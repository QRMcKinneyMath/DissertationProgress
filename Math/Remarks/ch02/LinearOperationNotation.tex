\begin{prop}[Linear Operator Notation]
\label{rmk:linearoperatornotation}
    $.$
    When dealing with mappings of 
    spaces of linear operators into
    spaces of other linear operators, 
    or even functions in general, 
    notation can get confusing, and
    presenting such things using ordinary notation without
    ambiguity can often require a plethora of parenthesis, 
    which hamper readability of an arguement. 

    For this reason, at points in this document, 
    I sometimes express the image $\beta(\alpha)$ using 
    \begin{equation*}
        \ip{\alpha, \beta}
    \end{equation*}
    Where $\beta:X \to Y$ 
    and $\alpha \in X$. 

    I combine this notation with usual function notation, 
    particularly in cases similar to the following. 
    For $i \in \{0,1\}$, 
    let $X_i, Y_i, Z_i$ be sets. 
    For $\alpha \in \{X,Y,Z\}$, let 
    $F_\alpha$ be the set of maps $f:\alpha_0 \to \alpha_1$. 
    If $T:F_X \to F_Y$, 
    $y \in Y_0$, 
    and $f \in F_X$, then I would notate
    \begin{equation*}
        \ip{y, Tf}
    \end{equation*}
    rather than $Tf(y)$ or $(T(f)(y))$
\end{prop}
