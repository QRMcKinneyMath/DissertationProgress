\begin{rmk}[\Seminorm Frechet]
\label{rmk:SeminormFrechet}
\rm
Let $X$ be a \LocallyConvex
\TVS.
An inspection of the proof of 
\ref{prop:LocallyConvexFromSeminorms}
reveals that if $\scB$ is a \LocalBasis 
for $X$, then there exists a collection of \Seminorms
$\scF = \{\norm{\cdot}_\alpha\}_{\alpha \in A}$ 
whose \Cardinality equals that of $\scB$ 
such that the \Topology on $X$ is the 
\WeakTopology on $X$ generated by $\scF$. 
In particular, if $X$ is \FirstCountable, 
then $\scF$ is \Countable as well. 
From $\scF$ we can define the translation invariant \Pseudometric 
\begin{equation*}
d(x,y) = \sum\limits_{i \in \N} \frac{1}{2^i} \frac{ \norm{x-y}_i}{1+\norm{x-y}_i}
\end{equation*}
on $X$. 
Much work has gone into the analysis \LocallyConvex \Metrizable spaces, 
but most of this work has been focused on the utility offered by the metrizability
of such spaces or geometric properties common to all \Seminorms, 
rather than the convenient geometric properties which may be possessed by the 
particular \Seminorms which define the \Metric on such spaces. 
This is not surprising, as many such properties are not preserved under 
\Homeomorphism, and therefore are not intrinsic to the space itself. 
I however, believe there is much to be gained from such investigation. 
This motivates the following definition. 
\end{rmk}