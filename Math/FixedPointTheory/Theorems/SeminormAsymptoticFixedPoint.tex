\begin{thm}[\AsymptoticallyNonexpansive \Seminorm fixed point]
\label{thm:AsymptoticallyNonexpansiveSeminormFixedPoint}
\rm
Let $X$ be a \UniformlyConvex \Complete \SeminormedSpace.
Let $K \subset X$ be \SetClosed, bounded, and \ConvexSet.
Let $T:K \to K$ be \AsymptoticallyNonexpansive. 
Then $Pfix(T) \neq \emptyset$. 
\begin{proof}
If $Diameter(K) = 0$, then $Pfix(T) = K$ so the result holds. 
Suppose otherwise. 
Let $x_0 \in K$. 
Define 
\begin{equation*}
R_{x_0} = \braces{\gamma \in [0,\infty) : \pa{\exists k \in \N}\pa{K \cap \bigcap\limits_{i=k}^{\infty} \overline{ B(T^ix_0, \gamma)} \neq \emptyset}}
\end{equation*}
Since $K$ is bounded,  $diam(K) \in R_{x_0}$. 
Define $\gamma_0 = \inf \pa{R_{x_0}}$. 
For each $\epsilon > 0$, define 
\begin{equation*}
C_\epsilon = \bigcup\limits_{k \in \N} \bigcap\limits_{i=k}^{\infty} \overline{B(T^ix_0, \gamma_0+\epsilon)}
\end{equation*}
Then each $C_\epsilon$ is the union of an increasing sequence of \ConvexSet sets, 
and is therefore itself
\ConvexSet.
Furthermore, each $C_\epsilon$ is clearly nonempty by definition of $\gamma_0$ and $R_{x_0}$. 
Hence, for each $ \epsilon > 0$, $\overline{C_\epsilon} \cap K$ is \SetClosed and \ConvexSet
and is therefore \weakly \SetClosed
by 
\ref{prop:ClosedAndConvexIsWeaklyClosed}. 
Since this set is bounded and $X$ is \Reflexive, 
$\overline{C_\epsilon} \cap K$ is \weakly \SetCompact 
by 
\ref{lem:ReflexiveSeparable}.
Thus, since $\{\overline{C_\epsilon} \cap K \}_{\epsilon > 0}$ possesses the 
\FiniteIntersectionProperty, we conclude
\begin{equation*}
\emptyset \neq \bigcap\limits_{\epsilon > 0} \pa{\overline{C_\epsilon} \cap K }
\end{equation*}
Denote the above intersection with $C$. 
Then if $x \in C$ and $\tau > 0$, then there exists $M \in \N$ such that if $m \geq M$, then 
\begin{equation*}
\norm{x-T^mx_0} \leq \gamma_0 + \tau
\end{equation*}
For sake of contradiction, let $y_0 \in C$ and suppose 
that both $T^n y_0 \not \to y_0$ and $\gamma_0 > 0$.
Then since $X$ is
\UniformlyConvex, 
there exists $\epsilon > 0$ and subsequence $\{n_k\}_{k \in \N} \subset \N$ such that 
for each $k \in \N$, 
\begin{equation*}
\norm{T^{n_k}y_0-y_0} \geq \epsilon
\end{equation*}
and for some $\xi > 0$, 
\begin{equation*}
1-\Delta \pa{\frac{\epsilon}{\gamma_0 + \xi}} < \frac{\gamma_0}{\gamma_0 + \xi}
\end{equation*}
Let $\{k_n\}_{n = N}^{\infty} \subset (1,\infty)$ such that $k_n \searrow 1$ and 
fo $n\geq N$, for each $x,y \in K$, 
\begin{equation*}
\norm{T^nx-T^ny} \leq k_n \norm{x-y}
\end{equation*}
Let $p \in \N$ such that 
\begin{equation*}
k_{n_p} \leq \frac{\gamma_0 + \xi}{\gamma_0 + \frac{\xi}{2}}
\end{equation*}
Since $y_0 \in C$, there exists $M \in \N$ such that if 
$m \geq M$, then 
\begin{equation*}
\norm{y_0-T^m x_0} \leq \gamma_0 + \frac{\xi}{2}
\end{equation*}
Define $\tilde{M}=n_p+M$.
If $m>\tilde{M}$, then $m>m-n_p > M$, so 
\begin{align*}
\norm{T^{n_p}y_0-T^mx_0} & = \norm{T^{n_p}y_0-T^{n_p}T^{m-n_p}x_0}\\
& \leq k_{n_p} \norm{y_0-T^{m-n_p}x_0}\\
& \leq \pa{\frac{\gamma_0+\xi}{\gamma_0+\frac{\xi}{2}}}\pa{\gamma_0+\frac{\xi}{2}} \\
& = \gamma_0+\xi
\end{align*}
and
\begin{equation*}
\norm{y_0-T^mx_0} \leq \gamma_0 + \xi
\end{equation*}
Furthermore, 
\begin{equation*}
\norm{\pa{T^{n_p}y_0-T^mx_0} - \pa{y_0-T^mx_0}} = \norm{T^{n_p}y_0-y_0} \geq \epsilon
\end{equation*}
Hence, by \ref{}, 
If $d = \max\pa{ \norm{y_0-T^mx_0}, \norm{T^{n_p}y_0-T^mx_0}} \leq \gamma_0+\xi$, then 
\begin{align*}
\norm{\pa{\frac{y_0+T^{n_p}y_0}{2}} - T^mx_0}& = \norm{\frac{\pa{T^{n_p}y_0-T^mx_0}+\pa{y_0-T^mx_0}}{2}} \\
& \leq d \pa{1-\Delta\pa{\frac{\epsilon}{d}}}\\
& \leq \pa{\gamma_0+\xi} \pa{ 1- \Delta \pa{\frac{\epsilon}{\gamma_0+\xi}}}\\
\end{align*}
Since $m > \tilde{m}$ is arbitrary, this implies that 
\begin{equation*}
\frac{y_0+T^{n_p}y_0}{2} \in \bigcap\limits_{k \geq \tilde{M}} \overline{B_X\pa{T^mx_0, \pa{\gamma_0+\xi}\pa{1-\Delta\pa{\frac{\epsilon}{\gamma_0+\xi}}}}}
\end{equation*}
so that $(\gamma_0+\xi)\pa{1-\Delta\pa{\frac{\epsilon}{\gamma_0+\xi}}}$.
This, however, is a contradiction because 
\begin{equation*}
(\gamma_0+\xi)\pa{1-\Delta\pa{\frac{\epsilon}{\gamma_0+\xi}}} < \gamma
\end{equation*}
Thus we conclude that either 
$\gamma_0 = 0$ or $y_0 \in \lim\limits_{n \to \infty} T^ny_0$.
In the latter case, since $T$ is \ContinuousFunction, 
\begin{equation*}
\norm{Ty_0-y_0} = d\pa{Ty_0,T\lim\limits_{n \to \infty} T^ny_0} \leq k_1 d\pa{y_0 ,  \lim\limits_{n \to \infty}T^ny_0} = 0
\end{equation*}
so that $Ty_0 \in y_0 + \overline{0}$.
\end{proof}
Again, since $T$ is \ContinuousFunction, $T\pa{y_0+\overline{0}} \subset y_0+\overline{0}$ and therefore 
$y_0 \in pFix(T)$. 
In the former case, $\{T^nx_0\}_{n \in \N}$ is a \PseudometricCauchySequence.
Since $K$ is \Complete, there exists $\tilde{x} \in K$ such that 
$\lim\limits_{n \to \infty} T^nx_0 = \tilde{x}+\overline{0}$. 
This then implies 
\begin{equation*}
\norm{T\tilde{x}-\tilde{x}} = d\pa{T\tilde{x}, T\lim\limits_{n \to \infty} T^n x_0} \leq k_1 d\pa{\tilde{x}, \lim\limits_{n \to \infty} T^nx_0} = 0
\end{equation*}
So that $T\tilde{x} \in \tilde{x}+\overline{0}$. 
Since $T$ is \ContinuousFunction, $T\pa{\tilde{x}+\overline{0}} \subset \pa{\tilde{x}+\overline{0}} \cap K$
\end{thm}

\begin{thm}
\label{thm:Asymptotic:Closed}
\rm
Let $X$ be a \Complete\SeminormedSpace.
Let $K \subset X$ be \SetClosed.
Let $T:K \to K$ be \ContinuousFunction.
Then $Pfix(T)$ is \SetClosed.
\begin{proof}
$Pfix(T)=\pa{T-I}^{-1}\{\overline{0}\}$ which is \SetClosed since $T-I$ is \ContinuousFunction.
\end{proof}
\end{thm}

\begin{thm}
\label{thm:Asymptotic:Convex}
\rm
Let $X$ be a \UniformlyConvex \SeminormedSpace.
Let $K \subset X$ be \ConvexSet.
Let $T:K \to K$ be \AsymptoticallyNonexpansive. 
Then $Pfix(T)$ is \ConvexSet.
\begin{proof}
Let $x_0,y_0 \in Pfix(T)$. 
Then, 
\begin{equation*}
\norm{T^n\pa{\frac{x_0+y_0}{2}} - x_0}  = \norm{T^n\pa{\frac{x_0+y_0}{2}} - T^nx_0}  \leq k_n \norm{\pa{x_0+y_0}{2}-x_0} = \frac{k_n}{2} \norm{x_0-y_0}
\end{equation*}
\begin{equation*}
\norm{T^n\pa{\frac{x_0+y_0}{2}} - y_0}  = \norm{T^n\pa{\frac{x_0+y_0}{2}} - T^ny_0}  \leq k_n \norm{\pa{x_0+y_0}{2}-y_0} = \frac{k_n}{2} \norm{x_0-y_0}
\end{equation*}
and
\begin{equation*}
\norm{\pa{T^n\pa{\frac{x_0+y_0}{2}} - x_0}-\pa{T^n\pa{\frac{x_0+y_0}{2}} - y_0}} = \norm{y_0-x_0}
\end{equation*}
Then, by \ref{}
with 
$\epsilon = \norm{x_0-y_0}$
and
$d = \frac{k_n}{2} \norm{x_0-y_0}$
we have 
\begin{align*}
\norm{T^n\pa{\frac{x_0+y_0}{2}}-\frac{x_0+y_0}{2}} & = \norm{\frac{\pa{T^n\pa{\frac{x_0+y_0}{2}} - x_0}+\pa{T^n\pa{\frac{x_0+y_0}{2}} - y_0}}{2}} \\
& \leq \frac{k_n}{2} \norm{x_0-y_0}\pa{1-\Delta\pa{\frac{2}{k_n}}}
\end{align*}
Since $k_n \to 1$ and since $\lim\limits_{\delta \nearrow 2} \Delta(\delta) =  1$, we conclude 
$\frac{x_0+y_0}{2} \in \lim\limits_{n \to \infty} T^n \pa{\frac{x_0+y_0}{2}}$.
Thus, by continuity
\begin{align*}
d\pa{T\pa{\frac{x_0+y_0}{2}}, \pa{\frac{x_0+y_0}{2}}} & = d \pa{T\pa{\frac{x_0+y_0}{2}},  \lim\limits_{n \to \infty} T^n \pa{\frac{x_0+y_0}{2}}}\\
&  \leq k_1 d \pa{\frac{x_0+y_0}{2},  \lim\limits_{n \to \infty} T^n \pa{\frac{x_0+y_0}{2}}}\\
& = 0
\end{align*}
By the standard continuity arguments, $T\pa{\frac{x_0+y_0}{2}+\overline{0}} \subset \frac{x_0+y_0}{2} + \overline{0} $
so that $\frac{x_0+y_0}{2} \in Pfix(T)$. 
\end{proof}
\end{thm}

\begin{thm}
\label{thm:PseudoAsymptoticNonexpansive}
\rm
Let $X$ be a \UniformlyConvex \Complete \SeminormedSpace.
Let $K \subset X$ be \SetClosed, bounded, and \ConvexSet.
Let $T:K \to K$ be \PseudoAsymptoticallyNonexpansive. 
Then $Pfix(T)$ is \SetClosed, \ConvexSet, and nonempty. 
\begin{proof}
Since $T$ is \PseudoAsymptoticallyNonexpansive, there exists $n \in \N$ such that 
$T^n$ is \AsymptoticallyNonexpansive. 
By 
\ref{thm:Asymptotic:Closed},
and \ref{thm:Asymptotic:Convex}, 
$Pfix(T^n)$ is \SetClosed and \ConvexSet. 
Since $K$ is bounded, $Pfix(T^n)$ is also bounded.
Furthermore, if $x_0 \in Pfix(T^n)$, then 
\begin{equation*}
Tx=TT^{n}x=T^nTx
\end{equation*}
Hence $T\pa{Pfix(T^n)} \subset Pfix(T^n)$. 
Also, if $x,y \in Pfix(T^n)$ and $k \in \N$, , then 
\begin{equation*}
\norm{Tx-Ty} = \norm{TT^{kn}x-TT^{kn}y} \leq k_{kn+1} \norm{x-y}
\end{equation*}
Hence, $T:Pfix(T^n) \to Pfix(T^n)$ is an \AsymptoticallyNonexpansive operator defined on a \SetClosed, 
\ConvexSet, bounded subset of a \Complete \UniformlyConvex \SeminormedSpace. 
By \ref{thm:AsymptoticallyNonexpansiveSeminormFixedPoint}, 
$Pfix(T) \neq \emptyset$. 
It is clear that $Pfix(T) \subset Pfix(T^n)$, and so 
we can also apply 
\ref{thm:Asymptotic:Closed}
and 
\ref{thm:Asymptotic:Convex}
to conclude that $Pfix(T)$ is \SetClosed and \ConvexSet.
\end{proof}
\end{thm}

\begin{thm}
\rm
Let $(X, \{\norm{\cdot}_i\_{i \in \N})$ be a \UniformlyConvex\GSpace such that for each $i \in \N$
$(X,\norm{\cdot}_i)$ is \UniformlyConvex.
Let $K \subset X$ be \SetClosed, bounded, and \ConvexSet such that for some $j\in \N$, 
for every $i \geq j$, 
$K + \overline{0}_i$ is \SetClosed in $(X, \norm{\cdot}_i)$. 
Let $T:K \to K$ be \Coherently \PseudoAsymptoticallyNonexpansive.
Then $fix(T) \neq \emptyset$. 
\begin{proof}
Let $\{i_k\}_{k \in \N}$ be an increasing \Sequence such that for each $k \in \N$, 
$T:K \subset (X, \norm{\cdot}_{i_k}) \to K \subset (X, \norm{\cdot}_{i_k})$ is \PseudoAsymptoticallyNonexpansive. 
Furthermore, by assumption, $K+\overline{0}_{i_k}$, for each $k \in \N$ is 
\SetClosed, bounded, and \ConvexSet in $\pa{X, \norm{\cdot}_{i_k}}$. 
Define $\tilde{T}_1:K+ \overline{0}_{i_1} \to K+\overline{0}_{i_1}$ by 
Let $Q:(X, \norm{\cdot}_{i_1}) \to (X, \norm{\cdot}_{i_1}) /\norm{\cdot}_{i_1}$ denote the 
\QuotientMap. 
By axiom choice let $\Gamma_i$ be a set containing 1 element of $K$ from each set in $Q^{-1}Q\pa{K}$. 
\begin{equation*}
\tilde{T}_1(x) = \begin{dcases}
Tx & x \in K \\
Ty & (x \not \in K)(\norm{y-\gamma} = 0)(\gamma \in \Gamma)
\end{dcases}
\end{equation*}
Then $\tilde{T}_1:K+\overline{0}_{i_1}  \subset (X, \norm{\cdot}_{i_1}) \to K+\overline{0}_{i_1}$ is 
\PseudoAsymptoticallyNonexpansive
and defined on a set which is \SetClosed, bounded and \ConvexSet on 
a \Complete, \UniformlyConvex \SeminormedSpace. 
Hence, by 
\ref{thm:PseudoAsymptoticNonexpansive}, 
$\tilde{T}_1$ possesses a \PseudoFixedPoint $x_1 \in K+\overline{0}_{i_1}$. 
Without loss of generality, $x_1 \in K$.
It is clear that $x_1$ is a \PseudoFixedPoint of $T$ with respect to the \Seminorm 
$\norm{\cdot}_{i_1}$. 
Thus, if we define $K_1 = K \cap \pa{x_0+ \overline{0}_{i_1}}$.
we can consider $T: K_1 \to K_1$ as a well defined \Coherently \PseudoAsymptoticallyNonexpansive map. 
It is clear that $K_1$ is \ConvexSet, as well as $\SetClosed$ in $X$. 
Furthermore, if $k \geq i_1$, 
then $\norm{\cdot}_k \geq \norm{\cdot}_{i_1}$, which implies
$\overline{0}_{i_1}+x_1$ is \SetClosed 
in $(X, \norm{\cdot}_{k})$ for each $k \geq i_1$. 
Hence, 
\begin{equation*}
K_1+ \overline{0}_k = K \cap (x_1+ \overline{0}_{i_1}) + \overline{0}_k = \pa{K+\overline{0}_k} \cap \pa{x_1 + \overline{0}_{i_1}}
\end{equation*}
is \SetClosed in $(X,\norm{\cdot}_k)$ for each $k \in \N$. 
Thus, we can apply the above arguement to claim that $T$ has a \PseudoFixedPoint $x_2$ with respect to $(X,\norm{\cdot}_{i_1})$ which is contained in $K_1$. 
Then defining $K_2 = K_1 \cap{x_2+\overline{0}_{i_2}}$ we can produce a sequence $\{x_i\}_{i \in \N}$ such that for each $i \in \N$, if $j>i$, then
\begin{equation*}
x_j \in x_i + \overline{0}_{i}
\end{equation*}
Hence, we can apply
\ref{thm:FixedPointConvergence}
to claim the existence of a \FixedPoint $\tilde{T}$ of $T$. 
\end{proof}
\end{thm}