\begin{rmk}[Introduction]
\rm
In this section, we utilize a different technique. 
If $X$ is a \SeminormedSpace and $T:X \to X$ is \ContinuousFunction, 
the continuity of $T$ allows us to 
induce a quotient operator $\tilde{T}:X/\overline{0} \to X/\overline{0}$ in a well defined sense. 
From this we can directly leverage \FixedPoint theorems for operators defined on normed spaces
do conclude the existence of a \FixedPoint for $\tilde{T}$ which then 
implies the existence of a \PseudoFixedPoint for $T$.
Thus the primary work is to ensure that the induced quotient operator still possesses the requisite properties to apply a given fixed point theorem. 
After this, we will apply
\ref{prop:FixedPointApproximation} to guarantee a \FixedPoint in the \GSpace case. 
\end{rmk}
\begin{prop}[Weakly Inward Pseudocontractive Path, \cite{morales2000}]
\label{prop:WeaklyInwardPathSeminorm}
\rm
Let $X$ be a \Complete \SeminormedSpace. 
Let $K \subset X$ be \SetClosed and \ConvexSet. 
Let $T:K \to X$ be a \ContinuousFunction \Pseudocontraction satisfying the 
\WeaklyInwardCondition. 
Let $x_0 \in K$. 
Then there exists a \ContinuousFunction $\gamma:[0,1) \to K$ such that 
for each $t \in [0,1]$, 
\begin{equation}
\label{eq:WeaklyInwardPath}
\gamma(t)  \in  \pa{tT \gamma(t)+(1-t)x_0} + \overline{0}
\end{equation}
Furthermore, if $\tilde{\gamma}:[0,1) \to K$ satisfies
\ref{eq:WeaklyInwardPath}, then
for each $t \in [0,1)$
$\norm{\gamma(t)-\tilde{\gamma}(t)} = 0$.
\begin{proof}
Let $Q:X \to X / \overline{0}$ denote the \QuotientMap. 
Since $T$ is \ContinuousFunction, there is a unique $\tilde{T}:Q(K) \to X/\overline{0}$ 
satisfying $\tilde{T} \circ Q = Q \circ T$.
We have $\tilde{T}[x] = [Tx]$. 
Since $T$ is \Pseudocontractive, 
if $x,y \in X$ and $\lambda > 0$, then
\begin{align*}
\norm{[x]-[y]} & = \norm{x-y}\\
& \leq \norm{x-y+\lambda\pa{(I-T)x-(I-T)y}}\\
& = \norm{[x-y+\lambda\pa{(I-T)x-(I-T)y}]}\\
& = \norm{[x]-[y]+ \lambda \pa{(I-\tilde{T})[x]-(I-\tilde{T})[y]}}
\end{align*}
Hence $\tilde{T}$ is \Pseudocontractive. 
It is clear that $Q\pa{I_K(x)} = I_{Q(K)}([x])$ for each $x \in K$, so 
we also see that $\tilde{T}$ satisfies the \WeaklyInwardCondition.
Thus By Proposition 1 of \cite{morales2000}, 
there is a unique function $\alpha:[0,1) \to Q(K)$ such that 
for each $t \in [0,1)$ $\alpha(t) = t \tilde{T} \alpha(t) + (1-t)[x_0]$
and $\alpha$ is \ContinuousFunction.
Fix $t_0 \in [0,1)$. 
Then there exists an $x_{t_0} \in K$ such that 
$\alpha(t_0) = [x_{t_0}]=Q(x_{t_0})$. 
Then, 
\begin{align*}
Q(x_{t_0}) & = [x_{t_0}] \\
& = \alpha(t) \\
& = t_0 \tilde{T}\alpha(t_0) + (1-t_0)[x_0]\\
& = t_0 \tilde{T} Q x_{t_0} + [(1-t_0)x_0]\\
& = t_0 Q T x_{t_0} + [(1-t_0)x_0]\\
& = [t_0Tx_{t_0}+(1-t_0)x_0]
\end{align*}
Hence, $x_{t_0} \in t_0Tx_{t_0}+(1-t)_0+\overline{0}$.
Define $\gamma:[0,1) \to K$ by $\gamma(t) = x_t$. 
Then, since $\alpha$ is continuous and 
\begin{equation*}
\norm{\gamma(t)-\gamma(s)} = \norm{x_t-x_s} = \norm{[x_t]-[x_s]} = \norm{\alpha(t)-\alpha(s)} 
\end{equation*}
Hence, we conclude that $\gamma$ is \ContinuousFunction. 
Also, by construction, $Q \circ \gamma = \alpha$.
Let $\tilde{\gamma}:[0,1) \to K$ be another 
function satisfying, for each $t \in [0,1)$, 
\ref{eq:WeaklyInwardPath}.
Then, for each $t \in [0,1)$, 
\begin{align*}
Q\circ \tilde{\gamma}(t) & = Q\pa{ tT \tilde{\gamma}(t) + (1-t)x_0}\\
& = tQ\circ T\tilde{\gamma}(t) +(1-t)[x_0]\\
& = t \tilde{T} \pa{Q \circ \tilde{\gamma}}(t) + (1-t)x_0
\end{align*}
The unique nature of $\alpha$ implies $Q \circ \tilde{\gamma} = \alpha$. 
But since $Q \circ \gamma = \alpha$, we conclude then that, 
for any $t \in [0,1)$, 
\begin{equation*}
\norm{\gamma(t)-\tilde{\gamma}(t) } = \norm{Q \circ \gamma (t) - Q \circ \tilde{\gamma}(t)} = \norm{\alpha(t)-\alpha(t)} = 0
\end{equation*}
Hence $\tilde{\gamma}(t) \in \gamma(t) + \overline{0}$.
\end{proof}
\end{prop}