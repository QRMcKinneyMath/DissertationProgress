\label{def:Direction}
\newcommand{\Direction}[0]{
    \bf \hyperref[def:Direction]{Direction} \rm
}

\newcommand{\Directions}[0]{
    \bf \hyperref[def:Direction]{Directions} \rm
}

\newcommand{\Directing}[0]{
    \bf \hyperref[def:Direction]{Directing} \rm
}

\newcommand{\Directings}[0]{
    \bf \hyperref[def:Direction]{Directings} \rm
}

\newcommand{\DirectedSet}[0]{
    \bf \hyperref[def:Direction]{Directed Set} \rm
}
\newcommand{\DirectedSets}[0]{
    \bf \hyperref[def:Direction]{Directed Sets} \rm
}
\newcommand{\DirectedSection}[0]{
    \textbf{\hyperref[def:Direction]{Section}}
}
\newcommand{\DirectedSections}[0]{
    \textbf{\hyperref[def:Direction]{Sections}}
}

\begin{df}[\Direction]
    Let $X \neq \emptyset$ be a set.
    Let $\leq$ be a \Preorder on X. 
	If every pair of elements in $X$ has an 
	\UpperBound with respect to $\leq$, then
    we call $\leq$ is a \Direction on $X$, 
	, we call $\leq$ is a \Directing of $X$
	, and we call $(X,\leq)$ is a \DirectedSet.
    If $x_0 \in X$, then we call
    $\{x \in X : x_0 \leq x\}$ the 
    \DirectedSection
    of $x_0$ under $\leq$. 
\end{df}
