\label{def:uniformlyconvergent}
\newcommand{\UniformlyConvergent}[0]{\textbf{\hyperref[def:uniformlyconvergent]{Uniformly Convergent}}\xspace}
\newcommand{\ConvergesUniformly}[0]{\textbf{\hyperref[def:uniformlyconvergent]{Converges Uniformly}}\xspace}
\newcommand{\UniformConvergence}[0]{\textbf{\hyperref[def:uniformlyconvergent]{Uniform Convergence}}\xspace}

\begin{df}[Uniform Convergence]
	Let $(X_\alpha, d_\alpha)$ be a \PseudometricSpace
	for $\alpha \in A$ where A is some indexing set. 
	For each $\alpha \in A$
	, let $\phi_\alpha :=\{x_i^\alpha\}_{i \in \N} \subset X_{\alpha}$
	be a sequence. 
	We say that the collection $\{\phi_\alpha\}_{\alpha \in A}$ 
    is \UniformlyConvergent to 
    $\{x_\alpha\}_{\alpha \in A} \in \prod\limits_{\alpha \in A} X_\alpha$
    if for each $\epsilon > 0$, 
    there is an $N \in \N$
    such that for each $n>N$, 
    and for every $\alpha \in A$, 
    we have 
    \begin{equation}
        d_\alpha(x^{\alpha}_i,x_\alpha) < \epsilon
    \end{equation}

    In this scenario, we may equivalently say that
    $\{\phi_\alpha\}$ demonstrates \UniformConvergence
    to $\{x_\alpha\}_{\alpha \in A}$ 
    or that it \ConvergesUniformly. 

    When we mention \UniformConvergence without
    reference to what the convergence is to, 
    we are merely claiming the existence of 
    such a limit. 
\end{df}
