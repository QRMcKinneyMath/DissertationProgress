\newcommand{\Pseudometric}[0]{\textbf{\hyperref[def:pseudometric]{Pseudometric}}\xspace}
\newcommand{\Pseudometrics}[0]{\textbf{\hyperref[def:pseudometric]{Pseudometrics}}\xspace}
\newcommand{\PseudometricSpaces}[0]{\textbf{\hyperref[def:pseudometric]{Pseudometric Spaces}}\xspace}
\newcommand{\PseudometricSpace}[0]{\textbf{\hyperref[def:pseudometric]{Pseudometric Space}}\xspace}
\begin{df}[Pseudometric]
\label{def:pseudometric}
\rm
    Let $X \neq \emptyset$. 
    Let $d:X \times X \to [0,\infty)$ be a \SymmetricMap that satisfies the \TriangleInequality and further satisfies, for each $x \in X$, 
    \begin{equation}
        d(x,x) = 0
    \end{equation}
    Under these conditions we call d a \Pseudometric on X and we call $\pa{X,d}$ a \PseudometricSpace.
    \end{df} 
	
	
	
\label{def:metric}
\newcommand{\Metric}[0]{
    \bf \hyperref[def:pseudometric]{Metric} \rm
}
\newcommand{\MetricSpace}[0]{
    \bf \hyperref[def:pseudometric]{Metric Space} \rm
}

\begin{df}[Metric]
	Let $(X,d)$ be a \PseudometricSpace. 
	If d has the property that for
	$x,y \in X$, if $x \neq y$, then
	\begin{equation*}
		d(x,y) \neq 0
	\end{equation*}
	Then we call d a \Metric on X 
	and we call $(X,d)$ a
	\MetricSpace
\end{df}
