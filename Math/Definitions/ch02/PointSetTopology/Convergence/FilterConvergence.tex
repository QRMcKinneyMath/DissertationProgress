\newcommand{\FilterLimit}[0]{
    \textbf{\hyperref[def:FilterConvergence]{Limit}}
}
\newcommand{\FilterLimits}[0]{
    \textbf{\hyperref[def:FilterConvergence]{Limits}}
}
\newcommand{\FilterConvergent}[0]{
    \textbf{\hyperref[def:FilterConvergence]{Convergent}}
}
\newcommand{\FilterConvergence}[0]{
    \textbf{\hyperref[def:FilterConvergence]{Convergence}}
}
\newcommand{\FilterConverges}[0]{
    \textbf{\hyperref[def:FilterConvergence]{Converges}}
}
\begin{df}[\FilterConvergence]
\label{def:FilterConvergence}
    Let $(X,\T)$ be a \TopologicalSpace. 
    Let $\scF$ be a \Filter on $X$. 
    Let $\scB$ be a \FilterBase for $\scF$. 
    Let \NeighborhoodFilterInstance{\T}(x) 
    denote the \NeighborhoodFilter of $\T$ at $x$. 
    Let \scF be \FinerFilter 
    than 
    \NeighborhoodFilterInstance{\T}(x).
    Then we say the following:
     $x$ is a \FilterLimit of $\scF$, 
     $x$ is a \FilterLimit of $\scB$, 
     $\scF$ \FilterConverges to $x$, 
     $\scB$ \FilterConverges to $x$, 
     $\scF$ is \FilterConvergent to $x$, 
     $\scB$ is \FilterConvergent to $x$, 
     $\scF$ posesses \FilterConvergence to $x$, and 
     $\scB$ posesses \FilterConvergence to $x$. 
\end{df}
