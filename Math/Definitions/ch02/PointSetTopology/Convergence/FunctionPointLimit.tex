\newcommand{\FunctionPointLimit}[0]{
    \textbf{\hyperref[def:FunctionPointLimit]{Limit}}
}
\newcommand{\FunctionPointLimits}[0]{
    \textbf{\hyperref[def:FunctionPointLimit]{Limits}}
}
\newcommand{\FunctionPointClusterPoint}[0]{
    \textbf{\hyperref[def:FunctionPointLimit]{Cluster Point}}
}
\newcommand{\FunctionPointClusterPoints}[0]{
    \textbf{\hyperref[def:FunctionPointLimit]{Cluster Points}}
}
\begin{df}[\FunctionPointLimit of a \Function at a point]
\label{def:FunctionPointLimit}
\rm
Let $(X,\T_X)$ be a \TopologicalSpace.
Let $(Y,\T_Y)$ be a \TopologicalSpace.
Let $f:X \to Y$. 
Let $a \in X$. 
Let $\scB$ be the \NeighborhoodFilter of $X$ at $a$. 
Let $y$ be a \FunctionLimit of $f$ with respect to $\scB$. 
Then instead of the standard notation
\begin{equation*}
y \in \lim\limits_{x, \scB}f(x)
\end{equation*}
we instead write
\begin{equation*}
y \in \lim\limits_{x \to a} f(x)
\end{equation*}
and we say that $y$ is a \FunctionPointLimit of $f$ at $a$. 
If $y$ is a \FunctionClusterPoint of $f$ with respect to $\scB$, then we say that
$y$ is a \FunctionPointClusterPoint of $f$ at $a$. 
\end{df}
