\newcommand{\FunctionLimit}[0]{
    \textbf{\hyperref[def:FunctionLimitPoint]{Limit}}
}
\newcommand{\FunctionLimits}[0]{
    \textbf{\hyperref[def:FunctionLimitPoint]{Limits}}
}
\newcommand{\FunctionClusterPoint}[0]{
    \textbf{\hyperref[def:FunctionLimitPoint]{Cluster Point}}
}
\newcommand{\FunctionClusterPoints}[0]{
    \textbf{\hyperref[def:FunctionLimitPoint]{Cluster Points}}
}
\begin{df}[\FunctionLimit]
\rm
    \label{def:FunctionLimitPoint}
    Let $X$ be a set and $Y$ be a topological space. 
    Let $\scF$ be a \Filter in $X$ and let 
    $f:X \to Y$. 
    Let $y \in Y$. 
    If $y$ is a \FilterLimit of $f(\scF)$, 
    then we say that 
    $y$ is a \FunctionLimit of $f$ with respect to $\scF$ 
    and we write $y \in \lim_{\scF} f$ or we may write
    $y \in \lim\limits_{x,\scF}f(x)$. 
    If $\lim_{\scF}f$ is a \Singleton then we will, 
    as an abuse of notation, write
    $y=\lim_{\scF}f$ and $y =\lim\limits_{x, \scF} f(x)$. 
    If $\scB$ is a \FilterBase on $X$ and 
    If $y$ is a \FilterClusterPoint $f(\scB)$ 
    then we sayt hat 
    $y$ is a \FunctionClusterPoint of $f$ with respect to $\scB$. 
\end{df}
