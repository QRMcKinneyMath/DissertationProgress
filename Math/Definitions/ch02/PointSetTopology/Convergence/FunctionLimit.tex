\newcommand{\FunctionLimit}[0]{\textbf{\hyperref[def:FunctionLimitPoint]{Limit}}\xspace}
\newcommand{\FunctionLimits}[0]{\textbf{\hyperref[def:FunctionLimitPoint]{Limits}}\xspace}
\newcommand{\FunctionClusterPoint}[0]{\textbf{\hyperref[def:FunctionLimitPoint]{Cluster Point}}\xspace}
\newcommand{\FunctionClusterPoints}[0]{\textbf{\hyperref[def:FunctionLimitPoint]{Cluster Points}}\xspace}
\begin{df}[\FunctionLimit]
\rm
    \label{def:FunctionLimitPoint}
    Let $X$ be a nonempty set and $Y$ be a \TopologicalSpace.
    Let $\scB$ be a \FilterBase in $X$ and let 
    $f:X \to Y$. 
    Let $y \in Y$. 
    If $y$ is a \FilterLimit of $f(\scB)$, 
    then we say that 
    $y$ is a \FunctionLimit of $f$ with respect to $\scB$ 
    and we write $y \in \lim_{\scB} f$ or we may write
    $y \in \lim\limits_{x,\scB}f(x)$. 
    If $\lim_{\scB}f$ contains just a single point then we will, 
    as an abuse of notation, write
    $y=\lim_{\scB}f$ and $y =\lim\limits_{x, \scB} f(x)$. 
    If $y$ is a \FilterClusterPoint $f(\scB)$ 
    then we say that 
    $y$ is a \FunctionClusterPoint of $f$ with respect to $\scB$. 
\end{df}
