\label{def:SubspaceTopology}
\newcommand{\SubspaceTopology}[0]{
    \textbf{\hyperref[def:SubspaceTopology]{Subspace Topology}}
}

\newcommand{\SubspaceTopologies}[0]{
    \textbf{\hyperref[def:SubspaceTopology]{Subspace Topologies}}
}

\newcommand{\SubspaceTopologicalSpace}[0]{
    \textbf{\hyperref[def:SubspaceTopology]{Subspace Topological Space}}
}

\newcommand{\SubspaceTopologicalSpaces}[0]{
    \textbf{\hyperref[def:SubspaceTopology]{Subspace Topological Spaces}}
}

\begin{df}[\SubspaceTopology]
    Let $(X, \T_X)$ 
    be a 
    \TopologicalSpace, 
    Let $Y \subset X$, 
    and let 
    $f$ be the 
    \InsertionFunction
    of $Y$ into $X$. 
    We call the \WeakTopology 
    on $Y$ generated by $f$
    which we will denote here with $\T_Y$, 
    the \SubspaceTopology
    of $Y$ relative to $(X,\T_X)$. 
    We call $(Y, \T_Y)$ the 
    \SubspaceTopologicalSpace.
    Unless otherwise specified, 
    when referring to a subset of a 
    \TopologicalSpace, 
    we consider that subset as 
    being a \TopologicalSpace 
    which is endowed with the \SubspaceTopology, 
    and when we say that a subset of a 
    \TopologicalSpace
    has a particular (Topological) property which has thus far only been defined 
    for a \TopologicalSpace, 
    we mean that the  \SubspaceTopologicalSpace 
    has that property. 
\end{df}


