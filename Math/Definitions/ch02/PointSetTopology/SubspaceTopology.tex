\newcommand{\SubspaceTopology}[0]{\textbf{\hyperref[def:SubspaceTopology]{Subspace Topology}}\xspace}
\newcommand{\SubspaceTopologies}[0]{\textbf{\hyperref[def:SubspaceTopology]{Subspace Topologies}}\xspace}
\newcommand{\SubspaceTopologicalSpace}[0]{\textbf{\hyperref[def:SubspaceTopology]{Subspace Topological Space}}\xspace}
\newcommand{\SubspaceTopologicalSpaces}[0]{\textbf{\hyperref[def:SubspaceTopology]{Subspace Topological Spaces}}\xspace}
\begin{df}[\SubspaceTopology]
\label{def:SubspaceTopology}

\rm
    Let $(X, \T_X)$ 
    be a 
    \TopologicalSpace, 
    Let $Y \subset X$.
    The \SubspaceTopology 
    of $Y$ relative to $(X, \scT_X)$
    is defined to be 
    the 
    \WeakTopology on $Y$ generated by the 
    \InsertionFunction $f:X \to Y$. 
    We denote the \SubspaceTopology
    with $\scT_Y$. 
    We call $(Y, \T_Y)$ the 
    \SubspaceTopologicalSpace.
    Unless otherwise specified, 
    when referring to a subset of a 
    \TopologicalSpace, 
    we consider that subset as 
    being a \TopologicalSpace 
    which is endowed with the \SubspaceTopology, 
    and when we say that a subset of a 
    \TopologicalSpace
    has a particular (Topological) property which has thus far only been defined 
    for a \TopologicalSpace, 
    we mean that the  \SubspaceTopologicalSpace 
    has that property. 
\end{df}


