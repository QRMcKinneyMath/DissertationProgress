\label{def:SubspaceTopology}
\newcommand{\SubspaceTopology}[0]{
    \textbf{\hyperref[def:SubspaceTopology]{Subspace Topology}}
}

\newcommand{\SubspaceTopologies}[0]{
    \textbf{\hyperref[def:SubspaceTopology]{Subspace Topologies}}
}

\newcommand{\SubspaceTopologicalSpace}[0]{
    \textbf{\hyperref[def:SubspaceTopology]{Subspace Topological Space}}
}

\newcommand{\SubspaceTopologicalSpaces}[0]{
    \textbf{\hyperref[def:SubspaceTopology]{Subspace Topological Spaces}}
}

\begin{df}[\SubspaceTopology]
    Let $(X, \T_X)$ 
    be a 
    \TopologicalSpace 
    and let 
    $Y \subset X$. 
    Define 
    \begin{equation*} 
        \T_Y = \left\{ U \cap Y | U \in \T_X\right\}
    \end{equation*} 
    Then $\T_Y$ 
    is a 
    \Topology 
    on $Y$ 
    which we call the 
    \SubspaceTopology
    on $Y$ of $(X, \T_X)$. 
    We call $(Y, \T_Y)$ the 
    \SubspaceTopologicalSpace.
    Unless otherwise specified, 
    when referring to a subset of a 
    \TopologicalSpace, 
    we consider that subset as 
    being a \TopologicalSpace 
    which is endowed with the \SubspaceTopology, 
    and when we say that a subset of a 
    \TopologicalSpace
    has a particular (Topological) property which has thus far only been defined 
    for a \TopologicalSpace, 
    we mean that the  \SubspaceTopologicalSpace 
    has that property. 
\end{df}


