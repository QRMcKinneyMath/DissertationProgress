\label{def:TopologyCoarseFine}
\newcommand{\TopologyCoarse}[0]{foo}

\newcommand{\TopologyFine}[0]{foo}

\newcommand{\TopologyCoarser}[0]{foo}

\newcommand{\TopologyFiner}[0]{foo}

\newcommand{\TopologyCoarsest}[0]{foo}

\newcommand{\TopologyFinest}[0]{}
\newcommand{\scTopologyCoarsenessRelation}[1]{
    %\hyperref[def:TopologyCoarseFine]{\ensuremath{\leq_{Top(#1)}}}
    foo
}
%
%\begin{df}[\TopologyCoarse, \TopologyFine]
%    Let $X$ be a set. 
%    Let $\T_1, \T_2$ be 
%    \TopologyRef s
%    on $X$
%    such that $\T_1 \subset \T_2$. 
%    In this case, we say that
%    $\T_1$ is more 
%    \TopologyCoarse
%    than 
%    $\T_2$, 
%    that $\T_1$
%    is 
%    \TopologyCoarser
%    than $\T_2$, 
%    that
%    $\T_2$ is more 
%    \TopologyFine
%    than $\T_1$, 
%    and that $\T_2$ is 
%    \TopologyFiner
%    than $\T_1$. 
%    
%    If $X$ is a set 
%    W
%    %Topology Coarseness partially orderes the set of topolgoies on X
%    %Intersection of elements of a subset of set of topologies is maximal element of that set. 
%    %Notation Defined Above
%   
%    Observe that the intersection of 
%%    any colleciton of \TopologyRef 's 
%    on $X$ 
%    is a \TopologyRef
%    on $X$. 
%
%    
%\end{df}
%
