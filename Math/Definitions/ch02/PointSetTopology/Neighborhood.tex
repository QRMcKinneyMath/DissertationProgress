\label{def:Neighborhood}
\newcommand{\Neighborhood}[0]{ \bf \hyperref[def:Neighborhood]{Neighborhood} \rm }
\newcommand{\Neighborhoods}[0]{ \bf \hyperref[def:Neighborhood]{Neighborhoods} \rm }
\newcommand{\NeighborhoodFilter}[0]{ \bf \hyperref[def:Neighborhood]{Neighborhood Filter} \rm }
\newcommand{\NeighborhoodFilters}[0]{ \bf \hyperref[def:Neighborhood]{Neighborhood Filters} \rm }
\newcommand{\NeighborhoodFilterInstance}[1]{\ensuremath{\scU_{#1}}}
\begin{df}[\Neighborhood, \NeighborhoodFilter]
    Let $(X, \T)$ be a \TopologicalSpace.
    Let $A \subset B \subset C \subset X$ 
    and let $B$ be \SetOpen in $(X,\T)$. 
    Then we call $C$ a 
    \Neighborhood of $A$ in $(X,\T)$. 
    If $x \in X$, then we call 
    a \Neighborhood of $\{x\}$ a 
    \Neighborhood of $x$. 
    We denote the collection of all \Neighborhoods 
    of $x \in X$ with 
    $\NeighborhoodFilterInstance{\T}(x)$.
    and we call this the 
	\NeighborhoodFilter of $x$.
    By 
    \ref{prop:NeighborhoodFilter}, 
    is a \Filter
    on $X$, and se we call this the
    \NeighborhoodFilter of 
    $\T$ at $x$. 
\end{df}
