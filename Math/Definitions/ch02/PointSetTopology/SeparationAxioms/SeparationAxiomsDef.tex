\newcommand{\Hausdorff}[0]{\textbf{\hyperref[def:Hausdorff]{Hausdorff}}\xspace}
\newcommand{\PseudoHausdorff}[0]{\textbf{\hyperref[def:Hausdorff]{Pseudo-Hausdorff}}\xspace}
\newcommand{\Hausdorffness}[0]{\textbf{\hyperref[def:Hausdorff]{Hausdorffness}}\xspace}
\newcommand{\TTwo}[0]{\textbf{\hyperref[def:Hausdorff]{T2}}\xspace}
\begin{df}[Separation Axioms]
    \label{def:SeparationAxioms}

    \rm
    Let $(X,\scT)$ be a \TopologicalSpace. 
    We define the following. 
    \begin{enumerate}[label=(\roman*), ref={\ref{def:SeparationAxioms}~\roman*}]
    \item \label{def:Hausdorff} We say $X$ is \Hausdorff, or \TTwo if 
    distinct points in $X$ have \Disjoint \Neighborhoods.
    \item \label{def:PseudoHausdorff} We say that $X$ is \PseudoHausdorff
    if, for each $x_0,x_1 \in X$, either $x_0$ and $x_1$ have \Disjoint 
    \Neighborhoods or $\scU_{\scT}(x_0) = \scU_{\scT}(x_1)$. 
    \end{enumerate}
\end{df}

\begin{rmk}
\rm
It is clear that any \Hausdorff space is \PseudoHausdorff, and that 
a space is \PseudoHausdorff if and only if its quotient under the 
\RelationOfEqualNeighborhoodFilters is \Hausdorff.
For this reason, many of the desirable properties of \Hausdorff spaces
are inherited by those which are \PseudoHausdorff.
As we will see later,  topological groups are \PseudoHausdorff spaces.
\end{rmk}
