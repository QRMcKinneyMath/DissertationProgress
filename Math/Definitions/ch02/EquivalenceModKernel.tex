\label{def:equivalencemodseminormkernel}
\newcommand{\EquivelanceModKernel}[0]{
    \bf \hyperref[def:equivalencemodseminormkernel]{Equivalence MOD-$\Ker$} \rm
}
\newcommand{\EquivalenceModKernel}[0]{
    \bf \hyperref[def:equivalencemodseminormkernel]{Equivalence MOD-$\Ker$} \rm
}
\newcommand{\EquivalentModKernel}[0]{
    \bf \hyperref[def:equivalencemodseminormkernel]{Equivalent MOD-$\Ker$} \rm
}
\newcommand{\SeminormKernelQuotientVectorSpace}[0]{
    \bf \hyperref[def:equivalencemodseminormkernel]{Seminorm Kernel Quotient Vector Space} \rm
}

\begin{df}[Quotient Space Mod Kernel]
Let $(X,\norm{\cdot})$ be a \SeminormedSpace over a field $\F \in \{\R,\C\}$.
with \SeminormKernel $\Ker$.
By \ref{prop:seminormkernelisavectorsubspace}, part 1, 
$\Ker$ is a vector subspace of $X's$ algebraic structure, and so if we define 
$\cong_{\Ker} \subset X \times X$ by setting, for $x,y \in X$
\begin{equation}
x \cong_{\Ker} y \iff x-y \in \Ker
\end{equation}
Then one recognizes $\cong_{\Ker}$ as \EquivelanceModKernel as would be commonly spoken of in Module or Vector Space theory. 
From this, alot of nice properties fall out. We list them here, without proof just to nail down notation. 
For proof, see any undergraduate algebra text.
\begin{enumerate}
%\item We denote the \QuotientSpace $X/\cong$ with $X/\Ker$. 
\item If $x \cong_{\Ker} y$, then we say that x and y are \EquivalentModKernel. 
\item For $x \in X$
    , we denote the \EquivalenceClass $[x]_{\cong_{\Ker}}$ with 
    $[x]_{\Ker}$ or 
    with $x+\Ker$, or 
    when confusion is unlikely, simply $[x]$. 
\item We denote $X/\cong_{\Ker}$ with $X/\Ker$. 
\item If we define $\oplus:X/\Ker \times X/\Ker \to X/\Ker$ by setting
    , for $x,y \in X$, $[x]_{\Ker}\oplus[y]_{\Ker}=[x+y]_{\Ker}$
    , then $\oplus$ is well defined and endows $X/\Ker$ with a group structure. 
\item If we further define $\odot:\F \times X/\Ker \to X/\Ker$ by 
    $\alpha [x]_{\Ker}=[\alpha x]_{\Ker}$
    , then $\pa{X/\Ker, \oplus, \odot, [0]_{\Ker}}$ is a Vector space over $\F$. 
\item Unless otherwise specified
    , when referring to the set $X/\Ker$
    , we endow it with the above vector space structure
    , and we call this space the \SeminormKernelQuotientVectorSpace 
    of the seminormed space $(X, \norm{\cdot})$.
\end{enumerate}
\end{df}

