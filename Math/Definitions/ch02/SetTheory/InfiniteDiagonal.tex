\newcommand{\InfiniteSetDiagonal}[0]{\textbf{\hyperref[def:InfiniteSetDiagonal]{Diagonal}}\xspace}
\newcommand{\InfiniteSetDiagonals}[0]{\textbf{\hyperref[def:InfiniteSetDiagonal]{Diagonals}}\xspace}
\newcommand{\scInfiniteSetDiagonal}[2]{\ensuremath{\hyperref[def:InfiniteSetDiagonal]{\Delta_{#1}\pa{#2}}}\xspace}
\begin{df}[\InfiniteSetDiagonal]
    \label{def:InfiniteSetDiagonal}
    \rm
    Let $X$ be a set.
    Let $A \neq \emptyset$. 
	For each $x \in X$, define $f_x:A \to X$
	by $f_x(\alpha) = x$ for each $\alpha \in A$. 
    We define 
    \begin{equation*}
    \scInfiniteSetDiagonal{A}{X}= \left\{ f_x \in \prod\limits_{\alpha \in A} X | x \in X\right\}
    \end{equation*}
    We call this the \InfiniteSetDiagonal of $X$ with respect to $A$, 
    or, when $A$ is understood, the \InfiniteSetDiagonal of $X$.
\end{df}
