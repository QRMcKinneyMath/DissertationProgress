\newcommand{\SetDiagonal}[0]{\textbf{\hyperref[def:SetDiagonal]{Diagonal}}\xspace}
\newcommand{\IdentityMap}[0]{\textbf{\hyperref[def:SetDiagonal]{Identity Map}}\xspace}
\newcommand{\IdentityMaps}[0]{\textbf{\hyperref[def:SetDiagonal]{Identity Maps}}\xspace}
\newcommand{\SetDiagonals}[0]{\textbf{\hyperref[def:SetDiagonal]{Diagonals}}\xspace}
\newcommand{\scSetDiagonal}[1]{\ensuremath{\hyperref[def:SetDiagonal]{\Delta\pa{#1}}}\xspace}
\newcommand{\scIdentity}[1]{\ensuremath{\hyperref[def:SetDiagonal]{I_{#1}}}\xspace}
\begin{df}[\IdentityMap]
\label{def:SetDiagonal}

\rm
    Let $X$ be a set. 
    We define 
    \begin{equation*}
	\scSetDiagonal{X}=\{(x,x)\in X \times X : x \in X\}
	\end{equation*}
    and we call 
    \scSetDiagonal{X} 
    the 
    \SetDiagonal
    of $X\times X$ or the 
    \IdentityMap of $X$.
    When viewing
    \scSetDiagonal{X} as a function, we may denote
    \scSetDiagonal{X} = \scIdentity{X}

\end{df}
