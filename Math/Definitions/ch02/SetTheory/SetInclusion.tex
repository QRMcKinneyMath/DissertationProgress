\newcommand{\SetInclusion}[0]{\textbf{\hyperref[def:SetInclusion]{Inclusion}}\xspace}
\begin{df}[$\in$]
\label{def:SetInclusion}

\rm
    The basic relation for sets is \SetInclusion, denoted $\in$.
    Let $X$ and $Y$ be sets. 
    We use the notation $Y \in X$ to indicate 
    that $Y$ is an element of $X$. 
    If $Y$ is not an element of $X$ then we write
    $Y \not \in X$. 
\end{df}
