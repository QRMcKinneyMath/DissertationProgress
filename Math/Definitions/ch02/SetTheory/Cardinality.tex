\newcommand{\Cardinal}[0]{\textbf{\hyperref[def:Cardinality]{Cardinal}}\xspace}
\newcommand{\Cardinals}[0]{\textbf{\hyperref[def:Cardinality]{Cardinals}}\xspace}
\newcommand{\Cardinality}[0]{\textbf{\hyperref[def:Cardinality]{Cardinality}}\xspace}
\newcommand{\Cardinalities}[0]{\textbf{\hyperref[def:Cardinality]{Cardinalities}}\xspace}
\newcommand{\FirstNaturals}[1]{\hyperref[def:Cardinality]{\ensuremath{N_{#1}}}\xspace}
\newcommand{\CardinalityFunction}[1]{\hyperref[def:Cardinality]{\ensuremath{\textbf{Card}\pa{#1}\xspace}}}
\newcommand{\Finite}[0]{\textbf{\hyperref[def:Cardinality]{Finite}}\xspace}
\newcommand{\Infinite}[0]{\textbf{\hyperref[def:Cardinality]{Infinite}}\xspace}
\newcommand{\Denumerable}[0]{\textbf{\hyperref[def:Cardinality]{Denumerable}}\xspace}
\newcommand{\Countable}[0]{\textbf{\hyperref[def:Cardinality]{Countable}}\xspace}
\newcommand{\Uncountable}[0]{\textbf{\hyperref[def:Cardinality]{Uncountable}}\xspace}
\begin{df}[Cardinality]
\label{def:Cardinality}

\rm
    We define $\mathbb{Z}^+=\{1, 2, 3, \cdots\}$. 
    We define $\mathbb{N} = \{0, 1, 2, 3, \cdots\}$.
    Let $n \in \Z^+$. We define 
    $\FirstNaturals{n} = \Z^+ \cap [1,n]$.
    Let $X$ be a set.
    Let $f:X \to \FirstNaturals{n}$ 
    be a 
    \Bijection. 
    Then, we say that 
    $X$ has 
    \Cardinality
    $n$
    and we write 
    $\CardinalityFunction{X}=n$.
    We say that the 
    \Cardinality 
    of the empty set is $0$
    and we write
    $\CardinalityFunction{\emptyset} = 0$.
    More generally, if there exists a 
    \Bijection
    between two sets 
    $Y$ and $Z$, then we write
    $\CardinalityFunction{Y}=\CardinalityFunction{Z}$
    and we say that they have the same 
    \Cardinalities. 
    Define
    $X_0=\N$
    and for $k \in \N$, define 
    $X_{k+1} = \scPowerSet{X_k}$. 
    Then for $k \in \N$, we define 
    $\aleph_k = \CardinalityFunction{X_k}$.
    If $\CardinalityFunction{X} \in \N$, then 
    we say that $X$ is \Finite. 
    If $\CardinalityFunction{Z} \in \N$ or 
    $\CardinalityFunction{Z} = \aleph_0$, 
    then we say that $Z$ is \Denumerable.
    If $\CardinalityFunction{Y} = \aleph_0$, then
    we say that $Y$ is \Countable.
    If $\CardinalityFunction{W}= \aleph_k$ for $k \geq 1$, 
    then we say that $W$ is \Uncountable. 
    If $\CardinalityFunction{V} = \aleph_j$ for $j \in \N$, 
    then we say that $V$ is \Infinite. 


\end{df}



