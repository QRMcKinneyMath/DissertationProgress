\newcommand{\Cover}[0]{\textbf{\hyperref[def:Cover]{Cover}}\xspace}
\newcommand{\CoveredBy}[0]{\textbf{\hyperref[def:Cover]{Covered By}}\xspace}
\newcommand{\Covers}[0]{\textbf{\hyperref[def:Cover]{Covers}}\xspace}
\newcommand{\Subcover}[0]{\textbf{\hyperref[def:Cover]{Subcover}}\xspace}
\newcommand{\Subcovers}[0]{\textbf{\hyperref[def:Cover]{Subcovers}}\xspace}
\begin{df}[\Cover, \Subcover] 
\label{def:Cover}
\rm
    Let $X$ be a set and let 
    $Y=\{Y_\alpha\}_{\alpha \in A}$ 
	be a collection of sets
    such that 
    \begin{equation*}
        X \subset \bigcup_{\alpha \in A} Y_{\alpha}
    \end{equation*}
    Then we say 
    $Y$ 
    is a 
    \Cover
    of $X$,
    we say $Y$ 
    \Covers $X$,
	and we say $X$ 
	is \CoveredBy
	$Y$. 
    In the context of talking about a 
    \Cover, if every member of a 
    \Cover posses a certain property
    then we may say that the \Cover 
    has that property. 
	An exception to this is that 
    when talking about the 
    \Cardinality
    or \Disjointedness 
    of a \Cover.
	In such cases we are saying
	that the \Cover itself is 
	\Disjoint or of a particular 
	\Cardinality, not that the constituent 
	sets each have that \Cardinality 
	or are themselves \Disjoint collections.
    If $Z \subset Y$ \Covers $X$, then
    we call $Z$ a \Subcover of $Y$. 
\end{df}
