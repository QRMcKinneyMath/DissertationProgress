\newcommand{\Cover}[0]{
    \bf \hyperref[def:Cover]{Cover} \rm
}
\newcommand{\Covers}[0]{
    \bf \hyperref[def:Cover]{Covers} \rm
}
\newcommand{\Subcover}[0]{
    \bf \hyperref[def:Cover]{Subcover} \rm
}
\newcommand{\Subcovers}[0]{
    \bf \hyperref[def:Cover]{Subcovers} \rm
}\begin{df}[Cover, Subcover]
\label{def:Cover}
\rm
    Let $X$ be a set and let 
    $Y=\{Y_\alpha\}_{\alpha \in A}$ 
    such that 
    \begin{equation*}
        X \subset \bigcup_{\alpha \in A} Y_{\alpha}
    \end{equation*}
    Then we say that 
    $Y$ 
    is a 
    \Cover
    for $X$ 
    or that $Y$ 
    \Covers $X$. 
    In the context of talking about a 
    \Cover, if every member of a 
    \Cover posses a certain property
    then we may say that the \Cover 
    has that property. 
    If $Z \subset Y$ \Covers $X$, then
    we call $Z$ a \Subcover of $Y$. 
    One exception to this is that 
    when talking about the 
    \Cardinality
    or \Disjointedness 
    of a \Cover, we are 
    talking about \Cover itself, 
    not each of its constituent sets. 
\end{df}
