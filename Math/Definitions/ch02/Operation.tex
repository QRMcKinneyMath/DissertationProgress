\label{def:Operation}
\newcommand{\Operation}[0]{
    \bf \hyperref[def:Opeartion]{Operation} \rm
}
\newcommand{\Operations}[0]{
    \bf \hyperref[def:Opeartion]{Operations} \rm
}
\newcommand{\UnaryOperation}[0]{
    \bf \hyperref[def:Opeartion]{Unary Operation} \rm
}
\newcommand{\UnaryOperations}[0]{
    \bf \hyperref[def:Opeartion]{U`nary Operations} \rm
}

\newcommand{\BinaryOperation}[0]{
    \bf \hyperref[def:Opeartion]{Binary Operation} \rm
}
\newcommand{\BinaryOperations}[0]{
    \bf \hyperref[def:Opeartion]{Binary Operations} \rm
}


\begin{df}[\Operation]
    Let $X \neq \emptyset$ be a set 
    and let $Y\neq \emptyset$
    be a set with
    $cardinality(Y)=n \in \N$. 
    Let 
    \begin{equuation*}
        T \in \prod\limits_{y \in Y} X
    \end{equation*}
    $n$-ary \Operation 
    on $X$, or when $n$
    is understood, just an 
    \Operation on $X$. 
    If $n=1$, we 
    may call 
    $T$ a 
    \UnaryOperation
    on $X$,
    If $n=2$, we 
    call T a 
    \BinaryOperation
    on $X$ 
    and may use the following notation for 
    $x,y \in X$. 
    \begin{equation*}
        T(x,y)=xTy
    \end{equation*}
\end{df}
