\label{def:IdentityElement}

\newcommand{\IdentityElement}[0]{
    \bf \hyperref[def:IdentityElement]{Identity Element} \rm
}
\newcommand{\IdentityElements}[0]{
    \bf \hyperref[def:IdentityElement]{Identity Elements} \rm
}
\newcommand{\LeftIdentityElement}[0]{
    \bf \hyperref[def:IdentityElement]{Left Identity Element} \rm
}
\newcommand{\LeftIdentityElements}[0]{
    \bf \hyperref[def:IdentityElement]{Left Identity Elements} \rm
}
\newcommand{\RightIdentityElement}[0]{
    \bf \hyperref[def:IdentityElement]{Right Identity Element} \rm
}
\newcommand{\RightIdentityElements}[0]{
    \bf \hyperref[def:IdentityElement]{Right Identity Elements} \rm
}

\begin{df}[\LeftIdentityElement, \RightIdentityElement]
    Let $X \neq 0$ be a set and 
    $L,R$ each be 
    \BinaryOperations 
    on X. 
    Let $l , r \in X$ 
    such that
    for every $x \in X$ 
    we have 
   \begin{align*}
        lLx=x\\
        xRr=x
   \end{align*}
   In such a scenario, we say that
   $l$ is a \LeftIdentityElement 
   of $L$, and
   we say that 
   $r$ is a 
   \RightIdentityElement
   of $R$. 
   When the 
   \Operation
   in question 
   is clear, 
   we may say that 
   a \LeftIdentityElement
   or a \RightIdentityElement
   is of the underlying set itself
   rather than of the 
   \Operation
   .
\end{df}

\begin{df}[\IdentityElement]
    Let $X \neq 0$ be a set.
    Let $\oplus$ be a \BinaryOperation on $X$. 
    Let $e \in X$ be both a 
    \LeftIdentityElement 
    and a 
    \RightIdentityElement 
    of $\oplus$. 
    In this scenario, we say that
    $e$ is an \IdentityElement of 
    $\oplus$. 
    We the 
    \Operation
    in question
    is clear, we may say that
    a \IdentityElement
    is of the underlying set itself
    rather than of the 
    \Operation. 
\end{df}


