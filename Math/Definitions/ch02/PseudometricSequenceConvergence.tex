\label{def:pseudometricsequenceconvergence}
\newcommand{\PseudometricConvergence}[0]{
    \bf \hyperref[def:pseudometricsequenceconvergence]{Pseudometric-Convergence} \rm
}
\newcommand{\PseudometricConvergent}[0]{
    \bf \hyperref[def:pseudometricsequenceconvergence]{Pseudometrically-Convergent} \rm
}
\newcommand{\PseudometricConverges}[0]{
    \bf \hyperref[def:pseudometricsequenceconvergence]{Pseudometric-Converges} \rm
}
\begin{df}[Pseudometric Convergence]
    Let $(X,d)$ be a pseudometric space and let $\{x_i\}_{i \in \N}$ be a sequence in $(X,d)$.
    Let $x_0 \in X$.  
    We say that $\{x_i\}_{i \in \N}$ exhibits \PseudometricConvergence to $x_0$ in d, or we say that $\{x_i\}_{i \in \N}$  \PseudometricConverges to $x_0$ in d or we say that $\{x_i\}_{i \in \N}$ is \PseudometricConvergent to $x_0 \in d$ if, 
    for every $\epsilon > 0$, there is an $N \in \N$ such that for every $n>N$, we have 
    \begin{equation}
        d(x_0, x_n) < \epsilon
    \end{equation}
\end{df}