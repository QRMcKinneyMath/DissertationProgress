\label{def:pseudometriccomplete}
\newcommand{\PseudometricComplete}[0]{
    \bf \hyperref[def:pseudometriccomplete]{Pseudometric-Complete} \rm
}
\newcommand{\Complete}[0]{
    \bf \hyperref[def:pseudometriccomplete]{Complete} \rm
}
\begin{df}[Pseudometric Complete]
    We say that a \PseudometricSpace $(X,d)$ is 
    \PseudometricComplete if each 
	\PseudometricCauchySequence 
	sequence in $(X,d)$ 
	\PseudometricConverges to a limit in $X$.


	In the case that d is a \Metric, then
	being \PseudometricComplete is 
	equivalent to beging \Complete
	in the classical sense, so 
	we will commonly refer to a \PseudometricSpace
	which is \PseudometricComplete as simply
	being \Complete. 
    \end{df}