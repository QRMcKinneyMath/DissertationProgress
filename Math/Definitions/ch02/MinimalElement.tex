\label{def:MinimalElement}
\newcommand{\MinimalElement}[0]{
    \bf \hyperref[def:MinimalElement]{Minimal Element} \rm
}

\newcommand{\Minimum}[0]{
    \bf \hyperref[def:MinimalElement]{Minimum} \rm
}

\newcommand{\Minima}[0]{
    \bf \hyperref[def:MinimalElement]{Minima} \rm
}

\begin{df}[\MinimalElement]
    Let $X \neq \emptyset$ be a set. 
    Let $R$ be an 
    \Relation on $X$. 
    Let $Y \subset X$.
    Let $a \in Y$. 
    We say that $a$ is a 
    \MinimalElement of $Y$, 
    or equivalently we say that 
    $a$ is a \Minimum of Y 
    if for every $b \in Y$,
	if $b R a$, then 
    we have $a = b$. 
	The Plural of \Minimum is \Minima, 
	and we represent the set of \Minima of Y with 
	respect to the relation $R$ with 
	$\Minima_R(Y)$, or if $R$ is understood, 
	we represent the set of $\Minima$ of Y with 
	$\Minima(Y)$. 
	
	
\end{df}

\begin{prop}[\MinimalElement unique if R is \AntiSymmetricRelation]
	Let $X \neq \emptyset$ be a set. 
	Let $R$ be an
	\AntiSymmetricRelation
	\Relation
	on X. 
	Let $Y \subset X$.
	Let $a$ and $b$ be 
	each be a \MinimalElement
	of Y. 
	Then $a=b$. 
	\begin{proof}
		Since $a \in \Minima(Y)$, 
		$a \leq b$. 
		Since $b \in \Minima(Y)$, 
		$b \leq a$. 
		By \RelationAntiSymmetry, 
		$b = a$. 
	\end{proof}
\end{prop}