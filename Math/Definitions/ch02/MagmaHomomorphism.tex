\label{def:MagmaHomomorphism}
\newcommand{\MagmaHomomorphism}[0]{
    \bf \hyperref[def:MagmaHomomorphism]{Magma Homomorphism} \rm
}
\newcommand{\MagmaHomomorphisms}[0]{
    \bf \hyperref[def:MagmaHomomorphism]{Magma Homomorphisms} \rm
}
\newcommand{\scMagma}[0]{
    \bf \hyperref[def:MagmaHomomorphism]{Magma} \rm
}
\newcommand{\Additive}[0]{
    \bf \hyperref[def:MagmaHomomorphism]{Additive} \rm
}
\newcommand{\Additivity}[0]{
    \bf \hyperref[def:MagmaHomomorphism]{Additivity} \rm
}
\begin{df}[\MagmaHomomorphism]
    Let $(X,\oplus_X)$
    and $(Y,\oplus_Y)$
    be \Magmas.
    Let $T:X \to Y$ satisfy, 
    for each $x_1, x_2 \in X$. 
    \begin{equation*}
        T\pa{x_1 \oplus_X x_2} = T\pa{x_1} \oplus_Y T\pa{x_2}
    \end{equation*}
    Then we call 
    T a \MagmaHomomorphism.
    We represent the collection of
    \MagmaHomomorphisms
    from $(X,\oplus_X)$ 
    to $(Y,\oplus_Y)$
    with 
    $H_{\scMagma}\pa{\pa{X, \oplus_X}, \pa{Y, \oplus_Y}}$, 
    or, when $\oplus_X$ and $\oplus_Y$ are clear, 
    $H_{\scMagma}\pa{X, Y}$. 
    A \MagmaHomomorphism
    is called \Additive
    and posseses the property 
    \Additivity. 
\end{df}
