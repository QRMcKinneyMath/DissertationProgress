\newcommand{\Magma}[0]{\textbf{\hyperref[def:Magma]{Magma}}\xspace}
\newcommand{\Magmas}[0]{\textbf{\hyperref[def:Magma]{Magmas}}\xspace}
\newcommand{\CommutativeMagma}[0]{\textbf{\hyperref[def:Magma]{Commutative Magma}}\xspace}
\newcommand{\CommutativeMagmas}[0]{\textbf{\hyperref[def:Magma]{Commutative Magmas}}\xspace}

\begin{df}[\Magma]
\label{def:Magma}
\rm
    Let $X$ be a nonempty set and
    $T:X \times X \to X$ be a 
    \BinaryOperation
    on $X$. 
    We call the pair $(X,T)$ a 
    \Magma.
    When it is clear what operation is being referred to, 
    we may simply refer to $X$
    as the 
    \Magma.
	If 
	$T$
	is
	\CommutativeFunction, 
	then we call 
	$(X,T)$ (or simply just $X$)
	a \CommutativeMagma.
	In general, this naming convention is used 
	for any algebraic structure defined on a set 
	via a \BinaryOperation with
	particular properties. 
	
	Now let $(M,+)$ be a \Magma, let $x \in M$, and let 
	$A,B \subset M$. 
	We define the following:
	\begin{equation*}
	x+A = \{x+y : y \in A\}
	\end{equation*}
    \begin{equation*}
    A+x = \{y+x : y \in A\}
    \end{equation*}
	\begin{equation*}
	A+B = \{a+b : a \in A \tab[.5cm] and \tab[.5cm] b \in B\}
	\end{equation*}
\end{df}
