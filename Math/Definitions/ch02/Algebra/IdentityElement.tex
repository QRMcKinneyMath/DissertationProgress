\newcommand{\IdentityElement}[0]{\textbf{\hyperref[def:IdentityElement]{Identity Element}}\xspace}
\newcommand{\IdentityElements}[0]{\textbf{\hyperref[def:IdentityElement]{Identity Elements}}\xspace}
\newcommand{\LeftIdentityElement}[0]{\textbf{\hyperref[def:IdentityElement]{Left Identity Element}}\xspace}
\newcommand{\LeftIdentityElements}[0]{\textbf{\hyperref[def:IdentityElement]{Left Identity Elements}}\xspace}
\newcommand{\RightIdentityElement}[0]{\textbf{\hyperref[def:IdentityElement]{Right Identity Element}}\xspace}
\newcommand{\RightIdentityElements}[0]{\textbf{\hyperref[def:IdentityElement]{Right Identity Elements}}\xspace}

\begin{df}[\LeftIdentityElement, \RightIdentityElement]
\label{def:IdentityElement}
\rm
    Let $(X,L)$ and
    $(X,R)$ be 
    \Magmas.
    Let $l , r \in X$ 
    such that
    for every $x \in X$ 
    we have 
   \begin{align*}
        lLx=x\\
        xRr=x
   \end{align*}
   In such a scenario, we say that
   $l$ is a \LeftIdentityElement 
   of $(X,L)$, and
   we say that 
   $r$ is a 
   \RightIdentityElement
   of $(X,R)$. 
\end{df}

\begin{df}[\IdentityElement]
\rm
    Let $(X,\oplus)$ be a 
    \Magma. 
    Let $e \in X$ be both a 
    \LeftIdentityElement 
    and a 
    \RightIdentityElement 
    of $\oplus$. 
    Then, we say that
    $e$ is an \IdentityElement of 
    $(X,\oplus)$. 
\end{df}


