\newcommand{\Ring}[0]{\textbf{\hyperref[def:Ring]{Ring}}\xspace}
\newcommand{\Rings}[0]{\textbf{\hyperref[def:Ring]{Rings}}\xspace}
\newcommand{\Field}[0]{\textbf{\hyperref[def:Ring]{Field}}\xspace}
\newcommand{\Fields}[0]{\textbf{\hyperref[def:Ring]{Fields}}\xspace}
\begin{df}[\Ring]
\label{def:Ring}
\rm
Let $R$ be a nonempty set.
Let $(R,+,0)$ be a \CommutativeFunction \Group. 
Let $(R \setminus \{0\}, \cdot, 1)$ be a \Monoid. 
Suppose that for each $a,b,c \in R$ we have 
\begin{equation*}
a \cdot \pa{b+c} = \pa{a \cdot b} + \pa{a \cdot c}
\end{equation*}
\begin{equation*}
\pa{a+b} \cdot c = \pa{a \cdot c} + \pa{b \cdot c}
\end{equation*}
Then we call $(R,+,\cdot,0,1)$ a \Ring. 
When the context is clear, we call $R$ a \Ring.
If $\cdot$ is \CommutativeFunction, then we call $R$ a 
\CommutativeFunction \Ring. 
If $(R \setminus \{0\},\cdot,1)$ is a \CommutativeGroup
then we call $R$ a \Field.
\end{df}
