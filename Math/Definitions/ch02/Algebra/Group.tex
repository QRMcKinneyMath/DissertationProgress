\label{def:Group}

\newcommand{\Group}[0]{
    \bf \hyperref[def:Group]{Group} \rm
}
\newcommand{\Groups}[0]{
    \bf \hyperref[def:Group]{Groups} \rm
}
\newcommand{\CommutativeGroup}[0]{
    \bf \hyperref[def:Group]{Commutative Group} \rm
}
\newcommand{\CommutativeGroups}[0]{
    \bf \hyperref[def:Group]{Commutative Groups} \rm
}
\newcommand{\AbelianGroup}[0]{
    \bf \hyperref[def:Group]{Abelian Group} \rm
}
\newcommand{\AbelianGroups}[0]{
    \bf \hyperref[def:Group]{Abelian Groups} \rm
}

\begin{df}[\Group]
    Let $(X,\oplus,e)$ 
    be a \Monoid
    such that
    each 
    $x \in X$
    is an
    \InvertibleElement.
    Then we call
    $(X,\oplus,e)$ a 
    \Group. 
	Out of respect, we call a 
	\CommutativeGroup
	an
	\AbelianGroup.
\end{df}

