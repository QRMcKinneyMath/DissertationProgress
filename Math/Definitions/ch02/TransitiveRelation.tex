\label{def:TransitiveRelation}
\newcommand{\TransitiveRelation}[0]{
    \bf \hyperref[def:TransitiveRelation]{Transitive} \rm
}

\newcommand{\RelationTransitivity}[0]{
    \bf \hyperref[def:TransitiveRelation]{Transitivity} \rm
}

\begin{df}[\TransitiveRelation]
    Let $X \neq \emptyset$ be a set. 
    Let $R$ be a \Relation on X. 
    We say that $R$ is \TransitiveRelation, 
    or equivalently we say that
    $R$ posseses 
    \RelationTransitivity
    if whenever $(a,b) \in R$ and $(b,c) \in R$, 
    we also have $(a,c) \in R$. 
\end{df}