\label{def:scalarhomogeneous}
\newcommand{\ScalarHomogeneous}[0]{
    \bf \hyperref[def:scalarhomogeneous]{Scalar Homogeneous} \rm
}

\newcommand{\ScalarHomogeneity}[0]{
    \bf \hyperref[def:scalarhomogeneous]{Scalar Homogeneity} \rm
}
\begin{df}[Scalar Homogeneous]
    Let V be a vector space over a field $\F \in \{\R, \C\}$. 
    
    We say that a map $p:V \to V$ is \ScalarHomogeneous, if
    , for each $\alpha \in \F$ and each $x \in V$, we have 
    \begin{equation}
        p(\alpha x) = \alpha p(x)
    \end{equation}
    Under these circumstances, we may instead say that the operator 
    p posesses \ScalarHomogeneity.
\end{df}

\label{def:absolutevaluescalarhomogeneous}
\newcommand{\AbsScalarHomogeneous}[0]{
    \bf \hyperref[def:absolutevaluescalarhomogeneous]{Absolutely Scalar Homogeneous} \rm
}

\newcommand{\AbsScalarHomogeneity}[0]{
    \bf \hyperref[def:absolutevaluescalarhomogeneous]{Absolute Scalar Homogeneity} \rm
}
\begin{df}[Scalar Homogeneous]
    Let V be a vector space over a field $\F \in \{\R, \C\}$. 
    
    We say that a map $p:V \to V$ is \AbsScalarHomogeneous, if
    , for each $\alpha \in \F$ and each $x \in V$, we have 
    \begin{equation}
        p(\alpha x) = \abs{\alpha} p(x)
    \end{equation}
    Under these circumstances, we may instead say that the operator 
    p posesses \AbsScalarHomogeneity.
\end{df}


\label{rmk:seminorm}
\begin{rmk}[\ScalarHomogeneous or \AbsScalarHomogeneous operator at 0 is 0]

If V is a vector space over $\mathbb{F} \in \{\R, \C\}$, then for each $x \in V$, $0x=0$.
Hence, if p is a \AbsScalarHomogeneous operator on v, then for any $x \in V$
\begin{equation}
p(0)=p(0x)=|0|p(x)=0p(x)=0
\end{equation}
If instead p is \ScalarHomogeneous operator on V, then we have
\begin{equation}
p(0)=p(0x)=0p(x)=0
\end{equation}
that is, in either case,  p(0)=0. 
\end{rmk}




