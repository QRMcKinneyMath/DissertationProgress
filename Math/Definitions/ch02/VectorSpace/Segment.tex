\newcommand{\Interval}[0]{\textbf{\hyperref[def:Interval]{Interval}}\xspace}
\newcommand{\Intervals}[0]{\textbf{\hyperref[def:Interval]{Intervals}}\xspace}
\newcommand{\ClosedInterval}[0]{\textbf{\hyperref[def:Interval]{Closed Interval}}\xspace}
\newcommand{\ClosedIntervals}[0]{\textbf{\hyperref[def:Interval]{Closed Intervals}}\xspace}
\newcommand{\OpenInterval}[0]{\textbf{\hyperref[def:Interval]{Open Interval}}\xspace}
\newcommand{\OpenIntervals}[0]{\textbf{\hyperref[def:Interval]{Open Intervals}}\xspace}
\newcommand{\HalfClosedInterval}[0]{\textbf{\hyperref[def:Interval]{Half-Closed Interval}}\xspace}
\newcommand{\HalfClosedIntervals}[0]{\textbf{\hyperref[def:Interval]{Half-Closed Intervals}}\xspace}
\newcommand{\HalfOpenInterval}[0]{\textbf{\hyperref[def:Interval]{Half-Open Interval}}\xspace}
\newcommand{\HalfOpenIntervals}[0]{\textbf{\hyperref[def:Interval]{Half-Open Intervals}}\xspace}
\begin{df}[\Interval]
\label{def:Interval}
\rm
    Let $V$ be a 
    \VectorSpace 
    over a 
    \Field
    $\F \in \{ \R, \C\}$. 
    Let $x,y \in V$. 
    We define the following sets:
    \begin{align*} 
        [x,y] = \{tx+(1-t)y : t \in [0,1] \} \\
        [x,y) = \{tx+(1-t)y : t \in [0,1) \} \\
        (x,y] = \{tx+(1-t)y : t \in (0,1] \} \\
        (x,y) = \{tx+(1-t)y : t \in (0,1) \}
    \end{align*}
    We refer to any of these sets as 
    \Intervals in $V$.
    Even in the absence of a topological structure, 
    we use the following language:
    \begin{enumerate}
        \item $[x,y]$ is called a \ClosedInterval.
        \item $(x,y)$ is called an \OpenInterval.
        \item $(x,y]$ and $[x,y)$ are called \HalfOpenIntervals or \HalfClosedIntervals.
    \end{enumerate}
\end{df}
