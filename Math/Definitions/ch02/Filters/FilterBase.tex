\newcommand{\FilterBase}[0]{\textbf{\hyperref[def:FilterBase]{Filter Base}}\xspace}
\newcommand{\FilterBases}[0]{\textbf{\hyperref[def:FilterBase]{Filter Bases}}\xspace}
\newcommand{\FilterBaseEquivalent}[0]{\textbf{\hyperref[def:FilterBase]{Equivalent}}\xspace}

\begin{df}[\FilterBase]
\label{def:FilterBase}\rm
    Let $X$ be a nonempty set.
    Let $\scB \subset \scPowerSet{X}$
    such that 
    \begin{enumerate}[label=(\roman*), ref={\ref{def:FilterBase}.~\roman*}]
        \item \label{def:FilterBase:IsNotEmpty}$
        \emptyset \neq \scB$. 
        \item \label{def:FilterBase:DoesntContainEmptySet}$
        \emptyset \not \in \scB$. 
        \item \label{def:FilterBase:IntersectionProperty}
        If $\scB_{Int}$ is the collection of binary intersections of elements of $\scB$, then
		Then \scNested{\scB}{\scB_{Int}} holds. 
    \end{enumerate}
    Then we call 
    $\scB$ 
    a
    \FilterBase
    on $X$. 
    By \ref{prop:FilterBase}, the 
    \Filter
    \FilterGeneratedBy
    a 
    \FilterBase
    $A$ 
    is given by 
    $\{U \subset X : (\exists Y \in A ) ( Y \subset U ) \}$. 
    If $A,B$ are \FilterBases
    on $X$ and they 
    \FilterGenerate
    the same 
    \Filter, 
    then we call them 
    \FilterBaseEquivalent. 
\end{df}

