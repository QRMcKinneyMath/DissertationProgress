\newcommand{\NetSectionFilter}[0]{\textbf{\hyperref[def:SectionFilter]{Section Filter}}\xspace}
\newcommand{\NetSectionFilters}[0]{\textbf{\hyperref[def:SectionFilter]{Section Filters}}\xspace}
\newcommand{\DirectedSectionFilter}[0]{\textbf{\hyperref[def:SectionFilter]{Section Filter}}\xspace}
\newcommand{\DirectedSectionFilters}[0]{\textbf{\hyperref[def:SectionFilter]{Section Filters}}\xspace}
\newcommand{\scSectionFilter}[1]{
    \ensuremath{\scF_{#1}}
}
\begin{df}[\NetSectionFilter]
    \label{def:SectionFilter}
    \rm
    Let $X$ be a nonempty set.
    Let $\sigma=\{x_{\alpha}\}_{\alpha \in A}$ be a 
    \Net in $X$. 
    For each $\alpha \in A$, let $S(\sigma, \alpha)$ 
    denote the \NetSection of $x_{\alpha}$ in $\sigma$. 
    Define $\scB$ by 
    \begin{equation*}
    \scB=\{S(\sigma, \alpha) : \alpha \in A\}
    \end{equation*}
    By \ref{prop:NetSectionsFormFilterBase}, $\scB$ is a 
    \FilterBase on $X$. 
    We call the \Filter \FilterGeneratedBy $\scB$ the 
    \NetSectionFilter of $\sigma$. 
    We call 
    the \NetSectionFilter of the 
    identity \Net in $A$ the \DirectedSectionFilter of $A$. 
    We denote the \DirectedSectionFilter of $A$ with \scSectionFilter{A}.
\end{df}
