\newcommand{\Subgradient}[0]{\textbf{\hyperref[def:Subgradient]{Subgradient}}\xspace}
\newcommand{\Subgradients}[0]{\textbf{\hyperref[def:Subgradient]{Subgradients}}\xspace}
\newcommand{\Subdifferential}[0]{\textbf{\hyperref[def:Subgradient]{Subdifferential}}\xspace}
\newcommand{\Subdifferentiable}[0]{\textbf{\hyperref[def:Subgradient]{Subdifferentiable}}\xspace}
\newcommand{\Subdifferentiability}[0]{\textbf{\hyperref[def:Subgradient]{Subdifferentiability}}\xspace}
\begin{df}[\Subgradient]
\label{def:Subgradient}
\rm
Let $X$ be a \TVS. 

Let $h:X \to (-\infty,\infty]$.
Let $x_0 \in X$. 
Suppose there exists $x^* \in X^*$ such that for every $y \in X$, we have
\begin{equation*}
\ip{y-x,x^*} \leq h(y)-h(x)
\end{equation*}
Then we say that $h$ is 
\Subdifferentiable at $x_0$ and we call
$x^*$ a \Subgradient of $h$ at $x_0$. 
We denote the collection of all \Subgradients of $h$ at $x_0$ with
$\partial h(x_0)$. 
If $D \subset X$ is the set on which $h$ is \Subdifferentiable, 
then we call $\partial h:D \to 2^{X^*}$ the \Subdifferential of $h$.
\end{df}
