\newcommand{\LeastUpperBound}[0]{\textbf{\hyperref[def:LeastUpperBound]{Least Upper Bound}}\xspace}
\newcommand{\LeastUpperBounds}[0]{\textbf{\hyperref[def:LeastUpperBound]{Least Upper Bounds}}\xspace}
\newcommand{\Sup}[0]{\textbf{\hyperref[def:LeastUpperBound]{Sup}}\xspace}
\newcommand{\Supremum}[0]{\textbf{\hyperref[def:LeastUpperBound]{Supremum}}\xspace}
\newcommand{\Suprema}[0]{\textbf{\hyperref[def:LeastUpperBound]{Suprema}}\xspace}
\newcommand{\LUB}[0]{\textbf{\hyperref[def:LeastUpperBound]{LUB}}\xspace}
\begin{df}[\LeastUpperBound]
\label{def:LeastUpperBound}
\rm
    Let $X \neq \emptyset$ be a set. 
    Let $R$ be a \Relation on X. 
    Let $Y \subset X$.
	Let $a \in X$. 
	We say that $a$ is a
	\LeastUpperBound of $Y$ if 
	$a \in \Minima(\UB(Y))$.
	We denote the set of \LeastUpperBounds
	for $Y$ with $\LUB(Y)$.
	If $b \in \LUB(Y)$, then we 
	also call $b$ a 
	\Supremum of $Y$. 
	The Plural of \Supremum is \Suprema.
	If $\LUB(Y)=\{c\}$, 
	then we write $c=\Sup(Y)$. 
\end{df}
