\label{def:LeastUpperBound}

\newcommand{\LeastUpperBound}[0]{
    \bf \hyperref[def:LeastUpperBound]{Least Upper Bound} \rm
}

\newcommand{\LeastUpperBounds}[0]{
    \bf \hyperref[def:LeastUpperBound]{Least Upper Bounds} \rm
}

\newcommand{\Sup}[0]{
    \bf \hyperref[def:LeastUpperBound]{Sup} \rm
}

\newcommand{\Supremum}[0]{
    \bf \hyperref[def:LeastUpperBound]{Supremum} \rm
}

\newcommand{\Suprema}[0]{
    \bf \hyperref[def:LeastUpperBound]{Suprema} \rm
}

\newcommand{\LUB}[0]{
	\bf \hyperref[def:LeastUpperBound]{LUB} \rm
}

\begin{df}[\LeastUpperBound]
    Let $X \neq \emptyset$ be a set. 
    Let $R$ be a \Relation on X. 
    Let $Y \subset X$.
	Let $a \in X$. 
	We say that $a$ is a
	\LeastUpperBound of $Y$ if 
	$a \in \Minima(\UB(Y))$.
	We denote the set of \LeastUpperBounds
	for $Y$ with $\LUB(Y)$.
	If $b \in \LUB(Y)$, then we 
	also call $b$ a 
	\Supremum of $Y$. 
	The Plural of \Supremum is \Suprema.
	If $\LUB(Y)=\{c\}$, 
	then we write $c=\Sup(Y)$. 
\end{df}