\newcommand{\SeminormTopology}[0]{\textbf{\hyperref[def:seminormtopology]{Seminorm Topology}}\xspace}
\newcommand{\SeminormInducedPseudometric}[0]{\textbf{\hyperref[def:seminormtopology]{Pseudometric induced by the Seminorm}}\xspace}
\newcommand{\SeminormSpaceInducedPseudometricSpace}[0]{\textbf{\hyperref[def:seminormtopology]{Pseudometric Space induced by the Seminormed Space}}\xspace}

\begin{df}[\SeminormTopology]
\label{def:seminormtopology}
\rm
    Let $(X,\norm{\cdot})$ be a \SeminormedSpace.
    define $d_{\norm{\cdot}}:V \times V \to [0,\infty)$  by setting,
    for $x,y \in X$, 
    \begin{equation*}
    d_{\norm{\cdot}}(x,y) = \norm{x-y}
    \end{equation*}
    Observe the following: 
    \begin{enumerate}
        \item \ref{rmk:seminorm} guarantees that $d_{\norm{\cdot}}(x,x)=0$ for $x \in X$. 
        \item 
        \ref{prop:subadditiveinducestriangleinequality} guarantees that d satisfies the \TriangleInequality. 
        \item d is \CommutativeFunction, as we have 
    \begin{equation*}
        d(x,y)_{\norm{\cdot}}=\norm{x-y}=|-1|\norm{x-y}=\norm{y-x}=d(y,x)
    \end{equation*}
    \end{enumerate}

    Hence, $d_{\norm{\cdot}}$  is a \Pseudometric on X, which we call the \SeminormInducedPseudometric on X. 
    We refer to $(X, d_{\norm{\cdot}})$ as the \SeminormSpaceInducedPseudometricSpace $(X,\norm{\cdot}$. 
    We refer to the \PseudometricTopology induced by $d_{\norm{\cdot}}$ as the \SeminormTopology induced by $\norm{\cdot}$, and unless otherwise specified, when we reference $(X,\norm{\cdot})$, we consider it to be endowed with this topology. 

\end{df}
