\label{def:UpperBound}
\newcommand{\UpperBound}[0]{\textbf{\hyperref[def:UpperBound]{Upper Bound}}\xspace}
\newcommand{\UpperBounds}[0]{\textbf{\hyperref[def:UpperBound]{Upper Bounds}}\xspace}
\newcommand{\BoundedFromAbove}[0]{\textbf{\hyperref[def:UpperBound]{Bounded From Above}}\xspace}
\newcommand{\UB}[0]{\textbf{\hyperref[def:UpperBound]{UpperBound}}\xspace}
\begin{df}[\UpperBound]
    Let $X \neq \emptyset$ be a set. 
    Let $R$ be a \Relation on X. 
    Let $Y \subset X$.
    Let $a \in X$. 
    We say that $a$ is an 
    \UpperBound for $Y$ if
    for every $x \in Y$, 
    we have $x R a$. 
    If $a$ is an \UpperBound
    then we also say that 
    the set Y is \BoundedFromAbove
    by a. 
	We denote the set of \UpperBounds of 
	$Y$ with respect to the relation $R$ with
	$\UB_R(Y)$. 
	When $R$ is understood, we denote this set with
	$\UB(Y)$. 
\end{df}
