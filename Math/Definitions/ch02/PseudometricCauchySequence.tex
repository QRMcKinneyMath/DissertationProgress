\label{def:pseudometriccauchysequence}
\newcommand{\PseudometricCauchySequence}[0]{
    \bf \hyperref[def:pseudometriccauchysequence]{Pseudometric Cauchy Sequence} \rm
}
\begin{df}[Pseudometric Cauchy Sequence]

    Let $(X,d)$ be a \PseudometricSpace.
    We say that a sequence $\{x_i\}_{i \in \N}$ is a \PseudometricCauchySequence
    if, for each $\epsilon > 0$, there exists an $N \in \N$, sucht that for 
    each pair $m,n \in \N$ such that $m>N$ and $n>N$, we have 
    \begin{equation}
        d(x_m,x_n) < \epsilon
    \end{equation}
\end{df}
\label{def:uniformlycauchy}
\newcommand{\UniformlyCauchy}[0]{
    \bf \hyperref[def:uniformlycauchy]{Uniformly Cauchy} \rm
}
\begin{df}[Uniformly Cauchy]
	Let $(X_\alpha, d_\alpha)$ be a \PseudometricSpace
	for $\alpha \in A$ where A is some indexing set. 
	For each $\alpha \in A$
	, let $\phi_\alpha :=\{x_i^\alpha\}_{i \in \N} \subset X_{\alpha}$
	be a sequence. 
	We say that the collection $\{\phi_\alpha\}_{\alpha \in A}$ 
	is 
	\UniformlyCauchy if for each $\epsilon > 0$, there exists an 
	$N \in \N$ such that for each pair $m,n \in N$
	such that $m>N$ and $n>N$, and for each $\alpha \in A$, 
	we have 
	\begin{equation}
	d_{\alpha} \pa{x^{\alpha}_n, x^{\alpha}_m} < \epsilon
	\end{equation}
\end{df}