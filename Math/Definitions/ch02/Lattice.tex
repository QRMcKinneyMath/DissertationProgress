\label{def:Lattice}
\newcommand{\Lattice}[0]{
    \textbf{\hyperref[def:Lattice]{Lattice}}
}
\newcommand{\Lattices}[0]{
    \textbf{\hyperref[def:Lattice]{Lattices}}
}
\newcommand{\CompleteLattice}[0]{
    \textbf{\hyperref[def:Lattice]{Complete Lattice}}
}
\newcommand{\CompleteLattices}[0]{
    \textbf{\hyperref[def:Lattice]{Complete Lattices}}
}
\newcommand{\LatticeJoin}[0]{
    \textbf{\hyperref[def:Lattice]{Join}}
}
\newcommand{\LatticeJoins}[0]{
    \textbf{\hyperref[def:Lattice]{Joins}}
}    
\newcommand{\LatticeMeet}[0]{
    \textbf{\hyperref[def:Lattice]{Meet}}
}
\newcommand{\LatticeMeets}[0]{
    \textbf{\hyperref[def:Lattice]{Meets}}
}    
\begin{df}[\Lattice, \LatticeJoin, \LatticeMeet]
    Let $(X, \leq)$ be a 
    \Poset
    such that, 
    for every $x,y \in X$, 
    the set 
    $\{x,y\}$ has both a 
    \Supremum 
    and an
    \Infimum. 
    Then we call $(X,\leq)$ 
    \Lattice. 
    Furthermore, we call 
    $\Sup\{x,y\}$
    the 
    \LatticeJoin of $x$ and $y$ 
    and we call
    $\Inf\{x,y\}$ the 
    \LatticeMeet
    of $x$ and $y$. 
    If every nonempty subset of 
    $X$ has both a 
    \Supremum
    and \Infimum 
    then we call $(X,\leq)$
    a \CompleteLattice. 
\end{df}
