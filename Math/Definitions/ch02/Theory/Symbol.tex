\newcommand{\Symbol}[0]{\textbf{\hyperref[def:Symbol]{Symbol}}\xspace}
\newcommand{\Symbols}[0]{\textbf{\hyperref[def:Symbol]{Symbols}}\xspace}
\newcommand{\SymbolEqual}[0]{\hyperref[def:Symbol]{\ensuremath{=_{\Symbol}}}\xspace}
\newcommand{\SymbolOccurence}[0]{\textbf{\hyperref[def:Symbol]{Symbol Occurence}}\xspace}
\newcommand{\SymbolOccurences}[0]{\textbf{\hyperref[def:Symbol]{Symbol Occurences}}\xspace}
\begin{df}[\Symbol]
\label{def:Symbol}

\rm
    The \Symbol is the atomic primitive of language theory. 
    Multiple instances of the same \Symbol are interchangeable. 
    Different \Symbols can be identified as such. 
    Within a certain context, 
    a character or sequence of characters 
    can represent a \Symbol.
    When we say that a character or a sequence 
    of characters is a \Symbol, 
    we mean that that character or sequence 
    of characters represents a \Symbol. 
    If $c$ and $b$ are both used to represent 
    the same symbol in a certain context, 
    then we write $c \SymbolEqual b$.
    and we either verbally describe the context in which that relationship
    is meant to hold, or we hope and pray that context makes it clear.
    Different instances of the same \Symbol 
    will sometimes be distinguished from each other, 
    which we will refer to as \SymbolOccurences.
\end{df}
