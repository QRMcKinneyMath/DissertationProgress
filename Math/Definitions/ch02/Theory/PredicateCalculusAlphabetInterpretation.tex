\newcommand{\Implication}[0]{\textbf{\hyperref[def:PredicateCalculusSymbols]{Implication}}\xspace}
\newcommand{\Negation}[0]{\textbf{\hyperref[def:PredicateCalculusSymbols]{Negation}}\xspace}
\newcommand{\LeftParenthesis}[0]{\textbf{\hyperref[def:PredicateCalculusSymbols]{Left Parenthesis}}\xspace}
\newcommand{\RightParenthesis}[0]{\textbf{\hyperref[def:PredicateCalculusSymbols]{Right Parenthesis}}\xspace}
\newcommand{\Comma}[0]{\textbf{\hyperref[def:PredicateCalculusSymbols]{Comma}}\xspace}
\newcommand{\UniversalQuantifier}[0]{\textbf{\hyperref[def:PredicateCalculusSymbols]{Universal Quantifier}}\xspace}
\newcommand{\IndividualVariable}[0]{\textbf{\hyperref[def:PredicateCalculusSymbols]{Individual Variable}}\xspace}
\newcommand{\IndividualVariables}[0]{\textbf{\hyperref[def:PredicateCalculusSymbols]{Individual Variables}}\xspace}
\newcommand{\IndividualConstant}[0]{\textbf{\hyperref[def:PredicateCalculusSymbols]{Individual Constant}}\xspace}
\newcommand{\IndividualConstants}[0]{\textbf{\hyperref[def:PredicateCalculusSymbols]{Individual Constants}}\xspace}
\newcommand{\FunctionLetter}[1]{\textbf{\hyperref[def:PredicateCalculusSymbols]{Function Letter Of Order #1}}\xspace}
\newcommand{\FunctionLetters}[1]{\textbf{\hyperref[def:PredicateCalculusSymbols]{Function Letters Of Order #1}}\xspace}
\newcommand{\PredicateLetter}[1]{\textbf{\hyperref[def:PredicateCalculusSymbols]{Predicate Letter Of Order #1}}\xspace}
\newcommand{\PredicateLetters}[1]{\textbf{\hyperref[def:PredicateCalculusSymbols]{Predicate Letters Of Order #1}}\xspace}
\newcommand{\PredicateCalculus}[0]{\textbf{\hyperref[def:PredicateCalculusSymbols]{Predicate Calculus}}\xspace}



\begin{df}[\Symbols of the \PredicateCalculus]
\label{def:PredicateCalculusSymbols}

\rm
    Throughout this section, we let \scA be an \Alphabet.
    The purpose of this section is to define a \Grammar \scG on \scA. 
    We start our definition of \scG by partitioning \scA in the following way
    \begin{enumerate}
        \item The singleton containing the \Implication \Symbol, which we represent with the \Character $\implies$. 
        \item The singleton containing the \Negation \Symbol which we represent with the \Character $\neg$. 
        \item The singleton containing the \LeftParenthesis \Symbol, which we represent with the \Character $($. 
        \item The singleton contianing the \RightParenthesis \Symbol, whcih we represent with the \Character $)$. 
        \item The singleton containing the \Comma \Symbol, which we represent with the \Character $,$. 
        \item The signleton containing the \UniversalQuantifier, which we represent witht he \Character $\forall$. 
        \item For each counting number n, the set of \FunctionLetters{n}, each of which may be empty.
        \item For each counting number n, the set of \PredicateLetters{n}, which is nonempty for at least 1 value of n. 
        \item The set of \IndividualVariables, which may be empty. 
        \item The set of \IndividualConstants, which may be empty. 
    \end{enumerate}

    Following the rules defined in the following steps, this partitining of 
    \Symbols will fully define \scG. 
    
\end{df}
