\newcommand{\String}[0]{\textbf{\hyperref[def:String]{String}}\xspace}
\newcommand{\Strings}[0]{\textbf{\hyperref[def:String]{Strings}}\xspace}
\newcommand{\StringEqual}[0]{\hyperref[def:String]{\ensuremath{=_{\String}}}\xspace}
\begin{df}[\String]
\label{def:String}

\rm
    A \String in an \Alphabet $A$ is a finite ordered list of \Symbols in $A$.
    The number of \Symbols in a string is called the length of that string.
    A string can contain multiple instances of the same \Symbol.
    A \String may be represented by a character, or a sequence of characters.
    If we say that a \Character or sequence of \Characters is a \String, then we mean that
    that \Character or sequence of \Characters represents a \String.
    Suppose $x$ is a \String of length m whose $i^{th}$ element
    is the \Symbol represented by $x_i$.
    Then, for every counting number i between 1 and m.
    Then we write $x_i \SymbolEqual x(i)$.
    If $x$ and $y$ each are strings of length $m$ such that for 
    $1 \leq i \leq m$ we have 
    $x(i) \SymbolEqual  y(i)$, then we write
    $x \StringEqual y$. 
	If $x$ is a \String of length m whose $i^{th}$ elemnt
	is the \Symbol represented by the \Character $x_i$, then we write
	$x \StringEqual x_1x_2x_3x_4x_5...x_{m-2}x_{m-1}x_m$
	
\end{df}
