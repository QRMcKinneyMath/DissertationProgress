\newcommand{\AppearsIn}[0]{\textbf{\hyperref[def:AppearsIn]{Appears In}}\xspace}
\newcommand{\AppearIn}[0]{\textbf{\hyperref[def:AppearsIn]{Appear In}}\xspace}
\newcommand{\Substring}[0]{\textbf{\hyperref[def:AppearsIn]{Substring}}\xspace}
\newcommand{\Substrings}[0]{\textbf{\hyperref[def:AppearsIn]{Substrings}}\xspace}
\newcommand{\StringOccurence}[0]{\textbf{\hyperref[def:AppearsIn]{Occurence}}\xspace}
\newcommand{\StringOccurences}[0]{\textbf{\hyperref[def:AppearsIn]{Occurences}}\xspace}
\begin{df}[\AppearsIn]
\label{def:AppearsIn}

\rm
    Let \scA be an \Alphabet.
    Let $x$ be a \String of length n in \scA. 
    Let $y$ be a \String of length m in \scA. 
    We say that $x$ 
    \AppearsIn
    y
    if there exists a $k >0$ such that 
    for each $i$, $1 \leq i \leq n$, we have 
    \begin{equation*}
        x(i) \SymbolEqual  y(i+k)
    \end{equation*}
    If $x$ \AppearsIn $y$
    then we may also say that 
    $x$ is a \Substring of $y$. 
    It is possible for a single \String
    to \AppearIn another string in multiple locations. 
    We call disjoint appearances of the same \String 
    in another a \StringOccurence of that \Substring.

\end{df}
