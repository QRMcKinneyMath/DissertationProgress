\label{def:LowerBound}
\newcommand{\LowerBound}[0]{
    \bf \hyperref[def:LowerBound]{Lower Bound} \rm
}

\newcommand{\BoundedFromBelow}[0]{
    \bf \hyperref[def:LowerBound]{Bounded From Below} \rm
}
\newcommand{\LowerBounds}[0]{
	\bf \hyperref[def:LowerBound]{Lower Bounds} \rm
}
\newcommand{\LB}[0]{
	\bf \hyperref[def:LowerBound]{LowerBound} \rm
}

\begin{df}[\LowerBound]
    Let $X \neq \emptyset$ be a set. 
    Let $R$ be a \Relation on X. 
    Let $Y \subset X$.
    Let $a \in X$. 
    We say that $a$ is an 
    \LowerBound for $Y$ if
    for every $x \in Y$, 
    we have $a R x$. 
    If $a$ is an \LowerBound
    then we also say that 
    the set Y is \BoundedFromBelow
    by a. 
	We denote the set of \LowerBounds of 
	$Y$ with respect to the relation $R$ with
	$\LB_R(Y)$. 
	When $R$ is understood, we denote this set with
	$\LB(Y)$. 
\end{df}