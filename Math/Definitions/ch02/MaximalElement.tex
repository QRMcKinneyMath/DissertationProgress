\newcommand{\MaximalElement}[0]{\textbf{\hyperref[def:MaximalElement]{Maximal Element}}\xspace}
\newcommand{\Maximum}[0]{\textbf{\hyperref[def:MaximalElement]{Maximum}}\xspace}
\newcommand{\Maxima}[0]{\textbf{\hyperref[def:MaximalElement]{Maxima}}\xspace}

\begin{df}[\MaximalElement]
\label{def:MaximalElement}
\rm
    Let $X \neq \emptyset$ be a set. 
    Let $R$ be an 
    \Relation on $X$. 
    Let $Y \subset X$.
    Let $a \in Y$. 
    We say that $a$ is a 
    \MaximalElement of $Y$, 
    or equivalently we say that 
    $a$ is a \Maximum of Y 
    if for every $b \in Y$, 
    if $a R b$, then $a=b$. 
	The Plural of \Maximum is a \Maxima, 
	and we represent the set of \Maxima of Y with 
	respect to the relation $R$ with 
	$\Maxima_R(Y)$, or if $R$ is understood, 
	we represent the set of $\Maxima$ of Y with 
	$\Maxima(Y)$. 
	
	
\end{df}

\begin{prop}[\MaximalElement unique if R is \AntiSymmetricRelation]
\label{prop:MaximalElementUniqueIfAntisymmetric}
\rm
	Let $X \neq \emptyset$ be a set. 
	Let $R$ be an
	\AntiSymmetricRelation
	\Relation
	on X. 
	Let $Y \subset X$.
	Let $a$ and $b$ be 
	each be a \MaximalElement
	of Y. 
	Then $a=b$. 
	\begin{proof}
		Since $a \in \Maxima(Y)$, 
		$b \leq a$. 
		Since $b \in \Maxima(Y)$, 
		$a \leq b$. 
		By \RelationAntiSymmetry, 
		$b = a$. 
	\end{proof}
\end{prop}
