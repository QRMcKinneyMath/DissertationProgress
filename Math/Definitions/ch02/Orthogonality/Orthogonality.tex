\begin{rmk}
\rm
Birkhoff-James orthogonality, first introduced by Birkhoff in \cite{birkhoff35}
and then studied in detail by James in \cite{james47}, 
is a generalization of the notion of orthogonality present in inner-product spaces
to the context of a \NormedSpace. 
In these papers and others about this concept since then, the unnecessary assumption of a \Hausdorff space has been made. 
I introduce this concept in a non-\Hausdorff space.
In general, orthogonality is not as well behaved in a \SeminormedSpace as it is 
in an inner-product space, but it retains interesting interactions with the 
smoothness and convexity of a space.
\end{rmk}
\newcommand{\Orthogonal}[0]{\textbf{\hyperref[def:Orthogonal]{Orthogonal}}\xspace}
\newcommand{\Orthogonality}[0]{\textbf{\hyperref[def:Orthogonal]{Orthogonality}}\xspace}
\begin{df}[\Orthogonal]
\label{def:Orthogonal}
\rm
Let $X$ be a \SeminormedSpace over $\F$.
Let $x_0, y_0 \in X$. 
We say that $x_0$ is \Orthogonal to $y_0$ and we write $x_0 \perp y_0$. 
if, for each $\lambda \in \F$, we have
\begin{equation*}
\norm{x_0} \leq \norm{x_0+\lambda y_0} 
\end{equation*} 
\end{df}
