\begin{prop}[Locally Uniformly Convex Spaces]
\label{prop:LocallyUniformlyConvex}
\rm
Let $X$ be a \SeminormedSpace. 
The following are equivalent.
\begin{enumerate}[label=(\roman*), ref={\ref{prop:LocallyUniformlyConvex}~\roman*}]
\item 
\label{prop:LocallyUniformlyConvex:Convex}
$X$ is \LocallyUniformlyConvex.
\item 
\label{prop:LocallyUniformlyConvex:EpsilonDelta}
For each $\epsilon > 0$ and for each  $x\in \partial B_X(0;1)$, there is a $\delta > 0$ such that if $y \in \partial B_X(0;1)$ and $\norm{x-y} \geq \epsilon$, then $\norm{x+y} \leq 2(1-\delta)$. 
\item 
\label{prop:LocallyUniformlyConvex:Convergence}
If $x \in \partial B_X(0;1)$, $\{x_n\}_{n \in \N} \subset \partial B_X(0;1)$, and $\norm{x+x_n} \to 2$, then $x_n \to x$. 
\end{enumerate} 
%Cioranescu 2.2.3, 2.2.6, 2.2.4, Cioranescu 2.2.11
\begin{proof}[Proof of \ref{prop:LocallyUniformlyConvex:Convex} implies \ref{prop:LocallyUniformlyConvex:EpsilonDelta}]
Let $X$ be 
\LocallyUniformlyConvex.
Let $\epsilon \in (0,2)$.
Let $x_0 \in X$ with $\norm{x_0} = 1$. 
Then $\tilde{\Delta}(\epsilon, x_0) > 0$. 
Set $\delta = \tilde{\Delta}(\epsilon, x_0)$.
Let $y \in \partial B_X(0;1)$ and 
$\norm{x-y} \geq \epsilon$.
Then $y \in X \setminus B_X(x;\epsilon)$. 
Hence, 
\begin{equation*}
\tilde{\Delta}(\epsilon, x_0) \leq 1 - \norm{\frac{x+y}{2}}
\end{equation*}
Letting $\delta = \tilde{\Delta}(\epsilon, x_0)$ and multiplying 
by two, we have 
\begin{equation*}
\norm{x+y} \leq 2\pa{1-\delta}
\end{equation*}
\end{proof}
\begin{proof}[Proof of \ref{prop:LocallyUniformlyConvex:EpsilonDelta} implies \ref{prop:LocallyUniformlyConvex:Convergence}]
For each $n \in \N$ there exists $\delta_n > 0$ such that 
if $y \in \partial B_X(0;1)$ then
$\norm{x+y}  > 2(1-\delta_n)$ implies $\norm{x-y} < \frac{1}{n}$. 
Thus if $\norm{x+x_n} \to 2$ then we have $\norm{x-x_n} \to 0$. 
\end{proof}
\begin{proof}[Proof of \ref{prop:LocallyUniformlyConvex:Convergence} implies \ref{prop:LocallyUniformlyConvex:Convex}]
Suppose otherwise. 
Then we could find $x_0 \in X$, $\epsilon > 0$, and 
$\{x_i\}_{i \in \N} \subset \pa{X \setminus B_X(x;\epsilon)} \cap \partial B_X(0;1)$
such that $\norm{\frac{x+x_n}{2}} \to 2$.
This is clearly a contradiction. 
\end{proof}
\end{prop}