\begin{prop}[Weakly Uniformly Convex Spaces]
\label{prop:WeaklyUniformlyConvex}
\rm
Let $X$ be a \SeminormedSpace.
The following are equivalent.
\begin{enumerate}[label=(\roman*), ref={\ref{prop:WeaklyUniformlyConvex}~\roman*}]
\item 
\label{prop:WeaklyUniformlyConvex:Convex}
$X$ is \WeaklyUniformlyConvex.
\item 
\label{prop:WeaklyUniformlyConvex:EpsilonDelta}
For each $\epsilon > 0$ and $x^* \in \overline{ B_{X^*}(0;1)}$, there is a $\delta > 0$ such that if $x,y \in \partial B_X(0;1)$ 
such that $\norm{x-y} > \epsilon$ then $\ip{x+y,x^*} \leq 2(1-\delta)$.
\item 
\label{prop:WeaklyUniformlyConvex:Convergence}
If $\{x_n\}_{n \in \N} , \{y_n\}_{n \in \N} \subset \overline{B_X(0;1)}$ and $\norm{x_n+y_n} \to 2$, then $x_n-y_n \overset{w}{\to} 0$. 
This One really is an open question... Gotta look into it. 
\item
\label{prop:WeaklyUniformlyConvex:WeakConvergence}
For every $x^* \in \partial B_{X^*}(0;1)$, if 
If $\{x_n\}_{n \in \N} \subset \overline{B_X(0;1)}$ and 
$\{y_n\}_{n \in \N} \subset \overline{B_X(0;1)}$ such that 
$\ip{x_k+y_k, x^*} \to 2$, then $x_n -y_n \to 0$. 
\end{enumerate}
\begin{proof}[Proof of \ref{prop:WeaklyUniformlyConvex:Convex} implies \ref{prop:WeaklyUniformlyConvex:EpsilonDelta}]
Let $X$ be \WeaklyUniformlyConvex.
Let $\epsilon > 0$. 
Let $x^* \in \overline{B_{X^*}(0;1)}$. 
Define $\delta = \Delta_w(\epsilon, x^*)>0$. 
Let $x \in \partial B_X(0;1)$. 
Let $y \in \partial B_X(0;1)$ such that $\norm{x-y} > \epsilon$. 
Then $y \in X \setminus B_X(x;\epsilon)$. 
Hence  
\begin{equation*}
1-\ip{\frac{x+y}{2}, x^*} \geq \tilde{\Delta}_w(\epsilon,x,x^*)  \geq \Delta_w(\epsilon,x^*) = \delta
\end{equation*}
Solving, we have 
\begin{equation*}
2(1-\delta) \geq \ip{x+y, x^*}
\end{equation*}
\end{proof}
\begin{proof}[Proof of \ref{prop:WeaklyUniformlyConvex:EpsilonDelta} implies \ref{prop:WeaklyUniformlyConvex:WeakConvergence}]
Let $x^* \in \partial B_{X^*}(0;1)$. 
Let $\{x_n\}_{n \in \N} \subset \overline{B_X(0;1)}$ 
and let $\{y_n\}_{n \in \N} \subset \overline{B_X(0;1)}$
such that $\ip{x_k+y_k,x^*} \to 2$. 
Let $\epsilon > 0$. 
Then there exists $\delta > 0$ such that 
if $x,y \in \overline{B_X(0;1)}$ and $\norm{x+y} > 2-2\delta$ 
Then $\norm{x-y} < \epsilon$. 
Hence $\norm{x_k-y_k} \to 0$ as $k \to \infty$. 
\end{proof}
\begin{proof}[Proof of \ref{prop:WeaklyUniformlyConvex:WeakConvergence} implies \ref{prop:WeaklyUniformlyConvex:Convex}]
I prove by converse. 
Suppose $X$ is not \WeaklyUniformlyConvex.
Then for some $x^* \in  \partial B_{X^*}(0;1)($, 
$X$ is not \WeaklyUniformlyConvex at $x^*$. 
For this $x^*$ there exists an $\epsilon > 0$
and a sequence $\{x_i\}_{i \in \N} \subsety \partial B_X(0;1)$
such that $\tilde{\Delta}(\epsilon, x_i, x^*) \searrow 0$ 
Hence, there is a corresponding sequence 
$\{y_i\}_{i \in \N} \subset \partial B_X(0;1)$ such that $\norm{x_i-y_i} \geq \epsilon$ 
for every $i$ and $1-\ip{\frac{x_i+y_i}{2}, x^*} \to 0$. 
That is, $\ip{\frac{x_i+y_i}{2}, x^*} \to 2$, but $x_i-y_i \not \to 0$. 
\end{proof}
\end{prop} 