\begin{prop}[Uniformly Convex Spaces]
\label{prop:UniformlyConvexSpace}
\rm
Let $X$ be a \SeminormedSpace.
The following conditions are equivalent. 
\begin{enumerate}[label=(\roman*), ref={\ref{prop:UniformlyConvexSpace}~\roman*}]
\item 
\label{prop:UniformlyConvexSpace:Convex}
X is \UniformlyConvex.
\item 
\label{prop:UniformlyConvexSpace:EpsilonDelta}
For each $\epsilon > 0$, there is a $\delta > 0$ such that if $x,y \in \overline{B_X(0;1)}$ and $\norm{x-y} \geq \epsilon$, then $\norm{x+y} \leq 2(1-\delta)$. 
\item 
\label{prop:UniformlyConvexSpace:Convergence}
If $\{x_i\}_{i \in \N} , \{y_i\}_{i \in \N} \subset \overline{B_X(0;1)}$ and $\norm{x_n+y_n} \to 2$, then $x_n-y_n \to 0$. 
\end{enumerate} 
\begin{proof}[Proof of \ref{prop:UniformlyConvexSpace:Convex} implies \ref{prop:UniformlyConvexSpace:EpsilonDelta}]
Let $X$ be \UniformlyConvex. 
Let $\epsilon > 0$. 
Define $\delta = \Delta(\epsilon)$. 
Let $x \in \overline{B_X(0;1)}$
Let $y \in \overline{B_X(0;1)}$
such that $\norm{x-y} \geq \epsilon$.
Then $0 < \delta = \Delta(\epsilon) \leq \tilde{\Delta}(\epsilon, x)  \leq 1- \norm{\frac{x+y}{2}}$. 
Hence, $\norm{x+y} \leq 2\pa{1-\delta}$.
\end{proof}
\begin{proof}[Proof of \ref{prop:UniformlyConvexSpace:EpsilonDelta} implies \ref{prop:UniformlyConvexSpace:Convergence}]
Let $\{x_i\}_{i \in \N},\{y_i\}_{i \in \N} \subset \overline{B_X(0;1)}$ 
such that $\norm{x_n+y_n} \to 2$. 
Let $\epsilon > 0$. 
Then there exists a $\delta > 0$ such that 
if $x,y  \in \overline{B_X(0;1)}$ and $\norm{x-y} \geq \epsilon$
then $\norm{x+y} \leq 2(1-\delta)=2-2\delta$. 
Since $\norm{x_n+y_n} \to 2$, there exists $K \in \N$ such that 
for $k > K$, 
$\norm{x_k+y_k} > 2\pa{1-\delta}$.
This then implies $\norm{x_k-y_k}  < \epsilon$, so $x_k-y_k \to 0$. 
\end{proof}
\begin{proof}[Proof of \ref{prop:UniformlyConvexSpace:Convergence} implies \ref{prop:UniformlyConvexSpace:Convex}]
Suppose $X$ is not \UniformlyConvex.
Then there exists an $\epsilon > 0$ 
and a sequence $\{x_i\}_{i \in \N} \subset \partial B_X(0;1)$ such that 
$\tilde{\Delta}(\epsilon, x_i) \searrow 0$. 
hence, there exists a corresponding sequence $\{y_i\}_{i \in \N} \subset \partial B_X(0;1)$ 
such that $\norm{x_i-y_i} \geq \epsilon $ for every $i$
and $1-\norm{\frac{x_i+y_i}{2}} \to 0$. 
That is, $\norm{x_i+y_i} \to 2$, but $x_i-y_i \not \to 0$. 
\end{proof}
\end{prop} 