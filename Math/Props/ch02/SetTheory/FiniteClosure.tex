\begin{prop}[Binary to Finite]
\label{prop:FiniteClosure}
\rm
    Let $X$ be a nonempty set. 
    The following are true. 
    \begin{enumerate}[label=(\roman*), ref={\ref{prop:FiniteClosure}.~\roman*}]
        \item \label{prop:FiniteClosure:Intersection}
        If $X$ is closed under binary intersections, then $X$ is closed under
        finite intersections.
        \item \label{prop:FiniteClosure:Union}
        If $X$ is closed under binary unions, 
        then $X$ is closed under finite unions.
    \end{enumerate}
    \begin{proof}[Proof of \ref{prop:FiniteClosure:Intersection}]
        We use induction.
        Let $M$ be the set of positive integers $n$ for which
        $X$ is closed under intersections of n sets. 
        The intersection of a single set equals that set, so $1 \in M$. 
        $2 \in M$ by direct application of the assumption of 
        \ref{prop:FiniteClosure:Intersection}. 
        Let $m \in M$. Let $\{x_i\}_{i=1}^{m+1} \subset 2^X$. 
        Then 
        \begin{equation*}
            \bigcap\limits_{i=1}^{m+1} x_i = \pa{ \bigcap\limits_{i=1}^m x_i} \cap x_{m+1} \in X
        \end{equation*}
        so $m+1 \in M$. 
        Hence $M= \Z^+$ and \ref{prop:FiniteClosure:Intersection} is proven. 
    \end{proof}
    \begin{proof}[Proof of \ref{prop:FiniteClosure:Union}]
         We use induction.
        Let $M$ be the set of positive integers $n$ for which
        $X$ is closed under unions of n sets. 
        The union of a single set equals that set, so $1 \in M$. 
        $2 \in M$ by direct application of the assumption of 
        \ref{prop:FiniteClosure:Union}. 
        Let $m \in M$. Let $\{x_i\}_{i=1}^{m+1} \subset 2^X$. 
        Then 
        \begin{equation*}
            \bigcup\limits_{i=1}^{m+1} x_i = \pa{ \bigcup\limits_{i=1}^m x_i} \cup x_{m+1} \in X
        \end{equation*}
        so $m+1 \in M$. 
        Hence $M= \Z^+$ and \ref{prop:FiniteClosure:Union} is proven. 
    \end{proof}
\end{prop}
