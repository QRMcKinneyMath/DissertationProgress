\begin{prop}[Metric Space Induced By Pseudometric Space]
    \label{prop:pseudometricinducedmetric}
    \rm
    Let $(X,d)$ be a \PseudometricSpace, $\cong$ the \RelationOfZeroDistance on $(X,d)$ and $\tilde{d}$ be defined as in \ref{def:pseudometricinducedmetric}.
    Let $(X/\cong, \T_{X/\cong})$ be the  \QuotientTopologicalSpace with \QuotientMap T, and let $(X/\cong, \T_{\tilde{d}})$ be the \TopologicalSpace induced by the \MetricSpace  $(X/\cong, \tilde{d})$. 
    The following are true. 
    \begin{enumerate}[label=(\roman*), ref={\ref{prop:pseudometricinducedmetric}~\roman*}]
        \item 
        \label{def:MFPM:IsMetric}
        $\tilde{d}$ is a well defined
        \Metric on $X/\cong$. 
        \item 
        \label{def:MFPM:TopolgiesEqual}
        $\T_{X/\cong} = \T_{\tilde{d}}$
        \item 
        \label{def:MFPM:IsIsometricSurjection}
        T is an \Isometric \Surjection of $(X,d)$ 
        onto $(X/\cong, \tilde{d})$
		\item
		\label{def:MFPM:Injection}
		$(X,d)$ is a \MetricSpace if and only if 
		$T$ is \Injective.
        \item 
        \label{def:MFPM:CompleteInherited}
        $(X/\cong, \tilde{d})$ is complete if and only if 
        $(X, d)$ is \PseudometricComplete.
    \end{enumerate}
    \begin{proof}[Proof of \ref{def:MFPM:IsMetric}]
        First we show that $\tilde{d}$ is well defined as a function, that is, that if
        $x_0,y_0 \in X$ and $x_1 \cong x_0$ and $y_1 \cong y_0$, then we should have 
        \begin{equation*}
            \tilde{d}\pa{\bra{x_0},\bra{y_0}}=\tilde{d}\pa{\bra{x_1},\bra{y_1}}
        \end{equation*}
        
        This is easy, as
        \begin{align*}
            d(x_0,y_0) & \leq d(x_0,x_1)+d(x_1,y_1)+d(y_1,y_0)\\
            & = d(x_1,y_1)\\
            & \leq d(x_1,x_0)+d(x_0,y_0)+d(y_0,y_1)\\
            &=d(x_0,y_0)
         \end{align*}
         Nonnegativity falls directly from the nonnegativity of d. 
         Proving that $\tilde{d}$ is \CommutativeFunction is equally trivial
         \begin{align*}
             \tilde{d}\pa{\bra{x}, \bra{y}}= d(x,y) = d(y,x) = \tilde{d}\pa{\bra{y}, \bra{x}}
         \end{align*}
         Proving that $\tilde{d}$ satisfies the \TriangleInequality is similarly simple, letting $x_0,y_0,z_0 \in X$, we have
         \begin{align*}
             \tilde{d}\pa{\bra{x_0}, \bra{z_0}} & = d(x_0,z_0) \\
             & \leq d(x_0, y_0)+d(y_0, z_0)\\
             & = \tilde{d}\pa{\bra{x_0}, \bra{y_0}}+ \tilde{d}\pa{ \bra{y_0}, \bra{z_0}}
         \end{align*}
         
         All that remains is to show positivity on nonequal arguements. Let $x_0, y_0 \in X$ such that $\bra{x_0} \neq \bra{y_0}$. Then $x_0 \not \cong y_0$. Hence \begin{equation*}
             \tilde{d}\pa{\bra{x_0}, \bra{y_0}}=d(x_0,y_0) \neq 0
             \end{equation*}
    \end{proof}
    \begin{proof}[Proof of \ref{def:MFPM:TopolgiesEqual}]
        By \ref{prop:QST:Basis}, $\scB_{\cong}:=\{T(B_d(x;\epsilon)) : x \in X, \epsilon > 0\}$ is a \TopologyBasis for $\T_{X/\cong}$. 
        By definition, $\scB_{\tilde{d}}:=\{B_{\tilde{d}}(\bra{x}; \epsilon) : x \in X , \epsilon > 0 \}$ is a \TopologyBasis for $\T_{\tilde{x}}$. 
        
        I claim that for each $x \in X$ and $\epsilon > 0$, 
        \begin{equation}
            T\pa{B_d(x;\epsilon)} = B_{\tilde{d}} \pa{\bra{x}; \epsilon}
        \end{equation}
        To see this, 
        suppose $\tilde{y} \in T\pa{B_d(x;\epsilon)}$. 
        Then $\tilde{y}=T(y)$ for some $y \in B_d(x;\epsilon)$. 
        Hence 
        \begin{align*}
            \tilde{d}(\tilde{y},\bra{x})& =\tilde{d}(T(y),\bra{x})\\
            &=\tilde{d}(\bra{y},\bra{x})\\
            & = d(y,x) \\
            & < \epsilon
        \end{align*}
        Hence $\tilde{y} \in B_d(\bra{x} ; \epsilon)$, and so 
                \begin{equation}
            T\pa{B_d(x;\epsilon)} \subset  B_{\tilde{d}} \pa{\bra{x}; \epsilon}
        \end{equation}
        Suppose $\bra{y} \in B_{\tilde{d}}\pa{\bra{x} ; \epsilon}$. 
        Then $d(x,y) = \tilde{d}()\bra{x},\bra{y}) < \epsilon$, so $y \in B_d(x; \epsilon)$. 
        Hence $[y]=T(y) \in T\pa{B_d(x;\epsilon)}$, so the reverse inclusion also holds, and so the above claim holds. 
        This, paired with the fact that 
        \begin{equation*}
            \{[x] : x \in X\}= X/\cong
        \end{equation*}
        finishes the result. 
        
    \end{proof}
    \begin{proof}[Proof of \ref{def:MFPM:IsIsometricSurjection}]
    Let $x,y \in X$. Then, 
    \begin{equation*}
        d(x,y) = \tilde{d}\pa{\bra{x}, \bra{y}} = \tilde{d}\pa{T(x), T(y)}
    \end{equation*}
        
        T is \Surjective by \ref{prop:QuotientMapSurjective}.
    \end{proof}
	\begin{proof}[Proof of \label{def:MFPM:Injection}]
	Let $T$ be \Injective. 
	Let $x,y \in X$ with $x \neq y$. 
	Then $T(x) \neq T(y)$, so by 
	\ref{def:MFPM:IsMetric}
	and
	\ref{def:MFPM:IsIsometricSurjection}, 
	$0 < \tilde{d}\pa{T(x),T(y)}= d(x,y)$. 
	Hence $d$ is a \Metric. 

	Now suppose $d$ is a \Metric. 
	Let $x,y \in X$ with $x \neq y$. 
	Then $d(x,y) > 0$. Hence by
	\ref{def:MFPM:IsIsometricSurjection}, 
	$\tilde{d}(T(x),T(y)) = d(x,y) > 0$. 
	Hence $T(x) \neq T(y)$. 
	\end{proof} 
    \begin{proof}[Proof of \ref{def:MFPM:CompleteInherited}]
        Let $(X,d)$ be \PseudometricComplete. 
        Let $\{[x_i]\}_{i \in \N} \subset (X/\cong, \tilde{d})$ be a \PseudometricCauchySequence. 
        Let $\epsilon > 0$. 
        Then there exists $N \in \N$ such that for $m,n > N$, we have 
        \begin{equation*}
            d(x_m, x_n) = \tilde{d}(Tx_m, Ty_m) =\tilde{d}([x_m], [x_n]) < \epsilon
        \end{equation*}
        So the \Sequence $\{x_i\}_{ i \in \N} \subset (X,d)$ is \PseudometricCauchySequence. 
        Since $(X,d)$ is \PseudometricComplete, this \Sequence has a limit, say $x_i \to x \in (X,d)$. 
        But, we have $[x_i]=Tx_i \to Tx = [x]$, so $\{[x_i]\}$ is convergent, and since that \Sequence was arbitrary, $(X/\cong, \tilde{d})$ is \PseudometricComplete. 
        
        Let $(X/\cong, \tilde{d})$ be \PseudometricComplete. 
        Let $\{x_i\} \subset X$ be a \PseudometricCauchySequence.
        Let $\epsilon > 0$. Then there exist $N \in \N$ such that for $m,n > N$, we have
        \begin{equation*}
            \tilde{d}\pa{[x_m], [x_n]} = \tilde{d}(Tx_m, Tx_n) = d(x_m, x_n) < \epsilon
        \end{equation*}
        so that $\{[x_i]\}_{i \in \N}$ is also a \PseudometricCauchySequence. 
        Since $(X/\cong, \tilde{d})$ is \PseudometricComplete, this \Sequence has a limit, say $[x_i] \to y \in X/\cong$. 
        Since T is \Surjective, for some $x \in X$, $Tx \in y$, and so
        \begin{equation*}
            d(x, x_i) = \tilde{d}(Tx, Tx_i) =\tilde{d}(y, [x_i]) \to 0
        \end{equation*}
        meaning $x_i \to x$ and we are done. 
                
\end{proof}
\end{prop}

\begin{rmk}[\MetricSpace  Correspondence]
\rm
Note that in the case of a \MetricSpace
the condition of being
\PseudometricComplete is equivalent to 
the condition being
Complete,  and
A \Sequence is a 
\PseudometricCauchySequence 
if and only if it is a Cauchy \Sequence. 
\end{rmk}
