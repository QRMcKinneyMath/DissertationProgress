\begin{prop}[Codomain Quotient Operator]
\label{prop:handedquotientoperators}
    Let X and Y be 
    \SeminormedSpaces
    with \CodomainQuotientMap $\scQ_Y$. 
    The following are true. 
    \begin{enumerate}
        \item $\scQ_Y$ is a well defined continuous linear surjective isometry. 
        \item If Y is a \NormedSpace, then $\scQ_Y$ is invertible with a continuous inverse. 
    \end{enumerate}
    \begin{proof}[Proof Of 1]
        Since $Tx \in Y$ for any $x \in X$, 
        $[Tx]_Y$ is defined for any $x \in X$. 
        Furthermore, if $q_y:Y \to Y/\Ker$
        is the \QuotientMap of Y under 
        \EquivalenceModKernel, then 
        $\scQ_YT = q_y \circ T$. 
        By \ref{prop:quotientnormspace}, 
        $q_y \in BL(Y, Y/\Ker)$. 
		Hence, 
		$\scQ_YT = q_y \circ T \in BL(X,Y/\Ker)$.
        Hence $\scQ_Y$ is well defined. 

        For linearity, let $\alpha \in \F$
        and $S,T \in BL(X,Y)$. 
        Let $x \in X$. 
        Then, 
        \begin{align*}
            \scQ_Y\pa{\alpha T+S}x & = \bra{\pa{\alpha T+S}x}_Y\\
            & = \bra{\alpha Tx+ Sx}_Y\\
            & = \bra{\alpha Tx}_Y+ \bra{Sx}_Y\\
            & = \alpha \bra{Tx}_Y+\bra{Sx_Y}\\
            & = \alpha \scQ_YTx+ \scQ_YSx\\
            & = \pa{\alpha \scQ_YT+\scQ_YS}x
        \end{align*}

        For being an isometry, 
        let $T \in BL(X,Y)$ and 
        let $x \in X$. Then, since $\norm{\bra{Tx}_Y}_{Y/\Ker} = \norm{Tx}_Y$, 
        \begin{align*}
            \frac{\norm{\scQ Tx}_{Y/\Ker}}{\norm{x}_X} & = \frac{\norm{\bra{Tx}_Y}_{Y/\Ker}}{\norm{x}_X} \\
            & = \frac{\norm{Tx}_Y}{\norm{x}_X} 
        \end{align*}
        and thus taking the norm over
        x with $\norm{x}_X \neq 0$ will yield the
        same result. Hence $\norm{T} = \norm{\scQ_Y T}$. 

        For surjectivity, let $\tilde{T} \in BL(X, Y/\Ker_Y)$. 
        Let $\{x_{\alpha}\}_{\alpha \in A}$ be a hamel basis for $X$. 
        For each $\alpha \in A$, let $y_{\alpha} \in \tilde{T}x_{\alpha}$. 
        Define $T:X \to Y$ by 
        \begin{equation}
            T\pa{\sum_{i=1}^n \beta_{\alpha_i} x_{\alpha_i}} = \sum_{i=1}^n \beta_{\alpha_i} y_{\alpha_i}
        \end{equation}
        T is obviously linear
        and has the property $[Tx]=\tilde{T}x$. 
        and since $\tilde{T} \in BL(X,Y/\Ker)$, 
        $\tilde{T}\Ker_X \subset \Ker_{ (Y/\Ker_Y)}=0$. 
        Hence $T \Ker_X \subset \Ker_Y$. 
        Furthermore, if $x \in X$ with $\norm{x}_X \neq 0$, then
        \begin{align*}
        \frac{\norm{Tx}_Y}{\norm{x}_X} & = \frac{\norm{\bra{Tx}_Y}_{Y/\Ker}}{\norm{x}_X}\\
        & = \frac{\norm{\tilde{T}x}_{Y/\Ker}}{\norm{x}_X}
        \end{align*}
        Therefore $T$ is bounded.
        Hence $T \in BL(X,Y)$, and $\scQ_YT=\tilde{T}$. 
        Thus we have surjectivity, and are done.
    \end{proof}
    \begin{proof}[Proof Of 2]
        If $Y$ is a \NormedSpace, 
        %then $q_y:Y \to Y/\Ker_Y$ is 
        a linear isometric homeomorphism by 
        \ref{prop:quotientnormspace}. 
        In particular, in this case, 
        $q_y$ is injective, meaning that 
        if $T,S \in BL(X,Y)$ where
        $T \neq S$, then 
        $Tx_0 \neq Sx_0$ for some $x_0 \in X$. 
        For this $x_0$, $q_yTx_0 \neq q_ySx_0$, so 
        $\scQ_YT \neq \scQ_YS$. 
        Therefore $\scQ_Y$ is injective, and therefore a bijection. 
        The inverse of an isometry is also an isometry 
        and therefore continuous, finishing this proof. 
    \end{proof}
\end{prop}
