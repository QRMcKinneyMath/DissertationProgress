\begin{prop}[Birkhoff-Kakutani]
\label{prop:BirkhoffKakutani}
\rm
Let $(G,\scT)$ be a \TopologicalGroup with \IdentityElement $e$.
Let $\scB=\{\scV_i\}_{i \in \mathbb{N}}$ be a \LocalBasis for $G$. 
Then there exists a \Pseudometric $d$ on $G$ with the following properties.
\begin{enumerate}
\item $\scT$ is the \PseudometricTopology on $G$ induced by $d$.
\item For each $x,y,z \in G$, $d(zx,zy)=d(x,y)$. 
\end{enumerate}
\begin{proof}
By 
\ref{prop:SymmetricContained}
$G$ has a \LocalBasis $\{V_i\}_{i \in \mathbb{N}}$ consisting
of \SymmetricSubset subsets satisfying, for each $n \in \mathbb{N}$, 
$V_{n+1}V_{n+1} \subset V_n$.
For each $r \in \mathbb{Q} \cap [0,1)$, let $C(n,r) \in \{0,1\}$ 
be the $n^{th}$ bit of $r's$ finite expansion. 
That is, for $r \in \mathbb{Q}$, let 
\begin{equation*}
r=\sum\limits_{n \in \mathbb{N}} \frac{C(n,r)}{2^n}
\end{equation*}
Let $P$ be the set of $r \in \mathbb{Q} \cap [0,1)$ 
for which $C(n,r)$ is nonzero for only finitely many $r$. 
For each $r \in P$, let $\Gamma_r$ denote 
the $n \in \mathbb{N}$ for which $C(n,r) \neq 0$. 
Since $G$ is not assumed to be \CommutativeFunction, 
define, for a \Finite set $K=\{x_k\}_{k=1}^m \subset \mathbb{N}$ in 
which $x_1<x_2<\cdots<x_m$, 
\begin{equation*}
\prod\limits_{n \in K} U_{n} = U_{x_1}U_{x_2}U_{x_3}\cdots U_{x_m}
\end{equation*}
Now, define $A:P \cup (1,\infty)\to \scT$ by 
\begin{equation*}
A(r) = \begin{dcases}
\prod\limits_{n \in \Gamma_r} V_n & r<1\\
G & r \geq 1
\end{dcases}
\end{equation*}

I first claim that if $r \leq s$, then $A(r) \subset A(s)$. 
To prove this, let $r \leq s$.
Define $\Gamma_r$ and $\Gamma_s$ as above. 
Then there exists a $k \in \mathbb{N}$ such that 
$\Gamma_r \cap [0,k] \subset \Gamma_s$
and $k+1 \in \Gamma_s \setminus \Gamma_r$.
Define $N=[k+2,\infty) \cap \Gamma_r$. 
Then since $V_{n+1}V_{n+1} \subset V_n$ for all $n$, it is clear that
\begin{equation*}
\prod\limits_{n \in N} V_n \subset V_{k+1}
\end{equation*}
Hence, 
\begin{align*}
A(r) & = \pa{\prod\limits_{n \in \Gamma_r \cap [0,k]} V_n} \pa{ \prod\limits_{n \in N} V_n} \\
& \subset\pa{ \prod\limits_{n \in \Gamma_r \cap [0,k]} V_n} V_{k+1} \\
& \subset\pa{\prod\limits_{n \in \Gamma_s \cap [0,k]} V_n} V_{k+1} \\
& \subset A(s)
\end{align*}

I now make a second claim: that if $r \in P \cap (0,1)$ and if $n>\max\{k \in \mathbb{Z}^+ : C(rk,r)=1\}$ then 
$A(r)A\pa{\frac{1}{2^n}} = A\pa{r+\frac{1}{2^n}}$. 
This is clear because, for all $k \in \mathbb{Z}^+$, 
\begin{equation*}
C(k,r)+C\pa{k,\frac{1}{2^n}} = C\pa{k, r+\frac{1}{2^n}}
\end{equation*}
Thus we have
\begin{equation*}
A(r)A\pa{\frac{1}{2^n}} = \pa{\prod\limits_{k \in \Gamma_r} U_k} U_n = \prod\limits_{k \in \Gamma_{r+\frac{1}{2^n}}} U_k = A\pa{r+\frac{1}{2^n}}
\end{equation*}

I now claim that for every $n \in \mathbb{Z}^+$, for every $r \in P$, we have 
\begin{equation*}
A(r)A\pa{\frac{1}{2^n}} \subset A\pa{r+\frac{3}{2^n}}
\end{equation*}
By the first and second claims, it is sufficient to prove 
in the case $n \leq \max\{n \in \mathbb{Z}^+ : C(n,r) = 1 \}$. 
Now, Define $\Gamma_r^{-} = \Gamma_r \cap [0,n)$ and $\Gamma_r^+ = \Gamma_r \cap [n,\infty)$.
Then by assumption $\Gamma_r^+ \neq \emptyset$. 
Define 
\begin{equation*}
r_1 = \frac{1}{2^{n-1}} - \sum\limits_{j \in \Gamma_r^+} \frac{1}{2^j}
\end{equation*}
Then $r_1 >0$ and 
Define 
\begin{equation*}
r_2 = r+r_1
\end{equation*}
Then $\max\{k \in \mathbb{Z}^+ : C(r_2,k) \neq 0 \leq n-1\}$.
Hence, by claim 2, $A(r_2)A\pa{\frac{1}{2^n}} = A\pa{r_2+\frac{1}{2^n}}$. 
Also $r < r_2 < r+\frac{1}{2^{n-1}}$.
By this observation, paired with claim 01, we have 
\begin{align*}
A(r)A\pa{\frac{1}{2^n}} & \subset A(r_2) A\pa{\frac{1}{2^n}}\\
&  = A\pa{r_2+\frac{1}{2^n}}\\
& \subset A\pa{r+\frac{1}{2^{n-1}}+\frac{1}{2^n}}\\
& \subset A\pa{r+\frac{3}{2^{n}}}
\end{align*}

Define, for $x \in G$, $\tilde{d}(x) = \inf\{r \in [0,\infty) : x \in A(r)\}$. 
Then $\tilde{d}(x) \leq 1$ for all $x \in G$. 
Define, for $x,y \in G$, 
\begin{equation*}
d\pa{x,y} = \sup\limits_{h \in G} \braces{ \abs{\tilde{d}(hx)-\tilde{d}(hy)}}
\end{equation*}

Since $\tilde{d}$ is bounded, $d$ is well defined. 
$d$ is clearly \CommutativeFunction, satisfies the 
\TriangleInequality, and satisfies $d(x,x)=0$, so 
$d$ is a \Pseudometric on $G$. 
That $d$ is left invariant is equally clear.

What remains to show is that the
\PseudometricTopology $\scT_d$ generated by $d$ on $G$ 
Since $d$ is left invariant, it suffices
to show that for each $\epsilon > 0$, 
there is an $n$ such that $V_n \subset B(e;\epsilon)$, 
and for each $k \in \mathbb{N}$, there is 
a $\delta > 0$ such that $B(e;\delta) \subset V_k$.

Showing $\scT \subset \scT_d$ is easy. 
Given $n \in \mathbb{N}$, 
\begin{equation*}
B\pa{e,\frac{1}{2^{n+1}}} \subset A\pa{\frac{1}{2^n}}=U_n
\end{equation*}

For the other direction, let $\epsilon >0$. 
Let $n \in \mathbb{N}$ such that $\frac{3}{2^n} < \epsilon$. 
Let $u \in U_n=A\pa{\frac{1}{2^n}}$. 
Let $z \in G$ and let $r$ be any positive number such that $z \in A(r)$. 
Then 
\begin{equation*}
zu \in A(r)A\pa{\frac{1}{2^n}} \subset A\pa{r+\frac{3}{2^n}}
\end{equation*}
Hence, 
\begin{equation*}
\tilde{d}(zu) \leq r+\frac{3}{2^n}
\end{equation*}
The nature of $r$ implies
\begin{equation}
\label{BirkhoffKakutaniProof:Dir2}
\tilde{d}(zu) \leq \inf\{r \in (0,\infty) : z \in A(r)\} + \frac{3}{2^n} = \tilde{d}(z)+\frac{3}{2^n}
\end{equation}

Now let $s$ be any positive number such that $zu \in A(s)$. 
Then 
\begin{equation*}
z \in A(s)u^{-1} \subset A(s)U_n^{-1} = A(s)U_n \subset A\pa{s+\frac{3}{2^n}}
\end{equation*}
Hence
\begin{equation*}
\tilde{d}(z) \leq s+\frac{3}{2^n}
\end{equation*}
Similar to the above, the nature of $s$ implies
\begin{equation}
\label{BirkhoffKakutaniProof:Dir1}
\tilde{d}(z) \leq \tilde{d}(zu)+\frac{3}{2^n}
\end{equation}
By 
\ref{BirkhoffKakutaniProof:Dir1}
and
\ref{BirkhoffKakutaniProof:Dir2}, 
and the arbitrary nature of $z$, 
$d(u,e) \leq \frac{3}{2^n}< \epsilon$. 
Since $u \in U_n$ was arbitrary, 
$U_n \subset B\pa{e;\epsilon}$.
\end{proof}
\end{prop}
