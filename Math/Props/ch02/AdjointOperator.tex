\begin{prop}[\AdjointOperator]
\label{prop:adjointoperator}
\rm
Let $X$, $Y$, and $Z$ be \SeminormedSpaces
over a \Field $\F \in \{\R, \C\}$.
Let $T \in BL(X,Y)$. 
Let $\T=T^\times_Z$ denote the 
\AdjointOperator of T
relative to the space $Z$. 
Let $Q_Y:Y \to Y/\Ker_Y$ denote the \QuotientMap
The following are true. 
\begin{enumerate}[label=(\roman*), ref={\ref{prop:adjointoperator}~\roman*}]
\item 
\label{prop:Adjoint:Linear}
$\T$ is \Linear. 
\item 
\label{prop:Adjoint:WellDefined}
If $S \in BL(Y,Z)$, then $\T S \in BL(X,Z)$. (That is, the \AdjointOperator is well defined as a concept).
\item 
\label{prop:Adjoint:Continuous}
$\T \in BL\pa{ BL(Y,Z), BL(X,Z)}$. 
\item 
\label{prop:Adjoint:Isometry}
$\norm{\T}=\norm{T}$
\item 
\label{prop:Adjoint:DenseRange}
If $Range(T)$ is dense in $Y$, then $\inf\limits_{\norm{x}=1}\norm{Tx} \leq \inf\limits_{\norm{S}_{BL(Y,Z)} = 1} \norm{\T S}$.
%			To Range(T) dense in Y. 
\item 
\label{prop:Adjoint:NotDenseRange}
If Range(T) is not dense in Y, then 
There exists $S \in BL(Y,Z)$ with $\norm{S} = 1$ and $\norm{\T S} = 0$. 
$\inf\limits_{\norm{S}_{BL(Y,Z)}=1} \norm{\T S} =0$
\item 
\label{prop:Adjoint:Surjective}
$\T$ is \Surjective if and only if T is \Injective and has \SetClosed range in Y. 
\end{enumerate}


\begin{proof}[Proof of \ref{prop:Adjoint:Linear}]
Let $S,R \in BL(Y,Z)$, 
$\alpha \in \F$, 
and $x \in X$. 
Then, 
\begin{align*}
\ip{x, \T(\alpha S+R)} & = \ip{Tx, \alpha S+R}\\
& = \alpha \ip{Tx, S}+ \ip{Tx, R} \\
& = \alpha \ip{x, \T S} + \ip{ x, \T R}\\
& = \ip{x, \alpha \T S} + \ip{x, \T R} \\
& = \ip{x, \alpha \T S + \T R}
\end{align*}
Since $x \in X$ was arbitrary, \Linearity is verified. 
\end{proof}
\begin{proof}[Proof of \ref{prop:Adjoint:WellDefined}]
Let $S \in BL(Y,Z)$. 
Then, 
$\T S = S \circ T$. 
The composition of \ContinuousFunction operators is \ContinuousFunction, so $\T S$ is 
\ContinuousFunction.
The composition of \Linear operators is \Linear, so $\T S$ is \Linear.
Hence, $\T S \in BL(X,Z)$.
\end{proof}
\begin{proof}[Proof of \ref{prop:Adjoint:Continuous}]
Let $S \in BL(Y,Z)$. Then, 
if $x \in X$
\begin{equation*}
\norm{\ip{x, \T S}} = \norm{\ip{Tx, S}} \leq \norm{S} \norm{Tx} \leq \norm{S} \norm{T} \norm{x}
\end{equation*}
Hence $\norm{\T S} \leq \norm{S} \norm{T}$
Since T is \Linear, and since S was arbitrary, 
by part  12 of \ref{prop:BLO}, $\T \in BL\pa{ BL(Y,Z), BL(X,Z)}$.
\end{proof}
\begin{proof}[Proof of \ref{prop:Adjoint:Isometry}]
For any $S \in BL(Y,Z)$, 
$\T S = S \circ T$, so
$\norm{\T S} \leq \norm{S} \norm{T}$. 
Hence $\norm{\T} \leq \norm{T}$. 
Now let $x_0 \in X$. 
Then, by part 4 of 
\ref{thm:hahnbanach}, 
there exists $S \in BL(Y,Z)$ with 
$\norm{S}=1$
and $\norm{STx_0} = \norm{Tx_0}$. 
Hence, 
\begin{align*}
\norm{T x_0} & = \norm{S Tx_0} \\
& = \norm{(S \circ T) x_0} \\
& = \norm{(\T S) x_0} \\
& \leq \norm{\T} \norm{S} \norm{x_0}\\
& = \norm{\T} \norm{x_0}
\end{align*}
Since $x_0 \in X$ is arbitrary, $\norm{T} \leq \norm{\T}$. 
Since the inequality goes both ways, $\norm{T}=\norm{\T}$.
\end{proof}
\begin{proof}[Proof of \ref{prop:Adjoint:DenseRange}]
Let $\Gamma=\inf\limits_{\norm{x}=1} \norm{Tx}$, 
and let $S \in BL(Y,Z)$ with $\norm{S} = 1$. 
Then, 
\begin{equation*}
\{x | \norm{Tx} \leq \Gamma\} \subset B_X(0;1)
\end{equation*}
so 
\begin{equation*}
\sup\limits_{\norm{x}\leq 1} \abs{\ip{Tx, S}} \geq \sup\limits_{\norm{Tx} \leq \Gamma}\abs{\ip{Tx, S}}
\end{equation*}
Also, since $Range(T)$ is dense in $Y$ and $S$ is \ContinuousFunction,
\begin{equation*}
\sup\limits_{\norm{Tx} \leq \Gamma} \abs{\ip{Tx, S}} = \sup\limits_{\norm{y} \leq \Gamma} \abs{\ip{y, S}}
\end{equation*}
From these two we arrive at the inequality
\begin{align*}
\norm{\T S} & = \sup\limits_{\norm{x}  \leq 1} \abs{\ip{x, \T S}}\\
& = \sup\limits_{\norm{x} \leq 1} \abs{\ip{Tx, S}}\\
& \geq \sup\limits_{\norm{Tx} \leq \Gamma} \abs{\ip{Tx, S}}\\
& = \sup\limits_{\norm{y} \leq \Gamma} \abs{\ip{y, S}}\\
& = \Gamma\\
& \inf\limits_{\norm{x} = 1} \norm{Tx}
\end{align*}
Since $S \in \partial B_{BL(Y,Z)}(0;1)$ was arbitrary, we conclude
$\inf\limits_{\norm{S} = 1} \norm{\T S} \geq \inf\limits_{\norm{x} = 1} \norm{Tx}$
\end{proof}
\begin{proof}[Proof of \ref{prop:Adjoint:NotDenseRange}]
Suppose $Range(T)$ is not dense in $Y$. 
Then there exists $y_0 \in Y \setminus \overline{Range(T)}$. 
By 
\ref{thm:HahnBanach:NullspaceOperator}, 
there exists $S_0 \in BL(Y,Z)$ with $S_0\pa{\overline{Range(T)}} = 0$ and 
$\norm{S_0(y_0)} = 1$. 
Set $S= \frac{S_0}{\norm{S_0}}$. 
Then $\norm{S} = 1$. 
Furthermore, 
\begin{align*}
\norm{\T S} & = \sup\limits_{\norm{x} \leq 1} \abs{\ip{x, \T S}}\\
& \leq \sup\limits_{y \in Range(T)} \abs{\ip{y, S}}\\
& = 0
\end{align*}
\end{proof}
\begin{proof}[Proof of \ref{prop:Adjoint:Surjective}]
Let $\T$ be \Surjective and let $x_0 \in X$ with $Tx_0 = 0$. 
Let $\tilde{S} \in BL(X,Z)$. 
Then there exists $S \in BL(Y,Z)$ with $\tilde{S} = \T S$. 
For this $S$, 
\begin{equation*}
\ip{x, \tilde{S}} = \ip{Tx, S} = 0
\end{equation*}
Hence $\tilde{S}x = 0$ for every \Linear \ContinuousFunction $\tilde{S}$. 
This implies $x=0$, so $T$ is \Injective. 
Now suppose 
\end{proof}
\end{prop}
