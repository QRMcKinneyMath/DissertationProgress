\begin{prop}[\MagmaHomomorphism]
\label{prop:MagmaHomomorphism}
\rm
Let $M_1$ and $M_2$ be \Magmas.
Let $T:M_1 \to M_2$ be a 
\MagmaHomomorphism.
Define $\cong \subset M_1 \times M_1$ by $x \cong y$ if and only if $T(x)=T(y)$.
The following are true.
\begin{enumerate}[label=(\roman*), ref={\ref{prop:MagmaHomomorphism}~\roman*}]
\item
\label{prop:MagmaHomomorphism:Compatible}
$\cong$ is a \Congruence on $M_1$. 
\item
\label{prop:MagmaHomomorphism:Quotient}
Let $Q:M_1 \to M_1/\cong$ denote the \QuotientMap.
Define $\tilde{T}:M_1/\cong \to Range(T)$ by setting, for each $x \in M_1/\cong$, 
$\tilde{T}\pa{[x]} = T(x)$. 
Then $\tilde{T}$ is a well defined \Bijective \MagmaHomomorphism, $T = \tilde{T} \circ Q$, and $Q = \tilde{T}^{-1} \circ T$. 
\end{enumerate}
\begin{proof}[Proof of \ref{prop:MagmaHomomorphism:Compatible}]
It is obvious that $T$ is an \EquivalenceRelation.
What remains to show is that $\cong$ is 
\AlgebraicallyConsistent with $M$. 
Suppose $x_0 \cong x_1$ and $y_0 \cong y_1$. 
Then since $T$ is a \MagmaHomomorphism, 
\begin{equation*}
T(x_0y_0) = T(x_0) T(y_0) = T(x_1) T(y_1) = T(x_1y_1)
\end{equation*}
Hence $x_0y_0 \cong x_1y_1$.
\end{proof}
\begin{proof}[Proof of \ref{prop:MagmaHomomorphism:Quotient}]
We first show that $T$ is well defined. 
Let $[x] = [y]$. We must show that $\tilde{T}([x]) = \tilde{T}([y])$. 
Since $[x] = [y]$, $x \cong y$. Hence, $T(x) = T(y)$, 
which gives us 
$\tilde{T}([x]) = T(x) = T(y) = \tilde{T} ([y])$. 

Suppose $\tilde{T}\pa{[x]} = \tilde{T}\pa{[y]}$. 
Then $T(x) = T(y)$, so $x \cong y$, so $[x] = [y]$. 
Hence $\tilde{T}$ is \Injective. 

Let $y \in Range(T)$. Then there exists $x \in M_1$ such that 
$T(x) = y$. Hence $\tilde{T}\pa{[x]} = T(x) = y$. 
Thus, $\tilde{T}$ is \Surjective. 

Now let $x,y \in M$. 
Then, $\tilde{T}\pa{[x][y]} = \tilde{T}\pa{[xy]} = T(xy) = T(x)T(y) = \tilde{T}([x]) \tilde{T}([y])$. 
Thus $\tilde{T}$ is a \MagmaHomomorphism. 

Finally, if $x \in M_1$, then $\tilde{T} \circ Q(x) = \tilde{T}([x]) = T(x)$, so 
$T = \tilde{T} \circ Q$.
The final equation falls from the \Bijectivity of $\tilde{T}$. 
\end{proof}
\end{prop}
