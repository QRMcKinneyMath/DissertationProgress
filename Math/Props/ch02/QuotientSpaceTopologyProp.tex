
\begin{prop}[Quotient Space Topology]
    \label{prop:QuotientSpaceTopology}
    
    Let $(Z,\T_Z)$ be a 
	\TopologicalSpace
    with \QuotientTopologicalSpace  $\pa{Z/\cong, \T_{Z/\cong}}$
    and \QuotientMap T.
    
    Then the following are true. 
    \begin{enumerate}
        \item $\T_{Z/\cong}$ is a \Topology on $Z/\cong$. 
        \item $T:(Z, \T_Z) \to (Z/\cong, \T_{Z/\cong})$ is 
        \ContinuousFunction. 
        \item If U is \SetOpen (\SetClosed) in $(Z,\T_Z)$ then $T(U)$ and $T(Z\setminus U)$ \Partition $Z/\cong$. 
        \item If U is \SetOpen in $(Z, \T_Z)$, then $T^{-1}(T(U))=U$. 
        \item If K is \SetClosed in $(Z,\T_Z)$, then $T^{-1}T(K)=K$. 
        \item $T:(Z, \T_Z) \to (Z/\cong, \T_{Z/\cong})$ is 
            \OpenFunction. 
        \item $T:(Z, \T_Z) \to (Z/\cong, \T_{Z/\cong})$ is 
        \ClosedFunction.
        \item $(Z, \T_Z)$ is a \SetCompact space if and only if $(Z/\cong, \T_{Z/\cong})$ is a \SetCompact space.
        \item If $\scB$ is a \TopologyBasis for $\T_z$, then $\{T(U) | U \in \scB\}$ is a \TopologyBasis for $\T_{Z/\cong}$. 
        \item If T is \Injective, then it is a \Homeomorphism. 
    \end{enumerate} 
    \begin{proof}[Proof of 1]
        Since $\emptyset \in \T_Z$, we have 
        \begin{equation}
            \emptyset = \bigcup\limits_{x \in \emptyset} \{Tx\} \in \T_{Z/\cong}
        \end{equation}
        Since $Z \in \T_Z$, and by \ref{rmk:quotientsetpartition}, 
        \begin{equation} 
            Z/\cong = \bigcup_{x \in Z} \{[x]\}= \bigcup\limits_{x \in Z} \{T(x)\} \in \T_{Z/\cong}
        \end{equation} 
        
        Let $\{U_{\alpha} | \alpha \in A\} \subset \T_{Z/\cong}$. 
        For each $\alpha \in A$, there exists $B_{\alpha} \in \T_{Z}$ such that we have
        \begin{equation} 
            U_{\alpha } = \bigcup_{x \in B_{\alpha}} \{Tx\}
        \end{equation} 
        Since $\bigcup_{\alpha \in A} B_\alpha \in \T_{Z}$, we have 
        \begin{equation}
            \bigcup_{\alpha \in A} U_{\alpha}= \bigcup\limits_{\alpha \in A} \bigcup\limits_{x \in U_\alpha} \{T(x)\} = \bigcup\limits_{x \in \bigcup\limits_{\alpha \in A} B_{\alpha}} \{T(x)\} \in \T_{Z/\cong}
        \end{equation} 
        Let $\{U_i\}_{i=1}^n \subset \T_{Z/\cong}$. 
        For each $i \in \{1, ..., n\}$, there exists $B_i \in \T_{Z}$ such that
        \begin{equation}
            U_i = \bigcup_{x \in B_{i}} \{T(x)\}
        \end{equation}
        Suppose 
        \begin{equation}
            [x_0] \in \bigcap\limits_{i=1}^n \bigcup\limits_{x \in B_i} \{T(x)\}
        \end{equation}
        Then for each $i \in \{1,..., n\}$, there is a $y_i \in B_i$ such that $ y_i \cong x_0$. 
        Since each $B_i$ is \SetOpen, the definition of $\cong$ implies that $x_0 \in B_i$ for every i. Hence, 
        \begin{equation} 
            x_0 \in \bigcap_{i=1}^n B_i
        \end{equation} 
        Implying 
        \begin{equation}
            [x_0] \in  \bigcup\limits_{x \in \bigcap\limits_{i=1}^n B_i} \{[x]\}
        \end{equation} 
        Hence, 
        \begin{equation} 
            \bigcap\limits_{i=1}^n \bigcup\limits_{x \in B_i} \{T(x)\}
            \subset
            \bigcup\limits_{x \in \bigcap\limits_{i=1}^n B_i} \{[x]\}
        \end{equation} 
        Furthermore, since the reverse inclusion is obvious, 
        and since $\bigcap_{i=1}^n B_i \in \T_{Z}$, we have 
        \begin{equation}
            \bigcap_{i=1}^n U_i = \bigcap_{i=1}^n \bigcup_{x \in B_i} \{T(x)\}= \bigcup\limits_{x \in \bigcap\limits_{i=1}^n B_i} \{T(x)\} \in \T_{Z/\cong}
        \end{equation}
    \end{proof}
    \begin{proof}[Proof of 2]
        Let $V \in \T_{Z/\cong}$. 
        Let $x_0 \in T^{-1}(V)$. 
        Then $[x_0] \in V$. 
        By definition, there is a $U \in \T_Z$ such that 
        \begin{equation}
            T(U) \subset \bigcup\limits_{x \in U} \{T(x)\}=V
        \end{equation}
        Hence there is a $y_0 \in U$  such that 
        \begin{equation}
            [x_0] \in T(y_0) = \{[y_0]\}
        \end{equation}
        Therefore, $x \cong y$. 
        Definition of the 
		\RelationOfEqualNeighborhoodFilters 
		implies $\scU(x_0)=\scU(y_0)$. 
        Hence, $x_0 \in U \subset T^{-1}(V)$.
    \end{proof}
    \begin{proof}[Proof of 3]
        Let $K$ be closed in $(Z,\T_Z)$. 
        Then each point $x_0$ in $Z\setminus K$ has some $U_{x_0} \in \scU_{\T_Z}(x_0)$ which is 
		\Disjoint 
		from K.
        Hence $y_0 \not \cong x_0$ for any $y_0 \in K$, $x_0 \in Z\setminus K$. 
        Hence $T(K)$ is 
		\Disjoint 
		from $T\pa{Z \setminus K}$. 
        This fact, paired with \ref{prop:QuotientMapSurjective}, implies $T(Z\setminus K)$ and T(K) 
		is a \Partition of $Z/\cong$.
    \end{proof}
    \begin{proof}[Proof of 4]
        Let $U \in \T_Z$. 
        The nontrivial direction to prove is $T^{-1}\pa{T(U)} \subset U$.
        Let $y \in T^{-1}\pa{T(U)}$. 
        Then $[y]=Ty \in T(U)$.
        Hence, $[y]=T(x)=[x]$ for some $x \in U$. 
        Since $y \cong x$ and $x \in U \in \scU_{\T_Z}(x)$, we have $U \in \scU_{\T_Z}(y)$. 
        Hence $y \in U$.
        Since y was arbitrary, $T^{-1}\pa{T(U)} \subset U$, and equality is obvious because the other direction of inclusion is trivial. 
    \end{proof}
    \begin{proof}[Proof of 5]
        Let K be 
		\SetClosed 
		in $(Z,\T_Z)$. Part 3 Of this result implies $Z/\cong$ is partitioned by $T(K)$ and $T(Z\setminus K)$. 
        
        By part 4 of this proposition, 
        \begin{align*}
            T^{-1}\pa{T(K)}&=T^{-1} \pa{T(Z) \setminus T(Z \setminus K)} \\
            &= T^{-1}\pa{Z/\cong \setminus T(Z \setminus K)}\\
            &=T^{-1}(Z/\cong) \setminus T^{-1}(T(Z\setminus K)) \\
            &= Z \setminus \pa{Z \setminus K} \\
            &= K
        \end{align*}      
    \end{proof}
    \begin{proof}[Proof of 6]
        Let $U \in \T_Z$.
        Then by definition of the \QuotientSpaceTopology
        \begin{equation}
            TU= \bigcup_{x \in U} \{T(x)\}  \in \T_{Z/\cong}
        \end{equation}
    \end{proof}  
    \begin{proof}[Proof of 7] 
        Let K be \SetClosed in $(Z,\T_Z)$. 
        Then $Z \setminus K \in \T_Z$. 
        By Parts 3 and five of this proposition, we know $T(K) = Z/\cong \setminus T(Z\setminus K)$ and also that $T(Z\setminus K) \in \T_{Z/\cong}$. Hence $T(K)$ is closed in $(Z/\cong, \T_{Z/\cong})$. 
    \end{proof} 
    \begin{proof}[Proof of 8]
        Let $(Z,\T_Z)$ be \SetCompact. 
        Let $\{U_{\alpha}\}_{\alpha \in A}$ be an open covering of $(Z/\cong, \T_{Z/\cong})$. 
        Then $\{T^{-1}\pa{U_{\alpha}} | \alpha \in A\}$ is an open covering of $(Z, \T_Z)$. 
        \SetCompactness of $(Z, \T_Z)$ guarantees the existence of a finite subcovering $\{T^{-1}\pa{U_{\alpha_i}} | i \in \{1, ..., n\}\}$. 
        Hence
        $\{U_{\alpha_i} | i \in \{1, ..., n\}\}=\{TT^{-1}(U_{\alpha_i}) | i \in \{1, ..., n\}\}$ is an 
		\SetOpen
        \Cover of $(Z/\cong, \T_{Z/\cong})$. 
         And the \SetCompactness of $(Z/\cong, \T_{Z/\cong})$ is verified. 
         
         
         Now, suppose $(Z/\cong, \T_{Z/\cong})$ is \SetCompact. 
         Let $\{V_{\beta} | \beta \in B\}$ be an 
         \SetOpen
         \Cover of $(Z, \T_Z)$. 
         Since T is an 
         \OpenFunction
         mapping, $\{T(V_{\beta}) | \beta \in B\}$ is an 
		 \SetOpen
         \Cover of $(Z/\cong, \T_{Z/\cong})$ which by 
		 \SetCompactness has a \Finite \Subcover $\{T(V_{\beta_i}) | i \in \{1, ..., n\}\}$. 
         By part 4 of \ref{prop:QuotientSpaceTopology}, 
         $\{V_{\beta_i}| i \in \{1, ..., n\}\} = \{T^{-1}(T(V_{\beta_i})) |i \in \{1, ..., n\}\}$ is then an \SetOpen \Subcover of $(Z, \T_Z)$. 
     %    
    \end{proof}
    \begin{proof}[Proof of 9]
        Let $\scB$ be a basis for $\T_z$ and let $V \in \T_{Z/\cong}$. 
        Then $T^{-1}(Z) \in \T_Z$, and so there is a subcollection $\{U_{\alpha}\}_{\alpha \in A} \subset \scB$ such that $T^{-1}(V) = \bigcup_{\alpha \in A} U_{\alpha}$. 
        Hence, 
        \begin{align*}
            V& =T(T^{-1}(V))\\
            & = T\pa{\bigcup_{\alpha \in A} U_{\alpha}}\\
            & = \bigcup_{\alpha \in A} T(U_{\alpha})
        \end{align*}
     \end{proof} 
     \begin{proof} [Proof of 10]
            If T is \Injective
			, then since it is 
            \ContinuousFunction
            Part 2 of this result, open by part 6 of this result, and \Surjective by \ref{prop:QuotientMapSurjective}, it is a 
			\Bicontinuous 
			\Bijection, that is, a \Homeomorphism. 
         \end{proof}
\end{prop} 
