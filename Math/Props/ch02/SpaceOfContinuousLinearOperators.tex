\begin{prop}[Space of Bounded Linear Operators On Seminormed Spaces]
\label{prop:BLO} 
Let $(X,\norm{\cdot}_X)$ be a \SeminormedSpace. 
Let $(Y, \norm{\cdot}_Y)$ be a \SeminormedSpace.
Let $BL(X,Y)$ denote the \SpaceOfBoundedLinearOperators from X to Y. 
Let $\norm{\cdot}$ denote the \OperatorSeminorm. 

The following are true. 
\begin{enumerate}
%For Item 1, may have to prove result connecting 
%pseudometric topology continuity to $\epsilon-delta$ cotninuity wrt the pseudometric. 
\item $\norm{\cdot}$ is in fact a well-defined \Seminorm on $BL(X,Y)$. 
\item If $\norm{\cdot}_Y$ is a \Norm, then so is $\norm{\cdot}$. 
\item If $T \in BL(X,Y)$ and $\alpha \in (0,\infty)$, then $\norm{T} = \sup\limits_{\norm{x}_X =\alpha} \frac{\norm{Tx}_Y}{\norm{x}_X}$. 
\item If $T \in BL(X,Y)$ and $\alpha \in (0,\infty)$, , then $\norm{T} = \sup\limits_{0<\norm{x}_X \leq \alpha} \frac{\norm{Tx}_Y}{\norm{x}_X}= \sup\limits_{0<\norm{x}_X < \alpha} \frac{\norm{Tx}_Y}{\norm{x}_X}$. 
\item If $T \in BL(X,Y)$ and $x \in X$, then $\norm{Tx}_Y \leq \norm{T} \norm{x}_X$. 
\item $S:X \to Y$ is linear
, $S(\Ker_X) \subset \Ker_Y$
, and $\sup\limits_{\norm{x}_X \neq 0} \frac{\norm{Sx}_Y}{\norm{x}_X} < \infty$
, if and only if $S \in BL(X,Y)$. 
\item A sequence $\{T_i\}_{i \in \N}$ is a \PseudometricCauchySequence
    if and only if
    there exists an $\alpha > 0$ 
    such that the collection of sequences 
    $\{\{T_ix\}_{i \in \N} | x \in B_X(0;\alpha)\}$ is
    \UniformlyCauchy
    if and only if
    for every $\beta > 0$, 
    the collection of sequences 
    $\{\{T_ix\}_{i \in \N} | x \in B_X(0;\beta)\}$ is
    is \UniformlyCauchy
\item If $T_i \to T$ with respect to $\norm{\cdot}$, then $T_ix \to Tx$ with respect to $\norm{\cdot}_Y$ for each $x \in X$
\item A sequence $\{T_i\}_{i \in \N} \subset BL(X,Y)$ 
    converges %TODO: Turn converges into a macro (net based) referenced via \Coverges, and insert that here. 
    with respect to $\norm{\cdot}$ 
    if and only if it is a \PseudometricCauchySequence 
    and for each $x_\alpha$ 
    in some Hamel basis $\{x_\alpha\}_{\alpha \in A} \subset X$,
    the sequence $\{T_ix_\alpha\}_{\alpha \in A}$
    converges with respect to $\norm{\cdot}_Y$. 
\item Let X be \NonDegenerate. Then $BL(X,Y)$ is complete if Y is complete. 
\item Let X be \NonDegenerate. If $BL(X,Y)$ is \NonDegenerate then Y is \NonDegenerate.
\end{enumerate}


\begin{proof}[Proof of 1] 
    Since X is \NonDegenerate, there exists an $x \in X$ with $\norm{x}_X \neq 0$, 
    so for each $T \in BL(X,Y)$, the set that the supremum is being taken over is nonempty.
    Also, it is clear that $Range(\norm{\cdot}) \subset [0,\infty)$, 

    For \Subadditivity, let $T_i \in BL(X,Y)$ for $i \in \{0,1\}$. and $x \in X$ with $\norm{x} > 0$.
    Then, since $\norm{\cdot}_Y$ is \Subadditive, 
    \begin{align*}
    \frac{\norm{(T_0+T_1)x}_Y}{\norm{x}_X} \leq \frac{\norm{T_0x}_Y}{\norm{x}_X}+ \frac{\norm{T_1x}_Y}{\norm{x}_X}
    \end{align*}
    Since this is true for each x with $\norm{x}_X \neq 0$, taking the supremum of each side yields

    \begin{align*}
    \sup\limits_{\norm{x}_X \neq 0} \pa{\frac{\norm{(T_0+T_1)x}_Y}{\norm{x}_X}} & \leq\sup\limits_{\norm{x}_X \neq 0} \pa{ \frac{\norm{T_0x}_Y}{\norm{x}_X}+ \frac{\norm{T_1x}_Y}{\norm{x}_X}}\\
& \leq\sup\limits_{\norm{x}_X \neq 0} \pa{ \frac{\norm{T_0x}_Y}{\norm{x}_X}} + \sup\limits_{\norm{x}_X \neq 0} \pa{\frac{\norm{T_1x}_Y}{\norm{x}_X}}\\
    \end{align*}
    Hence, $\norm{T_0+T_1} \leq \norm{T_0}+\norm{T_1}$ so that $\norm{\cdot}$ is \Subadditive. 
    For \ScalarHomogeneity, let $T \in BL(X,Y)$, $\alpha \in \F$, and $x \in X$ with $\norm{x}_X \neq 0$. 
    Then 
    \begin{align*}
        \frac{\norm{(\alpha T)x}_Y}{\norm{x}_X} = \frac{\norm{\alpha (Tx)}_Y}{\norm{x}_X} = \abs{\alpha} \frac{\norm{Tx}_Y}{\norm{x}_X}
    \end{align*}
    Hence taking the supremum finishes the proof.
\end{proof}
\begin{proof}[Proof of 2] 
   Let $T \neq 0 \in BL(X,Y)$. Then for some $x \in X$, $Tx \neq 0$. 
   Then $Tx$ has a neighborhood U disjoint from $0_Y$, 
   Hence $x \in T^{-1}(U)$ but not $0_X \in T^{-1}(U)$, since $T0_X = 0_Y$.
   Since U is a neighborhood of x disjoint from 0, 
   there is an $\epsilon > 0$ such that $0_X \subset \complement U \subset \complement \overline{B_X}(x;\epsilon)$,
   and therefore $\norm{x}_X > \epsilon$. 
   Since $\norm{x}_X > 0$, it is ranged over in the supremum defining $\norm{T}$, and so
   \begin{equation}
   0 < \frac{\norm{Tx}_Y}{\norm{x}_X} \leq \sup\limits_{\norm{x}_X \neq 0} \frac{\norm{Tx}_X}{\norm{x}_X}=\norm{T}
   \end{equation}
\end{proof}
\begin{proof}[Proof of 3] 
    Let $\alpha \in (0,\infty)$
   Let $T \in BL(X,Y)$. 
   Then, there is a sequence $\{x_i\} \subset X$ with each $\norm{x_i}_X \neq 0$ 
   such that 
   \begin{equation}
    \frac{\norm{Tx_i}_Y}{\norm{x_i}_X} \to \norm{T}
    \end{equation}
    For each $i \in \N$, define $y_i =\alpha  x_i/\norm{x_i}_X$. 
    then each $\norm{y_i} = \alpha$, 
    and by \ScalarHomogeneity
    of T, we have 
    \begin{equation}
    \frac{\norm{Ty_i}_Y}{\norm{y_i}_X} = \frac{\norm{Tx_i}_Y}{\norm{x_i}_X} \to \norm{T}
    \end{equation}
    , completing the proof. 
\end{proof}
\begin{proof}[Proof of 4] 
If we define, for 
$T \in BL(X,Y)$, 
$f(T) = \sup\limits_{0 < norm{x}_X \leq \alpha} \frac{\norm{Tx}_Y}{\norm{x}_X}$, then
since $\norm{\cdot}^{-1}((0,\alpha))\subset \norm{\cdot}^{-1}((0,\infty))$, we have $f(T) \leq \norm{T} $
and since $\norm{\cdot}^{-1}(\{\alpha\})\subset \norm{\cdot}^{-1}((0,\alpha))$, we have $\norm{T} \leq f(T)$. proving the first equality.
The second is found by applying the same arguement to $\alpha/2$ and realizing that $(0,\alpha/2] \subset (0,\alpha)$. 
\end{proof}
\begin{proof}[Proof of 5]
Let $T \in BL(X,Y)$ and $x \in X$. 
If $\norm{Tx}_Y \neq 0$, then $B_Y(Tx, \frac{\norm{Tx}_Y}{2})$ is a neighborhood of $Tx$ disjoint from 0.
Continuity of T impliese $x$ then has a neighborhood disjoint from $0 \in T^{-1}(0)$, implying
that $\norm{x}_X \neq 0$. 

Hence if $\norm{x}_X = 0$, then we know $\norm{Tx}_Y = 0$, so that the relation
\begin{equation}
\norm{Tx}_Y \leq \norm{T} \norm{x}_X
\end{equation}

If $\norm{x}_X \neq 0$, then by definition of supremum, 
\begin{equation*} 
\frac{\norm{Tx}_Y}{\norm{x}_X} \leq \norm{T}
\end{equation*}
so that $\norm{Tx}_Y \leq \norm{T} \norm{x}_X$. 
\end{proof}
\begin{proof}[Proof of 6]
    I assume the first 3 conditions
    and show that $S \in BL(X,Y)$.
    It is necessary and sufficient to 
    show that S is continuous.
    Let $F= \sup\limits_{\norm{x}_X \neq 0} \frac{\norm{Sx}_Y}{\norm{x}_X}$.
    If $F=0$, then $S(X) \subset \Ker_Y$. 
    Every neighborhood of every point in $\Ker_Y$
    contains $\Ker_Y$, so in that case continuity holds. 
    Suppose $F \neq 0$. 
    By translation invariance of the topology, 
    it is sufficient to consider neighborhoods of $0_Y \in Y$. 
    Let $\epsilon > 0$. 
    Define $V=B_X\pa{0; \frac{\epsilon}{F}}$. 
    Let $x_0 \in V$. 
    If $\norm{x_0}_X = 0$, then
    $S(x_0) \in S(\Ker_X) \subset \Ker_Y \subset B_Y(0;\epsilon)$. 
    If $\norm{x_0}_X \neq 0$, then 
    $\norm{Sx}_Y \leq F \norm{x}_X < \epsilon$, so
    $s(x_0) \in B_Y(0;\epsilon)$. 
    Hence $S\pa{B_X\pa{0;\frac{\epsilon}{F}}} \subset B_Y(0; \epsilon)$.
    so S is continuous,and this direction fo the proof is complete.

    Suppose conversely that $S \in BL(X,Y)$. 
    Then S is linear by definition
    , and the supremum expression is finite by part 1 
    of this result. 
    Since S is linear, $S0_X = 0_Y$. 
    Since S is continuous, 
    \begin{align*}
        S(\Ker_X) &= S\pa{\overline{\{0_X\}}}\\
        & \subset \overline{S\pa{\{0_X\}}}\\
        & =\overline{\{0_Y\}} \\
        & = \Ker_Y
    \end{align*}



\end{proof}
\begin{proof}[Proof of 7]
    $(3 \implies 2)$ is trivial, as is $(2 \implies 3)$.
    
    I now prove $(1\implies 3)$. 
    Let $\{T_i\}_{i \in \N}$ be a 
    \PseudometricCauchySequence.
    Let $\beta > 0$. 
    Let $\epsilon > 0$. 
    Then there exists $N \in \N$
    such that for $m,n > N$, 
    \begin{equation*}
    \norm{T_n-T_m} < \frac{\epsilon}{\beta}
    \end{equation*}
    Let $x \in B_X(0;\beta)$. 
    Then 
    \begin{align*}
        \norm{T_mx-T_nx}_Y & = \norm{(T_m-T_n)x}_Y\\
        & \leq \norm{T_m-T_n} \norm{x}_X\\
        & < \epsilon
    \end{align*}
    Since $x \in B_X(0;\beta)$ was arbitrary,
    $\{\{T_ix\}_{i \in \N} | x \in B_X(0;\beta)\}$ is
    \UniformlyCauchy.

    I now prove $(3 \implies 1)$. 
    Let $\epsilon > 0$.
    Then there is an $N \in \N$ 
    such that for $m,n>N$, 
    for each $x \in B_X(0;2)$, 
    \begin{equation*}
    \norm{T_mx-T_nx} < \epsilon
    \end{equation*}
    In particular, if $\norm{x}=1$, then 
    \begin{equation}
    \frac{\norm{(T_m-T_n)x}_Y}{\norm{x}_X} =\norm{(T_m-T_n)x}_Y < \epsilon
    \end{equation}
    Hence, by taking the supremum over such x
    and applying part 3 of this result, 
    $\norm{T_m-T_n} < \epsilon$. 
\end{proof}
\begin{proof}[Proof of 8]
    Let $T_i \to T$. 
    Let $x \in X$. 
    If $x \in \Ker_X$, then $T_i(x) \in \Ker_Y$ for $i \in \N$ and $T_x \in \Ker_Y$, 
    so convergence is obvious. 
    Suppose $\norm{x}_X > 0$. 
    Let $\epsilon > 0$. 
    Then there exists $N \in \N$ such that
    for $n>N$, $\norm{T_n-T} < \frac{\epsilon}{\norm{x}_X}$.
    For such n, 
    \begin{equation*}
    \norm{T_ix-Tx}_Y \leq \norm{T_i-T} \norm{x}_X < \epsilon
    \end{equation*}
\end{proof}
\begin{proof}[Proof of 9]
    
    $(\implies)$ 
    Suppose $T_i \to T$.
    Then, by \ref{prop:pseudometricconvergenceimpliespseudometriccauchy}, 
    $\{T_i\}_{i \in \N}$ is a \PseudometricCauchySequence.
    An application of part 8 of this result implies
    the pointwise convergence on a hamel basis. 


    $(\impliedby)$
    Let $\{x_\alpha\}_{\alpha \in A}$ be a Hamel basis for X. 
    Let $T_ix_\alpha \to y_\alpha$ for $\alpha \in A$. 
    Define $T: X \to Y$ by setting, for $x \in X$, 
    for any $\{\alpha_i\}_{i=1}^n \subset A$
    $\{\beta_i\}_{i=1}^n \subset \F$, 
    \begin{equation}
    T\pa{\sum_{i=1}^n \beta_i x_{\alpha_i}} = \sum_{i=1}^n \beta_i y_{\alpha_i}
    \end{equation}
    The uniqueness of a hamel basis representation implies
    that T is well defined. 
    It is clear also that T is linear, 
    and that $T\pa{\Ker_X} \subset \Ker_Y$. 

    Let $x \in X$. 
    Then we can find a unique representation, 
    $x=\sum_{j=1}^n \beta_j x_{\alpha_j}$ where $x_{\alpha_j} \in A$ and $\beta_j \in \F$ for every j. 
    For each $j \in \{1, ..., n\}$, there is an $N_j$ such that if
    $n_j > N_j$, the 
    \begin{equation}
    \norm{T_{n_j} x_{\alpha_j} -y_{\alpha_j}} < \frac{\epsilon}{n(\abs{\beta_j}+1)}
    \end{equation}
    Let $N=max\{N_j\}_{j=1}^n$. 
    Let $m>N$. 
    Then, we have 
    \begin{align*}
        \norm{T_mx-Tx} & = \norm{T_m\pa{\sum_{j=1}^n \beta_j x_{\alpha_j}}-T\pa{\sum_{j=1}^n \beta_j x_{\alpha_j}}}\\
        & = \norm{\sum_{j=1}^n \beta_j \pa{T_mx_{\alpha_j}-Tx_{\alpha_j}}}\\
        & = \sum_{j=1}^n \abs{\beta_j} \norm{T_mx_{\alpha_j}-y_{\alpha_j}}\\
        & < \epsilon 
    \end{align*}
    Since $m>N$ was arbitrtary , 
    $T_ix \to Tx$ for $x \in X$. 

    Since $T_ix \to Tx$ for $x \in B_X(0;2)$, 
    and since part 7 of this result, 
    paried with the assumption that
    $\{T_i\}_{i \in \N}$ is a \PseudometricCauchySequence, 
    $\left\{\{T_ix\}_{i \in \N}\right\}_{x \in B_X(0;2)}$ 
    is \UniformlyConvergent
    to $\{Tx\}_{x \in B_X(0;2)}$. 

    Let $\epsilon > 0$. 
    By \UniformConvergence, 
    there is an $N \in \N$ such that 
    for $n>N$, $x \in B_X(0;2)$, we have
    \begin{equation}
    \norm{T_nx-Tx} < \epsilon
    \end{equation}
    In particular, if $\norm{x} = 1$, 
    \begin{equation}
    \frac{\norm{(T_n-T)x}_Y}{\norm{x}_X} < \epsilon
    \end{equation}
    Implying first through part 6 of this result
    that $T \in BL(X,Y)$, and second that
    Hence $T_i \to T$ with respect to $\norm{\cdot}$. 
\end{proof}
\begin{proof}[Proof of 10]
    Let Y be Pseudometric Complete. 
    and let $X$ be \NonDegenerate. 
    Let $\{T_\alpha\}_{\alpha \in A} \subset BL(X,Y)$ 
    be a \PseudometricCauchySequence.
    Let $\{x_{\alpha}\}_{\alpha \in A}$ be a Hamel basis for X. 
    Let $\alpha \in A$. 
    If $\norm{x_\alpha}_X=0$, then $T_ix_{\alpha} \in \Ker$ for $i \in \N$, 
    and so $T_ix_{\alpha} \to 0$. 
    If $\norm{x_{\alpha}}_X > 0$, then 
    \begin{equation}
    \frac{(T_m-T_n)x_{\alpha}}{\norm{x_{\alpha}}} < 
    \end{equation}

%FINISH THIS RESULT
    I



\end{proof}

\end{prop}


\begin{rmk}

\end{rmk}

