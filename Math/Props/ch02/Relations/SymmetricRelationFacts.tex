\begin{prop}
    \label{prop:SymmetricRelation}
\rm
    Let $X \neq \emptyset$ 
    and let $R$ be a \Relation on $X$. 
    The following are true. 
    \begin{enumerate}[label=(\roman*), ref={\ref{prop:SymmetricRelation}~\roman*}]
    \item 
	\label{prop:SymmetricRelation:IntersectionSymmetric} 
	$R \cap R^{-1}$ is \SymmetricRelation.
    \item 
	\label{prop:SymmetricRelation:UnionSymmetric}
	$R \cup R^{-1}$ is \SymmetricRelation.
    \end{enumerate}
    \begin{proof}[Proof of \ref{prop:SymmetricRelation:IntersectionSymmetric}]
    If $R \cap R^{-1} = \emptyset$ 
	then it is trivally \SymmetricRelation.
    Suppose $R \cap R^{-1} \neq \emptyset$ and 
	let $(x,y) \in R \cap R^{-1}$. 
    Then $(x,y) \in R$, 
	implying by \ref{def:SymmetricRelation} that $(y,x) \in R$. 
    Also this implies $(x,y) \in R^{-1}$ so 
	by \ref{def:SymmetricRelation} we have 
    $(y,x) \in \pa{R^{-1}}^{-1}=R$. 
	Hence $(y,x) \in R \cap R^{-1}$, so \RelationSymmetry is 
    verified.
    \end{proof}
    \begin{proof}[Proof of \ref{prop:SymmetricRelation:UnionSymmetric}]
    Let $(x,y) \in R \cup R^{-1}$. Then either $(x,y) \in R$ 
    or $(x,y) \in R^{-1}$. 
    In the former case, 
	$(y,x) \in R^{-1} \subset R \cup R^{-1}$. 
    In the latter case, 
	$(y,x) \in R \subset R \cup R^{-1}$. 
    Hence in either case 
	$(y, x) \in R \cup R^{-1}$ and 
	\RelationSymmetry is verified.
    \end{proof}
\end{prop}
