\begin{prop}[\PseudometricTopology]
\label{prop:pseudometrictopology}
\rm
    Let $(X,d)$ be a  \PseudometricSpace and let $\scB$ be the set of \OpenBall's in $(X,d)$, along with $\emptyset$. Let $\scT_d$ denote $\scGeneratedTopology{X}{\scB}$. 
    The following are true. 
    \begin{enumerate}[label=(\roman*), ref={\ref{prop:pseudometrictopology}~\roman*}]
        \item 
        \label{prop:PseudometricTopology:Basis}
        $\scB$ is a \TopologyBasis for $\scT_d$. 
        \item 
        \label{prop:PseudometricTopology:FirstCountable}
        The \PseudometricInducedTopology is \FirstCountable.
    \end{enumerate}
    \begin{proof}[Proof of \ref{prop:PseudometricTopology:Basis}]
       Let $U,V \in \scB$. 
       Let $x \in U \cap V$.  
       Then there exists $x_u,x_v \in X$ and $\epsilon_u,\epsilon_v > 0$ 
       such that $U = B_d(x_u,\epsilon_u)$ and $V=B_d(x_v,\epsilon_v)$. 
       Define $\delta = min\pa{\frac{\epsilon_u - d(x,x_u)}{2}, \frac{\epsilon_v-d(x,x_v)}{2}}$.
       Let $y \in B(x;\delta)$. 
       Then
       \begin{equation*}
       d(y,x_u) \leq d(y,x)+d(x,x_u) \leq \delta+ d(x,x_u) \leq \frac{\epsilon_u}{2} + \frac{d(x,x_u)}{2} < \epsilon_u
       \end{equation*}
       Similarly, $d(y,x_v) < \epsilon_v$, so $x \in B_d(x;\delta) \subset U \cap V$. 
       By \ref{prop:BasisOfGeneratedTopology}, the result holds. 
    \end{proof}
    \begin{proof}[Proof of \ref{prop:PseudometricTopology:FirstCountable}]
        Let $x_0 \in X$. 
        I claim that 
        \begin{equation*}
            \scB_{x_0}:= \left\{ B_d\pa{x_0; \frac{1}{n}} : n \in \N\right\}
        \end{equation*}
        is a \NeighborhoodBasis for $(X,\T_d)$ at $x_0$. 
        Let $U \in \scU_{\T_d}(x)$ be \SetOpen in $\T_d$. 
        Since $\scB$ is a \TopologyBasis for $\T_d$, for some $y_0 \in X$ and $\epsilon > 0$, 
        $x_0 \in B_d(y_0; \epsilon) \subset U$. 
        Let $\delta = d(x_0, y_0)$. Then $\epsilon - \delta > 0$. 
        Define
        \begin{equation*}
            n = \ceil{ \frac{1}{\epsilon - \delta}}
        \end{equation*}
        Then we have 
        \begin{equation*}
            B_d\pa{x_0 ; \frac{1}{n}} \subset B_d(x_0 : \epsilon - \delta) \subset B(y_0 ; \epsilon) \subset U
        \end{equation*}
    \end{proof}
\end{prop}
