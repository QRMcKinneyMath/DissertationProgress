\begin{prop}[\Pseudometrizable Prequotient]
    \label{prop:pseudometrizableprequotient}
    \rm
    Let $(X,\T_X)$ be a \TopologicalSpace 
    with \RelationOfEqualNeighborhoodFilters $\cong$, and
    with \QuotientTopologicalSpace  $\pa{X/\cong, \T_{X/\cong}}$
    and \QuotientMap T. Let $\pa{X/\cong, \T_{X/\cong}}$ be \Pseudometrizable with \Pseudometric $\tilde{d}$. 
    The following hold. 
    \begin{enumerate}[label=(\roman*), ref={\ref{prop:pseudometrizableprequotient}~\roman*}]
        \item  
        \label{prop:PseudoPre:Pseudometrizable}
        Define $d:X^2 \to [0,\infty)$ by  $d(x,y) = \tilde{d}\pa{[x],[y]}$. 
        Then $\tilde{d}$ is a \Pseudometric on $X$ which is 
        \PseudometricCompatible with $\scT_X$. 
        \item 
        \label{prop:PseudoPre:Metrizable}
        $\tilde{d}$ is a \Metric $(X/\cong, \T_{X/\cong})$.
        \item 
        \label{prop:PseudoPre:Injective}
        If $T$ is \Injective, then $d$ as defined above is a \Metric on $X$.
    \end{enumerate}
    \begin{proof}[Proof Of \ref{prop:PseudoPre:Pseudometrizable}]
    We first prove $d$ to be a \Pseudometric on $X$.
        First, observe that if $x,y \in X$, then
        $d(x,y) =\tilde{d}([x],[y]) \in [0,\infty)$
        so that d is well defined. 
        Also, 
        $ d(x,y) = \tilde{d}([x],[y])=\tilde{d}([y],[x])=d(y,x)$,
        so d is \CommutativeFunction.
        Furthermore, 
        \begin{align*}
            d(x,z) & = \tilde{d}([x],[z])\\
            & \leq \tilde{d}([x],[y])+\tilde{d}([y], [z])\\
            & = d(x,y)+d(y,z)
        \end{align*}
        so d satisfies the \TriangleInequality. 
        Lastly, 
        $d(x,x)=\tilde{d}([x],[x])=0$, 
        and so $d$ is a \Pseudometric on $X$. 
        
        Let $\T_d$ denote the \PseudometricTopology on $(X,d)$. 
        What remains to show is that $\T_X=\T_d$. 
        Since $d(x,y)=\tilde{d}([x], [y])=\tilde{d}(Tx, Ty)$, T is an \Isometry. 
        Let $x \in U \in \T_X$. Then $[x] \in T(U) \in \T_{X/\cong}$. 
        Hence, there is an $\epsilon > 0$ such that $B_{\tilde{d}}([x], \epsilon) \subset T(U)$. 
        By \ref{prop:QST:OpenSetFiber}, $T^{-1}(B_{\tilde{d}}([x], \epsilon) \subset T^{-1}(T(U))=U$.
        Furthermore, by 
        \ref{prop:QST:QuotientMapContinuous}, 
        $T^{-1}(B_{\tilde{d}}([x], \epsilon) \in \T_X$. 
        Since T is an \Isometry $B_d(x, \epsilon) = T^{-1}(B_{\tilde{d}}([x],\epsilon) \subset U$. 
        Thus we have found an \OpenBall contained in $U$  containing an arbitrary point of U.
        Hence, $\T_X \subset \T_d$. As part of the preceeding arguement we also showed that an  arbitrary d-\OpenBall was in $\T_X$, so $\T_d \subset \T_X$, and so equality holds and we're done. 
    \end{proof} 
    \begin{proof}[Proof of \ref{prop:PseudoPre:Metrizable}]
        Let $x,y \in X$ wtih $[x] \neq [y]$. 
        Then $x \not \cong y$. By \ref{prop:relationofzerodistance}, $x \not \cong_d y$. 
        Hence $\tilde{d}([x],[y])=d(x,y) > 0$. 
    \end{proof}
    \begin{proof}[Proof of \ref{prop:PseudoPre:Injective}]
        Let T be \Injective, and suppose $x,y \in X$ with $x \neq y$. 
        Then $[x]=Tx\neq Ty=[y]$, 
        so by \ref{prop:PseudoPre:Metrizable}, $d(x,y) = \tilde{d}([x],[y]) > 0$. 
        
    \end{proof} 
\end{prop} 
