\begin{prop}[Pseudometrizable Prequotient]
    \label{prop:pseudometrizableprequotient}
    Let $(X,\T_X)$ be a topological space 
    with \QuotientTopologicalSpace  $\pa{X/\cong, \T_{X/\cong}}$
    and \QuotientMap T. Let $\pa{X/\cong, \T_{X/\cong}}$ be \Pseudometrizable with \Pseudometric $\tilde{d}$. 
    
    The following hold. 
    \begin{enumerate}
        \item  $(X,\T_X)$ is \Pseudometrizable. 
        \item $(X/\cong, \T_{X/\cong})$ is \Metrizable. 
        \item If T is injective, then $(X,\T_X)$ is metrizable. 
    \end{enumerate}
    \begin{proof}[Proof Of One]
        Define $d:X \times X \to [0,\infty)]$ by 
        \begin{equation*}
            d(x,y) = \tilde{d}\pa{[x], [y]}.
        \end{equation*}
        Then
        \begin{equation*}
            d(x,y) =\tilde{d}([x],[y]) \in [0,\infty)
        \end{equation*}
        so that d is well defined. 
        
        Also, 
        \begin{equation*}
            d(x,y) = \tilde{d}([x],[y])=\tilde{d}([y],[x])=d(y,x)
        \end{equation*}
        , so d is a \SymmetricMap.
        
        Also, 
        \begin{align*}
            d(x,z) & = \tilde{d}([x],[z])\\
            & \leq \tilde{d}([x],[y])+\tilde{d}([y], [z])\\
            & = d(x,y)+d(y,z)
        \end{align*}
        so d satisfies the \TriangleInequality. 
        Also, 
        \begin{equation}
            d(x,x)=\tilde{d}([x],[x])=0
        \end{equation}
        and so d is a \Pseudometric on X. 
        
        Let $\T_d$ denote the \PseudometricTopology on $(X,d)$. What remains to show is that $\T_X=\T_d$. 
        
        
           By \ref{def:pseudometricinducedmetric} paired with how d is defined, $\tilde{d}$ is the \PseudometricInducedMetric of $(X,d)$. Let $\cong_d$ denote the \RelationOfZeroDistance on $(X,d)$, and 
           let $\cong_{\T_X}$ denote the \RelationOfEqualNeighborhoodFilters on $(X,\T_X)$. 
           
           %claim: $\cong_d=\cong_{\T_X}$ to use a theormem. 

    \end{proof} 
    \begin{proof}[Proof of Two]
    \end{proof}
    \begin{proof}[Proof of Three]
    \end{proof} 
\end{prop} 
