\begin{prop}[\Topology from \NeighborhoodFilters]
\label{prop:TopFromNbhFilter}
\rm
Let $X$ be a nonempty set.
For each $x \in X$, let 
$\NeighborhoodFilterInstance{}(x) \subset \scPowerSet{X}$ such that 
each $\NeighborhoodFilterInstance{}(x)$ satisfies 
\ref{def:Filter:IsNonempty}, 
\ref{def:Filter:SubsetProperty}, 
\ref{def:Filter:FiniteIntersectionProperty}, 
and
\ref{prop:NeighborhoodFilter:Containsx}.
Further assume that the collection
$\{ \NeighborhoodFilterInstance{}(x) : x \in X \}$ 
satisfies 
\ref{prop:NeighborhoodFilter:CharacteristicProperty}.
Then there exists a unique topology $\T$ 
on $X$ such that 
for each $x \in X$, 
$\NeighborhoodFilterInstance{}(x)$ is the 
\NeighborhoodFilter for $\T$ at $x$. 
\begin{proof}
Define $\scT = \{U \subset X : (\forall x \in U) (U \in \scU_x)\}$. 
I first show that $\scT$ is a \Topology on $X$. 
Clearly, $\emptyset \in \scT$. 
By \ref{def:Filter:SubsetProperty}, for each $x \in X$, there 
exists $B \subset X$ such that $B \in \scU_x$. 
Hence, by 
\ref{def:Filter:SubsetProperty}, 
$X \in \scU_x$. 
Hence, $X \in \scT$.
Let $U_1,U_2 \in \scT$. 
Let $x \in U_1 \cap U_2$. 
Then $x \in U_1`$ and $x \in U_2$, so $U_1 \in \scU_x$ and $U_2 \in \scU_x$. 
By 
\ref{def:Filter:FiniteIntersectionProperty}, $U_1 \cap U_2 \in \scU_x$. 
Hence $U_1 \cap U_2 \in \scT$.
By 
\ref{prop:FiniteClosure:Intersection}, 
$\scT$ is closed under finite intersections.
Let $\{U_\alpha\}_{\alpha \in A} \subset \scT$. 
Let $x \in \bigcup\limits_{\alpha \in A} U_\alpha$. 
Then for some $\beta \in A$, $x \in U_\beta$. 
Hence $U_\beta \in \scU_x$. 
Since $U_\beta \subset\bigcup\limits_{\alpha \in A} U_\alpha$, 
by \ref{def:Filter:SubsetProperty}, 
$\bigcup\limits_{\alpha \in A} U_\alpha \in \scT$. 
Hence $\scT$ is a \Topology on $X$. 

I now show that $\scT$ has the required properties. 
Let $x \in X$. 
I must show $\scU_x = \scU_{\scT}(x)$.
Let $N \in \scU_{\scT}(x)$. 
Then there exists $U \in \scT$ with $x\in U \subset N$. 
Since $x \in U \in \scT$, $U \in \scU_x$. 
Since $U \subset N$, $N \in \scU_x$. 
Hence $\scU_{\scT}(x) \subset \scU_x$. 
Now let $U \in \scU_x$. 
Then by 
\ref{prop:NeighborhoodFilter:CharacteristicProperty}
there exists $V \in \scU_x$ 
such that for every $y \in V$, 
$U \in \scU_y$. 
Let $y \in V$. 
Then $\exists \tilde{U} \in \scU_y$ such that for 
every $z \in \tilde{U}$, $V \in \scU_{\scT}(z)$. 
Since $y \in \tilde{U}$, $V \in \scU_y$. 
Since $y \in V$ was arbitrary, $V \in \scT$. 
Hence $x \in V \subset U \in \scU_{\scT}(x)$. 
Thus $\scU_x \subset \scU_{\scT}(x)$. 

I now show that the $\scT$ is the unique 
\Topology with this property. 
Let $\scT_1$ be another \Topology on $X$ 
such that for each $x \in X$, 
$\scU_{\scT}(x) = \scU_x = \scU_{\scT_1}(x)$. 
Let $U \in \scT$. 
Let $x \in U$. 
Then $\exists U_x \in \scU_{\scT}(x) = \scU_{\scT_1}(x)$ such that
$U_x \subset U$. 
Since $x \in U_x \in \scU_{\scT_1}(x)$, 
there exists $x \in V_x \in \scT_1$ such that $V_x \subset U_x$. 
Hence $U = \bigcup\limits_{x \in U} V_x \in \scT_1$. 
Hence $\scT \subset \scT_1$.
The proof for the other direction is identical.
Hence uniqueness holds.
\end{proof}
\end{prop}
