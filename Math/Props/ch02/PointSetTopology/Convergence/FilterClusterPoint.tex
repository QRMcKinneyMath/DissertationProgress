\begin{prop}[\FilterClusterPoint]
\label{prop:FilterClusterPoint}

\rm
Let $(X,\scT)$ be a \TopologicalSpace.
Let $\scB$ and $\scD$ be \FilterBaseEquivalent \FilterBases. 
Let $\scF$ be the \Filter \FilterGeneratedBy $\scB$.
Let $\scE$ be a \FilterBase on $X$ which generates a \Filter 
which is \CoarserFilter than $\scF$. 
Let $\scG$ be an \UltraFilter in $X$. 
Let $x \in X$. 
The following are true. 
\begin{enumerate}[label=(\roman*), ref={\ref{prop:FilterClusterPoint}.~\roman*}]
\item 
\label{prop:FilterClusterPoint:Closed}
The set of $\FilterClusterPoints$ of $\scB$ is \SetClosed in $(X,\scT)$. 
\item
\label{prop:FilterClusterPoint:LimitIsClusterPoint}
If $x$ is a \FilterLimit of $\scB$, then $x$ is a \FilterClusterPoint of $\scB$. 
\item
\label{prop:FilterClusterPoint:CoarserFilter}
If $x$ is a \FilterClusterPoint of $\scB$, then 
$x$ is a \FilterClusterPoint of $\scE$. 
\item
\label{prop:FilterClusterPoint:Equivalent}
$x$ is a \FilterClusterPoint of $\scB$ 
if and only if 
$x$ is a \FilterClusterPoint of $\scD$.
\item 
\label{prop:FilterClusterPoint:Filter}
$x$ is a \FilterClusterPoint of $\scB$ 
if and only if 
$x$ is a \FilterClusterPoint of $\scF$, viewed as a \FilterBase.
\item 
\label{prop:FilterClusterPoint:FundamentalSystem}
Let $\scU$ be a \FundamentalSystemOfNeighborhoods for $\scT$ at $x$. 
Then $x$ is a \FilterClusterPoint for $\scB$ if and only if for each $U \in \scU$ and 
for each $B \in \scB$, $U \cap B\neq \emptyset$. 
\item 
\label{prop:FilterClusterPoint:Finer}
$x$ is a \FilterClusterPoint for $\scF$ if and only if there is a 
\Filter \FinerFilter than $\scF$ for which $x$ is a \FilterLimit. 
\item 
\label{prop:FilterClusterPoint:Ultrafilter}
$x$ is a \FilterClusterPoint of $\scG$ if and only if $x$ is a \FilterLimit of $\scG$. 
\item
\label{prop:FilterClusterPoint:CoarserTopology}
If $x$ is a \FilterClusterPoint for $\scB$ in $(X,\scT)$, 
and if $\scT_1$ is a \TopologyCoarser 
\Topology on $X$ than $\scT$, then 
$x$ is a \FilterClusterPoint for $\scB$ in $(X,\scT_1)$.
\end{enumerate}
\begin{proof}[Proof of \ref{prop:FilterClusterPoint:Closed}]
By \ref{def:FilterClusterPoint}, 
the set of \FilterClusterPoints of $\scB$ is an 
intersection of \SetClosed sets and is therfore itself \SetClosed.
\end{proof}
\begin{proof}[Proof of \ref{prop:FilterClusterPoint:CoarserFilter}]
Since $\scB$ generates a \Filter which is 
\FinerFilter than that generated by $\scE$, 
$\scNested{\scB}{\scE}$
holds. 
Let $x$ be a \FilterClusterPoint of $\scB$. 
Let $E \in \scE$. 
Then there exists $B \in \scB$ 
such that $B \subset E$. 
Since $x$ is a \FilterClusterPoint of $\scB$, 
$x \in \overline{B} \subset \overline{E}$. 
Since $E \in \scE$ was arbitrary, 
$x \in \bigcap\limits_{E \in \scE}\overline{E}$
and is therefore a \FilterClusterPoint of $\scE$. 
\end{proof}
\begin{proof}[Proof of \ref{prop:FilterClusterPoint:Equivalent}]
This is a direct result of two applications of \ref{prop:FilterClusterPoint:CoarserFilter}.
\end{proof}
\begin{proof}[Proof of \ref{prop:FilterClusterPoint:Filter}]
Since $\scF$, viewed as a \FilterBase
is \FilterBaseEquivalent to $\scB$, 
this result follows from a direct application of 
\ref{prop:FilterClusterPoint:Equivalent}.
\end{proof}
\begin{proof}[Proof of \ref{prop:FilterClusterPoint:FundamentalSystem}]
$(\implies)$
Let $U \in \scU$. 
Then $U \in \scU_{\scT}(x)$. 
Let $x$ be a \FilterClusterPoint for $\scB$. 
Let $B \in \scB$. 
Then $x \in \overline{B}$. 
Since $U \in \scU_{\scT}(x)$, $U \cap B \neq \emptyset$. 

$(\impliedby)$
Let $B \in \scB$. 
Let $U \in \scU_{\scT}(x)$. 
Since $\scU$ is a \FundamentalSystemOfNeighborhoods
for $X$ at $x$, there exists $V \in \scU$ with
$x \in V \subset U$. 
By assumption, $V \cap B \neq \emptyset$. 
Hence $U \cap B \neq \emptyset$. 
Since $U \in \scU_{\scT}(x)$ was arbitrary, 
$x \in \overline{B}$. 
Since $B \in \scB$ was arbitrary, 
$x$ is a \FilterClusterPoint of $x$. 
\end{proof}
\begin{proof}[Proof of \ref{prop:FilterClusterPoint:LimitIsClusterPoint}]
Let $x$ be a \FilterLimit of $\scB$. 
Let $B \in \scB$.
Let $U \in \scU_{\scT}(x)$. 
By \ref{prop:FilterConvergenceFacts:FundamentalSystemForAll}, 
there exists $V \in \scB$ such that $V \subset U$. 
By 
\ref{def:FilterBase:IntersectionProperty}, 
there exists a nonempty $W \in \scB$, such that $W \subset B \cap V$. 
Since $W \cap U \neq \emptyset$, $B \cap U \neq \emptyset$. 
Since $\scU_{\scT}(x)$ is a \FundamentalSystemOfNeighborhoods for $X$ at $x$,
since $U \in \scU_{\scT}(x)$ was arbitrary, 
and since $B \in \scB$ was arbitrary, 
we can apply
\ref{prop:FilterClusterPoint:FundamentalSystem} to conclude $x$ is a \FilterClusterPoint
of $\scB$. 


\end{proof}
\begin{proof}[Proof of \ref{prop:FilterClusterPoint:Finer}]
$(\implies)$
Define $A = \scF \cup \scU_{\scT}(x)$. 
Let $x$ be a \FilterClusterPoint of $\scF$. 
Then, since $\scU_{\scT}(x)$ is a 
\FundamentalSystemOfNeighborhoods for $X$ at $x$, 
by 
\ref{prop:FilterClusterPoint:FundamentalSystem}
and
\ref{prop:FiniteClosure:Intersection}, 
finite intersections
of elements of $A$  are nonempty. 
Hence, by 
\ref{prop:FilterExistence}, 
there is a \Filter $\scG$ on $X$ which contains $A$. 
By construction, 
$\scG$ is \FinerFilter than $\scF$
and $\scU_{\scT}(x) \leq \scG$, 
so $x$ is a \FilterLimit of $\scG$. 

$(\impliedby)$
Let $\scG$ be a \FinerFilter than $\scF$ 
and let $x$ be a \FilterLimit for $\scG$. 
By \ref{prop:FilterClusterPoint:LimitIsClusterPoint}, 
$x$ is a \FilterClusterPoint for $\scG$. 
By \ref{prop:FilterClusterPoint:CoarserFilter}, 
since $\scF$ is \CoarserFilter than $\scG$, 
$x$ is a \FilterClusterPoint of $\scF$. 
\end{proof}
\begin{proof}[Proof of \ref{prop:FilterClusterPoint:Ultrafilter}]
$(\impliedby)$
This direction is a direct consequence of 
\ref{prop:FilterClusterPoint:LimitIsClusterPoint}.

$(\implies)$
this direction is a direct consequence of 
\ref{prop:FilterClusterPoint:Finer}
combined with the fact that an \Ultrafilter 
is not properly contained in any \Filter.
\end{proof}
\begin{proof}[Proof of \ref{prop:FilterClusterPoint:CoarserTopology}]
If $\scT_1$ is \TopologyCoarser than $\scT$, than 
for any $B \in \scB$, $\overline{B}_{\scT} \subset \overline{B}_{\scT_1}$, implying
\begin{equation*}
\bigcap\limits_{B \in \scB} \overline{B}_{\scT} \subset \bigcap\limits_{B \in \scB} \overline{B}_{\scT_1}
\end{equation*}
\end{proof}
\end{prop}
