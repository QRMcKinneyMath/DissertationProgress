\begin{prop}[\Hausdorff Characterizations]
    \label{prop:HausdorffCharacterizations}
    \rm
    Let $(X,\scT)$ be a \TopologicalSpace. 
    The following are equivalent.
    \begin{enumerate}[label=(\roman*), ref={\ref{prop:HausdorffCharacterizations}~\roman*}]
        \item
        \label{prop:HausdorffCharacterizations:Hausdorff}
        $X$ is \Hausdorff.
        \item
        \label{prop:HausdorffCharacterizations:ClosedNeighborhoodsConvergeToPoint}
        For all $x \in X$, if $\NeighborhoodFilterInstance{\scT}(x)$ is the 
        \NeighborhoodFilter of $X$ at $x$, then 
        \begin{equation*}
        \bigcap\limits_{U \in \NeighborhoodFilterInstance{\scT}(x)} \ClosureMark{U} = \{x\}
        \end{equation*}
        \item
        \label{prop:HausdorffCharacterizations:ClosedBinaryDiagonal}
        \scSetDiagonal{X} is \SetClosed in $X \times X$. 
        \item
        \label{prop:HausdorffCharacterizations:ClosedDiagonal}
        For any index set $A$, \scInfiniteSetDiagonal{A}{X} is \SetClosed in $\prod\limits_{\alpha \in A} X$.
        \item
        \label{prop:HausdorffCharacterizations:UniqueLimit}
        A \Filter $\scF$ in $X$ has at most one \FilterLimit.
        \item
        \label{prop:HausdorffCharacterizations:UniqueClusterPoint}
        If a \Filter $\scF$ in $X$ \FilterConverges, say $\scF \to x$, then 
        $x$ is that \Filter's only \FilterClusterPoint.
    \end{enumerate}
\begin{proof}[Proof of \ref{prop:HausdorffCharacterizations:Hausdorff} $\implies$ \ref{prop:HausdorffCharacterizations:ClosedNeighborhoodsConvergeToPoint}]
Clearly, $\{ x \} \subset \bigcap\limits_{U \in \scU_{\scT}(x)} \overline{U}$.
For the other direction, let $y \neq x$. 
Then, since $X$ is \Hausdorff, there exist
$U_y \in \scU_{\scT}(y)$ and $U_x \in \scU_{\scT}(x)$
such that
$U_y \cap U_x = \emptyset$. 
Hence, $y \not \in \overline{U_x}$.
Hence, $y \not \in \bigcap\limits_{U \in \scU_{\scT}(x)} \overline{U}$.
\end{proof}
\begin{proof}[Proof of \ref{prop:HausdorffCharacterizations:ClosedNeighborhoodsConvergeToPoint} $\implies$ \ref{prop:HausdorffCharacterizations:ClosedBinaryDiagonal}]
Let $(x,y) \in \pa{X \times X} \setminus \scSetDiagonal{X}$. 
Then $y \neq x$, so $y \not \in \bigcap\limits_{U \in \scU_{\scT}(x)} \overline{U}$. 
Hence, there exists $U_x \in \scU_{\scT}(x)$ such that 
$y \not \in \overline{U_x}$. 
Hence, there exists $U_y \in \scU_{\scT}(y)$ such that
$U_y \cap U_x = \emptyset$. 
Then $U_x \cap U_y$ is a \Neighborhood of $(x,y)$ in $X \times X$
which is \Disjoint from $\scSetDiagonal{X}$.
Thus $U_x \times U_y \subset \pa{X \times X} \setminus \scSetDiagonal{X}$, so
that $\pa{X \times X} \setminus \scSetDiagonal{X}$ is \SetOpen,
\end{proof}
\begin{proof}[Proof of \ref{prop:HausdorffCharacterizations:ClosedBinaryDiagonal} $\implies$ \ref{prop:HausdorffCharacterizations:ClosedDiagonal}]
Let $\{x_\alpha\}_{\alpha \in A} \in \pa{\prod\limits_{\alpha \in A} X} \setminus \scInfiniteSetDiagonal{A}{X}$. 
Then there exists $\alpha,\beta \in A$ such that $x_\alpha \neq x_\beta$. 
This implies $(x_\alpha,x_\beta ) \in \pa{X \times X} \setminus \scSetDiagonal{X}$. 
Hence there is an $X \times X$-\Neighborhood $U$ of $(x,y)$ such that $U \cap \scSetDiagonal{X} = \emptyset$. 
There exists $U_\alpha \in \scU_{\scT}(x_\alpha)$ and $U_{\beta} \in \scU_{\scT}(x_\beta)$ such that $(x_\alpha,x_\beta) \in U_{\alpha} \times U_{\beta} \subset \scSetDiagonal{X}$. 
Then $\pi_{\alpha}^{-1}(U_\alpha) \cap \pi_{\beta}^{-1}\pa{U_{\beta}}$ 
is a \Neighborhood of $\{x_\alpha\}_{\alpha \in A}$ which is disjoint from 
$\scInfiniteSetDiagonal{A}{X}$.
Since
$\{x_\alpha\}_{\alpha \in A} \in \pa{\prod\limits_{\alpha \in A} X} \setminus \scInfiniteSetDiagonal{A}{X}$
was arbitrary, 
we conclude 
i$\pa{\prod\limits_{\alpha \in A} X} \setminus \scInfiniteSetDiagonal{A}{X}$
is \SetOpen.
\end{proof}
\begin{proof}[Proof of \ref{prop:HausdorffCharacterizations:ClosedDiagonal} $\implies$ \ref{prop:HausdorffCharacterizations:ClosedBinaryDiagonal}]
Trivial and obvious.
\end{proof}
\begin{proof}[Proof of \ref{prop:HausdorffCharacterizations:ClosedBinaryDiagonal} $\implies$ \ref{prop:HausdorffCharacterizations:Hausdorff}]
Let $x,y \in X$ with $x \neq y$. 
Then $(x,y) \in X \times X \setminus \scSetDiagonal{X}$. 
Since $\scSetDiagonal{X}$ is \SetClosed, there is an 
open $U \in \scU_{\scT_{X\times X}}((x,y))$ such that 
$U \cap \scSetDiagonal{X} = \emptyset$. 
There exists $U_x \in \scU_{\scT}(x)$ and $U_y \in \scU_{\scT}(y)$
such that $U_x \times U_y \subset U \subset \pa{X \times X} \setminus \scSetDiagonal{X}$. 
Hence, $U_x \cap U_y = \emptyset$. 
Since $x \neq y$ were arbitrary, $X$ is \Hausdorff.
\end{proof}
\begin{proof}[Proof of \ref{prop:HausdorffCharacterizations:Hausdorff} $\implies$ \ref{prop:HausdorffCharacterizations:UniqueClusterPoint}]
Let $X$ be \Hausdorff.
Let $\scF$ be a \Filter in $X$. 
Let $x_0$ be a \FilterLimit of $x_0$. 
Let $y \in X$ with $y \neq x_0$. 
Then there exists $U_y \in \scU_{\scT}(y)$ 
and $U_x \in \scU_{\scT}(x)$ with $U_x \cap U_y = \emptyset$. 
Since $x$ is a \FilterLimit of $\scF$, for some $F \in \scF$, 
$F \subset U_x$. 
Hence $F \cap U_y = \emptyset$. 
By 
\ref{prop:FilterClusterPoint:FundamentalSystem}, 
$y$ is not a \FilterClusterPoint of $\scF$, 
\end{proof}
\begin{proof}[Proof of \ref{prop:HausdorffCharacterizations:UniqueClusterPoint} $\implies$ \ref{prop:HausdorffCharacterizations:UniqueLimit}]
This result is a direct application of 
\ref{prop:FilterClusterPoint:LimitIsClusterPoint}.
\end{proof}
\begin{proof}[Proof of \ref{prop:HausdorffCharacterizations:UniqueLimit} $\implies$ \ref{prop:HausdorffCharacterizations:Hausdorff}]
Let $x,y \in X$ with $x \neq y$ and suppose $x$ and $y$ 
don't have any \Disjoint \Neighborhoods. 
Then by 
\ref{prop:FilterExistence}, 
there exists a \Filter on $X$
containing 
$\scU_{\scT}(x) \cup \scU_{\scT}(y)$, 
which would have both $x$ as a \FilterLimit
and $y$ as a \FilterLimit.
\end{proof}
\end{prop}
