\begin{prop}[Existence of \Balanced \NeighborhoodBasis of 0 in a \TVS]
    \label{prop:ExistenceOfBalancedNeighborhoods}
    \rm
    Let $(X,\T)$ be a 
    \TVS
    over a 
    \Field
    $\F$.
    The following are true. 
    \begin{enumerate}[label=(\roman*), ref={\ref{prop:ExistenceOfBalancedNeighborhoods}~\roman*}]
        \item 
        \label{prop:Bal1}
        If 
            $U \in \scU_{\T}(0)$,
            then there is a 
            \Balanced, \SetOpen
            $V \subset U$
            such that 
            $V \in \scU_{\T}(0)$.
        \item 
        \label{prop:Bal2}
        There exists a 
            \NeighborhoodBasis
            about $0 \in X$ 
            for $\T$ 
            consisting entirely 
            of \Balanced sets. 
        \item 
        \label{prop:Bal3}
        If 
            $U \in \scU_{\T}(0)$ is \ConvexSet,
            then there is a 
            \ConvexSet
            \Balanced, 
            \SetOpen
            $V \subset U$
            such that 
            $V \in \scU_{\T}(0)$.
        \item 
        \label{prop:Bal4}
        If $(X,\T)$ is 
            \LocallyConvex, 
            then there exists a 
            \NeighborhoodBasis
            about $0 \in X$ 
            for $\T$ 
            consisting entirely 
            of 
            \Balanced
            \ConvexSet 
            sets.
    \end{enumerate}

    \begin{proof}[Proof of \ref{prop:Bal1}] 
    Since scalar multiplication is \ContinuousAt $0$, 
    there is an \SetOpen disk $V \subset \mathbb{F}$
    and an \SetOpen $W \subset X$ with $0 \in W$ such that
    $VW \subset U$. 
    $VW$ is clearly balanced. 
    \end{proof}
    \begin{proof}[Proof of \ref{prop:Bal2}] 
    Let $\{U_{\alpha}\}_{\alpha \in A}$ be a 
    \LocalBasis for $\scT$. 
    Then, by \ref{prop:Bal1}, for each $\alpha \in A$, 
    there is a $W_\alpha \subset U_\alpha$ such that
    $W_\alpha$ is \BalancedSet and $0 \in W_\alpha$. 
    Clearly $\{W_\alpha\}_{\alpha \in A}$ forms a 
    \LocalBasis for $X$. 
    \end{proof}
    \begin{proof}[Proof of \ref{prop:Bal3}] 
    By \ref{prop:Bal1}, there is a 
    \BalancedSet \SetOpen $W \subset U$. 
    Let $\alpha \in \mathbb{C}$ with $\abs{\alpha} = 1$. 
    Then $\alpha^{-1}W = \subset W  \subset U$. 
    Hence $W \subset \alpha U$, so 
    if we define $A = \bigcap\limits_{\abs{\alpha} = 1 } \alpha U$, 
    then $0 \in W \subset A$.
    Thus $0 \InteriorMark{A}$. 
    For this reason, it suffices to show $\InteriorMark{A}$ is 
    \BalancedSet.
    It then suffices to show $A$ is \BalancedSet.
    Let $\alpha \in \mathbb{Z}$ with $\abs{\alpha} \leq 1$. 
    Then $\alpha = r \beta$ for some $\beta \in \mathbb{C}$ with $\abs{\beta} = 1$ 
    and $r \in [0,1]$. 
    Since $\alpha U$ is \ConvexSet and contains $0$, 
    $r\alpha U \subset \alpha U$. 
    Hence, 
    \begin{equation*}
    r \beta A = r \beta \bigcap\limits_{\abs{\alpha} = 1} \alpha U = \bigcap\limits_{\abs{\alpha} = 1 } r \alpha U \subset \bigcap\limits_{\abs{\alpha} = 1} \alpha U = A
    \end{equation*}
    Thus $A$ is \BalancedSet.
    \end{proof}
    \begin{proof}[Proof of \ref{prop:Bal4}] 
    This is a similar arguement to that of \ref{prop:Bal2}.
    \end{proof}
\end{prop}
