\begin{prop}
    \label{prop:FilterBase}\rm
    Let $X$ be a nonempty set.
    Let
    $A \subset \scPowerSet{X}$. 
    and define 
    \begin{equation*}
    \scU=\{U \subset X : (\exists a \in A ) ( a \subset U ) \}
    \end{equation*}
    The following are equivalent. 
    \begin{enumerate}
    \item $A$ is a \FilterBase on $X$. 
    \item $\scU$ is a \Filter on $X$. 
    \end{enumerate}
    \begin{proof}[1 $\implies$ 2]
    Supose $A$ is a \FilterBase on $X$.
    By $\ref{def:FilterBase:DoesntContainEmptySet}$, $\emptyset \not \in A$, so 
    $\emptyset \not \in \scU$, implying that $\scU$ satisfies
    \ref{def:Filter:DoesntContainEmpty}. 
    Also, by \ref{def:FilterBase:IsNotEmpty} $\emptyset \neq A \subset \scU$, so $\scU$ satisfies
    \ref{def:Filter:IsNonempty}. That $\scU$ satisfies \ref{def:Filter:SubsetProperty} is obvious. 
    Finally, if $G_1,G_2 \in \scU$, then there exists $U_1,U_2 \in A$ such that $A_i \subset G_i$. 
    By \ref{def:FilterBase:IntersectionProperty}, there is $B \in A$ satisfying 
    $B \subset U_1 \cap U_2 \subset G_1 \cap G_2$, so $G_1 \cap G_2 \in \scU$, implying $\scU$ is a \Filter on $X$. 
    \end{proof}
    \begin{proof}[1 $\impliedby$ 2]
        If $A = \emptyset$, then $\scU=\emptyset$, so $A$ failing $\ref{def:FilterBase:IsNotEmpty}$ implies $\scU$ fails \ref{def:Filter:IsNonempty}. 
        If $\emptyset \in A$, then $\emptyset \in \scU$, so $A$ failing $\ref{def:FilterBase:DoesntContainEmptySet}$ implies $\scU$ fails \ref{def:Filter:DoesntContainEmpty}. 
        Finally, if $\scA$ fails \ref{def:FilterBase:IntersectionProperty}, then 
        we can find $B,C \in A$ such that $B \cap C \not \in \scU$, implying $\scU$ fails $\ref{def:Filter:FiniteIntersectionProperty}$. 
        Hence necessity has been proven. 
    \end{proof}
\end{prop}
\begin{rmk}
\label{rmk:FilterBase}\rm
    If $A$ is a \FilterBase on $X$, then \scU defined in $\ref{prop:FilterBase}$ 
    is the \Filter \FilterGeneratedBy $A$. 
\end{rmk}
