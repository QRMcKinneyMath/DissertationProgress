\begin{prop}
\label{prop:FilterFacts}
\rm
    Let $X$ be a nonempty set.
    Let $\scF$ be a 
    \Filter on $X$. 
    Let $B \subset X$. 
    The following are true. 
    \begin{enumerate}[label=(\roman*), ref={\ref{prop:FilterFacts}~\roman*}]
        \item 
		\label{prop:FilterFacts:ContainsX} 
		$X \in \scF$. 
        \item 
		\label{prop:FilterFacts:ClosureUnderFiniteIntersections} 
		$\scF$ is closed under finite intersections.
        \item 
		\label{prop:FilterFacts:IntersectionOfFiltersIsAFilter} 
		The intersection of a collection of 
		\Filters on $X$ is a \Filter on $X$. 
        \item 
		\label{prop:FilterFacts:InducedFilterExistence}
        $\scF_B$ defined by $\scF_B = \braces{U \cap B | U \in \scF}$ satisfies 
        \ref{def:Filter:IsNonempty}, 
        \ref{def:Filter:SubsetProperty}, 
        and 
        \ref{def:Filter:FiniteIntersectionProperty}, 
        and is therefore a 
        \Filter on $B$ if and only if $\emptyset \not \in \scF_B$. 
    \end{enumerate}
    \begin{proof}[Proof of \ref{prop:FilterFacts:ContainsX}]
        By \ref{def:Filter:IsNonempty}, 
        $\scF$ contains a nonempty set $B$. 
        Since $B \subset X \subset X$, by 
        \ref{def:Filter:SubsetProperty}, 
        $X \in \scF$, so \ref{prop:FilterFacts:ContainsX} is proven. 
    \end{proof}
    \begin{proof}[Proof of \ref{prop:FilterFacts:ClosureUnderFiniteIntersections}]
	
        This result is a direct application of \ref{def:Filter:FiniteIntersectionProperty} paired with 
        \ref{prop:FiniteClosure:Intersection}.
    \end{proof}
    \begin{proof}[Proof of \ref{prop:FilterFacts:IntersectionOfFiltersIsAFilter}]
    	
        Let $\{\scF_\alpha\}_{\alpha \in A}$ be a collection of \Filters on $X$. 
        Define $\scF=\bigcap\limits_{\alpha \in A} \scF_\alpha$. 
        By \ref{prop:FilterFacts:ContainsX}, for each $\alpha \in A$, 
        $X \in \scF_\alpha$, so $X \in \scF$.
        Hence $\scF$ satisfies $\ref{def:Filter:IsNonempty}$.
        Furthermore, by $\ref{def:Filter:DoesntContainEmpty}$, for each 
        $\alpha \in A$, $\emptyset \not \in \scF_\alpha$, so 
        $\emptyset \not \in \scF$. Therefore $\scF$ satisfies \ref{def:Filter:DoesntContainEmpty}.
        Let $G_1 \in \scF$ and let $G_1 \subset G_2 \subset X$. 
        Then, for each $\alpha \in A$, $G_1 \in \scF_\alpha$, so 
        $G_2 \in \scF_\alpha$ for each $\alpha \in A$, so $G_2 \in \scF$. 
        Thus, $\scF$ satisfies \ref{def:Filter:SubsetProperty}. 
        Finally, let $\{G_1,G_2\} \subset \scF$. 
        Then for each $\alpha \in A$, $\{G_1,G_2\} \subset \scF_\alpha$, 
        implying by $\ref{def:Filter:FiniteIntersectionProperty}$ that 
        $G_1 \cap G_2 \in \scF_\alpha$, so $G_1\cap G_2 \in \scF$, 
        implying $\scF$ satisfies $\ref{def:Filter:FiniteIntersectionProperty}$. 
    \end{proof}
    \begin{proof}[Proof of \ref{prop:FilterFacts:InducedFilterExistence}]
       By 
       \ref{prop:FilterFacts:ContainsX}, 
       $X \in \scF$, so $B = X \cap B \in \scF_B$.
       Hence $\scF_B$ satisfies 
       \ref{def:Filter:IsNonempty}.
       If $G_1 \in \scF_B$, then there is an $H_1 \in \scF$ with $G_1 = H_1 \cap B$. 
       If $G_1 \subset G_2 \subset B$, 
       then $H_1 \cap B \subset G_2$ 
       and $H_1 \cap (X \setminus B) \subset (X \setminus B)$, 
       so $H_1 \subset G_2 \cup (X \setminus B)$, 
       which implies $G_2 \cup (X \setminus B) \in \scF$ by \ref{def:Filter:SubsetProperty}.
       By construction, then, $G_2 = \pa{G_2 \cup (X \setminus B) } \cap B \in \scF_B$.
       Hence $\scF_B$ satisfies $\ref{def:Filter:SubsetProperty}$. 
       Next, if $\{G_1, G_2 \} \subset \scF_B$, then there are $H_1,H_2 \in \scF$ 
       with $G_i = H_i \cap B$. 
       Since $\scF$ satisfies $\ref{def:Filter:FiniteIntersectionProperty}$, 
       $H_1 \cap H_2 \in \scF$. 
       This implies by construction that $B \cap \pa{H_1 \cap H_2} \in \scF_B$, but 
       $B \cap \pa{H_1 \cap H_2} = G_1 \cap G_2 $, so $\scF_B$ satisfies 
       \ref{def:Filter:FiniteIntersectionProperty}. 
    \end{proof}
\end{prop}

