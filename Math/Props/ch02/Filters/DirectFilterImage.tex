\begin{prop}[Direct Filter Image]
\label{prop:DirectFilterImage}
Suppose the following. 
\begin{enumerate}
    \item $X,Y \neq \emptyset$
    \item $f:X \to Y$ is \Surjective.
    \item For $i \in \{1,2\}$, $\scB_i$ is a \FilterBase for a \Filter $\scF_i$ on $X$. 
    \item $\scF_2$ is \FinerFilter than $\scF_1$. 
    \item $\scK$ is an \UltrafilterBase on $X$. 
\end{enumerate}
Then the following are true.
\begin{enumerate}[label=(\roman*), ref={\ref{prop:DirectFilterImage}.~\roman*}]
\item \label{prop:DirectFilterImage:Filter} $f(\scF_1)$ is a \Filter on $Y$. 
\item \label{prop:DirectFilterImage:Base} $f(\scB_1)$ is a \FilterBase for $f(\scF_1)$. 
\item \label{prop:DirectFilterImage:Order} $f(\scF_2)$ is a \FinerFilter than $f(\scF_1)$. 
\item \label{prop:DirectFilterImage:Ultrafilter} $f(\scK)$ is an \UltrafilterBase on $Y$. 
\end{enumerate}
\begin{proof}[Proof of \ref{prop:DirectFilterImage:Filter}]
    Since $\emptyset \not \in \scF_1$, 
    $\emptyset \not \in f(\scF_1)$, so 
    $f(\scF_1)$ satisfies \ref{def:Filter:DoesntContainEmpty}.
    Since $\emptyset \neq \scF_1$, $f(\scF_1) \neq \emptyset$, 
    so $f(\scF_1)$ satisfies $\ref{def:Filter:IsNonempty}$.
    Let $G_1 \in f(\scF_1)$. Let $G_1 \subset G_2 \subset Y$. 
    Then, since $f$ is \Surjective, $U \subset f^{-1}(G_2)$, 
    which by 
    \ref{def:Filter:SubsetProperty} implies $f^{-1}(G_2) \in \scF_1$. 
    Hence $G_2=f\pa{f^{-1}\pa{G_2}} \in f(\scF_1)$, so 
    $f(\scF_1)$ satisfies \ref{def:Filter:SubsetProperty}. 
    Finally, if $G_1,G_2 \in f(\scF_1)$, then 
    there are $K_1,K_2 \in \scF_1$ with $f(K_i) =G_i$ for $i \in \{1,2\}$. 
    By \ref{def:Filter:FiniteIntersectionProperty}, $K_1 \cap K_2 \in \scF_1$. 
    Also, $f(K_1 \cap K_2) \subset f(K_1) \cap f(K_2)$, 
    so by \ref{def:Filter:SubsetProperty}, $f(K_1) \cap f(K_2) \in f(\scF_1)$.
    Hence $f(\scF_1)$ satisfies $\ref{def:Filter:FiniteIntersectionProperty}$
    and is therefore a \Filter on $Y$. 
\end{proof}
\begin{proof}[Proof of \ref{prop:DirectFilterImage:Base}]
   By \ref{def:FilterBase:IsNotEmpty}, $\emptyset \neq \scB_1$, 
   so $\emptyset \neq f(\scB_1)$, and thus $f(\scB_1)$ satisfies \ref{def:FilterBase:IsNotEmpty}. 
   By \ref{def:FilterBase:DoesntContainEmptySet}, $\emptyset \not \in \scB_1$, so
   $\emptyset \not \in f(\scB_1)$, implying $\scB_1$ satisfies \ref{def:FilterBase:DoesntContainEmptySet}. 
   Finally, let $U_1,U_2 \in f(\scB_1)$.
   Then there exists $V_i \in \scB_1$ with $f(V_i) = U_i$. 
   Then by \ref{def:FilterBase:IntersectionProperty}, $V_1 \cap V_2 \in \scB_1$, 
   and $f(V_1 \cap V_2) \subset f(V_1) \cap f(V_2)$, and $f(V_1 \cap V_2) \in f(\scB_1)$
   by construction, so $f(\scB_1)$ satisfies \ref{def:FilterBase:IntersectionProperty}, 
   and therefore $f(\scB_1)$ is a \FilterBase on $Y$. 
   Now, if $V \in f(\scF_1)$, then by definition, there exists 
   $U \in \scF_1$ with $f(U) = V$. 
   By \ref{prop:FilterBaseFacts:BaseCondition}, there 
   exists a $b \in \scB_1$ with $b \subset U$. This implies $f(b) \subset f(U)=V$, but 
   $f(b) \in f(\scB_1)$.
   Furthermore, since $\scB_1 \subset \scF_1$, 
   $f(\scB_1)\subset f(\scF_1)$, so 
   we can apply \ref{prop:FilterBaseFacts:BaseCondition} to claim 
   that $f(\scB_1)$ is a \FilterBase for $f(\scF_1)$. 
\end{proof}
\begin{proof}[Proof of \ref{prop:DirectFilterImage:Order}]
    If $\scF_1 \subset \scF_2$ then $f(\scF_1) \subset f(\scF_2)$. 
    An invocation of \ref{prop:DirectFilterImage:Base} finishes the result. 
\end{proof}
\begin{proof}[Proof of \ref{prop:DirectFilterImage:Ultrafilter}]
   Let $\scG$ denote the \Ultrafilter for which 
   $\scK$ is an \UltrafilterBase.
   Let $U \subset Y$.
   Since $f$ is \Surjective, 
   \begin{equation}
   X= f^{-1}(U) \cup \pa{X \setminus f^{-1}(U)} = f^{-1}(U) \cup f^{-1} \pa{f(X) \setminus U}= f^{-1}(U) \cup f^{-1}(Y \setminus U)
   \end{equation}
   and by \ref{prop:FilterFacts:ContainsX}, we have 
    $f^{-1}(U) \cup f^{-1}(Y \setminus U) \in \scG$
    Since \scG is an \Ultrafilter, by \ref{def:UltrafilterFacts:BinaryUnion}, 
    either $f^{-1}(U) \in \scG$ or $f^{-1}(Y \setminus U) \in \scG$. 
    This implies either 
    $U \in f( \scG)$ or $Y \setminus U \in f(\scG)$. 
    By \ref{prop:UltrafilterFacts:UltrafilterCondition}, 
    $f(\scG)$ is an \Ultrafilter on $Y$. 
    An application of \ref{prop:DirectFilterImage:Base} completes the result.
\end{proof}
\end{prop}
