\begin{prop}[FilterBaseFacts]
\label{prop:FilterBaseFacts}
   Let $X \neq \emptyset$. 
   Let $\scF$ and $\scG$ be \Filters on $X$. 
   Let $F$ be a \FilterBase for $\scF$ and let 
   $G$ be a \FilterBase for $\scG$. 
   The following are true. 
   \begin{enumerate}[label=(\roman*), ref={\ref{prop:FilterBaseFacts}.~\roman*}]
    \item \label{prop:FilterBaseFacts:BaseFromSubbasis}
        The collection of \Finite intersections of a 
        \FilterSubbasis $A$ for $\scF$ 
        forms a \FilterBase for $\scF$. 
    \item \label{prop:FilterBaseFacts:BaseCondition}
    $B \subset \scF$ is a \FilterBase for $\scF$ 
    if and only if for each $Y \in \scF$, there exists
    $b \in B$ with $b \subset Y$. 
    \item \label{prop:FilterBaseFacts:FinerCondition}
       $\scF$ 
       is \FinerFilter than $\scG$ if and only if
       for each $g \in G$, there is an $f \in F$ with $f \subset g$. 
    \item  \label{prop:FilterBaseFact:EquivalenceCondition} 
    $F$ is 
    \FilterBaseEquivalent to 
    $G$ if and only if for each 
    $f \in F$ there  is a 
    $g \in G$ with $g \subset f$ 
    and for each $g \in G$ there is an 
    $f \in F$ with $f \subset g$. 
   \end{enumerate}
   \begin{proof}[Proof of \ref{prop:FilterBaseFacts:BaseFromSubbasis}]
   Let 
   \begin{equation}
    \scB = \left\{ \bigcap\limits_{i=1}^n A_i | \{A_i\}_{i=1}^n \subset A \wedge n \in \bbN\right\}
   \end{equation}
   By \ref{def:FilterSubbase:FiniteIntersectionsNonempty}, $\emptyset \not \in \scB$, so $\scB$ satisfies \ref{def:FilterBase:DoesntContainEmptySet}. 
   By \ref{def:FilterSubbase:IsNotEmpty}, $\emptyset \neq A \subset \scB$, so $\scB$ satisfies \ref{def:FilterBase:IsNotEmpty}. 
   Since $\emptyset \not \in \scB$ is \ClosedUnderFiniteIntersections, if $U,V \in \scB$, then $\emptyset \neq U \cap V \in \scB$, so $\scB$ can be seen to satisfy $\ref{def:FilterBase:IntersectionProperty}$. Hence $\scB$ is a \FilterBase. 
   Since $A \subset \scB$, $\scB$ is a \FilterBase for a \FinerFilter than $\scF$. 
   However, since $A \subset \scF$, 
   by \ref{prop:FilterFacts:ClosureUnderFiniteIntersections}, $\scB \subset \scF$. 
   Hence $\scF$ is the \Filter
   \FilterGeneratedBy $\scB$. 
   \end{proof}
   \begin{proof}[Proof of \ref{prop:FilterBaseFacts:BaseCondition}]
    $(\impliedby)$. Let $\scG$ denote the \Filter \FilterGeneratedBy $B$. 
    Then since $B \subset \scF$, $\scG \subset \scF$. 
    If for each $Y \in \scF$ there exists $b \in B$ with $b \subset Y$, then 
    \begin{equation}
    \scF \subset \{U \subset X | (\exists b \in B) (b \subset U) \} \subset \scG \subset \scF
    \end{equation}
    so that $\scF = \scG$ and 
    $\{U \subset X | (\exists b \in B) (b \subset U)\}= \scF$ is a \Filter on $X$. 
    Hence, by  \ref{prop:FilterBase}, $B$ is a \Filter. 

    $(\implies)$. If $B$ is a \FilterBase for $\scF$, then by 
    \ref{prop:FilterBase}, $\scF= \{Y \subset X | (\exists b \in B) ( b \subset Y) \}$
    so the desired property holds
   \end{proof}
   \begin{proof}[Proof of \ref{prop:FilterBaseFacts:FinerCondition}]
   Let $\scF$ be finer than $\scG$. Then 
   by applying \ref{prop:FilterBase}
   $\scG \subset \scF = \{ U \subset X | (\exists f \in F) (f \subset U) \}$, which is the desired result in one direction. 
   The other direction is equivalent again applying \ref{prop:FilterBase}
   to claim
   $\scG \subset \{U \subset X | (\exists f \in F) ( f \subset U ) \}$.
   \end{proof}
   \begin{proof}[Proof of \ref{prop:FilterBaseFact:EquivalenceCondition}]
   This is a result of two applications of \ref{prop:FilterBaseFacts:FinerCondition}, 
   one in each direction. 
   \end{proof}
\end{prop}
