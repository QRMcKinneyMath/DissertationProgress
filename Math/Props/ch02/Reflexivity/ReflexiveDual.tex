\begin{prop}[\Reflexive Dual]
\label{prop:ReflexiveDual}
\rm
Let $X$ be a \Complete \SeminormedSpace over $\F$.
Then $X$ is \Reflexive if and only if $X^*$ is \Reflexive. 
\begin{proof}
%Let $X$ be \Reflexive. 
%By \ref{}, 
%$\overline{B_X(0;1)}$ is \weakly \SetCompact.

Let $X$ be \Reflexive. 
Let $c_*$ denote the \CanonicalEmbedding of $X^*$. 
Let $c$ denote the \CanonicalEmbedding of $X$. 
Let $\tilde{j} \in X^{***}$. 
Define $j:X \to \F$ by 
$\ip{x,j} = \ip{c(x), \tilde{j}}$. 
Let $\alpha \in \F$ and $x,y \in X$. 
Then 
\begin{equation*}
\ip{\alpha x + y, j} = \ip{c\pa{\alpha x + y}, \tilde{j}} = \alpha \ip{c(x), \tilde{j}} + \ip{c(y), \tilde{j}} = \alpha \ip{x, j} + \ip{y, j}
\end{equation*}
so that $j$ is \Linear. 
$j$ is also \ContinuousFunction, as 
\begin{equation*}
\abs{\ip{x, j}} = \abs{\ip{c(x), \tilde{j}} }\leq \norm{c(x)} \norm{\tilde{j}} = \norm{x} \norm{\tilde{j}}
\end{equation*}
Furthermore, if $g^{*} \in X^{**}$, then 
there exists $g \in X$ such that $c(g)=g^*$ so that 
\begin{equation*}
\ip{g^*, \tilde{j}} = \ip{c(g), \tilde{j}} = \ip{g, j} = \ip{j, c(g)} = \ip{j, g^*} = \ip{g^*, c_*\pa{j}}
\end{equation*}
Since $g^* \in X^*$ was arbitrary, $\tilde{j}=c_*\pa{j}$.
Since $\tilde{j} \in X^{***}$ was arbitrary, $c_*$ is \Surjective and
$X^*$ is therefore \Reflexive. 

Suppose now that $X^*$ is \Reflexive. 
Then by the above implication, $X^{**}$ is \Reflexive. 
By 
\ref{lem:ReflexiveSeparable}, 
each \SetClosed \TopologySeparable subspace of $X^{**}$ is \Reflexive. 
Let $K \subset X$ be a \SetClosed \TopologySeparable subspace of $X$. 
Then $c(K)$ is a \SetClosed \TopologySeparable subspace of $X^*$. 
This implies $c(K)$ is \Reflexive, which implies 
$K$ is \Reflexive.
Hence, by \ref{lem:ReflexiveSeparable}, 
$X$ is \Reflexive.
\end{proof}
\end{prop}