\begin{prop}[Equivalence Classes Partition]
    \label{prop:EquivalenceClassesPartition}
    
    Let $X \neq \emptyset$. 
    Let $\cong$ be an 
	\EquivalenceRelation
	defined on X. 
    Let $x,y \in X$. 
    The following statements are equivalent. 
    \begin{enumerate}
        \item $[x]_{\cong}  \cap [y]_{\cong} \neq \emptyset$
        \item $x \cong y$
        \item $[x]_{\cong} = [y]_{\cong}$
        \item $[x]_{\cong} \subset [y]_{\cong}$
        \item $[y]_{\cong} \subset [x]_{\cong}$ 
    \end{enumerate}
    
\begin{proof}[Proof That $1 \implies 2$]
Suppose $M:=[x]_{\cong} \cap [y]_{\cong} \neq \emptyset$. 
Then there exists $z \in M$.
Then $z \cong x$, so by \RelationSymmetry, $x \cong z$. 
But by \RelationTransitivity, pair with $z \cong y$, we conclude $x \cong y$. 
\end{proof}
\begin{proof}[Proof That $2 \implies 4$]
    Let $x \cong y$ and let $z \in [x]_{\cong}$. 
    Then $z \cong x \cong y$, so $z \cong y$ and $z \in [y]_{\cong}$.
    Since z was arbitrary, we're done. 
\end{proof}
\begin{proof}[Proof That $2 \implies 5$]
    Let $x \cong y$. By 
	\RelationSymmetry, $y \cong x$, so by $(2 \implies 4)$, we are done. 
\end{proof}
\begin{proof}[Proof That $2 \implies 3$]
    Since  $2 \implies 4$ and $2 \implies 5$ and 5 and 4 together imply 3, we have this. 
\end{proof}
\begin{proof}[Proof That $5 \implies 1$]
    Let $[y]_{\cong} \subset [x]_{\cong}$. 
    Then $y \in [y]_{\cong} = [y]_{\cong} \cap [x]_{\cong} $.
    Hence 1 holds. 
\end{proof}

\end{prop} 
