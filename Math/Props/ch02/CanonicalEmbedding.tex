\begin{prop}[Canonical Embedding]
\label{prop:canonicalembedding}
    Let $X$ be a \SeminormedSpace 
    and let $c$ denote its 
    \CanonicalEmbedding. 
    The following are true. 
    \begin{enumerate}
        \item c is well defined
        \item c is Linear. 
        \item c is an isometry. 
        \item c is an injection if and only if X is a \NormedSpace. 
        \item If $q:X \to X/\Ker$ is the \QuotientMap, 
		$c_{X/\Ker}$ is the \CanonicalEmbedding of $(X/\Ker)$ into $(X/\Ker)^{**}$ 
		and $\Omega_2:X^{**} \to (X/\Ker)^{**}$ 
		is the linear bijective isometry defined in 
		\ref{def:higherorderdualspaceisomorphism}, 
		then $c = \Omega_2^{-1} \circ c_{X/\Ker} \circ q$. 
		//TODO: COME BACK TO THIS AND PROVE IT ONCE THE ISOS ARE CLEARED UP
        \item $c_X$ is surjective if and only if 
        $c_{X/\Ker}$ is surjective. 
        \item X is \Reflexive if and only if $X/\Ker$ is \Reflexive.
    \end{enumerate}
    \begin{proof}[Proof of 1]
		For any $x \in X$, 
		$c(x)$ as a function is obviously well defined. 
        Hence, I just need to show that, 
		for any
        $x \in X$, 
		$c(x) \in X^{**}$.
		That is
		, I must show that
		$c(x)$ is
        continuous and
        linear. 
        
		For linearity, if 
		$x^*, y^* \in X^*$ and 
		$\alpha \in \F$, 
		we have 
		\begin{align*}
		\ip{\alpha x^*+ y^* , c(x) } & = \ip{x, \alpha x^*+\alpha y^* }\\
		& = \alpha \ip{x, x^*}, \ip{y, y^*}\\
		& = \alpha \ip{x^*, c(x)} + \ip{y^*, c(x)}
		\end{align*}
		Thus linearity holds.
		
		For continuity, let $x \in X$
		and let $x^* \in X^*$. 
		\begin{align*}
		\abs{\ip{x^*, c(x)}} &= \abs{\ip{x, x^*}}\\
		& \leq \norm{x}\norm{x^*}
		\end{align*}
		so that $c(x)$ is bounded with 
		$\norm{c(x)} \leq \norm{x}$. 
    \end{proof}
    \begin{proof}[Proof of 2]
        Let $\alpha \in \F$ and $x,y \in X$.
        Let $x^* \in X$. 
		Then, 
		\begin{align*}
		\ip{x^*, c(\alpha x + y)} & = \ip{\alpha x + y , x^*}\\
		& = \alpha \ip{x,x^*} + \ip{y,x^*} \\
		& = \alpha \ip{x^*, c(x)} + \ip{x^*, c(y)} 
		\end{align*}
		, finishing the proof. 
    \end{proof}
    \begin{proof}[Proof of 3]
	Let $x_0 \in X$ and $x^* \in X^*$. 
	Then, 
		\begin{align*}
		\abs{\ip{x^*, c(x_0)}} &= \abs{\ip{x_0, x^*}}\\
		& \leq \norm{x_0}\norm{x^*}
		\end{align*}
	so that $\norm{c(x_0)} \leq \norm(x_0)$. 
	For the other direciton, by
	\ref{thm:hahnbanach}
	part 2, there exists an 
	$x_0^* \in X^*$ satisfying
	$\norm{x_0^*} = 1$ 
	and $\ip{x_0, x_0^*} = \norm{x}$
    \end{proof}
	We see that 
	$ \ip{x_0^*, c(x_0) } = \ip{x_0, x_0^*} = \norm{x} = \norm{x_0} \norm{x_0^*}$
	so that $\norm{c(x_0)} \geq \norm{x_0}$. 
	Since the inequality goes both ways, 
	$\norm{x_0} = \norm{c(x_0)}$, and 
	c is therefore an isometry. 
    \begin{proof}[Proof of 4]
	Let X be a \NormedSpace. 
	Then $X^*$ separates points in X. 
	Let $x \in X$ 
	and $y \in X$ 
	with $x \neq y$. 
	Since $X^*$ separates points in X, 
	there exists $x^* \in X^*$ with 
	$\ip{x^*, c(x) } =\ip{x,x^*} \neq \ip{y,x^*}, \ip{x^*, c(y)}$
	so that $c(x) \neq c(y)$.
	Hence $c$ is injective. 
	
	Now supose instead that $c$ is injective
	and let $x,y \in X$ with $\norm{x-y} = 0$.
	We find that for any $x^* \in X^*$, 
	\begin{align*}
	\abs{\ip{x^*, c(x)-c(y)}} & = \abs{\ip{x^*, c(x-y)}} \\
	& = \abs{\ip{x-y, x^*}}\\
	& =	\leq \norm{x^*} \norm{x-y} \\
	= 0
	\end{align*}
	so that $\norm{c(x)-c(y)} = 0$.
	Since $X^{**}$ is a normed space, 
	this implies $c(x) = c(y)$, which through injectivity 
	implies $x=y$. 
	Hence we have the implication $\norm{x-y} = 0 \implies x=y$, so
	that X is a $\NormedSpace$. 
    \end{proof}
    \begin{proof}[Proof of 5]
	Proceeding directly from the definition, we have 
	\begin{align*}
	\ip{x^*, \Omega_2^{-1} \circ c_{X/\Ker} \circ q (x) } & = \ip{x^*, \Omega^{\times} \circ c_{X/\Ker} \circ q(x)}\\
	& = \ip{\Omega x^*, c_{X/\Ker} \circ q(x) } \\
	& = \ip{q(x), \Omega x^*} \\
	& = \ip{x, x^* }\\
	& = \ip{x^*, c(x)}
	\end{align*}
	So we are done.
    \end{proof}
	\begin{proof}[Proof of 6]
	Since $\Omega_2$ is a Bijection, 
	by the prior part of this result, 
	c is a surjection if and only if
	$c_{X/\Ker} \circ q$ is a surjection
	where $q:X \to X/\Ker$ is the \QuotientMap. 
	Since q is a surjection, 
	c is a surjection 
	if and only if$c_{X/\Ker}$ 
	is a surjection. 
	\end{proof} 
	\begin{proof}[Proof of 7]
	This is a direct restatement of Part 06 of this result. 
	\end{proof} 
\end{prop}

