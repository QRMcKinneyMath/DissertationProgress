\begin{prop}[\VerySmooth dual implies \Reflexive]
\label{prop:VerySmoothDualImpliesReflexive}
\rm
%(Diestel Theorem 1 p.32)
Let $X$ be a \Complete \SeminormedSpace. 
Let $X^*$ be \VerySmooth. 
Then $X$ is \Reflexive.
\begin{proof}
Let $X^*$ be \VerySmooth.
Let $J_{*}$ denote the \NormalizedDualityMap of $X^*$. 
Then $J_{*}$ is \Norm to \weak \ContinuousFunction. 
Let $j \in \partial B_{X^*}(0;1)$. 
By \ref{thm:BishopPhelps}, there exists 
$\{j_i\}_{i \in \N} \subset \partial B_{X^*}(0;1)$ 
and $\{x_i\}_{i \in \N} \subset \partial B_X(0;1)$ such that 
$j_i \to j$ and $\ip{x_i, j_i} = 1$ for each $i \in \N$.
Then for each $i \in \N$, if $c:X \to X^{**}$ is the \CanonicalEmbedding, 
it is clear that $c(x_i) = J_*(j_i)$. 
Since $j_i \to j$, we conclude $c(x_i) \wto J_*(j) \in X^{**}$. 
Since $X$ is \Complete and $c$ is an \Isometry, we conclude $c(X)$ is \Complete
and therefore \SetClosed in $X^{**}$. since $c(X)$ is \ConvexSet in $X^{**}$, 
$c(X)$ is \weakly \SetClosed in $X^*$. 
Hence $J_*(j) \in c(X)$. 
Thus there exists $x \in X$ such that $c(x) = J_*(j)$. 
It is clear that $j$ attains its \Norm at $x$.
Since $j \in \partial B_X(0;1)$, 
by \ref{thm:James}, we conclude that $X$ is \Reflexive.
\end{proof}
\end{prop} 

\begin{prop}[\Reflexivity Conditions]
\label{prop:ReflexivityConditions}
\rm
Let $X$ be a \Complete \SeminormedSpace.
Any of the following conditions are sufficient to guarantee that $X$ is \Reflexive.
\begin{enumerate}[label=(\roman*), ref={\ref{prop:ReflexivityConditions}~\roman*}]
\item
\label{prop:ReflexivityConditions:Zero}
$X$ is \UniformlyConvex.
\item 
\label{prop:ReflexivityConditions:Single}
$X^*$ is \LocallyUniformlySmoothSpace.
\item 
\label{prop:ReflexivityConditions:Double}
$X^{**}$ is \LocallyWeaklyUniformlyConvex.
\item 
\label{prop:ReflexivityConditions:Triple}
$X^{***}$ is \SmoothSpace.
\item 
\label{prop:ReflexivityConditions:Quadruple}
$X^{****}$ is \StrictlyConvexSpace.
\end{enumerate}
\begin{proof}[Proof of \ref{prop:ReflexivityConditions:Zero}]
This is a restatement of \ref{thm:MilmanPettis}.
\end{proof}
\begin{proof}[Proof of \ref{prop:ReflexivityConditions:Single}]
Let $X^*$ be 
\LocallyUniformlySmoothSpace. 
By \ref{prop:DegreesOfSmoothness:LocalImpliesVery}, 
$X^*$ is \VerySmooth.
By \ref{prop:VerySmoothDualImpliesReflexive}, $X$ is \Reflexive. 
\end{proof}
\begin{proof}[Proof of \ref{prop:ReflexivityConditions:Double}]
Let $X^{**}$ be 
\LocallyWeaklyUniformlyConvex. 
By \ref{prop:VerySmoothWeakLocalConvexity}, 
$X^*$ is \VerySmooth.
By \ref{prop:VerySmoothDualImpliesReflexive}, $X$ is \Reflexive. 
\end{proof}
\begin{proof}[Proof of \ref{prop:ReflexivityConditions:Triple}]
I prove by converse. 
Suppose $X$ is not \Reflexive. 
By \ref{thm:James}, 
there exists $j \in \partial B_{X^*}(0;1)$ such that 
for each $x \in \partial B_X(0;1)$, $abs{\ip{x, j}} < 1$. 
By 
\ref{thm:BishopPhelps}, there is a sequence $\{j_i\}_{i \in \N} \subset \partial B_{X^*}(0;1)$ 
and a sequence $\{x_i\}_{i \in \N} \subset \partial B_X(0;1)$ such that 
$j_i \to j$ and $\ip{x_i,j_i} = 1 = \norm{j_i}\norm{x_i}$ for each $i \in \N$. 
Note that $\{x_i\}$ is not \weakly convergent. If this were the case, say $x_i \to \tilde{x}$, then we would have $\norm{\tilde{x}} \leq 1$ and 
\begin{align*}
\abs{\ip{\tilde{x},j}-1} & = \abs{\ip{\tilde{x}, j} - \ip{x_i, j_i}} \\
&= \abs{\ip{\tilde{x}, j} - \ip{x_i,j}+ \ip{x_i,j} - \ip{x_i,j_i}}\\
& \leq  \abs{\ip{\tilde{x} - x, j}} + \norm{x_i} \norm{j-j_i} \to 0
\end{align*}
which violates the fact that $j$ does not attain its supremum.
Let $c:X \to X^{**}$ denote the 
\CanonicalEmbedding. 
Since $X$ is \Complete and $c$ an \Isometry, 
$c(X)$ is \Complete and therefore \SetClosed in $X^{**}$. 
Since $c(X)$ is \ConvexSet, by 
\ref{prop:ClosedAndConvexIsWeaklyClosed}, 
$c(X)$ is \weakly \SetClosed in $X^{**}$. 
Since the \weak \Topology on $c(X)$ is \TopologyFiner 
than the \weak \Topology on $X$, 
we conclude $\{c(x_i)\}_{i \in \N}$ is not \weakly convergent, 
because if it were, this \weak limit would be an element of $c(X)$, say 
$c(y)$ but we would have $x_i \wto y$, a contradiction.
Let $J_{X^*}$ denote the \NormalizedDualityMap of $X^*$. 
Let $\tilde{j} \in J_{X^*}j$. 
Then since $\{c(x_i)\}_{i \in \N}$ has no \weak limit, 
$c(x_i) \not \wto  \tilde{j}$. 
Let $c_{X^{**}}:X^{**} \to X^{****}$ denote the \CanonicalEmbedding. 
Then since the \weakstar \Topology on $c_{X^{**}}(X^{**}) \subset X^{****}$ is 
equivalent to the \weak \Topology on $X^{**}$, we conclude 
$c_{X^{**}}\circ c(x_i) \not \wsto c_{X^{**}} \tilde{j}$.
Let $c_{X^*}:X^* \to X^{***}$ denote the \CanonicalEmbedding.
Then since $c_{X^*}$ is an \Isometry and 
$j_i \to j$, we conclude $c_{X^*}(j_i) \to c_{X^*}(j)$
Let $J_{X^{***}}$ denote the \NormalizedDualityMap of $X^{***}$. 
It is clear that $c_{X^{**}}(\tilde{j}) \in J_{X^{***}}c_{X^*}(j)$ and that
for each $i \in \N$, $c_{X^{**}} \circ c (x_i) \in J_{X^{***}}C_{X^*} j_i$.
Thus, since $c_{X^{**}} \circ c(x_i) \not \wsto c_{X^{**}}(\tilde{j})$, but 
$c_{X^*}\pa{j_i} \to c_{X^*}(j)$, we conclude that $J_{X^{***}}$ has a selection which is not 
\Norm to \weakstar continuous at $c_{X^*}(j)$. 
Thus by 
\ref{prop:Smooth:JContinuous}, $X^{***}$ is not \SmoothSpace.
\end{proof}
\begin{proof}[Proof of \ref{prop:ReflexivityConditions:Quadruple}]
Let $X^{****}$ by \StrictlyConvexSpace. 
By \ref{prop:StrictConvexDualImpliesSmooth}, $X^{***}$ is 
\SmoothSpace. 
By \ref{prop:ReflexivityConditions:Triple}, 
$X$ is \Reflexive. 
\end{proof}
\end{prop} 