The following result, which is well known, serves to provide an example of how 
the results relating the dual space of a \SeminormedSpace 
with that  of its quotient \NormedSpace can be used.  
\begin{thm}[Hahn Banach Theorem For Seminormed Spaces]
\label{thm:hahnbanach}
\rm
Let $(X,\norm{\cdot})$ be a \SeminormedSpace.
let $x_i \in X$ for $i \in \{0,1\}$ such that 
$x_0 \not \in x_1 + \overline{0}$.
Let $X^*$ denote the
\TopDualSpace of $X$. 
The following are true. 
\begin{enumerate}[label=(\roman*), ref={\ref{thm:hahnbanach}~\roman*}]
\item 
\label{thm:HahnBanach:Extension1}
If $Z \subset X$ is a \VectorSubspace of $X$
and $z^* \in Z^*$, then there 
is an \Extension. $x^*$ of $z^*$, 
$x^* \in X^*$ such that 
$\norm{z^*}_{Z^*} = \norm{x^*}_{X^*}$
\item 
\label{thm:HahnBanach:Point1}
If $x \in X$, 
with $\norm{x} \neq 0$, 
then there exists an
$x^* \in X$ with 
$\norm{x^*}=1$ and 
$\ip{x,x^*} = \norm{x}_X$. 
\item 
\label{thm:HahnBanach:Norm}
If $x \in X$, then 
\begin{equation*}
\norm{x}_X = \sup\limits_{0 \neq x^* \in X^*} \frac{\ip{x,x^*}}{\norm{x^*}}
\end{equation*}

\item 
\label{thm:HahnBanach:Operator1}
If $Y$ is a 
\NonDegenerate
\SeminormedSpace, and if 
$x_0 \in X$, with 
$\norm{x_0} \neq 0$, 
then there exists
an $S \in BL(X,Y)$ with 
$\norm{S} = 1$ and 
\begin{equation*}
\norm{Sx_0} = \norm{x_0}
\end{equation*}
\item 
\label{thm:HahnBanach:Open}
If $U \subset X$ is \SetOpen  and \ConvexSet, 
$C$ is \ConvexSet, and $U \cap C = \emptyset$, 
then there exists $j \in X^*$ and $\alpha \in \R$ such that 
for each $x_0 \in A$ and $y_0 \in C$, we have 
\begin{equation*}
Re \ip{x_0, j} < \alpha \leq \ip{y_0, j}
\end{equation*}
\item
\label{thm:HahnBanach:ClosedConvex}
If $C \subset X$ is \SetClosed and \ConvexSet, 
$K \subset X$ is \SetCompact and \ConvexSet, 
and $C \cap K = \emptyset$,
then there exists $j \in X^*$ and $\alpha_1 , \alpha_2 \in \mathbb{R}$ such that 
\begin{equation*}
\sup\limits_{x \in C} Re \ip{x, j} \leq \alpha_1 < \alpha_2 \leq \inf\limits_{x \in K} Re \ip{x, j}
\end{equation*}
\item 
\label{thm:HahnBanach:Nullspace}
If $C \subset X$ is a \SetClosed \VectorSubspace
and $x_0 \in X \setminus C$, then there exists
$j \in X^*$ such that $\ip{C,j} = 0$ and $\ip{x_0,j} = 1$. 
\item
\label{thm:HahnBanach:NullspaceOperator}
If $Y$ is a \NonDegenerate \SeminormedSpace,
$C \subset X$ is a \SetClosed \VectorSubspace of $X$, 
and $x_0 \in X \setminus C$, then there 
exists 
$S \in BL(X,Y)$ with $\norm{Sx_0} = 1$ and $S(C)=0$.
\end{enumerate}


\begin{proof}[Proof of \ref{thm:HahnBanach:Extension1}]
For $\alpha \in \{Z,X\}$, let 
$\Omega_\alpha:\alpha^* \to (\alpha/\Ker_\alpha)^*$ denote the isomorphism
defined in 
\ref{thm:dualspaceisomorphism}.
Let $q:X \to X/\Ker$ denote the 
\QuotientMap.
Define $T:Z/\Ker_Z \to q(Z)$ bv $T([z]_{\cong_Z} ) = [z]_{\cong_X}$. %Make a separate Result
T is clearly a \Linear \Isometric \Homeomorphism.          %Make a separate Result

Define $\Gamma_Z:(Z/\Ker_Z)^* \to q(Z)^*$ by setting, 
for $\phi^* \in (Z/\Ker_Z)^*$, 
for $[z]_Z \in Z/\Ker_Z$, 
\begin{equation*}
\ip{T[z]_Z, \Gamma_Z \phi^*} = \ip{[z]_Z, \phi^*}
\end{equation*}
Then $\Gamma_Z$ is a \Linear Bijective \Isometry. 
Hence $\Gamma_Z \circ \Omega_Z z^* \in q(Z)^*$ with 
$\norm{\Gamma_Z \circ \Omega_Z z^*}_{q(Z)^*} = \norm{z^*}_{Z^*}$. 
Thus we can apply the Hahn Banach theorem for \NormedSpaces to claim 
the existence of $x_q^* \in (X/\Ker_X)^*$ where
$x_q^*$ is an extension of $\Gamma_Z \circ \Omega_Z z^*$ and
\begin{equation*}
\norm{x_q^*}_{(X/\Ker_X)^*} = \norm{\Gamma_Z \circ \Omega_Z z^*}_{(q(Z))^*} = \norm{z^*}_{Z^*}
\end{equation*}
Finally, letting $x^* = \Omega_X^{-1} x_q^*$, we have 
$x^* \in X^*$, 
$\norm{x^*}_{X^*} = \norm{x_q^*}_{(X/\Ker_X)^*} =\norm{z^*}_{Z^*}$, 
and 
if $z \in Z$, then 
\begin{align*}
\ip{z, x^*} & = \ip{[z]_X, x_q^*} \\
& = \ip{[z]_X, \Gamma_Z \circ \Omega_Zz^*}\\
& = \ip{[z]_Z, \Omega_Z z^*}\\
& = \ip{z, z^*}
\end{align*}
\end{proof}
\begin{proof}[Proof of \ref{thm:HahnBanach:Point1}]
Let $Z=span(x)$. 
Define $z^* \in Z^*$ by 
$\ip{\alpha x, z^*} = \alpha \norm{x}$. 
Then $\norm{z^*} = 1$. 
Also, by \ref{thm:HahnBanach:Extension1},
it has an extension $x^* \in X^*$ with 
$\norm{x^*} = \norm{z^*} =1$ 
and $\ip{x,x^*} = \norm{x}$. 
\end{proof}
\begin{proof}[Proof of \ref{thm:HahnBanach:Norm}]
If $\norm{x} = 0$, then
for every $x^* \in X$, $\ip{x,x^*} = 0$.
Hence 
\begin{equation*} 
\norm{x}_X = \sup\limits_{0 \neq x^* \in X^*} \frac{\ip{x,x^*}}{\norm{x^*}} = \sup\limits_{x^* \in \partial B_{X^*}(0;1)} \frac{\ip{x,x^*}}{\norm{x^*}}=0
\end{equation*}
by \ref{thm:HahnBanach:Point1}, there exists 
$x^* \in X^*$ such that $\norm{x^*} = 1$ and $\ip{x,x^*} = \norm{x}$.
Then 
\begin{align*}
\norm{x} & = \frac{\ip{x,x^*}}{\norm{x^*}} \\
& \leq \sup\limits_{x^* \in \partial B_{X^*}(0;1)} \frac{\ip{x,x^*}}{\norm{x^*}} \\
& \leq    \sup\limits_{0 \neq x^* \in X^*} \frac{\ip{x,x^*}}{\norm{x^*}} 
\end{align*}
The other direction of the inequality is trivial.
\end{proof}
\begin{proof}[Proof of \ref{thm:HahnBanach:Operator1}]
By \ref{thm:HahnBanach:Point1},
there exists $x_0^* \in X^*$ with 
$\norm{x_0^*} = 1$ and $\ip{x_0, x_0^*}=\norm{x_0}$. 
Since Y is \NonDegenerate, there
exists $y_0 \in Y$ with $\norm{y_0} = 1$. 
Define $T: \F \to Y$ by $T \alpha = \alpha y$.
Then $\norm{T} = \norm{y} = 1$. 
Define $S: X \to Y$ by $S=T \circ x_0^*$. 
Then $\norm{S} \leq \norm{T} \norm{x_0^*} = 1$, and
$\norm{S x_0} = \norm{\ip{x_0, x_0^*} y} = \ip{x_0, x_0^*} = \norm{x_0}$. 
Hence $\norm{S} \geq 1$ and therefore $\norm{S} = 1$. 
\end{proof}
\begin{proof}[Proof of \ref{thm:HahnBanach:Open}]
We consider the case where $\F = \R$.
Let $u_0 \in U$ and $c_0 \in C$. 
Define $D = (U-u_0)-(C-c_0)$. 
Then $D$ is \SetOpen, \ConvexSet, and $0 \in D$. 
Let $\mu_D$ denote the \MinkowskiFunctional of $D$. 
By 
\ref{prop:MinkowskiFunctional}, $D$ is a \SublinearFunctional. 
Also, since $U \cap C = \emptyset$, $c_0-u_0 \not \in D$. 
Hance, $\mu_D(c_0-u_0) \geq 1$
Define $T:span(c_0-u_0) \to \F$ by $T(\alpha (c_0-u_0)) = \alpha$.
Then $T$ is \Linear and 
$Tx \leq \mu_D(x)$ for $x \in spn(c_0-u_0)$. 
Hence, by the Hahn Banach theorem for vector spaces, 
$T$ permits a \Linear \Extension to $X$, denoted
$\tilde{T}$, which is still dominated by $\mu_D$. 
Also, $\abs{\tilde{T}x} \leq \mu_D(x) \leq 1$ for $x \in D \cap (-D)$, 
so $\tilde{T}$ is \ContinuousFunction.
Let $u \in U$. 
Let $c \in C$. 
Then, since $\tilde{T}\pa{c_0-u_0}$, we have 
\begin{equation*}
\tilde{T}u-\tilde{T}c+1=\tilde{T}\pa{u-c+c_0-u_0} \leq \mu_D \pa{u-c+c_0-u_0} < 1
\end{equation*}
Hence, $\tilde{T} u < \tilde{T} c$ for every $c \in C$. 
Since $u$ and $c$ are arbitrary elements of $U$ and $C$ respectively, 
$\tilde{T}U \cap \tilde{T}C = \emptyset$
and $\sup\limits_{u \in U} \tilde{T}u \leq \inf\limits_{c \in C} \tilde{T} c$. 
Since $\tilde{T}U$ is \SetOpen, for each $u \in U$, 
$\tilde{T} u < \sup\limits_{u \in U} \tilde{T} u \leq \inf\limits_{c \in C} \tilde{T}c$.
\end{proof}
\begin{proof}[Proof of \ref{thm:HahnBanach:ClosedConvex}]
By 
\ref{prop:GroupSeparation} and 
\ref{prop:Bal3}, 
there exists a \ConvexSet \SetOpen $U$ 
such that 
$\pa{C+U} \cap {K+U} = \emptyset$.
In particular $C \cap \pa{K+U} = \emptyset$, 
and $K + U $ is \SetOpen and \ConvexSet.
Hence, by 
\ref{thm:HahnBanach:Open}, 
there is a $\tilde{T} \in X^*$ such that 
for each $x \in K+U$, 
\begin{equation*}
\tilde{T}x < \sup\limits_{x \in U+K} \tilde{T}x \leq \inf\limits_{c \in C}\tilde{T}c
\end{equation*}
Now, since $\tilde{T}$ is continuous, $\tilde{T}K$ is a \SetCompact 
subset of $\tilde{T}(K+U)$. 
Hence, the claimed result holds. 
\end{proof}
\begin{proof}[Proof of \ref{thm:HahnBanach:Nullspace}]
By \ref{thm:HahnBanach:ClosedConvex}, 
there exists $j_0 \in X^*$ such that 
$\sup\limits_{x \in C} \ip{x,j_0} < j_0(x_0)$.
Since $C$ is a \VectorSubspace of $X$, $j_0(C)$ is a 
\VectorSubspace of $\R$. Hence, $j_0(C) = 0$.
Hence $j=\frac{j_0}{\ip{x_0, j_0}}$ is our functional. 
\end{proof}
\begin{proof}[Proof of \ref{thm:HahnBanach:NullspaceOperator}]
By \ref{thm:HahnBanach:Nullspace}, there exists 
$j_0 \in X^*$ with $j(C) = 0$ and $\ip{x_0,j_0} = 1$. 
Since $Y$ is \NonDegenerate, there exists
$y_0 \in Y$ with $\norm{y_0} = 1$.
Define $T:\F \to Y$ by $T\alpha = \alpha y_0$. 
Define $S:X \to Y$ by $S = T \circ j_0$. 
Then $S \in BL(X,Y)$ 
and $\norm{Sx_0} = \norm{\ip{x_0,j_0}y} = \ip{x_0,j_0} = 1$.
\end{proof}
\end{thm}

