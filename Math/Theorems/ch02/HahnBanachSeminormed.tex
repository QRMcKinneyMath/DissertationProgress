\label{thm:hahnbanach}
\begin{thm}[Hahn Banach Theorem For Seminormed Spaces]
Let $(X,\norm{\cdot})$ be a \SeminormedSpace,
let $x_i \in X$ for $i \in \{0,1\}$ such that 
$\norm{x_0-x_1}_X \neq 0$, and
let $X^*$ denote $X's$
\TopDualSpace. 
The following are true. 
\begin{enumerate}
    \item If $Z \subset X$ is a subspace
        and $z^* \in Z^*$, then there 
        is an extension $x^*$ of $z^*$, 
        $x^* \in X^*$ such that 
        \begin{equation}
        \norm{z^*}_{Z^*} = \norm{x^*}_{X^*}
        \end{equation}
     \item If $x \in X$, 
        with $\norm{x} \neq 0$, 
        then there exists an
        $x^* \in X$ with 
        $\norm{x^*}=1$ and 
        $\ip{x,x^*} = \norm{x}_X$. 
    \item If $x \in X$, then 
    \begin{equation}
        \norm{x}_X = \sup\limits_{0 \neq x^* \in X^*} \frac{\ip{x,x^*}}{\norm{x^*}}
    \end{equation}
    \item If $Y$ is a 
        \NonDegenerate
        \SeminormedSpace, and if 
        $x_0 \in X$, with 
        $\norm{x_0} \neq 0$, 
        then there exists
        an $S \in BL(X,Y)$ with 
        $\norm{S} = 1$ and 
        \begin{equation}
            \norm{Sx_0} = \norm{x_0}
        \end{equation}
\end{enumerate}


\begin{proof}[Proof of 01]
    For $\alpha \in \{Z,X\}$, let 
    $\Omega_\alpha:\alpha^* \to (\alpha/\Ker_\alpha)^*$ denote the isomorphism
    defined in 
    \ref{thm:dualspaceisomorphism}.
    Let $q$ denote the quotient operator $q:X \to X/\Ker$. 
    Define $T:Z/\Ker_Z \to q(Z)$ bv $T([z]_{\cong_Z} ) = [z]_{\cong_X}$. %Make a separate Result
    Since Z is endowed with the subspace Topology,                       %Make a separate Result
    T is obviously a Linear Bijective Bicontinuous Isometry.          %Make a separate Result
    
    %Then $\Omega_Zz^* \in (Z/\Ker_Z)^*$ satisfies
    %$\norm{\Omega_Zz^*}_{Z/\Ker_Z}=\norm{z^*}_{Z^*}$ an 
    Define $\Gamma_Z:(Z/\Ker_Z)^* \to q(Z)^*$ by setting, 
    for $\phi^* \in (Z/\Ker_Z)^*$, 
    for $[z]_Z \in Z/\Ker_Z$, 
    \begin{equation}
        \ip{T[z]_Z, \Gamma_Z \phi^*} = \ip{[z]_Z, \phi^*}
    \end{equation}
    Then $\Gamma_Z$ is a Linear Bijective Isometry. 
    Hence $\Gamma_Z \circ \Omega_Z z^* \in q(Z)^*$ with 
    $\norm{\Gamma_Z \circ \Omega_Z z^*}_{q(Z)^*} = \norm{z^*}_{Z^*}$. 

    Thus we can apply the Hahn Banach theorem for \NormedSpaces to claim 
    the existence of $x_q^* \in (X/\Ker_X)^*$ where
    $x_q^*$ is an extension of $\Gamma_Z \circ \Omega_Z z^*$ and
    \begin{equation}
        \norm{x_q^*}_{(X/\Ker_X)^*} = \norm{\Gamma_Z \circ \Omega_Z z^*}_{(q(Z))^*} = \norm{z^*}_{Z^*}
    \end{equation}
    Finally, letting $x^* = \Omega_X^{-1} x_q^*$, we have 
    $x^* \in X^*$, 
    $\norm{x^*}_{X^*} = \norm{x_q^*}_{(X/\Ker_X)^*} =\norm{z^*}_{Z^*}$, 
    and 
    if $z \in Z$, then 
    \begin{align*}
        \ip{z, x^*} & = \ip{[z]_X, x_q^*} \\
        & = \ip{[z]_X, \Gamma_Z \circ \Omega_Zz^*}\\
        & = \ip{[z]_Z, \Omega_Z z^*}\\
        & = \ip{z, z^*}
    \end{align*}
\end{proof}

\begin{proof}[Proof of 2]
    Let $Z=span(x)$. 
    Define $z^* \in Z^*$ by 
    $\ip{\alpha x, z^*} = \alpha \norm{x}$. 
    Then $\norm{z^*} = 1$. 
    Also, by part 1 of this result, 
    it has an extension $x^* \in X^*$ with 
    $\norm{x^*} = \norm{z^*} =1$ 
    and $\ip{x,x^*} = \norm{x}$. 
\end{proof}
\begin{proof}[Proof of 3]
    If $\norm{x} = 0$, then
    for every $x^* \in X$, $\ip{x,x^*} = 0$.
    Hence 
    \begin{equation} 
    \norm{x}_X = \sup\limits_{0 \neq x^* \in X^*} \frac{\ip{x,x^*}}{\norm{x^*}} = \sup\limits_{x^* \in \partial B_{X^*}(0;1)} \frac{\ip{x,x^*}}{\norm{x^*}}=0
    \end{equation}

    Otherwise, let  $x^* \in X^*$ guaranteed to 
    exist by part 2 which satisfies $\norm{x^*}=1$, 
    $\ip{x,x^*} = \norm{x}$. 
    Then 
    \begin{align*}
    \norm{x} & = \frac{\ip{x,x^*}}{\norm{x^*}} \\
    & \leq \sup\limits_{x^* \in \partial B_{X^*}(0;1)} \frac{\ip{x,x^*}}{\norm{x^*}} \\
    & \leq    \sup\limits_{0 \neq x^* \in X^*} \frac{\ip{x,x^*}}{\norm{x^*}} 
    \end{align*}
    The other direction of the inequality
    falls directly from the definition of 
    the norm on $X^*$, and is trivial, so 
    we are done. 
\end{proof}
\begin{proof}[Proof of 4]
    By part 2 of this result, there exists $x_0^* \in X^*$ with 
    $\norm{x_0^*} = 1$ and $\ip{x_0, x_0^*}=\norm{x_0}$. 
    Since Y is \NonDegenerate, there
    exists $y_0 \in Y$ with $\norm{y_0} = 1$. 
    Define $T: \F \to Y$ by $T \alpha = \alpha y$.
    Then $\norm{T} = \norm{y} = 1$. 
    Define $S: X \to Y$ by $S=T \circ x_0^*$. 
    Then $\norm{S} \leq \norm{T} \norm{x_0^*} = 1$, and
    $\norm{S x_0} = \norm{\ip{x_0, x_0^*} y} = \ip{x_0, x_0^*} = \norm{x_0}$. 
    Hence $\norm{S} \geq 1$ and therefore $\norm{S} = 1$. 
\end{proof}
\end{thm}

