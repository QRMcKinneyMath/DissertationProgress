%%%
%%% Environments
%%%



\newsavebox{\speaker}
\newenvironment{chapterquote}[1]
    {\begin{flushright}\begin{minipage}{3 in}\sbox{\speaker}{#1}\itshape\singlespace}
    {\begin{flushright}---\usebox{\speaker}\end{flushright}\end{minipage}\end{flushright}}

% Note: chapterquote works (and looks) best when the chapter begins with a \section{} ...
% If you hadn't planned on beginning the chapter with a \section, try \section{Overview} :-)
% otherwise, some \vspace{} might be necessary, but that won't be consistent throughout the thesis

%%%
%%%  Functions---Commands that take parameters
%%%
\newcommand{\cfig}[3]{\centering\includegraphics[keepaspectratio=true,width=#3in]{./CH#1/EPSFDocs/#2}}
\newcommand{\acro}[1]{\textsc{#1}}
\newcommand{\sci}[2]{\ensuremath{#1 \!  \times \!  10^{#2}}}
\newcommand{\vect}[1]{\boldsymbol{\mathbf{#1}}}
\newcommand{\unitvec}[1]{\vect{\hat{#1}}}
\newcommand{\threebythree}[9]{\renewcommand{\arraystretch}{0.75}\begin{vmatrix}#1&#2&#3\\#4&#5&#6\\#7&#8&#9\end{vmatrix}\renewcommand{\arraystretch}{1.0}}
\newcommand{\mysci}[2]{\ensuremath{#1 \!  \times \!  10^{#2}}}
\newcommand{\myprop}[5]{g(#1,#2;#3,#4;#5)}
\newcommand{\myint}[3]{\int_{#1}^{#2}#3}
\newcommand{\myintinfinf}[1]{\myint{-\infty}{\infty}{#1}}


%%%
%%%  New Math Operators
%%%

\DeclareMathOperator{\polylog}{Li}


%%%
%%%  Symbols---Commands that are shorthand
%%%
\newcommand{\rf}{\acro{rf}\xspace}
\newcommand{\dc}{\acro{dc}\xspace}
\newcommand{\ac}{\acro{ac}\xspace}
\newcommand{\ccd}{\acro{ccd}\xspace}
\newcommand{\mach}{\acro{mach2}\xspace}
\newcommand{\mgmhd}{\acro{mgmhd}\xspace}
\newcommand{\cea}{\acro{cea}\xspace}
\newcommand{\half}{\nicefrac{1}{2}\,}
\newcommand{\xx}{\ensuremath{\unitvec{x}}\xspace}
\newcommand{\yy}{\ensuremath{\unitvec{y}}\xspace}
\newcommand{\zz}{\ensuremath{\unitvec{z}}\xspace}
\newcommand{\ttl}{\acro{ttl}\xspace}
\newcommand{\pmt}{\acro{pmt}\xspace}
\newcommand{\eg}{\textit{e.g.}\xspace}
\newcommand{\ie}{\textit{i.e.}\xspace}
\newcommand{\qmn}{\ensuremath{q^{mn}}\xspace}
\newcommand{\isp}{\ensuremath{I_{sp}}\xspace}



%%%
%%%  Other stuff I've found - Evaluate for usefulness
%%%

%\newcommand{\quan}[2][]{\mbox{$#1\,\mathrm{#2}$}}
%\newcommand{\vv}[1]{\ensuremath{\boldsymbol{#1}}}
%\newcommand{\tempc}[1]{\quan[#1]{^{\circ}C}}
%\DeclareMathOperator{\sinc}{sinc}
%\newcommand{\comment}[1]{\marginpar{\Large \hfill \ddag}\textsf{#1}}
%\newcommand{\comment}[1]{}

\newcommand\tab[1][3cm]{\hspace*{#1}}
\newcommand{\Real}{\ensuremath{\mathbb{R}}\xspace}
\newcommand{\Proj}{\ensuremath{\mathbb{P}}\xspace}
\newcommand{\Integer}{\ensuremath{\mathbb{Z}}\xspace}
\newcommand{\Natural}{\ensuremath{\mathbb{N}}\xspace}
\newcommand{\Complex}{\ensuremath{\mathbb{C}}\xspace}
\newcommand{\Rational}{\ensuremath{\mathbb{Q}}\xspace}

%%Even Shorter Notation
\newcommand{\N}{\Natural}
\newcommand{\Z}{\Integer}
\newcommand{\Q}{\Rational}
\newcommand{\R}{\Real}
\newcommand{\C}{\Complex}
\newcommand{\F}{\mathbb{F}\xspace}
\newcommand{\RP}{\R\P\xspace}
\newcommand{\CP}{\C\P\xspace}
\newcommand{\e}{\varepsilon\xspace}
\newcommand{\Funp}{\mathcal{P}\xspace}
\newcommand{\T}{\ensuremath{\mathcal{T}}\xspace}

%bbNotation
\newcommand{\bbA}{\ensuremath{\mathbb{A}}\xspace}
\newcommand{\bbB}{\ensuremath{\mathbb{B}}\xspace}
\newcommand{\bbC}{\ensuremath{\mathbb{C}}\xspace}
\newcommand{\bbD}{\ensuremath{\mathbb{D}}\xspace}
\newcommand{\bbE}{\ensuremath{\mathbb{E}}\xspace}
\newcommand{\bbF}{\ensuremath{\mathbb{F}}\xspace}
\newcommand{\bbG}{\ensuremath{\mathbb{G}}\xspace}
\newcommand{\bbH}{\ensuremath{\mathbb{H}}\xspace}
\newcommand{\bbI}{\ensuremath{\mathbb{I}}\xspace}
\newcommand{\bbJ}{\ensuremath{\mathbb{J}}\xspace}
\newcommand{\bbK}{\ensuremath{\mathbb{K}}\xspace}
\newcommand{\bbL}{\ensuremath{\mathbb{L}}\xspace}
\newcommand{\bbM}{\ensuremath{\mathbb{M}}\xspace}
\newcommand{\bbN}{\ensuremath{\mathbb{N}}\xspace}
\newcommand{\bbO}{\ensuremath{\mathbb{O}}\xspace}
\newcommand{\bbP}{\ensuremath{\mathbb{P}}\xspace}
\newcommand{\bbQ}{\ensuremath{\mathbb{Q}}\xspace}
\newcommand{\bbR}{\ensuremath{\mathbb{R}}\xspace}
\newcommand{\bbS}{\ensuremath{\mathbb{S}}\xspace}
\newcommand{\bbT}{\ensuremath{\mathbb{T}}\xspace}
\newcommand{\bbU}{\ensuremath{\mathbb{U}}\xspace}
\newcommand{\bbV}{\ensuremath{\mathbb{V}}\xspace}
\newcommand{\bbW}{\ensuremath{\mathbb{W}}\xspace}
\newcommand{\bbX}{\ensuremath{\mathbb{X}}\xspace}
\newcommand{\bbY}{\ensuremath{\mathbb{Y}}\xspace}
\newcommand{\bbZ}{\ensuremath{\mathbb{Z}}\xspace}

%
\newcommand{\scA}{\ensuremath{\mathcal{A}}\xspace}
\newcommand{\scB}{\ensuremath{\mathcal{B}}\xspace}
\newcommand{\scC}{\ensuremath{\mathcal{C}}\xspace}
\newcommand{\scD}{\ensuremath{\mathcal{D}}\xspace}
\newcommand{\scE}{\ensuremath{\mathcal{E}}\xspace}
\newcommand{\scF}{\ensuremath{\mathcal{F}}\xspace}
\newcommand{\scG}{\ensuremath{\mathcal{G}}\xspace}
\newcommand{\scH}{\ensuremath{\mathcal{H}}\xspace}
\newcommand{\scI}{\ensuremath{\mathcal{I}}\xspace}
\newcommand{\scJ}{\ensuremath{\mathcal{J}}\xspace}
\newcommand{\scK}{\ensuremath{\mathcal{K}}\xspace}
\newcommand{\scL}{\ensuremath{\mathcal{L}}\xspace}
\newcommand{\scM}{\ensuremath{\mathcal{M}}\xspace}
\newcommand{\scN}{\ensuremath{\mathcal{N}}\xspace}
\newcommand{\scO}{\ensuremath{\mathcal{O}}\xspace}
\newcommand{\scP}{\ensuremath{\mathcal{P}}\xspace}
\newcommand{\scQ}{\ensuremath{\mathcal{Q}}\xspace}
\newcommand{\scR}{\ensuremath{\mathcal{R}}\xspace}
\newcommand{\scS}{\ensuremath{\mathcal{S}}\xspace}
\newcommand{\scT}{\ensuremath{\mathcal{T}}\xspace}
\newcommand{\scU}{\ensuremath{\mathcal{U}}\xspace}
\newcommand{\scV}{\ensuremath{\mathcal{V}}\xspace}
\newcommand{\scW}{\ensuremath{\mathcal{W}}\xspace}
\newcommand{\scX}{\ensuremath{\mathcal{X}}\xspace}
\newcommand{\scY}{\ensuremath{\mathcal{Y}}\xspace}
\newcommand{\scZ}{\ensuremath{\mathcal{Z}}\xspace}

%% Groupings, Inner Products, Etc
\newcommand{\pa}[1]{\left( #1\right)}
\newcommand{\parens}[1]{\pa{#1}}
\newcommand{\bra}[1]{\left[ #1 \right]}
\newcommand{\brackets}[1]{\left[ #1 \right]}
\newcommand{\braces}[1]{\left\{ #1 \right\}}
\newcommand{\norm}[1]{\left|\left| #1 \right|\right|}
\newcommand{\floor}[1]{\left\lfloor #1 \right\rfloor}
\newcommand{\ip}[1]{\left\langle #1 \right\rangle}
\newcommand{\conjip}[1]{\overline{\innerprod{#1}}}
\newcommand{\ceil}[1]{\left\lceil #1 \right\rceil}
\newcommand{\abs}[1]{\left| #1 \right|}
\newcommand{\isthm}[1]{\vdash_{\mathcal{#1}}}



%% Function notation, Convergence
\newcommand{\spto}[1]{\overset{#1}{\to}}
\newcommand{\fto}{\spto{Functor}}
\newcommand{\mto}{\spto{\textbf{m}}}
\newcommand{\wto}{\spto{\textbf{w}}}
\newcommand{\wsto}{\spto{w^*}}

