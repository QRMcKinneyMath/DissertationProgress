\chapter{Introduction}
\label{ch:intro}
\begin{rmk}
\rm
Topologically speaking, 
the two properties that distinguish a normable space from a general topological vector space are local convexity and local boundedness. 
A norm, however, provides more information than what is necessary to 
define a topology on the vector space. 
That is, norm on a normed space can also impart geometric properties to the space
which may not be preserved under homeomorphism. 
Examples of these properties include differentiability 
of the norm and roundness of the unit ball.
These properties, which include
strict convexity, 
weak local uniform convexity,
local uniform convexity, 
uniform convexity, 
smoothness, 
very smoothness,
local uniform smoothness, 
and uniform smoothness, 
have been studied prolifically by many authors
and can be profitably leveraged to aid solving various problems. 
The overall goal of this paper is to, 
guided by the pre-existing geometry theory of norms, 
develop a geometry theory of 
geometry of seminorms which will enable the exploitation
of some of these geometric properties to find solutions to problems
in a space which is not locally bounded.

It is well known that a Frechet space $(X,\scT)$ has a topology which is generated 
by a sequence of seminorms $\{\norm{\cdot}\}_{i \in \N}$ 
which are the minkowski functionals 
of a countable neighborhood base of the space.
These seminorms can then define a
translation invariant metric 
\begin{equation*}
d(x,y) = \sum\limits_{n \in \N} \frac{1}{2^n} \frac{\norm{x-y}_n}{1+\norm{x-y}}
\end{equation*}
Fixed point theory for operators defined on certain classes of 
Frechet spaces have been studied, in perticular for those posessing the Dunford-Pettis property \cite{amar11}  or the particular case of $C(\R^+)$ \cite{olszowy12}, but
despite this, up until now, very little effort has gone into 
the analysis of the geometric properties of Frechet spaces 
which stem directly from the geometric properties of 
the individual seminorms which generate the topologies on such spaces.
The author believes the primary reasons for this fact to be as follows. 
\begin{enumerate}
\item Until now, discussion of the 
convexity and smoothness properties mentioned 
above had only every been published or discussed in the context of a Hausdorff space. 
In fact, 
prior this paper, 
the notion of a strictly convex semionorm had never even appeared 
in the literature. 
For this reason, the theory of the geometry of seminorms would need
to be developed prior to developing a theory of geometry of 
spaces generated by such seminorms. 
This is a lofty endeavor. 
\item The sequence of seminorms $\{\norm{\cdot}_i\}_{i \in \N}$ which generates the topology on $X$ is not unique, and indeed two sequences of seminorms may have wildly different geometric properties. Due to this, such analysis would likely need to be not analysis of a Frechet space $(X,\scT)$ but of a geometric space 
$\pa{X, \scT, \{\norm{\cdot}_i\}_{i \in \N}}$ for which the selection of 
the sequence of 
seminorms is considered intrinsic to the space itself. 

\end{enumerate}
Accordingly, this dissertation presents a solution to each of these issues 
and then applies the results to develop some previously unknown fixed point results for operators defined on Frechet spaces. 
The approach taken is as follows
\begin{enumerate}
\item Logical, topological, and algebraic prerequisites for this theory are developed in detail,
This includes a development of pseudo-Hausdorff spaces which satisfy none of the well known separability axioms but possess many properties which are analogous 
to those possessed by Hausdorff spaces.
Notably, all topological groups are pseudo-Hausdorff, including those which are not hausdorff.
\item Next, a theory of the duality of a seminormed space is explored, including the fact that the dual space of any seminormed space is itself a normed space which is 
the isometrically isomorphic dual space of a particular
canonical quotient of the seminormed space. 
This development includes an explicit representation of the isomorphism between the two
dual spaces in terms of a generalized adjoint operator for continuous linear operators
defined between seminormed spaces.
\item Next, the theory of reflexivity, convexity, and smoothness of a seminormed space are developed in detail. 
Many of these results rhyme with those that exist for normed spaces, 
but have never been considered in the 
but in publications have, to the author's knowledge, 
not been considered in the context of a non-Hausdorff space.
For this reason, full proofs are included for all of the included results in this category.
\item Next, results are developed regarding algebraic operations which preserve favorable properties of a sequence of seminorms. Of particular note are 
those transformations of a seminorm which preserve convexity, differentiability, accretivity, and pseudocontractivity of operators. 
\item Finally, the machinery built thus far is used to derive several novel fixed point results for operators defined on Frechet Spaces.
\end{enumerate}


The general technique to find a fixed point for a mapping
$T:X \subset (X, \{\norm{\cdot}\}_{i \in \N}) \to C$ 
that is used for item (5) is as follows
\begin{enumerate}
\item Determine sufficient conditions on T and or C to guarantee the existence of a preserved kernel. That is, find conditions such that for some $\tilde{x}_k \in \C$, $T\pa{\tilde{x}_k,+\overline{0}_k} \subset \tilde{x}_k+ \overline{0}_k$ where $\overline{0}_k = \{x \in X : \norm{x}_k = 0\}$. 
\item Determine additional conditions on $T$ to guarantee that, under some iterations scheme $\{x_n\}_{n \in \N}$, which is independent of k. That is, we aim to find conditions on $T$ and a function f
such that if $x_{n+1} = f(x_n, Tx_n, \alpha_n, \beta_n)$, $x_n \to \tilde{x_k}$ with respect to some topology on $X$ related to $\norm{\cdot}_k$.
\item Conclude that this implies $\{x_i\}_{i \in \N}$ is a Cauchy sequence which has a limit by completeness, and finally argue that this limit must be a fixed point of $T$. 
\end{enumerate}

\end{rmk}


