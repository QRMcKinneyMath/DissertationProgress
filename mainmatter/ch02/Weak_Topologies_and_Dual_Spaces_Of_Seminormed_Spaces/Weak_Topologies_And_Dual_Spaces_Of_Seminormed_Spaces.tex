\section{Weak Topologies And Dual Spaces Of Seminormed Spaces}

\subsection{Basics}



\newcommand{\scPowerSet}[1]{
	\ensuremath{2^{#1}}
}





\label{def:SetInclusion}
\begin{df}[$\in$]
    Let $X$ and $Y$ be sets. 
    We use the notation $Y \in X$ to indicate 
    that $Y$ is an element of $X$. 
    If $Y$ is not an element of $X$ then we write
    $Y \not \in X$. 
\end{df}

\newcommand{\QSubset}[0]{
    \textbf{\hyperref[def:Subset]{Subset}}
}
\newcommand{\QSubsets}[0]{
    \textbf{\hyperref[def:Subset]{Subsets}}
}
\newcommand{\Superset}[0]{
    \textbf{\hyperref[def:Subset]{Superset}}
}
\newcommand{\Supersets}[0]{
    \textbf{\hyperref[def:Subset]{Supersets}}
}
\begin{df}[Subset]
\label{def:Subset}

\rm
    Let $X$ and 
    $Y$ be sets
    such that $x \in Y\implies x \in X$. 
    Then we write $Y \subset X$ 
    and we say that $Y$ is a 
    \QSubset of $X$
    and we write $X \supset Y$
    and we say that $X$ is a 
    \Superset of $Y$. 
    
\end{df}

\newcommand{\SetComplement}[0]{
    \textbf{\hyperref[def:SetComplement]{Complement}}
}
\newcommand{\SetComplements}[0]{
    \textbf{\hyperref[def:SetComplement]{Complements}}
}
\begin{df}[Set Difference]
\label{def:SetComplement}

\rm
    Let $X$ and $Y$ be sets. 
    We define 
    \begin{equation*}
        X \setminus Y= \{x \in X | x \not \in Y\}
    \end{equation*}
    We call $X \setminus Y$ the 
    \SetComplement of $Y$ relative to $X$. 
\end{df}

\label{def:PairSet}
\newcommand{\PairSet}[0]{
    \textbf{\hyperref[def:PairSet]{Pair Set}}
}
\newcommand{\PairSets}[0]{
    \textbf{\hyperref[def:PairSet]{Pair Sets}}
}
\begin{df}[\PairSet]
    Let $X$ and $Y$. be sets. 
    We then assume that the set $Z$ 
    containing exactly $X$ and $Y$ is a 
    set which we call the \PairSet of $X$ and $Y$. 
\end{df}

\newcommand{\OrderedPair}[0]{\textbf{\hyperref[def:OrderedPair]{Ordered Pair}}\xspace}
\newcommand{\OrderedPairs}[0]{\textbf{\hyperref[def:OrderedPair]{Ordered Pairs}}\xspace}
\begin{df}[Ordered Pair]
\label{def:OrderedPair}

\rm
    Let $X$ and $Y$ be sets. 
    Then we define 
    $(X,Y)= \{\{X\}, \{X, Y\}\} \in \scPowerSet{\{X, Y\}}$. 
    We call $(X,Y)$ the \OrderedPair of $X$ with $Y$. 
\end{df}

\newcommand{\BinaryCartesianProduct}[0]{\textbf{\hyperref[def:BinaryCartesianProduct]{Binary Cartesian Product}}\xspace}
\begin{df}[Binary Cartesian Product]
\label{def:BinaryCartesianProduct}

\rm
    Let $X \neq \emptyset$ and let 
    $Y \neq \emptyset$. 
    We define 
    $X \times Y =\{(x,y) \in \scPowerSet{\scPowerSet{X \cup Y}} | x \in X \wedge y \in Y\}$
    We call 
    $X \times Y$ the 
    \BinaryCartesianProduct
    of $X$ with $Y$. 
\end{df}

\newcommand{\SetDiagonal}[0]{
    \textbf{\hyperref[def:SetDiagonal]{Diagonal}}
}
\newcommand{\SetDiagonals}[0]{
    \textbf{\hyperref[def:SetDiagonal]{Diagonals}}
}

\newcommand{\scSetDiagonal}[1]{
    \ensuremath{\hyperref[def:SetDiagonal]{\Delta\pa{#1}}}
}\begin{df}[Set Diagonal]
\label{def:SetDiagonal}

\rm
    Let $X$ be a set. 
    We define 
    $\scSetDiagonal{X}=\{(x,x)\in X \times X | x \in X\}$
    and we call 
    \scSetDiagonal{X} 
    the 
    \SetDiagonal
    of $X$.
\end{df}

\label{def:Relation}
\newcommand{\Relation}[0]{
    \bf \hyperref[def:Relation]{Relation} \rm
}
\begin{df}[\Relation]
    Let $X \neq \emptyset$ be a set
	and let $Y\neq \emptyset$ be a set. 
    We say that $R$ is a \Relation
    from $X$ to $Y$ if $R \subset X \times Y$. 
    If $(a,b) \in R$, then we may write
    $a R b$. 
\end{df}

\label{def:Function}
\newcommand{\Function}[0]{
    \textbf{\hyperref[def:Function]{Function}}
}
\newcommand{\Functions}[0]{
    \textbf{\hyperref[def:Function]{Functions}}
}
\newcommand{\Map}[0]{
    \textbf{\hyperref[def:Function]{Map}}
}
\newcommand{\Maps}[0]{
    \textbf{\hyperref[def:Function]{Maps}}
}
\newcommand{\Mapping}[0]{
    \textbf{\hyperref[def:Function]{Mapping}}
}
\newcommand{\Mappings}[0]{
    \textbf{\hyperref[def:Function]{Mappings}}
}

\newcommand{\FunctionDomain}[0]{
    \textbf{\hyperref[def:Function]{Domain}}
}
\newcommand{\FunctionDomains}[0]{
    \textbf{\hyperref[def:Function]{Domains}}
}
\newcommand{\FunctionCodomain}[0]{
    \textbf{\hyperref[def:Function]{Codomain}}
}
\newcommand{\FunctionCodomains}[0]{
    \textbf{\hyperref[def:Function]{Codomains}}
}
\newcommand{\FunctionRange}[0]{
    \textbf{\hyperref[def:Function]{Range}}
}
\newcommand{\FunctionRanges}[0]{
    \textbf{\hyperref[def:Function]{Ranges}}
}

\newcommand{\FunctionImage}[0]{
    \textbf{\hyperref[def:Function]{Image}}
}
\newcommand{\FunctionImages}[0]{
    \textbf{\hyperref[def:Function]{Images}}
}
\newcommand{\FunctionPreimage}[0]{
    \textbf{\hyperref[def:Function]{Preimage}}
}
\newcommand{\FunctionPreimages}[0]{
    \textbf{\hyperref[def:Function]{Preimages}}
}

\begin{df}[\Function]
    Let $X \neq \emptyset$ 
    and $Y \neq \emptyset$.
    Let $f \subset X \times Y$ 
    such that for each $x \in X$ 
    there is a Unique $y \in Y$ 
    such that 
    $(x,y) \in f$. 
    Then we say that $f$ 
    is a \Function
    from $X$ into $Y$. 
    and we write 
    $f:X \to Y$. 
    We may also call $f$ 
    a 
    \Map
    or a 
    \Mapping 
    from $X$ 
    into $Y$. 
    Primarily, though 
    we will rely on the notation 
    $f:X \to Y$ to indicate
    that 
    $f$ is a 
    \Function
    with 
    \FunctionDomain
    $X$
    and 
    \FunctionCodomain
    $Y$.
    If $A \subset X$ 
    and $B \subset Y$, 
    then we denote 
    \begin{equation*}
    f\pa{A}= \{f(x)\in Y | x \in A\} \\
    \; \; \;
    f^{-1}\pa{B} = \{x \in X | f(x) \in B \}
    \end{equation*}
    We call 
    $f(A)$ 
    the \FunctionImage
    of $A$ under $f$
    and we call
    $f^{-1}\pa{B}$ 
    the 
    \FunctionPreimage
    of $B$ under
    $f$. 
    we call $f(X)$ 
    the \FunctionRange
    of $f$. 
\end{df}

\label{def:InsertionFunction}
\newcommand{\InsertionFunction}[0]{
    \textbf{\hyperref[def:InsertionFunction]{Insertion Function}}
}
\newcommand{\InsertionFunctions}[0]{
    \textbf{\hyperref[def:InsertionFunction]{Insertion Functions}}
}
\begin{df}[\InsertionFunction]
    Let $A \subset B$ and define 
    $f:A \to B$ by $f(x)=x$. 
    The we call $f$ the 
    \InsertionFunction of $A$ into $B$. 
\end{df}


\newcommand{\Restriction}[0]{\textbf{\hyperref[def:FunctionRestriction]{Restriction}}\xspace}
\newcommand{\scRestriction}[2]{\ensuremath{\pa{#1}|_{#2}}\xspace}
\begin{df}[Restriction]
\label{def:FunctionRestriction}

\rm
    Let $X,Y$ be sets and 
    let $R$ be a 
    \Relation from 
    $X$ to $Y$. 
    Let $A \subset X$. 
    We define 
    \begin{equation*}
        \scRestriction{R}{A} = \{(x,y)\in R| x \in A \}
    \end{equation*}
    We call 
    \scRestriction{R}{A}
    the 
    \Restriction 
    of the \Relation
    $R$
    to the set 
    $A$. 
\end{df}

\label{def:RelationInverse}
\newcommand{\RelationInverse}[0]{
    \textbf{\hyperref[def:RelationInverse]{Inverse}}
}
\begin{df}[\RelationInverse]
    Let $X\neq \emptyset$ and
    $Y \neq \emptyset$. 
    Let $R$ be a 
    \Relation
    from $X$ to $Y$. 
    We define 
    \begin{equation*}
    R^{-1} = \{(y,x) \in Y\times X | (x,y) \in R\}
    \end{equation*}
    We call $R^{-1}$ the 
    \RelationInverse
    of $R$. 
\end{df}

\label{def:FunctionExtension}
\newcommand{\Extension}[0]{
    \textbf{\hyperref[def:FunctionExtension]{Extension}}
}
\newcommand{\Extensions}[0]{
    \textbf{\hyperref[def:FunctionExtension]{Extensions}}
}
\begin{df}[\Extension]
    Let $X,Y$ be sets and 
    let $f,g:X \to Y$.
    Let $f$ be a \Restriction
    of $g$. 
    Then we call $g$ an 
    \Extension of $f$. 
\end{df}

\label{def:Injective}
\newcommand{\Injective}[0]{
    \bf \hyperref[def:Injective]{Injective} \rm
}
\newcommand{\Injectivity}[0]{
    \bf \hyperref[def:Injective]{Injectiveness} \rm
}
\newcommand{\Injection}[0]{
    \bf \hyperref[def:Injective]{Injection} \rm
}
\newcommand{\Injections}[0]{
    \bf \hyperref[def:Injective]{Injections} \rm
}

\begin{df}[\Injective]
    Let $X,Y$ be sets and let 
    $f:X \to Y$. 
    We say that $f$ is 
    an 
    \Injection, 
    or that $f$ is 
    \Injective if 
    for all $x,y \in X$, 
    if $x \neq y$, then 
    $f(x) \neq f(y)$. 
\end{df}
    

\label{def:Surjective}
\newcommand{\Surjective}[0]{
    \bf \hyperref[def:Surjective]{Surjective} \rm
}
\newcommand{\Surjectivity}[0]{
    \bf \hyperref[def:Surjective]{Surjectivity} \rm
}
\newcommand{\Surjection}[0]{
    \bf \hyperref[def:Surjective]{Surjection} \rm
}
\newcommand{\Surjections}[0]{
    \bf \hyperref[def:Surjective]{Surjections} \rm
}
\begin{df}[\Surjective]
   Let $X,Y$ be sets and let 
   $f:X \to Y$. 
   Suppose that 
   for each $y \in Y$, 
   there exists an 
   $x \in X$ such that 
   $f(x) = y$. 
   Then we say that $f$ 
   is a
   \Surjection onto $Y$, 
   and we call $f$ 
   \Surjective
   onto $Y$. 
   When $Y$ is understood and the risk of 
   misunderstanding is minimal, 
   we may omit saying onto $Y$.
\end{df}

\newcommand{\Bijective}[0]{\textbf{\hyperref[def:Bijective]{Bijective}}\xspace}
\newcommand{\Bijectivity}[0]{\textbf{\hyperref[def:Bijective]{Bijectivity}}\xspace}
\newcommand{\Bijection}[0]{\textbf{\hyperref[def:Bijective]{Bijection}}\xspace}
\newcommand{\Bijections}[0]{\textbf{\hyperref[def:Bijective]{Bijections}}\xspace}
\begin{df}[Bijective]
\label{def:Bijective}

\rm
    Let $X$ and $Y$ be sets and let 
    $f:X \to Y$ be \Surjective and \Injective. 
    Then we say that $f$ is 
    \Bijective, or we say that f is a 
    \Bijection. 
\end{df}

\label{def:Cardinality}
\newcommand{\Cardinal}[0]{
    \bf \hyperref[def:Cardinality]{Cardinal} \rm
}
\newcommand{\Cardinals}[0]{
    \bf \hyperref[def:Cardinality]{Cardinals} \rm
}
\newcommand{\Cardinality}[0]{
    \bf \hyperref[def:Cardinality]{Cardinality} \rm
}
\newcommand{\Cardinalities}[0]{
    \bf \hyperref[def:Cardinality]{Cardinalities} \rm
}
\newcommand{\FirstNaturals}[1]{
    N_{#1}
}
\newcommand{\CardinalityFunction}[1]{
    \bf
    Card
    \rm
    \pa{#1}
}
\newcommand{\Finite}[0]{
    \bf \hyperref[def:Cardinality]{Finitje} \rm
}
\newcommand{\Infinite}[0]{
    \bf \hyperref[def:Cardinality]{Infinitje} \rm
}
\newcommand{\Denumerable}[0]{
    \bf \hyperref[def:Cardinality]{Denumerable} \rm
}
\newcommand{\Countable}[0]{
    \bf \hyperref[def:Cardinality]{Countable} \rm
}
\newcommand{\Uncountable}[0]{
    \bf \hyperref[def:Cardinality]{Uncountable} \rm
}
\begin{df}[\Cardinality]
    Let $n \in \N$. We define 
    \begin{equation*}
        \FirstNaturals{n}=\{ k \in \N | k \leq n\}
    \end{equation*}
    Let $X$ be a set.
    Let $f:X \to \FirstNaturals{n}$ 
    be a 
    \Bijection. 
    Then, we say that 
    $X$ has 
    \Cardinality
    $n$
    and we write 
    $\CardinalityFunction{X}=n$.
    More generaly, if there exists a 
    \Bijection
    between two sets 
    $Y$ and $Z$, then we write
    $\CardinalityFunction{Y}=\CardinalityFunction{Z}$
    and we say that they have the same 
    \Cardinalities. 
    Define
    $X_0=\N$
    and for $k \in \N$, define 
    $X_{k+1} = \scPowerSet{X_k}$. 
    Then for $k \in \N$, we define 
    $\aleph_k = \CardinalityFunction{X_k}$.
    If $\CardinalityFunction{X} \in \N$, then 
    we say that $X$ is \Finite. 
    If $\CardinalityFunction{Z} \in \N$ or 
    $\CardinalityFunction{Z} = \aleph_0$, 
    then we say that $Z$ is \Denumerable.
    If $\CardinalityFunction{Y} = \aleph_0$, then
    we say that $Y$ is \Countable.
    If $\CardinalityFunction{W}= \alpha_k$ for $k \geq 1$, 
    then we say that $W$ is \Uncountable. 
    If $\CardinalityFunction{V} = \alpha_j$ for $j \in \N$, 
    then we say that $V$ is \Infinite. 


\end{df}




\newcommand{\Disjoint}[0]{
    \bf \hyperref[def:Disjoint]{Disjoint} \rm
}
\newcommand{\Disjointedness}[0]{
    \bf \hyperref[def:Disjoint]{Disjointedness} \rm
}\begin{df}[Disjoint]
\label{def:Disjoint}

\rm
    Let $X$ and $Y$ be sets such that 
    $X \cap Y = \emptyset$. 
    Then we say that $X$ and $Y$ are 
    \Disjoint. 
    Let $F=\{X_{\alpha}\}_{\alpha \in A}$ 
    such that for each $\alpha, \beta \in A$ 
    with $\alpha \neq \beta$, we have 
    $X_\alpha$ 
    is \Disjoint
    to 
    $X_{\beta}$. 
    Then we say that $F$ is \Disjoint. 
\end{df}

\newcommand{\Cover}[0]{\textbf{\hyperref[def:Cover]{Cover}}\xspace}
\newcommand{\Covers}[0]{\textbf{\hyperref[def:Cover]{Covers}}\xspace}
\newcommand{\Subcover}[0]{\textbf{\hyperref[def:Cover]{Subcover}}\xspace}
\newcommand{\Subcovers}[0]{\textbf{\hyperref[def:Cover]{Subcovers}}\xspace}
\begin{df}[Cover, Subcover]
\label{def:Cover}
\rm
    Let $X$ be a set and let 
    $Y=\{Y_\alpha\}_{\alpha \in A}$ 
    such that 
    \begin{equation*}
        X \subset \bigcup_{\alpha \in A} Y_{\alpha}
    \end{equation*}
    Then we say that 
    $Y$ 
    is a 
    \Cover
    for $X$ 
    or that $Y$ 
    \Covers $X$. 
    In the context of talking about a 
    \Cover, if every member of a 
    \Cover posses a certain property
    then we may say that the \Cover 
    has that property. 
    If $Z \subset Y$ \Covers $X$, then
    we call $Z$ a \Subcover of $Y$. 
    One exception to this is that 
    when talking about the 
    \Cardinality
    or \Disjointedness 
    of a \Cover, we are 
    talking about \Cover itself, 
    not each of its constituent sets. 
\end{df}

\label{def:Partition}
\newcommand{\Partition}[0]{
    \bf \hyperref[def:Partition]{Partition} \rm
}
\newcommand{\Partitions}[0]{
    \bf \hyperref[def:Partition]{Partitions} \rm
}
\begin{df}[\Partition]
    Let $X$ be a set and 
    $Y \subset \scPowerSet{X}$ 
    be a \Disjoint \Cover for $X$. 
    Then we call $Y$ a \Partition
    for $X$. 
\end{df}

\label{def:InfiniteCartesianProduct}
\newcommand{\InfiniteCartesianProduct}[0]{
    \textbf{\hyperref[def:InfiniteCartesianProduct]{Cartesian Product}}
}
\newcommand{\InfiniteCartesianProducts}[0]{
    \textbf{\hyperref[def:InfiniteCartesianProduct]{Cartesian Products}}
}
\newcommand{\scCartesianProduct}[3]{
    \ensuremath{\prod\limits_{#1 \in #2}#3_{#1}}
}
\newcommand{\ProjectionMap}[0]{
    \textbf{\hyperref[def:InfiniteCartesianProduct]{Projection Map}}
}

\begin{df}[\InfiniteCartesianProduct]
    Let $A \neq \emptyset$. 
    For each $\alpha \in A$, let 
    $X_{\alpha} \neq \emptyset$. 
    Define 
    \begin{equation*}
        \prod\limits_{\alpha \in A} X_{\alpha } = \left\{f:A \to \bigcup\limits_{\alpha \in A} X_{\alpha} | (\forall \alpha \in A)(f(\alpha) \in X_{\alpha} ) \right\}
    \end{equation*}
    We call this the 
    \InfiniteCartesianProduct
    of $\{X_\alpha\}_{\alpha \in A}$. 
    For each $\alpha \in A$, we define 
    \begin{equation*}
        \pi_\alpha : \scCartesianProduct{\alpha}{A}{X} \to X_{\alpha }\tab[2cm] \pi_\alpha(f) = f(\alpha)
    \end{equation*}
    We call $\pi_\alpha$ the
    $\alpha-$\ProjectionMap.
\end{df}


\label{def:ClosureUnderUnion}
\newcommand{\ClosureUnderFiniteUnions}[0]{
    \bf \hyperref[def:ClosureUnderUnion]{Closure Under Finite Unions} \rm
}
\newcommand{\ClosureUnderCountableUnions}[0]{
    \bf \hyperref[def:ClosureUnderUnion]{Closure Under Countable Unions} \rm
}
\newcommand{\ClosureUnderUnions}[0]{
    \bf \hyperref[def:ClosureUnderUnion]{Closure Under Unions} \rm
}
\newcommand{\ClosureUnderArbitraryUnions}[0]{
    \bf \hyperref[def:ClosureUnderUnion]{Closure Under Aribtrary Unions} \rm
}
\newcommand{\ClosedUnderFiniteUnions}[0]{
    \bf \hyperref[def:ClosureUnderUnion]{Closed Under Finite Unions} \rm
}
\newcommand{\ClosedUnderCountableUnions}[0]{
    \bf \hyperref[def:ClosureUnderUnion]{Closed Under Countable Unions} \rm
}
\newcommand{\ClosedUnderUnions}[0]{
    \bf \hyperref[def:ClosureUnderUnion]{Closed Under Unions} \rm
}
\newcommand{\ClosedUnderArbitraryUnions}[0]{
    \bf \hyperref[def:ClosureUnderUnion]{Closed Under Aribtrary Unions} \rm
}

\begin{df}[\ClosureUnderUnions]
    Let $S$ be a set such that 
    \begin{equation*}
    \{ S_\alpha | \alpha \in A\} \subset S \implies \bigcup_{\alpha \in A} S_{\alpha}   \in S
    \end{equation*}
    for all index sets $A$. 
    Then we say that $S$ is 
    \ClosedUnderUnions
    or
    \ClosedUnderArbitraryUnions
    and that $S$ posesses 
    \ClosureUnderUnions
    or
    \ClosureUnderArbitraryUnions.
    If this relation only holds when A is a 
    \Countable set then we say that 
    $S$ is 
    \ClosedUnderCountableUnions
    and that $S$ posesses
    \ClosureUnderCountableUnions. 
    If this relation only holds when A is a 
    \Finite set then we say that 
    $S$ is 
    \ClosedUnderFiniteUnions
    and that $S$ posesses
    \ClosureUnderFiniteUnions    
\end{df}

\newcommand{\ClosureUnderFiniteIntersections}[0]{\textbf{\hyperref[def:ClosureUnderIntersection]{Closure Under Finite Intersections}}\xspace}
\newcommand{\ClosureUnderCountableIntersections}[0]{\textbf{\hyperref[def:ClosureUnderIntersection]{Closure Under Countable Intersections}}\xspace}
\newcommand{\ClosureUnderIntersections}[0]{\textbf{\hyperref[def:ClosureUnderIntersection]{Closure Under Intersections}}\xspace}
\newcommand{\ClosureUnderArbitraryIntersections}[0]{\textbf{\hyperref[def:ClosureUnderIntersection]{Closure Under Aribtrary Intersections}}\xspace}
\newcommand{\ClosedUnderFiniteIntersections}[0]{\textbf{\hyperref[def:ClosureUnderIntersection]{Closed Under Finite Intersections}}\xspace}
\newcommand{\ClosedUnderCountableIntersections}[0]{\textbf{\hyperref[def:ClosureUnderIntersection]{Closed Under Countable Intersections}}\xspace}
\newcommand{\ClosedUnderIntersections}[0]{\textbf{\hyperref[def:ClosureUnderIntersection]{Closed Under Intersections}}\xspace}
\newcommand{\ClosedUnderArbitraryIntersections}[0]{\textbf{\hyperref[def:ClosureUnderIntersection]{Closed Under Aribtrary Intersections}}\xspace}
\begin{df}[Closure Under Intersections]
\label{def:ClosureUnderIntersection}
\rm
    Let $S$ be a set such that 
    \begin{equation*}
    \{ S_\alpha | \alpha \in A\} \subset S \implies \bigcap_{\alpha \in A} S_{\alpha}   \in S
    \end{equation*}
    for all index sets $A$. 
    Then we say that $S$ is 
    \ClosedUnderIntersections
    or
    \ClosedUnderArbitraryIntersections
    and that $S$ posesses 
    \ClosureUnderIntersections
    or
    \ClosureUnderArbitraryIntersections.
    If this relation only holds when A is a 
    \Countable set then we say that 
    $S$ is 
    \ClosedUnderCountableIntersections
    and that $S$ posesses
    \ClosureUnderCountableIntersections. 
    If this relation only holds when A is a 
    \Finite set then we say that 
    $S$ is 
    \ClosedUnderFiniteIntersections
    and that $S$ posesses
    \ClosureUnderFiniteIntersections    
\end{df}

\begin{prop}
\label{prop:FiniteClosure}
\rm
    Let Let $X$ be a set. 
    The following are true. 
    \begin{enumerate}[label=(\roman*), ref={\ref{prop:FiniteClosure}.~\roman*}]
        \item \label{prop:FiniteClosure:Intersection}If $X$ has the property that $\{y,z\} \subset X \implies y \cap z \in X$, then $X$ 
        posesses \ClosureUnderFiniteIntersections. 
        \item \label{prop:FiniteClosure:Union}If $X$ has the property that $\{y,z\} \subset X \implies y \cup z \in X$, then $X$ posesses 
        \ClosureUnderFiniteUnions.
    \end{enumerate}
    \begin{proof}[Proof of \ref{prop:FiniteClosure:Intersection}]
        We use induction.
        Let $M$ be the set of natural numbers for which
        $X$ is closued under intersections of n sets. 
        The intersection of a single set equals that set, so $1 \in M$. 
        $2 \in M$ by direct application of the assumption of 
        \ref{prop:FiniteClosure:Intersection}. 
        Let $m \in M$. Let $\{x_i\}_{i=1}^{m+1} \subset M$. 
        Then 
        \begin{equation*}
            \bigcap\limits_{i=1}^{m+1} x_i = \pa{ \bigcap\limits_{i=1}^m x_i} \cap x_{m+1} \in X
        \end{equation*}
        so $m+1 \in M$. 
        Hence $M= \N$ and \ref{prop:FiniteClosure:Intersection} is proven. 
    \end{proof}
    \begin{proof}[Proof of \ref{prop:FiniteClosure:Union}]
         We use induction.
        Let $M$ be the set of natural numbers for which
        $X$ is closued under unions of n sets. 
        The union of a single set equals that set, so $1 \in M$. 
        $2 \in M$ by direct application of the assumption of 
        \ref{prop:FiniteClosure:Union}. 
        Let $m \in M$. Let $\{x_i\}_{i=1}^{m+1} \subset M$. 
        Then 
        \begin{equation*}
            \bigcup\limits_{i=1}^{m+1} x_i = \pa{ \bigcup\limits_{i=1}^m x_i} \cup x_{m+1} \in X
        \end{equation*}
        so $m+1 \in M$. 
        Hence $M= \N$ and \ref{prop:FiniteClosure:Union} is proven. 
    \end{proof}
\end{prop}



\subsection{Relations}
\label{def:ReflexiveRelation}
\newcommand{\ReflexiveRelation}[0]{
    \bf \hyperref[def:ReflexiveRelation]{Reflexive} \rm
}

\newcommand{\RelationReflexivity}[0]{
    \bf \hyperref[def:ReflexiveRelation]{Reflexivity} \rm
}

\begin{df}[\ReflexiveRelation]
    Let $X \neq \emptyset$ be a set. 
    Let $R$ be a \Relation on X. 
    We say that $R$ is \ReflexiveRelation with respect to X if, 
    or equivalently we say that
    $R$ posseses 
    \RelationReflexivity with respect to X
    if 
    $\{(a,a) | a \in X \} \subset R$.
    When X is understood, we may simply say that 
    $R$ is \ReflexiveRelation or that $R$
    posesses \RelationReflexivity. 
\end{df}
\label{def:TransitiveRelation}
\newcommand{\TransitiveRelation}[0]{
    \bf \hyperref[def:TransitiveRelation]{Transitive} \rm
}

\newcommand{\RelationTransitivity}[0]{
    \bf \hyperref[def:TransitiveRelation]{Transitivity} \rm
}

\begin{df}[\TransitiveRelation]
    Let $X \neq \emptyset$ be a set. 
    Let $R$ be a \Relation on X. 
    We say that $R$ is \TransitiveRelation, 
    or equivalently we say that
    $R$ posseses 
    \RelationTransitivity
    if whenever $(a,b) \in R$ and $(b,c) \in R$, 
    we also have $(a,c) \in R$. 
\end{df}
\label{def:Preorder}
\newcommand{\Preordering}[0]{\textbf{\hyperref[def:Preorder]{Preordering}}\xspace}
\newcommand{\Preorder}[0]{\textbf{\hyperref[def:Preorder]{Preorder}}\xspace}
\newcommand{\PreorderedSet}[0]{\textbf{\hyperref[def:Preorder]{Preordered Set}}\xspace}

\begin{df}[\Preorder]
    Let $X \neq \emptyset$ be a set. 
    Let $R$ be a \Relation on $X$. 
    If $R$ is
    \ReflexiveRelation
    and
    \TransitiveRelation
    then we call $R$
    a \Preorder on $X$, 
    or we equivalently call
    $R$ a \Preordering 
    of $X$ and we call 
    $(X,R)$ a \PreorderedSet.
    \end{df}

\newcommand{\Comparable}[0]{\textbf{\hyperref[def:Comparable]{Comparable}}\xspace}
\newcommand{\Comparability}[0]{\textbf{\hyperref[def:Comparable]{Comparability}}\xspace}

\begin{df}[\Comparable]
\label{def:Comparable}
\rm
   Let $(X,R)$ be a 
   \PreorderedSet. 
   We say that $x,y \in X$ are 
   \Comparable and that
   they possess 
   \Comparability
   if 
   $xRy$ or $yRx$. 
\end{df}

\label{def:SymmetricRelation}
\newcommand{\SymmetricRelation}[0]{\textbf{\hyperref[def:SymmetricRelation]{Symmetric}}\xspace}
\newcommand{\RelationSymmetry}[0]{\textbf{\hyperref[def:SymmetricRelation]{Symmetry}}\xspace}

\begin{df}[\SymmetricRelation]
    Let $X \neq \emptyset$ be a set. 
    Let $R$ be a \Relation on X. 
    We say that $R$ is \SymmetricRelation, 
    or equivalently we say that
    $R$ posseses 
    \RelationSymmetry
    if whenever $aRb$, we also have $bRa$. 
    This is equivalent to the condition 
    $R=R^{-1}$. 
\end{df}

\label{def:AntiSymmetricRelation}
\newcommand{\AntiSymmetricRelation}[0]{
    \bf \hyperref[def:AntiSymmetricRelation]{Anti-Symmetric} \rm
}

\newcommand{\RelationAntiSymmetry}[0]{
    \bf \hyperref[def:SymmetricRelation]{Anti-Symmetry} \rm
}

\begin{df}[\AntiSymmetricRelation]
    Let $X \neq \emptyset$ be a set. 
    Let $R$ be a \Relation on X. 
    We say that $R$ is \AntiSymmetricRelation, 
    or equivalently we say that
    $R$ posseses 
    \RelationAntiSymmetry
    if whenever $aRb$ and $bRa$, we 
    must have $a = b$.
\end{df}
\label{def:MaximalElement}
\newcommand{\MaximalElement}[0]{
    \bf \hyperref[def:MaximalElement]{Maximal Element} \rm
}

\newcommand{\Maximum}[0]{
    \bf \hyperref[def:MaximalElement]{Maximum} \rm
}

\begin{df}[Upper Bound]
    Let $X \neq \emptyset$ be a set. 
    Let $R$ be a \Relation on X. 
    Let $Y \subset X$.
    Let $a \in Y$. 
    We say that $a$ is a 
    \MaximalElement of $Y$, 
    or equivalently we say that 
    $a$ is a \Maximum of Y 
    if for every $b \in Y$, 
    we have $b \leq a$. 
    
    If we further assume that 
    Y has exactly 1 \MaximalElement 
    then we write 
    $a=max(Y)$. 
\end{df}
\label{def:MinimalElement}
\newcommand{\MinimalElement}[0]{
    \bf \hyperref[def:MinimalElement]{Minimal Element} \rm
}

\newcommand{\Minimum}[0]{
    \bf \hyperref[def:MinimalElement]{Minimum} \rm
}

\newcommand{\Minima}[0]{
    \bf \hyperref[def:MinimalElement]{Minima} \rm
}

\begin{df}[\MinimalElement]
    Let $X \neq \emptyset$ be a set. 
    Let $R$ be an 
    \Relation on $X$. 
    Let $Y \subset X$.
    Let $a \in Y$. 
    We say that $a$ is a 
    \MinimalElement of $Y$, 
    or equivalently we say that 
    $a$ is a \Minimum of Y 
    if for every $b \in Y$,
	if $b R a$, then 
    we have $a = b$. 
	The Plural of \Minimum is \Minima, 
	and we represent the set of \Minima of Y with 
	respect to the relation $R$ with 
	$\Minima_R(Y)$, or if $R$ is understood, 
	we represent the set of $\Minima$ of Y with 
	$\Minima(Y)$. 
	
	
\end{df}

\begin{prop}[\MinimalElement unique if R is \AntiSymmetricRelation]
	Let $X \neq \emptyset$ be a set. 
	Let $R$ be an
	\AntiSymmetricRelation
	\Relation
	on X. 
	Let $Y \subset X$.
	Let $a$ and $b$ be 
	each be a \MinimalElement
	of Y. 
	Then $a=b$. 
	\begin{proof}
		Since $a \in \Minima(Y)$, 
		$a \leq b$. 
		Since $b \in \Minima(Y)$, 
		$b \leq a$. 
		By \RelationAntiSymmetry, 
		$b = a$. 
	\end{proof}
\end{prop}
\label{def:UpperBound}
\newcommand{\UpperBound}[0]{
    \bf \hyperref[def:UpperBound]{Upper Bound} \rm
}

\newcommand{\UpperBounds}[0]{
    \bf \hyperref[def:UpperBound]{Upper Bounds} \rm
}

\newcommand{\BoundedFromAbove}[0]{
    \bf \hyperref[def:UpperBound]{Bounded From Above} \rm
}

\newcommand{\UB}[0]{
	\bf \hyperref[def:UpperBound]{UpperBound} \rm
}

\begin{df}[\UpperBound]
    Let $X \neq \emptyset$ be a set. 
    Let $R$ be a \Relation on X. 
    Let $Y \subset X$.
    Let $a \in X$. 
    We say that $a$ is an 
    \UpperBound for $Y$ if
    for every $x \in Y$, 
    we have $x R a$. 
    If $a$ is an \UpperBound
    then we also say that 
    the set Y is \BoundedFromAbove
    by a. 
	We denote the set of \UpperBounds of 
	$Y$ with respect to the relation $R$ with
	$\UB_R(Y)$. 
	When $R$ is understood, we denote this set with
	$\UB(Y)$. 
\end{df}
\label{def:LowerBound}
\newcommand{\LowerBound}[0]{
    \bf \hyperref[def:LowerBound]{Lower Bound} \rm
}

\newcommand{\BoundedFromBelow}[0]{
    \bf \hyperref[def:LowerBound]{Bounded From Below} \rm
}
\newcommand{\LowerBounds}[0]{
	\bf \hyperref[def:LowerBound]{Lower Bounds} \rm
}
\newcommand{\LB}[0]{
	\bf \hyperref[def:LowerBound]{LowerBound} \rm
}

\begin{df}[\LowerBound]
    Let $X \neq \emptyset$ be a set. 
    Let $R$ be a \Relation on X. 
    Let $Y \subset X$.
    Let $a \in X$. 
    We say that $a$ is an 
    \LowerBound for $Y$ if
    for every $x \in Y$, 
    we have $a R x$. 
    If $a$ is an \LowerBound
    then we also say that 
    the set Y is \BoundedFromBelow
    by a. 
	We denote the set of \LowerBounds of 
	$Y$ with respect to the relation $R$ with
	$\LB_R(Y)$. 
	When $R$ is understood, we denote this set with
	$\LB(Y)$. 
\end{df}
\newcommand{\LeastUpperBound}[0]{\textbf{\hyperref[def:LeastUpperBound]{Least Upper Bound}}\xspace}
\newcommand{\LeastUpperBounds}[0]{\textbf{\hyperref[def:LeastUpperBound]{Least Upper Bounds}}\xspace}
\newcommand{\Sup}[0]{\textbf{\hyperref[def:LeastUpperBound]{Sup}}\xspace}
\newcommand{\Supremum}[0]{\textbf{\hyperref[def:LeastUpperBound]{Supremum}}\xspace}
\newcommand{\Suprema}[0]{\textbf{\hyperref[def:LeastUpperBound]{Suprema}}\xspace}
\newcommand{\LUB}[0]{\textbf{\hyperref[def:LeastUpperBound]{LUB}}\xspace}
\begin{df}[\LeastUpperBound]
\label{def:LeastUpperBound}
\rm
    Let $X \neq \emptyset$ be a set. 
    Let $R$ be a \Relation on X. 
    Let $Y \subset X$.
	Let $a \in X$. 
	We say that $a$ is a
	\LeastUpperBound of $Y$ if 
	$a \in \Minima(\UB(Y))$.
	We denote the set of \LeastUpperBounds
	for $Y$ with $\LUB(Y)$.
	If $b \in \LUB(Y)$, then we 
	also call $b$ a 
	\Supremum of $Y$. 
	The Plural of \Supremum is \Suprema.
	If $\LUB(Y)=\{c\}$, 
	then we write $c=\Sup(Y)$. 
\end{df}

\label{def:GreatestLowerBound}

\newcommand{\GreatestLowerBound}[0]{
    \bf \hyperref[def:GreatestLowerBound]{Greatest Lower Bound} \rm
}

\newcommand{\GreatestLowerBounds}[0]{
    \bf \hyperref[def:GreatestLowerBound]{Greatest Lower Bounds} \rm
}

\newcommand{\Inf}[0]{
    \bf \hyperref[def:GreatestLowerBound]{Inf} \rm
}

\newcommand{\Infimum}[0]{
    \bf \hyperref[def:GreatestLowerBound]{Infimum} \rm
}

\newcommand{\Infima}[0]{
    \bf \hyperref[def:GreatestLowerBound]{Infima} \rm
}

\newcommand{\GLB}[0]{
	\bf \hyperref[def:GreatestLowerBound]{GLB} \rm
}

\begin{df}[\GreatestLowerBound]
    Let $X \neq \emptyset$ be a set. 
    Let $R$ be a \Relation on X. 
    Let $Y \subset X$.
	Let $a \in X$. 
	We say that $a$ is a
	\GreatestLowerBound of $Y$ if 
	$a \in \Maxima(\LB(Y))$.
	We denote the set of \GreatestLowerBounds
	for $Y$ with $\GLB(Y)$.
	If $b \in \GLB(Y)$, then we 
	also call $b$ a 
	\Infimum of $Y$. 
	The Plural of \Infimum is \Infima. 
	If $\GLB(Y)=\{c\}$, then 
	we write $c=\Inf(Y)$.  
\end{df}
\label{def:EquivalenceRelation}
\newcommand{\EquivalenceRelation}[0]{
    \bf \hyperref[def:EquivalenceRelation]{Equivalence Relation} \rm
}

\begin{df}[\EquivalenceRelation]
    Let $X \neq \emptyset$ be a set.
    Let $\cong$ be a \Preorder on X. 
    We say that $\cong$ is an \EquivalenceRelation on X
    if it is \SymmetricRelation.
\end{df}

\label{def:PartialOrder}
\newcommand{\PartialOrder}[0]{
    \bf \hyperref[def:PartialOrder]{Partial Order} \rm
}

\newcommand{\PartialOrdering}[0]{
    \bf \hyperref[def:PartialOrder]{Partial Ordering} \rm
}

\newcommand{\Poset}[0]{
    \bf \hyperref[def:PartialOrder]{Partially Ordered Set} \rm
}

\begin{df}[\PartialOrder]
    Let $X \neq \emptyset$ be a set.
    Let $\leq$ be a \Preorder on X. 
    We say that $\leq$ is a \PartialOrder on $X$
    and we say that $\leq$ is a \PartialOrdering of $X$ 
    if $\leq$ is \AntiSymmetricRelation.
    Let $\leq$ is a \PartialOrder on X, the we 
    refer to the pair $(X,\leq)$ as a 
    \Poset.    
\end{df}
\label{def:TotalOrder}
\newcommand{\TotalOrder}[0]{\textbf{\hyperref[def:TotalOrder]{Total Order}}\xspace}
\newcommand{\TotalOrdering}[0]{\textbf{\hyperref[def:TotalOrder]{Total Ordering}}\xspace}

\newcommand{\Toset}[0]{\textbf{\hyperref[def:TotalOrder]{Totally Ordered Set}}\xspace}
\begin{df}[\TotalOrder]
    Let $(X,R)$ be a 
    \Poset in which
    every pair of elements is 
    \Comparable. 
    Then we call 
    $R$ a 
    \TotalOrder
    on $X$
    and we call 
    $(X,R)$ a 
    \Toset.
\end{df}

\label{def:Chain}
\newcommand{\Chain}[0]{
    \textbf{\hyperref[def:Chain]{Chain}}
}
\newcommand{\Chains}[0]{
    \textbf{\hyperref[def:Chain]{Chains}}
}
\begin{df}[\Chain]
    Let $(X,\leq)$ be a \Poset. 
    Let $A \subset X$ such that
    $\pa{A, \leq \cap \pa{A \times A}}$ is a 
    \Toset. 
    Then we call $A$ a \Chain in $X$. 
\end{df}

\label{def:Direction}
\newcommand{\Direction}[0]{\textbf{\hyperref[def:Direction]{Direction}}\xspace}
\newcommand{\Directions}[0]{\textbf{\hyperref[def:Direction]{Directions}}\xspace}
\newcommand{\Directing}[0]{\textbf{\hyperref[def:Direction]{Directing}}\xspace}
\newcommand{\Directings}[0]{\textbf{\hyperref[def:Direction]{Directings}}\xspace}
\newcommand{\DirectedSet}[0]{\textbf{\hyperref[def:Direction]{Directed Set}}\xspace}
\newcommand{\DirectedSets}[0]{\textbf{\hyperref[def:Direction]{Directed Sets}}\xspace}

\begin{df}[\Direction]
    Let $X \neq \emptyset$ be a set.
    Let $\leq$ be a \Preorder on X. 
	If every pair of elements in $X$ has an 
	\UpperBound with respect to $\leq$, then
    we call $\leq$ is a \Direction on $X$, 
	, we call $\leq$ is a \Directing of $X$
	, and we call $(X,\leq)$ is a \DirectedSet.
\end{df}

\label{def:Lattice}
\newcommand{\Lattice}[0]{
    \textbf{\hyperref[def:Lattice]{Lattice}}
}
\newcommand{\Lattices}[0]{
    \textbf{\hyperref[def:Lattice]{Lattices}}
}
\newcommand{\CompleteLattice}[0]{
    \textbf{\hyperref[def:Lattice]{Complete Lattice}}
}
\newcommand{\CompleteLattices}[0]{
    \textbf{\hyperref[def:Lattice]{Complete Lattices}}
}
\newcommand{\LatticeJoin}[0]{
    \textbf{\hyperref[def:Lattice]{Join}}
}
\newcommand{\LatticeJoins}[0]{
    \textbf{\hyperref[def:Lattice]{Joins}}
}    
\newcommand{\LatticeMeet}[0]{
    \textbf{\hyperref[def:Lattice]{Meet}}
}
\newcommand{\LatticeMeets}[0]{
    \textbf{\hyperref[def:Lattice]{Meets}}
}    
\begin{df}[\Lattice, \LatticeJoin, \LatticeMeet]
    Let $(X, \leq)$ be a 
    \Poset
    such that, 
    for every $x,y \in X$, 
    the set 
    $\{x,y\}$ has both a 
    \Supremum 
    and an
    \Infimum. 
    Then we call $(X,\leq)$ 
    \Lattice. 
    Furthermore, we call 
    $\Sup\{x,y\}$
    the 
    \LatticeJoin of $x$ and $y$ 
    and we call
    $\Inf\{x,y\}$ the 
    \LatticeMeet
    of $x$ and $y$. 
    If every nonempty subset of 
    $X$ has both a 
    \Supremum
    and \Infimum 
    then we call $(X,\leq)$
    a \CompleteLattice. 
\end{df}

\newcommand{\Sequence}[0]{\textbf{\hyperref[def:Sequence]{Sequence}}\xspace}
\newcommand{\Sequences}[0]{\textbf{\hyperref[def:Sequence]{Sequences}}\xspace}
\begin{df}[\Sequence]
\label{def:Sequence}
\rm
    Let $X$ be a set. 
    A \Sequence in $X$ is a 
    \Function $f:\N \to X$. 
    If $f$ is a \Sequence 
    in $X$ and 
    $f(n) = x_n$ for $n \in \N$, then 
    we may refer to $\{x_n\}_{n \in \N}$ as the \Sequence itself. 
\end{df}

\label{def:Net}
\newcommand{\Net}[0]{
    \textbf{\hyperref[def:Net]{Net}}
}
\newcommand{\Nets}[0]{
    \textbf{\hyperref[def:Net]{Nets}}
}
\begin{df}[\Net]
    A \Net is a \Function 
    mapping from a directed set $(A, \leq)$
    into another set $X$. 
    If $f:A \to X$ is a \Net
    such that for $\alpha \in A$ we have
    $f(\alpha) = x_{\alpha}$, then we may 
    use the notation
    $\{x_\alpha\}_{\alpha \in A} \subset X$. 
\end{df}

\label{Axiom:ZornsLemma}
\begin{thm}[Zorns Lemma]
Let $(X,\leq)$. be a 
\Poset. If every 
\Chain in $X$ has an 
\UpperBound, then $\Maxima(X) \neq \emptyset$ 
\end{thm}
\begin{rmk}\ref{Axiom:ZornsLemma} Is equivalent to the axiom of choice
\end{rmk}


\subsection{Filters}
\newcommand{\Filter}[0]{\textbf{\hyperref[def:Filter]{Filter}}\xspace}
\newcommand{\Filters}[0]{\textbf{\hyperref[def:Filter]{Filters}}\xspace}
\begin{df}[\Filter]
\label{def:Filter}
\rm
    Let $X \neq \emptyset$. 
    Let $\scF \subset \scPowerSet{X}$ 
    satisfy the following.
    \begin{enumerate}[label=(\roman*), ref={\ref{def:Filter}.~\roman*}]
        \item 
		\label{def:Filter:IsNonempty} 
		$\scF \neq \emptyset$. 
        \item 
		\label{def:Filter:DoesntContainEmpty} 
		$\emptyset \not \in \scF$
        \item 
		\label{def:Filter:SubsetProperty} 
		If $G_1 \in \scF$ and $G_1 \subset G_2 \subset X$
		, then $G_2 \in \scF$. 
        \item 
		\label{def:Filter:FiniteIntersectionProperty} 
		If $\{G_1,G_2\} \subset \scF$, then $G_1 \cap G_2 \in \scF$. 
    \end{enumerate}
    Then we call $\scF$ 
    a \Filter
    on $X$. 
\end{df}

\begin{prop}
\label{prop:FilterFacts}
    Let $X \neq \emptyset$ and let $\scF$ be a 
    \Filter on $X$. The following are true. 
    \begin{enumerate}[label=(\roman*), ref={\ref{prop:FilterFacts}~\roman*}]
        \item \label{prop:FilterFacts:ContainsX} $X \in \scF$. 
        \item \label{prop:FilterFacts:ClosureUnderFiniteIntersections} $\scF$ is \ClosedUnderFiniteIntersections
        \item \label{prop:FilterFacts:IntersectionOfFiltersIsAFilter} The intersection of a collection of \Filters on $X$ is a \Filter on $X$. 
    \end{enumerate}
    \begin{proof}[Proof of \ref{prop:FilterFacts:ContainsX}]
        By \ref{def:Filter:IsNonempty}, 
        $\exists B \neq \emptyset \in \scF$. 
        Since $B \subset X \subset X$, by 
        \ref{def:Filter:SubsetProperty}, 
        $X \in \scF$, so \ref{prop:FilterFacts:ContainsX} is proven. 
    \end{proof}
    \begin{proof}[Proof of \ref{prop:FilterFacts:ClosureUnderFiniteIntersections}]
        Direct application of \ref{def:Filter:FiniteIntersectionProperty} paired with 
        \ref{prop:FiniteClosure:Union}.
    \end{proof}
    \begin{proof}[Proof of \ref{prop:FilterFacts:IntersectionOfFiltersIsAFilter}]
        Let $\{\scF_\alpha\}_{\alpha \in A}$ be a collection of \Filters on $X$. 
        Define $\scF=\bigcap\limits_{\alpha \in A} \scF_\alpha$. 
        By \ref{prop:FilterFacts:ContainsX}, for each $\alpha \in A$, 
        $X \in \scF_\alpha$, so $X \in \scF$.
        Hence $\scF$ satisfies $\ref{def:Filter:IsNonempty}$.
        Furthermore, by $\ref{def:Filter:DoesntContainEmpty}$, for each 
        $\alpha \in A$, $\emptyset \not \in \scF_\alpha$, so 
        $\emptyset \not \in \scF$. Therefore $\scF$ satisfies \ref{def:Filter:DoesntContainEmpty}.
        Since $G_1 \in \scF$ and $G_1 \subset G_2 \subset X$. 
        Then for each $\alpha \in A$, $G_1 \in \scF_\alpha$, so by 
        \ref{def:Filter:SubsetProperty}, $G_2 \in \scF_\alpha$.
        Hence $G_2 \in \scF$, so $\scF$ satisfies \ref{def:Filter:SubsetProperty}. 
        Finally, let $\{G_1,G_2\} \subset \scF$. 
        Then for each $\alpha \in A$, $\{G_1,G_2\} \subset \scF_\alpha$, 
        implying by $\ref{def:Filter:FiniteIntersectionProperty}$ that 
        $G_1 \cap G_2 \in \scF_\alpha$, so $G_1\cap G_2 \in \scF$, 
        implying $\scF$ satisfies $\ref{def:Filter:FiniteIntersectionProperty}$. 
        This concludes the proof of this result. 
    \end{proof}
    
\end{prop}


\begin{prop}
\label{prop:FilterExistence}
    Let $X$ be a set and
    $\emptyset \neq A \subset \scPowerSet{X}$.
    Then there is a \Filter
    on $X$ which contains $A$ if and only if 
    any \Finite intersection of elements of $A$ 
    is nonempty. 
    \begin{proof}
        The given condition is necessary by a combination 
        of \ref{prop:Filter:ContainsX} and \ref{def:Filter:DoesntContainEmpty}.
        For sufficiency,
        let $K$ be the collection of finite intersections
        of elements of $A$. 
        Define 
        \begin{equation}
        \scK = \{ F \cup Y | F \in K \wedge K \subset X\}
        \end{equation}
        Then $A \subset \scK$. 
        Since $A \neq \emptyset$, $\scK \neq \emptyset$, 
        so $\scK$ satisfies \ref{def:Filter:IsNonempty}. 
        Since \Finite intersections of elements of $A$
        are nonempty, $\emptyset \not \in K$, implying 
        $\emptyset \not \in \scK$, so $\ref{def:Filter:DoesntContainEmpty}$. 
        Now, let $P \in \scK$ and let $P \subset Q \subset X$.
        Then there exists $L_P \in K$ and $Y_P \subset X$ such that 
        $P = L_P \cup Y_P$, 
        and $Q = L_P \cup \pa{Y_P \cup Q} \in \scK$, so \ref{def:Filter:SubsetProperty}
        holds for $\scK$. 
        Finally, let 
        $G_1,G_2 \in \scK$. Then 
        $G_i=U_i \cup P_i$ for $U_i \in K$ and $P_i \subset X$. 
        By definition of $\scK$, 
        there are a $\{G_1^j\}_{j=1}^{n_1} \subset A$ 
        and $\{G_2^j\}_{j=1}^{n_2} \subset A$ with 
        $U_i = \bigcap\limits_{j=1}^{n_i} G_i^j$. 
        Clearly $U_1 \cap U_2$, being a finite subset of elemnts of $A$, 
        is an element of $\scK$. 
        Furthermore, 
        \begin{align*}
            U_1 \cap U_2 & \subset \pa{U_1 \cap U_2} \cup \pa{\pa{U_1 \cap P_2} \cup \pa{U_2 \cap P_1} \cup \pa{P_1 \cap P_2}}\\
                & = \pa{U_1 \cup P_1} \cap \pa{ U_2 \cup P_2 } \\
                & = G_1 \cap G_2
        \end{align*}
        so since $\scK$ satisfies $\ref{def:Filter:SubsetProperty}$, 
        $G_1 \cap G_2 \in \scK$, so $\ref{def:Filter:FiniteIntersectionProperty}$ 
        applies for $\scK$, so $\scK$ is a \Filter. 
        Hence,  \ref{prop:FilterExistence} is proven.
    \end{proof}
\end{prop}

\label{def:CoarseFineFilter}
\newcommand{\CoarserFilter}[0]{
    \textbf{\hyperref[def:CoarseFineFilter]{Coarser}}
}
\newcommand{\CoarsestFilter}[0]{
    \textbf{\hyperref[def:CoarseFineFilter]{Coarsest}}
}
\newcommand{\FinerFilter}[0]{
    \textbf{\hyperref[def:CoarseFineFilter]{Finer}}
}
\newcommand{\FinestFilter}[0]{
    \textbf{\hyperref[def:CoarseFineFilter]{Finest}}
}
\newcommand{\FilterFineness}[0]{
    \textbf{\hyperref[def:CoarseFineFilter]{Filter Fineness}}
}
\newcommand{\FilterCoarseness}[0]{
    \textbf{\hyperref[def:CoarseFineFilter]{Filter Coarseness}}
}
\newcommand{\UltraFilter}[0]{
    \textbf{\hyperref[def:CoarseFineFilter]{Ultrafilter}}
}
\newcommand{\UltraFilters}[0]{
    \textbf{\hyperref[def:CoarseFineFilter]{Ultrafilters}}
}


\begin{df}[\CoarserFilter, \FinerFilter]
    Let $X$ be a set and let $\scF_1$ 
    and $\scF_2$ be \Filters
    on $X$ such that 
    $\scF_1 \subset \scF_2$. 
    Then we say that 
    $\scF_1$ is \CoarserFilter
    than $\scF_2$ and we say that 
    $\scF_2$ is \FinerFilter than 
    $\scF_1$. 
    Let $A \subset X$ be a collection of filters
    and let $\scF_2 \in A$ be \FinerFilter 
    than every element of $A$. 
    Then we say that $\scF_2$ is the \FinestFilter
    element of $A$. 
    Let $\scF_3 \in A$ be \CoarserFilter
    than every element of $A$. 
    Then we say that $\scF_3$ is the \CoarsestFilter
    in $A$. 
    \FilterFineness
    defines a \PartialOrdering on
    the collection of \Filters on $X$, where 
    $\scF_1 \leq \scF_2$ if $\scF_2$ is a \FinerFilter than $\scF_1$. 
    A \Maximum
    of \FilterFineness is called an 
    \UltraFilter on $X$. 
\end{df}

\begin{prop}[Union/Intersection of chain of filters is a filter ]
    \label{prop:UnionIntersectionChainOfFiltersIsAFilter}
    Let \(X \neq \emptyset\).
    The following are true. 
    Then 
    \begin{enumerate}[label=(\roman*), ref={\ref{prop:UnionIntersectionChainOfFiltersIsAFilter}~\roman*}]
        \item \label{prop:UnionOfChainOfFiltersIsFinerFilter}
        
        $\bigcup\limits_{\alpha \in A} \scF_\alpha$ is a \Filter on $X$ which is \FinerFilter than each $X_\alpha$. 
        \item \label{prop:IntersectionOfFiltersIsCoarserFilter} 
        Let 
        \(\{ \scF_\alpha \}_{\alpha \in A}\) be a 
        collection of \Filters on $X$. 
        Then, 
        $\bigcap\limits_{\alpha \in A}\scF_\alpha$ is a \Filter on $X$ which is \CoarserFilter than each $X_\alpha$. 
    \end{enumerate}
    \begin{proof}[Proof of \ref{prop:UnionOfChainOfFiltersIsFinerFilter}]
        Denote 
        \begin{equation*}
        \scU = \bigcup\limits_{\alpha \in A} \scF_\alpha
        \end{equation*}
        
    \end{proof}
    \begin{proof}[Proof of \ref{prop:IntersectionOfFiltersIsCoarserFilter}]
        Denote 
        \begin{equation*}
            \scK = \bigcap\limits_{\alpha \in A} \scF_{\alpha}
        \end{equation*}
        By \ref{prop:FilterFacts:ContainsX}, $X \in \scK$, so \ref{def:Filter:IsNonempty}
        is satisfied by $\scK$. 
        Since $\emptyset \not \in \scF_\alpha$ for every $\alpha \in A$, $\emptyset \not \in \scK$, so $\ref{def:Filter:DoesntContainEmpty}$ is satisfied by $\scK$. 
        Let $G_1,G_2 \in \scK$. Then for each $\alpha \in A$, $G_1,G_2 \in \scF_\alpha$. 
        Hence, by \ref{def:Filter:FiniteIntersectionProperty}, for each $\alpha \in A$, 
        $G_1 \cap G_2 \in \scF_\alpha$. Hence $G_1 \cap G_2 \in \scK$. 
        Hence $\scK$ satisfies \ref{def:Filter:FiniteIntersectionProperty}. 
        Finally, let $G_1 \in \scK$. and let $G_1 \subset G_2 \subset X$. 
        Since $G_1 \in \scK$, for each $\alpha \in A$, $G_1 \in \scK$ implying that for each 
        $\alpha \in A$, $G_2 \in \scF_\alpha$. 
        Hence $G_2 \in \scK$, so $\scK$ satisfies $\ref{def:Filter:SubsetProperty}$
        and is therefore a \Filter. 
        For coarseness, note that by construction 
        $\scK \subset \scF_\alpha $ for each $\alpha \in A$, which directly mean $\scK$ is 
        \CoarserFilter than every $\scF_\alpha$ by \ref{def:CoarseFineFilter}
    \end{proof}
\end{prop}

\begin{prop}[Existence of finer ultrafilter]
\label{prop:ExistenceOfFinerUltrafilter}
    Let $\scF_0$ be a \Filter on a set $X$. 
    Then there exists an \Ultrafilter $\scF$ on $X$ 
    which is \FinerFilter than $\scF_0$. 
    \begin{proof}
        Let $
    \end{proof}
\end{prop}

\newcommand{\FilterBase}[0]{\textbf{\hyperref[def:FilterBase]{Filter Base}}\xspace}
\newcommand{\FilterBases}[0]{\textbf{\hyperref[def:FilterBase]{Filter Bases}}\xspace}
\newcommand{\FilterBaseEquivalent}[0]{\textbf{\hyperref[def:FilterBase]{Equivalent}}\xspace}

\begin{df}[\FilterBase]
\label{def:FilterBase}
    Let $X \neq \emptyset$. 
    Let $\scB \subset \scPowerSet{X}$
    such that 
    \begin{enumerate}[label=(\roman*), ref={\ref{def:FilterBase}.~\roman*}]
        \item \label{def:FilterBase:IsNotEmpty}$
        \emptyset \neq \scB$. 
        \item \label{def:FilterBase:DoesntContainEmptySet}$
        \emptyset \not \in \scB$. 
        \item \label{def:FilterBase:IntersectionProperty}
		Define $\scB_{Intersection}=\{U \cap V | \{U,V\} \subset \scB\}$.
		Then \scNested{\scB}{\scB_{Intersection}} holds. 
    \end{enumerate}
    Then we call 
    $\scB$ 
    a
    \FilterBase
    on $X$. 
    By \ref{prop:FilterBase}, the 
    \Filter
    \FilterGeneratedBy
    a 
    \FilterBase
    $A$ 
    is given by 
    $\{U \subset X | (\exists Y \subset A ) ( Y \subset U ) \}$. 

    If $A,B$ are \FilterBases
    on $X$ and they 
    \FilterGenerate
    the same 
    \Filter, 
    then we call them 
    \FilterBaseEquivalent. 
\end{df}


\begin{prop}
    \label{prop:FilterBaseGeneratesFilter}
    Let $\scB$ be a \FilterBase on a 
    set $X \neq \emptyset$. 
    Define 
    \begin{equation*}
    \scF=\{U \in \scPowerSet{X} | (\exists B \in \scB)(B \subset U) \}
    \end{equation*}
    Then the following are true. 
    \begin{enumerate}[label=(\roman*), ref={\ref{prop:FilterBaseGeneratesFilter}.~\roman*}]
        \item \label{prop:FilterBaseGeneratesFilter:IsAFilter} $\scF$ is a \Filter on $X$. 
        \item \label{prop:FilterBaseGeneratesFilter:IsCoarsest} $\scF$ is the \CoarsestFilter \Filter on $X$ which contain $\scB$. 
    \end{enumerate}
    \begin{proof}[Proof of \ref{prop:FilterBaseGeneratesFilter:IsAFilter}]
        By \ref{def:FilterBase:IsNotEmpty}, since $\scB \subset \scF$, we have $\ref{def:Filter:IsNonempty}$ holds for $\scF$. 
        Let $U \in \scF$. 
        Then, by definition, there is a $B \in \scB$ with $B \subset U$. 
            By \ref{def:FilterBase:DoesntContainEmptySet}, $B \neq \emptyset$, so $U \neq \emptyset$. 
            Hence $\emptyset \not \in \scF$, so 
            \ref{def:Filter:DoesntContainEmpty} holds for $\scF$. 
        Let $G_1,G_2 \in \scF$. Then there are 
        $B_1 \subset G_1$ and $B_2 \subset G_2$ with $B_i \in \scB$ for $i \in \{1,2\}$. 
        By \ref{def:FilterBase:IntersectionProperty}, there exists 
        $B \in \scB$ with $B \subset B_1 \cap B_2 \subset G_1 \cap G_2$. 
        By definition of $\scF$, then, $G_1 \cap G_2 \in \scF$, so that 
        \ref{def:Filter:FiniteIntersectionProperty}
        holds for $\scF$. 
        Since \ref{def:Filter:SubsetProperty} obviously holds for 
        $\scF$, we are done. 
    \end{proof}
    \begin{proof}[Proof of \ref{prop:FilterBaseGeneratesFilter:IsCoarsest}]
        Any $\scF_1 \subset 2^X$ containing $\scB$ which satisfies
        \ref{def:Filter:SubsetProperty} must by definition contain 
        $\scF$. By definition, all filters satisfy 
        \ref{def:Filter:SubsetProperty}, so we are done. 
    \end{proof}
\end{prop}





\subsection{Point Set Topology}
\subsubsection{Open Sets, Closed Sets, and Neighborhoods}

\newcommand{\TopologicalSpace}[2]{
    \pa{#1, \Topology{#1}{#2}}
}
\newcommand{\Topology}[2]{
    #2_{#1}
}
\newcommand{\TopologicalSpaceRef}[0]{
    \bf \hyperref[def:TopologicalSpace]{Topological Space} \rm
}
\newcommand{\TopologyRef}[0]{
        \bf \hyperref[def:TopologicalSpace]{Topology} \rm
}

\begin{df}[Topological Space]
    \label{def:TopologicalSpace}
    Let $X \neq \emptyset$ be a set 
    and let $\{\emptyset, X\} \subset \T \subset 2^X$ such that
    $\T$ is 
    \ClosedUnderArbitraryUnions and \ClosedUnderFiniteIntersections. 
    Then we call $\T$ a \TopologyRef on $X$ and we call 
    $(X,\T)$ a \TopologicalSpaceRef. 
\end{df}




\label{def:OpenSetClosedSet}
\newcommand{\SetOpen}[0]{\textbf{\hyperref[def:OpenSetClosedSet]{Set-Open}}\xspace}
\newcommand{\SetOpenness}[0]{\textbf{\hyperref[def:OpenSetClosedSet]{Set-Openness}}\xspace}
\newcommand{\SetClosed}[0]{\textbf{\hyperref[def:OpenSetClosedSet]{Set-Closed}}\xspace}
\newcommand{\SetClosedness}[0]{\textbf{\hyperref[def:OpenSetClosedSet]{Closedness}}\xspace}
\begin{df}[\SetOpen, \SetClosed]
    Let $(X, \T)$ 
    be a 
    \TopologicalSpace, 
    and let $A \in \T$. 
    We say that 
    $A$ is 
    \SetOpen
    in $(X,\T)$
    ( or \SetOpen in $X$ or \SetOpen in $\T$ or simply \SetOpen in cases where confusion
    won't result)
    and that $A$ posesses 
    \SetOpenness. 
    We say that 
    $X \setminus A$ 
    is 
    \SetClosed
    and that $X \setminus A$ 
    posesses 
    \SetClosedness
\end{df}

\label{def:FunctionContinuous}
\newcommand{\ContinuousFunction}[0]{\textbf{\hyperref[def:FunctionContinuous]{Continuous}}\xspace} 
\newcommand{\FunctionContinuity}[0]{\textbf{\hyperref[def:FunctionContinuous]{Continuity}}\xspace}
\begin{df}[global \FunctionContinuity of a function]
    Let $(X,\T_X)$ and 
    $(Y,\T_Y)$
    be
    \TopologicalSpaces.
    We say that a function
    $f:X \to Y$ is 
    \ContinuousFunction
    and that it exhibits
    \FunctionContinuity
    with respect to $\T_1$ and $\T_2$
	if
    $f^{-1}(\T_Y) \subset \T_X$. 
    We may make the \Topologies 
    explicit by writing 
    $f:(X, \T_X) \to (Y, \T_Y)$, 
    in which case we just say that
    $f$ is 
    \ContinuousFunction
    or that $f$ 
    posesses 
    \FunctionContinuity.
\end{df}

\label{def:OpenFunction}
\newcommand{\OpenFunction}[0]{\textbf{\hyperref[def:OpenFunction]{Open}}\xspace}
\newcommand{\FunctionOpenness}[0]{\textbf{\hyperref[def:OpenFunction]{Openness}}\xspace}
\begin{df}[\OpenFunction]
    Let $(X,\T_X)$ and
    $(Y, \T_Y)$ 
    be 
    \TopologicalSpaces. 
    We say that 
    $f:X \to Y$ 
    is 
    \OpenFunction
    if 
    $f(U)$ is 
    \SetOpen 
    in $(Y,\T_Y)$ 
    for every 
    \SetOpen
    $U \in (X,\T_X)$. 
\end{df}


\label{def:Homeomorphism}
\newcommand{\Homeomorphism}[0]{
    \textbf{\hyperref[def:Homeomorphism]{Homeomorphism}}
}
\newcommand{\Homeomorphisms}[0]{
    \textbf{\hyperref[def:Homeomorphism]{Homeomorphisms}}
}
\newcommand{\Homeomorphic}[0]{
    \textbf{\hyperref[def:Homeomorphism]{Homeomorphic}}
}

\begin{df}[\Homeomorphism]
    Let \(X,\T_X\)
    and \(Y, \T_Y\) be \TopologicalSpaces.
    Let $f:X \to Y$ such be a 
    \ContinuousFunction
    \Bijection
	such that $f^{-1}:Y \to X$ is also 
	\ContinuousFunction.
    Then we say that \(f\) is a 
    \Homeomorphism from \(X\) to \(Y\)
    and we say that 
    \(X\) and \(Y\) are 
    \Homeomorphic. 
\end{df}


\label{def:TopologyCoarseFine}
\newcommand{\TopologyCoarse}[0]{foo}

\newcommand{\TopologyFine}[0]{foo}

\newcommand{\TopologyCoarser}[0]{foo}

\newcommand{\TopologyFiner}[0]{foo}

\newcommand{\TopologyCoarsest}[0]{foo}

\newcommand{\TopologyFinest}[0]{}
\newcommand{\scTopologyCoarsenessRelation}[1]{
    %\hyperref[def:TopologyCoarseFine]{\ensuremath{\leq_{Top(#1)}}}
    foo
}
%
%\begin{df}[\TopologyCoarse, \TopologyFine]
%    Let $X$ be a set. 
%    Let $\T_1, \T_2$ be 
%    \TopologyRef s
%    on $X$
%    such that $\T_1 \subset \T_2$. 
%    In this case, we say that
%    $\T_1$ is more 
%    \TopologyCoarse
%    than 
%    $\T_2$, 
%    that $\T_1$
%    is 
%    \TopologyCoarser
%    than $\T_2$, 
%    that
%    $\T_2$ is more 
%    \TopologyFine
%    than $\T_1$, 
%    and that $\T_2$ is 
%    \TopologyFiner
%    than $\T_1$. 
%    
%    If $X$ is a set 
%    W
%    %Topology Coarseness partially orderes the set of topolgoies on X
%    %Intersection of elements of a subset of set of topologies is maximal element of that set. 
%    %Notation Defined Above
%   
%    Observe that the intersection of 
%%    any colleciton of \TopologyRef 's 
%    on $X$ 
%    is a \TopologyRef
%    on $X$. 
%
%    
%\end{df}
%

\label{def:WeakTopology}
\newcommand{\WeakTopology}[0]{
	\bf \hyperref[def:WeakTopology]{Weak Topology} \rm
}

\begin{df}[\WeakTopology]
	Let X be a set. 
    For each $\alpha \in A$, let 
    $(Y_\alpha, T_\alpha)$ be a 
    \TopologicalSpace, 
    and let $\phi_\alpha:X \to (Y_\alpha, T_\alpha)$. 
    Let $\T$ be the 
	\TopologyCoarsest possible 
    \Topology on X such that 
    for each $\alpha \in A$, 
    $\phi_\alpha:(X, \T) \to (Y, \T_\alpha)$ 
    is \ContinuousFunction. 
    We call $\T$ the
    \WeakTopology on 
    X induced by $\{\phi_\alpha\}_{\alpha \in A}$
\end{df}

\label{def:InductiveTopology}
\newcommand{\InductiveTopology}[0]{
    \textbf{\hyperref[def:InductiveTopology]{Inductive Topology}}
}
\newcommand{\InductiveTopologies}[0]{
    \textbf{\hyperref[def:InductiveTopology]{Inductive Topologies}}
}

\begin{df}[\InductiveTopology]
    Let $X$ be a set and for each 
    $\alpha \in A$, let 
    $(Y_\alpha, \T_\alpha)$ be a 
    \TopologicalSpace.
    Furthermore, for each $\alpha \in A$, let 
    $\phi_\alpha : (Y, \T_\alpha) \to X$. 
    Let $\T$ be the 
    \TopologyFinest
    topology on $X$ for whhich 
    each $\phi_\alpha$ is 
    \ContinuousFunction. 
    fWe call $\T$ the \InductiveTopology
    on $X$ induced by $\{\phi_\alpha\}_{\alpha \in A}$. 

    
\end{df}
:
\label{def:SubspaceTopology}
\newcommand{\SubspaceTopology}[0]{
    \textbf{\hyperref[def:SubspaceTopology]{Subspace Topology}}
}

\newcommand{\SubspaceTopologies}[0]{
    \textbf{\hyperref[def:SubspaceTopology]{Subspace Topologies}}
}

\newcommand{\SubspaceTopologicalSpace}[0]{
    \textbf{\hyperref[def:SubspaceTopology]{Subspace Topological Space}}
}

\newcommand{\SubspaceTopologicalSpaces}[0]{
    \textbf{\hyperref[def:SubspaceTopology]{Subspace Topological Spaces}}
}

\begin{df}[\SubspaceTopology]
    Let $(X, \T_X)$ 
    be a 
    \TopologicalSpace, 
    Let $Y \subset X$, 
    and let 
    $f$ be the 
    \InsertionFunction
    of $Y$ into $X$. 
    We call the \WeakTopology 
    on $Y$ generated by $f$
    which we will denote here with $\T_Y$, 
    the \SubspaceTopology
    of $Y$ relative to $(X,\T_X)$. 
    We call $(Y, \T_Y)$ the 
    \SubspaceTopologicalSpace.
    Unless otherwise specified, 
    when referring to a subset of a 
    \TopologicalSpace, 
    we consider that subset as 
    being a \TopologicalSpace 
    which is endowed with the \SubspaceTopology, 
    and when we say that a subset of a 
    \TopologicalSpace
    has a particular (Topological) property which has thus far only been defined 
    for a \TopologicalSpace, 
    we mean that the  \SubspaceTopologicalSpace 
    has that property. 
\end{df}



\subsubsection{Neighborhood}
\label{def:Neighborhood}
\newcommand{\Neighborhood}[0]{ \bf \hyperref[def:Neighborhood]{Neighborhood} \rm }
\newcommand{\Neighborhoods}[0]{ \bf \hyperref[def:Neighborhood]{Neighborhoods} \rm }
\newcommand{\NeighborhoodFilter}[0]{ \bf \hyperref[def:Neighborhood]{Neighborhood Filter} \rm }
\newcommand{\NeighborhoodFilters}[0]{ \bf \hyperref[def:Neighborhood]{Neighborhood Filters} \rm }
\newcommand{\NeighborhoodFilterInstance}[0]{\scU}
\begin{df}[\Neighborhood, \NeighborhoodFilter]
    Let $(X, \T)$ be a \TopologicalSpaceRef.
    $A$ be \SetOpen in $(X, \T)$, 
    and $x \in B \subset A$. 
    We call $A$ a 
    \Neighborhood
    of $x$ in $(X,\T)$. 
    We represent the collection of all \Neighborhoods 
    of $x$ in $(X,\T)$ with 
    $\NeighborhoodFilterInstance_{\T}(x)$
    and we call this the 
    \NeighborhoodFilter of 
    $\T$
    at 
    $x$. 
    We also call $A$ a 
    \Neighborhood of $B$ 
    in $(X,\T)$, and 
    we represent the collection of all 
    \Neighborhoods of 
    $B$ 
    with 
    $\NeighborhoodFilterInstance_{\T}(B)$.
\end{df}

\begin{prop}[\NeighborhoodFilter is a \Filter]
\label{prop:NeighborhoodFilter}
Let $(X,\T)$ be a \TopologicalSpace 
For each $x \in X$, 
let $\NeighborhoodFilterInstance{\T}(x)$ denote the 
\NeighborhoodFilter of $x$.
The following are true 
\begin{enumerate}[label=(\roman*), ref={\ref{prop:NeighborhoodFilter}~\roman*}]
\item \label{prop:NeighbhorhoodFilter:IsFilter} \NeighborhoodFilterInstance{\T}(x) is a \Filter on $X$. 
\item \label{prop:NeighborhoodFilter:Containsx} For each $U \in \NeighborhoodFilterInstance{\T}(x)$, $x \in U$. 
\item \label{prop:NeighborhoodFilter:CharacteristicProperty} 
Let $x,y \in X$. Then, if $U \in \NeighborhoodFilterInstance{\T}(x)$, then there exists 
$V \in \NeighborhoodFilterInstance{\T}(x)$ such that for each $y \in V$, $U \in \NeighborhoodFilterInstance{\T}(y)$. 
\end{enumerate}
\begin{proof}[Proof of \ref{prop:NeighbhorhoodFilter:IsFilter}]
Clearly $x \in X \subset X \subset X \subset X \in \T$, so $X \in \NeighborhoodFilterInstance{\T}(x)$. 
Thus $\NeighborhoodFilterInstance{\T}(x)$ satisfies \ref{def:Filter:IsNonempty}. 
Also, since $x \not \in \emptyset$
, $\emptyset \not \in\NeighborhoodFilterInstance{\T}(x)$.
Hence $\NeighborhoodFilterInstance{\T}(x)$ satisfies \ref{def:Filter:DoesntContainEmpty}.
If $\{G_1, G_2\} \subset \NeighborhoodFilterInstance{\T}(x)$ with 
$G_1 \cap G_2 \neq \emptyset$
then there are \SetOpen $U_i$ with $x \in U_i \subset G_i$.
For these $U_i$, 
$x \in U_1 \cap U_2 \subset U_1 \cap U_2 \subset G_1 \cap G_2$ and $U_1 \cap U_2 \in \T$. 
Hence, \NeighborhoodFilterInstance{\T} satifies
\ref{def:Filter:FiniteIntersectionProperty}.
It is obvious that $\NeighborhoodFilterInstance{\T}(x)$ satisfies
\ref{def:Filter:SubsetProperty}. 
\end{proof}
\begin{proof}[Proof of \ref{prop:NeighborhoodFilter:Containsx}]
Painfully Obvious
\end{proof}
\begin{proof}[Proof of \ref{prop:NeighborhoodFilter:CharacteristicProperty}]
Let $U \in \NeighborhoodFilterInstance{\T}(x)$. 
Then there exists \SetOpen $V$ with $x \in V \subset U$. 
Since $V$ is \SetOpen, $V \in \NeighborhoodFilterInstance{\T}(x)$. 
Let $Y \in V$. Then, $y \in V \subset V \subset U$, so $U \in \NeighborhoodFilterInstance{\T}(y)$. 
Hence $\ref{prop:NeighborhoodFilter:CharacteristicProperty}$ is satisfied. 
\end{proof}
\end{prop}


\label{def:DiscreteIndiscreteTopology}
\newcommand{\DiscreteTopology}[0]{
    \textbf{\hyperref[def:DiscreteIndiscreteTopology]{Discrete Topology}}
}
\newcommand{\DiscreteTopologies}[0]{
    \textbf{\hyperref[def:DiscreteIndiscreteTopology]{Discrete Topologies}}
}
\newcommand{\IndiscreteTopology}[0]{
    \textbf{\hyperref[def:DiscreteIndiscreteTopology]{Indiscrete Topology}}
}
\newcommand{\IndiscreteTopologies}[0]{
    \textbf{\hyperref[def:DiscreteIndiscreteTopology]{Indiscrete Topologies}}
}
\begin{df}[\DiscreteTopology, \IndiscreteTopology]
    Let $X$ be a set. 
    We call $\{X, \emptyset\}$ 
    the \IndiscreteTopology
    on $X$
    and we call 
    \scPowerSet{X}
    the 
    \DiscreteTopology 
    on $X$. 
\end{df}


\label{def:AccumulationClosureInterior}
\newcommand{\AccumulationPoint}[0]{ 
    \bf \hyperref[def:AccumulationClosureInterior]{Accumulation Point} \rm 
}
\newcommand{\AccumulationPoints}[0]{
    \bf \hyperref[def:AccumulationClosureInterior]{Accumulation Points} \rm 
}
\newcommand{\AccumulationPointMark}[1]{
    #1'
}
\newcommand{\Closure}[0]{
    \bf \hyperref[def:AccumulationClosureInterior]{Closure} \rm 
}
\newcommand{\Closures}[0]{
    \bf \hyperref[def:AccumulationClosureInterior]{Closures} \rm 
}
\newcommand{\ClosureMark}[1]{
    \overline{#1}
}
\newcommand{\Interior}[0]{
    \bf \hyperref[def:AccumulationClosureInterior]{Interior} \rm 
}
\newcommand{\Interiors}[0]{
    \bf \hyperref[def:AccumulationClosureInterior]{Interiors} \rm 
}
\newcommand{\InteriorMark}[1]{
    \overset{\circ}{#1}
}
\newcommand{\Boundary}[0]{
    \bf \hyperref[def:AccumulationClosureInterior]{Boundary} \rm
}
\newcommand{\Boundaries}[0]{
    \bf \hyperref[def:AccumulationClosureInterior]{Boundaries} \rm
}
\newcommand{\BoundaryMark}[1]{
    \partial\pa{#1}
}
\begin{df}[\AccumulationPoint, \Closure, \Interior, \Boundary]
    Let $(X,\T)$ be a \TopologicalSpace.
    Let $A \subset X$. We define the following. 
    \begin{enumerate}
        \item $\AccumulationPointMark{A}=\{x \in X | (\forall U \in \NeighborhoodFilterInstance_{\T}(A))((U \setminus A) \cap \{x\} \neq \emptyset)\} $
        \item $\ClosureMark{A}= A \cup \AccumulationPointMark{A}$
        \item $\BoundaryMark{A} = \ClosureMark{A} \cap \ClosureMark{X \setminus A}$
        \item $\InteriorMark{A} = A \setminus \ClosureMark{X \setminus A}$
    \end{enumerate}
    We call 
    an element of $\AccumulationPointMark{A}$ 
    an \AccumulationPoint of $A$. 
    We call $\ClosureMark{A}$ the 
    \Closure 
    of $A$. 
    We call $\InteriorMark{A}$ 
    the \Interior 
    of $A$. 
    We call $\BoundaryMark{A}$ 
    the \Boundary of $A$. 
\end{df}

\label{def:TopologySubBasis}
\newcommand{\TopologySubBasis}[0]{
    \textbf{\hyperref[def:TopologySubBasis]{SubBasis}}
}
\newcommand{\TopologySubBases}[0]{
    \textbf{\hyperref[def:TopologySubBasis]{SubBases}}
}
\newcommand{\scGeneratedTopology}[2]{
    \hyperref[def:TopologySubBasis]{\ensuremath{\T_{#1}\pa{#2}}}
}
\newcommand{\TopologyGeneratedBy}[0]{
    \textbf{\hyperref[def:TopologySubBasis]{Generated By}}
}

\begin{df}[\TopologySubBasis]
    Let $X \neq \emptyset$ 
    and let $B \subset \scPowerSet{X}$. 
    We denote the 
    \TopologyCoarsest
    \Topology on $X$ 
    containing $B$
    with
    \scGeneratedTopology{X}{B}.
    We say that $B$
    is a 
    \TopologySubBasis
    for 
    \scGeneratedTopology{X}{B}
    and we call 
    \scGeneratedTopology{X}{B}
    the 
    \Topology
    on $X$ 
    \TopologyGeneratedBy
    $B$. 
\end{df}

\begin{prop}[Characterization Of Generated Topology]
    \label{prop:CharacterizationOfGeneratedTopoology}
    Let $X \neq \emptyset$ 
    and $F \subset \scPowerSet{X}$. 
    Define 
    \begin{equation*}
    \T_{Prop}=\left\{\bigcup\limits_{\alpha \in A} \bigcap\limits_{i=1}^{N_\alpha} U_{i, \alpha} |(\forall \alpha \in A)\left((N_\alpha \in \N)\wedge \pa{(\forall i \in \{1, \cdots, N_\alpha\}} \pa{U_{i, \alpha} \in F}\right) \right\}\cup \{X, \emptyset\}
    \end{equation*}
    Then 
    $\scGeneratedTopology{X}{F} = \T_{Prop}$.
    \begin{proof}
        We first show that $\T_{Prop}$ is a
        \Topology
        on $X$. 
        For 
        \ClosureUnderUnions, 
        Let $B \neq \emptyset$ 
        and $\{B_\beta\}_{\beta \in B} \subset \T_{Prop}$
        Then for each $\beta \in B$, we can find $A_{\beta}$
        such that for each $\alpha_{\beta} \in A_{\beta}$, 
        there is an $N_{\alpha_{\beta}} \in \N$ 
        such that for each $i \in \{1, \cdots, N_{\alpha_\beta}\}$, 
        $U_{i, \alpha_\beta} \in F$ and
        \begin{equation}
            B_{\beta} = \bigcup\limits_{\alpha \in A_\beta} \bigcap\limits_{i=1}^{N_{\alpha_\beta}}U_{i, \alpha_\beta}
        \end{equation}
        Hence, we can write
         
        \begin{align*}
            \bigcup\limits_{\beta \in B} B_{\beta} & = \bigcup\limits_{\beta \in B} \bigcup\limits_{\alpha_\beta \in A_{\beta}} \bigcap\limits_{i=1}^{N_{\alpha_\beta}} U_{i, \alpha_\beta}\\
            & = \bigcup\limits_{\alpha_\beta \in \bigcup\limits_{\beta \in B} A_{\beta}} \bigcap\limits_{i =1 }^{N_{\alpha_\beta}} U_{i, \alpha_\beta} \in \T_{Prop}
        \end{align*}
        For \ClosureUnderFiniteIntersections, 
        let $N \in \N$ and 
        $\{B_j\}_{j=1}^N \subset \T_{Prop}$. 
        Then for each $j \in \{1, \cdots, N\}$, there 
        is an $A_{j}$
        such that for each $\alpha_{j} \in A_{j}$, 
        there is an $N_{\alpha_{j}} \in \N$ 
        such that for each $i \in \{1, \cdots, N_{\alpha_j}\}$, 
        $U_{i, \alpha_j} \in F$ and
        \begin{align*}
            \bigcap\limits_{j=1}^N B_j & = \bigcap\limits_{j=1}^N \bigcup\limits_{\alpha_j \in A_j} \bigcap\limits_{i=1}^{N_{\alpha_j}} U_{\alpha_j, i}\\
            & = \bigcup\limits_{\{\alpha_j\}_{j=1}^N \in \scCartesianProduct{j}{\{1, \cdots, N\}}{A}}\pa{\bigcap\limits_{j=1}^{N} \bigcap_{i=1}^{N_{\alpha_j}} U_{\alpha_j, i}} \in \T_{Prop}
        \end{align*}
        By construction, $X \in \T_{Prop}$ and $\emptyset \in \T_{Prop}$, so 
        $\T_{Prop}$ is in fact a 
        \Topology on 
        $X$. 
        By taking the union over the intersection of a single element, we have $F \subset \T_{Prop}$, so that 
        $\scGeneratedTopology{X}{F} \subset \T_{Prop}$. 
        Furthermore, $\scGeneratedTopology{X}{F}$ is closed under finite intersections and arbitrary unions
        so that it must contain $\T_{Prop}$. 
        Hence, equality holds. 



    \end{proof}
\end{prop}

\label{def:TopologyBasis}
\newcommand{\TopologyBasis}[0]{
    \textbf{\hyperref[def:TopologyBasis]{Basis}}
}
\newcommand{\TopologyBases}[0]{
    \textbf{\hyperref[def:TopologyBasis]{Bases}}
}
\begin{df}[\TopologyBasis]
    Let $(X,\T)$ 
    be a 
    \TopologicalSpace
    and let $B \subset \T$ 
    such that 
    each element of 
    $\T$ 
    can be written as a union 
    of elements of 
    $B$. 
    Then we call 
    $B$
    a 
    \TopologyBasis
    for $\T$. 
\end{df}

\begin{prop}
    \label{prop:TopologyBasisCharacterization}
    Let $(X,\T)$ be a 
    \TopologicalSpace 
    and let 
    $\scG \subset \T$ 
    such that $\{\emptyset, X \} \subset \scG$. 
    The following conditions are equivalent
    \begin{enumerate}
        \item For every
            \SetOpen
            $U$, 
            for every $x \in U$, 
            there exists an
            $G_x \in \scG$
            such that 
            $x \in G_x \subset U$. 
        \item $\scG$  is a \TopologyBasis for $\T$. 
    \end{enumerate}
    \begin{proof}[$1 \implies 2$]
        Let $U \in \T$. Then we can write $U = \bigcup_{x \in U} G_x$, implying that $\scG$ is a \TopologyBasis. 
    \end{proof}
    \begin{proof}[$2 \implies 1$]
        Let $U$ is \SetOpen, then since 
        $\scG$ is a \TopologyBasis, 
        there is a $\{G_{\alpha}\}_{\alpha \in A} \subset \scG$ such that 
        $U = \bigcup_{\alpha \in A} G_{\alpha}$. 
        Hence, if $x \in U$, then 
        $x \in G_{\alpha}$ for some $\alpha \in A$, and obviously $G_{\alpha} \subset U$, 
        so $1$ holds and we're done. 
    \end{proof}
\end{prop}

\begin{prop}[Bassis Of Generated Topology]
\label{prop:BasisOfGeneratedTopology}
    Let $X \neq \emptyset$ and let 
    $B \subset \scPowerSet{X}$
    Then $B$ is a 
    \TopologyBasis 
    for 
    \scGeneratedTopology{X}{B}
    if and only if 
    the following hold
    \begin{enumerate}
        \item $X \in B$
        \item $\emptyset \in B$. 
        \item For each $U,V \in B$, For each $x \in U \cap V$, there is a $W \in B$ with $x \in W \subset U \cap V$. 
    \end{enumerate}
    \begin{proof}
        I first claim that 
        it is sufficient to show that any finite intersection of elements of $B$ can 
        be written as a union of elements of $B$.
        By Induction, proving for a binary intersection is sufficient. 
        Hence, let $U,V \in B$ with $U \cap V \neq \emptyset$. 
        Then for each $x \in U \cap V$, by assumption, 
        there exists a $W_x \in B$ such that 
        $x \in W_x \subset U \cap V$. 
        Hence, we can write 
        \begin{equation*}
            U \cap V \subset \bigcup_{x \in U \cap V} W_x \subset U \cap V
        \end{equation*}
        showign that finite intersctions of 
        elements of $B$ can be written as unions of 
        elements of $B$. 
        Hence, by
        \ref{prop:CharacterizationOfGeneratedTopoology}, 
        $\scGeneratedTopology{X}{B}$ 
        consists of exactly the unions of elements of $B$, finishing one direction.
        For the other direction, 
        if $B$ is a 
        \TopologyBasis
        for $\scGeneratedTopology{X}{B}$, 
        then 
        by 
        \ref{prop:TopologyBasisCharacterization}, 
        sicnce $\scGeneratedTopology{X}{B}$ 
        contains finite intersections of elements of $B$, 
        the given properties hold. 
       
        

    \end{proof}
\end{prop}

\label{def:NeighborhoodBasis}
\newcommand{\NeighborhoodBasis}[0]{\textbf{\hyperref[def:NeighborhoodBasis]{Neighborhood Basis}}\xspace}
\newcommand{\NeighborhoodBases}[0]{\textbf{\hyperref[def:NeighborhoodBasis]{Neighborhood Bases}}\xspace}
\begin{df}[\NeighborhoodBasis]
    Let $(X,\T)$ be a
    \TopologicalSpace
    and let $X \in X$. 
    Let 
    $F \subset \T$ 
    such that
    for each $U \in \T$ with $x \in U$, there
    exists $f \in F$ with 
    $f \subset U$. 
    Further, let $x \in G$ for each $G \in F$. 
    Then we call $F$ a 
    \NeighborhoodBasis
    for $\T$ at $x$. 
\end{df}


\label{def:CompactSet}
\newcommand{\SetCompact}[0]{\textbf{\hyperref[def:CompactSet]{Compact}}\xspace}
\newcommand{\SetCompactness}[0]{\textbf{\hyperref[def:CompactSet]{Compactness}}\xspace}

\begin{df}[\SetCompact]
    We say that a 
    \TopologicalSpace
    is 
    \SetCompact
    if every 
    \SetOpen
    \Cover 
    for $X$ 
    has a 
    \Finite
    \Subcover.
\end{df}

\label{def:ClosedFunction}
\newcommand{\ClosedFunction}[0]{
    \textbf{\hyperref[def:ClosedFunction]{Closed}}
}
\newcommand{\FunctionClosedness}[0]{
    \textbf{\hyperref[def:ClosedFunction]{Closedness}}
}
\begin{df}[\ClosedFunction]
    Let $(X,\T_X)$ and
    $(Y, \T_Y)$ 
    be 
    \TopologicalSpaces. 
    We say that 
    $f:X \to Y$ 
    is 
    \ClosedFunction
    if 
    $f(K)$ is 
    \SetClosed 
    in $(Y,\T_Y)$ 
    for every 
    \SetClosed
    $K \in (X,\T_X)$. 
\end{df}


\label{def:CompactFunction}
\newcommand{\CompactFunction}[0]{
    \textbf{\hyperref[def:CompactFunction]{Compact}}
}
\newcommand{\FunctionCompactness}[0]{
    \textbf{\hyperref[def:CompactFunction]{Compactness}}
}
\begin{df}[\CompactFunction]
    Let $(X,\T_X)$ and
    $(Y, \T_Y)$ 
    be 
    \TopologicalSpaces. 
    We say that 
    $f:X \to Y$ 
    is 
    \CompactFunction
    if 
    $f(K)$ is 
    \SetCompact 
    in $(Y,\T_Y)$ 
    for every 
    \SetCompact
    $K \in (X,\T_X)$. 
\end{df}


\label{def:ProductTopology}
\newcommand{\ProductTopology}[0]{
    \textbf{\hyperref[def:ProductTopology]{Product Topology}}
}
\newcommand{\ProductTopologies}[0]{
    \textbf{\hyperref[def:ProductTopology]{Product Topologies}}
}

\begin{df}[\ProductTopology]
    Let $A \neq \emptyset$. 
    For each $\alpha \in A$, 
    let $(X_\alpha, \T_\alpha)$ 
    be a 
    \TopologicalSpace.
    We call the \WeakTopology
    on 
    \scCartesianProduct{\alpha}{A}{X}
    induced by 
    $\{\pi_\alpha:\scCartesianProduct{\alpha}{A}{X} \to (X_\alpha, \T_\alpha)\}_{\alpha \in A}$
    the \ProductTopology. 
\end{df}

\subsubsection{Countability  Axioms}
\label{def:FirstCountable}
\newcommand{\FirstCountable}[0]{
    \textbf{\hyperref[def:FirstCountable]{First Countable}}
}
\newcommand{\FirstCountability}[0]{
    \textbf{\hyperref[def:FirstCountable]{First Countability}}
}

\begin{df}[\FirstCountable]
    Let $(x,\T)$ be a 
    \TopologicalSpace. 
    We say that 
    $X$
    is 
    \FirstCountable
    if for each $x \in X$, 
    there is a 
    \Countable
    \NeighborhoodBasis
    for 
    $\T$
    at
    $X$. 
\end{df}    

\label{def:SecondCountable}
\newcommand{\SecondCountable}[0]{
    \textbf{\hyperref[def:SecondCountable]{Second Countable}}
}
\newcommand{\SecondCountability}[0]{
    \textbf{\hyperref[def:SecondCountable]{Second Countability}}
}

\begin{df}[\SecondCountable]
    A \TopologicalSpace
    which permits a 
    \Countable
    \TopologyBasis
    is called 
    \SecondCountable
\end{df}

\label{def:TopologyDense}
\newcommand{\TopologyDense}[0]{
    \textbf{\hyperref[def:TopologyDense]{Dense}}
}
\newcommand{\TopologyDensity}[0]{
    \textbf{\hyperref[def:TopologyDense]{Density}}
}
\begin{df}[\TopologyDense]
    Let $(X,\T)$ be a 
    \TopologicalSpace
    and let $A \subset X$. 
    We say that $A$ is
    \TopologyDense
    in $X$ if 
    $\ClosureMark(A) = X$. 
\end{df}

\label{def:TopologySeparable}
\newcommand{\TopologySeparable}[0]{
    \textbf{\hyperref[def:TopologySeparable]{Separable}}
}
\newcommand{\TopologySeparability}[0]{
    \textbf{\hyperref[def:TopologySeparable]{Separability}}
}
\begin{df}[\TopologySeparable]
    We say that a 
    \TopologicalSpace
    which permits a 
    \Countable
    \TopologyDense
    subset is 
    \TopologySeparable. 
\end{df}

\label{def:TopologyLindelof}
\newcommand{\TopologyLindelof}[0]{\textbf{\hyperref[def:TopologyLindelof]{Lindelof}}\xspace}
\begin{df}[\TopologyLindelof]
    A \TopologicalSpace 
    in which every
    \SetOpen
    \Cover
    permits a
    \Countable
    \Subcover
    is called a 
    \TopologyLindelof
    space.
\end{df}

\subsubsection{Separation Axioms}

\subsubsection{Uniformities}
\newcommand{\Uniformity}[0]{
    \textbf{\hyperref[def:Uniformity]{Uniformity}}
}
\newcommand{\Uniformities}[0]{
    \textbf{\hyperref[def:Uniformity]{Uniformities}}
}

\newcommand{\Entourage}[0]{
    \textbf{\hyperref[def:Uniformity]{Entourage}}
}
\newcommand{\Entourages}[0]{
    \textbf{\hyperref[def:Uniformity]{Entourages}}
}
\newcommand{\UniformityClose}[0]{
    \textbf{\hyperref[def:Uniformity]{Close}}
}
\newcommand{\UniformityCloseEnough}[0]{
    \textbf{\hyperref[def:Uniformity]{Close Enough}}
}
\begin{df}[\Uniformity]
\label{def:Uniformity}
    Let $X$ be a set and let $\scW \subset \scPowerSet{X \times X}$ such that
    \begin{enumerate}[label=(\roman*), ref={\ref{def:Uniformity}~\roman*}]
    \item \label{def:Uniformity:IsFilter} $\scW$ is a \Filter on $X \times X$. 
    \item \label{def:Uniformity:ContainsDiagonal} For each $W \in \scW$, $\scSetDiagonal{X} \subset W$. 
    \item \label{def:Uniformity:ContainsInverse} $W \in \scW \implies W^{-1} \in \scW$. 
    \item \label{def:Uniformity:ContainsCompositionSubset} For each $W \in \scW$, there exists $V \in \scW$ such that $V \circ V \subset W$. 
    \end{enumerate}
    Then we call $\scW$ a \Uniformity
    on $X$. 
    Furthermore, if $W \in \scW$, then we call 
    $W$ an \Entourage 
    of $\scW$. 
    If $(x,y) \in W$ is an \Entourage of $\scW$, 
    then we say that $x$ and $y$ are W-\UniformityClose.
    If $R$ is a \Relation on $X$ 
    then saying that $x_0Ry_0$ is true whenever
    $x_0$ and $y_0$ are \UniformityCloseEnough
    means that there exists an \Entourage $V$ such that $V \subset R$. 
\end{df}



%
\newcommand{\Homeomorphism}[0]{
    \bf \hyperref[def:Homeomorphism]{Homeomorphism} \rm
}








\newcommand{\NeighborhoodBasis}[0]{
	\bf \hyperref[def:TopologicalSpace]{Neighborhood Basis} \rm
}

\newcommand{\Bicontinuous}[0]{
	\bf \hyperref[def:TopologicalSpace]{Bicontinuous} \rm
}

\newcommand{\LetBeTopologicalSpace}[2]{
    Let $\scTopologicalSpace{#1}{#2}$ be a \TopologicalSpace.
}


\begin{df}[Homeomorphism]
    \label{def:Homeomorphism}
    \end{df} 









\subsection{Neighborhood Filter Of A Point}
\label{def:RelationOfEqualNeighborhoodFilters}
\newcommand{\RelationOfEqualNeighborhoodFilters}[1]{
    \bf \hyperref[def:RelationOfEqualNeighborhoodFilters]{Relation Of Equal Neighborhood Filters} \rm on #1
}
\begin{df}[Relation of Equal \NeighborhoodFilters]
    Let $(Z, \T_Z)$ be a \TopologicalSpace
	Define the relation 
	$\cong \subset Z \times Z$ 
	by setting, for $x,y \in Z$, 
    \begin{equation}
        x \cong y \iff \scU_{\T_Z}(x)=\scU_{\T_Z}(y)
    \end{equation}
    We call $\cong$ the \RelationOfEqualNeighborhoodFilters{$(Z,\T_Z)$}
\end{df} 


\begin{prop}[\RelationOfEqualNeighborhoodFilters]
    \label{prop:EqualNeighborhoodFiltersEquivalenceRelation}
    
    The
	\RelationOfEqualNeighborhoodFilters
	$\cong$ on a \TopologicalSpaceRef $(Z,\T_Z)$ forms an 
	\EquivalenceRelation	
	on Z. 
    \begin{proof}
        
        Let $x \in (Z,\T_Z)$. 
        Then $\NbhFilter{\Topology{Z}{\T}}{x}$=$\NbhFilter{\Topology{Z}{\T}}{x}$, so $x \cong x$.
        Thus $\cong$ is 
		\ReflexiveRelation. 
        
        Let $x,y \in (Z,\T_Z)$. 
        Suppose $x \cong y$. 
        Then  $\NbhFilter{\Topology{Z}{\T}}{x} = \NbhFilter{\Topology{Z}{\T}}{y}$
        , so trivially  $\NbhFilter{\Topology{Z}{\T}}{y} =\NbhFilter{\Topology{Z}{\T}}{x}$
        , and thus $y \cong x$.
        Hence, $\cong$ is 
		\SymmetricRelation
        
        Let $x,y,z \in (Z,\T_Z)$.
        Let $x \cong y$ and $y \cong z$. 
        Then, 
         $\NbhFilter{\Topology{Z}{\T}}{x}= \NbhFilter{\Topology{Z}{\T}}{y} =  \NbhFilter{\Topology{Z}{\T}}{z}$
         so that $x \cong z$.
         Thus $\cong$ is \TransitiveRelation
         
         Since $\cong$ is 
		 \ReflexiveRelation
		, \SymmetricRelation
		, and \TransitiveRelation, it is an 
		\EquivalenceRelation. 
        
    \end{proof}
\end{prop}

\subsection{Equivalence Relations, Quotient Sets, and Quotient Maps}
\label{def:EquivalenceClass}
\newcommand{\EquivalenceClass}[0]{\textbf{\hyperref[def:EquivalenceClass]{Equivalence Class}}\xspace}
\newcommand{\EqClass}[2]{\bra{#1}_{\cong}\xspace}
\begin{df}[Equivalence Class]
    
    Let $X \neq \emptyset$.
    Let $\cong$ be an 
	\EquivalenceRelation
	defined on X.  
    Let $x \in X$. 
    We define the set $[x]_{\cong}$ by 
    \begin{equation}
        [x]_{\cong} = \{y \in X | y \cong x\}
    \end{equation} 
    We call $\EqClass{x}{\cong}$ the \EquivalenceClass of x in $(X, \cong)$. 
\end{df}

\begin{prop}[Equivalence Classes Partition]
    \label{prop:EquivalenceClassesPartition}
    
    Let $X \neq \emptyset$. 
    Let $\cong$ be an equivalence relation defined on X. 
    Let $x,y \in X$. 
    The following statements are equivalent. 
    \begin{enumerate}
        \item $[x]_{\cong}  \cap [y]_{\cong} \neq \emptyset$
        \item $x \cong y$
        \item $[x]_{\cong} = [y]_{\cong}$
        \item $[x]_{\cong} \subset [y]_{\cong}$
        \item $[y]_{\cong} \subset [x]_{\cong}$ 
    \end{enumerate}
    
\begin{proof}[Proof That $1 \implies 2$]
Suppose $M:=[x]_{\cong} \cap [y]_{\cong} \neq \emptyset$. 
Then there exists $z \in M$.
Then $z \cong x$, so by symmetry, $x \cong z$. 
But by transitivity, pair with $z \cong y$, we conclude $x \cong y$. 
\end{proof}
\begin{proof}[Proof That $2 \implies 4$]
    Let $x \cong y$ and let $z \in [x]_{\cong}$. 
    Then $z \cong x \cong y$, so $z \cong y$ and $z \in [y]_{\cong}$.
    Since z was arbitrary, we're done. 
\end{proof}
\begin{proof}[Proof That $2 \implies 5$]
    Let $x \cong y$. By symmetry, $y \cong x$, so by $(2 \implies 4)$, we are done. 
\end{proof}
\begin{proof}[Proof That $2 \implies 3$]
    Since  $2 \implies 4$ and $2 \implies 5$ and 5 and 4 together imply 3, we have this. 
\end{proof}
\begin{proof}[Proof That $5 \implies 1$]
    Let $[y]_{\cong} \subset [x]_{\cong}$. 
    Then $y \in [y]_{\cong} = [y]_{\cong} \cap [x]_{\cong} $.
    Hence 1 holds. 
\end{proof}

\end{prop}  
\newcommand{\QuotientSet}[0]{\textbf{\hyperref[def:QuotientSet]{Quotient Set}}\xspace}
\newcommand{\QuoSet}[2]{\ensuremath{#1/#2}\xspace}
\newcommand{\LetBeQuotientSet}[2]{
    Let \ensuremath{\QuoSet{#1}{#2}} be the \QuotientSet of \ensuremath{#1} with respect to the relation \ensuremath{#2}.
}
\begin{df}[Quotient Set]  
\label{def:QuotientSet}
\rm
    Let $X \neq \emptyset$.
    Let $\cong$ be an 
	\EquivalenceRelation defined on X.
    We define the set $X/\cong$ by 
    \begin{equation}
        \QuoSet{X}{\cong} = \braces{ [x]_{\cong} : x \in X}
    \end{equation}
    We call $\QuoSet{X}{\cong}$ the \QuotientSet of X under the relation $\cong$. 
\end{df} 

\begin{rmk}[Quotient Set Partition]
    \label{rmk:quotientsetpartition}
    
    By \ref{prop:EquivalenceClassesPartition}, $X/\cong$ is a partition of X. 
\end{rmk} 
\newcommand{\QuotientMap}[0]{\textbf{\hyperref[df:quotient_map]{Quotient Map}}\xspace}
    
 \newcommand{\QuotientMapInstance}[3]{ #1 : #2\to #2/#3 }
\begin{df}[Quotient Map]
\label{df:quotient_map}
\rm
    Let $X \neq \emptyset$.
    Let $\cong$ be an 
	\EquivalenceRelation 
	on X.
    \LetBeQuotientSet{X}{\cong}
    Define $T:X \to X/\cong$ by setting, for each $x \in X$, 
    \begin{equation}
        T(x)=[x]
    \end{equation}    
    We call T the \QuotientMap of X under $\cong$. 
\end{df} 

\label{prop:QuotientMapSurjective}
\begin{prop}[Quotient Map Surjective]
    Let $X \neq \emptyset$. 
    Let $\cong$ be an equivalence relation on X.
    Let $\QuotientMapInstance{T}{X}{\cong}$  be the \QuotientMap of X under the relation $\cong$. 
    Then T is a surjection. 
    \begin{proof}
       Let $K \in Z/\cong$. 
       Then for some $x \in Z$, $K=[x]$. 
       Then $T(x) = K$. 
       Since K was arbitrary, we are done. 
    \end{proof}
\end{prop} 

\subsection{Quotient Space Topology}
\label{def:QuotientSpaceTopology}
\newcommand{\QuotientSpaceTopology}[0]{
    \bf \hyperref[def:QuotientSpaceTopology]{Quotient Topology} \rm
}
\newcommand{\QuotientTopologicalSpace}[0]{
    \bf \hyperref[def:QuotientSpaceTopology]{Quotient Topological Space} \rm 
}

\begin{df}[Quotient Space Topology]
    Let $(Z,\T_Z)$ be a topological space. 
    Let $\cong$ be the \RelationOfEqualNeighborhoodFilters{$(Z, \T_Z)$}. 
    Let T be the \QuotientMap of Z under the relation $\cong$. 
    Define $\T_{Z/\cong}$ by
    \begin{equation}
        \T_{Z/\cong} = \left\{ \bigcup_{x \in U}\{T(x)\} \in 2^{Z/\cong}| U \in \T_Z \right\}
    \end{equation}
    By \ref{prop:QuotientSpaceTopology}, $\T_{Z/\cong}$ is a topology on $Z/\cong$.
    We call $\T_{Z/\cong}$ the \QuotientSpaceTopology and we call $\pa{Z/\cong, \T_{Z/\cong}}$ the \QuotientTopologicalSpace of $(Z, \T_Z)$.
    
\end{df}

\begin{prop}[Quotient Space Topology]
    \label{prop:QuotientSpaceTopology}
    
    Let $(Z,\T_Z)$ be a topological space 
    with \QuotientTopologicalSpace  $\pa{Z/\cong, \T_{Z/\cong}}$
    and \QuotientMap T.
    
    Then the following are true. 
    \begin{enumerate}
        \item $\T_{Z/\cong}$ is a topology on $Z/\cong$. 
        \item $T:(Z, \T_Z) \to (Z/\cong, \T_{Z/\cong})$ is continuous. 
        \item If U is open (closed) in $(Z,\T_Z)$ then $T(U)$ and $T(Z\setminus U)$ partition $Z/\cong$. 
        \item If U is open in $(Z, \T_Z)$, then $T^{-1}(T(U))=U$. 
        \item If K is closed in $(Z,\T_Z)$, then $T^{-1}T(K)=K$. 
        \item $T:(Z, \T_Z) \to (Z/\cong, \T_{Z/\cong})$ is an open mapping. 
        \item $T:(Z, \T_Z) \to (Z/\cong, \T_{Z/\cong})$ is a  closed mapping.
        \item $(Z, \T_Z)$ is a compact space if and only if $(Z/\cong, \T_{Z/\cong})$ is a compact space
        
    \end{enumerate} 
    \begin{proof}[Proof of 1]
        Since $\emptyset \in \T_Z$, we have 
        \begin{equation}
            \emptyset = \bigcup\limits_{x \in \emptyset} \{Tx\} \in \T_{Z/\cong}
        \end{equation}
        Since $Z \in \T_Z$, and by \ref{rmk:quotientsetpartition}, 
        \begin{equation} 
            Z/\cong = \bigcup_{x \in Z} \{[x]\}= \bigcup\limits_{x \in Z} \{T(x)\} \in \T_{Z/\cong}
        \end{equation} 
        
        Let $\{U_{\alpha} | \alpha \in A\} \subset \T_{Z/\cong}$. 
        For each $\alpha \in A$, there exists $B_{\alpha} \in \T_{Z}$ such that we have
        \begin{equation} 
            U_{\alpha } = \bigcup_{x \in B_{\alpha}} \{Tx\}
        \end{equation} 
        Since $\bigcup_{\alpha \in A} B_\alpha \in \T_{Z}$, we have 
        \begin{equation}
            \bigcup_{\alpha \in A} U_{\alpha}= \bigcup\limits_{\alpha \in A} \bigcup\limits_{x \in U_\alpha} \{T(x)\} = \bigcup\limits_{x \in \bigcup\limits_{\alpha \in A} B_{\alpha}} \{T(x)\} \in \T_{Z/\cong}
        \end{equation} 
        Let $\{U_i\}_{i=1}^n \subset \T_{Z/\cong}$. 
        For each $i \in \{1, ..., n\}$, there exists $B_i \in \T_{Z}$ such that
        \begin{equation}
            U_i = \bigcup_{x \in B_{i}} \{T(x)\}
        \end{equation}
        Suppose 
        \begin{equation}
            [x_0] \in \bigcap\limits_{i=1}^n \bigcup\limits_{x \in B_i} \{T(x)\}
        \end{equation}
        Then for each $i \in \{1,..., n\}$, there is a $y_i \in B_i$ such that $ y_i \cong x_0$. 
        Since each $B_i$ is open, the definition of $\cong$ implies that $x_0 \in B_i$ for every i. Hence, 
        \begin{equation} 
            x_0 \in \bigcap_{i=1}^n B_i
        \end{equation} 
        Implying 
        \begin{equation}
            [x_0] \in  \bigcup\limits_{x \in \bigcap\limits_{i=1}^n B_i} \{[x]\}
        \end{equation} 
        Hence, 
        \begin{equation} 
            \bigcap\limits_{i=1}^n \bigcup\limits_{x \in B_i} \{T(x)\}
            \subset
            \bigcup\limits_{x \in \bigcap\limits_{i=1}^n B_i} \{[x]\}
        \end{equation} 
        Furthermore, since the reverse inclusion is obvious, 
        and since $\bigcap_{i=1}^n B_i \in \T_{Z}$, we have 
        \begin{equation}
            \bigcap_{i=1}^n U_i = \bigcap_{i=1}^n \bigcup_{x \in B_i} \{T(x)\}= \bigcup\limits_{x \in \bigcap\limits_{i=1}^n B_i} \{T(x)\} \in \T_{Z/\cong}
        \end{equation}
    \end{proof}
    \begin{proof}[Proof of 2]
        Let $V \in \T_{Z/\cong}$. 
        Let $x_0 \in T^{-1}V$. 
        Then $[x_0] \in V$. 
        By definition, there is a $U \in \T_Z$ such that 
        \begin{equation}
            T(U) \subset \bigcup\limits_{x \in U} \{T(x)\}=V
        \end{equation}
        Hence there is a $y_0 \in U$  such that 
        \begin{equation}
            [x_0] \in T(y_0) = \{[y_0]\}
        \end{equation}
        Therefore, $x \cong y$. 
        Definition of the relation of equal neighborhood filters implies $\scU(x_0)=\scU(y_0)$. 
        Hence, $x_0 \in U \subset T^{-1}(V)$.
    \end{proof}
    \begin{proof}[Proof of 3]
        Let $K$ be closed in $(Z,\T_Z)$. 
        Then each point $x_0$ in $Z\setminus K$ has some $U_{x_0} \in \scU_{\T_Z}(x_0)$ which is disjoint from K.
        Hence $y_0 \not \cong x_0$ for any $y_0 \in K$, $x_0 \in Z\setminus K$. 
        Hence $T(K)$ is disjoint from $T\pa{Z \setminus K}$. 
        This fact, paired with \ref{prop:QuotientMapSurjective}, implies $T(Z\setminus K)$ and T(K) partition $Z/\cong$.
    \end{proof}
    \begin{proof}[Proof of 4]
        Let $U \in \T_Z$. 
        The nontrivial direction to prove is $T^{-1}\pa{T(U)} \subset U$.
        Let $y \in T^{-1}\pa{T(U)}$. 
        Then $[y]=Ty \in T(U)$.
        Hence, $[y]=T(x)=[x]$ for some $x \in U$. 
        Since $y \cong x$ and $x \in U \in \scU_{\T_Z}(x)$, we have $U \in \scU_{\T_Z}(y)$. 
        Hence $y \in U$.
        Since y was arbitrary, $T^{-1}\pa{T(U)} \subset U$, and equality is obvious because the other direction of inclusion is trivial. 
    \end{proof}
    \begin{proof}[Proof of 5]
        Let K be closed in $(Z,\T_Z)$. Part 3 Of this result implies $Z/\cong$ is partitioned by $T(K)$ and $T(Z\setminus K)$. 
        
        By part 4 of this proposition, 
        \begin{align*}
            T^{-1}\pa{T(K)}&=T^{-1} \pa{T(Z) \setminus T(Z \setminus K)} \\
            &= T^{-1}\pa{Z/\cong \setminus T(Z \setminus K)}\\
            &=T^{-1}(Z/\cong) \setminus T^{-1}(T(Z\setminus K)) \\
            &= Z \setminus \pa{Z \setminus K} \\
            &= K
        \end{align*}      
    \end{proof}
    \begin{proof}[Proof of 6]
        Let $U \in \T_Z$.
        Then by definition of the \QuotientSpaceTopology
        \begin{equation}
            TU= \bigcup_{x \in U} \{T(x)\}  \in \T_{Z/\cong}
        \end{equation}
    \end{proof}  
    \begin{proof}[Proof of 7] 
        Let K be closed in $(Z,\T_Z)$. 
        Then $Z \setminus K \in \T_Z$. 
        By Parts 3 and five of this proposition, we know $T(K) = Z/\cong \setminus T(Z\setminus K)$ and also that $T(Z\setminus K) \in \T_{Z/\cong}$. Hence $T(K)$ is closed in $(Z/\cong, \T_{Z/\cong})$. 
    \end{proof} 
    \begin{proof}[Proof of 8]
        Let $(Z,\T_Z)$ be compact. 
        Let $\{U_{\alpha}\}_{\alpha \in A}$ be an open covering of $(Z/\cong, \T_{Z/\cong})$. 
        Then $\{T^{-1}\pa{U_{\alpha}} | \alpha \in A\}$ is an open covering of $(Z, \T_Z)$. 
        Compactness of $(Z, \T_Z)$ guarantees the existence of a finite subcovering $\{T^{-1}\pa{U_{\alpha_i}} | i \in \{1, ..., n\}\}$. 
        Hence
        $\{U_{\alpha_i} | i \in \{1, ..., n\}\}=\{TT^{-1}(U_{\alpha_i}) | i \in \{1, ..., n\}\}$ is an open covering of $(Z/\cong, \T_{Z/\cong})$. 
         And the compactness of $(Z/\cong, \T_{Z/\cong})$ is verified. 
         
         
         Now, suppose $(Z/\cong, \T_{Z/\cong})$ is compact. 
         Let $\{V_{\beta} | \beta \in B\}$ be an open covering of $(Z, \T_Z)$. 
         Since T is an open mapping, $\{T(V_{\beta}) | \beta \in B\}$ is an open covering of $(Z/\cong, \T_{Z/\cong})$ which by compactness has a finite subcover $\{T(V_{\beta_i}) | i \in \{1, ..., n\}\}$. 
         By part 4 of \ref{prop:QuotientSpaceTopology}, 
         $\{V_{\beta_i}| i \in \{1, ..., n\}\} = \{T^{-1}(T(V_{\beta_i})) |i \in \{1, ..., n\}\}$ is then an open subcovering of $(Z, \T_Z)$. 
     %    
    \end{proof}
\end{prop} 
\subsection{Weak Topologies}

\subsection{Algebraic Structures}
\label{def:AlgebraicDeclarations}



\begin{df}[Algebraic Declarations Placeholder]
\end{df}

\label{def:Symmetricmap}
\newcommand{\SymmetricMap}[0]{
    \bf \hyperref[def:Symmetricmap]{Symmetric Map} \rm
}
\newcommand{\CommutativeFunction}[0]{
    \bf \hyperref[def:Symmetricmap]{Commutative} \rm
}
\newcommand{\FunctionCommutativity}[0]{
    \bf \hyperref[def:Symmetricmap]{Commutativity} \rm
}
\begin{df}[\CommutativeFunction]
    Let X and Y be sets. 
    We say that a map 
    $f:X \times X \to Y$ is a \SymmetricMap 
    if for each 
    $x_0,x_1 \in X$, 
    $f(x_0,x_1)=f(x_1,x_0)$.
    In this situation, 
    we may also refer to $f$ as
    \CommutativeFunction, 
    or say that $f$ posesses 
    \FunctionCommutativity.
\end{df} 

\label{def:Operation}
\newcommand{\BinaryOperation}[0]{
    \bf \hyperref[def:Operation]{Binary Operation} \rm
}
\newcommand{\BinaryOperations}[0]{
    \bf \hyperref[def:Operation]{Binary Operations} \rm
}
\newcommand{\UnaryOperation}[0]{
    \bf \hyperref[def:Operation]{Unary Operation} \rm
}
\newcommand{\UnaryOperations}[0]{
    \bf \hyperref[def:Operation]{Unary Operations} \rm
}

\newcommand{\Operation}[0]{
    \bf \hyperref[def:Operation]{Operation} \rm
}
\newcommand{\Operations}[0]{
    \bf \hyperref[def:Operation]{Operations} \rm
}

\begin{df}[\Operation, \UnaryOperation, \BinaryOperation]
    Let $X \neq \emptyset$
    be a set. 
    Let $A \neq \emptyset$
    be a set with
    $cardinality(A)=n \in \N$. 
    We call a mapping 
    \begin{equation*}
        T:\prod\limits_{\alpha \in A} X \to X
    \end{equation*}
    an 
    n-ary \Operation
    on $X$. 
    If $n=1$ 
    then we call $T$ a
    \UnaryOperation
    on 
    $X$. 
    If $n=2$, 
    then we call $T$ a 
    \BinaryOperation
    on $X$. 
    If $T$ is a 
    \BinaryOperation
    on $X$, 
    we sometimes use the notation
    \begin{equation*}
        xTy=T(x,y)
    \end{equation*}
\end{df}


\label{def:Magma}
\newcommand{\Magma}[0]{
    \bf \hyperref[def:Magma]{Magma} \rm
}
\newcommand{\Magmas}[0]{
    \bf \hyperref[def:Magma]{Magmas} \rm
}
\newcommand{\CommutativeMagma}[0]{
    \bf \hyperref[def:Magma]{Commutative Magma} \rm
}
\newcommand{\CommutativeMagmas}[0]{
    \bf \hyperref[def:Magma]{Commutative Magmas} \rm
}

\begin{df}[\Magma]
    Let $X$ be a set and
    $T:X \times X \to X$ be a 
    \BinaryOperation
    on $X$. 
    We call the pair $(X,T)$ a 
    \Magma.
    When it is clear what operation is being referred to, 
    we may simply refer to $X$
    as the 
    \Magma.
	If 
	$T$
	is
	\CommutativeFunction, 
	then we call 
	$(X,T)$ (or simply just $X$)
	a \CommutativeMagma.
	In general, this naming convention is used 
	for any algebraic structure defined on a set 
	via a \BinaryOperation with
	particular properties. 
\end{df}

\label{def:MagmaHomomorphism}
\newcommand{\MagmaHomomorphism}[0]{\textbf{\hyperref[def:MagmaHomomorphism]{Magma Homomorphism}}\xspace}
\newcommand{\MagmaHomomorphisms}[0]{\textbf{\hyperref[def:MagmaHomomorphism]{Magma Homomorphisms}}\xspace}
\newcommand{\scMagma}[0]{\textbf{\hyperref[def:MagmaHomomorphism]{Magma}}\xspace}
\newcommand{\Additive}[0]{\textbf{\hyperref[def:MagmaHomomorphism]{Additive}}\xspace}
\newcommand{\Additivity}[0]{\textbf{\hyperref[def:MagmaHomomorphism]{Additivity}}\xspace}
\begin{df}[\MagmaHomomorphism]
    Let $(X,\oplus_X)$
    and $(Y,\oplus_Y)$
    be \Magmas.
    Let $T:X \to Y$ satisfy, 
    for each $x_1, x_2 \in X$. 
    \begin{equation*}
        T\pa{x_1 \oplus_X x_2} = T\pa{x_1} \oplus_Y T\pa{x_2}
    \end{equation*}
    Then we call 
    T a \MagmaHomomorphism.
    We represent the collection of
    \MagmaHomomorphisms
    from $(X,\oplus_X)$ 
    to $(Y,\oplus_Y)$
    with 
    $H_{\scMagma}\pa{\pa{X, \oplus_X}, \pa{Y, \oplus_Y}}$, 
    or, when $\oplus_X$ and $\oplus_Y$ are clear, 
    $H_{\scMagma}\pa{X, Y}$. 
    A \MagmaHomomorphism
    is called \Additive
    and posseses the property 
    \Additivity. 
\end{df}

\label{def:IdentityElement}

\newcommand{\IdentityElement}[0]{\textbf{\hyperref[def:IdentityElement]{Identity Element}}\xspace}
\newcommand{\IdentityElements}[0]{\textbf{\hyperref[def:IdentityElement]{Identity Elements}}\xspace}
\newcommand{\LeftIdentityElement}[0]{\textbf{\hyperref[def:IdentityElement]{Left Identity Element}}\xspace}
\newcommand{\LeftIdentityElements}[0]{\textbf{\hyperref[def:IdentityElement]{Left Identity Elements}}\xspace}
\newcommand{\RightIdentityElement}[0]{\textbf{\hyperref[def:IdentityElement]{Right Identity Element}}\xspace}
\newcommand{\RightIdentityElements}[0]{\textbf{\hyperref[def:IdentityElement]{Right Identity Elements}}\xspace}

\begin{df}[\LeftIdentityElement, \RightIdentityElement]
    Let $(X,L)$ and
    $(X,R)$ be 
    \Magmas.
    Let $l , r \in X$ 
    such that
    for every $x \in X$ 
    we have 
   \begin{align*}
        lLx=x\\
        xRr=x
   \end{align*}
   In such a scenario, we say that
   $l$ is a \LeftIdentityElement 
   of $(X,L)$, and
   we say that 
   $r$ is a 
   \RightIdentityElement
   of $(X,R)$. 
\end{df}

\begin{df}[\IdentityElement]
    Let $(X,\oplus)$ be a 
    \Magma. 
    Let $e \in X$ be both a 
    \LeftIdentityElement 
    and a 
    \RightIdentityElement 
    of $\oplus$. 
    Then, we say that
    $e$ is an \IdentityElement of 
    $(X,\oplus)$. 
\end{df}



\label{def:UnitalMagma}
\newcommand{\UnitalMagma}[0]{
    \bf \hyperref[def:UnitalMagma]{Unital Magma} \rm
}
\newcommand{\UnitalMagmas}[0]{
    \bf \hyperref[dsef:UnitalMagma]{Unital Magmas} \rm
}
\newcommand{\CommutativeUnitalMagma}[0]{
    \bf \hyperref[def:UnitalMagma]{Commutative Unital Magma} \rm
}
\newcommand{\CommutativeUnitalMagmas}[0]{
    \bf \hyperref[dsef:UnitalMagma]{Commutative Unital Magmas} \rm
}
\begin{df}[\UnitalMagma]
    Let $(X,\oplus)$ be a
    \Magma. 
    Let $e$ be an 
    \IdentityElement 
    of 
    $(X,\oplus)$. 
    Then we call 
    $(X,\oplus,e)$ a
    \UnitalMagma.
    If it is unambiguous what 
    operation is being referred to, 
    as in the case of 
    \Magmas, 
    we may simply say 
    let $X$ be a 
    \UnitalMagma, or
    potentially 
    Let $(X,e)$ be a 
    \UnitalMagma. 
\end{df}

\label{def:UnitalMagmaHomomorphism}
\newcommand{\UnitalMagmaHomomorphism}[0]{
 `   \bf \hyperref[def:UnitalMagmaHomomorphism]{Unital Magma Homomorphism} \rm
}
\newcommand{\UnitalMagmaHomomorphisms}[0]{
    \bf \hyperref[def:UnitalMagmaHomomorphism]{Unital Magma Homomorphisms} \rm
}
\newcommand{\scUnitalMagma}[0]{
    \bf \hyperref[def:UnitalMagmaHomomorphism]{UMagma} \rm
}

\begin{df}[\UnitalMagmaHomomorphism]
    Let $(X, \oplus_X, e_X)$ and $(Y, \oplus_Y, e_Y)$ be 
    \UnitalMagmas and 
    $T:X \to Y$ be a \MagmaHomomorphism such that
    $T(e_X)=e_Y$. 
    Then we call $T$ a 
    \UnitalMagmaHomomorphism.
    We represent the set of 
    \UnitalMagmaHomomorphisms
    between $X$ and $Y$ with 
    $H_{\scUnitalMagma}(X, Y)$. 
    Obviously, $H_{\scUnitalMagma}(X,Y) \subset H_{\scMagma}(X,Y)$. 
\end{df}

\label{def:InverseElement}

\newcommand{\InverseElement}[0]{\textbf{\hyperref[def:InverseElement]{Inverse}}\xspace}
\newcommand{\InvertibleElement}[0]{\textbf{\hyperref[def:InverseElement]{Invertible}}\xspace}
\newcommand{\InverseElements}[0]{\textbf{\hyperref[def:InverseElement]{Inverses}}\xspace}
\newcommand{\LeftInverseElement}[0]{\textbf{\hyperref[def:InverseElement]{Left Inverse}}\xspace}
\newcommand{\LeftInvertibleElement}[0]{\textbf{\hyperref[def:InverseElement]{Left Invertible}}\xspace}
\newcommand{\LeftInverseElements}[0]{\textbf{\hyperref[def:InverseElement]{Left Inverses}}\xspace}
\newcommand{\RightInverseElement}[0]{\textbf{\hyperref[def:InverseElement]{Right Inverse}}\xspace}
\newcommand{\RightInvertibleElement}[0]{\textbf{\hyperref[def:InverseElement]{Right Invertible}}\xspace}
\newcommand{\RightInverseElements}[0]{\textbf{\hyperref[def:InverseElement]{Right Inverses}}\xspace}

\begin{df}[\LeftInverseElement, \RightInverseElement]
    Let $(X,\oplus,e)$ be a 
    \UnitalMagma.
    Let $l,r \in X$ such that 
    \begin{equation}
        l \oplus r=e
    \end{equation}
    In this scenario, we say that 
    $l$ is a 
    \LeftInverseElement 
    of $r$  
    in
    $(X,\oplus,e)$
    and we say that 
    $r$
    is a 
    \RightInverseElement
    of $l$ 
    in
    $(x,\oplus,e)$.
    Furthermore, we say that 
    $r$ is 
    \LeftInvertibleElement
    in
    $(X,\oplus,e)$
    and that 
    $l$ is 
    \RightInvertibleElement 
    in
    $(X,\oplus,e)$
\end{df}

\begin{df}[\InverseElement]
    Let $(X,\oplus,e)$ be a
    \UnitalMagma. 
    Let $x,y \in X$ such that
    $x$ is a \LeftInverseElement
    of $y$
    and $x$
    is a 
    \RightInverseElement
    of $y$. 
    Then, we say that 
    $x$ is an 
    \InverseElement
    of $y$
    in 
    $(X,\oplus, e)$
    and we say 
    $y$ an 
    \InvertibleElement
    element
    of
    $(X,\oplus,e)$. 
\end{df}



\label{def:AssociativeFunction}
\newcommand{\AssociativeFunction}[0]{\textbf{\hyperref[def:AssociativeFunction]{Associative}}\xspace}
\newcommand{\AssociativeOperation}[0]{\textbf{\hyperref[def:AssociativeFunction]{Associative}}\xspace}
\newcommand{\FunctionAssociativity}[0]{\textbf{\hyperref[def:AssociativeFunction]{Associativity}}\xspace}
\newcommand{\OperationAssociativity}[0]{\textbf{\hyperref[def:AssociativeFunction]{Associativity}}\xspace}
\begin{df}[\AssociativeOperation]
	Let $T$ be a 
	\BinaryOperation
	on a set $X$.
    We say that T is 
    \AssociativeFunction 
    and we say that T posses
    \FunctionAssociativity
    if for each $x,y,z \in X$, we have 
    \begin{equation*}
        T\pa{x,T\pa{y,z}}=T\pa{T\pa{x,y},z}
    \end{equation*}
\end{df}

\label{def:Semigroup}
\newcommand{\Semigroup}[0]{\textbf{\hyperref[def:Semigroup]{Semigroup}}\xspace}
\newcommand{\Semigroups}[0]{\textbf{\hyperref[def:Semigroup]{Semigroups}}\xspace}
\newcommand{\CommutativeSemigroup}[0]{\textbf{\hyperref[def:Semigroup]{Commutative Semigroup}}\xspace}
\newcommand{\CommutativeSemigroups}[0]{\textbf{\hyperref[def:Semigroup]{Commutative Semigroups}}\xspace}

\begin{df}[\Semigroup]
    Let $(X,\oplus)$ be a 
    \Magma.
    Let $\oplus$ 
    be 
    \AssociativeFunction.
    Then we say that 
    $(X,\oplus)$ 
    is a 
    \Semigroup. 
\end{df}

\label{def:Monoid}

\newcommand{\Monoid}[0]{\textbf{\hyperref[def:Monoid]{Monoid}}\xspace}
\newcommand{\Monoids}[0]{\textbf{\hyperref[def:Monoid]{Monoids}}\xspace}
\newcommand{\CommutativeMonoid}[0]{\textbf{\hyperref[def:Monoid]{Commutative Monoid}}\xspace}
\newcommand{\CommutativeMonoids}[0]{\textbf{\hyperref[def:Monoid]{Commutative Monoids}}\xspace}

\begin{df}[\Monoid]
    Let $(X,\oplus,e)$ be a 
    \UnitalMagma
    and let
    $(X,\oplus)$ 
    be a \Semigroup. 
    Then we call
    $(X,\oplus,e)$ 
    a 
    \Monoid.
\end{df}

\label{def:AlgebraicConsistentRelation}
\newcommand{\AlgebraicallyConsistent}[0]{
    \bf \hyperref[def:AlgebraicConsistentRelation]{Consistent} \rm
}
\newcommand{\AlgebraicConsistency}[0]{
    \bf \hyperref[def:AlgebraicConsistentRelation]{Consistency} \rm
}

\begin{df}[\AlgebraicallyConsistent]
    Let $(X,\oplus)$ be a 
    \Magma
    and let 
    $R$ be a 
	\Relation 
	on 
    $X$
    such that
    For each 
	$x_0, x_1 \in X$, 
    if 
	$x_0Rx_1$ and
    $y_0R y_1$, then
    $\pa{x_0\oplus y_0}R \pa{x_1 \oplus y_1}$
    Then we say that 
    $R$ is
    \AlgebraicallyConsistent
    with $(X,\oplus)$, 
    and we say that
    $R$
    posesses 
    \AlgebraicConsistency
    with respect to 
    $(X,\oplus)$. 
\end{df}



\label{def:OrderedMagma}
\newcommand{\PartiallyOrderedMagma}[0]{
    \bf \hyperref[def:OrderedMagma]{Partially Ordered Magma} \rm
}
\newcommand{\TotallyOrderedMagma}[0]{
    \bf \hyperref[def:OrderedMagma]{Totally Ordered Magma} \rm
}
\newcommand{\DirectedMagma}[0]{
    \bf \hyperref[def:OrderedMagma]{Directed Magma} \rm
}
\newcommand{\PartiallyOrderedMagmas}[0]{
    \bf \hyperref[def:OrderedMagma]{Partially Ordered Magmas} \rm
}
\newcommand{\TotallyOrderedMagmas}[0]{
    \bf \hyperref[def:OrderedMagma]{Totally Ordered Magmas} \rm
}
\newcommand{\DirectedMagmas}[0]{
    \bf \hyperref[def:OrderedMagma]{Directed Magmas} \rm
}

\begin{df}[\PartiallyOrderedMagma, \TotallyOrderedMagma, \DirectedMagma]
    Let $(X,\oplus)$ 
    be a 
    \Magma.
    Let $T$ be a 
    \TotalOrdering 
    on $X$ 
    which is 
    \AlgebraicallyConsistent
    with $(X,\oplus)$. 
    Let $P$ be a 
    \PartialOrdering 
    on $X$
    which is 
    \AlgebraicallyConsistent
    with $(X,\oplus)$. 
    Let $D$ be a 
    \Directing 
    on $X$
    which is 
    \AlgebraicallyConsistent
    with $(X,\oplus)$. 
    We call 
    $(X,\oplus, T)$ a 
    \TotallyOrderedMagma.
    $(X,\oplus, P)$ a 
    \PartiallyOrderedMagma.
    $(X,\oplus, D)$ a 
    \DirectedMagma.
\end{df}

\label{def:Group}

\newcommand{\Group}[0]{
    \bf \hyperref[def:Group]{Group} \rm
}
\newcommand{\Groups}[0]{
    \bf \hyperref[def:Group]{Groups} \rm
}
\newcommand{\CommutativeGroup}[0]{
    \bf \hyperref[def:Group]{Commutative Group} \rm
}
\newcommand{\CommutativeGroups}[0]{
    \bf \hyperref[def:Group]{Commutative Groups} \rm
}
\newcommand{\AbelianGroup}[0]{
    \bf \hyperref[def:Group]{Abelian Group} \rm
}
\newcommand{\AbelianGroups}[0]{
    \bf \hyperref[def:Group]{Abelian Groups} \rm
}

\begin{df}[\Group]
    Let $(X,\oplus,e)$ 
    be a \Monoid
    such that
    each 
    $x \in X$
    is an
    \InvertibleElement.
    Then we call
    $(X,\oplus,e)$ a 
    \Group. 
	Out of respect, we call a 
	\CommutativeGroup
	an
	\AbelianGroup.
\end{df}


\label{def:GroupInverseOperator}
\newcommand{\GroupInverseOperator}[0]{\textbf{\hyperref[def:GroupInverseOperator]{Group Inverse Operator}}\xspace}
\newcommand{\GroupInverseOperators}[0]{\textbf{\hyperref[def:GroupInverseOperator]{Group Inverse Operators}}\xspace}
\newcommand{\scGroupInverseOperator}[0]{\textbf{\hyperref[def:GroupInverseOperator]{\ensuremath{T^{-1}}}}\xspace}

\begin{df}[\GroupInverseOperator]
    Let $(X, \oplus, e)$ be a group.
    We denote with
    $\scGroupInverseOperator_G$ the 
    function defined as follows:
    $\scGroupInverseOperator_G:X \to X$, 
    \begin{equation*}
        \scGroupInverseOperator_G(x)=-x
    \end{equation*}

    We call 
    $\scGroupInverseOperator_G$
    the 
    \GroupInverseOperator
    of 
    $(X,\oplus,e)$. 
\end{df}


\label{def:TranslationOperator}
\newcommand{\RightTranslation}[0]{\textbf{\hyperref[def:TranslationOperator]{Right Translation}}\xspace}
\newcommand{\LeftTranslation}[0]{\textbf{\hyperref[def:TranslationOperator]{Left Translation}}\xspace}
\newcommand{\Translation}[0]{\textbf{\hyperref[def:TranslationOperator]{Translation}}\xspace}
\newcommand{\scRightTranslationOperator}[0]{\textbf{\hyperref[def:TranslationOperator]{\ensuremath{T^R}}}\xspace}
\newcommand{\scLeftTranslationOperator}[0]{\textbf{\hyperref[def:TranslationOperator]{\ensuremath{T^L}}}\xspace}
\newcommand{\scTranslationOperator}[0]{\textbf{\hyperref[def:TranslationOperator]{T}}\xspace}
\newcommand{\RightTranslationOperator}[0]{\textbf{\hyperref[def:TranslationOperator]{Right Translation Operator}}\xspace}
\newcommand{\LeftTranslationOperator}[0]{\textbf{\hyperref[def:TranslationOperator]{Left Translation Operator}}\xspace}
\newcommand{\TranslationOperator}[0]{\textbf{\hyperref[def:TranslationOperator]{Translation Operator}}\xspace}
\newcommand{\RightTranslationOperators}[0]{\textbf{\hyperref[def:TranslationOperator]{Right Translation Operators}}\xspace}
\newcommand{\LeftTranslationOperators}[0]{\textbf{\hyperref[def:TranslationOperator]{Left Translation Operators}}\xspace}
\newcommand{\TranslationOperators}[0]{\textbf{\hyperref[def:TranslationOperator]{Translation Operators}}\xspace}
%
%%%Vector Space Version
%
%\begin{df}[\TranslationOperator]
%    Let $V$ be a 
%    \VectorSpace
%    over a 
%    \Field $\F$. 
%    Let $\alpha \in \F$. 
%    We define $T_\alpha:V \to V$ by 
%    setting, for each 
%    $x \in V$, 
%    \begin{equation}
%    T_\alpha(x)=\alpha+x
%    \end{equation}
%    We call $T_\alpha$ 
%    the 
%    \TranslationOperator
%\end{df}


%
%%
%%%Magma Version
%%
%
\begin{df}[\TranslationOperator]
    Let $(G,\oplus)$ be a 
    \Magma. 
    Let $g \in G$. 
    We define 
    $\scRightTranslationOperator_g:G \to G$
    and
    $\scLeftTranslationOperator_g:G \to G$
    by setting, for each 
    $x \in G$, 
    \begin{equation*}
        \scRightTranslationOperator_g (x) = x \oplus g
    \end{equation*}
    \begin{equation*}
        \scLeftTranslationOperator_g (x) = g \oplus x
    \end{equation*}
    We call $\scRightTranslationOperator_g$
    the \RightTranslation
    of $(G,\oplus)$
    by $g$, 
    and we call 
    $\scLeftTranslationOperator_g$
    the \LeftTranslation
    of 
    $(G, \oplus)$ 
    by $g$. 
    If $\oplus$ is 
    \CommutativeFunction,
    then we define 
    $\scTranslationOperator_g=\scRightTranslationOperator_g=\scLeftTranslationOperator_g$
    which we call 
    \Translation of $(G, \oplus)$ by $g$.
\end{df}


\subsection{Topological Algebra}
%\input{./Math/Definitions/ch02/TopologicalMagma}
\label{def:TopologicalGroup}
\newcommand{\TopologicalGroup}[0]{
    \bf \hyperref[def:TopologicalGroup]{Topological Group}  \rm`
}
\begin{df}[\TopologicalGroup]
    Let $(G,+,e)$ be a \Group.
    \Topology on
    $G$ such that 
    $+:G \times G \to G$ is 
    \ContinuousFunction
    with respect to the 
    %\ProductTopology 


    and $g_{-1}$ is 
    \ContinuousFunction.

    In this scenario, we call $(G, \T)$ a \TopologicalGroup.
\end{df}

\label{def:LocalBasis}
\newcommand{\LocalBasis}[0]{
    \bf \hyperref[def:LocalBasis]{Local Basis} \rm
}
\begin{df}[\LocalBasis]
    Let $(G,\T)$ be a 
    \TopologicalGroup
    with \IdentityElement $e$.
    We call a 
    \NeighborhoodBasis of $\T$ about $e$
    a $\LocalBasis$ for $(G,\T)$. 
\end{df}



\subsection{Vector Spaces}
\label{def:vectorspace}
\newcommand{\VectorSpace}[0]{ \textbf{\hyperref[def:vectorspace]{Vector Space}} }
\newcommand{\VectorSpaces}[0]{ \textbf{\hyperref[def:vectorspace]{Vector Spaces}}}
\newcommand{\Field}[0]{ \textbf{\hyperref[def:vectorspace]{Field}} }
\newcommand{\Fields}[0]{ \textbf{\hyperref[def:vectorspace]{Fields}} }
\newcommand{\VectorSubspace}[0]{ \textbf{\hyperref[def:vectorspace]{Vector Subspace}} }
\newcommand{\VectorSubspaces}[0]{ \textbf{\hyperref[def:vectorspace]{Vector Subspaces}}}
\newcommand{\StandardBasis}[0]{ \textbf{\hyperref[def:vectorspace]{Standard Basis}}}
\newcommand{\StandardBases}[0]{ \textbf{\hyperref[def:vectorspace]{Standard Bases}}}

\label{def:scalarhomogeneous}
\newcommand{\ScalarHomogeneous}[0]{
    \bf \hyperref[def:scalarhomogeneous]{Scalar Homogeneous} \rm
}

\newcommand{\ScalarHomogeneity}[0]{
    \bf \hyperref[def:scalarhomogeneous]{Scalar Homogeneity} \rm
}
\begin{df}[Scalar Homogeneous]
    Let V be a 
	\VectorSpace over a 
	\Field	$\F \in \{\R, \C\}$. 
    We say that a map $p:V \to V$ is \ScalarHomogeneous, if
    , for each $\alpha \in \F$ and each $x \in V$, we have 
    \begin{equation}
        p(\alpha x) = \alpha p(x)
    \end{equation}
    Under these circumstances, we may instead say that the operator 
    p posesses \ScalarHomogeneity.
\end{df}

\label{def:absolutevaluescalarhomogeneous}
\newcommand{\AbsScalarHomogeneous}[0]{
    \bf \hyperref[def:absolutevaluescalarhomogeneous]{Absolutely Scalar Homogeneous} \rm
}

\newcommand{\AbsScalarHomogeneity}[0]{
    \bf \hyperref[def:absolutevaluescalarhomogeneous]{Absolute Scalar Homogeneity} \rm
}
\begin{df}[Scalar Homogeneous]
    Let V be a \VectorSpace over a \Field $\F \in \{\R, \C\}$. 
    We say that a map $p:V \to V$ is \AbsScalarHomogeneous, if
    , for each $\alpha \in \F$ and each $x \in V$, we have 
    \begin{equation}
        p(\alpha x) = \abs{\alpha} p(x)
    \end{equation}
    Under these circumstances, we may instead say that the operator 
    p posesses \AbsScalarHomogeneity.
\end{df}


\label{rmk:seminorm}
\begin{rmk}[\ScalarHomogeneous or \AbsScalarHomogeneous operator at 0 is 0]

If V is a 
\VectorSpace over a 
\Field 
$\mathbb{F} \in \{\R, \C\}$, then for each 
$x \in V$,
 $0x=0$.
Hence, if p is a \AbsScalarHomogeneous operator on v, then for any $x \in V$
\begin{equation}
p(0)=p(0x)=|0|p(x)=0p(x)=0
\end{equation}
If instead p is \ScalarHomogeneous operator on V, then we have
\begin{equation}
p(0)=p(0x)=0p(x)=0
\end{equation}
that is, in either case,  p(0)=0. 
\end{rmk}





\label{def:subadditive}
\newcommand{\Subadditive}[0]{
    \bf \hyperref[def:subadditive]{Subadditive} \rm
}

\newcommand{\Subadditivity}[0]{
    \bf \hyperref[def:subadditive]{Subadditivity} \rm
}
\begin{df}[\Subadditive]
Let $(G, \oplus_G)$ be a 
\Magma 
and 
$(H, \oplus_H, \leq)$ 
be a 
\PartiallyOrderedMagma. 
We call a mapping $p:G \to H$ \Subadditive if, for every $x,y \in G$, we have 
\begin{equation}
    p(x\oplus_G y) \leq p(x)\oplus_H p(y)
\end{equation}
Under these circumstances, 
we may also say that
$p$
 posesses $\Subadditivity$. 
\end{df}

\label{def:linear}
\newcommand{\Linear}[0]{
    \bf \hyperref[def:linear]{Linear} \rm
}
\newcommand{\Linearity}[0]{
    \bf \hyperref[def:linear]{Linearity} \rm
}

\begin{df}[\Linear]
    Let V, U be  
    \VectorSpaces
    over a 
    \Field
    $\F$. 
    We say that 
    $T:V \to U$ 
    is \Linear
    or that T possesses 
    \Linearity
    if T is both 
    \Additive
    and \ScalarHomogeneous
\end{df}
\label{def:VectorSpaceSpaceOfLinearOperators}
\newcommand{\SpaceOfLinearOperators}[0]{
    \bf \hyperref[def:VectorSpaceSpaceOfLinearOperators]{Space of Linear Operators} \rm
}

\begin{df}[\SpaceOfLinearOperators]
    Let $U,V$ be \VectorSpaces 
    over the same \Field
    $\F$. 
    We denote with 
    $L(U,V)$ 
    the set of \Linear
    operators $T:U \to V$. 
    We refer to $L(U,V)$ as the 
    \SpaceOfLinearOperators from 
    U to V.
    We endow $L(U,V)$ with 
    the operations of pointwise addition
    and pointwise scalar multiplication, 
    which the reader can verify makes 
    $L(U,V)$ into a \VectorSpace. 
\end{df}


\label{def:BalancedSet}
\newcommand{\Balanced}[0]{
    \bf \hyperref[def:BalancedSet]{Balanced} \rm 
}
\newcommand{\BalancedSet}[0]{
    \bf \hyperref[def:BalancedSet]{Balanced Set} \rm 
}
\newcommand{\BalancedSets}[0]{
    \bf \hyperref[def:BalancedSet]{Balanced Sets} \rm 
}

\begin{df}[\Balanced]
    Let $V$ be a \VectorSpace 
    over a \Field 
    $\F \in \{\R, \C\}$.
    Let $S \subset V$. 
    We call $S$ a \BalancedSet and
    we say that $S$ is 
    \Balanced if 
    for each
    $\alpha \in \F$ 
    with 
    $\abs{\alpha} \leq 1$
    we have 
    $\alpha S \subset S$. 
\end{df}

\label{def:AbsorbingSet}
\newcommand{\Absorbing}[0] {
    \bf \hyperref[def:AbsorbingSet]{Absorbing} \rm
}
\newcommand{\AbsorbingSet}[0] {
    \bf \hyperref[def:AbsorbingSet]{Absorbing Set} \rm
}
\newcommand{\AbsorbingSets}[0] {
    \bf \hyperref[def:AbsorbingSet]{Absorbing Sets} \rm
}
\newcommand{\Absorbed}[0] {
    \bf \hyperref[def:AbsorbingSet]{Absorbed} \rm
}
\newcommand{\Absorb}[0] {
    \bf \hyperref[def:AbsorbingSet]{Absorb} \rm
}
\newcommand{\Absorbs}[0] {
    \bf \hyperref[def:AbsorbingSet]{Absorbs} \rm
}

\begin{df}[\Absorbing]
    Let V be a \VectorSpace
    over a 
    \Field
    $\F \in \{\R, \C\}$. 
    Let $A, B \subset V$. 
    We say that $A$ 
    \Absorbs
    $B$ if 
    there exists a 
    $c > 0$ 
    such that
    for every $d \in \F$ 
    with $\abs{d} > c$ 
    we have 
    $B \subset dA$. 
    In such a Scenario, 
    $A$ is also said to 
    \Absorb $B$, 
    and we say that $B$ is 
    \Absorbed by $A$.
    If $A$ \Absorbs every Singleton in $V$, 
    then we call $A$ an 
    \AbsorbingSet or we say that $A$
    is \Absorbing. 
\end{df}

\label{def:ScalingOperator}
\newcommand{\ScalingOperator}[0] {
    \bf \hyperref[def:ScalingOperator]{Scaling Operator} \rm
}

\begin{df}[\ScalingOperator]
    Let $V$ be a 
    \VectorSpace 
    over a 
    \Field $\F$. 
    Let $\alpha \in \F$. 
    We define $M_\alpha:V \to V$ by 
    setting, for each 
    $x \in V$, 
    \begin{equation}
    M_\alpha(x)=\alpha x
    \end{equation}
    We call $M_\alpha$ the
    \ScalingOperator
\end{df}

\begin{prop}[\ScalingOperator]
    \label{prop:ScalingOperatorAlgebraicProperties}
    Let $V$
    be a 
    \VectorSpace
    over a 
    \Field
    $\F$. 
    The following are true:
    \begin{enumerate}
        \item If $\alpha, \beta \in \F$, then $M_\alpha \circ M_\beta = M_{\alpha * \beta}$. 
    \end{enumerate}


    \begin{proof}[Proof of 01]
        Let $v \in V$. Then 
        \begin{align*}
            M_{\alpha} \circ M_{\beta} v & = M_\alpha \pa{\beta * v } \\
            & = \alpha * (\beta * v) \\
            & = (\alpha * \beta) * v \\
            & = M_{\alpha*\beta}v
        \end{align*}
    \end{proof} 

\end{prop}

\label{def:Interval}
\newcommand{\Interval}[0]{
    \bf \hyperref[def:Interval]{Interval} \rm
}
\newcommand{\Intervals}[0]{
    \bf \hyperref[def:Interval]{Intervals} \rm
}
\newcommand{\ClosedInterval}[0]{
    \bf \hyperref[def:Interval]{Closed Interval} \rm
}
\newcommand{\ClosedIntervals}[0]{
    \bf \hyperref[def:Interval]{Closed Intervals} \rm
}
\newcommand{\OpenInterval}[0]{
    \bf \hyperref[def:Interval]{Open Interval} \rm
}
\newcommand{\OpenIntervals}[0]{
    \bf \hyperref[def:Interval]{Open Intervals} \rm
}
\newcommand{\HalfClosedInterval}[0]{
    \bf \hyperref[def:Interval]{Half-Closed Interval} \rm
}
\newcommand{\HalfClosedIntervals}[0]{
    \bf \hyperref[def:Interval]{Half-Closed Intervals} \rm
}
\newcommand{\HalfOpenInterval}[0]{
    \bf \hyperref[def:Interval]{Half-Open Interval} \rm
}
\newcommand{\HalfOpenIntervals}[0]{
    \bf \hyperref[def:Interval]{Half-Open Intervals} \rm
}
\begin{df}[\Interval]
    Let $V$ be a 
    \VectorSpace 
    over a 
    \Field
    $\F$. 
    Let $x,y \in V$. 
    We define the following sets:
    \begin{align*} 
        [x,y] = \{tx+(1-t)y | t \in [0,1] \} \\
        [x,y) = \{tx+(1-t)y | t \in [0,1) \} \\
        (x,y] = \{tx+(1-t)y | t \in (0,1] \} \\
        (x,y) = \{tx+(1-t)y | t \in (0,1) \}
    \end{align*}
    We refer to any of these sets as 
    \Intervals in $V$.
    Even in the absence of a topological structure, 
    we use the following language:
    \begin{enumerate}
        \item $[x,y]$ is called a \ClosedInterval.
        \item $(x,y)$ is called an \OpenInterval.
        \item $(x,y]$ and $[x,y)$ are called \HalfOpenIntervals or \HalfClosedIntervals.
    \end{enumerate}
\end{df}

\label{def:ConvexSet}
\newcommand{\ConvexSet}[0]{
    \bf \hyperref[def:ConvexSet]{Convex} \rm 
}

\begin{df}[\ConvexSet]
    Let $V$ 
    be a 
    \VectorSpace
    over a $\Field$
    $\F \in \{\R, \C\}$. 
    Let 
    $K \subset \F$. 
    We say that 
    $K$ is
    \ConvexSet
    if 
    for every pair $x,y \in K$, 
    we have $[x,y] \subset K$.
\end{df}


\subsection{Pseudometrics}
\label{def:TriangleInequality}
\newcommand{\TriangleInequality}[0]{
    \bf \hyperref[def:TriangleInequality]{Triangle Inequality} \rm
}
\begin{df}[Symmetric Map]
    
    Let X be a set and $(Y,+, \leq)$ be a totally ordered magma.
    We say that a map $f:X \times X \to Y$ satisfies the \TriangleInequality if for each $x_0,x_1,x_3 \in X$, we have
    \begin{equation*}
        f(x_0,x_2) \leq  f(x_0,x_1)+f(x_1,x_2)
        \end{equation*}
\end{df} 
\label{def:pseudometric}
\newcommand{\Pseudometric}[0]{
    \bf \hyperref[def:pseudometric]{Pseudometric} \rm
}
\newcommand{\PseudometricSpace}[0]{
    \bf \hyperref[def:pseudometric]{Pseudometric Space} \rm
}

\begin{df}[Pseudometric]
    Let $X \neq \emptyset$. 
    Let $d:X \times X \to [0,\infty)$ be a \SymmetricMap that satisfies the \TriangleInequality.
    Under these conditions we call d a \Pseudometric on X and we call $\pa{X,d}$ a \PseudometricSpace.
    \end{df} 
\label{def:pseudometriccauchysequence}
\newcommand{\PseudometricCauchySequence}[0]{
    \bf \hyperref[def:pseudometriccauchysequence]{Pseudometric Cauchy Sequence} \rm
}
\begin{df}[Pseudometric Cauchy Sequence]

    Let $(X,d)$ be a \PseudometricSpace.
    We say that a sequence $\{x_i\}_{i \in \N}$ is a \PseudometricCauchySequence
    if, for each $\epsilon > 0$, there exists an $N \in \N$, sucht that for 
    each pair $m,n \in \N$ such that $m>N$ and $n>N$, we have 
    \begin{equation}
        d(x_m,x_n) < \epsilon
    \end{equation}
\end{df}
\label{def:uniformlycauchy}
\newcommand{\UniformlyCauchy}[0]{
    \bf \hyperref[def:uniformlycauchy]{Uniformly Cauchy} \rm
}
\begin{df}[Uniformly Cauchy]
	Let $(X_\alpha, d_\alpha)$ be a \PseudometricSpace
	for $\alpha \in A$ where A is some indexing set. 
	For each $\alpha \in A$
	, let $\phi_\alpha :=\{x_i^\alpha\}_{i \in \N} \subset X_{\alpha}$
	be a sequence. 
	We say that the collection $\{\phi_\alpha\}_{\alpha \in A}$ 
	is 
	\UniformlyCauchy if for each $\epsilon > 0$, there exists an 
	$N \in \N$ such that for each pair $m,n \in N$
	such that $m>N$ and $n>N$, and for each $\alpha \in A$, 
	we have 
	\begin{equation}
	d_{\alpha} \pa{x^{\alpha}_n, x^{\alpha}_m} < \epsilon
	\end{equation}
\end{df}
\label{def:pseudometricsequenceconvergence}
\newcommand{\PseudometricConvergence}[0]{\textbf{\hyperref[def:pseudometricsequenceconvergence]{Pseudometric-Convergence}}\xspace}
\newcommand{\PseudometricConvergent}[0]{\textbf{\hyperref[def:pseudometricsequenceconvergence]{Pseudometrically-Convergent}}\xspace}
\newcommand{\PseudometricConverges}[0]{\textbf{\hyperref[def:pseudometricsequenceconvergence]{Pseudometric-Converges}}\xspace}
\begin{df}[Pseudometric Convergence]
    Let $(X,d)$ be a \PseudometricSpace.
	Let $\{x_i\}_{i \in \N}$ be a sequence in $(X,d)$.
    Let $x_0 \in X$.  
    We say that 
	$\{x_i\}_{i \in \N}$ 
	exhibits 
	\PseudometricConvergence 
	to 
	$x_0$ 
	in d,
	or we say that 
	$\{x_i\}_{i \in \N}$  
	\PseudometricConverges 
	to 
	$x_0$ 
	in d, 
	or we say that 
	$\{x_i\}_{i \in \N}$ 
	is 
	\PseudometricConvergent 
	to 
	$x_0 \in d$ 
	if, 
    for every 
	$\epsilon > 0$, 
	there is an 
	$N \in \N$ 
	such that for every 
	$n>N$, 
	we have 
    \begin{equation}
        d(x_0, x_n) < \epsilon
    \end{equation}
\end{df}

\begin{prop}[Convergent Implies Cauchy]
\label{prop:pseudometricconvergenceimpliespseudometriccauchy}

    Let $(X,d)$ be a
    \PseudometricSpace.
    Let $\{x_i\}_{i \in \N}$ be a 
    \PseudometricConvergent sequence. 
    Then $\{x_i\}_{i \in \N}$
    is a \PseudometricCauchySequence.

    \begin{proof}
        Since $\{x_i\}$ converges, let 
        $x_i \to x$. 
        Let $\epsilon > 0$. 
        Then there exists $N \in \N$ 
        such that for $n>N$, we have
        $d(x_i, x) < \frac{\epsilon}{2}$. 
        For this N, if $m,n > N$, then we have 
        \begin{equation}
        d(x_m,x_n) \leq d(x_m,x) + d(x,x_n) < \epsilon
        \end{equation}
        and so the sequence is a
        \PseudometricCauchySequence, as advertised. 
    \end{proof}
\end{prop}

\label{def:uniformlycauchy}
\newcommand{\UniformlyCauchy}[0]{
    \bf \hyperref[def:uniformlycauchy]{Uniformly Cauchy} \rm
}
\begin{df}[Uniformly Cauchy]
	Let $(X_\alpha, d_\alpha)$ be a \PseudometricSpace
	for $\alpha \in A$ where A is some indexing set. 
	For each $\alpha \in A$
	, let $\phi_\alpha :=\{x_i^\alpha\}_{i \in \N} \subset X_{\alpha}$
	be a sequence. 
	We say that the collection $\{\phi_\alpha\}_{\alpha \in A}$ 
	is 
	\UniformlyCauchy if for each $\epsilon > 0$, there exists an 
	$N \in \N$ such that for each pair $m,n \in N$
	such that $m>N$ and $n>N$, and for each $\alpha \in A$, 
	we have 
	\begin{equation}
	d_{\alpha} \pa{x^{\alpha}_n, x^{\alpha}_m} < \epsilon
	\end{equation}
\end{df}

\label{def:uniformlyconvergent}
\newcommand{\UniformlyConvergent}[0]{\textbf{\hyperref[def:uniformlyconvergent]{Uniformly Convergent}}\xspace}
\newcommand{\ConvergesUniformly}[0]{\textbf{\hyperref[def:uniformlyconvergent]{Converges Uniformly}}\xspace}
\newcommand{\UniformConvergence}[0]{\textbf{\hyperref[def:uniformlyconvergent]{Uniform Convergence}}\xspace}

\begin{df}[Uniform Convergence]
	Let $(X_\alpha, d_\alpha)$ be a \PseudometricSpace
	for $\alpha \in A$ where A is some indexing set. 
	For each $\alpha \in A$
	, let $\phi_\alpha :=\{x_i^\alpha\}_{i \in \N} \subset X_{\alpha}$
	be a sequence. 
	We say that the collection $\{\phi_\alpha\}_{\alpha \in A}$ 
    is \UniformlyConvergent to 
    $\{x_\alpha\}_{\alpha \in A} \in \prod\limits_{\alpha \in A} X_\alpha$
    if for each $\epsilon > 0$, 
    there is an $N \in \N$
    such that for each $n>N$, 
    and for every $\alpha \in A$, 
    we have 
    \begin{equation}
        d_\alpha(x^{\alpha}_i,x_\alpha) < \epsilon
    \end{equation}

    In this scenario, we may equivalently say that
    $\{\phi_\alpha\}$ demonstrates \UniformConvergence
    to $\{x_\alpha\}_{\alpha \in A}$ 
    or that it \ConvergesUniformly. 

    When we mention \UniformConvergence without
    reference to what the convergence is to, 
    we are merely claiming the existence of 
    such a limit. 
\end{df}

\label{prop:uniformlycauchyplusconvergenceimpliesuniformconvergence}
\begin{prop}[Uniform Cauchy and Pointwise Convergence implies Uniform Convergence]

	Let $(X_\alpha, d_\alpha)$ be a \PseudometricSpace
	for $\alpha \in A$ where A is some indexing set. 
	For each $\alpha \in A$
	, let $\phi_\alpha :=\{x_i^\alpha\}_{i \in \N} \subset X_{\alpha}$
	be a sequence. 
    Suppose the collection $\{\phi_\alpha\}_{\alpha \in A}$ 
    is \UniformlyCauchy
    and that each $\phi_\alpha$ 
    is \PseudometricConvergent
    , say $x_i^{\alpha} \to x_\alpha$. 
    Then $\{\phi_\alpha\}_{\alpha \in A}$
    is \UniformlyConvergent
    to $\{x_\alpha\}_{\alpha \in A}$. 
    \begin{proof}
        Let $\epsilon > 0$. 
        Then, since $\{\phi_\alpha\}_{\alpha \in A}$ 
        is \UniformlyCauchy, 
        there is an 
        $N \in \N$
        such that 
        for $m, n>N$, we have
        $d_{\alpha}(x^{\alpha}_n,x^{\alpha}_m) < \frac{\epsilon}{2}$. 
        Also, since each $\phi_\alpha$ converges to $x_\alpha$, 
        there are $N_{\alpha} \in \N$. 
        such that for any $n_\alpha > N_\alpha$, 
        we have
        $d_\alpha(x^{\alpha}_{n_\alpha}, x_\alpha) < \frac{\epsilon }{2}$. 
        Define $M_{\alpha}=max(N+1, N_{\alpha}+1)$ for $\alpha \in A$. 
        Let $n>N$. 
        Then, for any $\alpha \in A$, we have. 
        \begin{align*}
            d_{\alpha}(x_n^{\alpha} , x_\alpha) & \leq d_{\alpha}(x_n^{\alpha} , x^{\alpha}_{M_{\alpha}}) + d_{\alpha}(x_{M_{\alpha}}, x_\alpha)\\
            & < \frac{\epsilon}{2}+\frac{\epsilon}{2} \\
            & = \epsilon
        \end{align*}
        completing the proof. 



    \end{proof} 
\end{prop}


\label{def:pseudometriccomplete}
\newcommand{\PseudometricComplete}[0]{
    \bf \hyperref[def:pseudometriccomplete]{Pseudometric-Complete} \rm
}
\newcommand{\Complete}[0]{
    \bf \hyperref[def:pseudometriccomplete]{Complete} \rm
}
\begin{df}[Pseudometric Complete]
    We say that a \PseudometricSpace $(X,d)$ is 
    \PseudometricComplete if each 
	\PseudometricCauchySequence 
	sequence in $(X,d)$ 
	\PseudometricConverges to a limit in $X$.


	In the case that d is a \Metric, then
	being \PseudometricComplete is 
	equivalent to beging \Complete
	in the classical sense, so 
	we will commonly refer to a \PseudometricSpace
	which is \PseudometricComplete as simply
	being \Complete. 
    \end{df}
\label{def:pseudometricball}
\newcommand{\OpenBall}[0]{
    \bf \hyperref[def:pseudometricball]{Open Ball} \rm
}
\newcommand{\ClosedBall}[0]{
    \bf \hyperref[def:pseudometricball]{Closed Ball} \rm
}
\begin{df}[Pseudometric Ball]
    Let $(X,d)$ be a \PseudometricSpace. 
    For each $x_0  \in X$ and each $\epsilon > 0$, we define the following.
    \begin{enumerate}
        \item  $B_d(x_0, \epsilon) := \{y \in X | d(x_0,y) < \epsilon\}$ denotes the \OpenBall about $x_0$ with radius $\epsilon$. 
    \item $\overline{B_d}(x_0,\epsilon) := \{y \in X | d(x_0,y) \leq \epsilon \}$ denotes the \ClosedBall about $x_0$ with radius $\epsilon$. 
    \end{enumerate} 
    
     
    \end{df} 
\label{def:pseudometrictopology}
\newcommand{\PseudometricTopology}[0]{
    \bf \hyperref[def:pseudometrictopology]{Pseudometric Topology} \rm
}
\newcommand{\PseudometricInducedTopology}[0]{
    \bf \hyperref[def:pseudometrictopology]{Pseudometric Topology} \rm
}
\begin{df}[Pseudometric Topology]
    Let $(X,d)$ be a \PseudometricSpace, and let $\scB$ be the set of \OpenBall's in $(X,d)$. 
    By \ref{prop:pseudometrictopology}, $\scB$ is the basis for a unique topology $\T_d$ on X. 
    We call $\T_d$ the \PseudometricInducedTopology induced by $d$ on X. 

\end{df}
\label{prop:pseudometrictopology}
\begin{prop}[Pseudometric Topology]
    Let $(X,d)$ by  \PseudometricSpace and let $\scB$ be the set of \OpenBall's in $(X,d)$. 
    The following are true. 
    \begin{enumerate}
        \item There exists a unique topology $\T_d$ on X which $\scB$ is a basis of. That is, the \PseudometricTopology $\T_d$ is well defined. 
        \item The \PseudometricInducedTopology is first countable. That is, each of its points permits a countable neighborhood basis. 
    \end{enumerate}
    \begin{proof}[Proof of 1]
        Uniqueness is guaranteed by closure under arbitrary unions of a topology. 
        For existense, it is sufficient to show that the collection of arbitrary unions
        of elements of $\scB$ is closed under finite intersections. 
        Suppose that for $1\leq i \leq n$, we have $\{U_{\alpha_i} | \alpha_i \in A_i\} \subset \scB$
        and consider the set
        \begin{equation}
            U=\bigcap_{i=1}^n \bigcup_{\alpha_i \in A_i} U_{\alpha_i}
        \end{equation}
        Let $x_0 \in U$. 
        For each $i \in \{1, ..., n\}$, there exists $\alpha_i \in A_i$ such that 
        \begin{equation}
            x_0 \in U_{\alpha_i} = B_d(x_i; \epsilon_i)
        \end{equation}
        For each $i \in \{1, ..., n \}$, define $\delta_i = d(x_0, x_i)$. Then $0 < \delta_i < \epsilon_i$. 
        Then, for each $i \in \{1, ..., n \}$, 
        \begin{equation}
            B_d(x_0; \epsilon_i-\delta_i) \subset U_{\alpha_i} \subset \bigcup_{\alpha_i \in A_i} U_{\alpha_i}
        \end{equation}
        Define 
        \begin{equation}
            \delta_{x_0} = \min\limits_{i=1}^n \pa{ \epsilon_i-\delta_i}
        \end{equation}
        Then $x_0 \in B(x_0; \delta_{x_0} ) \subset U$. 
        If $U=\{x_{\alpha} | \alpha \in A\}$, then the arbitrary nature of $x_0$ above means 
        we can repeat this construction, writing 
        \begin{equation}
            U \subset \bigcup_{\alpha \in A} B(x_{\alpha} ; \delta_{x_{\alpha}} )\subset \bigcup_{\alpha \in A} U = U
        \end{equation}
        Hence, $U \in B$ and the proof is complete. 
    \end{proof}
    \begin{proof}[Proof of 2]
        Let $x_0 \in X$. 
        I claim that 
        \begin{equation}
            \scB_{x_0}:= \left\{ B_d\pa{x_0; \frac{1}{n}} | n \in \N\right\}
        \end{equation}
        is a neighborhood basis for $(X,\T_d)$ at $x_0$. 
        Let $U \in \scU_{\T_d}(x)$ be open in $\T_d$. 
        Since $\scB$ is a basis for $\T_d$, for some $y0 \in X$ and $\epsilon > 0$, 
        $x_0 \in B_d(y_0; \epsilon) \subset U$. 
        Let $\delta = d(x_0, y_0)$. Then $\epsilon - \delta > 0$. 
        Define
        \begin{equation}
            n = \ceil{ \frac{1}{\epsilon - \delta}}
        \end{equation}
        Then we have 
        \begin{equation}
            B_d\pa{x_0 ; \frac{1}{n}} \subset B_d(x_0 : \epsilon - \delta) \subset B(y_0 ; \epsilon) \subset U
        \end{equation}
    \end{proof}
\end{prop}

\label{def:relationofzerodistance}
\newcommand{\RelationOfZeroDistance}[0]{
    \bf \hyperref[def:relationofzerodistance]{Relation Of Zero Distance} \rm
}
\begin{df}[Relation Of Zero Distance]
    Let $(X,d)$ be a \PseudometricSpace. 
    Define the relation  $\cong_d$ on $X \times X$ by setting, for $x,y \in X$, 
    \begin{equation}
        x \cong_d y \iff d(x,y) = 0
    \end{equation}
    We call $\cong_d$ the \RelationOfZeroDistance on $(X,d)$. 
\end{df} 

\begin{prop}[Relation Of Zero Distance is the Relation Of Equal Neighborhood Filters]
    \label{prop:relationofzerodistance}
    Let $(X,d)$ be a \PseudometricSpace.
    Let $\cong_{\T_d}$ be the \RelationOfEqualNeighborhoodFilters $(X,\T_d)$. 
    Let $\cong_d$ be the \RelationOfZeroDistance on $(X,d)$. 
    Then $\cong_{\T_d} = \cong_d$. 
    \begin{proof}
        Let $x,y \in X$ and suppose $x_0 \cong_d y_0$.
        Let $U \in \scU_{\T_d}(x_0)$. Then for some $\epsilon > 0$, 
        $x_0 \in B(x_0;\epsilon) \subset U$. 
        Since $x_0 \cong_d y_0$, $d(x_0,y_0) = 0$, so $y_0 \in B(x_0 ; \epsilon) \subset U$. 
        Hence $U \in \scU_{\T_d}(y_0)$. 
        The arbitrary nature of $U \in \scU_{\T_d}(x_0)$ implies 
        \begin{equation}
            \scU_{\T_d}(x_0) \subset \scU_{\T_d}(y_0)
        \end{equation}
        A reverse construction would just as easily show the reverse inclusion, so we conclude that $x_0 \cong_{\T_d} y_0$. 
        Now suppose $x_0 \cong_{\T_d} y $. Then or each $n \in \N$, 
        \begin{equation}
            y_0 \in B_{d} \pa{x_0 ; \frac{1}{n}}
        \end{equation}
        Hence $d(x_0, y_0) < \frac{1}{n}$ for each natural n, therefore $d(x_0,y_0) = 0$ and $x_0 \cong_d y_0$. 
    \end{proof}
\end{prop}

\newcommand{\PseudometricInducedMetric}[0]{
    \bf \hyperref[def:pseudometricinducedmetric]{Pseudometric Induced Metric} \rm
}
\newcommand{\MetricInducedByPseudometric}[0]{
    \bf \hyperref[def:pseudometricinducedmetric]{Metric Induced By The Pseudometric} \rm
}
\begin{df}[Metric Space Induced By Pseudometric]
    \label{def:pseudometricinducedmetric}
    Let $(X,d)$ be a \PseudometricSpace, and let $\cong$ be the \RelationOfZeroDistance, which by \ref{prop:relationofzerodistance} is also the \RelationOfEqualNeighborhoodFilters $(X,\T_d)$. 
    Define $\tilde{d}: X/\cong \to [0,\infty)$ by 
    \begin{equation}
        \tilde{d}\pa{\bra{x}, \bra{y}} = d(x,y)
    \end{equation}
    By \ref{prop:pseudometricinducedmetric}, $\tilde{d}$ is well defined and is in fact a metric on $X/\cong$, so we call $\tilde{d}$ the \MetricInducedByPseudometric d on X, or we call it the \PseudometricInducedMetric of $(X,d)$. 
\end{df}
\begin{prop}[Metric Space Induced By Pseudometric Space]
    \label{prop:pseudometricinducedmetric}
    %Let $X$, d, $\cong$, and $\tilde{d}$ be defined as in \ref{def:pseudometricinducedmetric}
    Let $(X,d)$ be a \PseudometricSpace, $\cong$ the \RelationOfZeroDistance on $(X,d)$ and $\tilde{d}$ be defined as in \ref{def:pseudometricinducedmetric}.
    Let $(X/\cong, \T_{X/\cong})$ be the  \QuotientTopologicalSpace with \QuotientMap T, and let $(X/\cong, \T_{\tilde{d}})$ be the topological space induced by the metric space $(X/\cong, \tilde{d})$. 
    The following are true. 
    \begin{enumerate}
        \item $\tilde{d}$ is in fact well defined, and is a metric on $X/\cong$, justifying calling it the \MetricInducedByPseudometric d.
        \item $\T_{X/\cong} = \T_{\tilde{d}}$
        \item T is an isometric surjection $(X,d)$ to $(X/\cong, \tilde{d})$
        \item $(X/\cong, \tilde{d})$ is complete if and only if $(X, d)$ is \PseudometricComplete.

    \end{enumerate}
    \begin{proof}[Proof of 01]
        First we show that $\tilde{d}$ is well defined as a mapping, that is, that if
        $x_0,y_0 \in X$ and $x_1 \cong x_0$ and $y_1 \cong y_0$, then we should have 
        \begin{equation}
            \tilde{d}\pa{\bra{x_0},\bra{y_0}}=\tilde{d}\pa{\bra{x_1},\bra{y_1}}
        \end{equation}
        
        This is easy, as
        \begin{align*}
            d(x_0,y_0) & \leq d(x_0,x_1)+d(x_1,y_1)+d(y_1,y_0)\\
            & = d(x_1,y_1)\\
            & \leq d(x_1,x_0)+d(x_0,y_0)+d(y_0,y_1)\\
            &=d(x_0,y_0)
         \end{align*}
         Nonnegativity falls directly from the nonnegativity of d. 
         Proving that $\tilde{d}$ is a \SymmetricMap is equally trivial
         \begin{align*}
             \tilde{d}\pa{\bra{x}, \bra{y}}= d(x,y) = d(y,x) = \tilde{d}\pa{\bra{y}, \bra{x}}
         \end{align*}
         Proving that $\tilde{d}$ satisfies the \TriangleInequality is similarly simple, letting $x_0,y_0,z_0 \in X$, we have
         \begin{align*}
             \tilde{d}\pa{\bra{x_0}, \bra{z_0}} & = d(x_0,z_0) \\
             & \leq d(x_0, y_0)+d(y_0, z_0)\\
             & = \tilde{d}\pa{\bra{x_0}, \bra{y_0}}+ \tilde{d}\pa{ \bra{y_0}, \bra{z_0}}
         \end{align*}
         
         All that remains is to show positivity on nonequal arguements. Let $x_0, y_0 \in X$ such that $\bra{x_0} \neq \bra{y_0}$. Then $x_0 \not \cong y_0$. Hence \begin{equation*}
             \tilde{d}\pa{\bra{x_0}, \bra{y_0}}=d(x_0,y_0) \neq 0
             \end{equation*}
    \end{proof}
    \begin{proof}[Proof of 02]
        By \ref{prop:QuotientSpaceTopology}, part 9, $\scB_{\cong}:=\{T(B_d(x;\epsilon)) | x \in X, \epsilon > 0\}$ is a basis for $\T_{X/\cong}$. 
        By definition, $\scB_{\tilde{d}}:=\{B_{\tilde{d}}(\bra{x}; \epsilon) | x \in X , \epsilon > 0 \}$ is a basis for $\T_{\tilde{x}}$. 
        
        I claim that for each $x \in X$ and $\epsilon > 0$, 
        \begin{equation}
            T\pa{B_d(x;\epsilon)} = B_{\tilde{d}} \pa{\bra{x}; \epsilon}
        \end{equation}
        To see this, 
        suppose $\tilde{y} \in T\pa{B_d(x;\epsilon)}$. 
        Then $\tilde{y}=T(y)$ for some $y \in B_d(x;\epsilon)$. 
        Hence 
        \begin{align*}
            \tilde{d}(\tilde{y},\bra{x})& =\tilde{d}(T(y),\bra{x})\\
            &=\tilde{d}(\bra{y},\bra{x})\\
            & = d(y,x) \\
            & < \epsilon
        \end{align*}
        Hence $\tilde{y} \in B_d(\bra{x} ; \epsilon)$, and so 
                \begin{equation}
            T\pa{B_d(x;\epsilon)} \subset  B_{\tilde{d}} \pa{\bra{x}; \epsilon}
        \end{equation}
        Suppose $\bra{y} \in B_{\tilde{d}}\pa{\bra{x} ; \epsilon}$. 
        Then $d(x,y) = \tilde{d}()\bra{x},\bra{y}) < \epsilon$, so $y \in B_d(x; \epsilon)$. 
        Hence $[y]=T(y) \in T\pa{B_d(x;\epsilon)}$, so the reverse inclusion also holds, and so the above claim holds. 
        This, paired witht he fact that 
        \begin{equation}
            \{[x] | x \in X\}= X/\cong
        \end{equation}
        finishes the result. 
        
    \end{proof}
    \begin{proof}[Proof of 03]
        Falls directly from the definition $T(x)=\bra{x}$, hence
        \begin{equation}
            d(x,y) = \tilde{d}\pa{\bra{x}, \bra{y}} = \tilde{d}\pa{T(x), T(y)}
        \end{equation}
        
        T is surjective by \ref{prop:QuotientMapSurjective}.
    \end{proof}
    \begin{proof}[Proof of 04]
        Let $(X,d)$ be \PseudometricComplete. 
        Let $\{[x_i]\}_{i \in \N} \subset (X/\cong, \tilde{d})$ be a \PseudometricCauchySequence. 
        Let $\epsilon > 0$. 
        Then there exists $N \in \N$ such that for $m,n > N$, we have 
        \begin{equation}
            d(x_m, x_n) = \tilde{d}(Tx_m, Ty_m) =\tilde{d}([x_m], [x_n]) < \epsilon
        \end{equation}
        So the sequence $\{x_i\}_{ i \in \N} \subset (X,d)$ is \PseudometricCauchySequence. 
        Since $(X,d)$ is \PseudometricComplete, this sequence has a limit, say $x_i \to x \in (X,d)$. 
        But, we have $[x_i]=Tx_i \to Tx = [x]$, so $\{[x_i]\}$ is convergent, and since that sequence was arbitrary, $(X/\cong, \tilde{d})$ is \PseudometricComplete. 
        
        Let $(X/\cong, \tilde{d})$ be \PseudometricComplete. 
        Let $\{x_i\} \subset X$ be a \PseudometricCauchySequence.
        Let $\epsilon > 0$. Then there exist $N \in \N$ such that for $m,n > N$, we have
        \begin{equation}
            \tilde{d}\pa{[x_m], [x_n]} = \tilde{d}(Tx_m, Tx_n) = d(x_m, x_n) < \epsilon
        \end{equation}
        so that $\{[x_i]\}_{i \in \N}$ is also a \PseudometricCauchySequence. 
        Since $(X/\cong, \tilde{d})$ is \PseudometricComplete, this sequence has a limit, say $[x_i] \to y \in X/\cong$. 
        Since T is surjective, for some $x \in X$, $Tx \in y$, and so
        \begin{equation}
            d(x, x_i) = \tilde{d}(Tx, Tx_i) =\tilde{d}(y, [x_i]) \to 0
        \end{equation}
        meaning $x_i \to x$ and we are done. 
                
\end{proof}
\end{prop}
\newcommand{\Pseudometrizable}[0]{\textbf{\hyperref[def:Pseudometrizable]{Pseudometrizable}}\xspace}
\newcommand{\Metrizable}[0]{\textbf{\hyperref[def:Pseudometrizable]{Metrizable}}\xspace}
\begin{df}[(Pseudo)Metrizable]
    \label{def:Pseudometrizable}
    Let $(X,\T)$ be a topological space. 
    \begin{enumerate}
        \item We say that $(X,\T)$ (Or $\T$ or X which it wouldn't cause confusion) is \Pseudometrizable if there exists a pseudometric d on X such that $\T$ is the \PseudometricInducedTopology on $(X,d)$. 
        \item We say that $(X,\T)$ (Or $\T$ or X when it wouldn't cause confusion) is \Metrizable if there exists a metric d on X such that $\T$ is the metric topology on $(X,d)$. 
    \end{enumerate}
\end{df}

\begin{prop}[Pseudometrizable Prequotient]
    \label{prop:pseudometrizableprequotient}
    Let $(X,\T_X)$ be a topological space 
    with \QuotientTopologicalSpace  $\pa{X/\cong, \T_{X/\cong}}$
    and \QuotientMap T. Let $\pa{X/\cong, \T_{X/\cong}}$ be \Pseudometrizable with \Pseudometric $\tilde{d}$. 
    
    The following hold. 
    \begin{enumerate}
        \item  $(X,\T_X)$ is \Pseudometrizable. 
        \item $(X/\cong, \T_{X/\cong})$ is \Metrizable. 
        \item If T is injective, then $(X,\T_X)$ is metrizable. 
    \end{enumerate}
    \begin{proof}[Proof Of One]
        Define $d:X \times X \to [0,\infty)]$ by 
        \begin{equation*}
            d(x,y) = \tilde{d}\pa{[x], [y]}.
        \end{equation*}
        Then
        \begin{equation*}
            d(x,y) =\tilde{d}([x],[y]) \in [0,\infty)
        \end{equation*}
        so that d is well defined. 
        
        Also, 
        \begin{equation*}
            d(x,y) = \tilde{d}([x],[y])=\tilde{d}([y],[x])=d(y,x)
        \end{equation*}
        , so d is a \SymmetricMap.
        
        Also, 
        \begin{align*}
            d(x,z) & = \tilde{d}([x],[z])\\
            & \leq \tilde{d}([x],[y])+\tilde{d}([y], [z])\\
            & = d(x,y)+d(y,z)
        \end{align*}
        so d satisfies the \TriangleInequality. 
        Also, 
        \begin{equation}
            d(x,x)=\tilde{d}([x],[x])=0
        \end{equation}
        and so d is a \Pseudometric on X. 
        
        Let $\T_d$ denote the \PseudometricTopology on $(X,d)$. What remains to show is that $\T_X=\T_d$. 
        
        
           By \ref{def:pseudometricinducedmetric} paired with how d is defined, $\tilde{d}$ is the \PseudometricInducedMetric of $(X,d)$. Let $\cong_d$ denote the \RelationOfZeroDistance on $(X,d)$, and 
           let $\cong_{\T_X}$ denote the \RelationOfEqualNeighborhoodFilters on $(X,\T_X)$. 
           
           %claim: $\cong_d=\cong_{\T_X}$ to use a theormem. 

    \end{proof} 
    \begin{proof}[Proof of Two]
    \end{proof}
    \begin{proof}[Proof of Three]
    \end{proof} 
\end{prop} 


%\input{./Math/Definitions/ch02/`

\subsection{Topological Vector Spaces} 
\label{def:VectorSpaceCompatible}
\newcommand{\VectorSpaceCompatible}[0]{\textbf{\hyperref[def:VectorSpaceCompatible]{Compatible}}\xspace}
\newcommand{\VectorSpaceCompatibility}[0]{\textbf{\hyperref[def:VectorSpaceCompatible]{Compatibility}}\xspace}
\begin{df}[\VectorSpaceCompatible]
    Let 
    $(V, +, \cdot, 0)$
    be a 
    \VectorSpace
    over $\F$
    and $\T$ be a 
    \Topology
    on $V$ such that 
    $(V,+,\T)$
    is a
    \TopologicalGroup
    and
    $\cdot:\F \times V \to V$
    is 
    \ContinuousFunction.
    Then we say that 
    $\T$
    is
    \VectorSpaceCompatible
    with
    $(V, +, \cdot, 0)$, 
    or when $+$ and $\cdot$ are obvious, 
    we say that 
    $\T$ 
    is 
    \VectorSpaceCompatible
    with 
    $\T$. 
\end{df}


\label{def:topologicalvectorspace}
\newcommand{\TVS}[0]{
    \bf \hyperref[def:topologicalvectorspace]{Topological Vector Space} \rm
}

\begin{df}[\TVS]
Let $(V,+,\cdot, 0)$ be a 
\VectorSpace over a \Field $\F \in \{\R, \C\}$. 
Let $\T$ be a 
\TopologyRef on V such that 
$(V, +, 0)$ is a 
\TopologicalGroup
and 
$\cdot: \F \times V \to V$ 
is
\Continuous. 
Then we call $(V,\T)$ a \TVS. 
\end{df}

\label{def:LocallyConvex}
\newcommand{\LocallyConvex}[0]{
    \bf \hyperref[def:LocallyConvex]{Locally Convex} \rm
}
\newcommand{\LocalConvexity}[0]{
    \bf \hyperref[def:LocallyConvex]{Local Convexity} \rm
}

\begin{df}[\LocallyConvex]
    We say that a 
    \TVS
    $(X,\T)$ is 
    \LocallyConvex 
    if $(X,\T)$ has a 
    \LocalBasis consisting only of 
    \ConvexSet sets.
    A \LocallyConvex 
    space is said to posess
    \LocalConvexity.
\end{df}


\begin{prop}[Existence of \Balanced \NeighborhoodBasis of 0 in a \TVS]
    \label{prop:ExistenceOfBalancedNeighborhoods}
    Let $(X,\T)$ be a 
    \TVS
    over a 
    \Field
    $\F$.
    The following are True. 
    \begin{enumerate}
        \item If 
            $U \in \scU_{\T}(0)$
            , then there is a 
            \BalancedSet
            $V \subset U$
            such that 
            $V \in \scU_{\T}(0)$.
        \item There exists a 
            \NeighborhoodBasis
            about $0 \in X$ 
            for $\T$ 
            consisting entirely 
            of \BalancedSet sets. 
        \item If 
            $U \in \scU_{\T}(0)$
            %is \ConvexSet
            , then there is a 
            \BalancedSet
            %, \ConvexSet
            $V \subset U$
            such that 
            $V \in \scU_{\T}(0)$.
        \item If $(X,\T)$ is 
            \LocallyConvex, 
            then there exists a 
            \NeighborhoodBasis
            about $0 \in X$ 
            for $\T$ 
            consisting entirely 
            of 
            \BalancedSet
            %, \ConvexSet 
            sets.
    \end{enumerate}

    \begin{proof}[Part 01] TODO
    \end{proof}
    \begin{proof}[Part 02] TODO
    \end{proof}
    \begin{proof}[Part 03] TODO
    \end{proof}
    \begin{proof}[Part 04] TODO
    \end{proof}
\end{prop}

\label{def:topologicalvectorspaceboundedset}
\newcommand{\TVSBounded}[0]{
    \bf \hyperref[def:topologicalvectorspaceboundedset]{TVS-Bounded} \rm
}

\begin{df}[TVS Bounded Set]
Let $(V,\T)$ be a 
\TVS.
Let $A \subset V$. 
We say that A is \TVSBounded with respect to $\T$,
or when confusion is unlikely we simply say that A is \TVSBounded
if for every $U \in \scU_{\T}(0)$, there exists an $\alpha \in \F$
, $\alpha > 0$
, such that $A \subset \alpha U$. 
\end{df}

\label{def:boundedlinearoperatorinatvs}
\newcommand{\BLO}[0]{
    \bf \hyperref[def:boundedlinearoperatorinatvs]{Bounded Linear Operator} \rm
}

\begin{df}[TVS Bounded Linear Operator]
Let $(V_i,\T_i)$ be a \TVS over $\F \in \{\R, \C\}$ for $i \in \{0,1\}$. 
We say that a linear operator $T:(V_1, \T_1) \to (V_2, \T_2)$ is a \BLO
if for each $U \in V_1$ with U \TVSBounded with respect to $\T_0$, 
$TU$ is \TVSBounded with respect to $\T_1$. 
\end{df}

\label{def:TVSSpaceOfContinuousLinearOperators}
\newcommand{\SpaceOfContinuousLinearOperators}[0]{\textbf{\hyperref[def:TVSSpaceOfContinuousLinearOperators]{Space Of Continuous Linear Operators}}\xspace}
\begin{df}[\SpaceOfContinuousLinearOperators]
    Let 
    $(U, \T_U)$
    and $(V, \T_V)$
    each be a \TVS
    over the same $\Field$
    $\F \in \{\R, \C\}$. 
    Let $L(U,V)$ denote the \SpaceOfLinearOperators
    from $U$
    to $V$.
    We denote with 
    $CL((U, \T_U), (V, \T_V))$ 
    the subset of 
    $L(U, V)$ consisting only of the 
    \ContinuousFunction operators. 
    When $\T_U$ and $\T_V$ are understood, 
    we may denote
    $CL((U, \T_U), (V, \T_V))=CL(U, V)$ 
\end{df}

\begin{rmk}[\SpaceOfContinuousLinearOperators is a \VectorSubspace]
    Let 
    $(U, \T_U)$
    and $(V, \T_V)$
    each be a \TVS
    over the same $\Field$
    $\F \in \{\R, \C\}$. 
    Let $L(U,V)$ denote the \SpaceOfLinearOperators
    from $U$
    to $V$.
    Let $CL(U,V)$ denote the \SpaceOfContinuousLinearOperators
    from U to V. 
    Then $CL(U,V)$
    is a \VectorSubspace of 
    $L(U,V)$. 
\end{rmk}

\label{def:TopologyOfUniformConvergence}
\newcommand{\TopologyOfUniformConvergence}[0]{
    \bf \hyperref[def:TopologyOfUniformConvergence]{Topology of Uniform Convergence} \rm
}
\begin{df}[\TopologyOfUniformConvergence]
    Let X be a set and 
    $(Y, \T_Y)$ be a \TVS.
    Let $\NbhFilter{\T_Y}{0}$ denote the 
    \NeighborhoodFilter of $0 \in (Y, \T_Y)$.
    Suppose $\scF$ is a 
    \VectorSubspace of the set of 
    functions  $T:X \to Y$. 
    Suppose
    $\scG \subset 2^X$  such that 
	$(\scG, \subset)$ is a \DirectedSet.
    For each $x \subset X$ and $y \subset Y$, and  define 
    $M(x, y) = \{f \in \scF | f(x) \subset y\}$
    Now we define 
    $\T(\scF, \T_Y, \scG)= \{f+ M(x,y) | x \in \scG \wedge y \in \NbhFilter{\T_Y}{0} \wedge f \in \scF\}$.
	We call $\T(\scF, \T_Y, \scG)$ the \TopologyOfUniformConvergence of $\scF$ on $\scG$ with respect to $\T_Y$. 
	When $\scF$, $\T_Y$ or $\scG$ are understood they may be omitted from the reference. 
	By \ref{prop:TopologyOfUniformConvergence}, $\T$ is a 
	\TopologyRef on $\scF$. 
\end{df}
\begin{prop}[\TopologyOfUniformConvergence]
\label{prop:TopologyOfUniformConvergence}
	
\end{prop}

%\begin{prop}[Bounded Linear Operator Continuous]
\label{prop:boundedlinearoperatorsarecontinuous}
Let $(V_i,\T_i)$ be a \TVS over a field $\F \in \{\R, \C\}$ for $i \in \{0,1\}$
Let $T:V_0 \to V_1$ be a \BLO. 
Then T is continous. 
\begin{proof} 
    Let $0_{V_1} \in U \in \T_1$. 
\end{proof} 
\end{prop}
 Not actually True.. In general a Sequentiqally continuous Operator is  is bounded, but bounded only implies continuous if trhe domain is a "BOrnological" space (Defined by bourbaki gives notion of boundeds grounding)

\subsection{Seminormed Spaces}


\label{def:seminorm}
\newcommand{\Seminorm}[0]{
    \bf \hyperref[def:seminorm]{Seminorm} \rm
}\newcommand{\Seminorms}[0]{
    \bf \hyperref[def:seminorm]{Seminorms} \rm
}
\newcommand{\NonDegenerate}[0]{
	\bf \hyperref[def:seminorm]{Non-Degenerate} \rm
}
\newcommand{\Degenerate}[0]{
	\bf \hyperref[def:seminorm]{Degenerate} \rm
}
\label{def:seminormedspace}
\newcommand{\SeminormedSpace}[0]{
    \bf \hyperref[def:seminormedspace]{Seminormed Space} \rm
}
\begin{df}[Seminorm]
    Let V be a vector space over a field $\F \in \{ \R, \C\}$.  
    We say that a map $\norm{\cdot}:V \to [0,\infty)$ is a \Seminorm on V 
	if it is both \Subadditive and \ScalarHomogeneous. 
	In this case, we refer to $(V, \norm{\cdot})$ as a \SeminormedSpace. 
	We say that $\norm{\cdot}$ is \NonDegenerate if there is at least one $v \in V$ with $\norm{v}>0$. 
	We say that $\norm{\cdot}$ is \Degenerate if it is not \NonDegenerate.  
	We may also refer to the \SeminormedSpace $(V, \norm{\cdot})$ as being
	\Degenerate
	or
	\NonDegenerate. 
\end{df} 





\label{def:norm}
\newcommand{\Norm}[0]{
    \bf \hyperref[def:norm]{Norm} \rm
}
\label{def:normedspace}
\newcommand{\NormedSpace}[0]{
    \bf \hyperref[def:normedspace]{Normed Space} \rm
}
\begin{df}[Norm]
    Let $(V,\norm{\cdot})$ be a \SeminormedSpace.
    If the following implication is true for $x \in V$, then we refer to $\norm{\cdot}$ as a \Norm on V, and we call $(V, \norm{\cdot})$ a \NormedSpace.
    \begin{equation}
    x \neq 0 \implies \norm{x} \neq 0
    \end{equation}
\end{df}

\begin{prop}[Subadditive Operator On a Group Induces a Metric]
    \label{prop:subadditiveinducestriangleinequality}
    Let $(G,+, e)$ be a group and let $(H,+,\leq)$ be a totally ordered magma. 
    Let $p:G \to H$ be \Subadditive. 
    define $d:G \times G \to H$ by setting, for each $x,y \in G$, 
    \begin{equation}
        d(x,y) =  p(x+(-y))
    \end{equation}

    Then d satisfies the triangle inequality. 

    \begin{proof}
    let $x,y, z \in G$. Then
    \begin{align*}
        d(x,z) &= p(x+(-z))\\
        & = p(x+e+(-z))\\
        & = p(x+(-y)+y+(-z))\\
        & \leq p(x+(-y))+p(y+(-z))\\
        & = d(x,y)+d(y,z)
    \end{align*}
    completing the proof. 
    \end{proof} 
\end{prop}
 
\newcommand{\SeminormTopology}[0]{\textbf{\hyperref[def:seminormtopology]{Seminorm Topology}}\xspace}
\newcommand{\SeminormInducedPseudometric}[0]{\textbf{\hyperref[def:seminormtopology]{Pseudometric induced by the Seminorm}}\xspace}
\newcommand{\SeminormSpaceInducedPseudometricSpace}[0]{\textbf{\hyperref[def:seminormtopology]{Pseudometric Space induced by the Seminormed Space}}\xspace}

\begin{df}[Seminorm Topology]
\label{def:seminormtopology}
\rm
    Let $(X,\norm{\cdot})$ be a \SeminormedSpace.
    define $d_{\norm{\cdot}}:V \times V \to [0,\infty)$  by setting,
    for $x,y \in X$, 
    \begin{equation}
    d_{\norm{\cdot}}(x,y) = \norm{x-y}
    \end{equation}
    Observe the following: 
    \begin{enumerate}
        \item \ref{rmk:seminorm} guarantees that $d_{\norm{\cdot}}(x,x)=0$ for $x \in X$. 
        \item 
        \ref{prop:subadditiveinducestriangleinequality} guarantees that d satisfies the \TriangleInequality. 
        \item d is a \SymmetricMap, as we have 
    \begin{equation}
        d(x,y)_{\norm{\cdot}}=\norm{x-y}=|-1|\norm{x-y}=\norm{y-x}=d(y,x)
    \end{equation}
    \end{enumerate}

    Hence, $d_{\norm{\cdot}}$  is a \Pseudometric on X, which we call the \SeminormInducedPseudometric on X. 
    We refer to $(X, d_{\norm{\cdot}})$ as the \SeminormSpaceInducedPseudometricSpace $(X,\norm{\cdot}$. 
    We refer to the \PseudometricTopology induced by $d_{\norm{\cdot}}$ as the \SeminormTopology induced by $\norm{\cdot}$, and unless otherwise specified, when we reference $(X,\norm{\cdot})$, we consider it to be endowed with this topology. 

\end{df}

\label{def:seminormkernel}
\newcommand{\SeminormKernel}[0]{
    \bf \hyperref[def:seminormkernel]{Seminorm Kernel} \rm
}
\newcommand{\SeminormKernels}[0]{
    \bf \hyperref[def:seminormkernel]{Seminorm Kernels} \rm
}
\newcommand{\Ker}[0]{
   \bf\mathcal{K}\rm^{ernel}
}


\begin{df}[Seminorm Kernel]
Let $(V, \norm{\cdot})$ be a \SeminormedSpace. 
Define the set $\Ker_{(V,\norm{\cdot})}$ by 
\begin{equation}
\Ker_{(B,\norm{\cdot})}=\{x \in V | \norm{x}=0\}
\end{equation}
We call this set the \SeminormKernel of the space $\Ker_{(V,\norm{\cdot})}$. 
When confusion is unlikely, we may denote this set with
$\Ker$, $\Ker_V$, or even $\Ker_{\norm{\cdot}}$, or we may just refer to it
as the \SeminormKernel, the \SeminormKernel of $V$, or the \SeminormKernel of $\norm{\cdot}$. 
\end{df}

\begin{prop}[Seminorm Kernel is a vector Subspace]
\label{prop:seminormkernelisavectorsubspace}
    Let $(X,\norm{\cdot})$ be a \SeminormedSpace over a field $\F \in \{\R, \C\}$  
    with corresponding \SeminormKernel $\Ker$. 
    Then the following are true. 
    \begin{enumerate}
        \item $\Ker$ is a vector subspace of X. 
        \item $\Ker$ is closed in the \SeminormTopology on X.
        \item By part 1 of this result, $\Ker$ is a subgroup of the additive structure of $X$. Hence, we can talk about the 
    \end{enumerate}


    \begin{proof}[Proof of One]
        \Subadditivity implies that, if $x,y \in \Ker$, then $\norm{x+y} \leq \norm{x}+\norm{y}=0$. 
        By \ScalarHomogeneity, if $x \in \Ker$  and $\alpha \in \F$, $\norm{\alpha x} =|\alpha| \norm{x}=0$
        so $\Ker$ is in fact a vector subspace of X. 
    \end{proof}
    \begin{proof}[Proof of Two]
        
        If $x \in X \setminus  \Ker$
        then $\norm{x} = \alpha > 0$ for some positive $\alpha$. 
        Hence $B(x;\alpha/2)$ is an open set containing x disjoint from $\Ker$. 
       We can then write $X \setminus \Ker$ as the union of all such open sets to see that $\Ker$ is closed. 
    \end{proof}
\end{prop}

\label{def:equivalencemodseminormkernel}
\newcommand{\EquivelanceModKernel}[0]{
    \bf \hyperref[def:equivalencemodseminormkernel]{Equivalence MOD-$\Ker$} \rm
}

\newcommand{\EquivalentModKernel}[0]{
    \bf \hyperref[def:equivalencemodseminormkernel]{Equivalent MOD-$\Ker$} \rm
}
\newcommand{\SeminormKernelQuotientVectorSpace}[0]{
    \bf \hyperref[def:equivalencemodseminormkernel]{Seminorm Kernel Quotient Vector Space} \rm
}

\begin{df}[Quotient Space Mod Kernel]
Let $(X,\norm{\cdot})$ be a \SeminormedSpace over a field $\F \in \{\R,\C\}$.
with \SeminormKernel $\Ker$.
By \ref{prop:seminormkernelisavectorsubspace}, part 1, 
$\Ker$ is a vector subspace of $X's$ algebraic structure, and so if we define 
$\cong_{\Ker} \subset X \times X$ by setting, for $x,y \in X$
\begin{equation}
x \cong_{\Ker} y \iff x-y \in \Ker
\end{equation}
Then one recognizes $\cong_{\Ker}$ as \EquivelanceModKernel as would be commonly spoken of in Module or Vector Space theory. 
From this, alot of nice properties fall out. We list them here, without proof just to nail down notation. 
For proof, see any undergraduate algebra text.
\begin{enumerate}
%\item We denote the \QuotientSpace $X/\cong$ with $X/\Ker$. 
\item If $x \cong_{\Ker} y$, then we say that x and y are \EquivalentModKernel. 
\item For $x \in X$
    , we denote the \EquivalenceClass $[x]_{\cong_{\Ker}}$ with 
    $[x]_{\Ker}$ or 
    with $x+\Ker$, or 
    when confusion is unlikely, simply $[x]$. 
\item We denote $X/\cong_{\Ker}$ with $X/\Ker$. 
\item If we define $\oplus:X/\Ker \times X/\Ker \to X/\Ker$ by setting
    , for $x,y \in X$, $[x]_{\Ker}\oplus[y]_{\Ker}=[x+y]_{\Ker}$
    , then $\oplus$ is well defined and endows $X/\Ker$ with a group structure. 
\item If we further define $\odot:\F \times X/\Ker \to X/\Ker$ by 
    $\alpha [x]_{\Ker}=[\alpha x]_{\Ker}$
    , then $\pa{X/\Ker, \oplus, \odot, [0]_{\Ker}}$ is a Vector space over $\F$. 
\item Unless otherwise specified
    , when referring to the set $X/\Ker$
    , we endow it with the above vector space structure
    , and we call this space the \SeminormKernelQuotientVectorSpace 
    of the seminormed space $(X, \norm{\cdot})$.
\end{enumerate}
\end{df}


\label{prop:equivalencemodkernelispseudometricequivalence}
\begin{prop}[Equivalence Mod Kernel is Pseudometric Equivalence]
    Let $(X,\norm{\cdot})$ be a seminromed space.
    with \SeminormKernel $\Ker$.
    Let $d$ denote the \SeminormInducedPseudometric.
	Let $\cong_{d}$ denote the 
	\RelationOfZeroDistance with respect to d. 
    
    Then $\cong_{\Ker}=\cong_{d}$. 
    \begin{proof}
        Let $x,y \in X$ and let $x \cong_{\Ker}y$.
        Then, since $x-y \in \Ker$, 
        Then $d(x,y) := \norm{x-y} =0$, so $x \cong_d y$. 
        Hence $\cong_{\Ker} \subset \cong_{d}$ 


        Now let $x,y \in X$ with $x \cong_d y$. 
        Then $\norm{x-y}=d(x,y) = 0$, so $x-y \in \Ker$
        , and therefore $x \cong_{\Ker} y$. 
        Hence, $\cong_{d} \subset \cong_{\Ker}$. 

        Since inclusion goes both directions, $\cong_{\Ker} = \cong_d$.

    \end{proof} 
\end{prop}

\label{def:quotientnormspace}
\newcommand{\QuotientNorm}[0]{
    \bf \hyperref[def:quotientnormspace]{Quotient Norm} \rm
}
\newcommand{\QuotientNormedSpace}[0]{
    \bf \hyperref[def:quotientnormspace]{Quotient Normed Space} \rm
}

\begin{df}[Quotient Norm Space]
Let $(X,\norm{\cdot})$ be a \SeminormedSpace
with \SeminormInducedPseudometric $d$, 
\SeminormKernel $\Ker$, and
\SeminormKernelQuotientVectorSpace $X/\Ker$.
Let $\tilde{d}:X/\Ker \times X/\Ker \to [0,\infty)$ be the \MetricInducedByPseudometric.

Define $\norm{\cdot}_{\Ker} : X/\Ker \to [0,\infty)$ by 
\begin{equation}
\norm{[x]}_{\Ker} = \tilde{d}([x], [0])
\end{equation}

By $\ref{prop:quotientnormspace}$, $(X/\Ker, \norm{\cdot}_{\Ker})$ is a normed space which we call the \QuotientNormedSpace of $(X,\norm{\cdot})$, and we call $\norm{\cdot}_{\Ker}$ the \QuotientNorm. 
Whenever we refer to $X/\Ker$, unless otherwise specified, we endow it with this norm and the topology generated by this norm.
Furthermore, whenever we consider $X/\Ker$, unless otherwise specified, we consider it as 
possesing the topology generated by the norm $\norm{\cdot}_{\Ker}$. 
\end{df}

\begin{prop}[Quotient Normed Space]
\label{prop:quotientnormspace}

Let $(X,\norm{\cdot})$ be a \SeminormedSpace
with \SeminormInducedPseudometric $d$, 
\SeminormKernel $\Ker$, and
\SeminormKernelQuotientVectorSpace $X/\Ker$.
Let $\tilde{d}:X/\Ker \times X/\Ker \to [0,\infty)$ be the \MetricInducedByPseudometric.
Let $T:X \to X/\Ker$ denote the \QuotientMap of X into $X/\Ker$ 
(Recalling that the 
\RelationOfEqualNeighborhoodFilters equals the 
\RelationOfZeroDistance equals the relation of 
\EquivelanceModKernel), so they would all produce the same quotient map)
Let $\norm{\cdot}_{\Ker}$ denote the \QuotientNorm.

The following are true. 
\begin{enumerate}
\item $\norm{\cdot}_{\Ker}$ is a norm on $X/\Ker$. 
\item $\tilde{d}$  is the \SeminormInducedPseudometric $\norm{\cdot}_{\Ker}$, and thus they produce the same topology. 
\item T has all of the properties described in $\ref{prop:QuotientSpaceTopology}$. 
\item T is Linear.
\item T is Surjective. 
\item T is an isometry. 
\item T is injective if and only if $\norm{\cdot}$ is a norm. 
\begin{proof}[Proof of 1]
    First, note that 
    $Range(\norm{\cdot}_{\Ker}) \subset Range(\tilde{d}) \subset [0,\infty)$,\
    so that $\norm{\cdot}_{\Ker}$ has the correct domain and codomain. 
    For \Subadditivity, let $[x],[y] \in X/\Ker$. Then 
    \begin{align*}
        \norm{[x]+[y]}_{\Ker}& = \norm{[x+y]}_{\Ker}\\
        & = \tilde{d}\pa{[x+y], [0]}\\
        & = d(x+y, 0)\\
        & = \norm{x+y} \\
        & \leq \norm{x}+\norm{y}\\
        & = d(x,0)+d(y,0)\\
        & = \tilde{d}\pa{[x],[0]}+ \tilde{d}\pa{[y],[0]}\\
        & = \norm{[x]}_{\Ker}+\norm{[y]}_{\Ker}
    \end{align*}
    For \ScalarHomogeneity, let $\alpha \in \F$ and $[x] \in X/\Ker$. 
    Then, 
    \begin{align*}
        \norm{[\alpha x]}_{\Ker} & = \tilde{d}\pa{[\alpha x], [0]}\\
        & = d(\alpha x, 0) \\
        & = \norm{\alpha x}\\
        & = \abs{\alpha} \norm{x} \\
        & = \abs{\alpha} \norm{[x]}_{\Ker}
    \end{align*}
    Finally, suppose $[x] \neq 0$. 
    Then, since the additive identity of $X/\Ker$ is $\Ker$, $x \not \in \Ker$. 
    Hence $\norm{[x]}_{\Ker} = \tilde{d}([x], 0) = d(x,0) =\norm{x} > 0$. 

\end{proof}
\begin{proof}[Proof of 2] 
Let $D$ denote the \SeminormInducedPseudometric $\norm{\cdot}_{\Ker}$. 
Then, for $[x], [y] \in X/\Ker$, 
\begin{align*}
\tilde{d}([x], [y]) & = d(x,y)\\
& = \norm{x-y}\\
& = \norm{x-y-0}\\
& = d(x-y, 0)\\
& = \tilde{d}([x-y],0)\\
& = \norm{[x-y]}_{\Ker}\\
& = \norm{[x]-[y]}_{\Ker}
& = D\pa{[x], [y]}
\end{align*}
Since these two \Pseudometric's are equal, they produce the same topology. 
Furthermore, by applying \ref{prop:pseudometricinducedmetric}, we see that the 
topology generated by $\norm{\cdot}_{\Ker}$ is also the \QuotientSpaceTopology on $X/\Ker$. 
\end{proof}
\begin{proof}[Proof of 3]
T is the topological \QuotientMap and the norm topology is the \QuotientSpaceTopology, so the assumptions of $\ref{prop:QuotientSpaceTopology}$ are satisfied. 
\end{proof} 
\begin{proof}[Proof of 4] 
Let $x,y \in X$ and $\alpha \in \F$. Then 
\begin{align*}
T(\alpha x + y) & = [\alpha x + y] \\
& = (\alpha x + y ) + \Ker\\
& =\alpha \pa{x+\Ker} + \pa{y+ \Ker}\\
& = \alpha [x] + [y]\\
& = \alpha T(x) + T(y) 
\end{align*}
\end{proof}
\begin{proof}[Proof of 5] 
Direct consequence of \ref{prop:QuotientMapSurjective}
\end{proof}
\begin{proof}[Proof of 6] 
Consequence of part 2 of this result combined with 
\end{proof}
\begin{proof}[Proof of 7] 
If $\norm{\cdot}$ is a \Norm, then $\Ker={0}$, so 
$Tx=Ty \implies T(x-y) =0 \implies x-y \in \Ker \implies x-y=0 \implies x=y$. 
\end{proof}

\end{enumerate} 

\end{prop} 

\begin{rmk}[Quotient Normed Space]
\label{rmk:quotientnormedspace}
    If $(X,\norm{\cdot}_X)$
    is a \NormedSpace
    then by parts
    4, 5, 6, and 7 of
    \ref{prop:quotientnormspace}, 
    $T:X \to \Ker_X$
    is an isomorphism of 
    \NormedSpaces whose 
    definition is literally
    \begin{equation}
    Tx=\{x\}
    \end{equation}
    For this reason, 
    as an admitted abuse of notation, 
    later in this document,
    I may not distinguish between the quotient
    $X/\Ker_X$ and the space $X$ if
    X is a \NormedSpace, 
    and similarly, I may not distinguish between 
    $x \in X$ and $\{x\} \in X/\Ker_X$. 
\end{rmk}

\begin{prop}
\label{prop:quotientspreservecompleteness}
Let $(X,\norm{\cdot})$ be a \SeminormedSpace with \QuotientNormedSpace $(X/\Ker, \norm{\cdot}_{\Ker})$. 

Then X is \PseudometricComplete if and only if $X/\Ker$ is complete. 

\begin{proof}
Let X be \PseudometricComplete. 
Let $\{[x_i]\}_{i \in \N} \subset X/\Ker$ be a \PseudometricCauchySequence. 
Let $\epsilon > 0$. 
Then there is an $N \in \N$ such that for $m,n > N$ we have 
\begin{equation}
\norm{[x_m-x_n]}_{\Ker} < \epsilon
\end{equation}

For this N, we have 
\begin{equation}
\norm{x_m-x_n} = \norm{[x_m-x_n]}_{\Ker} < \epsilon
\end{equation}
so that $\{x_i\}_{i \in \N}$ is a \PseudometricCauchySequence. 
Since X is \PseudometricComplete, 
there is a 
$x \in X$ such that $\norm{x_i-x} \to 0$, 
but since T is an isometry, 
\begin{equation}
\norm{[x]-[x_i]}=\norm{[x_i-x]}_{\Ker} \to 0
\end{equation}
and so 
$[x_i] \to [x]$.
so that $X/\Ker$ is complete. 

Now suppose instead that $X/\Ker$ is complete 
and suppose $\{x_i\}_{i \in \N}$ is a \PseudometricCauchySequence in X. 
Since $\norm{[x_i-x_j]}_{\Ker} = \norm{x_i-x_j}$, 
$\{[x_i]\}_{i \in \N}$ is a \PseudometricCauchySequence in $X/\Ker$, which therefor has a 
limit $y \in X/\Ker$. Since T is surjective, $y=[x]$ for some $x \in X$, and it is easy to see that
$x_i \to x$ so that $X$ is \PseudometricComplete. 

\end{proof}
\end{prop}

\label{def:BLO} 
\newcommand{\SpaceOfBoundedLinearOperators}[0]{ 
    \bf \hyperref[def:BLO]{Space of Bounded Linear Operators} \rm
}
\newcommand{\OperatorSeminorm}[0]{
    \bf \hyperref[def:BLO]{Operator Seminorm} \rm
}
\newcommand{\OperatorNorm}[0]{
    \bf \hyperref[def:BLO]{Operator Norm} \rm
}
\begin{df}[Space of Continuous Linear Operators From a Seminormed Space into a Normed Space]
Let $(X,\norm{\cdot}_X)$ be a \NonDegenerate \SeminormedSpace.
Let $(Y, \norm{\cdot}_Y)$ be a \SeminormedSpace.
We denote with $BL\pa{(X,\norm{\cdot}_X), (Y, \norm{\cdot}_Y)}$ 
the collection of
\Continuous
\Linear
operators
$T:(X, \norm{\cdot}_X) \to (Y, \norm{\cdot}_Y)$. 
When the topologies on X and Y are understood, we denote this set with
$BL\pa{X,Y}$. 
We refer to $BL\pa{X,Y}$ as the \SpaceOfBoundedLinearOperators 
from $(X, \norm{\cdot}_X)$ to $(Y, \norm{\cdot}_Y)$ 
, or when $\norm{\cdot}_X$ and $\norm{\cdot}_Y$ are understood, 
from X to Y. 

We endow $BL\pa{X,Y}$ with the algebraic operations
of pointwise scalar multiplication
and pointwise addition, making $BL\pa{X,Y}$ a vector space. 

We define $\norm{\cdot}:BL(X,Y) \to [0,\infty)$ by defining, 
for $T \in BL(X,Y)$
\begin{equation}
    \norm{T} = \sup\limits_{\norm{x}_X \neq 0} \frac{\norm{Tx}_Y}{\norm{x}_X}
\end{equation}
As will be proven in \ref{prop:BLO}, $\norm{\cdot}$ is a \Seminorm on $BL(X,Y)$, which 
we refer to as the \OperatorSeminorm on $BL(X,Y)$. induced by the
\Seminorm $\norm{\cdot}_X$ on X and the \Seminorm $\norm{\cdot}_Y$ on Y. 

In the case that $\norm{\cdot}_{Y}$ is a \Norm, rather than just a \Seminorm, by \ref{prop:BLO}
, $\norm{\cdot}$ is a \Norm on $BL(X,Y)$, which we instead call the \OperatorNorm. 
\end{df}

\begin{prop} 
\label{prop:BLO} 
Let $(X,\norm{\cdot}_X)$ be a \SeminormedSpace. 
Let $(Y, \norm{\cdot}_Y)$ be a \SeminormedSpace.
Let $BL(X,Y)$ denote the \SpaceOfBoundedLinearOperators from X to Y. 
Let $\norm{\cdot}$ denote the \OperatorSeminorm. 

The following are true. 
\begin{enumerate}
%For Item 1, may have to prove result connecting 
%pseudometric topology continuity to $\epsilon-delta$ cotninuity wrt the pseudometric. 
\item $\norm{\cdot}$ is in fact a well-defined \Seminorm on $BL(X,Y)$. 
\item If $\norm{\cdot}_Y$ is a \Norm, then so is $\norm{\cdot}$. 
\item $\norm{\cdot}$ is \NonDegenerate if and only if Y is. 
\item $BL(X,Y)$ is complete if and only if Y is. 
\item Convergence of a sequence $\{T_i\}_{i \in \N} \subset BL(X,Y)$ with respect to $\norm{\cdot}$
is equivalent to the following condition: $T_ix \to Tx$ uniformly for $x \in B_X(0;1)$. 
\end{enumerate}


\begin{proof}[Proof of 1] 
    Since X is nondegenerate, there exists at least 1 $x \in X$ with $\norm{x}_X \neq 0$, 
    so for each $T \in BL(X,Y)$, the set that the supremum is being taken over is nonempty.
    Also, it is clear that $Range(\norm{\cdot}) \subset [0,\infty)$, 

    For \Subadditivity, let $T_i \in BL(X,Y)$ for $i \in \{0,1\}$. and $x \in X$ with $\norm{x} > 0$.
    Then, since $\norm{\cdot}_Y$ is \Subadditive, 
    \begin{align*}
    \frac{\norm{(T_0+T_1)x}_Y}{\norm{x}_X} \leq \frac{\norm{T_0x}_Y}{\norm{x}_X}+ \frac{\norm{T_1x}_Y}{\norm{x}_X}
    \end{align*}
    Since this is true for each x with $\norm{x}_X \neq 0$, taking the supremum of each side yields

    \begin{align*}
    \sup\limits_{\norm{x}_X \neq 0} \pa{\frac{\norm{(T_0+T_1)x}_Y}{\norm{x}_X}} & \leq\sup\limits_{\norm{x}_X \neq 0} \pa{ \frac{\norm{T_0x}_Y}{\norm{x}_X}+ \frac{\norm{T_1x}_Y}{\norm{x}_X}}\\
& \leq\sup\limits_{\norm{x}_X \neq 0} \pa{ \frac{\norm{T_0x}_Y}{\norm{x}_X}} + \sup\limits_{\norm{x}_X \neq 0} \pa{\frac{\norm{T_1x}_Y}{\norm{x}_X}}\\
    \end{align*}
    Hence, $\norm{T_0+T_1} \leq \norm{T_0}+\norm{T_1}$ so that $\norm{\cdot}$ is \Subadditive. 
    For \ScalarHomogeneity, let $T \in BL(X,Y)$, $\alpha \in \F$, and $x \in X$ with $\norm{x}_X \neq 0$. 
    Then 
    \begin{align*}
        \frac{\norm{(\alpha T)x}_Y}{\norm{x}_X} = \frac{\norm{\alpha (Tx)}_Y}{\norm{x}_X} = \abs{\alpha} \frac{\norm{Tx}_Y}{\norm{x}_X}
    \end{align*}
    Hence taking the supremum finishes the proof.
\end{proof}
\begin{proof}[Proof of 2] 
   Let $T \neq 0 \in BL(X,Y)$. Then for some $x \in X$, $Tx \neq 0$. 
   Then $Tx$ has a neighborhood U disjoint from $0_Y$, 
   Hence $x \in T^{-1}(U)$ but not $0_X \in T^{-1}(U)$, since $T0_X = 0_Y$.
   Since U is a neighborhood of x disjoint from 0, 
   there is an $\epsilon > 0$ such that $0_X \subset \complement U \subset \complement B_X(x;\epsilon)$,
   and therefore $\norm{x}_X > \epsilon$. 
   Since $\norm{x}_X > 0$, it is ranged over in the supremum defining $\norm{T}$, and so
   \begin{equation}
   0 < \frac{\norm{Tx}_Y}{\norm{x}_X} \leq \sup\limits_{\norm{x}_X \neq 0} \frac{\norm{Tx}_X}{\norm{x}_X}=\norm{T}
   \end{equation}
\end{proof}
\begin{proof}[Proof of 3] 
\end{proof}
\begin{proof}[Proof of 4] 
\end{proof}

\end{prop}


\label{def:handedquotientoperators}
\newcommand{\CodomainQuotientOperator}[0]{
    \bf \hyperref[def:handedquotientoperators]{Codomain Quotient Operator} \rm
}
\newcommand{\CodomainQuotientMap}[0]{
    \bf \hyperref[def:handedquotientoperators]{Codomain Quotient Map} \rm
}
\begin{df}[Codomain Quotient Operator]
    Let X and Y be \SeminormedSpaces.
    Define $\scQ_Y:BL(X,Y) \to BL(X, Y/\Ker_Y)$ by setting, 
    for each $x \in X$, 
    \begin{equation*} 
        \scQ_YTx = [Tx]
    \end{equation*}
    Let $T \in BL(X,Y)$. 
    We call $\scQ_Y$ the \CodomainQuotientMap of X and Y
    and we call $\scQ_YT$ the 
    \CodomainQuotientOperator
    of T.
\end{df}

\begin{prop}[Codomain Quotient Operator]
\label{prop:handedquotientoperators}
    Let X and Y be 
    \SeminormedSpaces
    with \CodomainQuotientMap $\scQ_Y$. 
    The following are true. 
    \begin{enumerate}
        \item $\scQ_Y$ is a well defined continuous linear surjective isometry. 
        \item If Y is a \NormedSpace, then $\scQ_Y$ is invertible with a continuous inverse. 
    \end{enumerate}
    \begin{proof}[Proof Of 1]
        Since $Tx \in Y$ for any $x \in X$, 
        $[Tx]_Y$ is defined for any $x \in X$. 
        Furthermore, if $q_y:Y \to Y/\Ker$
        is the \QuotientMap of Y under 
        \EquivalenceModKernel, then 
        $\scQ_YT = q_y \circ T$. 
        By \ref{prop:quotientnormspace}, 
        $q_y$ is linear and an isometryu, and hence continuous.
        Therefore, 
        $q_y \in BL(Y, Y/\Ker)$. 
        Hence $\scQ_Y$ is well defined. 

        For linearity, let $\alpha \in \F$
        and $S,T \in BL(X,Y)$. 
        Let $x \in X$. 
        Then, 
        \begin{align*}
            \scQ_Y\pa{\alpha T+S}x & = \bra{\pa{\alpha T+S}x}_Y\\
            & = \bra{\alpha Tx+ Sx}_Y\\
            & = \bra{\alpha Tx}_Y+ \bra{Sx}_Y\\
            & = \alpha \bra{Tx}_Y+\bra{Sx_Y}\\
            & = \alpha \scQ_YTx+ \scQ_YSx\\
            & = \pa{\alpha \scQ_YT+\scQ_YS}x
        \end{align*}

        For being an isometry, 
        let $T \in BL(X,Y)$ and 
        let $x \in X$. Then, since $\norm{\bra{Tx}_Y}_{Y/\Ker} = \norm{Tx}_Y$, 
        \begin{align*}
            \frac{\norm{\scQ Tx}_{Y/\Ker}}{\norm{x}_X} & = \frac{\norm{\bra{Tx}_Y}_{Y/\Ker}}{\norm{x}_X} \\
            & = \frac{\norm{Tx}_Y}{\norm{x}_X} 
        \end{align*}
        and thus taking the norm over
        x with $\norm{x}_X \neq 0$ will yield the
        same result. Hence $\norm{T} = \norm{\scQ_Y T}$. 

        For surjectivity, let $\tilde{T} \in BL(X, Y/\Ker_Y)$. 
        Let $\{x_{\alpha}\}_{\alpha \in A}$ be a hamel basis for $X$. 
        For each $\alpha \in A$, let $y_{\alpha} \in \tilde{T}x_{\alpha}$. 
        Define $T:X \to Y$ by 
        \begin{equation}
            T\pa{\sum_{i=1}^n \beta_{\alpha_i} x_{\alpha_i}} = \sum_{i=1}^n \beta_{\alpha_i} y_{\alpha_i}
        \end{equation}
        T is obviously linear
        and has the property $[Tx]=\tilde{T}x$. 
        and since $\tilde{T} \in BL(X,Y/\Ker)$, 
        $\tilde{T}\Ker_X \subset \Ker_{ (Y/\Ker_Y)}=0$. 
        Hence $T \Ker_X \subset \Ker_Y$. 
        Furthermore, if $x \in X$ with $\norm{x}_X \neq 0$, then
        \begin{align*}
        \frac{\norm{Tx}_Y}{\norm{x}_X} & = \frac{\norm{\bra{Tx}_Y}_{Y/\Ker}}{\norm{x}_X}\\
        & = \frac{\norm{\tilde{T}x}_{Y/\Ker}}{\norm{x}_X}
        \end{align*}
        Therefore $T$ is bounded.
        Hence $T \in BL(X,Y)$, and $\scQ_YT=\tilde{T}$. 
        Thus we have surjectivity, and are done.
    \end{proof}
    \begin{proof}[Proof Of 2]
        If $Y$ is a \NormedSpace, 
        %then $q_y:Y \to Y/\Ker_Y$ is 
        a linear isometric homeomorphism by 
        \ref{prop:quotientnormspace}. 
        In particular, in this case, 
        $q_y$ is injective, meaning that 
        if $T,S \in BL(X,Y)$ where
        $T \neq S$, then 
        $Tx_0 \neq Sx_0$ for some $x_0 \in X$. 
        For this $x_0$, $q_yTx_0 \neq q_ySx_0$, so 
        $\scQ_YT \neq \scQ_YS$. 
        Therefore $\scQ_Y$ is injective, and therefore a bijection. 
        The inverse of an isometry is also an isometry 
        and therefore continuous, finishing this proof. 
    \end{proof}
\end{prop}

\label{def:quotientoperator}
\newcommand{\QuotientOperator}[0]{
    \bf \hyperref[def:quotientoperator]{Quotient Operator} \rm
}
\newcommand{\OperatorQuotientMap}[0]{
    \bf \hyperref[def:quotientoperator]{Operator Quotient Map} \rm
}
\begin{df}[Quotient Operator]
    Let $X,Y$ be \SeminormedSpaces
    with \SeminormKernels $\Ker_X$, $\Ker_Y$. 
    Define $Q:BL(X,Y) \to BL(X/\Ker_X, Y/\Ker_Y)$ by 
    setting, for $T \in BL(X,Y)$, 
    for $x \in X$, 
    \begin{equation}
    QT\bra{x}_X=\bra{Tx}_Y
    \end{equation}
    We call Q the \OperatorQuotientMap of X and Y and
    we call QT the \QuotientOperator of T. 
\end{df}



\begin{prop}[Quotient Operator]
\label{prop:quotientoperator}
    Let $X,Y$ be \SeminormedSpaces
    with \SeminormKernels $\Ker_X$, $\Ker_Y$
    and \OperatorQuotientMap Q. 
    Then Q is a well-defined linear surjective isometry. 
    \begin{proof} 
        We first show that Q is well defined. 
        Let $T \in BL(X,Y)$ and 
        let $x_0, x_1 \in X$ such that $\bra{x_0}=\bra{x_1}$. 
        Then $\norm{x_0-x_1}_X = 0$, so since T is continuous, 
        $\norm{Tx_0-Tx_1}_Y = 0$. 
        Hence $Tx_0 \cong Tx_1$, so
        $\bra{Tx_0} = \bra{Tx_1}$. 


        For linearity, let $\alpha\in \F$, and let
        $T,S \in BL(X,Y)$. 
        Let $x \in X$. 
        Then 
        \begin{align*}
            Q\pa{\alpha T+S}\bra{x}_X & = \bra{\pa{\alpha T+S}x}_Y\\
            & = \alpha \bra{Tx}_Y + \bra{Sx}_Y\\
            & = \alpha QT[x]_X +QS[x]_X\\
            & = \pa{\alpha QT+QS}[x]_X
        \end{align*}
        Since $x \in X$ was arbitrary, Q is linear. 

        As for being an isometry, let $T \in BL(X,Y)$ and let $x \in X$. 
        Since $\norm{\bra{x}}=\norm{x}$ and $\norm{Tx}=\norm{\bra{Tx}}$, 
        we have 
        \begin{align*}
        \frac{\norm{QT\bra{x}_{X/\Ker_X}}_{Y/\Ker_Y}}{\norm{\bra{x}}_{X/\Ker_X}}  & =  \frac{\norm{\bra{Tx}}_{Y/\Ker_Y}}{\norm{\bra{X}}_{X/\Ker_X}}\\
        & = \frac{\norm{Tx}_Y}{\norm{x}_X} 
        \end{align*}
        and so taking the supremum over $\norm{x} \neq 0$ gives us 
        that this is an isometry. 
        

        For surjectivity, let $\tilde{T} \in BL(X/\Ker_X, Y/\Ker_Y)$.
        Let $\{x_\alpha\}_{\alpha \in A}$ be a Hamel basis for X. 
        For each $\alpha \in A$, 
        let $y_\alpha \in \tilde{T}[x_\alpha]_X$. 
        Now define 
        \begin{equation}
            T\sum_{i=1}^n \beta_i x_{\alpha_i} = \sum_{i=1}^n \beta_i y_{\alpha_i}
        \end{equation}
        Then $T:X \to Y$ is obviously linear, and
        $Tx \in \tilde{T}[x]_X$ for $x \in X$. 
        Hence, 
        \begin{equation}
            \frac{`\norm{Tx}_{Y}}{\norm{x}_X} = \frac{\norm{\tilde{T}[x]_X}_{Y/\Ker_Y}}{\norm{[x]_X}_{X/\Ker_X}}
        \end{equation}
        so T is bounded, and hence $T \in BL(X,Y)$, 
        but that also implies that by definition, 
        $QT=\tilde{T}$, so we have proven surjectivity. 
    \end{proof}

\end{prop}

\label{def:canonicalisomorphism}
\newcommand{\CanonicalIso}[0]{
    \bf \hyperref[def:canonicalisomorphism]{Canonical Isomorphism Of The Quotient Space Of Continuous Linear Operators} \rm
}

\begin{df}[Canonical Isomorphism Of The Quotient Space Of Continuous Linear Operators]
    Let $X,Y$ be \SeminormedSpaces
    with \SeminormKernels $\Ker_X$, $\Ker_Y$.
    Let $\Ker$ denote the \SeminormKernel of $BL(X,Y)$. 
    Let Q denote the \OperatorQuotientMap of X and Y.
    Define $\Theta_{(X,Y)}:BL(X,Y)/\Ker \to BL(X/\Ker_X, Y/\Ker_Y)$ by 
    setting, for each $T \in BL(X,Y)$. 
    \begin{equation}
        \Theta_{(X,Y)}(\bra{T}) = QT
    \end{equation}
    We call $\Theta_{(X,Y)}$ the \CanonicalIso from X to Y. 
    When X and Y are understood, we may denote the
    \CanonicalIso simply with $\Theta$. 
    By \ref{prop:canonicalisomorphism}, $\Theta_{(X,Y)}$
    is an isomorphism of \NormedSpaces.
    That is, $\Theta$ is Linear, Bijective, Bicontinuous, and an isometry. 
\end{df}




\begin{prop}[Canonical Isomorphism Of The Quotient Space Of Continuous Linear Operators]
\label{prop:canonicalisomorphism}
    Let $X,Y$ be \SeminormedSpaces.
    Let $\Theta$ denote the \CanonicalIso
    from X to Y. 
    Then $\Theta$ is
    a bijective, bicontinuous, linear, isometry. 
    \begin{proof}
       By \ref{prop:quotientnormspace}, part 1, 
       $Y/\Ker_Y$ is a \NormedSpace, 
       Hence by \ref{prop:BLO}, part 2, 
       \newline
       $BL(X/\Ker_X, Y/\Ker_Y)$ is a \NormedSpace. 
       Similarly, by \ref{prop:quotientnormspace}, part 1, 
       $BL(X,Y)/\Ker$ is a normed space. 
       Hence, it is sufficient to show that $\Theta$ is a
       well-defined surjective linear isometry. 

       For well definedness, let $T,S \in BL(X,Y)$ with $[T]=[S]$. 
       Then, $\norm{T-S}=0$, so 
       if $x \in X$, $\norm{Tx-Sx}=0$. 
       Hence $Tx \cong Sx$ and since x was arbitrary, 
       $QT=QS$. 
        
       Let q denote the \QuotientMap $q:BL(X,Y) \to BL(X,Y)/\Ker$. 
       By parts 4, 5, and 6 of \ref{prop:quotientnormspace}, 
       q is a linear surjective isometry. 
       Also, by definition, $\Theta \circ q = Q$. 
       Since Q is surjective, $\Theta$ is surjective. 
       Since $Q$ is an isometry, and $q$ is a surjecive isometry, 
       $Theta$ is an isometry. 
       Since Q is linear, and since q is surjective and linear, 
       $\Theta$ is linear. 
    \end{proof}


\end{prop}

\label{def:topologicaldualspace}
\newcommand{\SemiTopDualSpace}[0]{
    \bf \hyperref[def:topologicaldualspace]{Topological Dual Space} \rm
}
\begin{df}[Seminorm Topological Dual Space]
    Let $(X,\norm{\cdot})$ be a 
    \SeminormedSpace
    over a field $\F$. 
    We denote with $X^*$ 
    \NormedSpace $BL(X,\F)$, 
    and we call $X^*$ the
    \SemiTopDualSpace of $X$. 
    If $x^* \in X^*$, then we may denote, 
    for $x \in X$, 
    $x^*(x)$
    with 
    $\ip{x,x^*}$. 


    
    Since $\F$ is a \NormedSpace, 
    by \ref{prop:BLO}, 
    part 02, $X^*$ is as well. 

    Since $X^*$ is a normed space 

    Also, $q:\F \to \F/\Ker_F$ 
    is a linear bijective isometry by 
    \ref{prop:quotientnormspace}, so
    if $Q:X^* \to BL(X/\Ker_X, \F/\Ker_{\F})$ is the \OperatorQuotientMap
    and if $\Theta:BL(X,\F)/\Ker \to BL(X/\Ker_X, \F/\Ker_\F)$ 
    is the \CanonicalIso, then we have 
    \begin{equation}
    \Theta=Q \circ q^{-1}
    \end{equation}


\end{df}

\label{def:dualspace} 
\newcommand{\TopDualSpace}[0]{ \bf \hyperref[def:dualspace]{Topological Dual Space} \rm } \newcommand{\FirstTopDualSpace}[0]{
	\bf \hyperref[def:dualspace]{$1^{st}$ Topological Dual Space} \rm
}
\begin{df}[Dual Space]
    Let $(X,\norm{\cdot})($ be a 
    \SeminormedSpace.
    We call $BL(X, \F)$ the 
    \TopDualSpace of 
    $(X,\norm{\cdot})$,
    and we denote 
    $BL(X,\F)$ with the symbol 
    $X^*$. 
    If $x^* \in X^*$, then
    we use the notational convention
    of writing, for $x \in X$. 
    \begin{equation}
    \ip{x, x^*}:= x^*(x)
    \end{equation}
	It would also be correct
	to refer to the 
	\TopDualSpace of 
	$(X,\norm{\cdot})$
	as the 
	\FirstTopDualSpace
	of X
	
\end{df}
\begin{rmk}[\TopDualSpace is a \NormedSpace]
    Let $X$ be a 
    \SeminormedSpace.
    Then, using 
    \ref{prop:BLO}, 
    since $\F$ is a 
    \NormedSpace, 
    so is $X^*$. 
\end{rmk}

\begin{thm}[\TopDualSpace Isomorphism]
\label{thm:dualspaceisomorphism}
    Let X be a 
    \SeminormedSpace.
    Define $\Omega:X^* \to \pa{X/\Ker_X}^*$
    by setting, for $x^* \in X$, 
    and for $x \in X$, 
    \begin{equation}
        \ip{x, x^*} = \ip{[x], \Omega x^*}
    \end{equation}
    Then $\Omega$ is a 
    Linear, 
    Bijective, 
    Isometric, 
    Bicontinuous operator. 
    That is, $X^*$ and $(X/\Ker_X)^*$ are 
    isomorphic, and that isomorphism is explicitly
    given by $\Omega$. 
    \begin{proof}
        Consider the following
        \begin{equation}
            BL(X,\F) \overset{q}{\to} BL(X,\F)/\Ker_{BL(X/\F)} \overset{\Theta}{\to} BL(X/\Ker_X, \F/\Ker_{\F}) \overset{\scQ_{\F}^{-1}}{\to} BL(X/\Ker_X,\F)
        \end{equation}
        where
        q is the \QuotientMap, 
        which is an linear bijective bicontinuous isometry in this case
        by parts 4, 5, 6, and 7 of 
        of \ref{prop:quotientnormspace}, 
        $\Theta$ is the \CanonicalIso, 
        which is a linear bijective bicontinuous isometry by 
        \ref{prop:canonicalisomorphism}
        and $\scQ_{\F}$ is the \CodomainQuotientMap.
        which is in this case a linear, bijective, bicontinuous isometry
        by \ref{prop:handedquotientoperators}


        Since $\Omega=\scQ_{\F}^{-1} \circ \Theta \circ q$, 
        and since each of the described properties
        are preserved under composition, 
        $\Omega$ is also a 
        linear bijective bicontinuous isometry. 
    \end{proof}
\end{thm}

\begin{rmk}[\TopDualSpace is a \NormedSpace]
\label{rmk:dualspaceisnormedspace}
	Let X be a 
	\SeminormedSpace.
	Since $X/\Ker_X$
	is a \NormedSpace, 
	so is $(X/\Ker_X)^*$. 
	By 
	\ref{thm:dualspaceisomorphism}, 
	we have a
	linear, bijective isometry
	between $X^*$
	and 
	$(X/\Ker_X)^*$. 
	Hence $X^*$ is a \NormedSpace. 
\end{rmk}



%\begin{prop}
    \label{prop:dualspacepushing}
    Let $X,Y$ be \SeminormedSpaces.
    Let $T:X \to Y$ be a
    linear, 
    surjective
    isometry. 
    Define $T^*:X^* \to Y^*$ by 
    setting, for $x^* \in X^*$ and
    $x \in X$, 
    \begin{equation}
    \ip{Tx, T^*x^*} = \ip{x,x^*}
    \end{equation}
    Then $T^*$ is a Linear Bijective Isometry. 
    \begin{proof} 
        I first need to show that $T^*$ is well defined.
        Suppose $Tx \cong Ty$. 
        Since T is an isometry, $x \cong y$. 
        Hence, $\ip{x,x^*} = \ip{y,x^*}$ for all 
        $x^* \in X^*$. so the equation 
        defining $T^*x^*$ is at least consistent. 
        Is $T^*x^*$ linear? Yes, as 
        if $x_1,y_1 \in Y$ and $\alpha \in \F$, 
        then there are $x,y \in X$ such that 
        $Tx=x_1$, $Ty=y_1$, and 
        \begin{align*}
            \ip{\alpha x_1+y_1, T^*x^*} & = \ip{\alpha x+y, x^*}\\
            & = \alpha \ip{x,x^*}+\ip{y,x^*}\\
            & = \alpha \ip{x_1, T^*x^*}+ \ip{y_1, T^*x^*}
        \end{align*}
        Boundedness of $T^*x^*$ will be shown during the 
        proof that $T^*$ is an isometry. 



    For linearity, let $x_i^* \in X^*$ for 
    $i \in \{0,1\}$ and let $\alpha \in \F$. 
    Let $y \in Y$. 
    Since T is surjective, there exists $x \in X$ 
    such that $y=Tx$.
    Then
    \begin{align*}
        \ip{y, T^*\pa{\alpha x_0^*+x_1^*}} & = \ip{x, \alpha x_0^*+x_1^*} \\
        & = \alpha \ip{x, x_0^*} + \ip{x, x_1^*} \\
        & = \alpha \ip{y, Tx_0^*}+ \ip{y, Tx_1^*}\\
        & = \ip{y, \alpha T^* x_0^* + T^* x_1^*}
    \end{align*}

    For surjectivity, let $y^* \in Y^*$. 
    To prove the existence of $x^* \in X^*$ 
    such that $T^*x^* = y^*$, 
    it is sufficient to find$ x^*\in X$
    with $Kernel(T x^*)=Kernel(y^*)$. 


    To see that $T^*$ is an isometry, 
    let $x^* \in X^*$. Then, 
    since T is surjective and an isometry, 
    \begin{equation*}
    \{y \in Y| \norm{y} \neq 0  \} = \{Tx | \norm{x} \neq 0\}
    \end{equation*}
    , which allows us to say that
    \begin{align*}
    \norm{T^*x^*} &= \sup\limits_{\norm{y} \neq 0} \frac{\abs{\ip{y, T^*x}}}{\norm{y}}\\
    &= \sup\limits_{\norm{x} \neq 0} \frac{\abs{\ip{Tx, T^*x^*}}}{\norm{Tx}} \\
    & = \sup\limits_{\norm{x} \neq 0} \frac{\abs{\ip{x,x^*}}}{\norm{x}}\\
    & = \norm{x^*}
    \end{align*}
    so $T^*$ is an isometry. 
    Since $T^*$ is an isometry, and 
    $X^*$ and $Y^*$ are both normed spaces by 
    \ref{thm:dualspaceisomorphism}, 
    $T^*$ is surjective. 

    \end{proof}



\end{prop}


\label{thm:hahnbanach}
\begin{thm}[Hahn Banach Theorem For Seminormed Spaces]
Let $(X,\norm{\cdot})$ be a \SeminormedSpace,
let $x_i \in X$ for $i \in \{0,1\}$ such that 
$\norm{x_0-x_1}_X \neq 0$, and
let $X^*$ denote $X's$
\TopDualSpace. 
The following are true. 
\begin{enumerate}
    \item If $Z \subset X$ is a subspace
        and $z^* \in Z^*$, then there 
        is an extension $x^*$ of $z^*$, 
        $x^* \in X^*$ such that 
        \begin{equation}
        \norm{z^*}_{Z^*} = \norm{x^*}_{X^*}
        \end{equation}
     \item If $x \in X$, 
        with $\norm{x} \neq 0$, 
        then there exists an
        $x^* \in X$ with 
        $\norm{x^*}=1$ and 
        $\ip{x,x^*} = \norm{x}_X$. 
    \item If $x \in X$, then 
    \begin{equation}
        \norm{x}_X = \sup\limits_{0 \neq x^* \in X^*} \frac{\ip{x,x^*}}{\norm{x^*}}
    \end{equation}
    \item If $Y$ is a 
        \NonDegenerate
        \SeminormedSpace, and if 
        $x_0 \in X$, with 
        $\norm{x_0} \neq 0$, 
        then there exists
        an $S \in BL(X,Y)$ with 
        $\norm{S} = 1$ and 
        \begin{equation}
            \norm{Sx_0} = \norm{x_0}
        \end{equation}
\end{enumerate}


\begin{proof}[Proof of 01]
    For $\alpha \in \{Z,X\}$, let 
    $\Omega_\alpha:\alpha^* \to (\alpha/\Ker_\alpha)^*$ denote the isomorphism
    defined in 
    \ref{thm:dualspaceisomorphism}.
    Let $q$ denote the quotient operator $q:X \to X/\Ker$. 
    Define $T:Z/\Ker_Z \to q(Z)$ bv $T([z]_{\cong_Z} ) = [z]_{\cong_X}$. %Make a separate Result
    Since Z is endowed with the subspace Topology,                       %Make a separate Result
    T is obviously a Linear Bijective Bicontinuous Isometry.          %Make a separate Result
    
    %Then $\Omega_Zz^* \in (Z/\Ker_Z)^*$ satisfies
    %$\norm{\Omega_Zz^*}_{Z/\Ker_Z}=\norm{z^*}_{Z^*}$ an 
    Define $\Gamma_Z:(Z/\Ker_Z)^* \to q(Z)^*$ by setting, 
    for $\phi^* \in (Z/\Ker_Z)^*$, 
    for $[z]_Z \in Z/\Ker_Z$, 
    \begin{equation}
        \ip{T[z]_Z, \Gamma_Z \phi^*} = \ip{[z]_Z, \phi^*}
    \end{equation}
    Then $\Gamma_Z$ is a Linear Bijective Isometry. 
    Hence $\Gamma_Z \circ \Omega_Z z^* \in q(Z)^*$ with 
    $\norm{\Gamma_Z \circ \Omega_Z z^*}_{q(Z)^*} = \norm{z^*}_{Z^*}$. 

    Thus we can apply the Hahn Banach theorem for \NormedSpaces to claim 
    the existence of $x_q^* \in (X/\Ker_X)^*$ where
    $x_q^*$ is an extension of $\Gamma_Z \circ \Omega_Z z^*$ and
    \begin{equation}
        \norm{x_q^*}_{(X/\Ker_X)^*} = \norm{\Gamma_Z \circ \Omega_Z z^*}_{(q(Z))^*} = \norm{z^*}_{Z^*}
    \end{equation}
    Finally, letting $x^* = \Omega_X^{-1} x_q^*$, we have 
    $x^* \in X^*$, 
    $\norm{x^*}_{X^*} = \norm{x_q^*}_{(X/\Ker_X)^*} =\norm{z^*}_{Z^*}$, 
    and 
    if $z \in Z$, then 
    \begin{align*}
        \ip{z, x^*} & = \ip{[z]_X, x_q^*} \\
        & = \ip{[z]_X, \Gamma_Z \circ \Omega_Zz^*}\\
        & = \ip{[z]_Z, \Omega_Z z^*}\\
        & = \ip{z, z^*}
    \end{align*}
\end{proof}

\begin{proof}[Proof of 2]
    Let $Z=span(x)$. 
    Define $z^* \in Z^*$ by 
    $\ip{\alpha x, z^*} = \alpha \norm{x}$. 
    Then $\norm{z^*} = 1$. 
    Also, by part 1 of this result, 
    it has an extension $x^* \in X^*$ with 
    $\norm{x^*} = \norm{z^*} =1$ 
    and $\ip{x,x^*} = \norm{x}$. 
\end{proof}
\begin{proof}[Proof of 3]
    If $\norm{x} = 0$, then
    for every $x^* \in X$, $\ip{x,x^*} = 0$.
    Hence 
    \begin{equation} 
    \norm{x}_X = \sup\limits_{0 \neq x^* \in X^*} \frac{\ip{x,x^*}}{\norm{x^*}} = \sup\limits_{x^* \in \partial B_{X^*}(0;1)} \frac{\ip{x,x^*}}{\norm{x^*}}=0
    \end{equation}

    Otherwise, let  $x^* \in X^*$ guaranteed to 
    exist by part 2 which satisfies $\norm{x^*}=1$, 
    $\ip{x,x^*} = \norm{x}$. 
    Then 
    \begin{align*}
    \norm{x} & = \frac{\ip{x,x^*}}{\norm{x^*}} \\
    & \leq \sup\limits_{x^* \in \partial B_{X^*}(0;1)} \frac{\ip{x,x^*}}{\norm{x^*}} \\
    & \leq    \sup\limits_{0 \neq x^* \in X^*} \frac{\ip{x,x^*}}{\norm{x^*}} 
    \end{align*}
    The other direction of the inequality
    falls directly from the definition of 
    the norm on $X^*$, and is trivial, so 
    we are done. 
\end{proof}
\begin{proof}[Proof of 4]
    By part 2 of this result, there exists $x_0^* \in X^*$ with 
    $\norm{x_0^*} = 1$ and $\ip{x_0, x_0^*}=\norm{x_0}$. 
    Since Y is \NonDegenerate, there
    exists $y_0 \in Y$ with $\norm{y_0} = 1$. 
    Define $T: \F \to Y$ by $T \alpha = \alpha y$.
    Then $\norm{T} = \norm{y} = 1$. 
    Define $S: X \to Y$ by $S=T \circ x_0^*$. 
    Then $\norm{S} \leq \norm{T} \norm{x_0^*} = 1$, and
    $\norm{S x_0} = \norm{\ip{x_0, x_0^*} y} = \ip{x_0, x_0^*} = \norm{x_0}$. 
    Hence $\norm{S} \geq 1$ and therefore $\norm{S} = 1$. 
\end{proof}
\end{thm}


\begin{prop}[Linear Operator Notation]
\label{rmk:linearoperatornotation}
    $.$
    When dealing with mappings of 
    spaces of linear operators into
    spaces of other linear operators, 
    or even functions in general, 
    notation can get confusing, and
    presenting such things using ordinary notation without
    ambiguity can often require a plethora of parenthesis, 
    which hamper readability of an arguement. 

    For this reason, at points in this document, 
    I sometimes express the image $\beta(\alpha)$ using 
    \begin{equation*}
        \ip{\alpha, \beta}
    \end{equation*}
    Where $\beta:X \to Y$ 
    and $\alpha \in X$. 

    I combine this notation with usual function notation, 
    particularly in cases similar to the following. 
    For $i \in \{0,1\}$, 
    let $X_i, Y_i, Z_i$ be sets. 
    For $\alpha \in \{X,Y,Z\}$, let 
    $F_\alpha$ be the set of maps $f:\alpha_0 \to \alpha_1$. 
    If $T:F_X \to F_Y$, 
    $y \in Y_0$, 
    and $f \in F_X$, then I would notate
    \begin{equation*}
        \ip{y, Tf}
    \end{equation*}
    rather than $Tf(y)$ or $(T(f)(y))$
\end{prop}

\label{def:adjointoperator}
\newcommand{\AdjointOperator}[0]{
    \bf \hyperref[def:adjointoperator]{Adjoint Operator} \rm
}
\begin{df}[\AdjointOperator]
    Let X, Y, and Z be \SeminormedSpaces
    over a field $\F \in \{\R, \C\}$. .
    Let $T \in BL(X,Y)$. 
    We define the operator
    $T^{\times}_Z:BL(Y,Z) \to BL(X,Z)$ by 
    setting, for $S \in BL(Y,Z)$ 
    and $x \in X$, 
    \begin{equation}
        \ip{x , \T^{\times}_{Z}S } = \ip{Tx, S}
    \end{equation}
    or, equivalently, 
    \begin{equation}
    \T^{\times}_Z S = S \circ T
    \end{equation}

    We call $T^\times_Z$ the \AdjointOperator
    of T relative to the space Z, we denote
    $T^\times_{\F}=T^\times$, and
    we refer to $T^\times:Y^* \to X^*$ as 
    simply the \AdjointOperator of T. 
\end{df}

\begin{prop}[\AdjointOperator]
    \label{prop:adjointoperator}
    Let X, Y, and Z be \SeminormedSpaces
    over a field $\F \in \{\R, \C\}$. .
    Let $T \in BL(X,Y)$. 
    Let $\T=T^\times_Z$ denote the 
    \AdjointOperator of T
    relative to the space Z. 
    Let $Q_y$ denote the \QuotientMap

    The following are true. 
    \begin{enumerate}
        \item $\T$ is Linear. 
        \item If $S \in BL(Y,Z)$, then $\T S \in BL(X,Z)$. (That is, the \AdjointOperator is well defined as a concept).
        \item $\T \in BL\pa{ BL(Y,Z), BL(X,Z)}$. 
        \item $\norm{\T}=\norm{T}$
        \item If T is surjective, then $\inf\limits_{\norm{x}=1}\norm{Tx} \leq \inf\limits_{\norm{S}_{BL(Y,Z)} = 1} \norm{\T S} $. Also TODO: Weaken T Surjectivity assumption to $Q_y \circ T$ being surjective. 
        
    \end{enumerate}


    \begin{proof}[Proof of 01]
        Let $S,R \in BL(Y,Z)$, 
        $\alpha \in \F$, 
        and $x \in X$. 
        Then, 
        \begin{align*}
            \ip{x, \T(\alpha S+R)} & = \ip{Tx, \alpha S+R}\\
            & = \alpha \ip{Tx, S}+ \ip{Tx, R} \\
            & = \alpha \ip{x, \T S} + \ip{ x, \T R}
            & = \ip{x, \alpha \T S} + \ip{x, \T R} \\
            & = \ip{x, \alpha \T S + \T R}
        \end{align*}
        Since $x \in X$ was arbitrary, linearity is verified. 
    \end{proof}
    \begin{proof}[Proof of 02]
        Let $S \in BL(Y,Z)$. 
        Then, 
        $\T S = S \circ T$. 
        The composition of continuous operators is continuous, so $\T S$ is 
        continuous.
        The composition of linear operators is linear, so $\T S$ is linear.
        This, paired with linearity, implies $\T S \in BL(X,Z)$.
    \end{proof}
    \begin{proof}[Proof of 3]
        Let $S \in BL(Y,Z)$. Then, 
        if $x \in X$
        \begin{equation}
        \norm{\ip{x, \T S}} = \norm{\ip{Tx, S}} \leq \norm{S} \norm{Tx} \leq \norm{S} \norm{T} \norm{x}
        \end{equation}
        Hence $\norm{\T S} \leq \norm{S} \norm{T}$
        Since T is linear, and since S was arbitrary, 
        by part  12 of \ref{prop:BLO}, $\T \in BL\pa{ BL(Y,Z), BL(X,Z)}$.
    \end{proof}
    \begin{proof}[Proof of 4]
        For any $S \in BL(Y,Z)$, 
        $\T S = S \circ T$, so
        $\norm{\T S} \leq \norm{S} \norm{T}$. 
        Hence $\norm{\T} \leq \norm{T}$. 
        Now let $x_0 \in X$. 
        Then, by part 4 of 
        \ref{thm:hahnbanach}, 
        there exists $S \in BL(Y,Z)$ with 
        $\norm{S}=1$
        and $\norm{STx_0} = \norm{Tx_0}$. 
        Hence, 
        \begin{align*}
            \norm{T x_0} & = \norm{S Tx_0} \\
            & = \norm{(S \circ T) x_0} \\
            & = \norm{(\T S) x_0} \\
            & \leq \norm{\T} \norm{S} \norm{x_0}\\
            & = \norm{\T} \norm{x_0}
        \end{align*}
        Since $x_0 \in X$ is arbitrary, $\norm{T} \leq \norm{\T}$. 
        Since the inequality goes both ways, $\norm{T}=\norm{\T}$.

    \end{proof}
    \begin{proof}[Proof of 05]
        Let $\Gamma=\inf\limits_{\norm{x}=1} \norm{Tx}$, 
        and let $S \in BL(Y,Z)$ with $\norm{S} = 1$. 
        Then, 
        \begin{equation*}
            \{x | \norm{Tx} \leq \Gamma\} \subset B_X(0;1)
        \end{equation*}
        so 
        \begin{equation*}
            \sup\limits_{\norm{x}\leq 1} \abs{\ip{Tx, S}} \geq \sup\limits_{\norm{Tx} \leq \Gamma}\abs{\ip{Tx, S}}
        \end{equation*}

        Also, since T is surjective by assumption, 
        \begin{equation*}
            \sup\limits_{\norm{Tx} \leq \Gamma} \abs{\ip{Tx, S}} = \sup\limits_{\norm{y} \leq \Gamma} \abs{\ip{y, S}}
        \end{equation*}
        From these two we arrive at the inequality
        \begin{align*}
            \norm{\T S} & = \sup\limits_{\norm{x}  \leq 1} \abs{\ip{x, \T S}}\\
            & = \sup\limits_{\norm{x} \leq 1} \abs{\ip{Tx, S}}\\
            & \geq \sup\limits_{\norm{Tx} \leq \Gamma} \abs{\ip{Tx, S}}\\
            & = \sup\limits_{\norm{y} \leq \Gamma} \abs{\ip{y, S}}\\
            & = \Gamma\\
            & = \inf\limits_{\norm{x} = 1} \norm{Tx}
        \end{align*}
        Since $S \in \partial B_{BL(Y,Z)}(0;1)$ was arbitrary, we conclude
        $\inf\limits_{\norm{S} = 1} \norm{\T S} \geq \inf\limits_{\norm{x} = 1} \norm{Tx}$
    \end{proof}
\end{prop}

\label{def:higherorderdualspaces}
\newcommand{\SecondTopDualSpace}[0]{\textbf{\hyperref[def:higherorderdualspaces]{\ensuremath{2^{nd}} Topological Dual Space}}\xspace}
\newcommand{\ThirdTopDualSpace}[0]{\textbf{\hyperref[def:higherorderdualspaces]{\ensuremath{3^{rd}} Topological Dual Space}}\xspace}
\newcommand{\NthTopDualSpace}[1]{\textbf{\hyperref[def:higherorderdualspaces]{\ensuremath{\pa{#1}^{th}} Topological Dual Space}}\xspace}
\begin{df}[Higher Order Dual Spaces]
    Let X be a 
   \SeminormedSpace. 
    From 
    \ref{def:dualspace}
    we know that the 
    \TopDualSpace
    of X, 
    $X^*$, 
    is also called the 
    \FirstTopDualSpace
    of X. 
    Building on this, 
    for $n \in \{2, 3, 4, ..., \}$
    we call the 
    \FirstTopDualSpace
    of $X^*$ the 
    \SecondTopDualSpace
    of X, 
    we call the 
    \FirstTopDualSpace
    of the 
    \SecondTopDualSpace
    of X the 
    \ThirdTopDualSpace
    of X, and 
    in general the 
    \FirstTopDualSpace
    of the 
    \NthTopDualSpace{n}
    of X
    the 
    \NthTopDualSpace{n+1}
    of X. 
   
    In general, we denote the 
    \NthTopDualSpace{n}
    of X with 
    $X^{n*}$, 
    though when n is small, 
    we may denote 
    $X^{**}=X^{2*}$, $X^{***}=X^{3*}$, 
    et cetera. 


\end{df}

\label{def:higherorderdualspaceisomorphism}
\newcommand{\NthDualSPaceIso}[1]{\textbf{\hyperref[def:higherorderdualspaceisomorphism]{\ensuremath{\pa{#1}^{th}} Dual Space Isomorphism}}\xspace}
\begin{df}[Higher Order Dual Space Isomorphism]
Let $X$ be a 
\SeminormedSpace
over a field
$\F$. 
Let $\Omega:X^* \to (X/\Ker_X)^*$ 
be the 
Linear 
Bijective 
Isometry
defined in 
\ref{thm:dualspaceisomorphism}.
Define 
\begin{equation*}
\Omega_1=\Omega
\end{equation*}
and also define, for $2 \leq n \in \N$, 
$\Omega_n:X^{n*} \to \pa{X/\Ker_X}^{n*}$
by 
\begin{equation*}
\Omega_n=\pa{\Omega_{n-1}^{\times}}^{-1}
\end{equation*}

By     
\label{prop:adjointoperator}
it is clear that the 
adjoint of a Linear Bijective isometry of normed spaces
is also a Linear Bijective isometry of normed spaces, and
so each $\Omega_n$ is as well. 

\end{df}

\label{def:canonicalembedding}
\newcommand{\CanonicalEmbedding}[0]{\textbf{\hyperref[def:canonicalembedding]{Canonical Embedding}}\xspace}
\newcommand{\Reflexive}[0]{\textbf{\hyperref[def:canonicalembedding]{Reflexive}}\xspace}
\begin{df}[\CanonicalEmbedding of X into $X^{**}$]
    Let X be a \SeminormedSpace. 
    Define $c_X:X \to X^{**}$ by 
    setting, for each $x^* \in X^*$, 
    for each $x \in X$
    \begin{equation}
        \ip{x^*, c(x)} = \ip{x, x^*}
    \end{equation}
    We call $c_X$ the
    \CanonicalEmbedding
    of X into $X^*$. 
    As normal, if X is understood, 
    we may denote $c_X=c$.
    If c is Surjective, then we 
    say that X is \Reflexive. 

\end{df}


\begin{prop}[Canonical Embedding]
\label{prop:canonicalembedding}
    Let $X$ be a \SeminormedSpace 
    and let $c$ denote its 
    \CanonicalEmbedding. 
    The following are true. 
    \begin{enumerate}
        \item c is well defined
        \item c is Linear. 
        \item c is an isometry. 
        \item c is an injection if and only if X is a \NormedSpace. 
        \item Using the abuse of notation described in 
        \ref{rmk:doubledualnotation}, 
        if $q:X \to X/\Ker$ is the \QuotientMap, 
        then $c_X=c_{X/\Ker} \circ q$. 
        \item $c_X$ is surjective if and only if 
        $c_{X/\Ker}$ is surjective. 
        \item X is \Reflexive if and only if $X/\Ker$ is \Reflexive.
    \end{enumerate}
    \begin{proof}[Proof of 1]
        I just need to show that for any 
        $x \in X$, $c(x)$ is
        conintuous and
        linear. 
        For continuity,
    \end{proof}
    \begin{proof}[Proof of 2]
        Let $\alpha \in \F$ and $x,y \in X$.
        Let $x^* \in X$. 
    \end{proof}
    \begin{proof}[Proof of 3]
    \end{proof}
    \begin{proof}[Proof of 4]
    \end{proof}
    \begin{proof}[Proof of 5]
    \end{proof}

\end{prop}


\label{def:SeminormWeakTopology}
\newcommand{\SeminormWeakTopology}[0]{
    \bf \hyperref[def:SeminormWeakTopology]{Seminorm Weak Topology} \rm
}
\newcommand{\NormWeakTopology}[0]{
    \bf \hyperref[def:SeminormWeakTopology]{Norm Weak Topology} \rm
}
\newcommand{\SeminormWeakStarTopology}[0]{
    \bf \hyperref[def:SeminormWeakTopology]{Seminorm Weak-* Topology} \rm
}
\newcommand{\NormWeakStarTopology}[0]{
    \bf \hyperref[def:SeminormWeakTopology]{Norm Weak-* Topology} \rm
}
\newcommand{\weak}[0]{
    \bf \hyperref[def:SeminormWeakTopology]{$\mathfrak{weak}$}\rm
}

\newcommand{\weakstar}[0]{
    \bf \hyperref[def:SeminormWeakTopology]{$\mathfrak{weak}^*$}\rm
}

\begin{df}[Weak Topologies Relating To Seminormed and Normed Spaces]
    %Let $(X,\norm{\dot})$ be a \SemiNormed space. 
    latex 

    \weak
    
    \weakstar


    


\end{df}                                

 Similar to in the context of a normed space, if X is a seminormed space, we define the weak topology on X to be the topology on X generated by $X^*$, and the $weak^*$ topology on $X^*$ to be the topology generated by $c(X)$.
Before moving on to the classical theory revamped, I present on more useful result about weak topologies of seminormed spaces. 
\begin{prop}[Weak Quotients]
    \label{prop:weakquotients}
    Let X be a seminormed space and $\{Y_\alpha\}_{\alpha \in A}$ be a collection of topological spaces. For each $\alpha \in A$ let $\phi_\alpha:X \to Y_\alpha$ have the property that for every $x,y \in X$, for every $\alpha \in A$, $\norm{x-y}=0 \implies \phi_\alpha(x)=\phi_\alpha(y)$. 
    For each $\alpha \in A$, define $\tilde{\phi}_\alpha:X/\norm{\cdot}^{-1}\{0\} \to Y_\alpha$ by
    $\tilde{\phi}_{\alpha}[x] = \phi_\alpha x$. Let $\T_w$ denote the weak topology on X induced by $\{\phi_\alpha\}_{\alpha \in A}$, and $\T_{\tilde{w}}$ denote the weak topology on $X/\norm{\cdot}^{-1}\{0\}$ induced by $\{\tilde{\phi}_{\alpha}\}_{\alpha \in A}$. Then 
    \begin{equation}
        (X,\T_w)/\norm{\cdot}^{-1}\{0\} = (X/\norm{\cdot}^{-1}\{0\}, \T_{\tilde{w}})
    \end{equation}
    \begin{proof}
    \end{proof} 
\end{prop} 


Finally, before we move on, recall that if $X,Y$ are Topological vector spaces, we can topologize the set of continuous linear operators from X to Y, denoted $BL(X,Y)$ by saying that $\{T_\alpha\}_{\alpha \in A} \subset BL(X,Y)$ converges to $T \in BL(X,Y)$ if there is a neighborhood U of 0 in X such that $T_{\alpha}x \to Tx$ uniformly for $x \in U$. 
