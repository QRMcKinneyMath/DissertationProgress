%\section{Foundations}
\subsection{Metatheoretical Constructs}
\newcommand{\Character}[0]{\textbf{\hyperref[def:Character]{Character}}\xspace}
\newcommand{\Characters}[0]{\textbf{\hyperref[def:Character]{Characters}}\xspace}
\begin{df}[\Character]
\label{def:Character}

\rm
   A \Character is a splotch of ink on a page 
   or lights on a screen that can be recognized as such.
   They are the means by which we communicate math, but are not
   in and of themselves mathematical objects.
   Roughly they act as the silent interfaces between different people's brains, 
   and notably between my own brain and this document, as well as your 
   brain and this document. 
   \Characters exist outside of mathematics. 
   A \Character can represent a mathematical concept, 
   but is not actually that concept, though we may sometimes say 
   that as an abuse of language. 
\end{df}

\newcommand{\IntuitiveSetTheory}[0]{\textbf{\hyperref[def:IntuitiveSetTheory]{Intuitive Set Theory}}\xspace}

\begin{df}[\IntuitiveSetTheory]
\label{def:IntuitiveSetTheory}

\rm
    Any mathematical theory has a collection of axioms, 
    or a set of axiom schema including set theory. 
    For this reason, any axiomatic theory, including set theory
    cannot itself be defined
    without some intuitive notion of set, collection, or class.
    Basically, an intuitive set is any well defined collection of items. 
    Intuitive set theory has its issues, but we will mostly sweep those under the rug
    and use it as the basis for propositional logic, and set theory itself. 
\end{df}

\subsection{Symbol, Alphabet, String, Grammar}
\newcommand{\Symbol}[0]{\textbf{\hyperref[def:Symbol]{Symbol}}\xspace}
\newcommand{\Symbols}[0]{\textbf{\hyperref[def:Symbol]{Symbols}}\xspace}
\newcommand{\SymbolEqual}[0]{\hyperref[def:Symbol]{\ensuremath{=_{\Symbol}}}\xspace}
\newcommand{\SymbolOccurence}[0]{\textbf{\hyperref[def:Symbol]{Symbol Occurence}}\xspace}
\newcommand{\SymbolOccurences}[0]{\textbf{\hyperref[def:Symbol]{Symbol Occurences}}\xspace}
\begin{df}[\Symbol]
\label{def:Symbol}

\rm
    The \Symbol is the atomic primitive of language theory. 
    Multiple instances of the same \Symbol are interchangeable. 
    Different \Symbols can be identified as such. 
    Within a certain context, 
    a character or sequence of characters 
    can represent a \Symbol.
    When we say that a character or a sequence 
    of characters is a \Symbol, 
    we mean that that character or sequence 
    of characters represents a \Symbol. 
    If $c$ and $b$ are both used to represent 
    the same symbol in a certain context, 
    then we write $c \SymbolEqual b$.
    and we either verbally describe the context in which that relationship
    is meant to hold, or we hope and pray that context makes it clear.
    Different instances of the same \Symbol 
    will sometimes be distinguished from each other, 
    which we will refer to as \SymbolOccurences.
\end{df}

\newcommand{\Alphabet}[0]{\textbf{\hyperref[def:Alphabet]{Alphabet}}\xspace}
\newcommand{\Alphabets}[0]{\textbf{\hyperref[def:Alphabet]{Alphabets}}\xspace}
\begin{df}[\Alphabet]
\label{def:Alphabet}

\rm 
    An \Alphabet is a collection of \Symbols which contains at least one \Symbol.
    A character or sequence of characters can represent an \Alphabet. 
    If we say that a character or string of characters is an \Alphabet, 
    then we mean that that character or string of characters represents that \Alphabet.
\end{df}

\newcommand{\String}[0]{\textbf{\hyperref[def:String]{String}}\xspace}
\newcommand{\Strings}[0]{\textbf{\hyperref[def:String]{Strings}}\xspace}
\newcommand{\StringEqual}[0]{\hyperref[def:String]{\ensuremath{=_{\String}}}\xspace}
\begin{df}[\String]
\label{def:String}

\rm
    A \String in an \Alphabet $A$ is a finite ordered list of \Symbols in $A$.
    The number of \Symbols in a string is called the length of that string.
    A string can contain multiple instances of the same \Symbol.
    A \String may be represented by a character, or a sequence of characters.
    If we say that a character or sequence of characters is a \String, then we mean that
    that character or sequence of characters represents a \String.
    Suppose $x$ is a \String of length m whose $i^{th}$ element
    is the \Symbol represented by $x_i$.
    Then, for every counting number i between 1 and m.
    Then we write $x_i \SymbolEqual x(i)$.
    If $x$ and $y$ each are strings of length $m$ such that for 
    $1 \leq i \leq m$ we have 
    $x(i) \SymbolEqual  y(i)$, then we write
    $x \StringEqual y$. 
\end{df}

\newcommand{\StringConcatenation}[0]{\textbf{\hyperref[def:StringConcatenation]{Concatenation}}\xspace}
\newcommand{\StringConcatenations}[0]{\textbf{\hyperref[def:StringConcatenation]{Concatenations}}\xspace}
\begin{df}[\StringConcatenation]

\rm
    Let $x$ be a \String of length $m$ in an \Alphabet $A$.
    Let $y$ be a \String of length $n$ in an \Alphabet $A$. 
    Let $z$ be the \String of length $m+n$ such that 
    for $1 \leq i \leq m$, 
    $z(i) \SymbolEqual x(i)$ 
    and for $m < i \leq m+n$, 
    $z(i) = y(i-m)$.
    Then we call 
    $z$ the 
    \StringConcatenation of $x$ with $y$
    and we write 
    $z \StringEqual xy$
\end{df}

\newcommand{\AppearsIn}[0]{\textbf{\hyperref[def:AppearsIn]{Appears In}}\xspace}
\newcommand{\AppearIn}[0]{\textbf{\hyperref[def:AppearsIn]{Appear In}}\xspace}
\newcommand{\Substring}[0]{\textbf{\hyperref[def:AppearsIn]{Substring}}\xspace}
\newcommand{\Substrings}[0]{\textbf{\hyperref[def:AppearsIn]{Substrings}}\xspace}
\newcommand{\StringOccurence}[0]{\textbf{\hyperref[def:AppearsIn]{Occurence}}\xspace}
\newcommand{\StringOccurences}[0]{\textbf{\hyperref[def:AppearsIn]{Occurences}}\xspace}
\begin{df}[\AppearsIn]
\label{def:AppearsIn}

\rm
    Let \scA be an \Alphabet.
    Let $x$ be a \String of length n in \scA. 
    Let $y$ be a \String of length m in \scA. 
    We say that $x$ 
    \AppearsIn
    y
    if there exists a $k >0$ such that 
    for each $i$, $1 \leq i \leq n$, we have 
    \begin{equation*}
        x(i) \SymbolEqual  y(i+k)
    \end{equation*}
    If $x$ \AppearsIn $y$
    then we may also say that 
    $x$ is a \Substring of $y$. 
    It is possible for a single \String
    to \AppearIn another string in multiple locations. 
    We call disjoint appearances of the same \String 
    in another a \StringOccurence of that \Substring.

\end{df}

\newcommand{\Grammar}[0]{\textbf{\hyperref[def:Grammar]{Grammar}}\xspace}
\newcommand{\Statement}[0]{\textbf{\hyperref[def:Grammar]{Statement}}\xspace}
\newcommand{\Statements}[0]{\textbf{\hyperref[def:Grammar]{Statements}}\xspace}
\begin{df}[\Grammar]
\label{def:Grammar}

\rm
    A \Grammar $G$ on an \Alphabet $A$ is a means of categorizing the 
	collection of \Strings in $A$ in which at least one of the categories 
    is called the collection of \Statements of \scG. 
	Some \Strings may fall into multiple categories, some into 1, and some into none.

\end{df}

\subsection{Axiom, Rule Of Inference, Axiomatic Theory}
\newcommand{\Theory}[0]{\textbf{\hyperref[def:Theory]{Theory}}\xspace}
\newcommand{\Theories}[0]{\textbf{\hyperref[def:Theory]{Theories}}\xspace}
\begin{df}[\Theory]
\label{def:Theory}
    TODO
\rm
\end{df}

\newcommand{\Axiom}[0]{\textbf{\hyperref[def:Axiom]{Axiom}}\xspace}
\newcommand{\Axioms}[0]{\textbf{\hyperref[def:Axiom]{Axioms}}\xspace}
\newcommand{\AxiomScheme}[0]{\textbf{\hyperref[def:Axiom]{Axiom Scheme}}\xspace}
\newcommand{\AxiomSchemes}[0]{\textbf{\hyperref[def:Axiom]{Axiom Schemes}}\xspace}
\begin{df}[\Axiom]
\label{def:Axiom}

\rm
    Let \scA be an \Alphabet.
    Let \scG be a \Grammar on \scA. 
    An \AxiomScheme in \scG is a subcollection \scB of the set of \Statements of \scG, 
    for which there exists an effictive means of deciding whether any given \Statement 
    in \scG is an element of \scB. 
    An element of an \AxiomScheme is called an \Axiom. 
\end{df}

\newcommand{\RuleOfInference}[0]{\textbf{\hyperref[def:RuleOfInference]{Rule Of Inference}}\xspace}
\newcommand{\RulesOfInference}[0]{\textbf{\hyperref[def:RuleOfInference]{Rules Of Inference}}\xspace}
\begin{df}[\RuleOfInference]
\label{def:RuleOfInference}

\rm
    Let \scA be an \Alphabet and \scG a \Grammar on \scA. 
    An n-th order \RuleOfInference on \scG is a
    collection \scR of n-tuples of \Statements of \scG such that 
    there is a concise effective mechanism for determining whether any
    arbitrary n-tuple of \Statements in \scG is an element of \scR. 
\end{df}


\newcommand{\SatisfyAxiom}[0]{\textbf{\hyperref[def:SatisfyAxiom]{Satisfy}}\xspace}
\newcommand{\SatisfiesAxiom}[0]{\textbf{\hyperref[def:SatisfyAxiom]{Satisfies}}\xspace}
\begin{df}[\SatisfyAxiom \AxiomScheme]
\label{def:SatisfyAxiom}

\rm
    Let \scT be a \Theory in a \Grammar \scG and \scV be an \AxiomScheme in \scG. 
    We say that \scT \SatisfiesAxiom the \AxiomScheme \scV if every \Axiom in \scV 
    is a \Theorem in \scT. 
\end{df}

\newcommand{\SatisfyRule}[0]{\textbf{\hyperref[def:SatisfyRule]{Satisfy}}\xspace}
\newcommand{\SatisfiesRule}[0]{\textbf{\hyperref[def:SatisfyRule]{Satisfies}}\xspace}
\begin{df}[\SatisfyRule \RuleOfInference]
\label{def:SatisfyRule}

\rm
    Let \scG be a \Grammar. 
    Let \scT be a \Theory in \scG. 
    Let n be a counting number where $n>1$. 
    Let \scR be an n-th order \RuleOfInference in \scG. 
    We say that \scT \SatisfiesRule \scR if 
    whenever $(B_1, B_2, \cdots, B_n)$ are elemnts of \scR 
    and $\isthm{\scT}B_1$, $\isthm{\scT}B_2$, $\cdots$, $\isthm{\scT}B_{n-1}$, 
    then $\isthm{\scT}B_n$. 


\end{df}

\newcommand{\AxiomaticTheory}[0]{\textbf{\hyperref[def:AxiomaticTheory]{Axiomatic Theory}}\xspace}
\newcommand{\AxiomaticTheories}[0]{\textbf{\hyperref[def:AxiomaticTheory]{Axiomatic Theories}}\xspace}
\begin{df}[\AxiomaticTheory]
\label{def:AxiomaticTheory}

\rm
    Let \scA be an \Alphabet and let \scG
    be a \Grammar on \scA. 
    Let $A_1, \cdots, A_n$ be a collection of 
    \AxiomSchemes on \scG. 
    Let $R_1, \cdots, R_m$ be a collection of 
    \RulesOfInference on \scG. 
    We define the 
    \AxiomaticTheory with \AxiomSchemes $A_1, \cdots, A_n$
    and \RulesOfInference $R_1, \cdots, R_n$ 
    to be the 
    smallest possible \Theory on \scG which 
    \SatisfiesAxiom each $A_i$ for $1 \leq i \leq n$
    and which \SatisfiesRule each $R_j$ for $1 \leq j \leq m$. 
\end{df}

\newcommand{\Proof}[0]{\textbf{\hyperref[def:Proof]{Proof}}\xspace}
\newcommand{\Proofs}[0]{\textbf{\hyperref[def:Proof]{Proofs}}\xspace}
\newcommand{\Proven}[0]{\textbf{\hyperref[def:Proof]{Proven}}\xspace}
\newcommand{\Proves}[0]{\textbf{\hyperref[def:Proof]{Proves}}\xspace}
\newcommand{\Prove}[0]{\textbf{\hyperref[def:Proof]{Prove}}\xspace}
\begin{df}[\Proof]
\label{def:Proof}

\rm
    Let \scT be an \AxiomaticTheory
    with \AxiomSchemes $A_1, \cdots, A_n$
    and \RulesOfInference $R_1, \cdot R_m$, 
    where for each $i$, $1 \leq i \leq m$, 
    $R_i$ is a \RuleOfInference of order $k_i$. 
    A \Proof in \scT  for a statement $S_t$ in \scG
    is a finite sequence of \Statements in \scG, 
    $S_1, \cdots, S_{t-2}, S_{t-1}, S_t$
    where for each $j$, $1 \leq j \leq t$, at least one of the following holds. 
    \begin{enumerate}
        \item $S_j$ is an \Axiom for \scT. 
        \item There exists a p, $1 \leq p \leq m$ and a collection of indices, 
        $l_1, l_2, \cdots, l_{k_p-1}$ such that each $l_q < j$ and 
        $(S_{l_1}, S_{l_2}, \cdots, S_{l_{k_p-1}}, S_j$ is in $R_p$
    \end{enumerate}
\end{df}


\begin{rmk}[\Theorem Iff \Proof]

\rm
    Notice that the \Theorems of an \AxiomaticTheory
    consist of exactly the \Statements for which there exists 
    a \Proof within that \Theory. 
\end{rmk}

\subsection{First Order Language}
\newcommand{\Implication}[0]{\textbf{\hyperref[def:PredicateCalculusSymbols]{Implication}}\xspace}
\newcommand{\Negation}[0]{\textbf{\hyperref[def:PredicateCalculusSymbols]{Negation}}\xspace}
\newcommand{\LeftParenthesis}[0]{\textbf{\hyperref[def:PredicateCalculusSymbols]{Left Parenthesis}}\xspace}
\newcommand{\RightParenthesis}[0]{\textbf{\hyperref[def:PredicateCalculusSymbols]{Right Parenthesis}}\xspace}
\newcommand{\Comma}[0]{\textbf{\hyperref[def:PredicateCalculusSymbols]{Comma}}\xspace}
\newcommand{\UniversalQuantifier}[0]{\textbf{\hyperref[def:PredicateCalculusSymbols]{Universal Quantifier}}\xspace}
\newcommand{\IndividualVariable}[0]{\textbf{\hyperref[def:PredicateCalculusSymbols]{Individual Variable}}\xspace}
\newcommand{\IndividualVariables}[0]{\textbf{\hyperref[def:PredicateCalculusSymbols]{Individual Variables}}\xspace}
\newcommand{\IndividualConstant}[0]{\textbf{\hyperref[def:PredicateCalculusSymbols]{Individual Constant}}\xspace}
\newcommand{\IndividualConstants}[0]{\textbf{\hyperref[def:PredicateCalculusSymbols]{Individual Constants}}\xspace}
\newcommand{\FunctionLetter}[1]{\textbf{\hyperref[def:PredicateCalculusSymbols]{Function Letter Of Order #1}}\xspace}
\newcommand{\FunctionLetters}[1]{\textbf{\hyperref[def:PredicateCalculusSymbols]{Function Letters Of Order #1}}\xspace}
\newcommand{\PredicateLetter}[1]{\textbf{\hyperref[def:PredicateCalculusSymbols]{Predicate Letter Of Order #1}}\xspace}
\newcommand{\PredicateLetters}[1]{\textbf{\hyperref[def:PredicateCalculusSymbols]{Predicate Letters Of Order #1}}\xspace}
\newcommand{\PredicateCalculus}[0]{\textbf{\hyperref[def:PredicateCalculusSymbols]{Predicate Calculus}}\xspace}



\begin{df}[\Symbols of the \PredicateCalculus]
\label{def:PredicateCalculusSymbols}

\rm
    Throughout this section, we let \scA be an
    \Alphabet with the following \Interpretation in which 
    \scA is partitioned into the following classes.
    \begin{enumerate}
        \item The singleton containing the \Implication \Symbol, which we represent with the \Character $\implies$. 
        \item The singleton containing the \Negation \Symbol which we represent with the \Character $\neg$. 
        \item The singleton containing the \LeftParenthesis \Symbol, which we represent with the \Character $($. 
        \item The singleton contianing the \RightParenthesis \Symbol, whcih we represent with the \Character $)$. 
        \item The singleton containing the \Comma \Symbol, which we represent with the \Character $,$. 
        \item The signleton containing the \UniversalQuantifier, which we represent witht he \Character $\forall$. 
        \item For each counting number n, the set of \FunctionLetters{n}, each of which may be empty.
        \item For each counting number n, the set of \PredicateLetters{n}, which is nonempty for at least 1 value of n. 
        \item The set of \IndividualVariables, which may be empty. 
        \item The set of \IndividualConstants, which may be empty. 
    \end{enumerate}
    
\end{df}

\newcommand{\Term}[0]{\textbf{\hyperref[def:Term]{Term}}\xspace}
\newcommand{\Terms}[0]{\textbf{\hyperref[def:Term]{Terms}}\xspace}

\begin{df}[\Term]
\label{def:Term}

\rm
    We say that a \String C in \scA is a \Term in \scG if and only if 
    one or both of the following hold. 
    \begin{enumerate}
        \item $C \StringEqual x$ where $x$ is an \IndividualVariable or an \IndividualConstant in \scA.
        \item $C \StringEqual f(x_1,x_2,\cdots,x_n)$ where $f$ is a \FunctionLetter{n} 
        and each $x_i$ is a \Term. 
    \end{enumerate}
\end{df}

\newcommand{\AtomicFormula}[0]{\textbf{\hyperref[def:AtomicFormula]{Atomic Formula}}\xspace}
\newcommand{\AtomicFormulae}[0]{\textbf{\hyperref[def:AtomicFormula]{Atomic Formulae}}\xspace}

\begin{df}[\AtomicFormulae]
\label{def:AtomicFormula}

\rm
    Let $b$ be a \PredicateLetter{n} in \scA. 
    For $1 \leq i \leq n$, let $x_i$ be a \Term in \scG. 
    Then we call 
    \begin{equation*}
        b(x_1,x_2,\cdots,x_n)
    \end{equation*}
    an \AtomicFormula in \scG. 
\end{df}

\newcommand{\WFF}[0]{\textbf{\hyperref[def:WellFormedFormula]{Well Formed Formula}}\xspace}
\newcommand{\WFFs}[0]{\textbf{\hyperref[def:WellFormedFormula]{Well Formed Formulas}}\xspace}
\newcommand{\Wff}[0]{\textbf{\hyperref[def:WellFormedFormula]{Well Formed Formula}}\xspace}
\newcommand{\Wffs}[0]{\textbf{\hyperref[def:WellFormedFormula]{Well Formed Formulas}}\xspace}
\begin{df}[\WFF]
\label{def:WellFormedFormula}

\rm
    Let $C$ be a \String in \scA. 
    We say that $C$ is a 
    \WFF in \scG if and only if one or more of the following hold. 
    \begin{enumerate}[label=(\roman*), ref={\ref{def:WellFormedFormula}.~\roman*}]
        \item 
        \label{def:WFF:AtomicFormula}
        $C$ is an \AtomicFormula in \scG.
        \item 
        \label{def:WFF:Negation}
        $C \StringEqual (\neg B)$ where $B$ is a \WFF in \scG.
        \item 
        \label{def:WFF:Implication}
        $C \StringEqual (B \implies D)$ where $B$ and $D$ are \WFFs in \scG.
        \item 
        \label{def:WFF:Quantification}
        $C \StringEqual ((\forall y)B)$ where $B$ is a \WFF in \scG and $y$ is an \IndividualVariable.
    \end{enumerate}
\end{df}

\newcommand{\WFFGeneratedByWFF}[0]{\textbf{\hyperref[def:WFFGeneratedByWFF]{Generated By  }}\xspace}
\newcommand{\CompleteCollectionOfStatementLetters}[0]{\textbf{\hyperref[def:WFFGeneratedByWFF]{Complete Collection Of Statement Letters}}\xspace}
\newcommand{\CompleteCollectionOfAtomicStatementLetters}[0]{\textbf{\hyperref[def:WFFGeneratedByWFF]{Complete Collection Of Atomic Statement Letters}}\xspace}

\begin{df}[\WFF \WFFGeneratedByWFF \WFF]
\label{def:WFFGeneratedByWFF}

\rm
    Let $C$ be a \WFF in \scG.
    Let $B_1, B_2, \cdots, B_k$ be a collection of \WFFs in \scG.
    We say that $C$ is \WFFGeneratedByWFF
    $B_1, B_2, \cdots, B_k$ if one or more of the following hold. 
    \begin{enumerate}
        \item For some $i$, $1 \leq i \leq k$, we have $C \StringEqual B_i$. 
        \item There is a \WFF $D$ in \scG such that $D$ is \WFFGeneratedByWFF 
        $B_1, B_2, \cdots, B_k$ and $C \StringEqual ( \neg D)$. 
        \item There are \WFFs $E$ and $F$ in 
        \scG, each of which are individually 
        \WFFGeneratedByWFF 
        $B_1, B_2, \cdots, B_k$
        such that 
        $C \StringEqual (E \implies F)$.
        \item There is an \IndividualVariable 
        $y$ in $\scA$ and a \WFF $G$ in \scG 
        such that $G$ is 
        \WFFGeneratedByWFF 
        $B_1, B_2, \cdots, B_k$, and 
        $C \StringEqual ((\forall y)G)$.
    \end{enumerate}
    If $C$ is \WFFGeneratedByWFF $B_1, B_2, \cdots, B_k$, 
    then we say that 
    $B_1, B_2,\cdots, B_k$ forms a 
    \CompleteCollectionOfStatementLetters
    for $C$. 
    If each member of $B_1, B_2, \cdots, B_k$ 
    is an \AtomicFormula
    (which is possible since all \AtomicFormulae
    are \WFFs 
    by \ref{def:WFF:AtomicFormula}
    )
    then we say that 
    $B_1, B_2,\cdots, B_k$ forms a 
    \CompleteCollectionOfAtomicStatementLetters
    for $C$.
\end{df}

\newcommand{\FundamentalAtomicGenerator}[0]{\textbf{\hyperref[rmk:FundamentalAtomicGenerator]{Fundamental Atomic Generator}}\xspace}
\begin{rmk}
\label{rmk:FundamentalAtomicGenerator}

\rm
    It is clear that
    for any \WFF $C$ 
    in \scG, 
    there exists a 
    unique \CompleteCollectionOfAtomicStatementLetters
    for $C$ 
    which is contained within every 
    other 
    \CompleteCollectionOfAtomicStatementLetters
    for $C$. 
    We call this the \FundamentalAtomicGenerator
    for $C$. 
    It turns out that the 
    \FundamentalAtomicGenerator for $C$ 
    is exactly the set of \AtomicFormulae
    which
    \AppearsIn
    $C$.      
\end{rmk}
 
\newcommand{\QuantifierScope}[0]{\textbf{\hyperref[def:QuantifierScope]{Scope}}\xspace}
\newcommand{\QuantifierScopes}[0]{\textbf{\hyperref[def:QuantifierScope]{Scopes}}\xspace}

\begin{df}[\QuantifierScope]
\label{def:QuantifierScope}
\rm
    Let $A$ be a \WFF in \scG such that
    $A \StringEqual ((\forall y)B)$ 
    where $B$ is a \WFF and $y$ is an \IndividualVariable in \scA.
    We say that the \QuantifierScope of $A$, 
    or when confusion is unlikely, the 
    \QuantifierScope of the quantifier $(\forall y)$ is the 
    collection of all \StringOccurences of \Terms in $B$. 
\end{df}

\newcommand{\BoundOccurence}[0]{\textbf{\hyperref[def:BoundOccurence]{Bound}}\xspace}

\begin{df}[\BoundOccurence]
\label{def:BoundOccurence}

\rm
    Let $B$ be a \Wff in \scG. 
    Let $x$ be an \IndividualVariable that \AppearsIn $B$.. 
    We say that an \StringOccurence of $x$ is \BoundOccurence in $B$. 
    if either of the following two hold. 
    \begin{enumerate}
        \item That \StringOccurence is immediately preceeded by the \Symbol $\forall$.
        \item That \StringOccurence is within the \QuantifierScope of the quantifier
        $(\forall x)$. 
    \end{enumerate}

\end{df}

\newcommand{\FreeOccurence}[0]{\textbf{\hyperref[def:FreeOccurence]{Free}}\xspace}

\begin{df}[\FreeOccurence]
\label{def:FreeOccurence}

\rm
    Let $B$ be a \Wff in \scG. 
    Let $x$ be an \IndividualVariable that \AppearsIn $B$. 
    We say that an \StringOccurence of $x$ is \FreeOccurence in $B$. 
    if it is not \BoundOccurence in $B$. 
\end{df}

\newcommand{\FreeVariable}[0]{\textbf{\hyperref[def:FreeBoundVariable]{Free}}\xspace}
\newcommand{\BoundVariable}[0]{\textbf{\hyperref[def:FreeBoundVariable]{Bound}}\xspace}

\begin{df}[\FreeVariable, \BoundVariable]
\label{def:FreeBoundVariable}

\rm
    Let $B$ be a \Wff in \scG.
    Let $x$ be an \IndividualVariable that \AppearsIn $B$. 
    We say that $x$ is \FreeVariable in $B$
    if it has an \StringOccurence that is \FreeOccurence in $B$. 
    We say that $x$ is \BoundVariable in $B$
    if it has an \StringOccurence that is \BoundOccurence in $B$. 

    Note that $x$ can be both
    \FreeVariable
    and 
    \BoundVariable
    in $B$. 
\end{df}

\begin{rmk}[\FreeVariable notation]
\label{rmk:FreeVariableNotation}

\rm
    Let $B$ be a \Wff in \scG. 
    Let $x_1, x_2, \cdots, x_n$ be the \IndividualVariables
    which are \FreeVariable in $B$. 
    We will often denote $B$ as
    $B(x_1,x_2, \cdots, x_n)$, to give more information. 
    Furthermore, if $t_1, t_2, \cdots, t_n$ are \Terms, 
    then we denote with 
    $B(t_1, t_2, \cdots, t_n)$ 
    the \String produced by substituting each
    instance of $x_i$ in
    $B(x_1, x_2, \cdots, x_n)$ 
    with $t_i$
    for $1 \leq i \leq n$. 
    Note that the resultant \String is still a \WFF.
\end{rmk}
 
\newcommand{\FreeTerm}[0]{\textbf{\hyperref[def:FreeTerm]{Free}}\xspace}

\begin{df}[\FreeTerm]
\label{def:FreeTerm}

\rm
    Let $B$ be a \WFF in \scG. 
    Let $t$ be a \Term in \scG. 
    Let $x$ be an \IndividualVariable that \AppearsIn $B$. 
    Then we say that 
    $t$ is \FreeTerm for $x$ in $B$ if 
    for every \IndividualVariable $y$ 
    that \AppearsIn $t$, there is no 
    \FreeOccurence of $x$ in $B$ within the \QuantifierScope 
    of the quantifier $(\forall y)$. 
\end{df}
 
\newcommand{\FirstOrderLanguage}[0]{\textbf{\hyperref[def:FirstOrderLanguage]{First Order Language}}\xspace}
\begin{df}[\FirstOrderLanguage]
\label{def:FirstOrderLanguage}
\rm
We call the \Grammar \scG defined throughout this section a \FirstOrderLanguage
on the \Alphabet \scA.
\end{df}
 
\subsection{Properties Of First Order Theories}
\newcommand{\ConsistentTheory}[0]{\textbf{\hyperref[def:ConsistentTheory]{Consistent}}\xspace}
\newcommand{\ConsistencyTheory}[0]{\textbf{\hyperref[def:ConsistentTheory]{Consistency}}\xspace}
\newcommand{\InconsistentTheory}[0]{\textbf{\hyperref[def:ConsistentTheory]{Inconsistent}}\xspace}
\newcommand{\InconsistencyTheory}[0]{\textbf{\hyperref[def:ConsistentTheory]{Inconsistency}}\xspace}

\begin{df}[\ConsistentTheory, \InconsistentTheory]
\label{def:ConsistentTheory}

\rm
    Let \scG be a \FirstOrderLanguage on an \Alphabet \scA. 
    Let \scT be a \Theory on \scG. 
    We say that \scT is \InconsistentTheory, or that \scT posesses
    \InconsistencyTheory if there
    exists a \Statement S such that 
    $\isthm{\scT}S$ and $\isthm{\scT}\pa{\neg S}$.
    If \scT is not \InconsistentTheory, 
    then we say that it is \ConsistentTheory and that it posesses \ConsistencyTheory.
\end{df}
 
\newcommand{\CompleteTheory}[0]{\textbf{\hyperref[def:CompleteTheory]{Consistent}}\xspace}
\newcommand{\CompletenessTheory}[0]{\textbf{\hyperref[def:CompleteTheory]{Completeness}}\xspace}
\newcommand{\IncompleteTheory}[0]{\textbf{\hyperref[def:CompleteTheory]{Incomplete}}\xspace}
\newcommand{\IncompletenessTheory}[0]{\textbf{\hyperref[def:CompleteTheory]{Incompleteness}}\xspace}

\begin{df}[\CompleteTheory, \IncompleteTheory]
\label{def:CompleteTheory}

\rm
    Let \scG be a \FirstOrderLanguage on an \Alphabet \scA. 
    Let \scT be a \Theory on \scG. 
    We say that \scT is \IncompleteTheory, or that \scT 
    posesses \IncompletenessTheory 
    if there exists  a \Statement S in \scG such that 
    Neither $\isthm{\scT}S$ nor $\isthm{\scG}\pa{\neg S}$. 
\end{df}
 


\section{Set Theory}



\newcommand{\scPowerSet}[1]{
	\ensuremath{2^{#1}}
}





\subsection{Functions}
\label{def:Relation}
\newcommand{\Relation}[0]{
    \bf \hyperref[def:Relation]{Relation} \rm
}
\begin{df}[\Relation]
    Let $X \neq \emptyset$ be a set. 
    We say that $R$ is a \Relation
    on X if $R \subset X \times X$. 
    If $(a,b) \in R$, then we may write
    $a R b$. 
\end{df}
\newcommand{\Function}[0]{\textbf{\hyperref[def:Function]{Function}}\xspace}
\newcommand{\Functions}[0]{\textbf{\hyperref[def:Function]{Functions}}\xspace}
\newcommand{\Map}[0]{\textbf{\hyperref[def:Function]{Map}}\xspace}
\newcommand{\Maps}[0]{\textbf{\hyperref[def:Function]{Maps}}\xspace}
\newcommand{\Mapping}[0]{\textbf{\hyperref[def:Function]{Mapping}}\xspace}
\newcommand{\Mappings}[0]{\textbf{\hyperref[def:Function]{Mappings}}\xspace}
\newcommand{\FunctionDomain}[0]{\textbf{\hyperref[def:Function]{Domain}}\xspace}
\newcommand{\FunctionDomains}[0]{\textbf{\hyperref[def:Function]{Domains}}\xspace}
\newcommand{\FunctionCodomain}[0]{\textbf{\hyperref[def:Function]{Codomain}}\xspace}
\newcommand{\FunctionCodomains}[0]{\textbf{\hyperref[def:Function]{Codomains}}\xspace}
\newcommand{\FunctionRange}[0]{\textbf{\hyperref[def:Function]{Range}}\xspace}
\newcommand{\FunctionRanges}[0]{\textbf{\hyperref[def:Function]{Ranges}}\xspace}
\newcommand{\FunctionImage}[0]{\textbf{\hyperref[def:Function]{Image}}\xspace}
\newcommand{\FunctionImages}[0]{\textbf{\hyperref[def:Function]{Images}}\xspace}
\newcommand{\FunctionPreimage}[0]{\textbf{\hyperref[def:Function]{Preimage}}\xspace}
\newcommand{\FunctionPreimages}[0]{\textbf{\hyperref[def:Function]{Preimages}}\xspace}
\begin{df}[\Function]
\label{def:Function}

\rm
    Let $X$ and $Y$ be nonempty sets.
    Let $f \subset X \times Y$ 
    such that for each $x \in X$ 
    there exists a unique $y \in Y$ 
    such that 
    $(x,y) \in f$. 
    Then we say that $f$ 
    is a \Function
    with \FunctionDomain $X$
    and \FunctionCodomain $Y$
    and we write 
    $f:X \to Y$. 
    If $(x,y) \in f$, then we write $f(x) = y$. 
    We may also call $f$ 
    a 
    \Map
    or a 
    \Mapping.
    If $A \subset X$ 
    and $B \subset Y$, 
    then we denote 
    \begin{equation*}
    f\pa{A}= \{f(x)\in Y : x \in A\} \\
    \; \; \;
    f^{-1}\pa{B} = \{x \in X : f(x) \in B \}
    \end{equation*}
    We call 
    $f(A)$ 
    the \FunctionImage
    of $A$ under $f$
    and we call
    $f^{-1}\pa{B}$ 
    the 
    \FunctionPreimage
    of $B$ under
    $f$. 
    We call $f(X)$ 
    the \FunctionRange
    of $f$. 
    When the domain of a function is understood, we may also refer to an unnamed map
    f by writing, 
    $x \to f(x)$
    If $\scA \subset 2^X$ and $\scB \subset 2^Y$, then we write
    \begin{equation*}
    f\pa{\scA} = \{ f(A) : A \in \scA\}\\
    \; \; \;
    f^{-1}\pa{\scB} = \{ f^{-1}(B) : B \in \scB\}
    \end{equation*}
\end{df}

\newcommand{\SetDiagonal}[0]{\textbf{\hyperref[def:SetDiagonal]{Diagonal}}\xspace}
\newcommand{\SetDiagonals}[0]{\textbf{\hyperref[def:SetDiagonal]{Diagonals}}\xspace}
\newcommand{\scSetDiagonal}[1]{\ensuremath{\hyperref[def:SetDiagonal]{\Delta\pa{#1}}}\xspace}
\begin{df}[Set Diagonal]
\label{def:SetDiagonal}

\rm
    Let $X$ be a set. 
    We define 
    $\scSetDiagonal{X}=\{(x,x)\in X \times X | x \in X\}$
    and we call 
    \scSetDiagonal{X} 
    the 
    \SetDiagonal
    of $X$.
\end{df}

\newcommand{\InsertionFunction}[0]{
    \textbf{\hyperref[def:InsertionFunction]{Insertion Function}}
}
\newcommand{\InsertionFunctions}[0]{
    \textbf{\hyperref[def:InsertionFunction]{Insertion Functions}}
}\begin{df}[Insertion Function]
\label{def:InsertionFunction}

\rm
    Let $A \subset B$ and define 
    $f:A \to B$ by $f(x)=x$. 
    The we call $f$ the 
    \InsertionFunction of $A$ into $B$. 
\end{df}


\newcommand{\Restriction}[0]{\textbf{\hyperref[def:FunctionRestriction]{Restriction}}\xspace}
\newcommand{\scRestriction}[2]{\ensuremath{\pa{#1}|_{#2}}\xspace}
\begin{df}[Restriction]
\label{def:FunctionRestriction}

\rm
    Let $X,Y$ be sets and 
    let $R$ be a 
    \Relation from 
    $X$ to $Y$. 
    Let $A \subset X$. 
    We define 
    \begin{equation*}
        \scRestriction{R}{A} = \{(x,y)\in R| x \in A \}
    \end{equation*}
    We call 
    \scRestriction{R}{A}
    the 
    \Restriction 
    of the \Relation
    $R$
    to the set 
    $A$. 
\end{df}

\newcommand{\Extension}[0]{\textbf{\hyperref[def:FunctionExtension]{Extension}}\xspace}
\newcommand{\Extensions}[0]{\textbf{\hyperref[def:FunctionExtension]{Extensions}}\xspace}
\begin{df}[\Extension]
\label{def:FunctionExtension}

\rm
    Let $X$ and $Y$ be nonempty sets.
	Let $g:X \to Y$. 
	Let $f$ be a \Restriction of $g$.  
    Then we call $g$ an 
    \Extension of $f$. 
\end{df}

\label{def:RelationInverse}
\newcommand{\RelationInverse}[0]{
    \textbf{\hyperref[def:RelationInverse]{Inverse}}
}
\begin{df}[\RelationInverse]
    Let $X\neq \emptyset$ and
    $Y \neq \emptyset$. 
    Let $R$ be a 
    \Relation
    from $X$ to $Y$. 
    We define 
    \begin{equation*}
    R^{-1} = \{(y,x) \in Y\times X | (x,y) \in R\}
    \end{equation*}
    We call $R^{-1}$ the 
    \RelationInverse
    of $R$. 
\end{df}

\newcommand{\Injective}[0]{
    \bf \hyperref[def:Injective]{Injective} \rm
}
\newcommand{\Injectivity}[0]{
    \bf \hyperref[def:Injective]{Injectiveness} \rm
}
\newcommand{\Injection}[0]{
    \bf \hyperref[def:Injective]{Injection} \rm
}
\newcommand{\Injections}[0]{
    \bf \hyperref[def:Injective]{Injections} \rm
}\begin{df}[Injective]
\label{def:Injective}

\rm
    Let $X,Y$ be sets and let 
    $f:X \to Y$. 
    We say that $f$ is 
    an 
    \Injection, 
    or that $f$ is 
    \Injective if 
    for all $x,y \in X$, 
    if $x \neq y$, then 
    $f(x) \neq f(y)$. 
\end{df}
    

\label{def:Surjective}
\newcommand{\Surjective}[0]{
    \bf \hyperref[def:Surjective]{Surjective} \rm
}
\newcommand{\Surjection}[0]{
    \bf \hyperref[def:Surjective]{Surjection} \rm
}
\begin{df}[\Surjective]
   Let $X,Y$ be sets and let 
   $f:X \to Y$. 
   Suppose that 
   for each $y \in Y$, 
   there exists an 
   $x \in X$ such that 
   $f(x) = y$. 
   Then we say that $f$ 
   is a
   \Surjection, 
   and we call $f$ 
   \Surjective. 
\end{df}

\newcommand{\Bijective}[0]{\textbf{\hyperref[def:Bijective]{Bijective}}\xspace}
\newcommand{\Bijectivity}[0]{\textbf{\hyperref[def:Bijective]{Bijectivity}}\xspace}
\newcommand{\Bijection}[0]{\textbf{\hyperref[def:Bijective]{Bijection}}\xspace}
\newcommand{\Bijections}[0]{\textbf{\hyperref[def:Bijective]{Bijections}}\xspace}
\begin{df}[\Bijection]
\label{def:Bijective}

\rm
    Let $X$ and $Y$ be nonempty sets and let 
    $f:X \to Y$ be \Surjective and \Injective. 
    Then we say that $f$ is 
    \Bijective, or we say that f is a 
    \Bijection. 
\end{df}

\newcommand{\Composition}[0]{\textbf{\hyperref[def:Composition]{Composition}}\xspace}
\newcommand{\Compositions}[0]{\textbf{\hyperref[def:Composition]{Compositions}}\xspace}
\newcommand{\Compose}[0]{\textbf{\hyperref[def:Composition]{Compose}}\xspace}
\newcommand{\scCompose}[0]{\hyperref[def:Composition]{\ensuremath{\circ}}\xspace}
\begin{df}[\Composition]
\label{def:Composition}
\rm
Let $X$, $Y$, and $Z$ be nonempty sets. 
Let $R$ be a \Relation from $X$ into $Y$. 
Let $S$ be a \Relation from $Y$ into $Z$.
We define 
\begin{equation*}
S \scCompose R = \{(x,z)  \in X \times Z: (\exists y \in Y)(xRy \tab[0.25cm] and \tab[0.25cm] ySz) \}
\end{equation*}
We call $S \scCompose R$ the \Composition of $S$ with $R$. 
\end{df}

\subsection{Cardinality}
\newcommand{\Cardinal}[0]{\textbf{\hyperref[def:Cardinality]{Cardinal}}\xspace}
\newcommand{\Cardinals}[0]{\textbf{\hyperref[def:Cardinality]{Cardinals}}\xspace}
\newcommand{\Cardinality}[0]{\textbf{\hyperref[def:Cardinality]{Cardinality}}\xspace}
\newcommand{\Cardinalities}[0]{\textbf{\hyperref[def:Cardinality]{Cardinalities}}\xspace}
\newcommand{\FirstNaturals}[1]{\hyperref[def:Cardinality]{\ensuremath{N_{#1}}}\xspace}
\newcommand{\CardinalityFunction}[1]{\hyperref[def:Cardinality]{\ensuremath{\textbf{Card}\pa{#1}\xspace}}}
\newcommand{\Finite}[0]{\textbf{\hyperref[def:Cardinality]{Finite}}\xspace}
\newcommand{\Infinite}[0]{\textbf{\hyperref[def:Cardinality]{Infinite}}\xspace}
\newcommand{\Denumerable}[0]{\textbf{\hyperref[def:Cardinality]{Denumerable}}\xspace}
\newcommand{\Countable}[0]{\textbf{\hyperref[def:Cardinality]{Countable}}\xspace}
\newcommand{\Uncountable}[0]{\textbf{\hyperref[def:Cardinality]{Uncountable}}\xspace}
\begin{df}[Cardinality]
\label{def:Cardinality}

\rm
    We define $\mathbb{Z}^+=\{1, 2, 3, \cdots\}$. 
    We define $\mathbb{N} = \{0, 1, 2, 3, \cdots\}$.
    Let $n \in \Z^+$. We define 
    $\FirstNaturals{n} = \Z^+ \cap [1,n]$.
    Let $X$ be a set.
    Let $f:X \to \FirstNaturals{n}$ 
    be a 
    \Bijection. 
    Then, we say that 
    $X$ has 
    \Cardinality
    $n$
    and we write 
    $\CardinalityFunction{X}=n$.
    We say that the 
    \Cardinality 
    of the empty set is $0$
    and we write
    $\CardinalityFunction{\emptyset} = 0$.
    More generally, if there exists a 
    \Bijection
    between two sets 
    $Y$ and $Z$, then we write
    $\CardinalityFunction{Y}=\CardinalityFunction{Z}$
    and we say that they have the same 
    \Cardinalities. 
    Define
    $X_0=\N$
    and for $k \in \N$, define 
    $X_{k+1} = \scPowerSet{X_k}$. 
    Then for $k \in \N$, we define 
    $\aleph_k = \CardinalityFunction{X_k}$.
    If $\CardinalityFunction{X} \in \N$, then 
    we say that $X$ is \Finite. 
    If $\CardinalityFunction{Z} \in \N$ or 
    $\CardinalityFunction{Z} = \aleph_0$, 
    then we say that $Z$ is \Denumerable.
    If $\CardinalityFunction{Y} = \aleph_0$, then
    we say that $Y$ is \Countable.
    If $\CardinalityFunction{W}= \aleph_k$ for $k \geq 1$, 
    then we say that $W$ is \Uncountable. 
    If $\CardinalityFunction{V} = \aleph_j$ for $j \in \N$, 
    then we say that $V$ is \Infinite. 


\end{df}




\begin{prop}[Binary to Finite]
\label{prop:FiniteClosure}
\rm
    Let $X$ be a nonempty set. 
    The following are true. 
    \begin{enumerate}[label=(\roman*), ref={\ref{prop:FiniteClosure}.~\roman*}]
        \item \label{prop:FiniteClosure:Intersection}
        If $X$ is closed under binary intersections, then $X$ is closed under
        finite intersections.
        \item \label{prop:FiniteClosure:Union}
        If $X$ is closed under binary unions, 
        then $X$ is closed under finite unions.
    \end{enumerate}
    \begin{proof}[Proof of \ref{prop:FiniteClosure:Intersection}]
        We use induction.
        Let $M$ be the set of positive integers $n$ for which
        $X$ is closed under intersections of n sets. 
        The intersection of a single set equals that set, so $1 \in M$. 
        $2 \in M$ by direct application of the assumption of 
        \ref{prop:FiniteClosure:Intersection}. 
        Let $m \in M$. Let $\{x_i\}_{i=1}^{m+1} \subset 2^X$. 
        Then 
        \begin{equation*}
            \bigcap\limits_{i=1}^{m+1} x_i = \pa{ \bigcap\limits_{i=1}^m x_i} \cap x_{m+1} \in X
        \end{equation*}
        so $m+1 \in M$. 
        Hence $M= \Z^+$ and \ref{prop:FiniteClosure:Intersection} is proven. 
    \end{proof}
    \begin{proof}[Proof of \ref{prop:FiniteClosure:Union}]
         We use induction.
        Let $M$ be the set of positive integers $n$ for which
        $X$ is closed under unions of n sets. 
        The union of a single set equals that set, so $1 \in M$. 
        $2 \in M$ by direct application of the assumption of 
        \ref{prop:FiniteClosure:Union}. 
        Let $m \in M$. Let $\{x_i\}_{i=1}^{m+1} \subset 2^X$. 
        Then 
        \begin{equation*}
            \bigcup\limits_{i=1}^{m+1} x_i = \pa{ \bigcup\limits_{i=1}^m x_i} \cup x_{m+1} \in X
        \end{equation*}
        so $m+1 \in M$. 
        Hence $M= \Z^+$ and \ref{prop:FiniteClosure:Union} is proven. 
    \end{proof}
\end{prop}

\newcommand{\scNested}[2]{\textbf{\hyperref[def:NestingCondition]{Nested}}\ensuremath{\pa{#1, #2}}\xspace}
\newcommand{\Nested}[0]{\textbf{\hyperref[def:NestingCondition]{Nested}}\xspace}
\begin{df}[\Nested]
\label{def:NestingCondition}
    Let $F,G \neq \emptyset$. 
    We say that \scNested{F}{G} holds if, 
    for each $g \in G$, there exists $f \in F$ 
    such that $f \subset g$. 
\end{df}

\subsection{Covers, Partitions}
\newcommand{\Disjoint}[0]{
    \bf \hyperref[def:Disjoint]{Disjoint} \rm
}
\newcommand{\Disjointedness}[0]{
    \bf \hyperref[def:Disjoint]{Disjointedness} \rm
}\begin{df}[Disjoint]
\label{def:Disjoint}

\rm
    Let $X$ and $Y$ be sets such that 
    $X \cap Y = \emptyset$. 
    Then we say that $X$ and $Y$ are 
    \Disjoint. 
    Let $F=\{X_{\alpha}\}_{\alpha \in A}$ 
	be a collection of sets
    such that for each $\alpha, \beta \in A$ 
    with $\alpha \neq \beta$, we have 
    $X_\alpha$ 
    is \Disjoint
    to 
    $X_{\beta}$. 
    Then we say that $F$ is \Disjoint. 
\end{df}

\newcommand{\Cover}[0]{\textbf{\hyperref[def:Cover]{Cover}}\xspace}
\newcommand{\CoveredBy}[0]{\textbf{\hyperref[def:Cover]{Covered By}}\xspace}
\newcommand{\Covers}[0]{\textbf{\hyperref[def:Cover]{Covers}}\xspace}
\newcommand{\Subcover}[0]{\textbf{\hyperref[def:Cover]{Subcover}}\xspace}
\newcommand{\Subcovers}[0]{\textbf{\hyperref[def:Cover]{Subcovers}}\xspace}
\begin{df}[\Cover, \Subcover] 
\label{def:Cover}
\rm
    Let $X$ be a set and let 
    $Y=\{Y_\alpha\}_{\alpha \in A}$ 
	be a collection of sets
    such that 
    \begin{equation*}
        X \subset \bigcup_{\alpha \in A} Y_{\alpha}
    \end{equation*}
    Then we say 
    $Y$ 
    is a 
    \Cover
    of $X$,
    we say $Y$ 
    \Covers $X$,
	and we say $X$ 
	is \CoveredBy
	$Y$. 
    In the context of talking about a 
    \Cover, if every member of a 
    \Cover posses a certain property
    then we may say that the \Cover 
    has that property. 
	An exception to this is that 
    when talking about the 
    \Cardinality
    or \Disjointedness 
    of a \Cover.
	In such cases we are saying
	that the \Cover itself is 
	\Disjoint or of a particular 
	\Cardinality, not that the constituent 
	sets each have that \Cardinality 
	or are themselves \Disjoint collections.
    If $Z \subset Y$ \Covers $X$, then
    we call $Z$ a \Subcover of $Y$. 
\end{df}

\label{def:Partition}
\newcommand{\Partition}[0]{
    \bf \hyperref[def:Partition]{Partition} \rm
}
\newcommand{\Partitions}[0]{
    \bf \hyperref[def:Partition]{Partitions} \rm
}
\begin{df}[\Partition]
    Let $X$ be a set and 
    $Y \subset \scPowerSet{X}$ 
    be a \Disjoint \Cover for $X$. 
    Then we call $Y$ a \Partition
    for $X$. 
\end{df}

\subsection{Infinite Cartesian Product}
\newcommand{\InfiniteCartesianProduct}[0]{\textbf{\hyperref[def:InfiniteCartesianProduct]{Cartesian Product}}\xspace}
\newcommand{\InfiniteCartesianProducts}[0]{\textbf{\hyperref[def:InfiniteCartesianProduct]{Cartesian Products}}\xspace}
\newcommand{\scCartesianProduct}[3]{\ensuremath{\prod\limits_{#1 \in #2}#3_{#1}}\xspace}
\newcommand{\ProjectionMap}[0]{\textbf{\hyperref[def:InfiniteCartesianProduct]{Projection Map}}\xspace}
\begin{df}[\InfiniteCartesianProduct]
\label{def:InfiniteCartesianProduct}
\rm
    Let $A$ be a nonempty set.
    For each $\alpha \in A$, let 
    $X_{\alpha}$ be a nonempty set.
    Define 
    \begin{equation*}
        \prod\limits_{\alpha \in A} X_{\alpha } = \left\{f:A \to \bigcup\limits_{\alpha \in A} X_{\alpha} \in 2^{A \times \bigcup\limits_{\alpha \in A} X_{\alpha}} : (\forall \alpha \in A)(f(\alpha) \in X_{\alpha} ) \right\}
    \end{equation*}
    We call this the 
    \InfiniteCartesianProduct
    of $\{X_\alpha\}_{\alpha \in A}$. 
    For each $\alpha \in A$, we define 
    \begin{equation*}
        \pi_\alpha : \scCartesianProduct{\alpha}{A}{X} \to X_{\alpha }\tab[2cm] \pi_\alpha(f) = f(\alpha)
    \end{equation*}
    We call $\pi_\alpha$ the
    $\alpha-$\ProjectionMap.
\end{df}

\newcommand{\InfiniteSetDiagonal}[0]{\textbf{\hyperref[def:InfiniteSetDiagonal]{Diagonal}}\xspace}
\newcommand{\InfiniteSetDiagonals}[0]{\textbf{\hyperref[def:InfiniteSetDiagonal]{Diagonals}}\xspace}
\newcommand{\scInfiniteSetDiagonal}[2]{\ensuremath{\hyperref[def:InfiniteSetDiagonal]{\Delta_{#1}\pa{#2}}}\xspace}
\begin{df}[\InfiniteSetDiagonal]
    \label{def:InfiniteSetDiagonal}
    rm
    Let $X$ be a set.
    Let $A \neq \emptyset$. 
    We define 
    \begin{equation*}
    \scInfiniteSetDiagonal{A}{X}= \left\{ \{x\}_{\alpha \in A} \in \prod\limits_{\alpha \in A} X | x \in X\right\}
    \end{equation*}
    We call this the \InfiniteSetDiagonal of $X$ with respect to $A$, 
    or, when $A$ is understood, the \InfiniteSetDiagonal of $X$.
\end{df}

\newcommand{\FunctionProduct}[0]{\textbf{\hyperref[def:FunctionProduct]{Function Product}}\xspace} 
\newcommand{\FunctionProducts}[0]{ \textbf{\hyperref[def:FunctionProduct]{Function Products}}\xpsace}
\begin{df}[Function Product]
\label{def:FunctionProduct}
    For $\alpha \in A$, let $f_\alpha : X_\alpha \to Y_\alpha$. 
    Define
    \begin{equation*}
    f: \prod\limits_{\alpha \in A} X_\alpha \to \prod\limits_{\alpha \in A} Y_{\alpha}
    \end{equation*}
    by 
    \begin{equation*}
      f\pa{\left\{x_\alpha\right\}_{\alpha \in A}} = \left\{f_\alpha \pa{x_\alpha} \right\}_{\alpha \in A}
    \end{equation*}
    Then we call $f$ the \FunctionProduct of $\{f_\alpha\}_{\alpha \in A}$
    and we denote 
    $\prod\limits_{\alpha \in A} f_\alpha:=f$.
    or in the case where $\{X_\alpha\}_{\alpha \in A} =\{X_1, \cdots, X_n\}$, 
    we may denote 
    $f=f_1 \times f_2 \times \cdot \times f_n$. 
\end{df}

\subsection{Relations and Orderings}
\label{def:ReflexiveRelation}
\newcommand{\ReflexiveRelation}[0]{
    \bf \hyperref[def:ReflexiveRelation]{Reflexive} \rm
}

\newcommand{\RelationReflexivity}[0]{
    \bf \hyperref[def:ReflexiveRelation]{Reflexivity} \rm
}

\begin{df}[\ReflexiveRelation]
    Let $X \neq \emptyset$ be a set. 
    Let $R$ be a \Relation on X. 
    We say that $R$ is \ReflexiveRelation with respect to X if, 
    or equivalently we say that
    $R$ posseses 
    \RelationReflexivity with respect to X
    if 
    $\{(a,a) | a \in X \} \subset R$.
    
    When X is understood, we may simply say that 
    $R$ is \ReflexiveRelation or that $R$
    posesses \RelationReflexivity. 
\end{df}
\label{def:TransitiveRelation}
\newcommand{\TransitiveRelation}[0]{\textbf{\hyperref[def:TransitiveRelation]{Transitive}}\xspace}
\newcommand{\RelationTransitivity}[0]{\textbf{\hyperref[def:TransitiveRelation]{Transitivity}}\xspace}
\begin{df}[\TransitiveRelation]
\rm
\rm
    Let $X \neq \emptyset$ be a set. 
    Let $R$ be a \Relation on X. 
    We say that $R$ is \TransitiveRelation, 
    or equivalently we say that
    $R$ posseses 
    \RelationTransitivity
    if whenever $(a,b) \in R$ and $(b,c) \in R$, 
    we also have $(a,c) \in R$. 
\end{df}

\label{def:Preorder}
\newcommand{\Preordering}[0]{
    \bf \hyperref[def:Preorder]{Preordering} \rm
}

\newcommand{\Preorder}[0]{
    \bf \hyperref[def:Preorder]{Preorder} \rm
}

\newcommand{\PreorderedSet}[0]{
    \bf \hyperref[def:Preorder]{Preordered Set} \rm
}

\begin{df}[\Preorder]
    Let $X \neq \emptyset$ be a set. 
    Let $R$ be a \Relation on $X$. 
    If $R$ is
    \ReflexiveRelation
    and
    \TransitiveRelation
    then we call $R$
    a \Preorder on $X$, 
    or we equivalently call
    $R$ a \Preordering 
    of $X$ and we call 
    $(X,R)$ a \PreorderedSet.
    \end{df}
\label{def:Comparable}
\newcommand{\Comparable}[0]{
    \bf \hyperref[def:Comparable]{Comparable} \rm
}
\newcommand{\Comparability}[0]{
    \bf \hyperref[def:Comparable]{Comparability} \rm
}

\begin{df}[\Comparable]
   Let $(X,R)$ be a 
   \PreorderedSet. 
   We say that $x,y \in X$ are 
   \Comparable and that
   they possess 
   \Comparability
   if 
   $xRy$ or $yRx$. 
\end{df}

\label{def:SymmetricRelation}
\newcommand{\SymmetricRelation}[0]{
    \bf \hyperref[def:SymmetricRelation]{Symmetric} \rm
}

\newcommand{\RelationSymmetry}[0]{
    \bf \hyperref[def:SymmetricRelation]{Symmetry} \rm
}

\begin{df}[\SymmetricRelation]
    Let $X \neq \emptyset$ be a set. 
    Let $R$ be a \Relation on X. 
    We say that $R$ is \SymmetricRelation, 
    or equivalently we say that
    $R$ posseses 
    \RelationSymmetry
    if whenever $aRb$, we also have $bRa$. 
\end{df}
\begin{prop}
    \label{prop:SymmetricRelation}
\rm
    Let $X \neq \emptyset$ 
    and let $R$ be a \Relation on $X$. 
    The following are true. 
    \begin{enumerate}[label=(\roman*), ref={\ref{prop:SymmetricRelation}~\roman*}]
    \item 
	\label{prop:SymmetricRelation:IntersectionSymmetric} 
	$R \cap R^{-1}$ is \SymmetricRelation.
    \item 
	\label{prop:SymmetricRelation:UnionSymmetric}
	$R \cup R^{-1}$ is \SymmetricRelation.
    \end{enumerate}
    \begin{proof}[Proof of \ref{prop:SymmetricRelation:IntersectionSymmetric}]
    If $R \cap R^{-1} = \emptyset$ 
	then it is trivally \SymmetricRelation.
    Suppose $R \cap R^{-1} \neq \emptyset$ and 
	let $(x,y) \in R \cap R^{-1}$. 
    Then $(x,y) \in R$, 
	implying by \ref{def:SymmetricRelation} that $(y,x) \in R$. 
    Also this implies $(x,y) \in R^{-1}$ so 
	by \ref{def:SymmetricRelation} we have 
    $(y,x) \in \pa{R^{-1}}^{-1}=R$. 
	Hence $(y,x) \in R \cap R^{-1}$, so \RelationSymmetry is 
    verified.
    \end{proof}
    \begin{proof}[Proof of \ref{prop:SymmetricRelation:UnionSymmetric}]
    Let $(x,y) \in R \cup R^{-1}$. Then either $(x,y) \in R$ 
    or $(x,y) \in R^{-1}$. 
    In the former case, 
	$(y,x) \in R^{-1} \subset R \cup R^{-1}$. 
    In the latter case, 
	$(y,x) \in R \subset R \cup R^{-1}$. 
    Hence in either case 
	$(y, x) \in R \cup R^{-1}$ and 
	\RelationSymmetry is verified.
    \end{proof}
\end{prop}

\newcommand{\AntiSymmetricRelation}[0]{\textbf{\hyperref[def:AntiSymmetricRelation]{Anti-Symmetric}}\xspace}
\newcommand{\RelationAntiSymmetry}[0]{\textbf{\hyperref[def:SymmetricRelation]{Anti-Symmetry}}\xspace}

\begin{df}[\AntiSymmetricRelation]
\label{def:AntiSymmetricRelation}
\rm
    Let $X$ be a nonempty set.
    Let $R$ be a \Relation on X. 
    We say that $R$ is \AntiSymmetricRelation
    and we say that
    $R$ posseses 
    \RelationAntiSymmetry
	if $R \cap R^{-1} \subset \scIdentity{X}$. 
\end{df}

\label{def:MaximalElement}
\newcommand{\MaximalElement}[0]{
    \bf \hyperref[def:MaximalElement]{Maximal Element} \rm
}

\newcommand{\Maximum}[0]{
    \bf \hyperref[def:MaximalElement]{Maximum} \rm
}

\begin{df}[Upper Bound]
    Let $X \neq \emptyset$ be a set. 
    Let $R$ be a \Relation on X. 
    Let $Y \subset X$.
    Let $a \in Y$. 
    We say that $a$ is a 
    \MaximalElement of $Y$, 
    or equivalently we say that 
    $a$ is a \Maximum of Y 
    if for every $b \in Y$, 
    we have $b \leq a$. 
    
    If we further assume that 
    Y has exactly 1 \MaximalElement 
    then we write 
    $a=max(Y)$. 
\end{df}
\newcommand{\MinimalElement}[0]{\textbf{\hyperref[def:MinimalElement]{Minimal Element}}\xspace}
\newcommand{\Minimum}[0]{\textbf{\hyperref[def:MinimalElement]{Minimum}}\xspace}
\newcommand{\Minima}[0]{\textbf{\hyperref[def:MinimalElement]{Minima}}\xspace}

\begin{df}[\MinimalElement]
\label{def:MinimalElement}
\rm
    Let $X \neq \emptyset$ be a set. 
    Let $R$ be an 
    \Relation on $X$. 
    Let $Y \subset X$.
    Let $a \in Y$. 
    We say that $a$ is a 
    \MinimalElement of $Y$, 
    or equivalently we say that 
    $a$ is a \Minimum of Y 
    if for every $b \in Y$,
	if $b R a$, then 
    we have $a = b$. 
	The Plural of \Minimum is \Minima, 
	and we represent the set of \Minima of Y with 
	respect to the relation $R$ with 
	$\Minima_R(Y)$, or if $R$ is understood, 
	we represent the set of $\Minima$ of Y with 
	$\Minima(Y)$. 
\end{df}

\begin{prop}[\MinimalElement unique if R is \AntiSymmetricRelation]
\label{prop:MinimalElementUnique}
\rm
	Let $X \neq \emptyset$ be a set. 
	Let $R$ be an
	\AntiSymmetricRelation
	\Relation
	on X. 
	Let $Y \subset X$.
	Let $a$ and $b$ be 
	each be a \MinimalElement
	of Y. 
	Then $a=b$. 
	\begin{proof}
		Since $a \in \Minima(Y)$, 
		$a \leq b$. 
		Since $b \in \Minima(Y)$, 
		$b \leq a$. 
		By \RelationAntiSymmetry, 
		$b = a$. 
	\end{proof}
\end{prop}
\label{def:UpperBound}
\newcommand{\UpperBound}[0]{
    \bf \hyperref[def:UpperBound]{Upper Bound} \rm
}

\newcommand{\UpperBounds}[0]{
    \bf \hyperref[def:UpperBound]{Upper Bounds} \rm
}

\newcommand{\BoundedFromAbove}[0]{
    \bf \hyperref[def:UpperBound]{Bounded From Above} \rm
}

\newcommand{\UB}[0]{
	\bf \hyperref[def:UpperBound]{UpperBound} \rm
}

\begin{df}[\UpperBound]
    Let $X \neq \emptyset$ be a set. 
    Let $R$ be a \Relation on X. 
    Let $Y \subset X$.
    Let $a \in X$. 
    We say that $a$ is an 
    \UpperBound for $Y$ if
    for every $x \in Y$, 
    we have $x R a$. 
    If $a$ is an \UpperBound
    then we also say that 
    the set Y is \BoundedFromAbove
    by a. 
	We denote the set of \UpperBounds of 
	$Y$ with respect to the relation $R$ with
	$\UB_R(Y)$. 
	When $R$ is understood, we denote this set with
	$\UB(Y)$. 
\end{df}
\label{def:LowerBound}
\newcommand{\LowerBound}[0]{
    \bf \hyperref[def:LowerBound]{Lower Bound} \rm
}

\newcommand{\BoundedFromBelow}[0]{
    \bf \hyperref[def:LowerBound]{Bounded From Below} \rm
}

\begin{df}[Lower Bound]
    Let $X \neq \emptyset$ be a set. 
    Let $R$ be a \Relation on X. 
    Let $Y \subset X$.
    Let $a \in X$. 
    We say that $a$ is an 
    \LowerBound for $Y$ if
    for every $x \in Y$, 
    we have $a R x$. 
    
    If $a$ is an \LowerBound
    then we also say that 
    the set Y is \BoundedFromBelow
    by a. 
\end{df}
\label{def:LeastUpperBound}

\newcommand{\LeastUpperBound}[0]{
    \bf \hyperref[def:LeastUpperBound]{Least Upper Bound} \rm
}

\newcommand{\LeastUpperBounds}[0]{
    \bf \hyperref[def:LeastUpperBound]{Least Upper Bounds} \rm
}

\newcommand{\Sup}[0]{
    \bf \hyperref[def:LeastUpperBound]{Sup} \rm
}

\newcommand{\Supremum}[0]{
    \bf \hyperref[def:LeastUpperBound]{Supremum} \rm
}

\newcommand{\Suprema}[0]{
    \bf \hyperref[def:LeastUpperBound]{Suprema} \rm
}

\newcommand{\LUB}[0]{
	\bf \hyperref[def:LeastUpperBound]{LUB} \rm
}

\begin{df}[\LeastUpperBound]
    Let $X \neq \emptyset$ be a set. 
    Let $R$ be a \Relation on X. 
    Let $Y \subset X$.
	Let $a \in X$. 
	We say that $a$ is a
	\LeastUpperBound of $Y$ if 
	$a \in \Minima(\UB(Y))$.
	We denote the set of \LeastUpperBounds
	for $Y$ with $\LUB(Y)$.
	If $b \in \LUB(Y)$, then we 
	also call $b$ a 
	\Supremum of $Y$. 
	The Plural of \Supremum is \Suprema.
	If $\LUB(Y)=\{c\}$, 
	then we write $c=\Sup(Y)$. 
\end{df}
\newcommand{\GreatestLowerBound}[0]{\textbf{\hyperref[def:GreatestLowerBound]{Greatest Lower Bound}}\xspace}
\newcommand{\GreatestLowerBounds}[0]{\textbf{\hyperref[def:GreatestLowerBound]{Greatest Lower Bounds}}\xspace}
\newcommand{\Inf}[0]{\textbf{\hyperref[def:GreatestLowerBound]{Inf}}\xspace}
\newcommand{\Infimum}[0]{\textbf{\hyperref[def:GreatestLowerBound]{Infimum}}\xspace}
\newcommand{\Infima}[0]{\textbf{\hyperref[def:GreatestLowerBound]{Infima}}\xspace}
\newcommand{\GLB}[0]{\textbf{\hyperref[def:GreatestLowerBound]{GLB}}\xspace}
\begin{df}[\GreatestLowerBound]
\label{def:GreatestLowerBound}
\rm
    Let $X$ be a nonempty set.
    Let $R$ be a \Relation on X. 
    Let $Y \subset X$.
	Let $a \in X$. 
	We say that $a$ is a
	\GreatestLowerBound of $Y$ if 
	$a \in \Maxima(\LB(Y))$.
	We denote the set of \GreatestLowerBounds
	for $Y$ with $\GLB(Y)$.
	If $b \in \GLB(Y)$, then we 
	also call $b$ an
	\Infimum of $Y$. 
	The Plural of \Infimum is \Infima. 
	If $\GLB(Y)=\{c\}$, then 
	we write $c=\Inf(Y)$.  
\end{df}

\label{def:EquivalenceRelation}
\newcommand{\EquivalenceRelation}[0]{\textbf{\hyperref[def:EquivalenceRelation]{Equivalence Relation}}\xspace}

\begin{df}[\EquivalenceRelation]
    Let $X \neq \emptyset$ be a set.
    Let $\cong$ be a \Preorder on X. 
    We say that $\cong$ is an \EquivalenceRelation on X
    if it is \SymmetricRelation.
\end{df}

\newcommand{\PartialOrder}[0]{\textbf{\hyperref[def:PartialOrder]{Partial Order}}\xspace}
\newcommand{\PartialOrdering}[0]{\textbf{\hyperref[def:PartialOrder]{Partial Ordering}}\xspace}
\newcommand{\Poset}[0]{\textbf{\hyperref[def:PartialOrder]{Partially Ordered Set}}\xspace}

\begin{df}[\PartialOrder]
\label{def:PartialOrder}
\rm
    Let $X$ be a nonempty set.
    Let $\leq$ be an \AntiSymmetricRelation \Preorder
    on $X$. 
    Then we say that $\leq$ is a \PartialOrder on $X$, 
    we say that $\leq$ is a \PartialOrdering of $X$, and
    we refer to the pair $(X,\leq)$ as a 
    \Poset.    
\end{df}

\label{def:TotalOrder}
\newcommand{\TotalOrder}[0]{
    \bf \hyperref[def:TotalOrder]{Total Order} \rm
}
\newcommand{\TotalOrdering}[0]{
    \bf \hyperref[def:TotalOrder]{Total Ordering} \rm
}

\newcommand{\Toset}[0]{
    \bf \hyperref[def:TotalOrder]{Totally Ordered Set} \rm
}
\begin{df}[\TotalOrder]
    Let $(X,R)$ be a 
    \Poset in which
    every pair of elements is 
    \Comparable. 
    Then we call 
    $R$ a 
    \TotalOrder
    on $X$
    and we call 
    $(X,R)$ a 
    \Toset.
\end{df}

\newcommand{\Chain}[0]{\textbf{\hyperref[def:Chain]{Chain}}\xspace}
\newcommand{\Chains}[0]{\textbf{\hyperref[def:Chain]{Chains}}\xspace}
\begin{df}[\Chain]
\label{def:Chain}
\rm
    Let $(X,\leq)$ be a \Poset. 
    Let $A \subset X$ such that
	$\pa{A, \scRestriction{\leq}{ A \times A}}$ 
	is a 
    \Toset. 
    Then we call $A$ a \Chain in $X$. 
\end{df}

\label{def:Direction}
\newcommand{\Direction}[0]{
    \bf \hyperref[def:Direction]{Direction} \rm
}

\newcommand{\Directions}[0]{
    \bf \hyperref[def:Direction]{Directions} \rm
}

\newcommand{\Directing}[0]{
    \bf \hyperref[def:Direction]{Directing} \rm
}

\newcommand{\Directings}[0]{
    \bf \hyperref[def:Direction]{Directings} \rm
}

\newcommand{\DirectedSet}[0]{
    \bf \hyperref[def:Direction]{Directed Set} \rm
}
\newcommand{\DirectedSets}[0]{
    \bf \hyperref[def:Direction]{Directed Sets} \rm
}
\newcommand{\DirectedSection}[0]{
    \textbf{\hyperref[def:Direction]{Section}}
}
\newcommand{\DirectedSections}[0]{
    \textbf{\hyperref[def:Direction]{Sections}}
}

\begin{df}[\Direction]
    Let $X \neq \emptyset$ be a set.
    Let $\leq$ be a \Preorder on X. 
	If every pair of elements in $X$ has an 
	\UpperBound with respect to $\leq$, then
    we call $\leq$ is a \Direction on $X$, 
	, we call $\leq$ is a \Directing of $X$
	, and we call $(X,\leq)$ is a \DirectedSet.
    If $x_0 \in X$, then we call
    $\{x \in X : x_0 \leq x\}$ the 
    \DirectedSection
    of $x_0$ under $\leq$. 
\end{df}

\label{def:DirectedSection}
\newcommand{\DirectedSection}[0]{
    \textbf{\hyperref[def:DirectedSection]{Section}}
}
\newcommand{\DirectedSections}[0]{
    \textbf{\hyperref[def:DirectedSection]{Sections}}
}
\begin{df}[\DirectedSection of a \DirectedSet]
    Let $(X, \leq)$ be a 
    \DirectedSet. 
    Let $x \in X$. 
    We define 
    \begin{equation}
    S(x,\leq) = \{y \in X | x \leq y\}
    \end{equation}
    We call $S(x,\leq)$ the 
    \DirectedSection of 
    $\leq$ corresponding to $x \in X$. 
\end{df}

\newcommand{\Lattice}[0]{\textbf{\hyperref[def:Lattice]{Lattice}}\xspace}
\newcommand{\Lattices}[0]{\textbf{\hyperref[def:Lattice]{Lattices}}\xspace}
\newcommand{\CompleteLattice}[0]{\textbf{\hyperref[def:Lattice]{Complete Lattice}}\xspace}
\newcommand{\CompleteLattices}[0]{\textbf{\hyperref[def:Lattice]{Complete Lattices}}\xspace}
\newcommand{\LatticeJoin}[0]{\textbf{\hyperref[def:Lattice]{Join}}\xspace}
\newcommand{\LatticeJoins}[0]{\textbf{\hyperref[def:Lattice]{Joins}}\xspace}    
\newcommand{\LatticeMeet}[0]{\textbf{\hyperref[def:Lattice]{Meet}}\xspace}
\newcommand{\LatticeMeets}[0]{\textbf{\hyperref[def:Lattice]{Meets}}\xspace}
\begin{df}[\Lattice, \LatticeJoin, \LatticeMeet]
\label{def:Lattice}
\rm
    Let $(X, \leq)$ be a 
    \Poset
    such that, 
    for every $x,y \in X$, 
    the set 
    $\{x,y\}$ has both a 
    \Supremum 
    and an
    \Infimum. 
    Then we call $(X,\leq)$ 
    \Lattice. 
    Furthermore, we call 
    $\Sup\{x,y\}$
    the 
    \LatticeJoin of $x$ and $y$ 
    and we call
    $\Inf\{x,y\}$ the 
    \LatticeMeet
    of $x$ and $y$. 
    If every nonempty subset of 
    $X$ has both a 
    \Supremum
    and \Infimum 
    then we call $(X,\leq)$
    a \CompleteLattice. 
\end{df}

\newcommand{\Sequence}[0]{\textbf{\hyperref[def:Sequence]{Sequence}}\xspace}
\newcommand{\Sequences}[0]{\textbf{\hyperref[def:Sequence]{Sequences}}\xspace}
\begin{df}[\Sequence]
\label{def:Sequence}
\rm
    Let $X$ be a set. 
    A \Sequence in $X$ is a 
    \Function $f:\N \to X$. 
    If $f$ is a \Sequence 
    in $X$ and 
    $f(n) = x_n$ for $n \in \N$, then 
    we may refer to $\{x_n\}_{n \in \N}$ as the \Sequence itself. 
\end{df}

\subsection{Equivalence Relations}
\label{def:EquivalenceClass}
\newcommand{\EquivalenceClass}[0]{\textbf{\hyperref[def:EquivalenceClass]{Equivalence Class}}\xspace}
\newcommand{\EqClass}[2]{\bra{#1}_{\cong}\xspace}
\begin{df}[Equivalence Class]
    
    Let $X \neq \emptyset$.
    Let $\cong$ be an 
	\EquivalenceRelation
	defined on X.  
    Let $x \in X$. 
    We define the set $[x]_{\cong}$ by 
    \begin{equation}
        [x]_{\cong} = \{y \in X | y \cong x\}
    \end{equation} 
    We call $\EqClass{x}{\cong}$ the \EquivalenceClass of x in $(X, \cong)$. 
\end{df}

\begin{prop}[Equivalence Classes Partition]
    \label{prop:EquivalenceClassesPartition}
    
    Let $X \neq \emptyset$. 
    Let $\cong$ be an equivalence relation defined on X. 
    Let $x,y \in X$. 
    The following are true.
    \begin{equation}
        [x]_{\cong}  \cap [y]_{\cong} \neq \emptyset \iff [x]_{\cong} = [y]_{\cong}  \iff x \cong y \iff [x]_{\cong} \subset [y]_{\cong} \iff [y]_{\cong} \subset [x]_{\cong} 
    \end{equation}
    \begin{equation}
        x \in [x]_{\cong} 
    \end{equation}
    \begin{proof} OBVIOUS \end{proof}
\end{prop}  
\label{def:QuotientSet}
\newcommand{\QuotientSet}[0]{
    \bf \hyperref[def:QuotientSet]{Quotient Set} \rm
}
\newcommand{\QuoSet}[2]{
    #1
    /
    #2
}
\newcommand{\LetBeQuotientSet}[2]{
    Let \QuoSet{#1}{#2} be the \QuotientSet of #1 with respect to the relation #2.
}
\begin{df}[Quotient Set]  
    Let $X \neq \emptyset$.
    Let $\cong$ be an 
	\EquivalenceRelation defined on X.
    We define the set $X/\cong$ by 
    \begin{equation}
        \QuoSet{X}{\cong} = \{ [x]_{\cong} : x \in X\}
    \end{equation}
    We call $\QuoSet{X}{\cong}$ the \QuotientSet of X under the relation $\cong$. 
\end{df} 
\begin{rmk}[Quotient Set Partition]
    \label{rmk:quotientsetpartition}
    \ref{prop:EquivalenceClassesPartition}
	, paired with the fact that 
	$x \in [x]_{\cong}$, 
	implies that 
	$X/\cong$ is a partition of X. 
\end{rmk} 
\label{df:quotient_map}
\newcommand{\QuotientMap}[0]{
        \bf \hyperref[df:quotient_map]{Quotient Map} \rm
    }
    
 \newcommand{\QuotientMapInstance}[3]{
     #1 : #2\to #2/#3
}
\begin{df}[Quotient Map]

    Let $X \neq \emptyset$.
    Let $\cong$ be an 
	\EquivalenceRelation 
	on X.
    \LetBeQuotientSet{X}{$\cong$}
    Define $T:X \to X/\cong$ by setting, for each $x \in X$, 
    \begin{equation}
        T(x)=[x]
    \end{equation}    
    We call T the \QuotientMap of X under $\cong$. 
\end{df} 
\begin{prop}[\QuotientMap is  \Surjective]
\label{prop:QuotientMapSurjective}
\rm
    Let $X$ be a nonempty set.
    Let $\cong$ be an 
	\EquivalenceRelation on 
	$X$.
    Let $\QuotientMapInstance{T}{X}{\cong}$  be the 
	\QuotientMap of $X$ under the 
	\Relation
	$\cong$. 
    Then T is a 
	\Surjection. 
    \begin{proof}
       Let $K \in X/\cong$. 
       Then for some $x \in X$, $K=[x]$. 
       Then $T(x) = K$. 
       Since K was arbitrary, we are done. 
    \end{proof}
\end{prop} 


\subsection{Nets}
\label{def:Net}
\newcommand{\Net}[0]{
    \textbf{\hyperref[def:Net]{Net}}
}
\newcommand{\Nets}[0]{
    \textbf{\hyperref[def:Net]{Nets}}
}
\begin{df}[\Net]
    A \Net is a \Function 
    mapping from a directed set $(A, \leq)$
    into another set $X$. 
    If $f:A \to X$ is a \Net
    such that for $\alpha \in A$ we have
    $f(\alpha) = x_{\alpha}$, then we may 
    use the notation
    $\{x_\alpha\}_{\alpha \in A} \subset X$. 
\end{df}

\newcommand{\NetSection}[0]{\textbf{\hyperref[def:NetSection]{Section}}\xspace}
\newcommand{\NetSections}[0]{\textbf{\hyperref[def:NetSection]{Sections}}\xspace}

\begin{df}[\NetSection of a \Net]
\label{def:NetSection}
\rm
    Let $X$ be a nonempty set.
Let $(A,\leq)$ be a \DirectedSet
and let $\sigma = \{x_\alpha\}_{\alpha \in A}$ be a 
\Net in $X$.
Let $\gamma \in A$. 
Let $S(\gamma, \leq)$ be the 
\DirectedSection of $\leq$ 
corresponding to $\gamma$. 
We define 
$\braces{x_\alpha : \alpha \in S(\gamma, \leq) }$
to be the \NetSection of $x_\gamma$ in $\sigma$. 
\end{df}

\begin{prop}[Net Section]
    \label{def:NetSection}
	\rm
        Let $X$ be a nonempty set.
    and let $\{x_\alpha\}_{\alpha \in A}$ 
    be a \Net in $X$. 
    Let $\beta \leq \gamma \in A$. 
    For $\alpha \in A$, let 
    $S(\alpha)$ denote the \NetSection
    of $x_\alpha$ in $\{x_\alpha\}_{\alpha \in A}$. 
    Then $S(\gamma) \subset S(\beta)$. 
    \begin{proof}
        Let $y \in S(\gamma)$.
        Then $y=x_\tau$ for some 
        $\tau \in A$ 
        satisfying 
        $\beta \leq \gamma \leq \tau$. 
        Hence, $y=x_\tau \in S(\beta)$. 
    \end{proof}
\end{prop}

\newcommand{\InductivelyOrdered}[0]{\textbf{\hyperref[def:InductiveOrder]{Inductively Ordered}}\xspace}
\newcommand{\InductiveOrder}[0]{\textbf{\hyperref[def:InductiveOrder]{Inductive Order}}\xspace}
\newcommand{\InductiveOrders}[0]{\textbf{\hyperref[def:InductiveOrder]{Inductive Orders}}\xspace}
\begin{df}[\InductiveOrder]
\label{def:InductiveOrder}
\rm
    Let $(X,\leq)$ be a \Poset.
    We say that $\leq$ is an 
    \InductiveOrder on $X$ 
    and we say that 
    $X$ is \InductivelyOrdered
    by $\leq$ if 
    each \Chain in $X$ has an \UpperBound in $X$. 
    we also call $(X,\leq)$ an 
    \InductivelyOrdered set in this circumstance.


\end{df}


\label{Axiom:ZornsLemma}
\begin{thm}[Zorns Lemma]
Let $(X,\leq)$. be a 
\Poset. If every 
\Chain in $X$ has an 
\UpperBound, then $\Maxima(X) \neq \emptyset$ 
\end{thm}
\begin{rmk}\ref{Axiom:ZornsLemma} Is equivalent to the axiom of choice
\end{rmk}




\section{Filters}
\subsection{Filter Basics}
\label{def:Filter}
\newcommand{\Filter}[0]{
    \textbf{\hyperref[def:Filter]{Filter}}
}
\newcommand{\Filters}[0]{
    \textbf{\hyperref[def:Filter]{Filters}}
}
\begin{df}[\Filter]
    Let $X \neq \emptyset$. 
    Let $\emptyset \neq \scF \subset \scPowerSet{X}$ 
    satisfy the following.
    \begin{enumerate}
        \item $\emptyset \not \in \scF$
        \item If $G_1 \in \scF$ and $G_1 \subset G_2 \subset X$, then $G_2 \in \scF$. 
        \item If $\{G_1,G_2\} \subset \scF$, then $G_1 \cap G_2 \in \scF$. 
    \end{enumerate}
    Then we call $\scF$ 
    a \Filter
    on $X$. 
\end{df}

\begin{prop}
\label{prop:FilterFacts}
    Let $X \neq \emptyset$ and let $\scF$ be a 
    \Filter on $X$. The following are true. 
    \begin{enumerate}[label=(\roman*), ref={\ref{prop:FilterFacts}~\roman*}]
        \item \label{prop:FilterFacts:ContainsX} $X \in \scF$. 
        \item \label{prop:FilterFacts:ClosureUnderFiniteIntersections} $\scF$ is \ClosedUnderFiniteIntersections
    \end{enumerate}
    \begin{proof}[Proof of \ref{prop:FilterFacts:ContainsX}]
        By \ref{def:Filter:IsNonempty}, 
        $\exists B \neq \emptyset \in \scF$. 
        Since $B \subset X \subset X$, by 
        \ref{def:Filter:SubsetProperty}, 
        $X \in \scF$, so \ref{prop:FilterFacts:ContainsX} is proven. 
    \end{proof}
    \begin{proof}[Proof of \ref{prop:FilterFacts:ClosureUnderFiniteIntersections}]
        Direct application of \ref{def:Filter:FiniteIntersectionProperty} paired with 
        \ref{prop:FiniteClosure:Union}.
    \end{proof}
\end{prop}


\newcommand{\CoarserFilter}[0]{\textbf{\hyperref[def:CoarseFineFilter]{Coarser}}\xspace}
\newcommand{\CoarsestFilter}[0]{\textbf{\hyperref[def:CoarseFineFilter]{Coarsest}}\xspace}
\newcommand{\FinerFilter}[0]{\textbf{\hyperref[def:CoarseFineFilter]{Finer}}\xspace}
\newcommand{\FinestFilter}[0]{\textbf{\hyperref[def:CoarseFineFilter]{Finest}}\xspace}
\newcommand{\FilterFineness}[0]{\textbf{\hyperref[def:CoarseFineFilter]{Filter Fineness}}\xspace}
\newcommand{\FilterCoarseness}[0]{\textbf{\hyperref[def:CoarseFineFilter]{Filter Coarseness}}\xspace}
\newcommand{\UltraFilter}[0]{\textbf{\hyperref[def:CoarseFineFilter]{Ultrafilter}}\xspace}
\newcommand{\UltraFilters}[0]{\textbf{\hyperref[def:CoarseFineFilter]{Ultrafilters}}\xspace}


\begin{df}[\CoarserFilter, \FinerFilter]
\rm
\label{def:CoarseFineFilter}
    Let $X$ be a nonempty set.
    Let $\scF_1$
    and $\scF_2$ be \Filters
    on $X$ such that 
    $\scF_1 \subset \scF_2$. 
    Then we say that 
    $\scF_1$ is \CoarserFilter
    than $\scF_2$ and we say that 
    $\scF_2$ is \FinerFilter than 
    $\scF_1$. 
    Let $A$ be a collection of \Filters on $X$. 
    Let $\scF_2 \in A$ be \FinerFilter 
    than every element of $A$. 
    Then we say that $\scF_2$ is the \FinestFilter
    element of $A$. 
    Let $\scF_3 \in A$ be \CoarserFilter
    than every element of $A$. 
    Then we say that $\scF_3$ is the \CoarsestFilter element
    in $A$. 
    \FilterFineness
    defines a \PartialOrdering on
    the collection of \Filters on $X$, where 
    $\scF_1 \leq \scF_2$ if $\scF_2$ is a \FinerFilter \Filter than $\scF_1$. 
    A \Maximum
    of the \FilterFineness relation is called an 
    \UltraFilter on $X$. 
\end{df}

\input{./Math/Props/ch02/Filters/FilterExistence.tex}
\begin{prop}
    \label{prop:FilterOrderFacts}
    Let $X \neq \emptyset$. The following are true. 
    \begin{enumerate}[label=(\roman*), ref={\ref{prop:FilterOrderFacts}.~\roman*}]
    \item \label{prop:FilterOrderFacts:ContainingFilter} Let $\scF$ be a \Filter on $X$ and let $A \subset X$. Then there is a \Filter containing $A$ on $X$ which is $\FinerFilter$ than $\scF$ if and only if $A \cap U \neq \emptyset$ for each $U \in \scF$. 
    \item \label{prop:FilterOrderFacts:FinestFilterExistence}
    Let $\sc{K}=\{\scF_\alpha\}_{\alpha \in A}$ be a collection of $\Filters$ on $X$. 
    There exists a \Filter on $X$ which is \FinerFilter than each $\scF_\alpha$ if and only if, for every \Finite subset $\{\scF_{\alpha_i}\}_{i=1}^n \subset \scK$, 
    for each $\{U_i\}_{i=1}^n \in \prod\limits_{i=1}^n \scF_{\alpha_i}$
    , $\bigcap\limits_{i=1}^n U_i \neq \emptyset$.
    \item \label{prop:FilterOrderFacts:FiltersAreInductive}

    \end{enumerate}
\end{prop}

\subsection{Filter Base}
\newcommand{\FilterGeneratedBy}[0]{
    \textbf{\hyperref[def:GeneratedFilter]{Generated By}}
}
\newcommand{\FilterGenerate}[0]{
    \textbf{\hyperref[def:GeneratedFilter]{Generate}}
}
\newcommand{\FilterGenerates}[0]{
    \textbf{\hyperref[def:GeneratedFilter]{Generates}}
}
\newcommand{\FilterSubbasis}[0]{
    \textbf{\hyperref[def:GeneratedFilter]{Subbasis}}
}
\newcommand{\FilterSubBases}[0]{
    \textbf{\hyperref[def:GeneratedFilter]{Subbases}}
}

\begin{df}[\FilterSubbasis]
\label{def:GeneratedFilter}
    Let $X$ be a set and $A \subset X$ such that 
    \begin{enumerate}[label=(\roman*), ref={\ref{def:GeneratedFilter}.~\roman*}]
       \item \label{def:FilterSubbase:IsNotEmpty} $\emptyset \not \in A$.
       \item \label{def:FilterSubbase:FiniteIntersectionsNonempty} 
       \begin{equation}
        \emptyset \not \in K:=\left\{ \bigcap\limits_{i=1}^n A_i | \{A_i\}_{i=1}^n \subset A \wedge n \in \bbN\right\}
       \end{equation}
    \end{enumerate}
    Define 
    \begin{equation*}
        \scK=\left\{ U \cup P | U \in K \wedge P \subset X \right\}
    \end{equation*}
    We call $\scK$ the \Filter
    on $X$
    \FilterGeneratedBy
    $A$. 
    and we call $A$ a 
    \FilterSubbasis
    for $G$. 
    By \ref{prop:FilterExistence}, 
    $\scK$ is in fact a \Filter, and is the 
    \CoarsestFilter \Filter on $X$ containing $A$. 
\end{df}

\newcommand{\FilterBase}[0]{\textbf{\hyperref[def:FilterBase]{Filter Base}}\xspace}
\newcommand{\FilterBases}[0]{\textbf{\hyperref[def:FilterBase]{Filter Bases}}\xspace}
\newcommand{\FilterBaseEquivalent}[0]{\textbf{\hyperref[def:FilterBase]{Equivalent}}\xspace}

\begin{df}[\FilterBase]
\label{def:FilterBase}\rm
    Let $X$ be a nonempty set.
    Let $\scB \subset \scPowerSet{X}$
    such that 
    \begin{enumerate}[label=(\roman*), ref={\ref{def:FilterBase}.~\roman*}]
        \item \label{def:FilterBase:IsNotEmpty}$
        \emptyset \neq \scB$. 
        \item \label{def:FilterBase:DoesntContainEmptySet}$
        \emptyset \not \in \scB$. 
        \item \label{def:FilterBase:IntersectionProperty}
        If $\scB_{Int}$ is the collection of binary intersections of elements of $\scB$, then
		Then \scNested{\scB}{\scB_{Int}} holds. 
    \end{enumerate}
    Then we call 
    $\scB$ 
    a
    \FilterBase
    on $X$. 
    By \ref{prop:FilterBase}, the 
    \Filter
    \FilterGeneratedBy
    a 
    \FilterBase
    $A$ 
    is given by 
    $\{U \subset X : (\exists Y \in A ) ( Y \subset U ) \}$. 
    If $A,B$ are \FilterBases
    on $X$ and they 
    \FilterGenerate
    the same 
    \Filter, 
    then we call them 
    \FilterBaseEquivalent. 
\end{df}


\begin{prop}
    \label{prop:FilterBase}\rm
    Let $X$ be a nonempty set.
    Let
    $A \subset \scPowerSet{X}$. 
    and define 
    \begin{equation*}
    \scU=\{U \subset X : (\exists a \in A ) ( a \subset U ) \}
    \end{equation*}
    The following are equivalent. 
    \begin{enumerate}
    \item $A$ is a \FilterBase on $X$. 
    \item $\scU$ is a \Filter on $X$. 
    \end{enumerate}
    \begin{proof}[1 $\implies$ 2]
    Supose $A$ is a \FilterBase on $X$.
    By $\ref{def:FilterBase:DoesntContainEmptySet}$, $\emptyset \not \in A$, so 
    $\emptyset \not \in \scU$, implying that $\scU$ satisfies
    \ref{def:Filter:DoesntContainEmpty}. 
    Also, by \ref{def:FilterBase:IsNotEmpty} $\emptyset \neq A \subset \scU$, so $\scU$ satisfies
    \ref{def:Filter:IsNonempty}. That $\scU$ satisfies \ref{def:Filter:SubsetProperty} is obvious. 
    Finally, if $G_1,G_2 \in \scU$, then there exists $U_1,U_2 \in A$ such that $A_i \subset G_i$. 
    By \ref{def:FilterBase:IntersectionProperty}, there is $B \in A$ satisfying 
    $B \subset U_1 \cap U_2 \subset G_1 \cap G_2$, so $G_1 \cap G_2 \in \scU$, implying $\scU$ is a \Filter on $X$. 
    \end{proof}
    \begin{proof}[1 $\impliedby$ 2]
        If $A = \emptyset$, then $\scU=\emptyset$, so $A$ failing $\ref{def:FilterBase:IsNotEmpty}$ implies $\scU$ fails \ref{def:Filter:IsNonempty}. 
        If $\emptyset \in A$, then $\emptyset \in \scU$, so $A$ failing $\ref{def:FilterBase:DoesntContainEmptySet}$ implies $\scU$ fails \ref{def:Filter:DoesntContainEmpty}. 
        Finally, if $\scA$ fails \ref{def:FilterBase:IntersectionProperty}, then 
        we can find $B,C \in A$ such that $B \cap C \not \in \scU$, implying $\scU$ fails $\ref{def:Filter:FiniteIntersectionProperty}$. 
        Hence necessity has been proven. 
    \end{proof}
\end{prop}
\begin{rmk}
\label{rmk:FilterBase}\rm
    If $A$ is a \FilterBase on $X$, then \scU defined in $\ref{prop:FilterBase}$ 
    is the \Filter \FilterGeneratedBy $A$. 
\end{rmk}

\begin{prop}[FilterBaseFacts]
\label{prop:FilterBaseFacts}
   Let $X \neq \emptyset$. 
   Let $\scF$ and $\scG$ be \Filters on $X$. 
   Let $F$ be a \FilterBase for $\scF$ and let 
   $G$ be a \FilterBase for $\scG$. 
   The following are true. 
   \begin{enumerate}[label=(\roman*), ref={\ref{prop:FilterBaseFacts}.~\roman*}]
    \item \label{prop:FilterBaseFacts:BaseFromSubbasis}
        The collection of \Finite intersections of a 
        \FilterSubbasis $A$ for $\scF$ 
        forms a \FilterBase for $\scF$. 
    \item \label{prop:FilterBaseFacts:BaseCondition}
    $B \subset \scF$ is a \FilterBase for $\scF$ 
    if and only if \scNested{B}{\scF} holds. 
    \item \label{prop:FilterBaseFacts:FinerCondition}
       $\scF$ 
       is \FinerFilter than $\scG$ if and only if \scNested{F}{G} holds. 
    \item  \label{prop:FilterBaseFact:EquivalenceCondition} 
    $F$ is 
    \FilterBaseEquivalent to 
    $G$ if and only if \scNested{F}{G} and \scNested{G}{F} both hold.
    \item \label{prop:FilterBaseFacts:FiltersAreFilterBases}
    $\scF$ is a \FilterBase for $\scF$. 
   \end{enumerate}
   \begin{proof}[Proof of \ref{prop:FilterBaseFacts:BaseFromSubbasis}]
   Let 
   \begin{equation}
    \scB = \left\{ \bigcap\limits_{i=1}^n A_i | \{A_i\}_{i=1}^n \subset A \wedge n \in \bbN\right\}
   \end{equation}
   By \ref{def:FilterSubbase:FiniteIntersectionsNonempty}, $\emptyset \not \in \scB$, so $\scB$ satisfies \ref{def:FilterBase:DoesntContainEmptySet}. 
   By \ref{def:FilterSubbase:IsNotEmpty}, $\emptyset \neq A \subset \scB$, so $\scB$ satisfies \ref{def:FilterBase:IsNotEmpty}. 
   Since $\emptyset \not \in \scB$ is \ClosedUnderFiniteIntersections, if $U,V \in \scB$, then $\emptyset \neq U \cap V \in \scB$, so $\scB$ can be seen to satisfy $\ref{def:FilterBase:IntersectionProperty}$. Hence $\scB$ is a \FilterBase. 
   Since $A \subset \scB$, $\scB$ is a \FilterBase for a \FinerFilter than $\scF$. 
   However, since $A \subset \scF$, 
   by \ref{prop:FilterFacts:ClosureUnderFiniteIntersections}, $\scB \subset \scF$. 
   Hence $\scF$ is the \Filter
   \FilterGeneratedBy $\scB$. 
   \end{proof}
   \begin{proof}[Proof of \ref{prop:FilterBaseFacts:BaseCondition}]
    $(\impliedby)$. Let $\scG$ denote the \Filter \FilterGeneratedBy $B$. 
    Then since $B \subset \scF$, $\scG \subset \scF$. 
    If for each $Y \in \scF$ there exists $b \in B$ with $b \subset Y$, then 
    \begin{equation}
    \scF \subset \{U \subset X | (\exists b \in B) (b \subset U) \} \subset \scG \subset \scF
    \end{equation}
    so that $\scF = \scG$ and 
    $\{U \subset X | (\exists b \in B) (b \subset U)\}= \scF$ is a \Filter on $X$. 
    Hence, by  \ref{prop:FilterBase}, $B$ is a \Filter. 

    $(\implies)$. If $B$ is a \FilterBase for $\scF$, then by 
    \ref{prop:FilterBase}, $\scF= \{Y \subset X | (\exists b \in B) ( b \subset Y) \}$
    so the desired property holds
   \end{proof}
   \begin{proof}[Proof of \ref{prop:FilterBaseFacts:FinerCondition}]
   Let $\scF$ be finer than $\scG$. Then 
   by applying \ref{prop:FilterBase}
   $\scG \subset \scF = \{ U \subset X | (\exists f \in F) (f \subset U) \}$, which is the desired result in one direction. 
   The other direction is equivalent again applying \ref{prop:FilterBase}
   to claim
   $\scG \subset \{U \subset X | (\exists f \in F) ( f \subset U ) \}$.
   \end{proof}
   \begin{proof}[Proof of \ref{prop:FilterBaseFact:EquivalenceCondition}]
   This is a result of two applications of \ref{prop:FilterBaseFacts:FinerCondition}, 
   one in each direction. 
   \end{proof}
   \begin{proof}[Proof of \ref{prop:FilterBaseFacts:FiltersAreFilterBases}]
    Define $\scU = \{U \subset X | (\exists f \in \scF) \pitchfork(f \subset U )\}$. 
    By construction $\scF \subset \scU$. 
    By \ref{def:Filter:SubsetProperty}, $\scU \subset \scF$.
    Hence, by \ref{prop:FilterBase}, 
    $\scU = \scF$ is a \FilterBase on $X$. 
    Clearly $\scF \subset \scF$ and \scNested{\scF}{\scF} hold, so we can apply
    \ref{prop:FilterBaseFacts:BaseCondition} so see that $\scF$ is a \FilterBase
    for $\scF$. 
   \end{proof}
\end{prop}

\begin{prop}[\Net \NetSections form a \FilterBase]
    \label{prop:NetSectionsFormFilterBase}
    Let $X \neq \emptyset$
    and let $\sigma=\{x_\alpha\}_{\alpha \in A}$
    be a \Net in $X$. 
    For each 
    $\alpha \in A$, denote  with
    $S(\sigma, \alpha ) $ 
    the \NetSection of $x_\alpha$ in $\sigma$. 
    Define 
    \begin{equation*}
        \scB=\{S(\sigma, \alpha ) | \alpha \in A\}
    \end{equation*}
    Then $\scB$ is a \FilterBase on $X$. 
    \begin{proof}
        Since $\sigma$ is a \Net, 
        $(A,\leq)$ is a \DirectedSet, 
        implying that $(A,\leq)$ is a \PreorderedSet.
        Hence, $\leq$ is \ReflexiveRelation so that
        if $\alpha \in A$, then $x_\alpha \in S(\sigma, \alpha)$. 
        Hence, $\emptyset \not \in \scB$, so 
        $\scB$ satisfies 
        \ref{def:FilterBase:DoesntContainEmptySet}.
        Furthermore, since 
        $(A,\leq)$ is a \PreorderedSet, 
        $A$ is nonempty, so 
        $\emptyset \neq \scB$, implying $\scB$ satisfeis 
        \ref{def:FilterBase:IsNotEmpty}. 
        Finally, let $U,V \in \scB$. 
        Then we can find $u,v \in A$ such that 
        $U = S(\sigma, U)$, $V=S(\sigma, V)$. 
        Since $A$ is a \DirectedSet, 
        there exists $w \in A$ with $u \leq w$ and $v \leq w$. 
        Hence by \ref{def:NetSection}, 
        $S(\sigma, w ) \subset S(\sigma, u) \cap S(\sigma, v)$. 
        Since $S(\sigma, w ) \in \scB$, 
        $\scB$ satisfies 
        \ref{def:FilterBase:IntersectionProperty}, 
        and we're done. 
    \end{proof}
\end{prop}

\newcommand{\NetSectionFilter}[0]{\textbf{\hyperref[def:SectionFilter]{Section Filter}}\xspace}
\newcommand{\NetSectionFilters}[0]{\textbf{\hyperref[def:SectionFilter]{Section Filters}}\xspace}
\newcommand{\DirectedSectionFilter}[0]{\textbf{\hyperref[def:SectionFilter]{Section Filter}}\xspace}
\newcommand{\DirectedSectionFilters}[0]{\textbf{\hyperref[def:SectionFilter]{Section Filters}}\xspace}
\newcommand{\scSectionFilter}[1]{
    \ensuremath{\scF_{#1}}
}
\begin{df}[\NetSectionFilter]
    \label{def:SectionFilter}
    \rm
    Let $X \neq \emptyset$. 
    Let $\sigma=\{x_{\alpha}\}_{\alpha \in A}$ be a 
    \Net in $X$. 
    For each $\alpha \in A$, let $S(\sigma, \alpha)$ 
    denote the \NetSection of $x_{\alpha}$ in $\sigma$. 
    Define $\scB=\{S(\sigma, \alpha) | \alpha \in A\}$. 
    By \ref{prop:NetSectionFormFilterBase}, $\scB$ is a 
    \FilterBase on $X$. 
    We call the \FilterGeneratedBy $\scB$ the 
    \NetSectionFilter of $\sigma$. 
    We call 
    the \NetSectionFilter of the 
    identity \Net in $A$ the \DirectedSectionFilter of $A$. 
    We denote the \DirectedSectionFilter of $A$ with \scSectionFilter{A}.
\end{df}

\subsection{Ultrafilters}
\label{def:Ultrafilter}
\newcommand{\Ultrafilter}[0]{
    \textbf{\hyperref[def:Ultrafilter]{Ultrafilter}}
}
\newcommand{\Ultrafilters}[0]{
    \textbf{\hyperref[def:Ultrafilter]{Ultrafilters}}
}
\begin{df}[\Ultrafilter]
    Let $X \neq \emptyset$. 
    An \Ultrafilter 
    on $X$ is a \Maximum
    of the relation of \FilterFineness
    on $X$. 
\end{df}
\begin{rmk}[\Ultrafilter Existence]
    Let $\scF$ be a \Filter on $X \neq \emptyset$. 
    By 
    \ref{prop:FilterOrderFacts:UnionOfChainOfFiltersIsAFilter}
    Not only is \FilterFineness an \InductiveOrder 
    on the set of \Filters of $X$, 
    (as stated in \ref{prop:FilterOrderFacts:FilterFinenessIsInductive})
    , but \FilterFineness is also an \InductiveOrder
    on the set of \Filters \FinerFilter than $\scF$. 
    Hence by \ref{Axiom:ZornsLemma}, 
    $\scF$ is contained in an \UltraFilter on $X$. 
\end{rmk}

\begin{prop}[Ultrafilter Facts]
\label{prop:UltrafilterFacts}
    Suppose the following
    \begin{enumerate}[label=(\Roman*), ref={\ref{prop:UltrafilterFacts}.~\Roman*}]
        \item \label{prop:UltrafilterFacts:Ass1} $X \neq \emptyset$. 
        \item \label{prop:UltrafilterFacts:Ass2} $\scF$ is an \Ultrafilter on $X$. 
        \item \label{prop:UltrafilterFacts:Ass3} $\scG$ is a \Filter on $X$. 
        \item \label{prop:UltrafilterFacts:Ass4} $\scK$ is a \FilterSubbasis on $X$. 
        \item \label{prop:UltrafilterFacts:Ass5} 
        $\scM=\scK \cup \{K \subset X | X \setminus K \in \scK\}$. 
    \end{enumerate}

    Then the following are true
    \begin{enumerate}
        \item \label{prop:UltrafilterFacts:BinaryUnion} 
        If $\{A,B\} \subset \scPowerSet{X}$ and 
        $A \cup B \in \scF$, 
        then  $A \in \scF$ or $B \in \scF$. 
        \item \label{prop:UltrafilterFacts:FiniteUnion}
        If $\{A_i\}_{i=1}^n \subset \scPowerSet{X}$ such that 
        $\bigcup\limits_{i=1}^n A_i \in \scF$, 
        then for some $j \in \{1, \cdots, n\}$, 
        $A_j \in \scF$. 
        \item \label{prop:UltraFilterFacts:UltrafilterCondition}
        If $\scM = \scPowerSet{X}$, then $\scK$ is an \UltraFilter on $X$. 
        \item \label{prop:UltraFilterFacts:UltrafilterIntersection} 
        $\scG$ is the intersection of all $\Ultrafilters$ on $X$ which contain $\scG$.
    \end{enumerate}


    \begin{proof}[Proof of \ref{prop:UltrafilterFacts:BinaryUnion}]
        We use contradiction.
        Suppose $A \not \in \scF$ and $B \not \in \scF$. 
        Define $\scT=\{G \in \scPowerSet{X} | A \cup G \in \scF\}$. 
        Then $B \in \scT$, so $\scT$ satisfies 
        \ref{def:Filter:IsNonempty}.
        Furthermore, if $G_1 \in \scT$ and $G_1 \subset G_2 \subset X$, then 
        by \ref{def:Filter:SubsetProperty},  
        $A \cup G_1 \subset A \cup G_2 \in \scF$. Hence $G_2 \in \scT$ so 
        $\scT$ satisfies \ref{def:Filter:SubsetProperty}.
        Let $G_3,G_4 \in \scT$ 
        \begin{align*}
            A \cup \pa{G_3 \cap G_4} & = \pa{A \cup G_3} \cap \pa{A \cup G_4} \in \scF
        \end{align*}
        so that $G_3 \cap G_4 \in \scT$ and therefore $\scT$ satisfies
        \ref{def:Filter:FiniteIntersectionProperty}.
        Finally since $A \not \in \scF$, $\emptyset \not \in \scT$, so 
        $\scT$ satisfeis $\ref{def:Filter:DoesntContainEmpty}$, and therefore 
        $\scT$ is a \Filter on $X$. 
        Trivially, $\scF \subset \scT$ but since $B \in \scT \setminus \scF$, 
        this contradicts
        \ref{prop:UltrafilterFacts:Ass2}
        Hence the result holds. 
    \end{proof}
    \begin{proof}[Proof of \ref{prop:UltrafilterFacts:FiniteUnion}]
        We use induction on $n$. Obviously the result holds for $n=1$ 
        and by \ref{prop:UltrafilterFacts:BinaryUnion}, the result
        also holds for $n=2$. 
        Suppose the result holds for $n=k$ 
        Let $\{A_i\}_{i=1}^{k+1} \subset \scPowerSet{X}$ such that
        $\bigcup\limits_{i=1}^{k+1} A_i \in \scF$. 
        Then since the result holds for $n=2$, either 
        $A_{k+1} \in \scF$ or $\bigcup\limits_{i=1}^k A_i \in \scF$. 
        Since the result holds for $n=k$, either
        $A_{k+1} \in \scF$ or $A_i \in \scF$ for $i \in \{1, \cdots, k\}$. 
        Hence the result holds for $n=k+1$.
        Hence the result holds in general. 
    \end{proof}
    \begin{proof}[Proof of \ref{prop:UltraFilterFacts:UltrafilterCondition}]
        I first prove that $\scM=\scPowerSet{X}$, 
        paired with 
        \ref{prop:UltrafilterFacts:Ass4} and 
        \ref{prop:UltrafilterFacts:Ass5} 
        implies $\scK$ is a \Filter
        on $X$.
        By \ref{prop:UltrafilterFacts:Ass1}, $\scM \neq \emptyset$. 
        By \ref{prop:UltrafilterFacts:Ass5}, then $\scK \neq \emptyset$, so 
        $\scK$ satisfies \ref{def:Filter:IsNonempty}.
        By \ref{def:FilterSubbase:IsNotEmpty}, $\scK$ satisfies \ref{def:Filter:DoesntContainEmpty}.
        Let $G_1 \in \scK$ and let $G_1 \subset G_2 \subset X$. 
        Then $G_1 \cap \pa{X \setminus G_2} = \emptyset$, 
        which by 
        \ref{def:FilterSubbase:FiniteIntersectionsNonempty}
        implies $X \setminus G_2 \not \in \scK$. 
        Since $\scM = \scPowerSet{X}$, we conclude $G_2 \in \scK$, so
        $\scK$ satisfies \ref{def:Filter:SubsetProperty}. 
        Finally, let 
        $G_1,G_2 \in \scK$. 
        By assumption, either $G_1 \cap G_2 \in \scK$  or $X \setminus \pa{G_1 \cap G_2 } \in \scK$. 
        If $X \setminus \pa{G_1 \cap _G2} \in \scK$, then by 
        \ref{def:FilterSubbase:FiniteIntersectionsNonempty}, 
        $G_1 \cap G_2 \cap \pa{X \setminus \pa{G_1 \cap G_2}} \neq \emptyset$, a contradiction.
        Hence, $G_1 \cap G_2 \in \scK$, so that \ref{def:Filter:FiniteIntersectionProperty}
        is satisfied by $\scK$. 
        Hence $\scK$ is a \Filter on $X$. 
        By \ref{rmk:Ultrafilter}, there is an \Ultrafilter $\scL$ containing $\scK$. 
        If $\scK$ is not an \Ultrafilter, then $\exists B \in \scL \setminus \scK$. 
        Since $\scM = \scPowerSet{X}$, $X \setminus B \in \scK \subset \scL$, 
        implying $\emptyset =B \cap \pa{X \setminus B}  \in \scL$, contradicting 
        \ref{def:Filter:DoesntContainEmpty}, thus
        $\scK$ is an \Ultrafilter. 
    \end{proof}
    \begin{proof}[Proof of \ref{prop:UltraFilterFacts:UltrafilterIntersection}]
    Obviously 

    \end{proof}
\end{prop}

\subsection{Induced Filters}
\newcommand{\InducedFilter}[0]{\textbf{\hyperref[def:InducedFilter]{Induced}}\xspace}
\begin{df}[\InducedFilter]
\label{def:InducedFilter}
\rm
    Let $X$ be a nonempty set.
    Let $\scF$ be a \Filter on $X$.
    Let $A \subset X$. 
    Define
    \begin{equation*}
        \scF_A:= \{U \cap A : U \in \scF\}
    \end{equation*}
    Suppose $\emptyset \not \in \scF_A$.
    Then by \ref{prop:FilterFacts:InducedFilterExistence}
    $\scF_A$ is a \Filter on $A$ 
    which we call the \Filter
    \InducedFilter
    by $\scF$. 
\end{df}

\begin{prop}[\InducedFilter \Filter facts]
\label{prop:InducedFilter}
\rm
    Let $X$ be a nonempty set.
    Let $\scF$ be a \Filter on $X$. 
    Let $\scB$ be a \FilterBase for $\scF$. 
    Let $\scG$ be an \Ultrafilter on $X$. 
    Let $A \subset X$. 
    Define
    \begin{align*}
        \scF_A =\{A \cap U : U \in \scF\}\\
        \scB_A =\{A \cap U : U \in \scB\}\\
        \scG_A= \{A \cap U : U \in \scG\}
    \end{align*}
    The following are true
    \begin{enumerate}[label=(\roman*), ref={\ref{prop:InducedFilter}.~\roman*}]
    \item  \label{prop:InducedFilter:Base} If $\scF_A$ is a \Filter on $A$, then $\scB_A$ is a \FilterBase for $\scF_A$. 
    \item  \label{prop:InducedFilter:Ultrafilter} $\scG_A$ is a \Filter on $A$ if and only if $A \in \scG$. In this case, $\scG_A$ is an \Ultrafilter on $A$. 
    \end{enumerate}
    \begin{proof}[Proof of \ref{prop:InducedFilter:Base}]
       Let $U \in \scF_\alpha$. 
       Then there exists $V \in \scF$ such that $U = A \cap V$. 
       Since $\scB$ is a \FilterBase for $\scF$, 
       by \ref{prop:FilterBaseFacts:BaseCondition}, 
       there exists $B \in \scB$ satisfying $B \subset V$. 
       Then $B \cap A \subset A \cap V = U$. 
       But $B \cap A \in \scB_A$, so since $\scB_A \subset \scF_A$, 
       we can the apply $\ref{prop:FilterBaseFacts:BaseCondition}$ 
       to claim that $\scB_A$ is a \FilterBase for $\scF_A$. 
    \end{proof}
    \begin{proof}[Proof of \ref{prop:InducedFilter:Ultrafilter}]
        Even if $\scG$ was merely a \Filter on $X$, by
        \ref{def:Filter:DoesntContainEmpty}
        and
        \ref{prop:FilterFacts:ClosureUnderFiniteIntersections}
        $A \in \scG$ is sufficient to guarantee that $\scG_A$ is a \Filter
        on $A$. 
        Now suppose $\scG_A$ is a \Filter on $A$
        Then $A \cap U \neq \emptyset$ for $U \in \scG$.
        Since  $\scG$ is an \Ultrafilter on $X$, 
        we can apply maximality with 
        \ref{prop:FilterOrderFacts:ContainingFilter}, 
        to see that
        $A \in \scG$. 
        Finally, if $P \subset A$ satisfies 
        $P \not \in \scG_A$, then 
        $P \not \in \scG$,
        Since $\scG$ is an \Ultrafilter, 
        by \ref{prop:FilterOrderFacts:ContainingFilter}, 
        $P \cap U = \emptyset$ for some $U \in \scG$. 
        This implies $P \cap \pa{U \cap A} = \emptyset$, 
        and since $U \cap A \in \scG_A$, 
        we can apply \ref{prop:FilterOrderFacts:ContainingFilter}
        to conclude that there is no \Filter \FinerFilter 
        than $\scG_A$ on $A$ which contains $P$. 
        Since $P \subset A$ was arbitrary, 
        $\scG_A$ is an \Ultrafilter on $A$. 


    \end{proof}
\end{prop}

\subsection{Direct and Inverse Images of a Filter Base}
\begin{prop}[Direct Filter Image]
\label{prop:DirectFilterImage}
\rm
Suppose the following. 
\begin{enumerate}
    \item $X$ and $Y$ are nonempty sets.
    \item $f:X \to Y$ is \Surjective.
    \item For $i \in \{1,2\}$, $\scB_i$ is a \FilterBase for a \Filter $\scF_i$ on $X$. 
    \item $\scF_2$ is \FinerFilter than $\scF_1$. 
    \item $\scK$ is an \UltrafilterBase on $X$. 
\end{enumerate}
Then the following are true.
\begin{enumerate}[label=(\roman*), ref={\ref{prop:DirectFilterImage}.~\roman*}]
\item \label{prop:DirectFilterImage:Filter} $f(\scF_1)$ is a \Filter on $Y$. 
\item \label{prop:DirectFilterImage:Base} $f(\scB_1)$ is a \FilterBase for $f(\scF_1)$. 
\item \label{prop:DirectFilterImage:Order} $f(\scF_2)$ is a \FinerFilter than $f(\scF_1)$. 
\item \label{prop:DirectFilterImage:Ultrafilter} $f(\scK)$ is an \UltrafilterBase on $Y$. 
\end{enumerate}
\begin{proof}[Proof of \ref{prop:DirectFilterImage:Filter}]
    Since $\emptyset \not \in \scF_1$, 
    $\emptyset \not \in f(\scF_1)$, so 
    $f(\scF_1)$ satisfies \ref{def:Filter:DoesntContainEmpty}.
    Since $\emptyset \neq \scF_1$, $f(\scF_1) \neq \emptyset$, 
    so $f(\scF_1)$ satisfies $\ref{def:Filter:IsNonempty}$.
    Let $G_1 \in f(\scF_1)$. 
    The there exists $U \in \scF_1$ such that $f(U) = G_1$. 
    Let $G_1 \subset G_2 \subset Y$. 
    Then, 
    $U \subset f^{-1}(G_2)$, 
    which by 
    \ref{def:Filter:SubsetProperty} implies $f^{-1}(G_2) \in \scF_1$. 
    Then, since $f$ is \Surjective, 
    $G_2=f\pa{f^{-1}\pa{G_2}} \in f(\scF_1)$, so 
    $f(\scF_1)$ satisfies \ref{def:Filter:SubsetProperty}. 
    Finally, if $G_1,G_2 \in f(\scF_1)$, then 
    there are $K_1,K_2 \in \scF_1$ with $f(K_i) =G_i$ for $i \in \{1,2\}$. 
    By \ref{def:Filter:FiniteIntersectionProperty}, $K_1 \cap K_2 \in \scF_1$. 
    Also, $f(K_1 \cap K_2) \subset f(K_1) \cap f(K_2)$, 
    so by \ref{def:Filter:SubsetProperty}, $f(K_1) \cap f(K_2) \in f(\scF_1)$.
    Hence $f(\scF_1)$ satisfies $\ref{def:Filter:FiniteIntersectionProperty}$
    and is therefore a \Filter on $Y$. 
\end{proof}
\begin{proof}[Proof of \ref{prop:DirectFilterImage:Base}]
   By \ref{def:FilterBase:IsNotEmpty}, $\emptyset \neq \scB_1$, 
   so $\emptyset \neq f(\scB_1)$, and thus $f(\scB_1)$ satisfies \ref{def:FilterBase:IsNotEmpty}. 
   By \ref{def:FilterBase:DoesntContainEmptySet}, $\emptyset \not \in \scB_1$, so
   $\emptyset \not \in f(\scB_1)$, implying $\scB_1$ satisfies \ref{def:FilterBase:DoesntContainEmptySet}. 
   Finally, let $U_1,U_2 \in f(\scB_1)$.
   Then there exists $V_i \in \scB_1$ with $f(V_i) = U_i$. 
   Then, since $\scB_1$ satisfies
   \ref{def:FilterBase:IntersectionProperty}, 
   there exists $V \in \scB_1$ such that $V \subset V_1 \cap V_2$, so 
   $f(V) \subset f(V_1) \cap f(V_2)$. 
   Also, $f(V) \in f(\scB_1)$, so $f(\scB_1)$ satisfies 
   \ref{def:FilterBase:IntersectionProperty}.
   Hence, $f(\scB_1)$ is a \FilterBase on $Y$. 
   Now, if $V \in f(\scF_1)$, then by definition, there exists 
   $U \in \scF_1$ with $f(U) = V$. 
   By \ref{prop:FilterBaseFacts:BaseCondition}, there 
   exists a $b \in \scB_1$ with $b \subset U$. This implies $f(b) \subset f(U)=V$, but 
   $f(b) \in f(\scB_1)$.
   Furthermore, since $\scB_1 \subset \scF_1$, 
   $f(\scB_1)\subset f(\scF_1)$, so 
   we can apply \ref{prop:FilterBaseFacts:BaseCondition} to claim 
   that $f(\scB_1)$ is a \FilterBase for $f(\scF_1)$. 
\end{proof}
\begin{proof}[Proof of \ref{prop:DirectFilterImage:Order}]
    If $\scF_1 \subset \scF_2$ then $f(\scF_1) \subset f(\scF_2)$. 
    An invocation of \ref{prop:DirectFilterImage:Base} finishes the result. 
\end{proof}
\begin{proof}[Proof of \ref{prop:DirectFilterImage:Ultrafilter}]
   Let $\scG$ denote the \Ultrafilter for which 
   $\scK$ is an \UltrafilterBase.
   Let $U \subset Y$.
   Since $f$ is \Surjective, 
   \begin{equation}
   X= f^{-1}(U) \cup \pa{X \setminus f^{-1}(U)} = f^{-1}(U) \cup f^{-1} \pa{f(X) \setminus U}= f^{-1}(U) \cup f^{-1}(Y \setminus U)
   \end{equation}
   and by \ref{prop:FilterFacts:ContainsX}, we have 
    $f^{-1}(U) \cup f^{-1}(Y \setminus U) \in \scG$.
    Since \scG is an \Ultrafilter, by 
    \ref{prop:UltrafilterFacts:BinaryUnion}, 
    either $f^{-1}(U) \in \scG$ or $f^{-1}(Y \setminus U) \in \scG$. 
    This implies either 
    $U \in f( \scG)$ or $Y \setminus U \in f(\scG)$. 
    Since $U$ is arbitrary, by
    \ref{prop:UltraFilterFacts:UltrafilterCondition}, 
    $f\pa{\scG}$ is an \UltraFilter on $X$. 
    Finally, by 
    \ref{prop:DirectFilterImage:Base}, 
    $f\pa{\scK}$ is a \FilterBase for 
    $f(\scG)$, so the result holds. 
\end{proof}
\end{prop}

\begin{prop}[Inverse Filter Image]
\label{prop:InverseFilterImage}
Suppose the following
\begin{enumerate}
\item $X,Y \neq \emptyset$
\item $f:X \to Y$. 
\item $\scB$ is a \FilterBase for a \Filter $\scF$ on $Y$. 
\item $\scF_{f(X)}:=\{U \cap f(X) | U \in \scF\}$
\end{enumerate}
Then the following are true
\begin{enumerate}[label=(\roman*), ref={\ref{prop:InverseFilterImage}.~\roman*}]
    \item \label{prop:InverseFilterImage:Base}
    $f^{-1}\pa{\scB}$ is a \FilterBase on $X$ if and only if 
    $\emptyset \not \in f^{-1}\pa{\scB}$. 
    \item \label{prop:InverseFilterImage:Finer} If $f^{-1}\pa{\scB}$ is a \FilterBase on $X$, 
    then $f\pa{f^{-1}\pa{\scB}}$ is a \FilterBase for a \Filter on $Y$ finer than $\scF$. 
    \item \label{prop:InverseFilterImage:Subspace} If $f\pa{f^{-1}\pa{\scB}}$ is a \FilterBase for a \Filter
    $\scG$ on $Y$, then $\scG_{f(X)}=\scF_{f(X)}$. 
\end{enumerate}
\begin{proof}[Proof of \ref{prop:InverseFilterImage:Base}]
    Necessity of $\emptyset \not \in f^{-1}\pa{\scB}$ is obvious by 
    \ref{def:FilterBase:DoesntContainEmptySet}.
    For sufficiency, suppose $\emptyset \not \i f^{-1}\pa{\scB}$. 
    Then $f^{-1}\pa{\scB}$ satisfies \ref{def:FilterBase:DoesntContainEmptySet} trivially.
    Furthermore, by $\ref{def:FilterBase:IsNotEmpty}$, 
    $\scB \neq \emptyset$, so $f^{-1}\pa{\scB} \neq \emptyset$, so 
    $f^{-1}\pa{\scB}$ satisfies \ref{def:FilterBase:IsNotEmpty}. 
    Finally, let $U_1,U_2 \in f^{-1}\pa{\scB}$. 
    Then there exist $V_1,V_2 \in \scB$ such that 
    $U_i = f^{-1}(V_i)$. 
    By \ref{def:FilterBase:IntersectionProperty}, there exists $W \in \scB$ such that
    $W \subset V_1 \cap V_2$, and $f^{-1}(W) \in f^{-1}\pa{\scB}$. 
    Clearly, 
    \begin{equation*}
        f^{-1}(W) \subset f^{-1}\pa{V_1 \cap V_2} = f^{-1}(V_1) \cap f^{-1}(V_2) = U_1 \cap U_2
    \end{equation*}
    Hence $f^{-1}(\scB)$ satisfies \ref{def:FilterBase:IntersectionProperty}
\end{proof}
\begin{proof}[Proof of \ref{prop:InverseFilterImage:Finer}]
    If $f^{-1}\pa{\scB}$ is a \FilterBase on $X$, 
    then we can leverage \ref{prop:DirectFilterImage:Base} to 
    claim that $f\pa{f^{-1}\pa{\scB}}$ is a \FilterBase for a \Filter on $f(X)$, 
    and therefore also a \FilterBase for a \Filter on $Y$. 
    In particular, $f\pa{f^{-1}\pa{\scB}}$ is a \FilterBase. 
    Furthermore, if $b \in \scB$, then $f(f^{-1}(b)) \in f(f^{-1}\pa{\scB})$ and 
    $f(f^{-1}(b))= f(X) \cap b \subset b$, so by \ref{prop:FilterBaseFacts:FinerCondition}, 
    $f(f^{-1}\pa{\scB})$ is a \FilterBase for a \FinerFilter \Filter than $\scF$. 
\end{proof}
\begin{proof}[Proof of \ref{prop:InverseFilterImage:Subspace}]
   This is because given the assumptions, 
   $f\pa{f^{-1}\pa{\scB}} = \{f(X) \cap \scB | x \in \scB\}$, 
   which lets us apply \ref{prop:FilterBaseFacts:EquivalenceCondition}
\end{proof}
\end{prop}

\subsection{Filter Products}
\newcommand{\ProductFilter}[0]{\textbf{\hyperref[def:ProductFilter]{Product Filter}}\xspace}
\newcommand{\ProductFilters}[0]{\textbf{\hyperref[def:ProductFilter]{Product Filters}}\xspace}
\newcommand{\ProductFilterBase}[0]{\textbf{\hyperref[def:ProductFilter]{Product Filter Base}}\xspace}
\newcommand{\ProductFilterBases}[0]{\textbf{\hyperref[def:ProductFilter]{Product Filter Base}}\xspace}
\begin{df}[\ProductFilter]
\label{def:ProductFilter}
\rm
Suppose the following
\begin{enumerate}
    \item $A$ is a nonempty set.
    \item $\{X_\alpha\}_{\alpha \in A}$ is a collection of nonemptyset sets. 
    \item For each $\alpha \in A$, $\scF_\alpha$ is a \Filter on $X_\alpha$. 
    \item For each $\gamma \in A$, $\pi_\gamma : \prod\limits_{\alpha \in A} X_\alpha \to  X_\gamma$ represents the \ProjectionMap. 
\end{enumerate}
Then we define the \Filter on $\prod\limits_{\alpha \in A} X_\alpha$ 
\FilterGeneratedBy 
\begin{equation*}
\bigcup\limits_{\alpha \in A} \pi_\alpha^{-1}(\scF_\alpha)
\end{equation*}
to be the \ProductFilter on $\prod\limits_{\alpha \in A}(X_\alpha, \scF_\alpha)$.

\end{df}

\begin{prop}
    \label{prop:ProductFilterFacts}
    Suppose the following
\begin{enumerate}
    \item $A \neq \emptyset$.
    \item $\{X_\alpha\}_{\alpha \in A}$ is a collection of nonemptyset sets. 
    \item For each $\alpha \in A$
    , $\scB_\alpha$ is a \FilterBase for a \Filter $\scF_\alpha$ on $X_\alpha$.
    \item For each $\alpha \in A$, $\scG_\alpha$ is a \FilterSubbasis
    which \FilterGenerates $\scF_\alpha$. 
    \item For each $\gamma \in A$, $\pi_\gamma : \prod\limits_{\alpha \in A} X_\alpha \to  X_\gamma$ represents the \ProjectionMap. 
    \item $\scG:=\bigcup\limits_{\alpha \in A} \pi_\alpha^{-1}(\scG_\alpha)$. 
    \item $\scB$ is the collection of finite intersections of elements of 
        $\bigcup\limits_{\alpha \in A} \pi_{\alpha}^{-1}(\scB_\alpha)$
    \item $\scF$ is the \ProductFilter. 
\end{enumerate}
Then the following are true 
\begin{enumerate}
\item  $\scF$ is in fact a filter on $\prod\limits_{\alpha\in A} X_\alpha$. 
\item $\scG$ is a \FilterSubbasis for $\scF$. 
\item $\scB$ is a \FilterBase for $\scF$. 
\item $\scF$ is the \CoarsestFilter \Filter on $\prod\limits_{\alpha \in A} X_\alpha$ such that 
for every $\alpha \in A$, $\pi_\alpha (\scF) = \scF_\alpha$. 
\end{enumerate}
\end{prop}





\section{Topological Spaces}
\subsection{Open Sets, Closed Sets, and Neighborhoods}

\newcommand{\scTopologicalSpace}[2]{
    \pa{#1, \scTopology{#1}{#2}}
}
\newcommand{\scTopology}[2]{
    #2_{#1}
}
\newcommand{\TopologicalSpace}[0]{\textbf{\hyperref[def:TopologicalSpace]{Topological Space}}\xspace}
\newcommand{\Topology}[0]{\textbf{\hyperref[def:TopologicalSpace]{Topology}}\xspace}
\newcommand{\TopologicalSpaces}[0]{\textbf{\hyperref[def:TopologicalSpace]{Topological Spaces}}\xspace}
\newcommand{\Topologies}[0]{\textbf{\hyperref[def:TopologicalSpace]{Topologies}}\xspace}

\begin{df}[Topological Space]
    \label{def:TopologicalSpace}

\rm	
    Let $X$ be a nonempty set and let $\T \subset 2^X$ such that 
    \begin{enumerate}[label=(\roman*), ref={\ref{def:TopologicalSpace}~\roman*}]
    \item \label{def:TopologicalSpace:ContainsX} $X \in \T$. 
    \item \label{def:TopologicalSpace:ContainsEmptyset} $\emptyset \in \T$. 
    \item \label{def:TopologicalSpace:ClosureUnderUnions} $\T$ is 
    closed under arbitrary unions.
    \item \label{def:TopologicalSpace:ClosureUnderFiniteIntersections} $\T$
    is closed under finite intersections. 
    \end{enumerate}
    Then we call $\T$ a \Topology on $X$ and we call 
    $(X,\T)$ a \TopologicalSpace. 
\end{df}




\label{def:DiscreteIndiscreteTopology}
\newcommand{\DiscreteTopology}[0]{\textbf{\hyperref[def:DiscreteIndiscreteTopology]{Discrete Topology}}\xspace}
\newcommand{\DiscreteTopologies}[0]{\textbf{\hyperref[def:DiscreteIndiscreteTopology]{Discrete Topologies}}\xspace}
\newcommand{\IndiscreteTopology}[0]{\textbf{\hyperref[def:DiscreteIndiscreteTopology]{Indiscrete Topology}}\xspace}
\newcommand{\IndiscreteTopologies}[0]{\textbf{\hyperref[def:DiscreteIndiscreteTopology]{Indiscrete Topologies}}\xspace}
\begin{df}[\DiscreteTopology, \IndiscreteTopology]
    Let $X$ be a set. 
    We call $\{X, \emptyset\}$ 
    the \IndiscreteTopology
    on $X$
    and we call 
    \scPowerSet{X}
    the 
    \DiscreteTopology 
    on $X$. 
\end{df}


\newcommand{\SetOpen}[0]{\textbf{\hyperref[def:OpenSetClosedSet]{Open}}\xspace}
\newcommand{\SetOpenness}[0]{\textbf{\hyperref[def:OpenSetClosedSet]{Openness}}\xspace}
\newcommand{\SetClosed}[0]{\textbf{\hyperref[def:OpenSetClosedSet]{Closed}}\xspace}
\newcommand{\SetClosedness}[0]{\textbf{\hyperref[def:OpenSetClosedSet]{Closedness}}\xspace}
\begin{df}[\SetOpen, \SetClosed]
\label{def:OpenSetClosedSet}

\rm
    Let $(X, \T)$ 
    be a 
    \TopologicalSpace, 
    and let $A \in \T$. 
    Then we say that 
    $A$ is 
    \SetOpen
    in $(X,\T)$
    and we say that 
    $X \setminus A$ 
    is 
    \SetClosed
    in $(X,\T)$.
    When confusion is unlikely we say 
    that A is \SetOpen in $X$ or in $\T$, or that $A$ is \SetOpen
    with no qualification.
\end{df}

\label{def:CompactSet}
\newcommand{\SetCompact}[0]{
    \textbf{\hyperref[def:CompactSet]{Compact}}
}
\newcommand{\SetCompactness}[0]{
    \textbf{\hyperref[def:CompactSet]{Compactness}}
}

\begin{df}[\SetCompact]
    We say that a 
    \TopologicalSpace
    is 
    \SetCompact
    if every 
    \SetOpen
    \Cover 
    for $X$ 
    has a 
    \Finite
    \Subcover.
\end{df}

\label{def:CompactFunction}
\newcommand{\CompactFunction}[0]{
    \textbf{\hyperref[def:CompactFunction]{Compact}}
}
\newcommand{\FunctionCompactness}[0]{
    \textbf{\hyperref[def:CompactFunction]{Compactness}}
}
\begin{df}[\CompactFunction]
    Let $(X,\T_X)$ and
    $(Y, \T_Y)$ 
    be 
    \TopologicalSpaces. 
    We say that 
    $f:X \to Y$ 
    is 
    \CompactFunction
    if 
    $f(K)$ is 
    \SetCompact 
    in $(Y,\T_Y)$ 
    for every 
    \SetCompact
    $K \in (X,\T_X)$. 
\end{df}


\label{def:TopologyCoarseFine}
\newcommand{\TopologyCoarse}[0]{\textbf{\hyperref[def:TopologicalSpace]{Coarse}}}
\newcommand{\TopologyFine}[0]{\textbf{\hyperref[def:TopologicalSpace]{Fine}}}
\newcommand{\TopologyCoarser}[0]{\textbf{\hyperref[def:TopologicalSpace]{Coarser}}}
\newcommand{\TopologyFiner}[0]{\textbf{\hyperref[def:TopologicalSpace]{Finer}}}
\newcommand{\TopologyCoarsest}[0]{\textbf{\hyperref[def:TopologicalSpace]{Coarsest}}}
\newcommand{\TopologyFinest}[0]{\textbf{\hyperref[def:TopologicalSpace]{Finest}}}
\newcommand{\TopologyFineness}[0]{\textbf{\hyperref[def:TopologicalSpace]{Fineness}}}
\newcommand{\TopologyCoarseness}[0]{\textbf{\hyperref[def:TopologicalSpace]{Coarseness}}}

\newcommand{\scTopologyCoarsenessRelation}[1]{
    \hyperref[def:TopologyCoarseFine]{\ensuremath{\leq_{TopCoarse(#1)}}}
}
\newcommand{\scTopologyFinenessRelation}[1]{
    \hyperref[def:TopologyCoarseFine]{\ensuremath{\leq_{TopFine(#1)}}}
}

\begin{df}[\TopologyCoarse, \TopologyFine]
    Let $X$ be a set. 
    Let $\T_1, \T_2$ be 
    \TopologyRef s
    on $X$
    such that $\T_1 \subset \T_2$. 
    In this case, we say that
    $\T_1$ is more 
    \TopologyCoarse
    than 
    $\T_2$, 
    that $\T_1$
    is 
    \TopologyCoarser
    than $\T_2$, 
    that
    $\T_2$ is more 
    \TopologyFine
    than $\T_1$, 
    that $\T_2$ is 
    \TopologyFiner
    than $\T_1$,
    and we write
    $\T_1 \scTopologyFinenessRelation{X} \T_2$
    and we write
    $\T_2 \scTopologyCoarsenessRelation{X} \T_1$.
    
    Let $K$ denote the collection of topologies on $X$. 
    Let $A\subset K$. 
    If one exists, a  
    \Maximum
    of 
    $A$
    with respect to $\scTopologyFinenessRelation{X}$
    is called the 
    \TopologyFinest
    topology in $A$. 
    If one exists, a
    \Maximum
    of 
    $A$
    with respect to 
    $\scTopologyCoarsenessRelation{X}$
    is called the 
    \TopologyCoarsest
    topology in $A$. 



    Since $\subset$ defines a 
    \PartialOrder
    on the power set of $X$, 
    \TopologyFineness
    defines a 
    \PartialOrder
    on the set of 
    \TopologyRef 's
    of $X$. 

    The intersection of any 2 topologies on $X$ 
    is a topology on $X$, so that 
    $\scTopologyCoarsenessRelation{X}$
    is a \Direction 
    on $X$. 
\end{df}
%

\label{def:Neighborhood}
\newcommand{\Neighborhood}[0]{ \bf \hyperref[def:Neighborhood]{Neighborhood} \rm }
\newcommand{\Neighborhoods}[0]{ \bf \hyperref[def:Neighborhood]{Neighborhoods} \rm }
\newcommand{\NeighborhoodFilter}[0]{ \bf \hyperref[def:Neighborhood]{Neighborhood Filter} \rm }
\newcommand{\NeighborhoodFilters}[0]{ \bf \hyperref[def:Neighborhood]{Neighborhood Filters} \rm }
\newcommand{\NeighborhoodFilterInstance}[0]{\scU}
\begin{df}[\Neighborhood, \NeighborhoodFilter]
    Let $(X, \T)$ be a \TopologicalSpace.
    $A$ be \SetOpen in $(X, \T)$, 
    and $x \in B \subset A$. 
    We call $A$ a 
    \Neighborhood
    of $x$ in $(X,\T)$. 
    We represent the collection of all \Neighborhoods 
    of $x$ in $(X,\T)$ with 
    $\NeighborhoodFilterInstance_{\T}(x)$
    and we call this the 
    \NeighborhoodFilter of 
    $\T$
    at 
    $x$. 
    We also call $A$ a 
    \Neighborhood of $B$ 
    in $(X,\T)$, and 
    we represent the collection of all 
    \Neighborhoods of 
    $B$ 
    with 
    $\NeighborhoodFilterInstance_{\T}(B)$.
\end{df}

\begin{prop}[\NeighborhoodFilter is a \Filter]
\label{prop:NeighborhoodFilter}
Let $(X,\T)$ be a \TopologicalSpace 
and let $x \in X$. 
Then $\NeighborhoodFilterInstance{\T}(x)$, the
\NeighborhoodFilter of $x$, is in fact a 
\Filter on $X$. 
\begin{proof}
For part 1 of 
\ref{def:Filter:DoesntContainEmpty}
, since $x \not \in \emptyset$, $\emptyset \not \in\NeighborhoodFilterInstance{\T}(x)$
If $\{G_1, G_2\} \subset \NeighborhoodFilterInstance{\T}(x)$ with 
$G_1 \cap G_2 \neq \emptyset$
then there are \SetOpen $x \in U_i \subset G_i$ and we have
$x \in U_1 \cap U_2 \subset U_1 \cap U_2 \subset G_1 \cap G_2$ with 
$U_1 \cap U_2 \in \T$. Hence, 
\ref{def:Filter:FiniteIntersectionProperty}
holds. 
\ref{def:Filter:SubsetProperty} is obvious
\end{proof}
\end{prop}


\begin{prop}[\Topology from \NeighborhoodFilters]
\label{prop:TopFromNbhFilter}
Let $X \neq \emptyset$. 
For each $x \in X$, let 
$\NeighborhoodFilterInstance{}(x) \subset \scPowerSet{X}$ such that 
each $\NeighborhoodFilterInstance{}(x)$ satisfies 
\ref{def:Filter:SubsetProperty}
\ref{def:Filter:FiniteIntersectionProperty}
and
\ref{prop:NeighborhoodFilter:Containsx}
and the collection
$\{ \NeighborhoodFilterInstance{}(x) | x \in X \}$ 
satisfies 
\ref{prop:NeighborhoodFilter:CharacteristicProperty}.
Then there existss a Unique topology $\T$ 
on $X$ such that 
for each $x \in X$, 
$\NeighborhoodFilterInstance{}(x)$ is the 
\NeighborhoodFilter for $\T$ at $x$. 
\begin{proof}
\end{proof}

\end{prop}

\label{def:RelationOfEqualNeighborhoodFilters}
\newcommand{\RelationOfEqualNeighborhoodFilters}[1]{
    \bf \hyperref[def:RelationOfEqualNeighborhoodFilters]{Relation Of Equal Neighborhood Filters} \rm on #1
}
\begin{df}[relation of equal neighborhood filters]
    
    Let $(Z, \T_Z)$ be a topological space. Define the relation $\cong$ on Z by setting, for $x,y \in Z$, 
    \begin{equation}
        x \cong y \iff \scU_{\T_Z}(x)=\scU_{\T_Z}(y)
    \end{equation}
    We call $\cong$ the \RelationOfEqualNeighborhoodFilters{$(Z,\T_Z)$}
\end{df} 

\begin{prop}[\RelationOfEqualNeighborhoodFilters]
    \label{prop:EqualNeighborhoodFiltersEquivalenceRelation}
    
    The
	\RelationOfEqualNeighborhoodFilters
	$\cong$ on a \TopologicalSpaceRef $(Z,\T_Z)$ forms an 
	\EquivalenceRelation	
	on Z. 
    \begin{proof}
        
        Let $x \in (Z,\T_Z)$. 
        Then $\NbhFilter{\Topology{Z}{\T}}{x}$=$\NbhFilter{\Topology{Z}{\T}}{x}$, so $x \cong x$.
        Thus $\cong$ is 
		\ReflexiveRelation. 
        
        Let $x,y \in (Z,\T_Z)$. 
        Suppose $x \cong y$. 
        Then  $\NbhFilter{\Topology{Z}{\T}}{x} = \NbhFilter{\Topology{Z}{\T}}{y}$
        , so trivially  $\NbhFilter{\Topology{Z}{\T}}{y} =\NbhFilter{\Topology{Z}{\T}}{x}$
        , and thus $y \cong x$.
        Hence, $\cong$ is 
		\SymmetricRelation
        
        Let $x,y,z \in (Z,\T_Z)$.
        Let $x \cong y$ and $y \cong z$. 
        Then, 
         $\NbhFilter{\Topology{Z}{\T}}{x}= \NbhFilter{\Topology{Z}{\T}}{y} =  \NbhFilter{\Topology{Z}{\T}}{z}$
         so that $x \cong z$.
         Thus $\cong$ is \TransitiveRelation
         
         Since $\cong$ is 
		 \ReflexiveRelation
		, \SymmetricRelation
		, and \TransitiveRelation, it is an 
		\EquivalenceRelation. 
        
    \end{proof}
\end{prop}
\label{def:AccumulationClosureInterior}
\newcommand{\AccumulationPoint}[0]{ 
    \bf \hyperref[def:AccumulationClosureInterior]{Accumulation Point} \rm 
}
\newcommand{\AccumulationPoints}[0]{
    \bf \hyperref[def:AccumulationClosureInterior]{Accumulation Points} \rm 
}
\newcommand{\AccumulationPointMark}[1]{
    #1'
}
\newcommand{\Closure}[0]{
    \bf \hyperref[def:AccumulationClosureInterior]{Closure} \rm 
}
\newcommand{\Closures}[0]{
    \bf \hyperref[def:AccumulationClosureInterior]{Closures} \rm 
}
\newcommand{\ClosureMark}[1]{
    \overline{#1}
}
\newcommand{\Interior}[0]{
    \bf \hyperref[def:AccumulationClosureInterior]{Interior} \rm 
}
\newcommand{\Interiors}[0]{
    \bf \hyperref[def:AccumulationClosureInterior]{Interiors} \rm 
}
\newcommand{\InteriorMark}[1]{
    \overset{\circ}{#1}
}
\newcommand{\Boundary}[0]{
    \bf \hyperref[def:AccumulationClosureInterior]{Boundary} \rm
}
\newcommand{\Boundaries}[0]{
    \bf \hyperref[def:AccumulationClosureInterior]{Boundaries} \rm
}
\newcommand{\BoundaryMark}[1]{
    \partial\pa{#1}
}
\begin{df}[\AccumulationPoint, \Closure, \Interior, \Boundary]
    Let $(X,\T)$ be a \TopologicalSpace.
    Let $A \subset X$. We define the following. 
    \begin{enumerate}
        \item $\AccumulationPointMark{A}=\{x \in X | (\forall U \in \NeighborhoodFilterInstance_{\T}(A))((U \setminus A) \cap \{x\} \neq \emptyset)\} $
        \item $\ClosureMark{A}= A \cup \AccumulationPointMark{A}$
        \item $\BoundaryMark{A} = \ClosureMark{A} \cap \ClosureMark{X \setminus A}$
        \item $\InteriorMark{A} = A \setminus \ClosureMark{X \setminus A}$
    \end{enumerate}
    We call 
    an element of $\AccumulationPointMark{A}$ 
    an \AccumulationPoint of $A$. 
    We call $\ClosureMark{A}$ the 
    \Closure 
    of $A$. 
    We call $\InteriorMark{A}$ 
    the \Interior 
    of $A$. 
    We call $\BoundaryMark{A}$ 
    the \Boundary of $A$. 
\end{df}

\label{def:TopologySubBasis}
\newcommand{\TopologySubBasis}[0]{
    \textbf{\hyperref[def:TopologySubBasis]{SubBasis}}
}
\newcommand{\TopologySubBases}[0]{
    \textbf{\hyperref[def:TopologySubBasis]{SubBases}}
}
\newcommand{\scGeneratedTopology}[2]{
    \hyperref[def:TopologySubBasis]{\ensuremath{\T_{#1}\pa{#2}}}
}
\newcommand{\TopologyGeneratedBy}[0]{
    \textbf{\hyperref[def:TopologySubBasis]{Generated By}}
}

\begin{df}[\TopologySubBasis]
    Let $X \neq \emptyset$ 
    and let $B \subset \scPowerSet{X}$. 
    We denote the 
    \TopologyCoarsest
    \TopologyRef on $X$ 
    containing $B$
    with
    \scGeneratedTopology{X}{B}.
    We say that $B$
    is a 
    \TopologySubBasis
    for 
    \scGeneratedTopology{X}{B}
    and we call 
    \scGeneratedTopology{X}{B}
    the 
    \TopologyRef
    on $X$ 
    \TopologyGeneratedBy
    $B$. 
\end{df}

\begin{prop}[Characterization Of Generated Topology]
    \label{prop:CharacterizationOfGeneratedTopoology}

    \rm
    Let $X$ be a nonempty set.
    Let $F \subset \scPowerSet{X}$. 
    Let $K$ be the collection of finite intersections of elements of $F$. 
    Let $\scK$ be the collection of unions of elements of $K$. 
    Define 
    \begin{equation*}
    \T_{Prop} = \scK \cup \{X, \emptyset\}
    \end{equation*}
    Define 
    Then 
    $\scGeneratedTopology{X}{F} = \T_{Prop}$.
    \begin{proof}
        We first show that $\T_{Prop}$ is a
        \Topology
        on $X$. 
        To prove closure under arbitrary unions, 
        Let $B \neq \emptyset$ 
        and $\{B_\beta\}_{\beta \in B} \subset \T_{Prop}$
        Then for each $\beta \in B$, we can find $A_{\beta}$
        such that for each $\alpha_{\beta} \in A_{\beta}$, 
        there is an $N_{\alpha_{\beta}} \in \N$ 
        such that for each $i \in \{1, \cdots, N_{\alpha_\beta}\}$, 
        $U_{i, \alpha_\beta} \in F$ and
        \begin{equation}
            B_{\beta} = \bigcup\limits_{\alpha \in A_\beta} \bigcap\limits_{i=1}^{N_{\alpha_\beta}}U_{i, \alpha_\beta}
        \end{equation}
        Hence, we can write
         
        \begin{align*}
            \bigcup\limits_{\beta \in B} B_{\beta} & = \bigcup\limits_{\beta \in B} \bigcup\limits_{\alpha_\beta \in A_{\beta}} \bigcap\limits_{i=1}^{N_{\alpha_\beta}} U_{i, \alpha_\beta}\\
            & = \bigcup\limits_{\alpha_\beta \in \bigcup\limits_{\beta \in B} A_{\beta}} \bigcap\limits_{i =1 }^{N_{\alpha_\beta}} U_{i, \alpha_\beta} \in \T_{Prop}
        \end{align*}
        To prove closure under finite intersections, 
        let $N \in \N$ and 
        $\{B_j\}_{j=1}^N \subset \T_{Prop}$. 
        Then for each $j \in \{1, \cdots, N\}$, there 
        is an $A_{j}$
        such that for each $\alpha_{j} \in A_{j}$, 
        there is an $N_{\alpha_{j}} \in \N$ 
        such that for each $i \in \{1, \cdots, N_{\alpha_j}\}$, 
        $U_{i, \alpha_j} \in F$ and
        \begin{align*}
            \bigcap\limits_{j=1}^N B_j & = \bigcap\limits_{j=1}^N \bigcup\limits_{\alpha_j \in A_j} \bigcap\limits_{i=1}^{N_{\alpha_j}} U_{\alpha_j, i}\\
            & = \bigcup\limits_{\{\alpha_j\}_{j=1}^N \in \scCartesianProduct{j}{\{1, \cdots, N\}}{A}}\pa{\bigcap\limits_{j=1}^{N} \bigcap_{i=1}^{N_{\alpha_j}} U_{\alpha_j, i}} \in \T_{Prop}
        \end{align*}
        By construction, $X \in \T_{Prop}$ and $\emptyset \in \T_{Prop}$, so 
        $\T_{Prop}$ is in fact a 
        \Topology on 
        $X$. 
        By taking the union over the intersection of a single element, we have $F \subset \T_{Prop}$, so that 
        $\scGeneratedTopology{X}{F} \subset \T_{Prop}$. 
        Furthermore, $\scGeneratedTopology{X}{F}$ is closed under finite intersections and arbitrary unions
        so that it must contain $\T_{Prop}$. 
        Hence, equality holds. 



    \end{proof}
\end{prop}

\label{def:TopologyBasis}
\newcommand{\TopologyBasis}[0]{
    \textbf{\hyperref[def:TopologyBasis]{Basis}}
}
\newcommand{\TopologyBases}[0]{
    \textbf{\hyperref[def:TopologyBasis]{Bases}}
}
\begin{df}[\TopologyBasis]
    Let $(X,\T)$ 
    be a 
    \TopologicalSpaceRef
    and let $B \subset \T$ 
    such that 
    each element of 
    $\T$ 
    can be written as a union 
    of elements of 
    $B$. 
    Then we call 
    $B$
    a 
    \TopologyBasis
    for $\T$. 
\end{df}

\begin{prop}
    \label{prop:TopologyBasisCharacterization}
    Let $(X,\T)$ be a 
    \TopologicalSpace 
    and let 
    $\scG \subset \T$ 
    such that $\{\emptyset, X \} \subset \scG$. 
    The following conditions are equivalent
    \begin{enumerate}
        \item For every
            \SetOpen
            $U$, 
            for every $x \in U$, 
            there exists an
            $G_x \in \scG$
            such that 
            $x \in G_x \subset U$. 
        \item $\scG$  is a \TopologyBasis for $\T$. 
    \end{enumerate}
    \begin{proof}[$1 \implies 2$]
        Let $U \in \T$. Then we can write $U = \bigcup_{x \in U} G_x$, implying that $\scG$ is a \TopologyBasis. 
    \end{proof}
    \begin{proof}[$2 \implies 1$]
        Let $U$ is \SetOpen, then since 
        $\scG$ is a \TopologyBasis, 
        there is a $\{G_{\alpha}\}_{\alpha \in A} \subset \scG$ such that 
        $U = \bigcup_{\alpha \in A} G_{\alpha}$. 
        Hence, if $x \in U$, then 
        $x \in G_{\alpha}$ for some $\alpha \in A$, and obviously $G_{\alpha} \subset U$, 
        so $1$ holds and we're done. 
    \end{proof}
\end{prop}

\begin{prop}[Bassis Of Generated Topology]
\label{prop:BasisOfGeneratedTopology}
    Let $X \neq \emptyset$ and let 
    $B \subset \scPowerSet{X}$
    Then $B$ is a 
    \TopologyBasis 
    for 
    \scGeneratedTopology{X}{B}
    if and only if 
    the following hold
    \begin{enumerate}
        \item $X \in B$
        \item $\emptyset \in B$. 
        \item For each $U,V \in B$, For each $x \in U \cap V$, there is a $W \in B$ with $x \in W \subset U \cap V$. 
    \end{enumerate}
    \begin{proof}
        I first claim that 
        it is sufficient to show that any finite intersection of elements of $B$ can 
        be written as a union of elements of $B$.
        By Induction, proving for a binary intersection is sufficient. 
        Hence, let $U,V \in B$ with $U \cap V \neq \emptyset$. 
        Then for each $x \in U \cap V$, by assumption, 
        there exists a $W_x \in B$ such that 
        $x \in W_x \subset U \cap V$. 
        Hence, we can write 
        \begin{equation*}
            U \cap V \subset \bigcup_{x \in U \cap V} W_x \subset U \cap V
        \end{equation*}
        showign that finite intersctions of 
        elements of $B$ can be written as unions of 
        elements of $B$. 
        Hence, by
        \ref{prop:CharacterizationOfGeneratedTopoology}, 
        $\scGeneratedTopology{X}{B}$ 
        consists of exactly the unions of elements of $B$, finishing one direction.
        For the other direction, 
        if $B$ is a 
        \TopologyBasis
        for $\scGeneratedTopology{X}{B}$, 
        then 
        by 
        \ref{prop:TopologyBasisCharacterization}, 
        sicnce $\scGeneratedTopology{X}{B}$ 
        contains finite intersections of elements of $B$, 
        the given properties hold. 
       
        

    \end{proof}
\end{prop}

\newcommand{\FundamentalSystemOfNeighborhoods}[0]{\textbf{\hyperref[def:FundamentalSystemOfNeighborhoods]{Fundamental System Of Neighborhoods}}\xspace}
\newcommand{\FundamentalSystemsOfNeighborhoods}[0]{\textbf{\hyperref[def:FundamentalSystemOfNeighborhoods]{Fundamental Systems Of Neighborhoods}}\xspace}
\begin{df}[\FundamentalSystemOfNeighborhoods]
\label{def:FundamentalSystemOfNeighborhoods}

\rm
    Let $(X,\scT)$ be a \TopologicalSpace.
    Let $x \in X$. 
    Let $\NeighborhoodFilterInstance{\scT}(x)$ denote
    the \NeighborhoodFilter of $\scT$ at $x$. 
    We say that $\scK$ is a \FundamentalSystemOfNeighborhoods for $X$ at $x$ if 
    \begin{enumerate}[label=(\roman*), ref={\ref{def:FundamentalSystemOfNeighborhoods}~\roman*}]
    \item \label{def:FundamentalSystemOfNeighborhoods:AreNeighborhoods}
    $\scK \subset \NeighborhoodFilterInstance{\scT}(x)$. 
    \item \label{def:FundamentalSystemOfNeighborhoods:Nesting}
    For each $U \in \NeighborhoodFilterInstance{\scT}(x)$, there exists $V \in \scK$ such that $V \subset U$.
    \end{enumerate}
    It is clear that 
    $\scU_{\scT}(x)$
    is a \FundamentalSystemOfNeighborhoods
    for $X$ at $x$. 
\end{df}

\newcommand{\NeighborhoodBasis}[0]{\textbf{\hyperref[def:NeighborhoodBasis]{Neighborhood Basis}}\xspace}
\newcommand{\NeighborhoodBases}[0]{\textbf{\hyperref[def:NeighborhoodBasis]{Neighborhood Bases}}\xspace}
\begin{df}[\NeighborhoodBasis]
\label{def:NeighborhoodBasis}

\rm
    Let $(X,\T)$ be a
    \TopologicalSpace
    and let $x \in X$. 
    Let 
    $F \subset \T$ 
    such that
    $\scNested{U}{\scU_{\scT}(x)}$.
    Further, let $x \in G$ for each $G \in F$. 
    Then we call $F$ a 
    \NeighborhoodBasis
    for $\T$ at $x$. 
\end{df}


\begin{prop}[\NeighborhoodBasis Facts]
    \label{prop:NeighborhoodBasisFacts}
    Let $(X,\T)$ be a Topological Space and 
    let $x \in X$. 
   % Let $\scB_X$ be a \NeighborhoodBaiss for $\T$ at $X$. 
    The following are true
    

\end{prop}


\subsection{Continuous Functions}
\newcommand{\ContinuityAt}[0]{\textbf{\hyperref[def:FunctionContinuityAtAPoint]{Continuity At}}\xspace}
\newcommand{\ContinuousAt}[0]{\textbf{\hyperref[def:FunctionContinuityAtAPoint]{Continuous At}}\xspace}
\begin{df}[\ContinuityAt a point]
\label{def:FunctionContinuityAtAPoint}
Let $(X, \T_X)$ be a \TopologicalSpace.
Let $(Y, \T_Y)$ be a \TopologicalSpace.
Let $f:X \to Y$. 
Let $x_0 \in X$. 
Let $\scU_{\T_X}(x_0)$ be the
\NeighborhoodFilter
of $\T_X$ at $x_0$. 
Let $\scU_{\T_Y}(f(x_0))$ be the
\NeighborhoodFilter
of $\T_Y$ at $f(x_0)$.
We say that $f$ is 
\ContinuousAt $x_0$, 
and we say that $f$ posesses
\ContinuityAt $x_0$, if 
\scNested{\scU_{\T_X}(x_0)}{f^{-1} \pa{\scU_{\T_Y}(f(x_0)}}
holds.
\end{df}

\begin{prop}
\label{prop:ContinuityAtAPoint:Closure}
Let $(X, \T_X)$ be a \TopologicalSpace. 
Let $(Y, |T_Y)$ be a \TopologicalSpace. 
Let $A \subset X$. 
Let $x_0 \in \ClosureMark{A}$.
Let $f:X \to Y$ be 
\ContinuousAt $x_0$. 
Then $f(x_0) \in \ClosureMark{f(A)}$. 
\begin{proof}
Let $\scV$ be the \NeighborhoodFilter 
of $\T_Y$ about $f(x_0)$. 
Let $V \in \scV$. 


\end{proof}
\end{prop}

\label{def:FunctionContinuous}
\newcommand{\ContinuousFunction}[0]{
    \textbf{\hyperref[def:FunctionContinuous]{Continuous}}
}
\newcommand{\FunctionContinuity}[0]{
    \textbf{\hyperref[def:FunctionContinuous]{Continuity}}
}

\begin{df}[\ContinuousFunction]
    Let $(X,\T_X)$ and 
    $(Y,\T_Y)$
    be
    \TopologicalSpaces.
    We say that a function
    $f:X \to Y$ is 
    \ContinuousFunction
    and that it exhibits
    \FunctionContinuity
    with respect to $\T_1$ and $\T_2$
    $f^{-1}(\T_Y) \subset \T_X$. 
    We may make the \Topologies 
    explicit by writing 
    $f:(X, \T_X) \to (Y, \T_Y)$, 
    in which case we just say that
    $f$ is 
    \ContinuousFunction
    or that $f$ 
    posesses 
    \FunctionContinuity.
\end{df}

\begin{prop}[global \FunctionContinuity iff \ContinuityAt each point]
\label{prop:ContinuityCharacterization}
Let $(X, \T_X)$ and 
$(Y, \T_Y)$ 
be 
\TopologicalSpaces.
Let $f:X \to Y$. 
The following are equivalent:
\begin{enumerate}[label=(\roman*), ref={\ref{prop:ContinuityCharacterization}.~\roman*}]
\item
\label{prop:ContinuityCharacterizations:GlobalContinuity}
$f$ is \ContinuousFunction.
\item 
\label{prop:ContinuityCharacterizations:ContinuousAtEachPoint}
$f$ is \ContinuousAt each point in $X$. 
\item 
\label{prop:ContinuityCharacterizations:Closure}
For each $A \subset X$, 
$f\pa{\ClosureMark{A}} \subset \ClosureMark{f(A)}$
\item
\label{prop:ContinuityCharacterizations:ClosedInverseImage}
If $B \subset Y$ is 
\SetClosed in $Y$, 
then 
$f^{-1}(B)$ is \SetClosed in $X$. 
\end{enumerate}
\begin{proof}[Proof of $\ref{prop:ContinuityCharacterizations:GlobalContinuity} \implies \ref{prop:ContinuityCharacterizations:ContinuousAtEachPoint}$]

Let $x_0 \in X$. 
Let $\scU$ denote the \NeighborhoodFilter of $\T_Y$ at $f(x_0)$. 
Let $\scV$ denote the \NeighborhoodFilter of $\T_X$ at $x_0$.
Let $U \in \scU$. 
Then by 
\ref{def:Neighborhood}
there exists \SetOpen $\tilde{U} \in \scU$ such that $f(x_0) \in \tilde{U} \subset U$. 
By the assumption 
\ref{prop:ContinuityCharacterizations:GlobalContinuity}, 
$f^{-1}\pa{\tilde{U}} \in \T_X$. 
Also, since $f(x_0) \in \tilde{U}$, $x_0 \in f^{-1}\pa{\tilde{U}}$.
Hence, $f^{-1}\pa{\tilde{U}} \in \scV$. 
Since $f^{-1}\pa{\tilde{U}} \subset f^{-1}\pa{U}$, 
we conclude 
\scNested{\scV}{f^{-1}(\scU)}.
Hence $f$ is \ContinuousAt $x_0$. 
Since $x_0 \in X$ was arbitrary, we are done. 
\end{proof}
\begin{proof}[Proof of $\ref{prop:ContinuityCharacterizations:ContinuousAtEachPoint} \implies \ref{prop:ContinuityCharacterizations:Closure}$]

This result is a direct application of 
\ref{prop:ContinuityAtAPoint:Closure}.
\end{proof}
\begin{proof}[Proof of $\ref{prop:ContinuityCharacterizations:Closure} \implies \ref{prop:ContinuityCharacterizations:ClosedInverseImage}$]
\end{proof}
\begin{proof}[Proof of $\ref{prop:ContinuityCharacterizations:ClosedInverseImage} \implies \ref{prop:ContinuityCharacterizations:GlobalContinuity}$]
\end{proof}
\end{prop}

\newcommand{\Homeomorphism}[0]{\textbf{\hyperref[def:Homeomorphism]{Homeomorphism}}\xspace}
\newcommand{\Homeomorphisms}[0]{\textbf{\hyperref[def:Homeomorphism]{Homeomorphisms}}\xspace}
\newcommand{\Homeomorphic}[0]{\textbf{\hyperref[def:Homeomorphism]{Homeomorphic}}\xspace}
\newcommand{\Homeomorphically}[0]{\textbf{\hyperref[def:Homeomorphism]{Homeomorphically}}\xspace}

\begin{df}[\Homeomorphism]
\label{def:Homeomorphism}

\rm
    Let $\pa{X,\T_X}$
    and $\pa{Y, \T_Y}$ be \TopologicalSpaces.
    Let $f:X \to Y$ such be a 
    \ContinuousFunction
    \Bijection
	such that $f^{-1}:Y \to X$ is also 
	\ContinuousFunction.
    Then we say that \(f\) is a 
    \Homeomorphism from \(X\) to \(Y\)
    and we say that 
    \(X\) and \(Y\) are 
    \Homeomorphic
	and we say that \(f\) operates 
	\Homeomorphically.
\end{df}

 %LATER
\label{def:WeakTopology}
\newcommand{\WeakTopology}[0]{
	\bf \hyperref[def:WeakTopology]{Weak Topology} \rm
}

\begin{df}[\WeakTopology]
	Let X be a set. 
    For each $\alpha \in A$, let 
    $(Y_\alpha, T_\alpha)$ be a 
    \TopologicalSpace, 
    and let $\phi_\alpha:X \to (Y_\alpha, T_\alpha)$. 
    Let $\T$ be the 
	\TopologyCoarsest possible 
    \Topology on X such that 
    for each $\alpha \in A$, 
    $\phi_\alpha:(X, \T) \to (Y, \T_\alpha)$ 
    is \ContinuousFunction. 
    We call $\T$ the
    \WeakTopology on 
    X induced by $\{\phi_\alpha\}_{\alpha \in A}$
\end{df}
 %LATER
\label{def:InductiveTopology}
\newcommand{\InductiveTopology}[0]{
    \textbf{\hyperref[def:InductiveTopology]{Inductive Topology}}
}
\newcommand{\InductiveTopologies}[0]{
    \textbf{\hyperref[def:InductiveTopology]{Inductive Topologies}}
}

\begin{df}[\InductiveTopology]
    Let $X$ be a set and for each 
    $\alpha \in A$, let 
    $(Y_\alpha, \T_\alpha)$ be a 
    \TopologicalSpace.
    Furthermore, for each $\alpha \in A$, let 
    $\phi_\alpha : (Y, \T_\alpha) \to X$. 
    Let $\T$ be the 
    \TopologyFinest
    topology on $X$ for which 
    each $\phi_\alpha$ is 
    \ContinuousFunction. 
    fWe call $\T$ the \InductiveTopology
    on $X$ induced by $\{\phi_\alpha\}_{\alpha \in A}$. 

    
\end{df}
 %LATER
\label{def:OpenFunction}
\newcommand{\OpenFunction}[0]{
    \textbf{\hyperref[def:OpenFunction]{Open}}
}
\newcommand{\FunctionOpenness}[0]{
    \textbf{\hyperref[def:OpenFunction]{Openness}}
}
\begin{df}[\OpenFunction]
    Let $(X,\T_X)$ and
    $(Y, \T_Y)$ 
    be 
    \TopologicalSpaces. 
    We say that 
    $f:X \to Y$ 
    is 
    \OpenFunction
    if 
    $f(U)$ is 
    \SetOpen 
    in $(Y,\T_Y)$ 
    for every 
    \SetOpen
    $U \in (X,\T_X)$. 
\end{df}

 %LATER
\label{def:ClosedFunction}
\newcommand{\ClosedFunction}[0]{
    \textbf{\hyperref[def:ClosedFunction]{Closed}}
}
\newcommand{\FunctionClosedness}[0]{
    \textbf{\hyperref[def:ClosedFunction]{Closedness}}
}
\begin{df}[\ClosedFunction]
    Let $(X,\T_X)$ and
    $(Y, \T_Y)$ 
    be 
    \TopologicalSpaces. 
    We say that 
    $f:X \to Y$ 
    is 
    \ClosedFunction
    if 
    $f(K)$ is 
    \SetClosed 
    in $(Y,\T_Y)$ 
    for every 
    \SetClosed
    $K \in (X,\T_X)$. 
\end{df}


\subsection{Subspaces And Quotient Spaces}
\label{def:SubspaceTopology}
\newcommand{\SubspaceTopology}[0]{
    \textbf{\hyperref[def:SubspaceTopology]{Subspace Topology}}
}

\newcommand{\SubspaceTopologies}[0]{
    \textbf{\hyperref[def:SubspaceTopology]{Subspace Topologies}}
}

\newcommand{\SubspaceTopologicalSpace}[0]{
    \textbf{\hyperref[def:SubspaceTop9ology]{Subspace Topological Space}}
}

\newcommand{\SubspaceTopologicalSpaces}[0]{
    \textbf{\hyperref[def:SubspaceTop9ology]{Subspace Topological Spaces}}
}

\begin{df}[\SubspaceTopology]
    Let $(X, \T_X)$ 
    be a 
    \TopologicalSpace 
    and let 
    $Y \subset X$. 
    Define 
    \begin{equation*} 
        \T_Y = \left\{ U \cap Y | U \in \T_X\right\}
    \end{equation*} 
    Then $\T_Y$ 
    is a 
    \Topology 
    on $Y$ 
    which we call the 
    \SubspaceTopology
    on $Y$ of $(X, \T_X)$. 
    We call $(Y, \T_Y)$ the 
    \SubspaceTopologicalSpace.
    Unless otherwise specified, 
    when referring to a subset of a 
    \TopologicalSpace, 
    we consider that subset as 
    being a \TopologicalSpace 
    which is endowed with the \SubspaceTopology, 
    and when we say that a subset of a 
    \TopologicalSpace
    has a particular (Topological) property which has thus far only been defined 
    for a \TopologicalSpace, 
    we mean that the  \SubspaceTopologicalSpace 
    has that property. 
\end{df}


 %LATER
\label{def:QuotientSpaceTopology}
\newcommand{\QuotientSpaceTopology}[0]{
    \bf \hyperref[def:QuotientSpaceTopology]{Quotient Topology} \rm
}
\newcommand{\QuotientTopologicalSpace}[0]{
    \bf \hyperref[def:QuotientSpaceTopology]{Quotient Topological Space} \rm 
}

\begin{df}[Quotient Space Topology]
    Let $(Z,\T_Z)$ be a topological space. 
    Let $\cong$ be the \RelationOfEqualNeighborhoodFilters{$(Z, \T_Z)$}. 
    Let T be the \QuotientMap of Z under the relation $\cong$. 
    Define $\T_{Z/\cong}$ by
    \begin{equation}
        \T_{Z/\cong} = \left\{ \bigcup_{x \in U}\{T(x)\} \in 2^{Z/\cong}| U \in \T_Z \right\}
    \end{equation}
    By \ref{prop:QuotientSpaceTopology}, $\T_{Z/\cong}$ is a topology on $Z/\cong$.
    We call $\T_{Z/\cong}$ the \QuotientSpaceTopology and we call $\pa{Z/\cong, \T_{Z/\cong}}$ the \QuotientTopologicalSpace of $(Z, \T_Z)$.
    
\end{df}

\begin{prop}[Quotient Space Topology]
    \label{prop:QuotientSpaceTopology}
    
    Let $(Z,\T_Z)$ be a 
	\TopologicalSpaceRef
    with \QuotientTopologicalSpace  $\pa{Z/\cong, \T_{Z/\cong}}$
    and \QuotientMap T.
    
    Then the following are true. 
    \begin{enumerate}
        \item $\T_{Z/\cong}$ is a \TopologyRef on $Z/\cong$. 
        \item $T:(Z, \T_Z) \to (Z/\cong, \T_{Z/\cong})$ is \Continuous. 
        \item If U is \SetOpen (\SetClosed) in $(Z,\T_Z)$ then $T(U)$ and $T(Z\setminus U)$ \Partition $Z/\cong$. 
        \item If U is \SetOpen in $(Z, \T_Z)$, then $T^{-1}(T(U))=U$. 
        \item If K is \SetClosed in $(Z,\T_Z)$, then $T^{-1}T(K)=K$. 
        \item $T:(Z, \T_Z) \to (Z/\cong, \T_{Z/\cong})$ is \MapOpen. 
        \item $T:(Z, \T_Z) \to (Z/\cong, \T_{Z/\cong})$ is \MapClosed.
        \item $(Z, \T_Z)$ is a \Compact space if and only if $(Z/\cong, \T_{Z/\cong})$ is a \Compact space.
        \item If $\scB$ is a \TopologyBasis for $\T_z$, then $\{T(U) | U \in \scB\}$ is a \TopologyBasis for $\T_{Z/\cong}$. 
        \item If T is \Injective, then it is a \Homeomorphism. 
    \end{enumerate} 
    \begin{proof}[Proof of 1]
        Since $\emptyset \in \T_Z$, we have 
        \begin{equation}
            \emptyset = \bigcup\limits_{x \in \emptyset} \{Tx\} \in \T_{Z/\cong}
        \end{equation}
        Since $Z \in \T_Z$, and by \ref{rmk:quotientsetpartition}, 
        \begin{equation} 
            Z/\cong = \bigcup_{x \in Z} \{[x]\}= \bigcup\limits_{x \in Z} \{T(x)\} \in \T_{Z/\cong}
        \end{equation} 
        
        Let $\{U_{\alpha} | \alpha \in A\} \subset \T_{Z/\cong}$. 
        For each $\alpha \in A$, there exists $B_{\alpha} \in \T_{Z}$ such that we have
        \begin{equation} 
            U_{\alpha } = \bigcup_{x \in B_{\alpha}} \{Tx\}
        \end{equation} 
        Since $\bigcup_{\alpha \in A} B_\alpha \in \T_{Z}$, we have 
        \begin{equation}
            \bigcup_{\alpha \in A} U_{\alpha}= \bigcup\limits_{\alpha \in A} \bigcup\limits_{x \in U_\alpha} \{T(x)\} = \bigcup\limits_{x \in \bigcup\limits_{\alpha \in A} B_{\alpha}} \{T(x)\} \in \T_{Z/\cong}
        \end{equation} 
        Let $\{U_i\}_{i=1}^n \subset \T_{Z/\cong}$. 
        For each $i \in \{1, ..., n\}$, there exists $B_i \in \T_{Z}$ such that
        \begin{equation}
            U_i = \bigcup_{x \in B_{i}} \{T(x)\}
        \end{equation}
        Suppose 
        \begin{equation}
            [x_0] \in \bigcap\limits_{i=1}^n \bigcup\limits_{x \in B_i} \{T(x)\}
        \end{equation}
        Then for each $i \in \{1,..., n\}$, there is a $y_i \in B_i$ such that $ y_i \cong x_0$. 
        Since each $B_i$ is \SetOpen, the definition of $\cong$ implies that $x_0 \in B_i$ for every i. Hence, 
        \begin{equation} 
            x_0 \in \bigcap_{i=1}^n B_i
        \end{equation} 
        Implying 
        \begin{equation}
            [x_0] \in  \bigcup\limits_{x \in \bigcap\limits_{i=1}^n B_i} \{[x]\}
        \end{equation} 
        Hence, 
        \begin{equation} 
            \bigcap\limits_{i=1}^n \bigcup\limits_{x \in B_i} \{T(x)\}
            \subset
            \bigcup\limits_{x \in \bigcap\limits_{i=1}^n B_i} \{[x]\}
        \end{equation} 
        Furthermore, since the reverse inclusion is obvious, 
        and since $\bigcap_{i=1}^n B_i \in \T_{Z}$, we have 
        \begin{equation}
            \bigcap_{i=1}^n U_i = \bigcap_{i=1}^n \bigcup_{x \in B_i} \{T(x)\}= \bigcup\limits_{x \in \bigcap\limits_{i=1}^n B_i} \{T(x)\} \in \T_{Z/\cong}
        \end{equation}
    \end{proof}
    \begin{proof}[Proof of 2]
        Let $V \in \T_{Z/\cong}$. 
        Let $x_0 \in T^{-1}(V)$. 
        Then $[x_0] \in V$. 
        By definition, there is a $U \in \T_Z$ such that 
        \begin{equation}
            T(U) \subset \bigcup\limits_{x \in U} \{T(x)\}=V
        \end{equation}
        Hence there is a $y_0 \in U$  such that 
        \begin{equation}
            [x_0] \in T(y_0) = \{[y_0]\}
        \end{equation}
        Therefore, $x \cong y$. 
        Definition of the 
		\RelationOfEqualNeighborhoodFilters 
		implies $\scU(x_0)=\scU(y_0)$. 
        Hence, $x_0 \in U \subset T^{-1}(V)$.
    \end{proof}
    \begin{proof}[Proof of 3]
        Let $K$ be closed in $(Z,\T_Z)$. 
        Then each point $x_0$ in $Z\setminus K$ has some $U_{x_0} \in \scU_{\T_Z}(x_0)$ which is 
		\Disjoint 
		from K.
        Hence $y_0 \not \cong x_0$ for any $y_0 \in K$, $x_0 \in Z\setminus K$. 
        Hence $T(K)$ is 
		\Disjoint 
		from $T\pa{Z \setminus K}$. 
        This fact, paired with \ref{prop:QuotientMapSurjective}, implies $T(Z\setminus K)$ and T(K) 
		is a \Partition of $Z/\cong$.
    \end{proof}
    \begin{proof}[Proof of 4]
        Let $U \in \T_Z$. 
        The nontrivial direction to prove is $T^{-1}\pa{T(U)} \subset U$.
        Let $y \in T^{-1}\pa{T(U)}$. 
        Then $[y]=Ty \in T(U)$.
        Hence, $[y]=T(x)=[x]$ for some $x \in U$. 
        Since $y \cong x$ and $x \in U \in \scU_{\T_Z}(x)$, we have $U \in \scU_{\T_Z}(y)$. 
        Hence $y \in U$.
        Since y was arbitrary, $T^{-1}\pa{T(U)} \subset U$, and equality is obvious because the other direction of inclusion is trivial. 
    \end{proof}
    \begin{proof}[Proof of 5]
        Let K be 
		\SetClosed 
		in $(Z,\T_Z)$. Part 3 Of this result implies $Z/\cong$ is partitioned by $T(K)$ and $T(Z\setminus K)$. 
        
        By part 4 of this proposition, 
        \begin{align*}
            T^{-1}\pa{T(K)}&=T^{-1} \pa{T(Z) \setminus T(Z \setminus K)} \\
            &= T^{-1}\pa{Z/\cong \setminus T(Z \setminus K)}\\
            &=T^{-1}(Z/\cong) \setminus T^{-1}(T(Z\setminus K)) \\
            &= Z \setminus \pa{Z \setminus K} \\
            &= K
        \end{align*}      
    \end{proof}
    \begin{proof}[Proof of 6]
        Let $U \in \T_Z$.
        Then by definition of the \QuotientSpaceTopology
        \begin{equation}
            TU= \bigcup_{x \in U} \{T(x)\}  \in \T_{Z/\cong}
        \end{equation}
    \end{proof}  
    \begin{proof}[Proof of 7] 
        Let K be \SetClosed in $(Z,\T_Z)$. 
        Then $Z \setminus K \in \T_Z$. 
        By Parts 3 and five of this proposition, we know $T(K) = Z/\cong \setminus T(Z\setminus K)$ and also that $T(Z\setminus K) \in \T_{Z/\cong}$. Hence $T(K)$ is closed in $(Z/\cong, \T_{Z/\cong})$. 
    \end{proof} 
    \begin{proof}[Proof of 8]
        Let $(Z,\T_Z)$ be \SetCompact. 
        Let $\{U_{\alpha}\}_{\alpha \in A}$ be an open covering of $(Z/\cong, \T_{Z/\cong})$. 
        Then $\{T^{-1}\pa{U_{\alpha}} | \alpha \in A\}$ is an open covering of $(Z, \T_Z)$. 
        \SetCompactness of $(Z, \T_Z)$ guarantees the existence of a finite subcovering $\{T^{-1}\pa{U_{\alpha_i}} | i \in \{1, ..., n\}\}$. 
        Hence
        $\{U_{\alpha_i} | i \in \{1, ..., n\}\}=\{TT^{-1}(U_{\alpha_i}) | i \in \{1, ..., n\}\}$ is an 
		\OpenCover of $(Z/\cong, \T_{Z/\cong})$. 
         And the \SetCompactness of $(Z/\cong, \T_{Z/\cong})$ is verified. 
         
         
         Now, suppose $(Z/\cong, \T_{Z/\cong})$ is \SetCompact. 
         Let $\{V_{\beta} | \beta \in B\}$ be an \OpenCover of $(Z, \T_Z)$. 
         Since T is an \MapOpen mapping, $\{T(V_{\beta}) | \beta \in B\}$ is an 
		 \OpenCover of $(Z/\cong, \T_{Z/\cong})$ which by 
		 \SetCompactness has a \Finite \SubCover $\{T(V_{\beta_i}) | i \in \{1, ..., n\}\}$. 
         By part 4 of \ref{prop:QuotientSpaceTopology}, 
         $\{V_{\beta_i}| i \in \{1, ..., n\}\} = \{T^{-1}(T(V_{\beta_i})) |i \in \{1, ..., n\}\}$ is then an \OpenSubCover of $(Z, \T_Z)$. 
     %    
    \end{proof}
    \begin{proof}[Proof of 9]
        Let $\scB$ be a basis for $\T_z$ and let $V \in \T_{Z/\cong}$. 
        Then $T^{-1}(Z) \in \T_Z$, and so there is a subcollection $\{U_{\alpha}\}_{\alpha \in A} \subset \scB$ such that $T^{-1}(V) = \bigcup_{\alpha \in A} U_{\alpha}$. 
        Hence, 
        \begin{align*}
            V& =T(T^{-1}(V))\\
            & = T\pa{\bigcup_{\alpha \in A} U_{\alpha}}\\
            & = \bigcup_{\alpha \in A} T(U_{\alpha})
        \end{align*}
     \end{proof} 
     \begin{proof} [Proof of 10]
            If T is \Injective
			, then since it is \Continuous Part 2 of this result, open by part 6 of this result, and \Surjective by \ref{prop:QuotientMapSurjective}, it is a 
			\Bicontinuous 
			\Bijection, that is, a \Homeomorphism. 
         \end{proof}
\end{prop} 
\subsection{Product Spaces}
\label{def:ProductTopology}
\newcommand{\ProductTopology}[0]{
    \textbf{\hyperref[def:ProductTopology]{Product Topology}}
}
\newcommand{\ProductTopologies}[0]{
    \textbf{\hyperref[def:ProductTopology]{Product Topologies}}
}

\begin{df}[\ProductTopology]
    Let $A \neq \emptyset$. 
    For each $\alpha \in A$, 
    let $(X_\alpha, \T_\alpha)$ 
    be a 
    \TopologicalSpace.
    We call the \WeakTopology
    on 
    \scCartesianProduct{\alpha}{A}{X}
    induced by 
    $\{\pi_\alpha:\scCartesianProduct{\alpha}{A}{X} \to (X_\alpha, \T_\alpha)\}_{\alpha \in A}$
    the \ProductTopology. 
\end{df}

\subsection{Open and Closed Maps}
\subsection{Convergence of Filters}
\begin{rmk}

\rm
    By \ref{prop:FilterBaseFacts:FiltersAreFilterBases}, 
    a \Filter $\scF$ is a \FilterBase for itself, 
    and any \FilterBase is a \FilterSubbasis 
    for the \Filter it generates.
    For this reason, when stating definitions or results 
    about \Filters, 
    we will prefer to assume that an 
    object is a \FilterSubbasis if possible, 
    only resorting to assuming something is a 
    \FilterBase or \Filter when necessary, 
    and we will prefer to show that things
    are \Filters.
\end{rmk}

\newcommand{\FilterLimit}[0]{\textbf{\hyperref[def:FilterConvergence]{Limit}}\xspace}
\newcommand{\FilterLimits}[0]{\textbf{\hyperref[def:FilterConvergence]{Limits}}\xspace}
\newcommand{\FilterConvergent}[0]{\textbf{\hyperref[def:FilterConvergence]{Convergent}}\xspace}
\newcommand{\FilterConvergence}[0]{\textbf{\hyperref[def:FilterConvergence]{Convergence}}\xspace}
\newcommand{\FilterConverges}[0]{\textbf{\hyperref[def:FilterConvergence]{Converges}}\xspace}
\begin{df}[\FilterConvergence]
\label{def:FilterConvergence}
    Let $(X,\T)$ be a \TopologicalSpace. 
    Let $\scF$ be a \Filter on $X$. 
    Let $\scB$ be a \FilterBase for $\scF$. 
    Let \NeighborhoodFilterInstance{\T}(x) 
    denote the \NeighborhoodFilter of $\T$ at $x$. 
    Let \scF be \FinerFilter 
    than 
    \NeighborhoodFilterInstance{\T}(x).
    Then we say the following:
     $x$ is a \FilterLimit of $\scF$, 
     $x$ is a \FilterLimit of $\scB$, 
     $\scF$ \FilterConverges to $x$, 
     $\scB$ \FilterConverges to $x$, 
     $\scF$ is \FilterConvergent to $x$, 
     $\scB$ is \FilterConvergent to $x$, 
     $\scF$ posesses \FilterConvergence to $x$, and 
     $\scB$ posesses \FilterConvergence to $x$. 
\end{df}

\begin{prop}[\FilterConvergence Facts]
\label{prop:FilterConvergenceFacts}

\rm
    Let $(X,\scT)$ be a \TopologicalSpace. 
    Let $x \in X$. 
    Let \NeighborhoodFilterInstance{\scT}(x) denote the 
    \NeighborhoodFilter for $\scT$ at $x$. 
    Let $\scF$ be a \Filter on $X$. 
    Let $\scB$ be a \FilterBase for $\scF$. 
    The following are true. 
    \begin{enumerate}[label=(\roman*), ref={\ref{prop:FilterConvergenceFacts}.~\roman*}]
    \item 
    \label{prop:FilterConvergenceFacts:BaseEquivalence}
    $x$ is a \FilterLimit of $\scF$ if and only if $x$ is a \FilterLimit of $\scB$. 
    \item 
    \label{prop:FilterConvergenceFacts:FundamentalSystemExistence}
    $x$ is a \FilterLimit of $\scB$ if and only if, for some 
    \FundamentalSystemOfNeighborhoods $\scU$ of $x$, 
    \scNested{\scB}{\scU} holds. 
    \item 
    \label{prop:FilterConvergenceFacts:FundamentalSystemForAll}
    $x$ is a \FilterLimit of $\scB$ if and only if, 
    for every 
    \FundamentalSystemOfNeighborhoods $\scU$ of $x$, 
    \scNested{\scB}{\scU} holds. 
    \item 
    \label{prop:FilterConvergenceFacts:FinerFilter}
    If $x$ is a \FilterLimit of $\scF$ and $\scG$ is \FinerFilter
    than $\scF$ then $x$ is a \FilterLimit of $\scG$. 
    \item 
    \label{prop:FilterConvergenceFacts:CoarserTopology}
    If $x$ is a \FilterLimit of $\scF$ in $(X,\scT)$, and 
    $\scT_1$ is a \TopologyCoarser \Topology on $X$, then 
    $x$ is a \FilterLimit of $\scF$ in $(X, \scT_1)$. 
    \item 
    \label{prop:FilterConvergenceFacts:Intersection}
    Let $\{\scF_\alpha\}_{\alpha \in A}$ be a collection of 
    \Filters on $X$ each of which have $x$ as a \FilterLimit. 
    Then $\bigcap\limits_{\alpha \in A} \scF_\alpha$ has $x$
    as a \FilterLimit. 
    \item 
    \label{prop:FilterConvergenceFacts:UltrafilterCondition}
    $x$ is a \FilterLimit of $\scF$ if and only if 
    $x$ is a \FilterLimit of every $\UltraFilter$ 
    which is \FinerFilter than $\scF$. 
    \end{enumerate}
    \begin{proof}[Proof of \ref{prop:FilterConvergenceFacts:BaseEquivalence}]
       This is a direct result of 
       \ref{prop:FilterBaseFacts:FiltersAreFilterBases}. 
    \end{proof}
    \begin{proof}[Proof of \ref{prop:FilterConvergenceFacts:FundamentalSystemExistence}]
    $(\implies)$ 
    Let $x$ be a \FilterLimit for $\scB$.
    By 
    \ref{def:FilterConvergence}, 
    $\scU_{\scT}(x) \subset \scF$.
    Hence, 
    \scNested{\scF}{\scU_{\scT}(x)} holds.
    Since \Nested is 
    \TransitiveRelation 
    and 
    by 
    \ref{prop:FilterBaseFacts:BaseCondition}, 
    $\scNested{\scB}{\scU_{\scT}(x)}$ holds.
    Since $\scU_{\scT}(x)$ is a 
    \FundamentalSystemOfNeighborhoods for $x$, 
    this direction of the proof is complete. 

    $(\impliedby)$
    Let $\tilde{\scU}$ be a 
    \FundamentalSystemOfNeighborhoods
    for $x$
    such that $\scNested{\scB}{\tilde{\scU}}$ holds. 
    Then by \ref{def:FundamentalSystemOfNeighborhoods:Nesting}, 
    $\scNested{\tilde{\scU}}{\scU_{\scT}(x)}$.
    Since \Nested is 
    \TransitiveRelation 
    we conclude 
    $\scNested{\scB}{\scU_{\scT}(x)}$. 
    Since $\scB \subset \scF$, 
    $\scNested{\scF}{\scU_{\scT}(x)}$ holds. 
    By \ref{prop:FilterBaseFacts:FinerCondition}, 
    we conclude $\scU_{\scT}(x) \leq \scF$, 
    and so $x$ is a \FilterLimit of $\scB$  
    by \ref{def:FilterConvergence}.
    \end{proof}
    \begin{proof}[Proof of \ref{prop:FilterConvergenceFacts:FundamentalSystemForAll}]
    $(\impliedby)$ 
    This is a trivial consequence of the $(\impliedby)$ direction of
    \ref{prop:FilterConvergenceFacts:FundamentalSystemExistence}.
    
    $(\implies)$ 
    Let $x$ be a \FilterLimit of $\scB$. 
    Then $\scU_{\scT}(x) \subset \scF$. 
    Let $\scU$ be a 
    \FundamentalSystemOfNeighborhoods
    For $X$ at $x$. 
    By 
    \ref{def:FundamentalSystemOfNeighborhoods:AreNeighborhoods}, 
    $\scU \subset \scU_{\scT}(x)$, so 
    $\scU \subset \scU_{\scT}(x) \subset \scF$. 
    Hence $\scNested{\scF}{ \scU}$ holds. 
    Since $\scNested{\scB}{\scF}$ holds and since 
    \Nested is a \TransitiveRelation, 
    $\scNested{\scB}{\scU}$ holds. 
    \end{proof}
    \begin{proof}[Proof of \ref{prop:FilterConvergenceFacts:FinerFilter}]
    This is clear since \FilterFineness is a \PartialOrdering and is therefore \TransitiveRelation.
    \end{proof}
    \begin{proof}[Proof of \ref{prop:FilterConvergenceFacts:CoarserTopology}]
    If $\scT_1$ is \TopologyCoarser than $\scT$, then 
    $\scU_{\scT_1}(x) \subset \scU_{\scT}(x)$. 
    If $x$ is a \FilterLimit of $\scF$ in $(X, \scT)$, then 
    $\scU_{\scT}(x) \subset \scF$. 
    Hence, $\scU_{\scT_1}(x) \subset\scF$, so 
    $x$ is a \FilterLimit of $\scF$ in $(X,\scT_1)$. 
    \end{proof}
    \begin{proof}[Proof of \ref{prop:FilterConvergenceFacts:Intersection}]
    Let $U \in \scU_{\scT}(x)$. 
    Then for each $\alpha \in A$, $U \in \scF_{\alpha}$. 
    Hence $U \in \bigcap\limits_{\alpha \in A} \scF_{\alpha}$. 
    Hence 
    $\scU_{\scT}(x) \subset \bigcap\limits_{\alpha \in A} \scF_{\alpha}$. 
    so $x$ is a \FilterLimit of 
    $\bigcap\limits_{\alpha \in A} \scF_{\alpha}$.
    \end{proof}
    \begin{proof}[Proof of \ref{prop:FilterConvergenceFacts:UltrafilterCondition}]
    $(\implies)$ 
    This direction is a direct consequence of 
    \ref{prop:FilterConvergenceFacts:FinerFilter}.

    $(\impliedby)$
    By \ref{prop:UltraFilterFacts:UltrafilterIntersection}, 
    $\scF$ is the intersection of all \Ultrafilters on $X$ 
    which contain $\scF$. 
    By \ref{prop:FilterConvergenceFacts:Intersection}, if 
    these all have $x$ as a \FilterLimit, then $\scF$ must also
    have $x$ as a \FilterLimit.
    \end{proof}
\end{prop}

\newcommand{\FilterClusterPoint}[0]{\textbf{\hyperref[def:FilterClusterPoint]{Cluster Point}}\xspace}
\newcommand{\FilterClusterPoints}[0]{\textbf{\hyperref[def:FilterClusterPoint]{Cluster Points}}\xspace}
\begin{df}[\FilterClusterPoint]
\label{def:FilterClusterPoint}
    Let $(X,\scT)$ be a \TopologicalSpace. 
    Let $x \in X$. 
    Let $\scB$ be a \FilterBase in $X$. 
    We say that $x$ is a \FilterClusterPoint of $\scB$ if 
    \begin{equation}
        x \in \bigcap\limits_{U \in \scB} \ClosureMark{U}
    \end{equation}
\end{df}


\begin{prop}[\FilterClusterPoint]
\label{prop:FilterClusterPoint}

\rm
Let $(X,\scT)$ be a \TopologicalSpace.
Let $\scB$ and $\scD$ be \FilterBaseEquivalent \FilterBases. 
Let $\scF$ be the \Filter \FilterGeneratedBy $\scB$.
Let $\scE$ be a \FilterBase on $X$ which generates a \Filter 
which is \CoarserFilter than $\scF$. 
Let $\scG$ be an \UltraFilter in $X$. 
Let $x \in X$. 
The following are true. 
\begin{enumerate}[label=(\roman*), ref={\ref{prop:FilterClusterPoint}.~\roman*}]
\item 
\label{prop:FilterClusterPoint:Closed}
The set of $\FilterClusterPoints$ of $\scB$ is \SetClosed in $(X,\scT)$. 
\item
\label{prop:FilterClusterPoint:LimitIsClusterPoint}
If $x$ is a \FilterLimit of $\scB$, then $x$ is a \FilterClusterPoint of $\scB$. 
\item
\label{prop:FilterClusterPoint:CoarserFilter}
If $x$ is a \FilterClusterPoint of $\scB$, then 
$x$ is a \FilterClusterPoint of $\scE$. 
\item
\label{prop:FilterClusterPoint:Equivalent}
$x$ is a \FilterClusterPoint of $\scB$ 
if and only if 
$x$ is a \FilterClusterPoint of $\scD$.
\item 
\label{prop:FilterClusterPoint:Filter}
$x$ is a \FilterClusterPoint of $\scB$ 
if and only if 
$x$ is a \FilterClusterPoint of $\scF$, viewed as a \FilterBase.
\item 
\label{prop:FilterClusterPoint:FundamentalSystem}
Let $\scU$ be a \FundamentalSystemOfNeighborhoods for $\scT$ at $x$. 
Then $x$ is a \FilterClusterPoint for $\scB$ if and only if for each $U \in \scU$ and 
for each $B \in \scB$, $U \cap B\neq \emptyset$. 
\item 
\label{prop:FilterClusterPoint:Finer}
$x$ is a \FilterClusterPoint for $\scF$ if and only if there is a 
\Filter \FinerFilter than $\scF$ for which $x$ is a \FilterLimit. 
\item 
\label{prop:FilterClusterPoint:Ultrafilter}
$x$ is a \FilterClusterPoint of $\scG$ if and only if $x$ is a \FilterLimit of $\scG$. 
\item
\label{prop:FilterClusterPoint:CoarserTopology}
If $x$ is a \FilterClusterPoint for $\scB$ in $(X,\scT)$, 
and if $\scT_1$ is a \TopologyCoarser 
\Topology on $X$ than $\scT$, then 
$x$ is a \FilterClusterPoint for $\scB$ in $(X,\scT_1)$.
\end{enumerate}
\begin{proof}[Proof of \ref{prop:FilterClusterPoint:Closed}]
By \ref{def:FilterClusterPoint}, 
the set of \FilterClusterPoints of $\scB$ is an 
intersection of \SetClosed sets and is therfore itself \SetClosed.
\end{proof}
\begin{proof}[Proof of \ref{prop:FilterClusterPoint:CoarserFilter}]
Since $\scB$ generates a \Filter which is 
\FinerFilter than that generated by $\scE$, 
$\scNested{\scB}{\scE}$
holds. 
Let $x$ be a \FilterClusterPoint of $\scB$. 
Let $E \in \scE$. 
Then there exists $B \in \scB$ 
such that $B \subset E$. 
Since $x$ is a \FilterClusterPoint of $\scB$, 
$x \in \overline{B} \subset \overline{E}$. 
Since $E \in \scE$ was arbitrary, 
$x \in \bigcap\limits_{E \in \scE}\overline{E}$
and is therefore a \FilterClusterPoint of $\scE$. 
\end{proof}
\begin{proof}[Proof of \ref{prop:FilterClusterPoint:Equivalent}]
This is a direct result of two applications of \ref{prop:FilterClusterPoint:CoarserFilter}.
\end{proof}
\begin{proof}[Proof of \ref{prop:FilterClusterPoint:Filter}]
Since $\scF$, viewed as a \FilterBase
is \FilterBaseEquivalent to $\scB$, 
this result follows from a direct application of 
\ref{prop:FilterClusterPoint:Equivalent}.
\end{proof}
\begin{proof}[Proof of \ref{prop:FilterClusterPoint:FundamentalSystem}]
$(\implies)$
Let $U \in \scU$. 
Then $U \in \scU_{\scT}(x)$. 
Let $x$ be a \FilterClusterPoint for $\scB$. 
Let $B \in \scB$. 
Then $x \in \overline{B}$. 
Since $U \in \scU_{\scT}(x)$, $U \cap B \neq \emptyset$. 

$(\impliedby)$
Let $B \in \scB$. 
Let $U \in \scU_{\scT}(x)$. 
Since $\scU$ is a \FundamentalSystemOfNeighborhoods
for $X$ at $x$, there exists $V \in \scU$ with
$x \in V \subset U$. 
By assumption, $V \cap B \neq \emptyset$. 
Hence $U \cap B \neq \emptyset$. 
Since $U \in \scU_{\scT}(x)$ was arbitrary, 
$x \in \overline{B}$. 
Since $B \in \scB$ was arbitrary, 
$x$ is a \FilterClusterPoint of $x$. 
\end{proof}
\begin{proof}[Proof of \ref{prop:FilterClusterPoint:LimitIsClusterPoint}]
Let $x$ be a \FilterLimit of $\scB$. 
Let $B \in \scB$.
Let $U \in \scU_{\scT}(x)$. 
By \ref{prop:FilterConvergenceFacts:FundamentalSystemForAll}, 
there exists $V \in \scB$ such that $V \subset U$. 
By 
\ref{def:FilterBase:IntersectionProperty}, 
there exists a nonempty $W \in \scB$, such that $W \subset B \cap V$. 
Since $W \cap U \neq \emptyset$, $B \cap U \neq \emptyset$. 
Since $\scU_{\scT}(x)$ is a \FundamentalSystemOfNeighborhoods for $X$ at $x$,
since $U \in \scU_{\scT}(x)$ was arbitrary, 
and since $B \in \scB$ was arbitrary, 
we can apply
\ref{prop:FilterClusterPoint:FundamentalSystem} to conclude $x$ is a \FilterClusterPoint
of $\scB$. 


\end{proof}
\begin{proof}[Proof of \ref{prop:FilterClusterPoint:Finer}]
$(\implies)$
Define $A = \scF \cup \scU_{\scT}(x)$. 
Let $x$ be a \FilterClusterPoint of $\scF$. 
Then, since $\scU_{\scT}(x)$ is a 
\FundamentalSystemOfNeighborhoods for $X$ at $x$, 
by 
\ref{prop:FilterClusterPoint:FundamentalSystem}
and
\ref{prop:FiniteClosure:Intersection}, 
finite intersections
of elements of $A$  are nonempty. 
Hence, by 
\ref{prop:FilterExistence}, 
there is a \Filter $\scG$ on $X$ which contains $A$. 
By construction, 
$\scG$ is \FinerFilter than $\scF$
and $\scU_{\scT}(x) \leq \scG$, 
so $x$ is a \FilterLimit of $\scG$. 

$(\impliedby)$
Let $\scG$ be a \FinerFilter than $\scF$ 
and let $x$ be a \FilterLimit for $\scG$. 
By \ref{prop:FilterClusterPoint:LimitIsClusterPoint}, 
$x$ is a \FilterClusterPoint for $\scG$. 
By \ref{prop:FilterClusterPoint:CoarserFilter}, 
since $\scF$ is \CoarserFilter than $\scG$, 
$x$ is a \FilterClusterPoint of $\scF$. 
\end{proof}
\begin{proof}[Proof of \ref{prop:FilterClusterPoint:Ultrafilter}]
$(\impliedby)$
This direction is a direct consequence of 
\ref{prop:FilterClusterPoint:LimitIsClusterPoint}.

$(\implies)$
this direction is a direct consequence of 
\ref{prop:FilterClusterPoint:Finer}
combined with the fact that an \Ultrafilter 
is not properly contained in any \Filter.
\end{proof}
\begin{proof}[Proof of \ref{prop:FilterClusterPoint:CoarserTopology}]
If $\scT_1$ is \TopologyCoarser than $\scT$, than 
for any $B \in \scB$, $\overline{B}_{\scT} \subset \overline{B}_{\scT_1}$, implying
\begin{equation*}
\bigcap\limits_{B \in \scB} \overline{B}_{\scT} \subset \bigcap\limits_{B \in \scB} \overline{B}_{\scT_1}
\end{equation*}
\end{proof}
\end{prop}

\newcommand{\FunctionLimit}[0]{\textbf{\hyperref[def:FunctionLimitPoint]{Limit}}\xspace}
\newcommand{\FunctionLimits}[0]{\textbf{\hyperref[def:FunctionLimitPoint]{Limits}}\xspace}
\newcommand{\FunctionClusterPoint}[0]{\textbf{\hyperref[def:FunctionLimitPoint]{Cluster Point}}\xspace}
\newcommand{\FunctionClusterPoints}[0]{\textbf{\hyperref[def:FunctionLimitPoint]{Cluster Points}}\xspace}
\begin{df}[\FunctionLimit]
\rm
    \label{def:FunctionLimitPoint}
    Let $X$ be a nonempty set and $Y$ be a \TopologicalSpace.
    Let $\scB$ be a \FilterBase in $X$ and let 
    $f:X \to Y$. 
    Let $y \in Y$. 
    If $y$ is a \FilterLimit of $f(\scB)$, 
    then we say that 
    $y$ is a \FunctionLimit of $f$ with respect to $\scB$ 
    and we write $y \in \lim_{\scB} f$ or we may write
    $y \in \lim\limits_{x,\scB}f(x)$. 
    If $\lim_{\scB}f$ contains just a single point then we will, 
    as an abuse of notation, write
    $y=\lim_{\scB}f$ and $y =\lim\limits_{x, \scB} f(x)$. 
    If $y$ is a \FilterClusterPoint $f(\scB)$ 
    then we say that 
    $y$ is a \FunctionClusterPoint of $f$ with respect to $\scB$. 
\end{df}

\newcommand{\NetLimit}[0]{
    \textbf{\hyperref[def:NetLimitPoint]{Limit}}
}
\newcommand{\NetLimits}[0]{
    \textbf{\hyperref[def:NetLimitPoint]{Limits}}
}
\newcommand{\NetClusterPoint}[0]{
    \textbf{\hyperref[def:NetLimitPoint]{Cluster Point}}
}
\newcommand{\NetClusterPoints}[0]{
    \textbf{\hyperref[def:NetLimitPoint]{Cluster Points}}
}
\begin{df}[\NetLimit]
\label{def:NetLimitPoint}
\rm
    Let $(X,\scT)$ be a \TopologicalSpace. 
    Let $\sigma=\{x_\alpha\}_{\alpha \in A} \subset X$ be a \Net in $X$. 
    We say that $x \in X$ is a \NetLimit 
    of $\sigma$ if 
    $x$ is a \FunctionLimit of $\sigma$ with respect to the 
    \DirectedSectionFilter on $A$. 
    We say that $x \in X$ is a \NetClusterPoint 
    of $\sigma$ if 
    $x$ is a \FunctionClusterPoint of $\sigma$ with respect to the 
    \DirectedSectionFilter on $A$. 
\end{df}


\begin{prop}[\FunctionLimit]
\label{prop:FunctionLimit}

\rm
Let $X$ be a nonempty set.
Let $(Y,\scT)$ be a \TopologicalSpace. 
Let $f:X \to Y$. 
Let $\scF$ be a \Filter in $X$. 
Let $\scB$ be a \FilterBase for $\scF$. 
Let $y \in Y$. 
The following are true 
\begin{enumerate}[label=(\roman*), ref={\ref{prop:FunctionLimit}.~\roman*}]
    \item
    \label{prop:FunctionLimitCharacterization1}
    $y$ is a \FunctionLimit of $f$ with respect to $\scB$ if and only if,
    for each \Neighborhood $V$ of $y$, there exists $M \in \scB$ with 
    $f(M) \subset V$. 
    \item
    \label{prop:FunctionLimitCharacterization2}
    $y$ is a \FunctionLimit of $f$ with respect to $\scB$ 
    if and only if 
    for each \Neighborhood $V$ of $y$, $f^{-1}(V) \in \scF$. 
    \item 
    \label{prop:FunctionClusterPointCharacterization1}
    $y$ is a \FunctionClusterPoint of $f$ with respect to $\scB$ 
    if and only if
    for each \Neighborhood $V$ of $y$ and each
    $M \in \scB$, $f(M) \cap V \neq \emptyset$. 
    \item
    \label{prop:FunctionLimitCoarser}
    If $y$ is a \FunctionLimit of $f$ with respect to $\scB$
    in $(Y,\scT)$ and $\scT_1$ is a \TopologyCoarser 
    \Topology on $Y$ than $\scT$, then $y$ is a 
    \FunctionLimit of $f$ with respect to $\scB$ in $(Y, \scT_1)$. 
    \item
    \label{prop:FunctionClusterPointCoarser}
    If $y$ is a \FunctionClusterPoint of $f$ with respect to $\scB$
    in $(Y, \scT)$ and $\scT_1$ is a \TopologyCoarser
    \Topology on $Y$ than $\scT$, then $y$ is a 
    \FunctionClusterPoint of $f$ with respect to $\scB$ in $(Y, \scT_1)$. 
    \item 
    \label{prop:FunctionLimitFiner}
    If $y$ is a \FunctionLimit of $f$ with respect to $\scB$ and 
    $\scF_1$ is a \FinerFilter \Filter on $X$ than $\scF$, then 
    $y$ is a \FunctionLimit of $f$ with respect to $\scF_1$. 
    \item 
    \label{prop:FunctionClusterPointFiner}
    If $y$ is a \FunctionClusterPoint of $f$ with respect to $\scB$ and 
    $\scF_1$ is a \CoarserFilter \Filter on $X$ than $\scF$, then 
    $y$ is a \FunctionClusterPoint of $f$ with respect to $\scF_1$. 
    \item
    \label{prop:FunctionClusterPointCharacterization2}
    $y$ is a \FunctionClusterPoint of $f$ with repsect to $\scB$ if and only
    if there is a \Filter $\scG$ on $X$ which is \FinerFilter than $\scF$ such that 
    $y$ is a \FunctionLimit of $f$ with respect to $\scG$. 
    \item
    \label{prop:FunctionClusterPoints:Closed}
    The set of \FunctionClusterPoints of $f$ with respect to $\scB$ is \SetClosed in $Y$ and 
    may be empty. 
\end{enumerate}
\begin{proof}[Proof of \ref{prop:FunctionLimitCharacterization1}]
This is obvious from an application of 
\ref{def:FunctionLimitPoint}
and
\ref{prop:FilterConvergenceFacts:FundamentalSystemExistence}.
\end{proof}
\begin{proof}[Proof of \ref{prop:FunctionLimitCharacterization2}]
This is a direct application of \ref{prop:FunctionLimitCharacterization1}.
\end{proof}
\begin{proof}[Proof of \ref{prop:FunctionClusterPointCharacterization1}]
This result comes from an application of 
\ref{prop:FilterClusterPoint:FundamentalSystem}.
\end{proof}
\begin{proof}[Proof of \ref{prop:FunctionLimitCoarser}]
This is a direct consequence of 
\ref{prop:FilterConvergenceFacts:CoarserTopology}.
\end{proof}
\begin{proof}[Proof of \ref{prop:FunctionClusterPointCoarser}]
This is a direct consequence of 
\ref{prop:FilterClusterPoint:CoarserTopology}.
\end{proof}
\begin{proof}[Proof of \ref{prop:FunctionLimitFiner}]
This is a direct consequence of 
\ref{prop:FilterConvergenceFacts:FinerFilter}.
\end{proof}
\begin{proof}[Proof of \ref{prop:FunctionClusterPointFiner}]
This is a direct consequence of 
\ref{prop:FilterClusterPoint:CoarserFilter}.
\end{proof}
\begin{proof}[Proof of \ref{prop:FunctionClusterPointCharacterization2}]
This result falls from a direct application of 
\ref{prop:FilterClusterPoint:Finer}
paired with
\ref{prop:InverseFilterImage}.
\end{proof}
\begin{proof}[Proof of \ref{prop:FunctionClusterPoints:Closed}]
This is obvious.
\end{proof}
\end{prop}

\newcommand{\FunctionPointLimit}[0]{\textbf{\hyperref[def:FunctionPointLimit]{Limit}}\xspace}
\newcommand{\FunctionPointLimits}[0]{\textbf{\hyperref[def:FunctionPointLimit]{Limits}}\xspace}
\newcommand{\FunctionPointClusterPoint}[0]{\textbf{\hyperref[def:FunctionPointLimit]{Cluster Point}}\xspace}
\newcommand{\FunctionPointClusterPoints}[0]{\textbf{\hyperref[def:FunctionPointLimit]{Cluster Points}}\xspace}
\begin{df}[\FunctionPointLimit of a \Function at a point]
\label{def:FunctionPointLimit}
\rm
Let $(X,\T_X)$ be a \TopologicalSpace.
Let $(Y,\T_Y)$ be a \TopologicalSpace.
Let $f:X \to Y$. 
Let $a \in X$. 
Let $\scB$ be the \NeighborhoodFilter of $X$ at $a$. 
Let $y$ be a \FunctionLimit of $f$ with respect to $\scB$. 
Then instead of the standard notation
\begin{equation*}
y \in \lim\limits_{x, \scB}f(x)
\end{equation*}
we instead write
\begin{equation*}
y \in \lim\limits_{x \to a} f(x)
\end{equation*}
and we say that $y$ is a \FunctionPointLimit of $f$ at $a$. 
If $y$ is a \FunctionClusterPoint of $f$ with respect to $\scB$, then we say that
$y$ is a \FunctionPointClusterPoint of $f$ at $a$. 
\end{df}

\begin{prop}
\label{prop:FunctionLimitIffContinuous}

\rm
For $i \in \{0,1\}$, let $(X_i, \scT_i)$ be \TopologicalSpaces. 
Let $f:X_0 \to X_1$. 
Let $x_0 \in X_0$. 
The following are true. 
\begin{enumerate}[label=(\roman*), ref={\ref{prop:FunctionLimitIffContinuous}~\roman*}]
    \item 
    \label{prop:Continuity:ContinuousIffLimit}
    
    $f$ is 
    \ContinuousAt
    $x_0$ 
    if and only if  
    $f(x_0) \in \lim\limits_{x \to x_0} f(x)$. 
    \item 
    \label{prop:Continuity:FilterBaseConvergence}
    If $f$ is \ContinuousFunction at $x_0$, then 
    for every $\FilterBase$ \scB in $X$ which \FilterConverges 
    to $x_0$, we have $f(\scB)$ \FilterConverges to $f(x_0)$.
    \item  
    \label{prop:Continuity:UltrafilterConvergence}
    If, for every $\UltraFilter$ $\scU$ on $X$ which \FilterConverges to $a$
    , we have $f(\scU)$ converges to $f(a)$, then $f$ is \ContinuousFunction at $a$. 
    \item 
    \label{prop:Continuity:LimitComposition}
    Let $X_2$ be a nonempty set.
    Let $\scF$ be a \Filter on $X_2$. Let 
    $f_1:X_2 \to X_0$. 
    Let $x_0 \in \lim\limits_{x, \scF}g(x)$. 
    If $f$ is  
    \ContinuousAt
    $x_0$, then 
    $f(x_0) \in \lim\limits_{x, \scF} f \circ g(x)$. 
\end{enumerate}
\begin{proof}[Proof of \ref{prop:Continuity:ContinuousIffLimit}]
$(\implies)$
Let $f$ be \ContinuousFunction at $x_0$. 
Then $\scNested{\scU_{\scT_0}(x_0)}{f^{-1}\pa{\scU_{\scT_1}(f(x_0))}}$ holds.
Let $V \in \scU_{\scT_1}(f(x_0))$. 
Then there exists $U \in \scU_{\scT_0}(x_0)$ such that 
$U \subset f^{-1}(V)$. 
Hence $f(U) \subset V$. 
Since $\scU_{\scT_1}(f(x_0))$ was arbitrary, 
$\scNested{f\pa{\scU_{\scT_0}(x_0)}}{\scU_{\scT_1}(f(x_0))}$ holds, 
so $f(x_0) \in \lim\limits_{x \to x_0}f(x)$.

$(\impliedby)$
Let $f(x_0) \in \lim\limits_{x \to x_0}f(x_0)$. 
Then $\scNested{f\pa{\scU_{\scT_0}(x_0)}}{\scU_{\scT_1}(f(x_0))}$ holds.
Let $U \in f^{-1} \pa{\scU_{\scT_1}(f(x_0))}$. 
Then there is a $\tilde{U} \in \scU_{\scT_1}(f(x_0))$ such that 
$U = f^{-1}\pa{\tilde{U}}$. 
By nesting there exists $V \in f\pa{\scU_{\scT_0}(x_0)}$ such that 
$V \subset \tilde{U}$. 
By definition there exists $\tilde{V} \in \scU_{\scT_0}(x_0)$ 
such that $V= f\pa{\tilde{V}}$. 
Then we have $f\pa{\tilde{V}} = VB \subset \tilde{U}$, 
so 
$\tilde{V} \subset f^{-1}\pa{\tilde{U}} = U$.
Since $U \in f^{-1} \pa{\scU_{\scT_1}(f(x_0))}$
was arbitrary and $V \in \scU_{\scT_0}(x_0)$, we conclude
$\scNested{\scU_{\scT_0}(x_0)}{ f^{-1} \pa{\scU_{\scT_1}(f(x_0))}}$ holds.
Thus $f$ is \ContinuousFunction at $x_0$. 
\end{proof} 
\begin{proof}[Proof of \ref{prop:Continuity:FilterBaseConvergence}]
Let $f$ be \ContinuousFunction at $x_0$.
Then $\scNested{f\pa{\scU_{\scT_0}(x_0)}}{\scU_{\scT_1}(f(x_0))}$ holds.
Let $\scB$ be a \FilterBase in $X$. 
Let $x_0$ be a \FilterLimit of $\scB$. 
Then $\scNested{\scB}{\scU_{\scT_0}(x_0)}$ holds.
Hence $\scNested{f\pa{\scB}}{f\pa{\scU_{\scT_0}(x_0)}}$ holds. 
By \RelationTransitivity, we conclude
\begin{equation*}
\scNested{f\pa{\scB}}{\scU_{\scT_1}(f(x_0))}
\end{equation*}
so $x_0$ is a \FunctionLimit of $f\pa{\scB}$. 
\end{proof} 
\begin{proof}[Proof of \ref{prop:Continuity:UltrafilterConvergence}]
Suppose $f$ is not \ContinuousFunction at $x_0$. 
Then it is not the case that 
$\scNested{f\pa{\scU_{\scT_0}(x_0)}}{\scU_{\scT_1}(f(x_0))}$ holds.
Hence, for some $V \in \scU_{\scT_1}(f(x_0))$, 
and for each $U \in \scU_{\scT_0}(x_0)$, 
$f(U) \cap \pa{X_1 \setminus V} \neq \emptyset$. 
Hence, by 
\ref{prop:FilterExistence}, 
$\{f^{-1}\pa{X_1 \setminus V}\} \cup \scU_{\scT_0}(x_0)$ is
contianed within a \Filter \scG on $X_1$. 
Since $\scU_{\scT_00}(x_0) \subset \scG$, $x_0$ is a 
\FilterLimit of $\scG$. 
Furthermore, since $V \in \scU_{\scT_1}(f(x_0))$, 
and since $f\pa{f^{-1}\pa{X_1 \setminus V}} \in f\pa{\scG}$, 
by
\ref{prop:FilterClusterPoint:FundamentalSystem}, 
since $V \cap f\pa{f^{-1}\pa{X_1 \setminus V}} \subset V \cap \pa{X_1 \setminus V} = \emptyset$, we conclude
$f(x_0)$ is not a \FilterClusterPoint of $f\pa{\scG}$. 
By 
\ref{prop:FilterClusterPoint:Finer}, 
there is no \Filter 
on $X_1$ which is 
\FinerFilter
than $f(\scG)$ which has $f(x_0)$ as a \FilterLimit.
By \ref{def:Ultrafilter}, 
there is an \Ultrafilter $\tilde{\scG}$ on $X$ 
containing $\scG$. 
By 
\ref{prop:FilterConvergenceFacts:FinerFilter}, 
Since $x_0$ is a \FilterLimit of $\scG$, 
$x_0$ is a \FilterLimit of $\tilde{\scG}$. 
But by the above argumentation, $f(x_0)$ is not a 
\FilterLimit of $f\pa{\tilde{\scG}}$.
Thus we have, under the assumption that $f$ 
is not \ContinuousFunction at $x_0$, 
constructed an \Ultrafilter $\tilde{\scG}$
such that $x_0$ is a \FilterLimit of $\tilde{\scG}$
but $f(x_0)$ is not a \FilterLimit of 
$f\pa{\tilde{\scG}}$.
\end{proof} 
\begin{proof}[Proof of \ref{prop:Continuity:LimitComposition}]
This falls from several applications of 
\ref{prop:Continuity:FilterBaseConvergence}
Let $U \in \scU_{\scT_1}(f(x_0)$. 
Since $f$ is \ContinuousFunction at $x_0$, 
there exists $V \in \scU_{\scT_0}(x_0)$ such that 
$f(V) \subset U$. 
Since $x_0$ is a \FilterLimit of $g\pa{\scF}$, there exists 
$F \in \scF$ such that $g(F) \subset V$. 
Hence $\pa{f\circ g}(F) \subset U$. 
Since $U \in \scU_{\scT_1}(f(x_0))$ was arbitrary, we conclude
$\scNested{\pa{f \circ g}\pa{\scF}}{\scU_{\scT_1}\pa{f(x_0)}}$. 
Hence $f(x_0) \in \lim\limits_{x,\scF}\pa{f\circ g}\pa{x}$.
\end{proof} 
\end{prop}




\subsection{Separation Axioms}
\newcommand{\Hausdorff}[0]{\textbf{\hyperref[def:Hausdorff]{Hausdorff}}\xspace}
\newcommand{\PseudoHausdorff}[0]{\textbf{\hyperref[def:Hausdorff]{Pseudo-Hausdorff}}\xspace}
\newcommand{\Hausdorffness}[0]{\textbf{\hyperref[def:Hausdorff]{Hausdorffness}}\xspace}
\newcommand{\TTwo}[0]{\textbf{\hyperref[def:Hausdorff]{T2}}\xspace}
\begin{df}[Separation Axioms]
    \label{def:SeparationAxioms}

    \rm
    Let $(X,\scT)$ be a \TopologicalSpace. 
    We define the following. 
    \begin{enumerate}[label=(\roman*), ref={\ref{def:SeparationAxioms}~\roman*}]
    \item \label{def:Hausdorff} We say $X$ is \Hausdorff, or \TTwo if 
    distinct points in $X$ have \Disjoint \Neighborhoods.
    \item \label{def:PseudoHausdorff} We say that $X$ is \PseudoHausdorff
    if, for each $x_0,x_1 \in X$, either $x_0$ and $x_1$ have \Disjoint 
    \Neighborhoods or $\scU_{\scT}(x_0) = \scU_{\scT}(x_1)$. 
    \end{enumerate}
\end{df}

\begin{rmk}
\rm
It is clear that any \Hausdorff space is \PseudoHausdorff, and that 
a space is \PseudoHausdorff if and only if its quotient under the 
\RelationOfEqualNeighborhoodFilters is \Hausdorff.
For this reason, many of the desirable properties of \Hausdorff spaces
are inherited by those which are \PseudoHausdorff.
As we will see later,  topological groups are \PseudoHausdorff spaces.
\end{rmk}

\begin{prop}[\Hausdorff Characterizations]
    \label{prop:HausdorffCharacterizations}
    \rm
    Let $(X,\scT)$ be a \TopologicalSpace. 
    The following are equivalent.
    \begin{enumerate}[label=(\roman*), ref={\ref{prop:HausdorffCharacterizations}~\roman*}]
        \item
        \label{prop:HausdorffCharacterizations:Hausdorff}
        $X$ is \Hausdorff.
        \item
        \label{prop:HausdorffCharacterizations:ClosedNeighborhoodsConvergeToPoint}
        For all $x \in X$, if $\NeighborhoodFilterInstance{\scT}(x)$ is the 
        \NeighborhoodFilter of $X$ at $x$, then 
        \begin{equation*}
        \bigcap\limits_{U \in \NeighborhoodFilterInstance{\scT}(x)} \ClosureMark{U} = \{x\}
        \end{equation*}
        \item
        \label{prop:HausdorffCharacterizations:ClosedBinaryDiagonal}
        \scSetDiagonal{X} is \SetClosed in $X \times X$. 
        \item
        \label{prop:HausdorffCharacterizations:ClosedDiagonal}
        For any index set $A$, \scInfiniteSetDiagonal{A}{X} is \SetClosed in $\prod\limits_{\alpha \in A} X$.
        \item
        \label{prop:HausdorffCharacterizations:UniqueLimit}
        A \Filter $\scF$ in $X$ has at most one \FilterLimit.
        \item
        \label{prop:HausdorffCharacterizations:UniqueClusterPoint}
        If a \Filter $\scF$ in $X$ \FilterConverges, say $\scF \to x$, then 
        $x$ is that \Filter's only \FilterClusterPoint.
    \end{enumerate}
\end{prop}

\subsection{Compactness}
\begin{prop}
\label{prop:ClosedCompact}
\rm
Let $X$ be a \SetCompact \TopologicalSpace. 
Let $F \subset X$ be \SetClosed. 
Then $F$ is \SetCompact.
\begin{proof}
Let $\{U_{\alpha}\}_{\alpha \in A}$ 
be an \SetOpen \Cover for $F$.
Then $\{U_{\alpha}\}_{\alpha \in A} \cup \{X \setminus F\}$
is a \SetOpen \Cover for $X$. 
Since $X$ is \SetCompact
there is $\{\alpha_i\}_{i=1}^n \subset A$
such that $\{U_{\alpha_i}\}_{i=1}^n \cup \{X \setminus F\}$ 
is an \SetOpen \Cover for $X$. 
Since $\pa{X \setminus F} \cap F = \emptyset$, 
$\{U_{\alpha_i}\}_{i=1}^n$ is an \SetOpen \Cover
for $F$. 
Hence $F$ is \SetCompact.
\end{proof}
\end{prop}

\begin{prop}
\label{prop:CompactClosed}
\rm
Let $X$ be a \PseudoHausdorff \TopologicalSpace. 
Let $\cong$ denote the \RelationOfEqualNeighborhoodFilters 
on $X$. 
Let $\pi:X \to X/\cong$ denote the \QuotientMap. 
Let $K \subset X$ be a \SetCompact \Fiber under $\cong$. 
Then $K$ is \SetClosed. 
\begin{proof}
Let $x_0 \in X \setminus K$. 
Then since $K$ is a \Fiber.
$\pi(x_0) \in \pa{X / \cong} \setminus \pi(K)$. 
Since $X/\cong$ is \Hausdorff, for each $x \in K$, 
there is an $X/\cong$ \Neighborhood  $U_x$ of $\pi(x_0)$ and
and $X/\cong$ \Neighborhood $V_x$ of $x$ such that $U \cap V = \emptyset$. 
Then $\{V_x\}_{x \in \pi(K)}$ is an \SetOpen \Cover 
of $\pi(K)$. Since $\pi$ is continuous, $\pi(K)$ is \SetCompact. 
Hence, there is an $\{x_i\}_{i=1}^n \subset \pi(K)$ such that 
$K \subset \bigcup_{i=1}^n V_{x_i}$. 
Define $U = \bigcap_{i=1}^n U_{x_i}$. 
Then $\pi(x_0) \in U$
and $U \cap V = \emptyset$. 
Since $\pi(K) \subset V$, $U \cap \pi(K) = \emptyset$. 
Since $\pi(x_0) \in U$, $x_0 \in \pi^{-1}(U)$, and 
$K \cap \pi^{-1}(U) = \emptyset$, so $x \in X \setminus \overline{K}$. 
Hence $K=\overline{K}$, so $K$ is \SetCompact. 
\end{proof}
\end{prop}

\begin{cor}[Compact Closed]
\label{cor:CompactClosed}
\rm
A \SetCompact subset of a \Hausdorff space is \SetClosed. 
\begin{proof}
Since every subset of a \Hausdorff space is a \Fiber, we 
can apply
\ref{prop:CompactClosed} to get this result. 
\end{proof}
\end{cor}

\begin{prop}[Compact Contained]
\label{prop:CompactContained}
\rm
Let $X$ be a nonempty set.
Let $\scT_1$ and $\scT_2$ be
\Topologies on $X$. 
Let $\cong_i$ denote the \RelationOfEqualNeighborhoodFilters on $\scT_i$. 
Let $\cong_1=\cong_2$. 
Let $\scT_1$ be \PseudoHausdorff.
Let $\scT_2$ be \SetCompact.
Let $\scT_1 \subset \scT_2$. 
Then $\scT_1 = \scT_2$. 
\begin{proof}
Let $F \subset X$ be \SetClosed in $\scT_2$. 
By \ref{prop:ClosedCompact}, $F$, endowed with its $\scT_2$ subspace \Topology, 
is \SetCompact. 
Since $\scT_1 \subset \scT_2$, $F$, 
endowed with its $\scT_1$ subspace \Topology, is \SetCompact. 
Since $F$ is $\scT_2-\SetClosed$, by \ref{prop:QuotientSpaceTopology}, 
$F$ is a \Fiber under $\cong_2$. 
Hence $F$ is a \Fiber under $\cong_1$. 
Hence, we can apply \ref{prop:CompactClosed}
to claim that $F$ is \SetClosed in $\scT_1$. 
Since an arbitrary $\scT_2-\SetClosed$ set is 
$\scT_1-\SetClosed$, $\scT_2 \subset \scT_1$. 
By assumption, the reverse inclusion holds, so equality is verified. 
\end{proof}
\end{prop}

\begin{prop}
\label{prop:CompactClosure}
\rm
Let $X$ be a \PseudoHausdorff
\TopologicalSpace. 
Let $K \subset X$ be \SetCompact. 
Let $\cong$ denote the 
\RelationOfEqualNeighborhoodFilters in $X$. 
Then $\pi(K)$ is \SetClosed, 
$\overline{K} = \pi^{-1}\pa{\pi\pa{K}}$, 
and $\overline{K}$ is \SetCompact.
\begin{proof}
Since $K$ is \SetCompact, 
$\pi(K)$ is \SetCompact. 
Since $X/\cong$ is \Hausdorff, we can apply
\ref{cor:CompactClosed} to see that $\pi(K)$ is \SetClosed. 
Furthermore, by continuity of $\pi$, 
$\pi^{-1}\pa{\pi\pa{K}}$ is a \SetClosed set containing 
$K$.
Also, by 
\ref{prop:QST:ClosedSetFiber}, 
$\overline{K}$ is a \Fiber.
Thus, we have 
\begin{equation*}
\overline{K} \subset \pi^{-1}(\pi (K)) \subset \pi^{-1}\pa{\pi\pa{\overline{K}}} = \overline{K}
\end{equation*}
Hence $\overline{K} = \pi^{-1}(\pi(K))$. 

Finally, let $\{U_{\alpha}\}_{\alpha \in A}$ be a \SetOpen 
\Cover of $\overline{K}$. Then $\{\pi(U_{\alpha})\}_{\alpha \in A}$ is a
\SetOpen \Cover of $\pi(\overline{K}) = \pi(K)$.
Since $\pi(K)$ is \SetCompact, there is a 
finite $\{\alpha_i\}_{i=1}^n \subset A$ 
such that $\{\pi(V_{\alpha_i})\}_{i=1}^n$ covers $\pi(K)=\pi\pa{\overline{K}}$. 
Hence $\{V_{\alpha_i}\}_{i=1}^n$ covers $\overline{K}$ so $\overline{K}$ is \SetCompact.


\end{proof}
\end{prop}

\subsection{Countability  Axioms}
\label{def:FirstCountable}
\newcommand{\FirstCountable}[0]{\textbf{\hyperref[def:FirstCountable]{First Countable}}\xspace}
\newcommand{\FirstCountability}[0]{\textbf{\hyperref[def:FirstCountable]{First Countability}}\xspace}

\begin{df}[\FirstCountable]
\label{def:FirstCountable}
\rm
    Let $(x,\T)$ be a 
    \TopologicalSpace. 
    We say that 
    $X$
    is 
    \FirstCountable
    if for each $x \in X$, 
    there is a 
    \Countable
    \NeighborhoodBasis
    for 
    $\T$
    at
    $X$. 
\end{df}    

\label{def:SecondCountable}
\newcommand{\SecondCountable}[0]{
    \textbf{\hyperref[def:SecondCountable]{Second Countable}}
}
\newcommand{\SecondCountability}[0]{
    \textbf{\hyperref[def:SecondCountable]{Second Countability}}
}

\begin{df}[\SecondCountable]
    A \TopologicalSpace
    which permits a 
    \Countable
    \TopologyBasis
    is called 
    \SecondCountable
\end{df}

\label{def:TopologyDense}
\newcommand{\TopologyDense}[0]{
    \textbf{\hyperref[def:TopologyDense]{Dense}}
}
\newcommand{\TopologyDensity}[0]{
    \textbf{\hyperref[def:TopologyDense]{Density}}
}
\begin{df}[\TopologyDense]
    Let $(X,\T)$ be a 
    \TopologicalSpace
    and let $A \subset X$. 
    We say that $A$ is
    \TopologyDense
    in $X$ if 
    $\ClosureMark(A) = X$. 
\end{df}

\label{def:TopologySeparable}
\newcommand{\TopologySeparable}[0]{
    \textbf{\hyperref[def:TopologySeparable]{Separable}}
}
\newcommand{\TopologySeparability}[0]{
    \textbf{\hyperref[def:TopologySeparable]{Separability}}
}
\begin{df}[\TopologySeparable]
    We say that a 
    \TopologicalSpace
    which permits a 
    \Countable
    \TopologyDense
    subset is 
    \TopologySeparable. 
\end{df}

\label{def:TopologyLindelof}
\newcommand{\TopologyLindelof}[0]{
    \textbf{\hyperref[def:TopologyLindelof]{Lindelof}}
}
\begin{df}[\TopologyLindelof]
    A \TopologicalSpace 
    in which every
    \SetOpen
    \Cover
    permits a
    \Countable
    \Subcover
    is called a 
    \TopologyLindelof
    space.
\end{df}



\section{Algebraic Structures}
\subsection{Magma, Semigroup, Group}
%\label{def:AlgebraicDeclarations}

\begin{df}\bf REMOVE \rm
\end{df}

\newcommand{\CommutativeFunction}[0]{\textbf{\hyperref[def:Symmetricmap]{Commutative}}\xspace}
\newcommand{\FunctionCommutativity}[0]{\textbf{\hyperref[def:Symmetricmap]{Commutativity}}\xspace}
\begin{df}[\CommutativeFunction]
\label{def:Symmetricmap}
\rm
    Let X and Y be nonempty sets. 
    We say that a map 
    $f:X \times X \to Y$ is \CommutativeFunction
    if for each 
    $x_0,x_1 \in X$, 
    $f(x_0,x_1)=f(x_1,x_0)$.
\end{df} 

\newcommand{\BinaryOperation}[0]{\textbf{\hyperref[def:Operation]{Binary Operation}}\xspace}
\newcommand{\BinaryOperations}[0]{\textbf{\hyperref[def:Operation]{Binary Operations}}\xspace}
\newcommand{\UnaryOperation}[0]{\textbf{\hyperref[def:Operation]{Unary Operation}}\xspace}
\newcommand{\UnaryOperations}[0]{\textbf{\hyperref[def:Operation]{Unary Operations}}\xspace}
\newcommand{\Operation}[0]{\textbf{\hyperref[def:Operation]{Operation}}\xspace}
\newcommand{\Operations}[0]{\textbf{\hyperref[def:Operation]{Operations}}\xspace}

\begin{df}[\Operation, \UnaryOperation, \BinaryOperation]
\label{def:Operation}
\rm
    Let $X$ be a nonempty set.
    be a set. 
    Let $A$ be a nonempty set with
    $cardinality(A)=n \in \Z^+$.
    We call a mapping 
    \begin{equation*}
        T:\prod\limits_{\alpha \in A} X \to X
    \end{equation*}
    an 
    n-ary \Operation
    on $X$. 
    If $n=1$ 
    then we call $T$ a
    \UnaryOperation
    on 
    $X$. 
    If $n=2$, 
    then we call $T$ a 
    \BinaryOperation
    on $X$. 
    If $T$ is a 
    \BinaryOperation
    on $X$, 
    we sometimes use the notation
    \begin{equation*}
        xTy=T(x,y)
    \end{equation*}
\end{df}


\newcommand{\Magma}[0]{\textbf{\hyperref[def:Magma]{Magma}}\xspace}
\newcommand{\Magmas}[0]{\textbf{\hyperref[def:Magma]{Magmas}}\xspace}
\newcommand{\CommutativeMagma}[0]{\textbf{\hyperref[def:Magma]{Commutative Magma}}\xspace}
\newcommand{\CommutativeMagmas}[0]{\textbf{\hyperref[def:Magma]{Commutative Magmas}}\xspace}

\begin{df}[\Magma]
\label{def:Magma}
\rm
    Let $X$ be a nonempty set.
    Let $T:X \times X \to X$ be a 
    \BinaryOperation
    on $X$. 
    We call the pair $(X,T)$ a 
    \Magma.
    When $T$ is clear, we call $X$ a 
    \Magma.
	If 
	$T$
	is
	\CommutativeFunction, 
	then we call 
	$(X,T)$
	a \CommutativeMagma.
	In general, this naming convention is used 
	for any algebraic structure defined on a set 
	via a \BinaryOperation with
	particular properties. 
	Now let $(M,+)$ be a \Magma, let $x \in M$, and let 
	$A,B \subset M$. Let $\scU,\scV \subset 2^M$.
	We define the following:
	\begin{equation*}
	x+A = \{x+y : y \in A\}
	\end{equation*}
    \begin{equation*}
    A+x = \{y+x : y \in A\}
    \end{equation*}
	\begin{equation*}
	A+B = \{a+b : a \in A \tab[.5cm] and \tab[.5cm] b \in B\}
	\end{equation*}
    \begin{equation*}
    \scU+\scV = \{U+V : (U \in \scU)(V \in \scV)\}
    \end{equation*}
    \begin{equation*}
    \scU + A = \{U+A : U \in \scU\}
    \end{equation*}
    \begin{equation*}
    A+\scU = \{A +U : U \in \scU\}
    \end{equation*}
    \begin{equation*}
    \scU+x=\{U+x : U \in \scU\}
    \end{equation*}
    \begin{equation*}
    x+\scU = \{x+U : U \in \scU\}
    \end{equation*}
\end{df}

\begin{prop}[\MagmaHomomorphism]
\label{prop:MagmaHomomorphism}
\rm
Let $M_1$ and $M_2$ be \Magmas.
Let $T:M_1 \to M_2$ be a 
\MagmaHomomorphism.
Define $\cong \subset M_1 \times M_1$ by $x \cong y$ if and only if $T(x)=T(y)$.
The following are true.
\begin{enumerate}[label=(\roman*), ref={\ref{prop:MagmaHomomorphism}~\roman*}]
\item
\label{prop:MagmaHomomorphism:Compatible}
$\cong$ is a \Congruence on $M_1$. 
\item
\label{prop:MagmaHomomorphism:Quotient}
Let $Q:M_1 \to M_1/\cong$ denote the \QuotientMap.
Define $\tilde{T}:M_1/\cong \to Range(T)$ by setting, for each $x \in M_1/\cong$, 
$\tilde{T}\pa{[x]} = T(x)$. 
Then $\tilde{T}$ is a well defined \Bijective \MagmaHomomorphism, $T = \tilde{T} \circ Q$, and $Q = \tilde{T}^{-1} \circ T$. 
\end{enumerate}
\begin{proof}[Proof of \ref{prop:MagmaHomomorphism:Compatible}]
It is obvious that $T$ is an \EquivalenceRelation.
What remains to show is that $\cong$ is 
\AlgebraicallyConsistent with $M$. 
Suppose $x_0 \cong x_1$ and $y_0 \cong y_1$. 
Then since $T$ is a \MagmaHomomorphism, 
\begin{equation*}
T(x_0y_0) = T(x_0) T(y_0) = T(x_1) T(y_1) = T(x_1y_1)
\end{equation*}
Hence $x_0y_0 \cong x_1y_1$.
\end{proof}
\begin{proof}[Proof of \ref{prop:MagmaHomomorphism:Quotient}]
We first show that $T$ is well defined. 
Let $[x] = [y]$. We must show that $\tilde{T}([x]) = \tilde{T}([y])$. 
Since $[x] = [y]$, $x \cong y$. Hence, $T(x) = T(y)$, 
which gives us 
$\tilde{T}([x]) = T(x) = T(y) = \tilde{T} ([y])$. 

Suppose $\tilde{T}\pa{[x]} = \tilde{T}\pa{[y]}$. 
Then $T(x) = T(y)$, so $x \cong y$, so $[x] = [y]$. 
Hence $\tilde{T}$ is \Injective. 

Let $y \in Range(T)$. Then there exists $x \in M_1$ such that 
$T(x) = y$. Hence $\tilde{T}\pa{[x]} = T(x) = y$. 
Thus, $\tilde{T}$ is \Surjective. 

Now let $x,y \in M$. 
Then, $\tilde{T}\pa{[x][y]} = \tilde{T}\pa{[xy]} = T(xy) = T(x)T(y) = \tilde{T}([x]) \tilde{T}([y])$. 
Thus $\tilde{T}$ is a \MagmaHomomorphism. 

Finally, if $x \in M_1$, then $\tilde{T} \circ Q(x) = \tilde{T}([x]) = T(x)$, so 
$T = \tilde{T} \circ Q$.
The final equation falls from the \Bijectivity of $\tilde{T}$. 
\end{proof}
\end{prop}

\newcommand{\IdentityElement}[0]{\textbf{\hyperref[def:IdentityElement]{Identity Element}}\xspace}
\newcommand{\IdentityElements}[0]{\textbf{\hyperref[def:IdentityElement]{Identity Elements}}\xspace}
\newcommand{\LeftIdentityElement}[0]{\textbf{\hyperref[def:IdentityElement]{Left Identity Element}}\xspace}
\newcommand{\LeftIdentityElements}[0]{\textbf{\hyperref[def:IdentityElement]{Left Identity Elements}}\xspace}
\newcommand{\RightIdentityElement}[0]{\textbf{\hyperref[def:IdentityElement]{Right Identity Element}}\xspace}
\newcommand{\RightIdentityElements}[0]{\textbf{\hyperref[def:IdentityElement]{Right Identity Elements}}\xspace}

\begin{df}[\LeftIdentityElement, \RightIdentityElement]
\label{def:IdentityElement}
\rm
    Let $(X,L)$ and
    $(X,R)$ be 
    \Magmas.
    Let $l , r \in X$ 
    such that
    for every $x \in X$ 
    we have 
   \begin{align*}
        lLx=x\\
        xRr=x
   \end{align*}
   In such a scenario, we say that
   $l$ is a \LeftIdentityElement 
   of $(X,L)$, and
   we say that 
   $r$ is a 
   \RightIdentityElement
   of $(X,R)$. 
\end{df}

\begin{df}[\IdentityElement]
\rm
    Let $(X,\oplus)$ be a 
    \Magma. 
    Let $e \in X$ be both a 
    \LeftIdentityElement 
    and a 
    \RightIdentityElement 
    of $\oplus$. 
    Then, we say that
    $e$ is an \IdentityElement of 
    $(X,\oplus)$. 
\end{df}



\newcommand{\UnitalMagma}[0]{\textbf{\hyperref[def:UnitalMagma]{Unital Magma}}\xspace}
\newcommand{\UnitalMagmas}[0]{\textbf{\hyperref[def:UnitalMagma]{Unital Magmas}}\xspace}
\newcommand{\CommutativeUnitalMagma}[0]{\textbf{\hyperref[def:UnitalMagma]{Commutative Unital Magma}}\xspace}
\newcommand{\CommutativeUnitalMagmas}[0]{\textbf{\hyperref[dsef:UnitalMagma]{Commutative Unital Magmas}}\xspace}
\begin{df}[\UnitalMagma]
\label{def:UnitalMagma}
\rm
    Let $(X,\oplus)$ be a
    \Magma with \IdentityElement $e$. 
    Then we call 
    $(X,\oplus,e)$ a
    \UnitalMagma.
\end{df}

\newcommand{\UnitalMagmaHomomorphism}[0]{\textbf{\hyperref[def:UnitalMagmaHomomorphism]{Unital Magma Homomorphism}}\xspace}
\newcommand{\UnitalMagmaHomomorphisms}[0]{\textbf{\hyperref[def:UnitalMagmaHomomorphism]{Unital Magma Homomorphisms}}\xspace}
\newcommand{\HomoKernel}[0]{\textbf{\hyperref[def:UnitalMagmaHomomorphism]{Kernel}}\xspace}
\newcommand{\scUnitalMagma}[0]{\textbf{\hyperref[def:UnitalMagmaHomomorphism]{UMagma}}\xspace}

\begin{df}[\UnitalMagmaHomomorphism]
\label{def:UnitalMagmaHomomorphism}
\rm
    Let $(X, \oplus_X, e_X)$ and $(Y, \oplus_Y, e_Y)$ be 
    \UnitalMagmas and 
    $T:X \to Y$ be a \MagmaHomomorphism such that
    $T(e_X)=e_Y$. 
    Then we call $T$ a 
    \UnitalMagmaHomomorphism.
    We represent the set of 
    \UnitalMagmaHomomorphisms
    between $X$ and $Y$ with 
    $H_{\scUnitalMagma}(X, Y)$. 
    If $T$ is a \UnitalMagmaHomomorphism, 
    then we call $T^{-1}(e_Y)$ the 
    \HomoKernel of $T$. 
    We denote the \HomoKernel of $T$ with 
    $Kernel(T)$. 
\end{df}


\newcommand{\InverseElement}[0]{\textbf{\hyperref[def:InverseElement]{Inverse}}\xspace}
\newcommand{\InvertibleElement}[0]{\textbf{\hyperref[def:InverseElement]{Invertible}}\xspace}
\newcommand{\InverseElements}[0]{\textbf{\hyperref[def:InverseElement]{Inverses}}\xspace}
\newcommand{\LeftInverseElement}[0]{\textbf{\hyperref[def:InverseElement]{Left Inverse}}\xspace}
\newcommand{\LeftInvertibleElement}[0]{\textbf{\hyperref[def:InverseElement]{Left Invertible}}\xspace}
\newcommand{\LeftInverseElements}[0]{\textbf{\hyperref[def:InverseElement]{Left Inverses}}\xspace}
\newcommand{\RightInverseElement}[0]{\textbf{\hyperref[def:InverseElement]{Right Inverse}}\xspace}
\newcommand{\RightInvertibleElement}[0]{\textbf{\hyperref[def:InverseElement]{Right Invertible}}\xspace}
\newcommand{\RightInverseElements}[0]{\textbf{\hyperref[def:InverseElement]{Right Inverses}}\xspace}

\begin{df}[\LeftInverseElement, \RightInverseElement]
\label{def:InverseElement}
\rm
    Let $(X,\oplus,e)$ be a 
    \UnitalMagma.
    Let $l,r \in X$ such that 
    \begin{equation*}
        l \oplus r=e
    \end{equation*}
    Then we say that 
    $l$ is a 
    \LeftInverseElement 
    of $r$  
    in
    $(X,\oplus,e)$
    and we say that 
    $r$
    is a 
    \RightInverseElement
    of $l$ 
    in
    $(x,\oplus,e)$.
    Furthermore, we say that 
    $r$ is 
    \LeftInvertibleElement
    in
    $(X,\oplus,e)$
    and that 
    $l$ is 
    \RightInvertibleElement 
    in
    $(X,\oplus,e)$.
\end{df}

\begin{df}[\InverseElement]
\rm
    Let $(X,\oplus,e)$ be a
    \UnitalMagma. 
    Let $x,y \in X$ such that
    $x$ is a \LeftInverseElement
    of $y$
    and $x$
    is a 
    \RightInverseElement
    of $y$. 
    Then, we say that 
    $x$ is an 
    \InverseElement
    of $y$
    in 
    $(X,\oplus, e)$,
    we say 
    $y$ an 
    \InvertibleElement
    element
    of
    $(X,\oplus,e)$, 
    and we write $x=y^{-1}$. 
\end{df}



\newcommand{\AssociativeFunction}[0]{\textbf{\hyperref[def:AssociativeFunction]{Associative}}\xspace}
\newcommand{\AssociativeOperation}[0]{\textbf{\hyperref[def:AssociativeFunction]{Associative}}\xspace}
\newcommand{\FunctionAssociativity}[0]{\textbf{\hyperref[def:AssociativeFunction]{Associativity}}\xspace}
\newcommand{\OperationAssociativity}[0]{\textbf{\hyperref[def:AssociativeFunction]{Associativity}}\xspace}
\begin{df}[\AssociativeOperation]
\label{def:AssociativeFunction}
\rm
	Let $T$ be a 
	\BinaryOperation
	on a set $X$.
    We say that T is 
    \AssociativeFunction 
    and we say that T posses
    \FunctionAssociativity
    if for each $x,y,z \in X$, we have 
    \begin{equation*}
        T\pa{x,T\pa{y,z}}=T\pa{T\pa{x,y},z}
    \end{equation*}
\end{df}

\newcommand{\Semigroup}[0]{\textbf{\hyperref[def:Semigroup]{Semigroup}}\xspace}
\newcommand{\Semigroups}[0]{\textbf{\hyperref[def:Semigroup]{Semigroups}}\xspace}
\newcommand{\CommutativeSemigroup}[0]{\textbf{\hyperref[def:Semigroup]{Commutative Semigroup}}\xspace}
\newcommand{\CommutativeSemigroups}[0]{\textbf{\hyperref[def:Semigroup]{Commutative Semigroups}}\xspace}

\begin{df}[\Semigroup]
\label{def:Semigroup}
\rm
    An \AssociativeFunction \Magma
    is called a \Semigroup.
\end{df}

\newcommand{\Monoid}[0]{\textbf{\hyperref[def:Monoid]{Monoid}}\xspace}
\newcommand{\Monoids}[0]{\textbf{\hyperref[def:Monoid]{Monoids}}\xspace}
\newcommand{\CommutativeMonoid}[0]{\textbf{\hyperref[def:Monoid]{Commutative Monoid}}\xspace}
\newcommand{\CommutativeMonoids}[0]{\textbf{\hyperref[def:Monoid]{Commutative Monoids}}\xspace}

\begin{df}[\Monoid]
\label{def:Monoid}
\rm
    A \Monoid
    is a \UnitalMagma
    which is also a \Semigroup.
\end{df}

\newcommand{\AlgebraicallyConsistent}[0]{\textbf{\hyperref[def:AlgebraicConsistentRelation]{Consistent}}\xspace}
\newcommand{\AlgebraicConsistency}[0]{\textbf{\hyperref[def:AlgebraicConsistentRelation]{Consistency}}\xspace}
\newcommand{\Congruence}[0]{\textbf{\hyperref[def:AlgebraicConsistentRelation]{Congruence}}\xspace}
\newcommand{\Congruences}[0]{\textbf{\hyperref[def:AlgebraicConsistentRelation]{Congruences}}\xspace}

\begin{df}[\AlgebraicallyConsistent]
\label{def:AlgebraicConsistentRelation}
\rm
    Let $(X,\oplus)$ be a 
    \Magma
    and let 
    $R$ be a 
	\Relation 
	on 
    $X$
    such that
    for each 
	$\{x_0, x_1,y_0,y_1\} \subset X$, 
    if 
	$x_0Rx_1$ and
    $y_0R y_1$, then
    \begin{equation*}
    \pa{x_0\oplus y_0}R \pa{x_1 \oplus y_1}
    \end{equation*}
    Then we say that 
    $R$ is
    \AlgebraicallyConsistent
    with $(X,\oplus)$, 
    and we say that
    $R$
    posesses 
    \AlgebraicConsistency
    with respect to 
    $(X,\oplus)$. 
    A $\Congruence$ on $M$ is an
    \EquivalenceRelation which is \AlgebraicallyConsistent
    with $M$. 
\end{df}



\label{def:OrderedMagma}
\newcommand{\PartiallyOrderedMagma}[0]{
    \bf \hyperref[def:OrderedMagma]{Partially Ordered Magma} \rm
}
\newcommand{\TotallyOrderedMagma}[0]{
    \bf \hyperref[def:OrderedMagma]{Totally Ordered Magma} \rm
}
\newcommand{\DirectedMagma}[0]{
    \bf \hyperref[def:OrderedMagma]{Directed Magma} \rm
}
\newcommand{\PartiallyOrderedMagmas}[0]{
    \bf \hyperref[def:OrderedMagma]{Partially Ordered Magmas} \rm
}
\newcommand{\TotallyOrderedMagmas}[0]{
    \bf \hyperref[def:OrderedMagma]{Totally Ordered Magmas} \rm
}
\newcommand{\DirectedMagmas}[0]{
    \bf \hyperref[def:OrderedMagma]{Directed Magmas} \rm
}

\begin{df}[\PartiallyOrderedMagma, \TotallyOrderedMagma, \DirectedMagma]
    Let $(X,\oplus)$ 
    be a 
    \Magma.
    Let $T$ be a 
    \TotalOrdering 
    on $X$ 
    which is 
    \AlgebraicallyConsistent
    with $(X,\oplus)$. 
    Let $P$ be a 
    \PartialOrdering 
    on $X$
    which is 
    \AlgebraicallyConsistent
    with $(X,\oplus)$. 
    Let $D$ be a 
    \Directing 
    on $X$
    which is 
    \AlgebraicallyConsistent
    with $(X,\oplus)$. 
    We call 
    $(X,\oplus, T)$ a 
    \TotallyOrderedMagma.
    $(X,\oplus, P)$ a 
    \PartiallyOrderedMagma.
    $(X,\oplus, D)$ a 
    \DirectedMagma.
\end{df}

\newcommand{\Group}[0]{\textbf{\hyperref[def:Group]{Group}}\xspace}
\newcommand{\Groups}[0]{\textbf{\hyperref[def:Group]{Groups}}\xspace}
\newcommand{\CommutativeGroup}[0]{\textbf{\hyperref[def:Group]{Commutative Group}}\xspace}
\newcommand{\CommutativeGroups}[0]{\textbf{\hyperref[def:Group]{Commutative Groups}}\xspace}
\newcommand{\AbelianGroup}[0]{\textbf{\hyperref[def:Group]{Abelian Group}}\xspace}
\newcommand{\AbelianGroups}[0]{\textbf{\hyperref[def:Group]{Abelian Groups}}\xspace}
\begin{df}[\Group]
\label{def:Group}
\rm
    Let $(X,\oplus,e)$ 
    be a \Monoid
    such that
    each 
    $x \in X$
    is an
    \InvertibleElement.
    Then we call
    $(X,\oplus,e)$ a 
    \Group. 
	Out of respect, we call a 
	\CommutativeGroup
	an
	\AbelianGroup.
\end{df}


\label{def:GroupInverseOperator}
\newcommand{\GroupInverseOperator}[0]{
    \bf \hyperref[def:GroupInverseOperator]{Group Inverse Operator} \rm
}
\newcommand{\GroupInverseOperators}[0]{
    \bf \hyperref[def:GroupInverseOperator]{Group Inverse Operators} \rm
}

\newcommand{\scGroupInverseOperator}[0]{
    \bf \hyperref[def:GroupInverseOperator]{T^{-1}} \rm
}

\begin{df}[\GroupInverseOperator]
    Let $(X, \oplus, e)$ be a group.
    We denote with
    $\scGroupInverseOperator_G$ the 
    function defined as follows:
    $\scGroupInverseOperator_G:X \to X$, 
    \begin{equation*}
        \scGroupInverseOperator_G(x)=-x
    \end{equation*}

    We call 
    $\scGroupInverseOperator_G$
    the 
    \GroupInverseOperator
    of 
    $(X,\oplus,e)$. 
\end{df}


\label{def:TranslationOperator}
\newcommand{\RightTranslation}[0]{\textbf{\hyperref[def:TranslationOperator]{Right Translation}}\xspace}
\newcommand{\LeftTranslation}[0]{\textbf{\hyperref[def:TranslationOperator]{Left Translation}}\xspace}
\newcommand{\Translation}[0]{\textbf{\hyperref[def:TranslationOperator]{Translation}}\xspace}
\newcommand{\scRightTranslationOperator}[0] {\textbf{\hyperref[def:TranslationOperator]{T^R}\xspace}
\newcommand{\scLeftTranslationOperator}[0] {\textbf{\hyperref[def:TranslationOperator]{T^L}}\xspace}
\newcommand{\scTranslationOperator}[0] {\textbf{\hyperref[def:TranslationOperator]{T}}\xspace}
\newcommand{\RightTranslationOperator}[0] {\textbf{\hyperref[def:TranslationOperator]{Right Translation Operator}\xspace}
\newcommand{\LeftTranslationOperator}[0] {\textbf{\hyperref[def:TranslationOperator]{Left Translation Operator}}\xspace}
\newcommand{\TranslationOperator}[0] {\textbf{\hyperref[def:TranslationOperator]{Translation Operator}}\xspace}
\newcommand{\RightTranslationOperators}[0] {\textbf{\hyperref[def:TranslationOperator]{Right Translation Operators}\xspace}
\newcommand{\LeftTranslationOperators}[0] {\textbf{\hyperref[def:TranslationOperator]{Left Translation Operators}}\xspace}
\newcommand{\TranslationOperators}[0] {\textbf{\hyperref[def:TranslationOperator]{Translation Operators}}\xspace}
%
%%%Vector Space Version
%
%\begin{df}[\TranslationOperator]
%    Let $V$ be a 
%    \VectorSpace
%    over a 
%    \Field $\F$. 
%    Let $\alpha \in \F$. 
%    We define $T_\alpha:V \to V$ by 
%    setting, for each 
%    $x \in V$, 
%    \begin{equation}
%    T_\alpha(x)=\alpha+x
%    \end{equation}
%    We call $T_\alpha$ 
%    the 
%    \TranslationOperator
%\end{df}


%
%%
%%%Magma Version
%%
%
\begin{df}[\TranslationOperator]
    Let $(G,\oplus)$ be a 
    \Magma. 
    Let $g \in G$. 
    We define 
    $\scRightTranslationOperator_g:G \to G$
    and
    $\scLeftTranslationOperator_g:G \to G$
    by setting, for each 
    $x \in G$, 
    \begin{equation*}
        \scRightTranslationOperator_g (x) = x \oplus g
    \end{equation*}
    \begin{equation*}
        \scLeftTranslationOperator_g (x) = g \oplus x
    \end{equation*}
    We call $\scRightTranslationOperator_g$
    the \RightTranslation
    of $(G,\oplus)$
    by $g$, 
    and we call 
    $\scLeftTranslationOperator_g$
    the \LeftTranslation
    of 
    $(G, \oplus)$ 
    by $g$. 
    If $\oplus$ is 
    \CommutativeFunction,
    then we define 
    $\scTranslationOperator_g=\scRightTranslationOperator_g=\scLeftTranslationOperator_g$
    which we call 
    \Translation of $(G, \oplus)$ by $g$.
\end{df}

\newcommand{\SymmetricSubset}[0]{\textbf{\hyperref[def:SymmetricSubset]{Symmetric}}\xspace}
\begin{df}[\SymmetricSubset]
\label{def:SymmetricSubset}
\rm
Let $G$ be a group and $A \subset G$. 
We define 
\begin{equation*}
A^{-1} = \{a^{-1} : a \in A \}
\end{equation*}
If $A = A^{-1}$ then we say that $A$ is \SymmetricSubset. 

\end{df}


\subsection{Vector Space}
\newcommand{\VectorSpace}[0]{ \textbf{\hyperref[def:vectorspace]{Vector Space}}\xspace}
\newcommand{\VectorSpaces}[0]{ \textbf{\hyperref[def:vectorspace]{Vector Spaces}}\xspace}
\newcommand{\Field}[0]{ \textbf{\hyperref[def:vectorspace]{Field}}\xspace}
\newcommand{\Fields}[0]{ \textbf{\hyperref[def:vectorspace]{Fields}}\xspace}
\newcommand{\VectorSubspace}[0]{ \textbf{\hyperref[def:vectorspace]{Vector Subspace}}\xspace}
\newcommand{\VectorSubspaces}[0]{ \textbf{\hyperref[def:vectorspace]{Vector Subspaces}}\xspace}
\newcommand{\StandardBasis}[0]{ \textbf{\hyperref[def:vectorspace]{Standard Basis}}\xspace}
\newcommand{\StandardBases}[0]{ \textbf{\hyperref[def:vectorspace]{Standard Bases}}\xspace}
\label{def:vectorspace}

\newcommand{\LinearlyIndependent}[0]{\textbf{\hyperref[def:LinearlyIndependent]{Linearly Independent}}\xspace}
\newcommand{\LinearIndependence}[0]{\textbf{\hyperref[def:LinearlyIndependent]{Linear Independence}}\xspace}
\newcommand{\LinearlyDependent}[0]{\textbf{\hyperref[def:LinearlyIndependent]{Linearly Dependent}}\xspace}
\newcommand{\LinearDependence}[0]{\textbf{\hyperref[def:LinearlyIndependent]{Linear Dependence}}\xspace}

\begin{df}[\LinearlyIndependent]
    \label{def:LinearlyIndependent}
    \rm
    Let $V$ be a \VectorSpace over a 
    \Field $\scK$.
    Let $A \subset V$. 
    We say that $A$ is \LinearlyIndependent 
    if, 
    for any finite subset 
    $\{x_i\}_{i=1}^n \subset A$, 
    the only solution to 
    \begin{equation*}
    \sum\limits_{i=1}^n \beta_i x_i = 0 \tab[2cm] \{\beta_i\}_{i=1}^n \subset \scK
    \end{equation*}
    is $\beta_i = 0$ for $1 \leq i \leq n$.
    In such a scenario, we may also say that $A$ 
    posesses \LinearIndependence. 
    A subset of a \VectorSpace which is not 
    \LinearlyIndependent is said to be 
    \LinearlyDependent and to posess
    \LinearDependence. 
\end{df}

\newcommand{\ScalarHomogeneous}[0]{
    \bf \hyperref[def:scalarhomogeneous]{Scalar Homogeneous} \rm
}

\newcommand{\ScalarHomogeneity}[0]{
    \bf \hyperref[def:scalarhomogeneous]{Scalar Homogeneity} \rm
}
\begin{df}[\ScalarHomogeneous]
\label{def:scalarhomogeneous}
\rm
    Let V be a 
	\VectorSpace over a 
	\Field	$\F \in \{\R, \C\}$. 
    We say that a map $p:V \to V$ is \ScalarHomogeneous if,
    for each $\alpha \in \F$ and each $x \in V$,
    \begin{equation*}
        p(\alpha x) = \alpha p(x)
    \end{equation*}
    Under these circumstances, we may instead say that the operator 
    p posesses \ScalarHomogeneity.
\end{df}

\newcommand{\AbsScalarHomogeneous}[0]{
    \bf \hyperref[def:absolutevaluescalarhomogeneous]{Absolutely Scalar Homogeneous} \rm
}

\newcommand{\AbsScalarHomogeneity}[0]{
    \bf \hyperref[def:absolutevaluescalarhomogeneous]{Absolute Scalar Homogeneity} \rm
}
\begin{df}[\AbsScalarHomogeneous]
\label{def:absolutevaluescalarhomogeneous}
\rm
    Let V be a \VectorSpace over a \Field $\F \in \{\R, \C\}$. 
    We say that a map $p:V \to V$ is \AbsScalarHomogeneous if,
    for each $\alpha \in \F$ and each $x \in V$, 
    \begin{equation*}
        p(\alpha x) = \abs{\alpha} p(x)
    \end{equation*}
    Under these circumstances, we may instead say that the operator 
    p posesses \AbsScalarHomogeneity.
\end{df}


\begin{rmk}[\ScalarHomogeneous or \AbsScalarHomogeneous operator at 0 is 0]
\label{rmk:seminorm}
\rm
If V is a 
\VectorSpace over a 
\Field 
$\mathbb{F} \in \{\R, \C\}$, then for each 
$x \in V$,
 $0x=0$.
Hence, if p is an \AbsScalarHomogeneous operator on V, then for any $x \in V$
\begin{equation*}
p(0)=p(0x)=|0|p(x)=0p(x)=0
\end{equation*}
If instead p is \ScalarHomogeneous operator on V, then we have
\begin{equation*}
p(0)=p(0x)=0p(x)=0
\end{equation*}
that is, in either case,  p(0)=0. 
\end{rmk}





\newcommand{\Subadditive}[0]{\textbf{\hyperref[def:subadditive]{Subadditive}}\xspace}
\newcommand{\Subadditivity}[0]{\textbf{\hyperref[def:subadditive]{Subadditivity}}\xspace}
\begin{df}[\Subadditive]
\label{def:subadditive}
\rm
Let $G$ be a 
\Magma 
and 
$(H, \leq)$ 
be a 
\PartiallyOrderedMagma. 
We call a mapping $p:G \to H$ \Subadditive if, for every $x,y \in G$, 
\begin{equation*}
    p(xy) \leq p(x)p(y)
\end{equation*}
Under these circumstances, 
we may also say that
$p$
 posesses $\Subadditivity$. 
\end{df}

\newcommand{\Linear}[0]{\textbf{\hyperref[def:linear]{Linear}}\xspace}
\newcommand{\Linearity}[0]{\textbf{\hyperref[def:linear]{Linearity}}\xspace}

\begin{df}[\Linear]
\label{def:linear}
\rm
    Let V, U be  
    \VectorSpaces
    over a 
    \Field
    $\F$. 
    Let $T:V \to U$ 
    be
    \Additive
    and \ScalarHomogeneous.
    Then we say that $T$
    is \Linear.
\end{df}

\label{def:VectorSpaceSpaceOfLinearOperators}
\newcommand{\SpaceOfLinearOperators}[0]{
    \bf \hyperref[def:VectorSpaceSpaceOfLinearOperators]{Space of Linear Operators} \rm
}

\begin{df}[\SpaceOfLinearOperators]
    Let $U,V$ be \VectorSpaces 
    over the same \Field
    $\F$. 
    We denote with 
    $L(U,V)$ 
    the set of \Linear
    operators $T:U \to V$. 
    We refer to $L(U,V)$ as the 
    \SpaceOfLinearOperators from 
    U to V.
    We endow $L(U,V)$ with 
    the operations of pointwise addition
    and pointwise scalar multiplication, 
    which the reader can verify makes 
    $L(U,V)$ into a \VectorSpace. 
\end{df}


\newcommand{\AlgebraicDual}[0]{\textbf{\hyperref[def:AlgebraicDual]{Algebraic Dual}}\xspace}
\newcommand{\AlgebraicDuals}[0]{ \textbf{\hyperref[def:AlgebraicDual]{Algebraic Duals}}\xspace}
\newcommand{\LinearFunctional}[0]{ \textbf{\hyperref[def:AlgebraicDual]{Linear Functional}}\xspace}
\newcommand{\LinearFunctionals}[0]{ \textbf{\hyperref[def:AlgebraicDual]{Linear Functionals}}\xspace}
\newcommand{\scAlgebraicDual}[1]{\hyperref[def:AlgebraicDual]{\ensuremath{#1'}}}
\begin{df}[\AlgebraicDual]
\label{def:AlgebraicDual}
\rm
Let $V$ be a \VectorSpace 
over a \Field $\bbF$. 
We define $\scAlgebraicDual{V}=L(V, \bbF)$.
We call $\scAlgebraicDual{V}$ the \AlgebraicDual
of $V$. 
If $x^* \in \scAlgebraicDual{V}$, then we call 
$x^*$ a \LinearFunctional.
\end{df}

\begin{prop}[\LinearlyIndependent \LinearFunctionals]
    \label{prop:LinearlyIndependentLinearFunctionals}
    \rm
        Let $V$ be a \VectorSpace
        over a \Field $\bbF$.
        Let $\{x_i^*\}_{i=1}^n \subset \scAlgebraicDual{V}$ be 
        \LinearlyIndependent. 
        Let $\{e_i\}_{i=1}^n$ be the \StandardBasis for $\bbF^n$. 
        Define $S:V \to  \bbF^n$ by 
        $S(v)= \sum\limits_{i=1}^n \ip{v, x_i^*}e_i$
        Then $S$ is \Surjective. 
        \begin{proof}
            Supose otherwise.
            Then there exists $0 \neq c \in \bbF^n$, represented
            $c= \sum\limits_{i=1}^n c_i e_i$
            such that $c \perp Range(S)$. Hence, for every $x \in X$, 
            \begin{align*}
            0 
            & = \ip{\sum\limits_{i=1}^n \ip{x, x_i^*} e_i, c} \\
            & =  \ip{\sum\limits_{i=1}^n \ip{x, x_i^*}e_i, \sum\limits_{i=1}^n c_i e_i} \\
            & = \sum\limits_{i=1}^n \ip{x, x_i^*} \overline{c_i}\\
            & = \ip{x, \sum\limits_{i=1}^n \overline{c_i} x_i^*} 
            \end{align*}
            This implies that $\sum\limits_{i=1}^n \overline{c_i} x_i^* = 0$, a contradiction. 
        \end{proof}
\end{prop}

\begin{prop}[Functional Kernel Intersection]
\label{prop:FunctionalKernelIntersection}
\rm
Let $V$ be a \VectorSpace
over a \Field $\bbF$. 
Let $\{v_i\}_{i=1}^n \subset \scAlgebraicDual{V}$. 
Let $v \in \scAlgebraicDual{V}$. 
Suppose
\begin{equation*}
\bigcap\limits_{i=1}^n Kernel(v_i) \subset Kernel(v)
\end{equation*}
Then $v \in span(v_1, \cdots, v_n)$. 
\begin{proof}
   Without loss of generality 
    we let $\{v_i\}_{i=1}^n$ be \LinearlyIndependent.
    Let $\{e_i\}_{i=1}^n$ be the \StandardBasis of $\bbF^n$.
   Define $S:V \to \bbF^n$  by $S(x)=\sum\limits_{i=1}^n \ip{x, v_i} e_i$. 
   By \ref{prop:LinearlyIndependentLinearFunctionals}, $S$ is \Surjective.
   Hence, if we let $Q:V \to V/Kernel(S)$ be the quotient map. 
   Then $S$ has a invertible quotient 
   $\tilde{S}:V/Kernel(S) \to \bbF^n$ which satisfies
   $S=\tilde{S} \circ Q$ and $Q = \tilde{S}^{-1} \circ S$. 
   Since $Kernel(S) = \bigcap\limits_{i=1}^n Kernel(v_i) \subset Kernel(v)$, 
   there exists $\tilde{v}:X/Kernel(S) \to \bbF$ such that 
   $v=\tilde{v} \circ Q$. 
    Also, $\tilde{v}\circ \tilde{S}^{-1}:\bbF^n \to \bbF$
    is linear, so there are $\{\sigma_i\}_{i=1}^n \subset \bbF$ 
    such that if $\sum_{i=1}^n x_i e_i \in \bbF^n$, we have 
    \begin{equation*}
    \tilde{v} \circ \tilde{S}^{-1} \pa{\sum\limits_{i=1}^n x_i e_i}
    = \sum\limits_{i=1}^n x_i \pa{\tilde{v} \circ \tilde{S}^{-1}}(e_i)
    = \sum\limits_{i=1}^n x_i \sigma_i
    \end{equation*}
    Hence, for $x \in V$, we have 
    \begin{align*}
    \ip{x, v} & = \tilde{v} \circ Q (x) \\
    & = \tilde{v} \circ \tilde{S}^{-1} \circ S(x)\\
    & = \pa{\tilde{v} \circ \tilde{S}^{-1}} \pa{ \sum\limits_{i=1}^n \ip{x, v_i} e_i}\\
    & = \sum\limits_{i=1}^n \ip{x, v_i} \sigma_i\\
    & = \ip{x, \sum\limits_{i=1}^n \sigma_i v_i}
    \end{align*}
    Hence $v \in span(v_1, \cdots, v_n)$.
\end{proof}
\end{prop}

\newcommand{\Balanced}[0]{
    \bf \hyperref[def:BalancedSet]{Balanced} \rm 
}
\newcommand{\BalancedSet}[0]{
    \bf \hyperref[def:BalancedSet]{Balanced Set} \rm 
}
\newcommand{\BalancedSets}[0]{
    \bf \hyperref[def:BalancedSet]{Balanced Sets} \rm 
}

\begin{df}[\Balanced]
\label{def:BalancedSet}
\rm
    Let $V$ be a \VectorSpace 
    over a \Field 
    $\F \in \{\R, \C\}$.
    Let $S \subset V$. 
    We call $S$ a \BalancedSet and
    we say that $S$ is 
    \Balanced if 
    for each
    $\alpha \in \F$ 
    with 
    $\abs{\alpha} \leq 1$
    we have 
    $\alpha S \subset S$. 
\end{df}

\newcommand{\Absorbing}[0]{\textbf{\hyperref[def:AbsorbingSet]{Absorbing}}\xspace}
\newcommand{\AbsorbingSet}[0]{\textbf{\hyperref[def:AbsorbingSet]{Absorbing Set}}\xspace}
\newcommand{\AbsorbingSets}[0]{\textbf{\hyperref[def:AbsorbingSet]{Absorbing Sets}}\xspace}
\newcommand{\Absorbed}[0]{\textbf{\hyperref[def:AbsorbingSet]{Absorbed}}\xspace}
\newcommand{\Absorb}[0]{\textbf{\hyperref[def:AbsorbingSet]{Absorb}}\xspace}
\newcommand{\Absorbs}[0]{\textbf{\hyperref[def:AbsorbingSet]{Absorbs}}\xspace}

\begin{df}[\Absorbing]
\label{def:AbsorbingSet}
\rm
    Let V be a \VectorSpace
    over a 
    \Field
    $\F \in \{\R, \C\}$. 
    Let $A, B \subset V$. 
    We say that $A$ 
    \Absorbs
    $B$ if 
    there exists a 
    $c > 0$ 
    such that
    $B \subset cA$. 
    In such a scenario, 
    $A$ is also said to 
    \Absorb $B$, 
    and we say that $B$ is 
    \Absorbed by $A$.
    If $A$ \Absorbs every singleton in $V$, 
    then we call $A$ an 
    \AbsorbingSet or we say that $A$
    is \Absorbing. 
\end{df}

\newcommand{\ScalingOperator}[0] {
    \bf \hyperref[def:ScalingOperator]{Scaling Operator} \rm
}

\begin{df}[\ScalingOperator]
\label{def:ScalingOperator}
\rm
    Let $V$ be a 
    \VectorSpace 
    over a 
    \Field $\F$. 
    Let $\alpha \in \F$. 
    We define $M_\alpha:V \to V$ by 
    setting, for each 
    $x \in V$, 
    $M_\alpha(x)=\alpha x$.
    We call $M_\alpha$ the
    \ScalingOperator
\end{df}

\begin{prop}[\ScalingOperator]
    \label{prop:ScalingOperatorAlgebraicProperties}
    Let $V$
    be a 
    \VectorSpace
    over a 
    \Field
    $\F$. 
    The following are true:
    \begin{enumerate}
        \item If $\alpha, \beta \in \F$, then $M_\alpha \circ M_\beta = M_{\alpha * \beta}$. 
    \end{enumerate}


    \begin{proof}[Proof of 01]
        Let $v \in V$. Then 
        \begin{align*}
            M_{\alpha} \circ M_{\beta} v & = M_\alpha \pa{\beta * v } \\
            & = \alpha * (\beta * v) \\
            & = (\alpha * \beta) * v \\
            & = M_{\alpha*\beta}v
        \end{align*}
    \end{proof} 

\end{prop}

\newcommand{\Interval}[0]{\textbf{\hyperref[def:Interval]{Interval}}\xspace}
\newcommand{\Intervals}[0]{\textbf{\hyperref[def:Interval]{Intervals}}\xspace}
\newcommand{\ClosedInterval}[0]{\textbf{\hyperref[def:Interval]{Closed Interval}}\xspace}
\newcommand{\ClosedIntervals}[0]{\textbf{\hyperref[def:Interval]{Closed Intervals}}\xspace}
\newcommand{\OpenInterval}[0]{\textbf{\hyperref[def:Interval]{Open Interval}}\xspace}
\newcommand{\OpenIntervals}[0]{\textbf{\hyperref[def:Interval]{Open Intervals}}\xspace}
\newcommand{\HalfClosedInterval}[0]{\textbf{\hyperref[def:Interval]{Half-Closed Interval}}\xspace}
\newcommand{\HalfClosedIntervals}[0]{\textbf{\hyperref[def:Interval]{Half-Closed Intervals}}\xspace}
\newcommand{\HalfOpenInterval}[0]{\textbf{\hyperref[def:Interval]{Half-Open Interval}}\xspace}
\newcommand{\HalfOpenIntervals}[0]{\textbf{\hyperref[def:Interval]{Half-Open Intervals}}\xspace}
\begin{df}[\Interval]
\label{def:Interval}
\rm
    Let $V$ be a 
    \VectorSpace 
    over a 
    \Field
    $\F \in \{ \R, \C\}$. 
    Let $x,y \in V$. 
    We define the following sets:
    \begin{align*} 
        [x,y] = \{tx+(1-t)y : t \in [0,1] \} \\
        [x,y) = \{tx+(1-t)y : t \in [0,1) \} \\
        (x,y] = \{tx+(1-t)y : t \in (0,1] \} \\
        (x,y) = \{tx+(1-t)y : t \in (0,1) \}
    \end{align*}
    We refer to any of these sets as 
    \Intervals in $V$.
    Even in the absence of a topological structure, 
    we use the following language:
    \begin{enumerate}
        \item $[x,y]$ is called a \ClosedInterval.
        \item $(x,y)$ is called an \OpenInterval.
        \item $(x,y]$ and $[x,y)$ are called \HalfOpenIntervals or \HalfClosedIntervals.
    \end{enumerate}
\end{df}

\newcommand{\ConvexSet}[0]{
    \bf \hyperref[def:ConvexSet]{Convex} \rm 
}

\begin{df}[\ConvexSet]
\label{def:ConvexSet}
\rm
    Let $V$ 
    be a 
    \VectorSpace
    over a $\Field$
    $\F \in \{\R, \C\}$. 
    Let 
    $K \subset V$. 
    We say that 
    $K$ is
    \ConvexSet
    if 
    for every pair $x,y \in K$, 
    we have $[x,y] \subset K$.
\end{df}






\section{Topological Algebra}
\subsection{Topological Groups}
\label{def:TopologicalGroup}
\newcommand{\TopologicalGroup}[0]{
    \bf \hyperref[def:TopologicalGroup]{Topological Group}  \rm`
}
\begin{df}[\TopologicalGroup]
    Let $(G,+,e)$ be a \Group. 
    Let $g_{-1}:G \to G$ be defined by 
    $g(x)=-x$. 
    Let $\T$ be a 
    \TopologyRef on
    $G$ such that 
    $+:G \times G \to G$ is \Continuous
    and $g_{-1}$ is \Continuous.
    In this scenario, we call $(G, \T)$ a \TopologicalGroup.
\end{df}

\begin{prop}[Group Invariance]
\label{prop:GroupInvariance}
\rm
Let $(G,\scT)$ be a \TopologicalGroup.
Let $x \in G$. 
Then $\scRightTranslationOperator_x$, 
$\scLeftTranslationOperator_x$, and 
$\scGroupInverseOperator_G$
are 
\Homeomorphisms of $G$ with itself. 
\begin{proof}
Let $x \in G$. Let $y \in G$. 
Let $V \in \T$ with 
$\scLeftTranslationOperator_x(y) \in V$. 
Then $x  y  \in V$. 
That is, $(x,y) \in *^{-1}(V)$
By \FunctionContinuity  of $*$, there are 
$U_1,U_2 \in \T$, $x \in U_1$, $y \in U_2$ such that 
$ U_1  U_2 \subset V$. 
Since 
$\scLeftTranslationOperator_x(U_2) = x  U_2 \subset U_1  U_2 \subset V$, 
$U_2 \subset L_x^{-1}(V)$. 
Since $V$ was arbitrary, $\scLeftTranslationOperator_x$ is \ContinuousAt $y$.
Since $y$ was arbitrary, $\scLeftTranslationOperator_x$ is \ContinuousFunction.
Similarly,  
$\scLeftTranslationOperator_{x^{-1}}$ is \ContinuousFunction. 
Since 
$\pa{\scLeftTranslationOperator_x}^{-1} =\scLeftTranslationOperator_{x^{-1}}$, 
$\scLeftTranslationOperator_x$ has a \ContinuousFunction inverse and so is a \Homeomorphism. 
The proof for $\scRightTranslationOperator_x$ is virtually unchanged. 
Since $\pa{\scGroupInverseOperator_G}^{-1} = \scGroupInverseOperator$
and  $\scGroupInverseOperator_G$ is \ContinuousFunction, 
it is a \Homeomorphism.
\end{proof}
\end{prop}

\newcommand{\LocalBasis}[0]{\textbf{\hyperref[def:LocalBasis]{Local Basis}}\xspace}
\begin{df}[\LocalBasis]
\label{def:LocalBasis}
\rm
    Let $(G,\T)$ be a 
    \TopologicalGroup
    with \IdentityElement $e$.
    We call a 
    \NeighborhoodBasis of $\T$ about $e$
    a \LocalBasis for $(G,\T)$. 
\end{df}


\begin{prop}[Symmetric Contained]
\label{prop:SymmetricContained}
\rm
Let $(G,\T)$ be a
\TopologicalGroup with identity $e$.
Let $e \in U \in \T$. 
Then there exists a \SetOpen \SymmetricSubset $V$ containing $e$
such that $VV \subset U$. 
\begin{proof}
Since $ee=e \in U$, and the \Group operation is  is \ContinuousFunction,  
there are 
\SetOpen $V_1$ and $V_2$ with 
$e \in V_1$, $e \in V_2$ such that $V_1V_2 \subset U$. 
Define $V = V_1 \cap V_2 \cap V_1^{-1} \cap V_2^{-1}$. 
Since $\scGroupInverseOperator_G$ is a  \Homeomorphism $V_1^{-1}$ and 
$V_2^{-1}$ are \SetOpen. Hence $V$ is \SetOpen. Also, 
$e \in V$, so $\emptyset \neq V$ and $V$ is clearly \SymmetricSubset.
Furthermore 
$VV \subset V_1  V_2 \subset U$. concluding the proof.
\end{proof}
\end{prop}

\begin{prop}[Local Basis Symmetric]
\label{rmk:LocalBasisSymmetric}
\rm
It is clear by \ref{prop:SymmetricContained} 
that any \TopologicalGroup contains a 
\LocalBasis consisting of only 
\SymmetricSubset sets, 
and for any \LocalBasis
of a \TopologicalGroup, 
we can find a \LocalBasis
of the same \Cardinality
which is \SymmetricSubset.
\end{prop}

\begin{prop}[Group Separation]
\label{prop:GroupSeparation}
\rm
Let $(G,\T)$ be a \TopologicalGroup with \IdentityElement $e$.
Let $K \subset G$ be 
\SetCompact. Let $C \subset G$ be \SetClosed. 
Let $K \cap C = \emptyset$. 
Then there exists an 
\SymmetricSubset \SetOpen $V$ containing $e$ such that 
$VKV \cap VCV = \emptyset$.
\begin{proof}
Let $y \in K$. 
Then $y \in G \setminus C$. 
Since $C$ is \SetClosed, 
there is an \SetOpen  
\SymmetricSubset 
$\tilde{V}_y$
containing $e$ such that
$\tilde{V}_y\{y\} \cap C = \emptyset$. 
By 3 applications of
\ref{prop:SymmetricContained}
there exists an 
\SetOpen
\SymmetricSubset $V_y$ containing $e$
such that $V_y^8 \subset \tilde{V}_y$. 
Then $V_y^8 \{y\} \cap C = \emptyset$. 
Then, since by \ref{prop:GroupInvariance}, 
$\scLeftTranslationOperator_y$ is a 
\Homeomorphism, there is an 
\SetOpen $U_y$ containing $e$ 
such that $yU_y \subset V_y y$. 
By \ref{prop:SymmetricContained} 
we can without loss of generality assume 
$U_y$ is \SymmetricSubset.
Hence $V_y^4 \{y\} U_y^4 \cap C = \emptyset$. 
Define $W_y = U_y \cap V_y$. 
Then $W_y$ is \SetOpen, \SymmetricSubset, $e \in W_y$, and 
$W_y^4 \{y\} W_y^4 \cap C = \emptyset$.
Since $W_y$ is \SymmetricSubset, 
$\pa{W_y^2 \{y\} W_y^2} \cap \pa{W_y^2 C W_y^2} = \emptyset$.

Thus we have, for each $x \in K$, a 
\SetOpen , \SymmetricSubset $W_x$ containing $e$ such that $\pa{W_x^2 \{x\}W_x^2} \cap  \pa{ W_x^2 C W_x^2}= \emptyset$.
Since $K \subset \bigcup\limits_{x \in K} W_x \{x\} W_x$, we can find a 
$\{x_i\}_{i=1}^n \subset K$ such that 
$K \subset \bigcup\limits_{i=1}^n W_{x_i} \{x_i\} W_{x_i}$.
Define $W = \bigcap\limits_{i=1}^n W_{x_i}$.
Then 
\begin{align*}
WKW & \subset W\pa{\bigcup\limits_{i=1}^n W_{x_i}\{x_i\}W_{x_i}} W\\
& = \bigcup\limits_{i=1}^n \pa{W W_{x_i} \{x_i\} W_{x_i} W}\\
& \subset \bigcup\limits_{i=1}^n \pa{ W_{x_i}^2 \{x_i\} W_{x_i}^2}\\
& \subset G \setminus \pa{\bigcap\limits_{i=1}^n W_{x_i}^2 C W_{x_i}^2}\\
& \subset G \setminus \pa{\bigcap\limits_{i=1}^n W_{x_i} C W_{x_i}}\\
& \subset G \setminus \pa{ \pa{\bigcap\limits_{i=1}^n W_{x_i}} C \pa{\bigcap\limits_{i=1}^n W_{x_i}}}\\
& = G \setminus \pa{WCW}
\end{align*}
\end{proof}
\end{prop}

\begin{prop}[\SetClosed Product]
\label{prop:ClosedSum}
\rm
Let $(G, \T)$ be a \TopologicalGroup with \IdentityElement $e$.
Let $C \subset G$ be \SetClosed. 
Let $K \subset G$ be \SetCompact. 
Then $CK$ is \SetClosed and $KC$ is \SetClosed. 
\begin{proof}
Let $x \not \in CK$. 
Then $\pa{C^{-1}x} \cap K = \emptyset$.
By \ref{prop:GroupInvariance}, $C^{-1}x$ is \SetClosed.
Hence, by \ref{prop:GroupSeparation}, there exists
an
\SetOpen \SymmetricSubset
$V$ contining $e$ such that 
$\pa{C^{-1}xV } \cap \pa{KV} = \emptyset$.
Hence $\pa{xVV^{-1}} \cap \pa{CK} = \emptyset$, so 
$x \not \in \overline{CK}$, so 
$CK = \overline{CK}$.
Hence $CK$ is \SetClosed. 
The proof for $KC$ is virtually identical.
\end{proof}
\end{prop}

\begin{prop}[Local Base Nesting]
\label{prop:LocalBaseNesting}
\rm
Let $G$ be a \TopologicalGroup with \IdentityElement $e$.
Let $\scB$ be a \LocalBasis for $G$. 
Let $b_0 \in \scB$. 
Then there exists $b_1 \in \scB$ such that 
$\overline{b_1} \subset b_0$. 
\begin{proof}
Let $b_0 \in B$. 
Then $C=G \setminus b_0$  is \SetClosed and 
$K = \{e\}$ is \SetCompact. 
Furthermore $e \in b_0$, so $K \cap C = \emptyset$, 
so by \ref{prop:GroupSeparation}, 
there is an \SetOpen $V$ such that 
\begin{equation*}
V \cap \pa{\pa{G \setminus b_0}V }= \pa{eV} \cap \pa{\pa{G \setminus b_0}V } = \emptyset
\end{equation*}
If $x \not \in b_0$, then $xV \subset \pa{G \setminus b_0}  V $ is \Disjoint from $V$, 
so $x \not \in \overline{V}$. 
Hence $\overline{V} \subset b_0$. 
Since $\scB$ is a \LocalBasis, 
there is a $b_1 \in \scB$ with $b_1 \subset V$. 
Hence $\overline{b_1} \subset \overline{V} \subset b_0$. 
\end{proof}
\end{prop}

\begin{prop}[Closure Characterization]
\label{prop:ClosureCharacterization}
\rm 
Let $G$ be a \TopologicalGroup, 
$\scB= \{V_\alpha \}_{\alpha \in B}$ be a \LocalBasis for $G$, 
and $A \subset X$. 
Then 
\begin{equation*}
\overline{A} = \bigcap\limits_{\alpha \in B}\pa{ A+V_{\alpha}}
\end{equation*}
\begin{proof}
It is clear that by \ref{rmk:LocalBasisSymmetric}, we can, 
without loss of generality, 
assume that $\scB$ consists of \SymmetricSubset
sets. 
Let $x \in \overline{A}$. 
This is true if and only if, for each $\alpha \in B$, 
$x+V_\alpha \cap A = \emptyset$. 
Since $\scB$ consists of \SymmetricSubset sets, 
this is true if and only if, for each $\alpha \in B$, 
$\{x\} \cap \pa{A+V_\alpha} = \{x\} \cap \pa{A - V_\alpha} \neq \emptyset$. 
This is equivalent to saying that $x \in A+V_{\alpha}$ for every $\alpha \in B$. 
This is then equivalent to 
\begin{equation*}
x \in \bigcap\limits_{\alpha\in B} \pa{A+V_{\alpha}}
\end{equation*}
Since every step involved here was a logical equivalence, 
\begin{equation*}
\overline{A} = \bigcap\limits_{\alpha \in B} \pa{A+V_{\alpha}}
\end{equation*}
\end{proof}
\end{prop}

\begin{prop}[Sum Closure]
\label{prop:SumClosure}
\rm
Let $(G, \scT)$ be a \TopologicalGroup with \IdentityElement $e$
and let $A_0,A_1 \subset G$. 
Then 
\begin{equation}
\overline{A_0}+\overline{A_1} \subset \overline{A_0+A_1}
\end{equation}
Furthermore, if there exists a \SetCompact $K$ such that either 
$A_0 \subset K$ or $A_1 \subset K$, then 
\begin{equation*}
\overline{A_0}+\overline{A_1} = \overline{A_0+A_1}
\end{equation*}

\begin{proof}
Let $a_i \in \overline{A_i}$ for $i \in \{0,1\}$. 
Let $e \in V \in \scT$. 
By \ref{prop:SymmetricContained} there is a
\SymmetricSubset $V'$ satisfying
$e \in V' \in \scT$ and $V'+V' \subset V$. 
By \ref{prop:GroupInvariance}, 
there exists 
a
\SymmetricSubset
\SetOpen
$\tilde{V}$ 
containing $e$ 
such that $\tilde{V} a_1 \subset a_1V'$
and $\tilde{V} \subset V'$. 
Since $a_i \in \overline{A_i}$, 
$A_i \cap \pa{a_i + \tilde{V}} \neq \emptyset$. 
Hence, for $i \in \{1,2\}$, there exists 
$x_i \in A_i  \cap \pa{a_i+\tilde{V}}$.
Hence, 
\begin{align*}
x_0+x_1 &  \in \pa{A_0 \cap (a_0+\tilde{V})} + \pa{A_1 \cap (a_1+\tilde{V})}\\
& \subset (A_0+A_1) \cap \pa{a_0 + \tilde{V} + a_1 + \tilde{V}}\\
& \subset \pa{A_0+A_1} \cap \pa{a_0+a_1+V'+V'}\\
& \subset \pa{A_0+A_1} \cap \pa{a_0+a_1+V} \neq \emptyset
\end{align*}
Since $e \in V \in \scT$ was arbitrary, 
$a_0+a_1 \in \overline{A_0+A_1}$, 
Hence 
\begin{equation*}
\overline{A_0}+ \overline{A_1} \subset \overline{A_0+A_1}
\end{equation*}

Now, if there is a compact $K$ such that $A_0 \subset K$, then 
$\overline{A_0} \subset \overline{K}$. 
Furthermore, by \ref{prop:CompactClosure}, $\overline{K}$ is \SetCompact.
Hence, by  \ref{prop:ClosedCompact}, $\overline{A_0}$ is \SetCompact. 
Hence, by \ref{prop:ClosedSum} $\overline{A_0}+\overline{A_1}$ is \SetClosed. 
Thus, we have
\begin{align*}
\overline{A_0+A_1} & \subset \overline{\overline{A_0}+\overline{A_1}}\\
& = \overline{A_0}+\overline{A_1}
\end{align*}
So equality holds. 
If instead $A_1 \subset K$, the argument is identical. 

\end{proof}
\end{prop}

\begin{prop}[Birkhoff-Kakutani]
\label{prop:BirkhoffKakutani}
\rm
Let $(G,\scT)$ be a \TopologicalGroup with \IdentityElement $e$.
Let $\scB=\{\scV_i\}_{i \in \mathbb{N}}$ be a \LocalBasis for $G$. 
Then there exists a \Pseudometric $d$ on $G$ with the following properties.
\begin{enumerate}
\item $\scT$ is the \PseudometricTopology on $G$ induced by $d$.
\item For each $x,y,z \in G$, $d(zx,zy)=d(x,y)$. 
\end{enumerate}
\begin{proof}
By 
\ref{prop:SymmetricContained}
$G$ has a \LocalBasis $\{V_i\}_{i \in \mathbb{N}}$ consisting
of \SymmetricSubset subsets satisfying, for each $n \in \mathbb{N}$, 
$V_{n+1}V_{n+1} \subset V_n$.
For each $r \in \mathbb{Q} \cap [0,1)$, let $C(n,r) \in \{0,1\}$ 
be the $n^{th}$ bit of $r's$ finite expansion. 
That is, for $r \in \mathbb{Q}$, let 
\begin{equation*}
r=\sum\limits_{n \in \mathbb{N}} \frac{C(n,r)}{2^n}
\end{equation*}
Let $P$ be the set of $r \in \mathbb{Q} \cap [0,1)$ 
for which $C(n,r)$ is nonzero for only finitely many $r$. 
For each $r \in P$, let $\Gamma_r$ denote 
the $n \in \mathbb{N}$ for which $C(n,r) \neq 0$. 
Since $G$ is not assumed to be \CommutativeFunction, 
define, for a \Finite set $K=\{x_k\}_{k=1}^m \subset \mathbb{N}$ in 
which $x_1<x_2<\cdots<x_m$, 
\begin{equation*}
\prod\limits_{n \in K} U_{n} = U_{x_1}U_{x_2}U_{x_3}\cdots U_{x_m}
\end{equation*}
Now, define $A:P \cup (1,\infty)\to \scT$ by 
\begin{equation*}
A(r) = \begin{dcases}
\prod\limits_{n \in \Gamma_r} V_n & r<1\\
G & r \geq 1
\end{dcases}
\end{equation*}

I first claim that if $r \leq s$, then $A(r) \subset A(s)$. 
To prove this, let $r \leq s$.
Define $\Gamma_r$ and $\Gamma_s$ as above. 
Then there exists a $k \in \mathbb{N}$ such that 
$\Gamma_r \cap [0,k] \subset \Gamma_s$
and $k+1 \in \Gamma_s \setminus \Gamma_r$.
Define $N=[k+2,\infty) \cap \Gamma_r$. 
Then since $V_{n+1}V_{n+1} \subset V_n$ for all $n$, it is clear that
\begin{equation*}
\prod\limits_{n \in N} V_n \subset V_{k+1}
\end{equation*}
Hence, 
\begin{align*}
A(r) & = \pa{\prod\limits_{n \in \Gamma_r \cap [0,k]} V_n} \pa{ \prod\limits_{n \in N} V_n} \\
& \subset\pa{ \prod\limits_{n \in \Gamma_r \cap [0,k]} V_n} V_{k+1} \\
& \subset\pa{\prod\limits_{n \in \Gamma_s \cap [0,k]} V_n} V_{k+1} \\
& \subset A(s)
\end{align*}

I now make a second claim: that if $r \in P \cap (0,1)$ and if $n>\max\{k \in \mathbb{Z}^+ : C(rk,r)=1\}$ then 
$A(r)A\pa{\frac{1}{2^n}} = A\pa{r+\frac{1}{2^n}}$. 
This is clear because, for all $k \in \mathbb{Z}^+$, 
\begin{equation*}
C(k,r)+C\pa{k,\frac{1}{2^n}} = C\pa{k, r+\frac{1}{2^n}}
\end{equation*}
Thus we have
\begin{equation*}
A(r)A\pa{\frac{1}{2^n}} = \pa{\prod\limits_{k \in \Gamma_r} U_k} U_n = \prod\limits_{k \in \Gamma_{r+\frac{1}{2^n}}} U_k = A\pa{r+\frac{1}{2^n}}
\end{equation*}

I now claim that for every $n \in \mathbb{Z}^+$, for every $r \in P$, we have 
\begin{equation*}
A(r)A\pa{\frac{1}{2^n}} \subset A\pa{r+\frac{3}{2^n}}
\end{equation*}
By the first and second claims, it is sufficient to prove 
in the case $n \leq \max\{n \in \mathbb{Z}^+ : C(n,r) = 1 \}$. 
Now, Define $\Gamma_r^{-} = \Gamma_r \cap [0,n)$ and $\Gamma_r^+ = \Gamma_r \cap [n,\infty)$.
Then by assumption $\Gamma_r^+ \neq \emptyset$. 
Define 
\begin{equation*}
r_1 = \frac{1}{2^{n-1}} - \sum\limits_{j \in \Gamma_r^+} \frac{1}{2^j}
\end{equation*}
Then $r_1 >0$ and 
Define 
\begin{equation*}
r_2 = r+r_1
\end{equation*}
Then $\max\{k \in \mathbb{Z}^+ : C(r_2,k) \neq 0 \leq n-1\}$.
Hence, by claim 2, $A(r_2)A\pa{\frac{1}{2^n}} = A\pa{r_2+\frac{1}{2^n}}$. 
Also $r < r_2 < r+\frac{1}{2^{n-1}}$.
By this observation, paired with claim 01, we have 
\begin{align*}
A(r)A\pa{\frac{1}{2^n}} & \subset A(r_2) A\pa{\frac{1}{2^n}}\\
&  = A\pa{r_2+\frac{1}{2^n}}\\
& \subset A\pa{r+\frac{1}{2^{n-1}}+\frac{1}{2^n}}\\
& \subset A\pa{r+\frac{3}{2^{n}}}
\end{align*}

Define, for $x \in G$, $\tilde{d}(x) = \inf\{r \in [0,\infty) : x \in A(r)\}$. 
Then $\tilde{d}(x) \leq 1$ for all $x \in G$. 
Define, for $x,y \in G$, 
\begin{equation*}
d\pa{x,y} = \sup\limits_{h \in G} \braces{ \abs{\tilde{d}(hx)-\tilde{d}(hy)}}
\end{equation*}

Since $\tilde{d}$ is bounded, $d$ is well defined. 
$d$ is clearly \CommutativeFunction, satisfies the 
\TriangleInequality, and satisfies $d(x,x)=0$, so 
$d$ is a \Pseudometric on $G$. 
That $d$ is left invariant is equally clear.

What remains to show is that the
\PseudometricTopology $\scT_d$ generated by $d$ on $G$ 
Since $d$ is left invariant, it suffices
to show that for each $\epsilon > 0$, 
there is an $n$ such that $V_n \subset B(e;\epsilon)$, 
and for each $k \in \mathbb{N}$, there is 
a $\delta > 0$ such that $B(e;\delta) \subset V_k$.

Showing $\scT \subset \scT_d$ is easy. 
Given $n \in \mathbb{N}$, 
\begin{equation*}
B\pa{e,\frac{1}{2^{n+1}}} \subset A\pa{\frac{1}{2^n}}=U_n
\end{equation*}

For the other direction, let $\epsilon >0$. 
Let $n \in \mathbb{N}$ such that $\frac{3}{2^n} < \epsilon$. 
Let $u \in U_n=A\pa{\frac{1}{2^n}}$. 
Let $z \in G$ and let $r$ be any positive number such that $z \in A(r)$. 
Then 
\begin{equation*}
zu \in A(r)A\pa{\frac{1}{2^n}} \subset A\pa{r+\frac{3}{2^n}}
\end{equation*}
Hence, 
\begin{equation*}
\tilde{d}(zu) \leq r+\frac{3}{2^n}
\end{equation*}
The nature of $r$ implies
\begin{equation}
\label{BirkhoffKakutaniProof:Dir2}
\tilde{d}(zu) \leq \inf\{r \in (0,\infty) : z \in A(r)\} + \frac{3}{2^n} = \tilde{d}(z)+\frac{3}{2^n}
\end{equation}

Now let $s$ be any positive number such that $zu \in A(s)$. 
Then 
\begin{equation*}
z \in A(s)u^{-1} \subset A(s)U_n^{-1} = A(s)U_n \subset A\pa{s+\frac{3}{2^n}}
\end{equation*}
Hence
\begin{equation*}
\tilde{d}(z) \leq s+\frac{3}{2^n}
\end{equation*}
Similar to the above, the nature of $s$ implies
\begin{equation}
\label{BirkhoffKakutaniProof:Dir1}
\tilde{d}(z) \leq \tilde{d}(zu)+\frac{3}{2^n}
\end{equation}
By 
\ref{BirkhoffKakutaniProof:Dir1}
and
\ref{BirkhoffKakutaniProof:Dir2}, 
and the arbitrary nature of $z$, 
$d(u,e) \leq \frac{3}{2^n}< \epsilon$. 
Since $u \in U_n$ was arbitrary, 
$U_n \subset B\pa{e;\epsilon}$.
\end{proof}
\end{prop}

\begin{prop}[\NormalSubgroup]
\label{prop:TopGroup:NormalSubgroup}
\rm
Let $G$ be a \TopologicalGroup with \IdentityElement $e$. 
Then $\overline{\{e\}}$ is a \NormalSubgroup of $G$. 
\begin{proof}
Let $x \in \overline{\{e\}}$ and $y \in \overline{\{e\}}$. 
Then $y^{-1} \in \overline{\{e\}}$. 
Hence 
$xy^{-1} \in \overline{\{e\}} \cdot \overline{\{e\}} = \overline{\{e\}\{e\}} = \overline{\{e\}}$
so $\overline{\{e\}}$ is a \Subgroup of $G$. 
Let $h \in \overline{\{0\}}$ and let $x \in G$. 
Let $V \in \scU_{\scT}(0)$. 
Then there exists \SetOpen $U \in \scU_{\scT}(0)$ such that $xU \subset Vx$. 
Since $h \in \overline{\{e\}}$, $e \in Uh$. 
Hence, 
$e=xx^{-1} =xex^{-1} \in xUhx^{-1} \subset Vxhx^{-1}$
Hence $xhx^{-1} \in \overline{\{e\}}$, so $x\overline{\{e\}}x^{-1} \subset \overline{\{e\}} $.
In otherwords, $\overline{\{e}\}$ is a 
\NormalSubgroup of $G$. 
\end{proof}
\end{prop}

\label{def:TopologyOfUniformConvergence}
\newcommand{\TopologyOfUniformConvergence}[0]{
    \bf \hyperref[def:TopologyOfUniformConvergence]{Topology of Uniform Convergence} \rm
}
\begin{df}[\TopologyOfUniformConvergence]
    Let X be a set and 
    $(Y, \T_Y)$ be a \TVS.
    Let $\NbhFilter{\T_Y}{0}$ denote the 
    \NeighborhoodFilter of $0 \in (Y, \T_Y)$.
    Suppose $\scF$ is a 
    \VectorSubspace of the set of 
    functions  $T:X \to Y$. 
    Suppose
    $\scG \subset 2^X$  such that 
	$(\scG, \subset)$ is a \DirectedSet.
    For each $x \subset X$ and $y \subset Y$, and  define 
    $M(x, y) = \{f \in \scF | f(x) \subset y\}$
    Now we define 
    $\T(\scF, \T_Y, \scG)= \{f+ M(x,y) | x \in \scG \wedge y \in \NbhFilter{\T_Y}{0} \wedge f \in \scF\}$.
	We call $\T(\scF, \T_Y, \scG)$ the \TopologyOfUniformConvergence of $\scF$ on $\scG$ with respect to $\T_Y$. 
	When $\scF$, $\T_Y$ or $\scG$ are understood they may be omitted from the reference. 
	By \ref{prop:TopologyOfUniformConvergence}, $\T$ is a 
	\TopologyRef on $\scF$. 
\end{df}
\begin{prop}\bf REMOVE\rm \end{prop}


\begin{prop}[\TopologyOfUniformConvergence is Compatible]
\label{prop:TopologyOfUniformConvergenceCompatible}
\rm
    Let $X$ be a nonempty set. 
    Let $(Y,\scT_Y)$ be a \CommutativeFunction \TopologicalGroup with 
    \IdentityElement $e$.
    Let $\scF$ be a 
    \Subgroup of the set of 
    functions  $T:X \to Y$. 
    Let 
    $\emptyset \neq \scG \subset 2^X$  such that 
	$(\scG, \subset)$ is a \DirectedSet.
    Let $\scT$ denote the \TopologyOfUniformConvergence 
    on $\scF$ determined by $\scG$. 
    Then $(\scF, \scT)$ is a \TopologicalGroup.
\begin{proof}
Let $f,g \in \scF$.
Let $U \in \scU_{fg}$.
Then there exists $U_1 \in \scG$ and $V \in \scT_Y$ such that 
$fgM(U_1,V) \subset U$.
By 
\ref{prop:SymmetricContained}
there exists \SetOpen $\tilde{V} \subset V$ such that 
$e \in \tilde{V}$ and $\tilde{V}\tilde{V} \subset V$. 
Then, since $\cdot$ is \CommutativeFunction, 
\begin{align*}
\pa{fM(U_1,\tilde{V})}\pa{gM(U_1,\tilde{V})} & = fgM(U_1,\tilde{V})M(U_1,\tilde{V})\\
& \subset fgM(U_1, \tilde{V} \tilde{V}) \\
& \subset fgM(U_1,V) \\
& \subset U
\end{align*}
Hence the \Group \Operation of $\scF$ is \ContinuousAt $(f,g)$. 
Since $f,g \in \scF$ were arbitrary, the \Group \Operation of $\scF$
is \ContinuousFunction with respect to $\scT$. 
I now show that $\scGroupInverseOperator_{\scF}$ is \ContinuousFunction
with respect to $\scT$. 
Let $f \in \scF$. Let $U \in \scU_{f^{-1}}$. 
Then there exists $U_1 \in \scG$ and $V \in \scT_{Y}$ containing $e$ such that
$f^{-1}M(U_1,V) \subset U$. 
Let $g \in fM(U_1,V^{-1})$. 
Then $g=fh$ for some $h \in M(U_1,V^{-1})$. Hence, if $x \in U$, then 
\begin{equation*}
g^{-1}(x) = \pa{g(x)}^{-1} = \pa{f(x)h(x)}^{-1} = h^{-1}(x)f^{-1}(x) \in \pa{V^{-1}}^{-1}f^{-1}(x) = f^{-1}(x) V
\end{equation*}
Hence $g^{-1} \in f^{-1}M(U_1,V) \subset U$. 
Hence $\scGroupInverseOperator_{\scF}\pa{ fM(U_1,V^{-1})} \subset f^{-1}M(U_1,V)$.
This concludes the proof. 
\end{proof}
\end{prop}



\subsection{Topological Vector Spaces} 
\label{def:VectorSpaceCompatible}
\newcommand{\VectorSpaceCompatible}[0]{
    \bf \hyperref[def:VectorSpaceCompatible]{Compatible} \rm
}
\newcommand{\VectorSpaceCompatibility}[0]{
    \bf \hyperref[def:VectorSpaceCompatible]{Compatibility} \rm
}
\begin{df}[\VectorSpaceCompatible]
    Let 
    $(V, +, \cdot, 0)$
    be a 
    \VectorSpace
    over $\F$
    and $\T$ be a 
    \Topology
    on $V$ such that 
    $(V,+,\T)$
    is a
    \TopologicalGroup
    and
    $\cdot:\F \times V \to V$
    is \Continuous.
    Then we say that 
    $\T$
    is
    \VectorSpaceCompatible
    with
    $(V, +, \cdot, 0)$, 
    or when $+$ and $\cdot$ are obvious, 
    we say that 
    $\T$ 
    is 
    \VectorSpaceCompatible
    with 
    $\T$. 
\end{df}


\label{def:topologicalvectorspace}
\newcommand{\TVS}[0]{
    \bf \hyperref[def:topologicalvectorspace]{Topological Vector Space} \rm
}

\begin{df}[\TVS]
Let $(V,+,\cdot, 0)$ be a 
\VectorSpace over a \Field $\F \in \{\R, \C\}$. 
Let $\T$ be a 
\Topology on $V$
which is 
\VectorSpaceCompatible
with $(V, + , \cdot, 0)$. 
Then we call 
$(V,\T)$ a 
\TVS.
\end{df}

\newcommand{\LocallyConvex}[0]{\textbf{\hyperref[def:LocallyConvex]{Locally Convex}}\xspace}
\newcommand{\LocalConvexity}[0]{\textbf{\hyperref[def:LocallyConvex]{Local Convexity}}\xspace}

\begin{df}[\LocallyConvex]
\label{def:LocallyConvex}
\rm
    We say that a 
    \TVS
    $(X,\T)$ is 
    \LocallyConvex 
    if $(X,\T)$ has a 
    \LocalBasis consisting only of 
    \ConvexSet sets.
    A \LocallyConvex 
    space is said to posess
    \LocalConvexity.
\end{df}


\begin{prop}[Existence of \Balanced \NeighborhoodBasis of 0 in a \TVS]
    \label{prop:ExistenceOfBalancedNeighborhoods}
    \rm
    Let $(X,\T)$ be a 
    \TVS
    over a 
    \Field
    $\F$.
    The following are true. 
    \begin{enumerate}[label=(\roman*), ref={\ref{prop:ExistenceOfBalancedNeighborhoods}~\roman*}]
        \item 
        \label{prop:Bal1}
        If 
            $U \in \scU_{\T}(0)$,
            then there is a 
            \Balanced, \SetOpen
            $V \subset U$
            such that 
            $V \in \scU_{\T}(0)$.
        \item 
        \label{prop:Bal2}
        There exists a 
            \NeighborhoodBasis
            about $0 \in X$ 
            for $\T$ 
            consisting entirely 
            of \Balanced sets. 
        \item 
        \label{prop:Bal3}
        If 
            $U \in \scU_{\T}(0)$ is \ConvexSet,
            then there is a 
            \ConvexSet
            \Balanced, 
            \SetOpen
            $V \subset U$
            such that 
            $V \in \scU_{\T}(0)$.
        \item 
        \label{prop:Bal4}
        If $(X,\T)$ is 
            \LocallyConvex, 
            then there exists a 
            \NeighborhoodBasis
            about $0 \in X$ 
            for $\T$ 
            consisting entirely 
            of 
            \Balanced
            \ConvexSet 
            sets.
    \end{enumerate}

    \begin{proof}[Proof of \ref{prop:Bal1}] 
    Since scalar multiplication is \ContinuousAt $0$, 
    there is an \SetOpen disk $V \subset \mathbb{F}$
    and an \SetOpen $W \subset X$ with $0 \in W$ such that
    $VW \subset U$. 
    $VW$ is clearly balanced. 
    \end{proof}
    \begin{proof}[Proof of \ref{prop:Bal2}] 
    Let $\{U_{\alpha}\}_{\alpha \in A}$ be a 
    \LocalBasis for $\scT$. 
    Then, by \ref{prop:Bal1}, for each $\alpha \in A$, 
    there is a $W_\alpha \subset U_\alpha$ such that
    $W_\alpha$ is \BalancedSet and $0 \in W_\alpha$. 
    Clearly $\{W_\alpha\}_{\alpha \in A}$ forms a 
    \LocalBasis for $X$. 
    \end{proof}
    \begin{proof}[Proof of \ref{prop:Bal3}] 
    By \ref{prop:Bal1}, there is a 
    \BalancedSet \SetOpen $W \subset U$. 
    Let $\alpha \in \mathbb{C}$ with $\abs{\alpha} = 1$. 
    Then $\alpha^{-1}W = \subset W  \subset U$. 
    Hence $W \subset \alpha U$, so 
    if we define $A = \bigcap\limits_{\abs{\alpha} = 1 } \alpha U$, 
    then $0 \in W \subset A$.
    Thus $0 \InteriorMark{A}$. 
    For this reason, it suffices to show $\InteriorMark{A}$ is 
    \BalancedSet.
    It then suffices to show $A$ is \BalancedSet.
    Let $\alpha \in \mathbb{Z}$ with $\abs{\alpha} \leq 1$. 
    Then $\alpha = r \beta$ for some $\beta \in \mathbb{C}$ with $\abs{\beta} = 1$ 
    and $r \in [0,1]$. 
    Since $\alpha U$ is \ConvexSet and contains $0$, 
    $r\alpha U \subset \alpha U$. 
    Hence, 
    \begin{equation*}
    r \beta A = r \beta \bigcap\limits_{\abs{\alpha} = 1} \alpha U = \bigcap\limits_{\abs{\alpha} = 1 } r \alpha U \subset \bigcap\limits_{\abs{\alpha} = 1} \alpha U = A
    \end{equation*}
    Thus $A$ is \BalancedSet.
    \end{proof}
    \begin{proof}[Proof of \ref{prop:Bal4}] 
    This is a similar arguement to that of \ref{prop:Bal2}.
    \end{proof}
\end{prop}

\label{def:topologicalvectorspaceboundedset}
\newcommand{\TVSBounded}[0]{\textbf{\hyperref[def:topologicalvectorspaceboundedset]{TVS-Bounded}}\xspace}

\begin{df}[TVS Bounded Set]
Let $(V,\T)$ be a 
\TVS.
Let $A \subset V$. 
We say that A is \TVSBounded with respect to $\T$,
or when confusion is unlikely we simply say that A is \TVSBounded
if for every $U \in \scU_{\T}(0)$, there exists an $\alpha \in \F$
, $\alpha > 0$
, such that $A \subset \alpha U$. 
\end{df}

\newcommand{\BoundedLinearOperator}[0]{\textbf{\hyperref[def:boundedlinearoperatorinatvs]{Bounded Linear Operator}}\xspace}

\begin{df}[\BoundedLinearOperator]
\label{def:boundedlinearoperatorinatvs}
\rm
Let $\F \in \{\R, \C\}$.
For $i \in \{0,1\}$, let $(V_i,\T_i)$ be  a \TVSs over $\F$.
We say that a \Linear operator $T:(V_1, \T_1) \to (V_2, \T_2)$ is a 
\BoundedLinearOperator
if for each $U \in V_1$ with U \TVSBounded with respect to $\T_0$, 
$TU$ is \TVSBounded with respect to $\T_1$. 
\end{df}



TODO: Clean this Up. It isn't actually clear to me how I should topologize
CL(U,V) for arbitrary TVS's' U and V. 
For now its fine, because so far i've only used the convergence in the case of a 
Seminormed space but eventualy I want to define a topology that
in the case where U and V are seminormed spaces, $CL(U,V)=BL(U,V)=$ the seminormed topology generated by their seminorms.
\label{def:TVSSpaceOfContinuousLinearOperators}
\newcommand{\SpaceOfContinuousLinearOperators}[0]{
    \bf \hyperref[def:TVSSpaceOfContinuousLinearOperators]{Space Of Continuous Linear Operators} \rm
}
\begin{df}[\SpaceOfContinuousLinearOperators]
    Let 
    $(U, \T_U)$
    and $(V, \T_V)$
    each be a \TVS
    over the same $\Field$
    $\F \in \{\R, \C\}$. 
    Let $L(U,V)$ denote the \SpaceOfLinearOperators
    from $U$
    to $V$.
    We denote with 
    $CL((U, \T_U), (V, \T_V))$ 
    the subset of 
    $L(U, V)$ consisting only of the \Continuous operators. 
    When $\T_U$ and $\T_V$ are understood, 
    we may denote
    $CL((U, \T_U), (V, \T_V))=CL(U, V)$ 

    
\end{df}

\begin{prop}[\TopologyOfUniformConvergence is a Tvs]
\label{prop:TOUC:Tvs}
\rm
Let $\bbF \in \{\R, \C\}$. 
Let $Y$ be a \TVS
over $\bbF$. 
Let $X$ be a nonempty set.
Let $\scF$ be a \VectorSubspace 
of the set of maps $T:X \to Y$. 
Let $\scG \subset X$. 
Let $\scT$ denote the 
\TopologyOfUniformConvergence
on $\scF$ induced by $\scG$. 
Then $\pa{\scF, \scT}$ is a \TVS
over $\bbF$ if and only if, 
for each $U \in \scG$, for each 
$f \in \scF$, the set $f(U)$ is 
\TVSBounded in $Y$. 
\begin{proof}
Suppose $(\scF, \scT)$ is a \TVS. 
Then $\cdot:\bbF \times \scF \to \scF$ is \ContinuousFunction. 
Let $f \in \scF$. Let $U \in \scG$. Let $V \in \scT_Y$. 
Since $\cdot$ is \ContinuousFunction, 
$T:\bbF \to \scF$ defined by $T(\alpha) = \alpha f$ is 
\ContinuousFunction.
Hence $T^{-1}(M(U,V))$ contains some $\alpha_0>0$.
Hence $alpha_0f \in M(U,V)$. That is $\alpha_0f(U) \subset V$.
That is, $f(U) \subset \frac{1}{\alpha_0} V$, so $f(U)$ is absorbed by $V$. 
Since $V \in \scT_Y$ is an arbitrary \SetOpen set containing $0$, $f(U)$ 
is \TVSBounded. 
Since $U \in \scG$ is arbitrary, this direciton of the theorem is proven.

Now let each $f(U)$ be \TVSBounded. 
Let $\alpha \in \bbF$. 
Let $f \in \scF$.
Let $U \in \scG$.
Let $V \in \scT_{Y}$ such that $0 \in V$. 
By 
\ref{prop:SymmetricContained}
there exists 
$\tilde{V} \in \scT_Y$ such that 
$0 \in \tilde{V}$ and
$\tilde{V}+\tilde{V}+\tilde{V}+\tilde{V} \subset V$.
Let $V' \in \scT_Y$ such that $\alpha V' \subset \tilde{V}$.
Let $V''$ be a \Balanced \SetOpen subset of $Y$ containing $0$ such that  $V'' \subset V' \cap \tilde{V}$. 
Since $f(U)$ is \TVSBounded there exists $\epsilon_0 > 0$ such that 
\begin{equation*}
\epsilon_0 f(U) \subset V'' \tab[1cm] \abs{\epsilon_0} < \abs{\alpha}
\end{equation*}
Set $\Gamma = B_{\mathbb{C}}(0;\epsilon_0)$. 
Set $\Omega = M(U,V'')$. 
Let $h \in \Gamma\pa{f+\Omega}$. 
Then there exists $\gamma \in \C$ with $\abs{\gamma} < \epsilon_0$ 
and there exists $g \in M(U,V'')$ such that $h=(\alpha+\gamma)\pa{f+g}$. 
Let $x \in U$. 
Then, 
\begin{align*}
h(x) & = \pa{\alpha+\epsilon}\pa{f+g}(x)\\
& = \alpha f(x) + \gamma f(x) + \alpha g(x) + \gamma g(x) \\
& \in \alpha f(x) + \gamma f(U) + \alpha g(U) + \gamma g(U) \\
& = \alpha f(x) + \frac{\gamma}{\epsilon_0} \epsilon_0 f(U) + \alpha g(U) + \gamma g(U) \\
& \subset  \alpha f(x) + \frac{\gamma}{\epsilon_0} V'' + \alpha V'' + \gamma V''\\
& \subset \alpha f(x) + V'' +\alpha V'' +\alpha V'' \\
& \subset \alpha f(x) + \tilde{V}+ \tilde{V}+ \tilde{V}\\
& \subset \alpha f(x) + V
\end{align*}
Hence, $h \in\alpha f + M(U,V)$.
Hence, $\Gamma \Omega \subset \alpha f + M(U,V)$.
Since $U \in \scG$ and $V \in \scT_Y$ containing 0 were arbitrary, 
$\cdot$ is 
\ContinuousAt $(\alpha, f )$. 
Since $\alpha \in \bbF$ and $f \in \scF$ were arbitrary,
$\cdot$ is \ContinuousFunction. 
By 
\ref{prop:TopologyOfUniformConvergenceCompatible}, $(\scF, \scT)$ 
is a \TopologicalGroup, so \FunctionContinuity of $\cdot$ is enough
to guarantee that 
$(\scF, \scT)$ is a \TVS.

\end{proof}
\end{prop}



\section{Seminormed Spaces}
\subsection{Introduction}
\newcommand{\Seminorm}[0]{\textbf{\hyperref[def:seminorm]{Seminorm}}\xspace}
\newcommand{\Seminorms}[0]{\textbf{\hyperref[def:seminorm]{Seminorms}}\xspace}
\newcommand{\NonDegenerate}[0]{\textbf{\hyperref[def:seminorm]{Non-Degenerate}}\xspace}
\newcommand{\Degenerate}[0]{\textbf{\hyperref[def:seminorm]{Degenerate}}\xspace}
\newcommand{\SeminormedSpace}[0]{\textbf{\hyperref[def:seminorm]{Seminormed Space}}\xspace}
\newcommand{\SeminormedSpaces}[0]{\textbf{\hyperref[def:seminorm]{Seminormed Spaces}}\xspace}
\begin{df}[Seminorm]
\label{def:seminorm}
    Let 
	$V$ be a 
	\VectorSpace
	over a 
	\Field 
	$\F \in \{ \R, \C\}$.  
    We say that a map 
	$\norm{\cdot}:V \to [0,\infty)$ 
	is a 
	\Seminorm on 
	$V$ 
	if it is both \Subadditive and \AbsScalarHomogeneous. 
	In this case, we refer to $(V, \norm{\cdot})$ as a \SeminormedSpace. 
	We say that $\norm{\cdot}$ is \NonDegenerate if there is at least one $v \in V$ with $\norm{v}>0$. 
	We say that $\norm{\cdot}$ is \Degenerate if it is not \NonDegenerate.  
	We may also refer to the \SeminormedSpace $(V, \norm{\cdot})$ as being
	\Degenerate
	or
	\NonDegenerate. 
\end{df} 





\label{def:norm}
\newcommand{\Norm}[0]{
    \bf \hyperref[def:norm]{Norm} \rm
}
\label{def:normedspace}
\newcommand{\NormedSpace}[0]{
    \bf \hyperref[def:normedspace]{Normed Space} \rm
}\newcommand{\NormedSpaces}[0]{
    \bf \hyperref[def:normedspace]{Normed Spaces} \rm
}
\begin{df}[Norm]
    Let $(V,\norm{\cdot})$ be a \SeminormedSpace.
    If the following implication is true for $x \in V$, then we refer to $\norm{\cdot}$ as a \Norm on V, and we call $(V, \norm{\cdot})$ a \NormedSpace.
    \begin{equation}
    x \neq 0 \implies \norm{x} \neq 0
    \end{equation}
\end{df}

\begin{prop}[Subadditive Operator On a Group Induces a Metric]
    \label{prop:subadditiveinducestriangleinequality}
    Let $(G,+, e)$ be a group and let $(H,+,\leq)$ be a totally ordered magma. 
    Let $p:G \to H$ be \Subadditive. 
    define $d:G \times G \to H$ by setting, for each $x,y \in G$, 
    \begin{equation}
        d(x,y) =  p(x+(-y))
    \end{equation}

    Then d satisfies the triangle inequality. 

    \begin{proof}
    let $x,y, z \in G$. Then
    \begin{align*}
        d(x,z) &= p(x+(-z))\\
        & = p(x+e+(-z))\\
        & = p(x+(-y)+y+(-z))\\
        & \leq p(x+(-y))+p(y+(-z))\\
        & = d(x,y)+d(y,z)
    \end{align*}
    completing the proof. 
    \end{proof} 
\end{prop}
 
\label{def:seminormtopology}
\newcommand{\SeminormTopology}[0]{
    \bf \hyperref[def:seminormtopology]{Seminorm Topology} \rm
}

\newcommand{\SeminormInducedPseudometric}[0]{
    \bf \hyperref[def:seminormtopology]{Pseudometric induced by the Seminorm} \rm
}

\newcommand{\SeminormSpaceInducedPseudometricSpace}[0]{
    \bf \hyperref[def:seminormtopology]{Pseudometric Space induced by the Seminormed Space} \rm
}


\begin{df}[Seminorm Topology]
    Let $(X,\norm{\cdot})$ be a \SeminormedSpace.
    define $d_{\norm{\cdot}}:V \times V \to [0,\infty)$  by setting,
    for $x,y \in X$, 
    \begin{equation}
    d_{\norm{\cdot}}(x,y) = \norm{x-y}
    \end{equation}
    Observe the following: 
    \begin{enumerate}
        \item \ref{rmk:seminorm} guarantees that $d_{\norm{\cdot}}(x,x)=0$ for $x \in X$. 
        \item 
        \ref{prop:subadditiveinducestriangleinequality} guarantees that d satisfies the \TriangleInequality. 
        \item d is a \SymmetricMap, as we have 
    \begin{equation}
        d(x,y)_{\norm{\cdot}}=\norm{x-y}=|-1|\norm{x-y}=\norm{y-x}=d(y,x)
    \end{equation}
    \end{enumerate}

    Hence, $d_{\norm{\cdot}}$  is a \Pseudometric on X, which we call the \SeminormInducedPseudometric on X. 
    We refer to $(X, d_{\norm{\cdot}})$ as the \SeminormSpaceInducedPseudometricSpace $(X,\norm{\cdot}$. 
    We refer to the \PseudometricTopology induced by $d_{\norm{\cdot}}$ as the \SeminormTopology induced by $\norm{\cdot}$, and unless otherwise specified, when we reference $(X,\norm{\cdot})$, we consider it to be endowed with this topology. 

\end{df}

\newcommand{\CompleteSeminormedSpace}[0]{\textbf{\hyperref[df:BanachSpace]{Complete Seminormed Space}}\xspace}
\newcommand{\CompleteNormedSpace}[0]{\textbf{\hyperref[df:BanachSpace]{Complete Normed Space}}\xspace}
\newcommand{\CompleteSeminormedSpaces}[0]{\textbf{\hyperref[df:BanachSpace]{Complete Seminormed Spaces}}\xspace}
\newcommand{\CompleteNormedSpaces}[0]{\textbf{\hyperref[df:BanachSpace]{Complete Normed Spaces}}\xspace}
\newcommand{\BanachSpace}[0]{\textbf{\hyperref[df:BanachSpace]{Banach Space}}\xspace}
\newcommand{\BanachSpaces}[0]{\textbf{\hyperref[df:BanachSpace]{Banach Spaces}}\xspace}
\begin{df}[\CompleteSeminormedSpace, \BanachSpace]
\label{df:BanachSpace}
Let $(X,\norm{\cdot})$ be a \SeminormedSpace.
Let $d$ be the \Pseudometric induced on $X$ by $\norm{\cdot}$. 
If $(X,d)$ is \PseudometricComplete, then we call 
$(X, \norm{\cdot})$ a \CompleteSeminormedSpace. 
$(X, \norm{\cdot})$ is a \NormedSpace then under these 
same circumstances we call $(X, \norm{\cdot})$ a 
\CompleteNormedSpace. 
A \CompleteNormedSpace is called a \BanachSpace. 
\end{df}

\label{def:seminormkernel}
\newcommand{\SeminormKernel}[0]{\textbf{\hyperref[def:seminormkernel]{Seminorm Kernel}}\xspace}
\newcommand{\SeminormKernels}[0]{\textbf{\hyperref[def:seminormkernel]{Seminorm Kernels}}\xspace}
\newcommand{\Ker}[0]{\textbf{\ensuremath{\mathcal{K}\rm^{ernel}}}\xspace}


\begin{df}[Seminorm Kernel]
Let $(V, \norm{\cdot})$ be a \SeminormedSpace. 
Define the set $\Ker_{(V,\norm{\cdot})}$ by 
\begin{equation}
\Ker_{(B,\norm{\cdot})}=\{x \in V | \norm{x}=0\}
\end{equation}
We call this set the \SeminormKernel of the space $\Ker_{(V,\norm{\cdot})}$. 
When confusion is unlikely, we may denote this set with
$\Ker$, $\Ker_V$, or even $\Ker_{\norm{\cdot}}$, or we may just refer to it
as the \SeminormKernel, the \SeminormKernel of $V$, or the \SeminormKernel of $\norm{\cdot}$. 
\end{df}

\begin{prop}[Seminorm Kernel is a vector Subspace]
\label{prop:seminormkernelisavectorsubspace}
    Let $(X,\norm{\cdot})$ be a \SeminormedSpace over a field $\F \in \{\R, \C\}$  
    with corresponding \SeminormKernel $\Ker$. 
    Then the following are true. 
    \begin{enumerate}
        \item $\Ker$ is a vector subspace of X. 
        \item $\Ker$ is closed in the \SeminormTopology on X.
        \item $\Ker=X$ if and only if X is \Degenerate. 
    \end{enumerate}


    \begin{proof}[Proof of One]
        \Subadditivity implies that, if $x,y \in \Ker$, then $\norm{x+y} \leq \norm{x}+\norm{y}=0$. 
        By \ScalarHomogeneity, if $x \in \Ker$  and $\alpha \in \F$, $\norm{\alpha x} =|\alpha| \norm{x}=0$
        so $\Ker$ is in fact a vector subspace of X. 
    \end{proof}
    \begin{proof}[Proof of Two]
        
        If $x \in X \setminus  \Ker$
        then $\norm{x} = \alpha > 0$ for some positive $\alpha$. 
        Hence $B(x;\alpha/2)$ is an open set containing x disjoint from $\Ker$. 
       We can then write $X \setminus \Ker$ as the union of all such open sets to see that $\Ker$ is closed. 
    \end{proof}
	
	\begin{proof}[Proof of Three]
		Direct application of the definitions of the \SeminormKernel
		and \Degenerate \Seminorm. 
	\end{proof} 
\end{prop}

\label{prop:equivalencemodkernelispseudometricequivalence}
\begin{prop}[Equivalence Mod Kernel is Pseudometric Equivalence]
    Let $(X,\norm{\cdot})$ be a seminromed space.
    with \SeminormKernel $\Ker$.
    Let $d$ denote the \SeminormInducedPseudometric.
	Let $\cong_{d}$ denote the 
	\RelationOfZeroDistance with respect to d. 
    
    Then $\cong_{\Ker}=\cong_{d}$. 
    \begin{proof}
        Let $x,y \in X$ and let $x \cong_{\Ker}y$.
        Then, since $x-y \in \Ker$, 
        Then $d(x,y) := \norm{x-y} =0$, so $x \cong_d y$. 
        Hence $\cong_{\Ker} \subset \cong_{d}$ 


        Now let $x,y \in X$ with $x \cong_d y$. 
        Then $\norm{x-y}=d(x,y) = 0$, so $x-y \in \Ker$
        , and therefore $x \cong_{\Ker} y$. 
        Hence, $\cong_{d} \subset \cong_{\Ker}$. 

        Since inclusion goes both directions, $\cong_{\Ker} = \cong_d$.

    \end{proof} 
\end{prop}

\newcommand{\EquivelanceModKernel}[0]{\textbf{\hyperref[def:equivalencemodseminormkernel]{Equivalence MOD-$\Ker$}}\xspace}
\newcommand{\EquivalenceModKernel}[0]{\textbf{\hyperref[def:equivalencemodseminormkernel]{Equivalence MOD-$\Ker$}}\xspace}
\newcommand{\EquivalentModKernel}[0]{\textbf{\hyperref[def:equivalencemodseminormkernel]{Equivalent MOD-$\Ker$}}\xspace}
\newcommand{\SeminormKernelQuotientVectorSpace}[0]{\textbf{\hyperref[def:equivalencemodseminormkernel]{Seminorm Kernel Quotient Vector Space}}\xspace}

\begin{df}[Quotient Space Mod Kernel]
\label{def:equivalencemodseminormkernel}
\rm
Let $\F \in \{\R,\C\}$. 
Let $(X,\norm{\cdot})$ be a \SeminormedSpace over $\F$.
Denote the \SeminormKernel of $(X,\norm{\cdot})$ with  $\Ker$.
By \ref{prop:SeminormKernel:VectorSubspace} and \ref{prop:SeminormKernel:Closure0}, 
$\Ker$ is a \SetClosed \VectorSubspace of $X$. 
We endow $X/\Ker$ with the \QuotientVectorSpace structure, 
which we call the 
\SeminormKernelQuotientVectorSpace of the \SeminormedSpace $(X,\norm{\cdot})$.
\end{df}


\label{def:quotientnormspace}
\newcommand{\QuotientNorm}[0]{
    \bf \hyperref[def:quotientnormspace]{Quotient Norm} \rm
}
\newcommand{\QuotientNormedSpace}[0]{
    \bf \hyperref[def:quotientnormspace]{Quotient Normed Space} \rm
}

\begin{df}[Quotient Norm Space]
Let $(X,\norm{\cdot})$ be a \SeminormedSpace
with \SeminormInducedPseudometric $d$, 
\SeminormKernel $\Ker$, and
\SeminormKernelQuotientVectorSpace $X/\Ker$.
Let $\tilde{d}:X/\Ker \times X/\Ker \to [0,\infty)$ be the \MetricInducedByPseudometric.

Define $\norm{\cdot}_{\Ker} : X/\Ker \to [0,\infty)$ by 
\begin{equation}
\norm{[x]}_{\Ker} = \tilde{d}([x], [0])
\end{equation}

By $\ref{prop:quotientnormspace}$, $(X/\Ker, \norm{\cdot}_{\Ker})$ is a normed space which we call the \QuotientNormedSpace of $(X,\norm{\cdot})$, and we call $\norm{\cdot}_{\Ker}$ the \QuotientNorm. 
Whenever we refer to $X/\Ker$, unless otherwise specified, we endow it with this norm and the topology generated by this norm.
Furthermore, whenever we consider $X/\Ker$, unless otherwise specified, we consider it as 
possesing the topology generated by the norm $\norm{\cdot}_{\Ker}$. 
\end{df}

\begin{prop}[Quotient Normed Space]
\label{prop:quotientnormspace}
\rm
Let $(X,\norm{\cdot})$ be a \SeminormedSpace
with \SeminormInducedPseudometric $d$, 
\SeminormKernel $\Ker$, and
\SeminormKernelQuotientVectorSpace $X/\Ker$.
Let $\tilde{d}:X/\Ker \times X/\Ker \to [0,\infty)$ be the \MetricInducedByPseudometric.
Let $T:X \to X/\Ker$ denote the \QuotientMap of X into $X/\Ker$ 
(Recalling that the 
\RelationOfEqualNeighborhoodFilters equals the 
\RelationOfZeroDistance equals the relation of 
\EquivelanceModKernel), so they would all produce the same quotient map)
Let $\norm{\cdot}_{\Ker}$ denote the \QuotientNorm.

The following are true. 
\begin{enumerate}[label=(\roman*), ref={\ref{prop:quotientnormspace}~\roman*}]
\item 
\label{prop:QNS:WellDefined}
$\norm{\cdot}_{\Ker}$ is a \Norm on $X/\Ker$. 
\item 
\label{prop:QNS:Compatible}
$\tilde{d}$  is the \SeminormInducedPseudometric $\norm{\cdot}_{\Ker}$, 
and thus they produce the same \Topology. 
\item 
\label{prop:QNS:Topology}
T has all of the properties described in $\ref{prop:QuotientSpaceTopology}$. 
\item 
\label{prop:QNS:Linear}
T is \Linear.
\item 
\label{prop:QNS:Surjective}
T is \Surjective. 
\item 
\label{prop:QNS:Isometry}
T is an \Isometry. 
\item 
\label{prop:QNS:Injective}
T is \Injective if and only if $\norm{\cdot}$ is a \Norm. 
\begin{proof}[Proof of \ref{prop:QNS:WellDefined}]
    First, note that 
    $Range(\norm{\cdot}_{\Ker}) \subset Range(\tilde{d}) \subset [0,\infty)$,\
    so that $\norm{\cdot}_{\Ker}$ has the correct \FunctionDomain and \FunctionCodomain. 
    For \Subadditivity, let $[x],[y] \in X/\Ker$. Then 
    \begin{align*}
        \norm{[x]+[y]}_{\Ker}& = \norm{[x+y]}_{\Ker}\\
        & = \tilde{d}\pa{[x+y], [0]}\\
        & = d(x+y, 0)\\
        & = \norm{x+y} \\
        & \leq \norm{x}+\norm{y}\\
        & = d(x,0)+d(y,0)\\
        & = \tilde{d}\pa{[x],[0]}+ \tilde{d}\pa{[y],[0]}\\
        & = \norm{[x]}_{\Ker}+\norm{[y]}_{\Ker}
    \end{align*}
    For \AbsScalarHomogeneity, let $\alpha \in \F$ and $[x] \in X/\Ker$. 
    Then, 
    \begin{align*}
        \norm{[\alpha x]}_{\Ker} & = \tilde{d}\pa{[\alpha x], [0]}\\
        & = d(\alpha x, 0) \\
        & = \norm{\alpha x}\\
        & = \abs{\alpha} \norm{x} \\
        & = \abs{\alpha} \norm{[x]}_{\Ker}
    \end{align*}
    Finally, suppose $[x] \neq 0$. 
    Then, since the additive identity of $X/\Ker$ is $\Ker$, $x \not \in \Ker$. 
    Hence $\norm{[x]}_{\Ker} = \tilde{d}([x], 0) = d(x,0) =\norm{x} > 0$. 

\end{proof}
\begin{proof}[Proof of \ref{prop:QNS:Compatible}] 
Let $D$ denote the \SeminormInducedPseudometric $\norm{\cdot}_{\Ker}$. 
Then, for $[x], [y] \in X/\Ker$, 
\begin{align*}
\tilde{d}([x], [y]) & = d(x,y)\\
& = \norm{x-y}\\
& = \norm{x-y-0}\\
& = d(x-y, 0)\\
& = \tilde{d}([x-y],0)\\
& = \norm{[x-y]}_{\Ker}\\
& = \norm{[x]-[y]}_{\Ker}\\
& = D\pa{[x], [y]}
\end{align*}
Since these two \Pseudometric's are equal, they produce the same \Topology. 
Furthermore, by applying \ref{prop:pseudometricinducedmetric}, we see that the 
\Topology generated by $\norm{\cdot}_{\Ker}$ is also the \QuotientSpaceTopology on $X/\Ker$. 
\end{proof}
\begin{proof}[Proof of \ref{prop:QNS:Topology}]
T is the topological \QuotientMap and the \Norm \Topology is the \QuotientSpaceTopology, so the assumptions of $\ref{prop:QuotientSpaceTopology}$ are satisfied. 
\end{proof} 
\begin{proof}[Proof of \ref{prop:QNS:Linear}] 
This is a direct consequence of \ref{prop:QuotientVectorSpace:QuotientMapLinear}.
\end{proof}
\begin{proof}[Proof of \ref{prop:QNS:Surjective}] 
This is a direct consequence of \ref{prop:QuotientMapSurjective}.
\end{proof}
\begin{proof}[Proof of \ref{prop:QNS:Isometry}] 
This is a direct consequence of \ref{def:MFPM:IsIsometricSurjection}.
\end{proof}
\begin{proof}[Proof of \ref{prop:QNS:Injective}] 
This is a direct consequence of \ref{def:MFPM:Injection}.
\end{proof}

\end{enumerate} 

\end{prop} 

\begin{rmk}[Quotient Normed Space]
\label{rmk:quotientnormedspace}
\rm
    If $(X,\norm{\cdot}_X)$
    is a \NormedSpace
    then by parts
    \ref{prop:QNS:Linear}, 
    \ref{prop:QNS:Surjective}, 
    \ref{prop:QNS:Isometry}, 
    and
    \ref{prop:QNS:Injective}, 
    $T:X \to \Ker_X$
    is an isomorphism of 
    \NormedSpaces satisfying
    $Tx=\{x\}$.
    For this reason, 
    as an abuse of notation, 
    later in this document,
    I may not distinguish between the quotient
    $X/\Ker_X$ and the space $X$ if
    X is a \NormedSpace, 
    and similarly, I may not distinguish between 
    $x \in X$ and $\{x\} \in X/\Ker_X$. 
\end{rmk}

\begin{prop}
\label{prop:quotientspreservecompleteness}
Let $(X,\norm{\cdot})$ be a \SeminormedSpace with \QuotientNormedSpace $(X/\Ker, \norm{\cdot}_{\Ker})$. 

Then X is \PseudometricComplete if and only if $X/\Ker$ is complete. 

\begin{proof}
Let X be \PseudometricComplete. 
Let $\{[x_i]\}_{i \in \N} \subset X/\Ker$ be a \PseudometricCauchySequence. 
Let $\epsilon > 0$. 
Then there is an $N \in \N$ such that for $m,n > N$ we have 
\begin{equation}
\norm{[x_m-x_n]}_{\Ker} < \epsilon
\end{equation}

For this N, we have 
\begin{equation}
\norm{x_m-x_n} = \norm{[x_m-x_n]}_{\Ker} < \epsilon
\end{equation}
so that $\{x_i\}_{i \in \N}$ is a \PseudometricCauchySequence. 
Since X is \PseudometricComplete, 
there is a 
$x \in X$ such that $\norm{x_i-x} \to 0$, 
but since T is an isometry, 
\begin{equation}
\norm{[x]-[x_i]}=\norm{[x_i-x]}_{\Ker} \to 0
\end{equation}
and so 
$[x_i] \to [x]$.
so that $X/\Ker$ is complete. 

Now suppose instead that $X/\Ker$ is complete 
and suppose $\{x_i\}_{i \in \N}$ is a \PseudometricCauchySequence in X. 
Since $\norm{[x_i-x_j]}_{\Ker} = \norm{x_i-x_j}$, 
$\{[x_i]\}_{i \in \N}$ is a \PseudometricCauchySequence in $X/\Ker$, which therefore has a 
limit $y \in X/\Ker$. Since T is surjective, $y=[x]$ for some $x \in X$, and it is easy to see that
$x_i \to x$ so that $X$ is \PseudometricComplete. 

\end{proof}
\end{prop}

\subsection{Continuous Linear Operators On Seminormed Spaces}
\label{def:BLO} 
\newcommand{\SpaceOfBoundedLinearOperators}[0]{ 
    \bf \hyperref[def:BLO]{Space of Bounded Linear Operators} \rm
}
\newcommand{\OperatorSeminorm}[0]{
    \bf \hyperref[def:BLO]{Operator Seminorm} \rm
}
\newcommand{\OperatorNorm}[0]{
    \bf \hyperref[def:BLO]{Operator Norm} \rm
}
\begin{df}[Space of Continuous Linear Operators From a Seminormed Space into a Normed Space]
Let $(X,\norm{\cdot}_X)$ be a \NonDegenerate \SeminormedSpace.
Let $(Y, \norm{\cdot}_Y)$ be a \SeminormedSpace.
We denote with $BL\pa{(X,\norm{\cdot}_X), (Y, \norm{\cdot}_Y)}$ 
the collection of
\ContinuousFunction
\Linear
operators
$T:(X, \norm{\cdot}_X) \to (Y, \norm{\cdot}_Y)$. 
When the topologies on X and Y are understood, we denote this set with
$BL\pa{X,Y}$. 
We refer to $BL\pa{X,Y}$ as the \SpaceOfBoundedLinearOperators 
from $(X, \norm{\cdot}_X)$ to $(Y, \norm{\cdot}_Y)$ 
, or when $\norm{\cdot}_X$ and $\norm{\cdot}_Y$ are understood, 
from X to Y. 

We endow $BL\pa{X,Y}$ with the algebraic operations
of pointwise scalar multiplication
and pointwise addition, making $BL\pa{X,Y}$ a vector space. 

We define $\norm{\cdot}:BL(X,Y) \to [0,\infty)$ by defining, 
for $T \in BL(X,Y)$
\begin{equation}
    \norm{T} = \sup\limits_{\norm{x}_X \neq 0} \frac{\norm{Tx}_Y}{\norm{x}_X}
\end{equation}
As will be proven in \ref{prop:BLO}, $\norm{\cdot}$ is a \Seminorm on $BL(X,Y)$, which 
we refer to as the \OperatorSeminorm on $BL(X,Y)$. induced by the
\Seminorm $\norm{\cdot}_X$ on X and the \Seminorm $\norm{\cdot}_Y$ on Y. 

In the case that $\norm{\cdot}_{Y}$ is a \Norm, rather than just a \Seminorm, by \ref{prop:BLO}
, $\norm{\cdot}$ is a \Norm on $BL(X,Y)$, which we instead call the \OperatorNorm. 
\end{df}

\begin{prop}[Space of Bounded Linear Operators On Seminormed Spaces]
\label{prop:BLO} 
Let $(X,\norm{\cdot}_X)$ be a \SeminormedSpace. 
Let $(Y, \norm{\cdot}_Y)$ be a \SeminormedSpace.
Let $BL(X,Y)$ denote the \SpaceOfBoundedLinearOperators from X to Y. 
Let $\norm{\cdot}$ denote the \OperatorSeminorm. 

The following are true. 
\begin{enumerate}
%For Item 1, may have to prove result connecting 
%pseudometric topology continuity to $\epsilon-delta$ cotninuity wrt the pseudometric. 
\item $\norm{\cdot}$ is in fact a well-defined \Seminorm on $BL(X,Y)$. 
\item If $\norm{\cdot}_Y$ is a \Norm, then so is $\norm{\cdot}$. 
\item If $T \in BL(X,Y)$ and $\alpha \in (0,\infty)$, then $\norm{T} = \sup\limits_{\norm{x}_X =\alpha} \frac{\norm{Tx}_Y}{\norm{x}_X}$. 
\item If $T \in BL(X,Y)$ and $\alpha \in (0,\infty)$, , then $\norm{T} = \sup\limits_{0<\norm{x}_X \leq \alpha} \frac{\norm{Tx}_Y}{\norm{x}_X}= \sup\limits_{0<\norm{x}_X < \alpha} \frac{\norm{Tx}_Y}{\norm{x}_X}$. 
\item If $T \in BL(X,Y)$ and $x \in X$, then $\norm{Tx}_Y \leq \norm{T} \norm{x}_X$. 
\item $S:X \to Y$ is linear
, $S(\Ker_X) \subset \Ker_Y$
, and $\sup\limits_{\norm{x}_X \neq 0} \frac{\norm{Sx}_Y}{\norm{x}_X} < \infty$
, if and only if $S \in BL(X,Y)$. 
\item A sequence $\{T_i\}_{i \in \N}$ is a \PseudometricCauchySequence
    if and only if
    there exists an $\alpha > 0$ 
    such that the collection of sequences 
    $\{\{T_ix\}_{i \in \N} | x \in B_X(0;\alpha)\}$ is
    \UniformlyCauchy
    if and only if
    for every $\beta > 0$, 
    the collection of sequences 
    $\{\{T_ix\}_{i \in \N} | x \in B_X(0;\beta)\}$ is
    is \UniformlyCauchy
\item If $T_i \to T$ with respect to $\norm{\cdot}$, then $T_ix \to Tx$ with respect to $\norm{\cdot}_Y$ for each $x \in X$
\item A sequence $\{T_i\}_{i \in \N} \subset BL(X,Y)$ 
    converges %TODO: Turn converges into a macro (net based) referenced via \Coverges, and insert that here. 
    with respect to $\norm{\cdot}$ 
    if and only if it is a \PseudometricCauchySequence 
    and for each $x_\alpha$ 
    in some Hamel basis $\{x_\alpha\}_{\alpha \in A} \subset X$,
    the sequence $\{T_ix_\alpha\}_{\alpha \in A}$
    converges with respect to $\norm{\cdot}_Y$. 
\item $BL(X,Y)$ is complete if and only if Y is. 

\item $\norm{\cdot}$ is \NonDegenerate if and only if Y is. \bf THIS WILL NEED TO BER MOVED LATER, UNTIL AFTER THE SEMINORMED HAHN BANACH THEOREM \rm
\end{enumerate}


\begin{proof}[Proof of 1] 
    Since X is nondegenerate, there exists at least 1 $x \in X$ with $\norm{x}_X \neq 0$, 
    so for each $T \in BL(X,Y)$, the set that the supremum is being taken over is nonempty.
    Also, it is clear that $Range(\norm{\cdot}) \subset [0,\infty)$, 

    For \Subadditivity, let $T_i \in BL(X,Y)$ for $i \in \{0,1\}$. and $x \in X$ with $\norm{x} > 0$.
    Then, since $\norm{\cdot}_Y$ is \Subadditive, 
    \begin{align*}
    \frac{\norm{(T_0+T_1)x}_Y}{\norm{x}_X} \leq \frac{\norm{T_0x}_Y}{\norm{x}_X}+ \frac{\norm{T_1x}_Y}{\norm{x}_X}
    \end{align*}
    Since this is true for each x with $\norm{x}_X \neq 0$, taking the supremum of each side yields

    \begin{align*}
    \sup\limits_{\norm{x}_X \neq 0} \pa{\frac{\norm{(T_0+T_1)x}_Y}{\norm{x}_X}} & \leq\sup\limits_{\norm{x}_X \neq 0} \pa{ \frac{\norm{T_0x}_Y}{\norm{x}_X}+ \frac{\norm{T_1x}_Y}{\norm{x}_X}}\\
& \leq\sup\limits_{\norm{x}_X \neq 0} \pa{ \frac{\norm{T_0x}_Y}{\norm{x}_X}} + \sup\limits_{\norm{x}_X \neq 0} \pa{\frac{\norm{T_1x}_Y}{\norm{x}_X}}\\
    \end{align*}
    Hence, $\norm{T_0+T_1} \leq \norm{T_0}+\norm{T_1}$ so that $\norm{\cdot}$ is \Subadditive. 
    For \ScalarHomogeneity, let $T \in BL(X,Y)$, $\alpha \in \F$, and $x \in X$ with $\norm{x}_X \neq 0$. 
    Then 
    \begin{align*}
        \frac{\norm{(\alpha T)x}_Y}{\norm{x}_X} = \frac{\norm{\alpha (Tx)}_Y}{\norm{x}_X} = \abs{\alpha} \frac{\norm{Tx}_Y}{\norm{x}_X}
    \end{align*}
    Hence taking the supremum finishes the proof.
\end{proof}
\begin{proof}[Proof of 2] 
   Let $T \neq 0 \in BL(X,Y)$. Then for some $x \in X$, $Tx \neq 0$. 
   Then $Tx$ has a neighborhood U disjoint from $0_Y$, 
   Hence $x \in T^{-1}(U)$ but not $0_X \in T^{-1}(U)$, since $T0_X = 0_Y$.
   Since U is a neighborhood of x disjoint from 0, 
   there is an $\epsilon > 0$ such that $0_X \subset \complement U \subset \complement \overline{B_X}(x;\epsilon)$,
   and therefore $\norm{x}_X > \epsilon$. 
   Since $\norm{x}_X > 0$, it is ranged over in the supremum defining $\norm{T}$, and so
   \begin{equation}
   0 < \frac{\norm{Tx}_Y}{\norm{x}_X} \leq \sup\limits_{\norm{x}_X \neq 0} \frac{\norm{Tx}_X}{\norm{x}_X}=\norm{T}
   \end{equation}
\end{proof}
\begin{proof}[Proof of 3] 
    Let $\alpha \in (0,\infty)$
   Let $T \in BL(X,Y)$. 
   Then, there is a sequence $\{x_i\} \subset X$ with each $\norm{x_i}_X \neq 0$ 
   such that 
   \begin{equation}
    \frac{\norm{Tx_i}_Y}{\norm{x_i}_X} \to \norm{T}
    \end{equation}
    For each $i \in \N$, define $y_i =\alpha  x_i/\norm{x_i}_X$. 
    then each $\norm{y_i} = \alpha$, 
    and by \ScalarHomogeneity
    of T, we have 
    \begin{equation}
    \frac{\norm{Ty_i}_Y}{\norm{y_i}_X} = \frac{\norm{Tx_i}_Y}{\norm{x_i}_X} \to \norm{T}
    \end{equation}
    , completing the proof. 
\end{proof}
\begin{proof}[Proof of 4] 
If we define, for 
$T \in BL(X,Y)$, 
$f(T) = \sup\limits_{0 < norm{x}_X \leq \alpha} \frac{\norm{Tx}_Y}{\norm{x}_X}$, then
since $\norm{\cdot}^{-1}((0,\alpha))\subset \norm{\cdot}^{-1}((0,\infty))$, we have $f(T) \leq \norm{T} $
and since $\norm{\cdot}^{-1}(\{\alpha\})\subset \norm{\cdot}^{-1}((0,\alpha))$, we have $\norm{T} \leq f(T)$. proving the first equality.
The second is found by applying the same arguement to $\alpha/2$ and realizing that $(0,\alpha/2] \subset (0,\alpha)$. 
\end{proof}
\begin{proof}[Proof of 5]
Let $T \in BL(X,Y)$ and $x \in X$. 
If $\norm{Tx}_Y \neq 0$, then $B_Y(Tx, \frac{\norm{Tx}_Y}{2})$ is a neighborhood of $Tx$ disjoint from 0.
Continuity of T impliese $x$ then has a neighborhood disjoint from $0 \in T^{-1}(0)$, implying
that $\norm{x}_X \neq 0$. 

Hence if $\norm{x}_X = 0$, then we know $\norm{Tx}_Y = 0$, so that the relation
\begin{equation}
\norm{Tx}_Y \leq \norm{T} \norm{x}_X
\end{equation}

If $\norm{x}_X \neq 0$, then by definition of supremum, 
\begin{equation*} 
\frac{\norm{Tx}_Y}{\norm{x}_X} \leq \norm{T}
\end{equation*}
so that $\norm{Tx}_Y \leq \norm{T} \norm{x}_X$. 
\end{proof}
\begin{proof}[Proof of 6]
    I assume the first 3 conditions
    and show that $S \in BL(X,Y)$.
    It is necessary and sufficient to 
    show that S is continuous.
    Let $F= \sup\limits_{\norm{x}_X \neq 0} \frac{\norm{Sx}_Y}{\norm{x}_X}$.
    If $F=0$, then $S(X) \subset \Ker_Y$. 
    Every neighborhood of every point in $\Ker_Y$
    contains $\Ker_Y$, so in that case continuity holds. 
    Suppose $F \neq 0$. 
    By translation invariance of the topology, 
    it is sufficient to consider neighborhoods of $0_Y \in Y$. 
    Let $\epsilon > 0$. 
    Define $V=B_X\pa{0; \frac{\epsilon}{F}}$. 
    Let $x_0 \in V$. 
    If $\norm{x_0}_X = 0$, then
    $S(x_0) \in S(\Ker_X) \subset \Ker_Y \subset B_Y(0;\epsilon)$. 
    If $\norm{x_0}_X \neq 0$, then 
    $\norm{Sx}_Y \leq F \norm{x}_X < \epsilon$, so
    $s(x_0) \in B_Y(0;\epsilon)$. 
    Hence $S\pa{B_X\pa{0;\frac{\epsilon}{F}}} \subset B_Y(0; \epsilon)$.
    so S is continuous,and this direction fo the proof is complete.

    Suppose conversely that $S \in BL(X,Y)$. 
    Then S is linear by definition
    , and the supremum expression is finite by part 1 
    of this result. 
    Since S is linear, $S0_X = 0_Y$. 
    Since S is continuous, 
    \begin{align*}
        S(\Ker_X) &= S\pa{\overline{\{0_X\}}}\\
        & \subset \overline{S\pa{\{0_X\}}}\\
        & =\overline{\{0_Y\}} \\
        & = \Ker_Y
    \end{align*}



\end{proof}
\begin{proof}[Proof of 7]
    $(3 \implies 2)$ is trivial, as is $(2 \implies 3)$.
    
    I now prove $(1\implies 3)$. 
    Let $\{T_i\}_{i \in \N}$ be a 
    \PseudometricCauchySequence.
    Let $\beta > 0$. 
    Let $\epsilon > 0$. 
    Then there exists $N \in \N$
    such that for $m,n > N$, 
    \begin{equation*}
    \norm{T_n-T_m} < \frac{\epsilon}{\beta}
    \end{equation*}
    Let $x \in B_X(0;\beta)$. 
    Then 
    \begin{align*}
        \norm{T_mx-T_nx}_Y & = \norm{(T_m-T_n)x}_Y\\
        & \leq \norm{T_m-T_n} \norm{x}_X\\
        & < \epsilon
    \end{align*}
    Since $x \in B_X(0;\beta)$ was arbitrary,
    $\{\{T_ix\}_{i \in \N} | x \in B_X(0;\beta)\}$ is
    \UniformlyCauchy.

    I now prove $(3 \implies 1)$. 
    Let $\epsilon > 0$.
    Then there is an $N \in \N$ 
    such that for $m,n>N$, 
    for each $x \in B_X(0;2)$, 
    \begin{equation*}
    \norm{T_mx-T_nx} < \epsilon
    \end{equation*}
    In particular, if $\norm{x}=1$, then 
    \begin{equation}
    \frac{\norm{(T_m-T-n)x}_Y}{\norm{x}_X} =\norm{(T_m-T_n)x}_Y < \epsilon
    \end{equation}
    Hence, by taking the supremum over such x
    and applying part 3 of this result, 
    $\norm{T_m-T_n} < \epsilon$. 
\end{proof}
\begin{proof}[Proof of 8]
    Let $T_i \to T$. 
    Let $x \in X$. 
    If $x \in \Ker_X$, then $T_i(x) \in \Ker_Y$ for $i \in \N$ and $T_x \in \Ker_Y$, 
    so convergence is obvious. 
    Suppose $\norm{x}_X > 0$. 
    Let $\epsilon > 0$. 
    Then there exists $N \in \N$ such that
    for $n>N$, $\norm{T_n-T} < \frac{\epsilon}{\norm{x}_X}$.
    For such n, 
    \begin{equation*}
    \norm{T_ix-T_x}_Y \leq \norm{T_i-T} \norm{x}_X < \epsilon
    \end{equation*}
\end{proof}
\begin{proof}[Proof of 9]
%Pick up here, and prove it
\end{proof}
\begin{proof}[Proof of 6] %Remove
    Let $T_i \subset BL(X,Y)$ conveerge, say $T_i \to T \in BL(X,Y)$. 
    Let $\epsilon > 0$. 
    Then there is an $N \in \N$
    such that for $n>N$, 
    we have 
    $\norm{T_i-T} < \epsilon$. 
    For these n, for any $x \in \overline{B_X}(0;1)$, we have
    \begin{align*}
    \norm{T_ix-Tx}_Y& = \norm{(T_i-T)x}_Y \\
    & \leq \norm{T_i-T} \norm{x}_X\\
    & < \epsilon 
    \end{align*}
    So that $T_ix \to Tx$ uniformly for $x \in \overline{B_X}(0;1)$. 

    Suppose conversely that we have a sequence
    $\{T_i\}_{i \in \N}$ 
    such that $T_ix \to y(x)$ for each $x \in \overline{B_X}(0;1)$. 
    I proceed through three separate claims.
    
    \bf claim 01 \rm if $T_ix \to y(x)$ uniformly for $x \in \overline{B_X}(0;1)$, then
    there is a linear operator $T:X \to Y$ such that pointiwise everywhere in X, $T_ix \to Tx$. 
    \bf FOR CLAIM 01 \rm: Take a hamel bnasis $\{x_\alpha\}$, normalize it to get $y(z_\alpha)$ define T in terms of linear combinations. SHow that T is linear. Then work on the boundedness arguement. 
    

    \bf claim 02 \rm The linear operator T guaranteed to exist in claim 01 is bounded, that is, 
    that $\norm{T}$ is well defined and finite.

    \bf claim 03 \rm  $\norm{T_i-T} \to 0$.


    \bf Proof of claim 01 \rm Let $x \in X$.
    Define $z=x/\norm{x}_X$. 
    Then $z \in \overline{B_X}(0;1)$. 
    Hence, by assumption, there is a (not necessisarily unique)
    $y(z) \in Y$ such that 
    $T_iz \to y(z)$. 
    Let $\epsilon > 0$. 
    Then, there is an $N \in \N$ such that for $n>N$, we have
    $\norm{T_nz-y(z)}_Y < \frac{\epsilon}{\norm{x}_X}$. 
    For these n, 
    \begin{align*}
    \norm{T_nx-\norm{x}_Xy(z)}_Y & = \norm{x}_X \norm{T_nz-y(z)} \\
    & < \epsilon
    \end{align*}
    Hence $T_i(x) \to \norm{x}_Xy(z)$
    , so our candidate map is $T:X \to Y$ defined by 
    \begin{equation*}
        \begin{dcases}
            T(x) = y(x)  & x \in \overline{B_X}0;1
        \end{dcases}
    \end{equation*}
    Hence there is a map $T:X \to Y$ such that $T_ix \to Tx$ pointwise for $x \in X$. 
    %This map is linear, 
    %as if $\alpha\in \F$ 
    %and $x,y \in X$, and 
    %$\epsilon > 0$, then
    %There is an $N \in \N$ 
    %such that for $n>N$, we have
    %\begin{equation}
    %FILL IN CONDITION
    %\end{equation}
    %and for these n, we have 
    %\begin{align*}
    %T(\alpha x + y)
    %\end{align*}
\end{proof}

\end{prop}


\label{def:handedquotientoperators}
\newcommand{\CodomainQuotientOperator}[0]{
    \bf \hyperref[def:handedquotientoperators]{Codomain Quotient Operator} \rm
}
\newcommand{\CodomainQuotientMap}[0]{
    \bf \hyperref[def:handedquotientoperators]{Codomain Quotient Map} \rm
}
\begin{df}[Codomain Quotient Operator]
    Let X and Y be \SeminormedSpaces.
    Define $\scQ_Y:BL(X,Y) \to BL(X, Y/\Ker_Y)$ by setting, 
    for each $x \in X$, 
    \begin{equation*} 
        \scQ_YTx = [Tx]
    \end{equation*}
    Let $T \in BL(X,Y)$. 
    We call $\scQ_Y$ the \CodomainQuotientMap of X and Y
    and we call $\scQ_YT$ the 
    \CodomainQuotientOperator
    of T.
\end{df}

\begin{prop}[Codomain Quotient Operator]
\label{prop:handedquotientoperators}
    Let X and Y be 
    \SeminormedSpaces
    with \CodomainQuotientMap $\scQ_Y$. 
    The following are true. 
    \begin{enumerate}
        \item $\scQ_Y$ is a well defined continuous linear surjective isometry. 
        \item If Y is a \NormedSpace, then $\scQ_Y$ is invertible with a continuous inverse. 
    \end{enumerate}
    \begin{proof}[Proof Of 1]
        Since $Tx \in Y$ for any $x \in X$, 
        $[Tx]_Y$ is defined for any $x \in X$. 
        Furthermore, if $q_y:Y \to Y/\Ker$
        is the \QuotientMap of Y under 
        \EquivalenceModKernel, then 
        $\scQ_YT = q_y \circ T$. 
        By \ref{prop:quotientnormspace}, 
        $q_y$ is linear and an isometryu, and hence continuous.
        Therefore, 
        $q_y \in BL(Y, Y/\Ker)$. 
        Hence $\scQ_Y$ is well defined. 

        For linearity, let $\alpha \in \F$
        and $S,T \in BL(X,Y)$. 
        Let $x \in X$. 
        Then, 
        \begin{align*}
            \scQ_Y\pa{\alpha T+S}x & = \bra{\pa{\alpha T+S}x}_Y\\
            & = \bra{\alpha Tx+ Sx}_Y\\
            & = \bra{\alpha Tx}_Y+ \bra{Sx}_Y\\
            & = \alpha \bra{Tx}_Y+\bra{Sx_Y}\\
            & = \alpha \scQ_YTx+ \scQ_YSx\\
            & = \pa{\alpha \scQ_YT+\scQ_YS}x
        \end{align*}

        For being an isometry, 
        let $T \in BL(X,Y)$ and 
        let $x \in X$. Then, since $\norm{\bra{Tx}_Y}_{Y/\Ker} = \norm{Tx}_Y$, 
        \begin{align*}
            \frac{\norm{\scQ Tx}_{Y/\Ker}}{\norm{x}_X} & = \frac{\norm{\bra{Tx}_Y}_{Y/\Ker}}{\norm{x}_X} \\
            & = \frac{\norm{Tx}_Y}{\norm{x}_X} 
        \end{align*}
        and thus taking the norm over
        x with $\norm{x}_X \neq 0$ will yield the
        same result. Hence $\norm{T} = \norm{\scQ_Y T}$. 

        For surjectivity, let $\tilde{T} \in BL(X, Y/\Ker_Y)$. 
        Let $\{x_{\alpha}\}_{\alpha \in A}$ be a hamel basis for $X$. 
        For each $\alpha \in A$, let $y_{\alpha} \in \tilde{T}x_{\alpha}$. 
        Define $T:X \to Y$ by 
        \begin{equation}
            T\pa{\sum_{i=1}^n \beta_{\alpha_i} x_{\alpha_i}} = \sum_{i=1}^n \beta_{\alpha_i} y_{\alpha_i}
        \end{equation}
        T is obviously linear
        and has the property $[Tx]=\tilde{T}x$. 
        and since $\tilde{T} \in BL(X,Y/\Ker)$, 
        $\tilde{T}\Ker_X \subset \Ker_{ (Y/\Ker_Y)}=0$. 
        Hence $T \Ker_X \subset \Ker_Y$. 
        Furthermore, if $x \in X$ with $\norm{x}_X \neq 0$, then
        \begin{align*}
        \frac{\norm{Tx}_Y}{\norm{x}_X} & = \frac{\norm{\bra{Tx}_Y}_{Y/\Ker}}{\norm{x}_X}\\
        & = \frac{\norm{\tilde{T}x}_{Y/\Ker}}{\norm{x}_X}
        \end{align*}
        Therefore $T$ is bounded.
        Hence $T \in BL(X,Y)$, and $\scQ_YT=\tilde{T}$. 
        Thus we have surjectivity, and are done.
    \end{proof}
    \begin{proof}[Proof Of 2]
        If $Y$ is a \NormedSpace, 
        %then $q_y:Y \to Y/\Ker_Y$ is 
        a linear isometric homeomorphism by 
        \ref{prop:quotientnormspace}. 
        In particular, in this case, 
        $q_y$ is injective, meaning that 
        if $T,S \in BL(X,Y)$ where
        $T \neq S$, then 
        $Tx_0 \neq Sx_0$ for some $x_0 \in X$. 
        For this $x_0$, $q_yTx_0 \neq q_ySx_0$, so 
        $\scQ_YT \neq \scQ_YS$. 
        Therefore $\scQ_Y$ is injective, and therefore a bijection. 
        The inverse of an isometry is also an isometry 
        and therefore continuous, finishing this proof. 
    \end{proof}
\end{prop}

\label{def:quotientoperator}
\newcommand{\QuotientOperator}[0]{\textbf{\hyperref[def:quotientoperator]{Quotient Operator}}\xspace}
\newcommand{\OperatorQuotientMap}[0]{\textbf{\hyperref[def:quotientoperator]{Operator Quotient Map}}\xspace}
\begin{df}[Quotient Operator]
    Let $X,Y$ be \SeminormedSpaces
    with \SeminormKernels $\Ker_X$, $\Ker_Y$. 
    Define $Q:BL(X,Y) \to BL(X/\Ker_X, Y/\Ker_Y)$ by 
    setting, for $T \in BL(X,Y)$, 
    for $x \in X$, 
    \begin{equation}
    QT\bra{x}_X=\bra{Tx}_Y
    \end{equation}
    We call Q the \OperatorQuotientMap of X and Y and
    we call QT the \QuotientOperator of T. 
\end{df}



\begin{prop}[Quotient Operator]
\label{prop:quotientoperator}
    Let $X,Y$ be \SeminormedSpaces
    with \SeminormKernels $\Ker_X$, $\Ker_Y$
    and \OperatorQuotientMap Q. 
    Then Q is a well-defined linear surjective isometry. 
    \begin{proof} 
        We first show that Q is well defined. 
        Let $T \in BL(X,Y)$ and 
        let $x_0, x_1 \in X$ such that $\bra{x_0}=\bra{x_1}$. 
        Then $\norm{x_0-x_1}_X = 0$, so since T is continuous, 
        $\norm{Tx_0-Tx_1}_Y = 0$. 
        Hence $Tx_0 \cong Tx_1$, so
        $\bra{Tx_0} = \bra{Tx_1}$. 


        For linearity, let $\alpha\in \F$, and let
        $T,S \in BL(X,Y)$. 
        Let $x \in X$. 
        Then 
        \begin{align*}
            Q\pa{\alpha T+S}\bra{x}_X & = \bra{\pa{\alpha T+S}x}_Y\\
            & = \alpha \bra{Tx}_Y + \bra{Sx}_Y\\
            & = \alpha QT[x]_X +QS[x]_X\\
            & = \pa{\alpha QT+QS}[x]_X
        \end{align*}
        Since $x \in X$ was arbitrary, Q is linear. 

        As for being an isometry, let $T \in BL(X,Y)$ and let $x \in X$. 
        Since $\norm{\bra{x}}=\norm{x}$ and $\norm{Tx}=\norm{\bra{Tx}}$, 
        we have 
        \begin{align*}
        \frac{\norm{QT\bra{x}_{X/\Ker_X}}_{Y/\Ker_Y}}{\norm{\bra{x}}_{X/\Ker_X}}  & =  \frac{\norm{\bra{Tx}}_{Y/\Ker_Y}}{\norm{\bra{X}}_{X/\Ker_X}}\\
        & = \frac{\norm{Tx}_Y}{\norm{x}_X} 
        \end{align*}
        and so taking the supremum over $\norm{x} \neq 0$ gives us 
        that this is an isometry. 
        

        For surjectivity, let $\tilde{T} \in BL(X/\Ker_X, Y/\Ker_Y)$.
        Let $\{x_\alpha\}_{\alpha \in A}$ be a Hamel basis for X. 
        For each $\alpha \in A$, 
        let $y_\alpha \in \tilde{T}[x_\alpha]_X$. 
        Now define 
        \begin{equation}
            T\sum_{i=1}^n \beta_i x_{\alpha_i} = \sum_{i=1}^n \beta_i y_{\alpha_i}
        \end{equation}
        Then $T:X \to Y$ is obviously linear, and
        $Tx \in \tilde{T}[x]_X$ for $x \in X$. 
        Hence, 
        \begin{equation}
            \frac{`\norm{Tx}_{Y}}{\norm{x}_X} = \frac{\norm{\tilde{T}[x]_X}_{Y/\Ker_Y}}{\norm{[x]_X}_{X/\Ker_X}}
        \end{equation}
        so T is bounded, and hence $T \in BL(X,Y)$, 
        but that also implies that by definition, 
        $QT=\tilde{T}$, so we have proven surjectivity. 
    \end{proof}

\end{prop}

\label{def:canonicalisomorphism}
\newcommand{\CanonicalIso}[0]{
    \bf \hyperref[def:canonicalisomorphism]{Canonical Isomorphism Of The Quotient Space Of Continuous Linear Operators} \rm
}

\begin{df}[Canonical Isomorphism Of The Quotient Space Of Continuous Linear Operators]
    Let $X,Y$ be \SeminormedSpaces
    with \SeminormKernels $\Ker_X$, $\Ker_Y$.
    Let $\Ker$ denote the \SeminormKernel of $BL(X,Y)$. 
    Let Q denote the \OperatorQuotientMap of X and Y.
    Define $\Theta_{(X,Y)}:BL(X,Y)/\Ker \to BL(X/\Ker_X, Y/\Ker_Y)$ by 
    setting, for each $T \in BL(X,Y)$. 
    \begin{equation}
        \Theta_{(X,Y)}(\bra{T}) = QT
    \end{equation}
    We call $\Theta_{(X,Y)}$ the \CanonicalIso from X to Y. 
    When X and Y are understood, we may denote the
    \CanonicalIso simply with $\Theta$. 
    By \ref{prop:canonicalisomorphism}, $\Theta_{(X,Y)}$
    is an isomorphism of \NormedSpaces.
    That is, $\Theta$ is Linear, Bijective, Bicontinuous, and an isometry. 
\end{df}




\begin{prop}[Canonical Isomorphism Of The Quotient Space Of Continuous Linear Operators]
\label{prop:canonicalisomorphism}
    Let $X,Y$ be \SeminormedSpaces.
    Let $\Theta$ denote the \CanonicalIso
    from X to Y. 
    Then $\Theta$ is
    a bijective, bicontinuous, linear, isometry. 
    \begin{proof}
       By \ref{prop:quotientnormspace}, part 1, 
       $Y/\Ker_Y$ is a \NormedSpace, 
       Hence by \ref{prop:BLO}, part 2, 
       \newline
       $BL(X/\Ker_X, Y/\Ker_Y)$ is a \NormedSpace. 
       Similarly, by \ref{prop:quotientnormspace}, part 1, 
       $BL(X,Y)/\Ker$ is a normed space. 
       Hence, it is sufficient to show that $\Theta$ is a
       well-defined surjective linear isometry. 

       For well definedness, let $T,S \in BL(X,Y)$ with $[T]=[S]$. 
       Then, $\norm{T-S}=0$, so 
       if $x \in X$, $\norm{Tx-Sx}=0$. 
       Hence $Tx \cong Sx$ and since x was arbitrary, 
       $QT=QS$. 
        
       Let q denote the \QuotientMap $q:BL(X,Y) \to BL(X,Y)/\Ker$. 
       By parts 4, 5, and 6 of \ref{prop:quotientnormspace}, 
       q is a linear surjective isometry. 
       Also, by definition, $\Theta \circ q = Q$. 
       Since Q is surjective, $\Theta$ is surjective. 
       Since $Q$ is an isometry, and $q$ is a surjecive isometry, 
       $Theta$ is an isometry. 
       Since Q is linear, and since q is surjective and linear, 
       $\Theta$ is linear. 
    \end{proof}


\end{prop}

\newcommand{\SemiTopDualSpace}[0]{\textbf{\hyperref[def:topologicaldualspace]{Topological Dual Space}}\xspace}
\begin{df}\bf REMOVE \rm
\label{def:topologicaldualspace}
\rm
\end{df}

\label{def:dualspace}
\newcommand{\TopDualSpace}[0]{
    \bf \hyperref[def:dualspace]{Topological Dual Space} \rm
}
\begin{df}[Dual Space]
    Let $(X,\norm{\cdot})($ be a 
    \SeminormedSpace.
    We call $BL(X, \F)$ the 
    \TopDualSpace of 
    $(X,\norm{\cdot})$,
    and we denote 
    $BL(X,\F)$ with the symbol 
    $X^*$. 
    If $x^* \in X^*$, then
    we use the notational convention
    of writing, for $x \in X$. 
    \begin{equation}
    \ip{x, x^*}:= x^*(x)
    \end{equation}
\end{df}
\begin{rmk}[\TopDualSpace is a \NormedSpace]
    Let $X$ be a 
    \SeminormedSpace.
    Then, using 
    \ref{prop:BLO}, 
    since $\F$ is a 
    \NormedSpace, 
    so is $X^*$. 
\end{rmk}

\newcommand{\Reflexive}[0]{\textbf{\hyperref[def:Reflexive]{Reflexive}}\xspace}
\newcommand{\Reflexivity}[0]{\textbf{\hyperref[def:Reflexive]{Reflexivity}}\xspace}

\begin{df}[\Reflexive space]
\label{def:Reflexive}
\rm
Let $X$ be a \SeminormedSpace. 
Let $X^*$ denote the \TvsDual of $X$. 
We denote with $X^{**}$ the \TvsDual of $X^*$, when 
$X^*$ is endowed with the \Norm \Topology. 
Let $c:X \to X''$ denote the \CanonicalEmbedding of $X$ into 
$X''$, the second \AlgebraicDual of $X$.
If $x_0 \in X$ $j_1,j_2 \in X^*$, then 
\begin{equation*}
\abs{\ip{j_1-j_2, c(x_0)}} = \abs{\ip{x_0, j_1-j_2}} \leq \norm{x_0} \norm{j_1-j_2}
\end{equation*}
so that $c(x_0) \in X^{**}$. 
Thus, when $X$ is a \SeminormedSpace, we view $c$
as a function from $X$ into the second \TvsDual
instead of into the second \AlgebraicDual. 
Furthermore, when $c$ is \Surjective, 
we say that $X$ is \Reflexive. 
\end{df}
\begin{thm}[\TopDualSpace Isomorphism]
\label{thm:dualspaceisomorphism}
    Let X be a 
    \SeminormedSpace.
    Define $\Omega:X^* \to \pa{X/\Ker_X}^*$
    by setting, for $x^* \in X$, 
    and for $x \in X$, 
    \begin{equation}
        \ip{x, x^*} = \ip{[x], \Omega x^*}
    \end{equation}
    Then $\Omega$ is a 
    Linear, 
    Bijective, 
    Isometric, 
    Bicontinuous operator. 
    That is, $X^*$ and $(X/\Ker_X)^*$ are 
    isomorphic, and that isomorphism is explicitly
    given by $\Omega$. 
    \begin{proof}
        Consider the following
        \begin{equation}
            BL(X,\F) \overset{q}{\to} BL(X,\F)/\Ker_{BL(X/\F)} \overset{\Theta}{\to} BL(X/\Ker_X, \F/\Ker_{\F}) \overset{\scQ_{\F}^{-1}}{\to} BL(X/\Ker_X,\F)
        \end{equation}
        where
        q is the \QuotientMap, 
        which is an linear bijective bicontinuous isometry in this case
        by parts 4, 5, 6, and 7 of 
        of \ref{prop:quotientnormspace}, 
        $\Theta$ is the \CanonicalIso, 
        which is a linear bijective bicontinuous isometry by 
        \ref{prop:canonicalisomorphism}
        and $\scQ_{\F}$ is the \CodomainQuotientMap.
        which is in this case a linear, bijective, bicontinuous isometry
        by \ref{prop:handedquotientoperators}


        Since $\Omega=\scQ_{\F}^{-1} \circ \Theta \circ q$, 
        and since each of the described properties
        are preserved under composition, 
        $\Omega$ is also a 
        linear bijective bicontinuous isometry. 
    \end{proof}
\end{thm}

\subsection{Minkowski Functionals}
\newcommand{\MinkowskiFunctional}[0]{\textbf{\hyperref[def:MinkowskiFunctional]{Minkowski Functional}}\xspace}
\begin{df}[\MinkowskiFunctional]
\label{def:MinkowskiFunctional}
\rm
Let $X$ be a \VectorSpace over $\F \in \{ \R, \C\}$. 
Let $A \subset X$ be \Absorbing and \ConvexSet.
Define $\mu_A:X \to [0,\infty)$ by 
\begin{equation*}\
\mu_A(x) = \inf\limits \braces{ \lambda  \in (0,\infty): \frac{1}{\lambda} x \in A}
\end{equation*}
We call $\mu_A$ the \MinkowskiFunctional 
of $A$. 
\end{df}
\begin{prop}[\MinkowskiFunctional]
\label{prop:MinkowskiFunctional}
\rm
Let $X$ be a \VectorSpace over $\F \in \{ \C, \R\}$. 
Let $A \subset X$ be 
\ConvexSet, 
and \Absorbing.
Let $\mu_A$ denote the 
\MinkowskiFunctional
of $A$. 
Then $\mu_A$ is a \SublinearFunctional.
If in addition we assume that $A$ is 
\Balanced,
then $\mu_A$ is a \Seminorm on $X$. 
\begin{proof}
It is clear that $\mu_A$ is well defined since $A$ is \AbsorbingSet.
Now let $x_0,y_0 \in X$. 
Let $\epsilon > 0$. 
Then there exists $\lambda_x, \lambda_y > 0$ such that 
$\frac{1}{\lambda_x} x_0 \in A$ and $\frac{1}{\lambda_y}  y_0 \in A$.
and 
\begin{equation*}
\gamma_x < \mu_A(x_0) + \epsilon \tab[1cm] \gamma_y < \mu_A(y_0)+\epsilon
\end{equation*}
Since $A$ is \ConvexSet, 
\begin{align*}
\frac{x_0+y_0}{\lambda_x+\lambda_y} &= \frac{\lambda_x}{\lambda_x+\lambda_y} \frac{x_0}{\lambda_x} + \frac{\lambda_y}{\lambda_x+\lambda_y} \frac{y_0}{\lambda_y}\\
& = \frac{\lambda_x}{\lambda_x-\lambda_y} \frac{x_0}{\lambda_x} + \pa{ 1- \frac{\lambda_x}{\lambda_x+\lambda_y}}\frac{y_0}{\lambda_y}  \\
& \in A
\end{align*}
Hence $\mu_A(x_0+y_0) \leq \lambda_x+\lambda_y \leq \mu_A(x_0)+\mu_A(y_0) + 2 \epsilon$. 
Since $\epsilon > 0$ was arbitrary, 
$\mu_A$ is \Subadditive. 
Now let $s > 0$. 
Then 
\begin{align*}
\mu_A(sx_0) & = \inf\braces{ \lambda > 0 : \frac{1}{\lambda} sx_0 \in A}\\
& = \inf \braces{\frac{s \lambda}{s} > 0 : \frac{s}{\lambda} x_0 \in A} \\
& =  \inf \braces{s \tilde{\lambda} > 0 : \frac{1}{\tilde{\lambda}} x_0 \in A} \\
& =s \inf \braces{\tilde{\lambda} > 0 : \frac{1}{\tilde{\lambda}} x_0 \in A } \\
& = s \mu_A(x_0)
\end{align*}

Suppose now that $A$ is \Balanced.
Since $A$ is \ConvexSet and \Absorbing, $0 \in A$. 
Hence, $\mu_A(0)=0$. 
Let $t \in \F$.
If $t=0$, then by the above argument, 
$\mu_A(0x) = 0=0\mu_A(x)$. 
Let $x_0 \in X$ and let $t \in \F$ be nonzero. 
Write $t=t_0t_1$ where $\abs{t_1} = 1$ and $t_0 \geq  0$. 
Suppose now, that $t \neq 0$.
Then, since $A$ is \Balanced, $\frac{1}{t_1}A = A$, so 
\begin{align*}
\mu_A(tx_0) & = \inf\braces{\lambda > 0 : \frac{1}{\lambda} tx_0 \in A}\\
& = \inf \braces{\lambda > 0 : \frac{t_0}{\lambda} x \in \frac{A}{t_1}} \\
& = \inf \braces{\lambda > 0 : \frac{t_0}{\lambda} x \in A}\\
& = t_0 \mu_A(x)\\
& = \abs{t} \mu_A(x)
\end{align*}
Hence $\mu_A$ is \AbsScalarHomogeneous
and is therefore a 
\Seminorm.
\end{proof}
\end{prop}
\begin{prop}[\Seminorm Induces \MinkowskiFunctional]
\label{prop:SeminormInducesMinkowski}
\rm
Let $(X, \norm{\cdot})$ be a \SeminormedSpace. 
Let $B$ denote the \ClosedUnitBall of $X$. 
Then $B$  is \Absorbing, \BalancedSet, and \ConvexSet and 
therefore induces a \MinkowskiFunctional $\mu_B$. 
Furthermore $\mu_B=\norm{\cdot}$. 
\begin{proof}
For each $x_0 \in X$, we have $x_0 \in \norm{x_0} B$, so $B$ is \Absorbing. 
That $B$ is \Balanced is a direct consequence of the 
\AbsScalarHomogeneity
of $\norm{\cdot}$. 
Finally, by the \TriangleInequality, if $x_0 ,y_0 \in B$, and 
if $\lambda \in (0,1)$, we have 
\begin{equation*}
\norm{\lambda x_0 + (1-\lambda) y_0} \leq \lambda \norm{x_0} + (1-\lambda) \norm{y_0} = 1
\end{equation*}
so $B$ is \ConvexSet. 
Now for every $\lambda > \norm{x_0}$, $\norm{\frac{x_0}{\lambda}} < 1$, so $\frac{1}{\lambda} x_0 \in B$. 
Hence $\mu_B(x_0) \leq \norm{x_0}$. 
If $\lambda < \norm{x_0}$, then $\norm{\frac{x_0}{\lambda}} > 1$, so $\frac{x_0}{\lambda} \not \in B$. 
Hence $\norm{x_0} \leq \mu_B(x_0)$. 
hence equality holds. 
\end{proof}
\end{prop}
\begin{prop}[\LocallyConvex from \Seminorms]
\label{prop:LocallyConvexFromSeminorms}
\rm
Let $V$ be a 
\VectorSpace over $\F \in \{\R, \C\}$.
Let $\scF = \{\norm{\cdot}_\gamma\}_{\gamma \in B}$ be a  
collection of \Seminorms on $V$.
Then the \WeakTopology  on $V$
induced by $\scF$ makes $V$ into a 
\TVS.
\begin{proof}
The inequalities 
\begin{equation*}
\norm{x+y - \pa{x'+y'}}_\gamma \leq \norm{x-x'}_\gamma + \norm{y-y'}_\gamma
\end{equation*}
and 
\begin{equation*}
\norm{ \lambda x- \lambda' x'}_\gamma \leq \abs{\lambda - \lambda ' } \norm{x}_\gamma + \abs{\lambda ' } \norm{x-x'}_\gamma
\end{equation*}
imply that vector addition and scalar multiplication are 
\ContinuousFunction in the \WeakTopology 
generated by $\scF$. 
\end{proof}
\end{prop}
\begin{prop}[\LocallyConvex \Seminorm]
\label{prop:LocallyConvexSeminorm}
\rm
Let $(X, \scT)$ be a \LocallyConvex 
\TVS.
Then there exists a collection of 
\Seminorms 
$\scF=\{\norm{\cdot}_\alpha\}_{\alpha \in A}$ 
on $X$ such that 
$\scT$ is the \WeakTopology on $X$ 
induced by $\scF$.
\begin{proof}[First Claim]
By \ref{prop:Bal4}, $X$ permits a 
\LocalBasis $B_X= \{B_\alpha\}_{\alpha \in A}$ consisting of 
\ConvexSet, \Absorbing, \Balanced sets.
Define $\scF = \{ \mu_{B_\alpha}\}_{\alpha \in A}$. 
By \ref{prop:MinkowskiFunctional}, each 
$\mu_{B_\alpha}$ is a \Seminorm on $X$. 
Denote the \WeakTopology on $X$ generated by 
$\scF$ with $\scT_F$. 
\ref{prop:LocallyConvexFromSeminorms}
implies that $(X, \scT_F)$ is a \TVS.
For this reason, it is sufficient to show that for some
\LocalBasis $\scB_{\scF}$ of $\scT_F$, we have
\begin{equation*}
 \scNested{\scB_X}{\scB_F} \tab[1cm] \scNested{\scB_F}{\scB_X}
\end{equation*}
For the first claim, given $\alpha \in A$ and $\epsilon > 0$, 
$\frac{\epsilon}{2} B_{\alpha} \subset \mu_{B_\alpha}^{-1}[0, \epsilon)$, so 
given $\{\epsilon_i\}_{i=1}^n \subset (0,\infty)$ and $\{\alpha_i\}_{i=1}^n \subset B$, 
\begin{equation*}
\bigcap\limits_{i=1}^n \frac{\epsilon_i}{2}B_{\alpha_i}  \in \scT
\end{equation*}
So there exists $\gamma$ with 
\begin{equation*}
B_\gamma \subset \bigcap\limits_{i=1}^n \frac{\epsilon_i}{2}B_{\alpha_i}  \subset \bigcap\limits_{i=1}^n \mu_{B_{\alpha_i}}^{-1}\pa{[0,\epsilon_i)}
\end{equation*}
Hence $ \scNested{\scB_X}{\scB_F}$ holds. 
For the other direction, given $\gamma \in B$, 
$\mu_{B_\gamma}^{-1}[0,1) \subset B_\gamma$, so $\scNested{\scB_F}{\scB_X}$ holds. 
\end{proof}
\end{prop}
\begin{rmk}[\Seminorm Frechet]
\label{rmk:SeminormFrechet}
\rm
Let $X$ be a \LocallyConvex
\TVS.
An inspection of the proof of 
\ref{prop:LocallyConvexFromSeminorms}
reveals that if $\scB$ is a \LocalBasis 
for $X$, then there exists a collection of \Seminorms
$\scF = \{\norm{\cdot}_\alpha\}_{\alpha \in A}$ 
whose \Cardinality equals that of $\scB$ 
such that the \Topology on $X$ is the 
\WeakTopology on $X$ generated by $\scF$. 
In particular, if $X$ is \FirstCountable, 
then $\scF$ is \Countable as well. 
From $\scF$ we can define the translation invariant \Pseudometric 
\begin{equation*}
d(x,y) = \sum\limits_{i \in \N} \frac{1}{2^i} \frac{ \norm{x-y}_i}{1+\norm{x-y}_i}
\end{equation*}
on $X$. 
Much work has gone into the analysis \LocallyConvex \Metrizable spaces, 
but most of this work has been focused on the utility offered by the metrizability
of such spaces or geometric properties common to all \Seminorms, 
rather than the convenient geometric properties which may be possessed by the 
particular \Seminorms which define the \Metric on such spaces. 
This is not surprising, as many such properties are not preserved under 
\Homeomorphism, and therefore are not intrinsic to the space itself. 
I however, believe there is much to be gained from such investigation. 
This motivates the following definition. 
\end{rmk}
\newcommand{\GSpace}[0]{\textbf{\hyperref[def:GSpace]{GSpace}}\xspace}
\newcommand{\GSpaces}[0]{\textbf{\hyperref[def:GSpace]{GSpaces}}\xspace}
\begin{df}[\GSpace]
\label{def:GSpace}
\rm
Let $(X, \scT)$ be a 
\FirstCountable 
\LocallyConvex \TVS.
Let $\{\norm{\cdot}_i\}_{i \in \mathbb{N}}$ be an increasing sequence of 
\Seminorms on $X$ such that 
$\scT$ is the \WeakTopology on $X$ 
generated by $\{\norm{\cdot}_i\}_{i \in \mathbb{N}}$. 
Then we call $\pa{X, \scT, \{\norm{\cdot}_i\}_{i \in \mathbb{N}}}$
a \GSpace. 
When confusion is unlikely, we may call 
$X$ a \GSpace, $(X,\scT)$ a \GSpace, or 
$\pa{X,\{\norm{\cdot}_i\}_{i \in \mathbb{N}}}$ a \GSpace.
Generally speaking, we will say that a 
\GSpace $(X, \{\norm{\cdot}_i\}_{i \in \N}$ has a certain property if 
for each $i \in \N$ the \SeminormedSpace 
$(X, \norm{\cdot}_i)$ posesses that property. 
An exception to this convention is that of completeness. 
A \GSpace is said to be \Complete if the \Metric $d$ defined by 
$d(x,y)= \sum\limits_{i \in \N} \frac{1}{2^i} \frac{\norm{x-y}_i}{1+\norm{x-y}_i}$ is \Complete.
\end{df}
\subsection{Seminormed Hahn Banach Theorem}
\begin{thm}[Hahn Banach Theorem For Seminormed Spaces]
\label{thm:hahnbanach}
\rm
Let $(X,\norm{\cdot})$ be a \SeminormedSpace,
let $x_i \in X$ for $i \in \{0,1\}$ such that 
$\norm{x_0-x_1}_X \neq 0$, and
let $X^*$ denote $X's$
\TopDualSpace. 
The following are true. 
\begin{enumerate}[label=(\roman*), ref={\ref{thm:hahnbanach}~\roman*}]
    \item 
	\label{thm:HahnBanach:Extension1}
	If $Z \subset X$ is a subspace
        and $z^* \in Z^*$, then there 
        is an extension $x^*$ of $z^*$, 
        $x^* \in X^*$ such that 
        \begin{equation*}
        \norm{z^*}_{Z^*} = \norm{x^*}_{X^*}
        \end{equation*}
     \item 
	 \label{thm:HahnBanach:Point1}
	 If $x \in X$, 
        with $\norm{x} \neq 0$, 
        then there exists an
        $x^* \in X$ with 
        $\norm{x^*}=1$ and 
        $\ip{x,x^*} = \norm{x}_X$. 
    \item 
	\label{thm:HahnBanach:Norm}
	If $x \in X$, then 
    \begin{equation*}
        \norm{x}_X = \sup\limits_{0 \neq x^* \in X^*} \frac{\ip{x,x^*}}{\norm{x^*}}
    \end{equation*}
	
    \item 
	\label{thm:HahnBanach:Operator1}
	If $Y$ is a 
        \NonDegenerate
        \SeminormedSpace, and if 
        $x_0 \in X$, with 
        $\norm{x_0} \neq 0$, 
        then there exists
        an $S \in BL(X,Y)$ with 
        $\norm{S} = 1$ and 
        \begin{equation*}
            \norm{Sx_0} = \norm{x_0}
        \end{equation*}
\end{enumerate}


\begin{proof}[Proof of \ref{thm:HahnBanach:Extension1}]
    For $\alpha \in \{Z,X\}$, let 
    $\Omega_\alpha:\alpha^* \to (\alpha/\Ker_\alpha)^*$ denote the isomorphism
    defined in 
    \ref{thm:dualspaceisomorphism}.
    Let $q$ denote the quotient operator $q:X \to X/\Ker$. 
    Define $T:Z/\Ker_Z \to q(Z)$ bv $T([z]_{\cong_Z} ) = [z]_{\cong_X}$. %Make a separate Result
    Since Z is endowed with the subspace Topology,                       %Make a separate Result
    T is obviously a \Linear \Isometric \Homeomorphism.          %Make a separate Result
    
    %Then $\Omega_Zz^* \in (Z/\Ker_Z)^*$ satisfies
    %$\norm{\Omega_Zz^*}_{Z/\Ker_Z}=\norm{z^*}_{Z^*}$ an 
    Define $\Gamma_Z:(Z/\Ker_Z)^* \to q(Z)^*$ by setting, 
    for $\phi^* \in (Z/\Ker_Z)^*$, 
    for $[z]_Z \in Z/\Ker_Z$, 
    \begin{equation*}
        \ip{T[z]_Z, \Gamma_Z \phi^*} = \ip{[z]_Z, \phi^*}
    \end{equation*}
    Then $\Gamma_Z$ is a \Linear Bijective \Isometry. 
    Hence $\Gamma_Z \circ \Omega_Z z^* \in q(Z)^*$ with 
    $\norm{\Gamma_Z \circ \Omega_Z z^*}_{q(Z)^*} = \norm{z^*}_{Z^*}$. 

    Thus we can apply the Hahn Banach theorem for \NormedSpaces to claim 
    the existence of $x_q^* \in (X/\Ker_X)^*$ where
    $x_q^*$ is an extension of $\Gamma_Z \circ \Omega_Z z^*$ and
    \begin{equation*}
        \norm{x_q^*}_{(X/\Ker_X)^*} = \norm{\Gamma_Z \circ \Omega_Z z^*}_{(q(Z))^*} = \norm{z^*}_{Z^*}
    \end{equation*}
    Finally, letting $x^* = \Omega_X^{-1} x_q^*$, we have 
    $x^* \in X^*$, 
    $\norm{x^*}_{X^*} = \norm{x_q^*}_{(X/\Ker_X)^*} =\norm{z^*}_{Z^*}$, 
    and 
    if $z \in Z$, then 
    \begin{align*}
        \ip{z, x^*} & = \ip{[z]_X, x_q^*} \\
        & = \ip{[z]_X, \Gamma_Z \circ \Omega_Zz^*}\\
        & = \ip{[z]_Z, \Omega_Z z^*}\\
        & = \ip{z, z^*}
    \end{align*}
\end{proof}

\begin{proof}[Proof of 2]
    Let $Z=span(x)$. 
    Define $z^* \in Z^*$ by 
    $\ip{\alpha x, z^*} = \alpha \norm{x}$. 
    Then $\norm{z^*} = 1$. 
    Also, by part 1 of this result, 
    it has an extension $x^* \in X^*$ with 
    $\norm{x^*} = \norm{z^*} =1$ 
    and $\ip{x,x^*} = \norm{x}$. 
\end{proof}
\begin{proof}[Proof of 3]
    If $\norm{x} = 0$, then
    for every $x^* \in X$, $\ip{x,x^*} = 0$.
    Hence 
    \begin{equation*} 
    \norm{x}_X = \sup\limits_{0 \neq x^* \in X^*} \frac{\ip{x,x^*}}{\norm{x^*}} = \sup\limits_{x^* \in \partial B_{X^*}(0;1)} \frac{\ip{x,x^*}}{\norm{x^*}}=0
    \end{equation*}

    Otherwise, let  $x^* \in X^*$ guaranteed to 
    exist by part 2 which satisfies $\norm{x^*}=1$, 
    $\ip{x,x^*} = \norm{x}$. 
    Then 
    \begin{align*}
    \norm{x} & = \frac{\ip{x,x^*}}{\norm{x^*}} \\
    & \leq \sup\limits_{x^* \in \partial B_{X^*}(0;1)} \frac{\ip{x,x^*}}{\norm{x^*}} \\
    & \leq    \sup\limits_{0 \neq x^* \in X^*} \frac{\ip{x,x^*}}{\norm{x^*}} 
    \end{align*}
    The other direction of the inequality
    falls directly from the definition of 
    the norm on $X^*$, and is trivial, so 
    we are done. 
\end{proof}
\begin{proof}[Proof of 4]
    By part 2 of this result, there exists $x_0^* \in X^*$ with 
    $\norm{x_0^*} = 1$ and $\ip{x_0, x_0^*}=\norm{x_0}$. 
    Since Y is \NonDegenerate, there
    exists $y_0 \in Y$ with $\norm{y_0} = 1$. 
    Define $T: \F \to Y$ by $T \alpha = \alpha y$.
    Then $\norm{T} = \norm{y} = 1$. 
    Define $S: X \to Y$ by $S=T \circ x_0^*$. 
    Then $\norm{S} \leq \norm{T} \norm{x_0^*} = 1$, and
    $\norm{S x_0} = \norm{\ip{x_0, x_0^*} y} = \ip{x_0, x_0^*} = \norm{x_0}$. 
    Hence $\norm{S} \geq 1$ and therefore $\norm{S} = 1$. 
\end{proof}
\end{thm}


\begin{prop}[\SetClosed and \ConvexSet is \weak \SetClosed]
\label{prop:ClosedAndConvexIsWeaklyClosed}
\rm
Let $X$ be a \SeminormedSpace. 
Let $K \subset X$ be 
\ConvexSet and \SetClosed. 
Then $K$ is \weak \SetClosed. 
\begin{proof}
Let $y_0 \in  X \setminus  K$. 
Then since $\{y_0\}$ is \SetCompact and \ConvexSet and $K$ is \ConvexSet
and \SetClosed, we can apply 
\ref{thm:HahnBanach:ClosedConvex}
to conclude the existence of $j_y \in X^*$ and $\gamma_y^1, \gamma_y^2 \in \mathbb{R} $ 
such that 
$\sup\limits_{k \in K}(\ip{k, j_y}) = \gamma_y^1 < \gamma_y^2 < \ip{y_0, j_y}$.
Hence
$y_0 \in j_y^{-1}(\gamma_y^2, \infty)$ and $K \cap j_y^{-1}(\gamma_y^2, \infty) = \emptyset$. 
Hence, we can write 
\begin{equation*}
X \setminus K = \bigcup\limits_{y \in X \setminus K } j_y^{-1}(\gamma_y^2, \infty)
\end{equation*}
which is a union of \weak \SetOpen sets, and is therefore \weakly \SetOpen. 
Hence $K$ is \weakly \SetClosed. 
\end{proof}
\end{prop}
\begin{rmk}
\label{rmk:ClosedAndConvexIsWeaklyClosed}
\rm
\ref{prop:ClosedAndConvexIsWeaklyClosed}
can be shown to hold in an arbitrary \LocallyConvex \TVS, 
using a more general form of the Hahn Banach theorem.
\end{rmk}
\begin{prop}[\weak]
\label{prop:WeakNorm}
\rm
Let $X$ be a \SeminormedSpace.
Let $x_0 \in X$
Let $x_0^* \in X^*$. 
Let $\{x_\alpha\}_{\alpha \in A}  \subset X$
such that $x_\alpha \wto x_0$. 
Let $\{x_\alpha^*\}_{\alpha \in A} \subset X^*$ such
that $x_\alpha^* \wsto x_0^*$. 
The following are true. 
\begin{enumerate}[label=(\roman*), ref={\ref{prop:WeakNorm}~\roman*}]
\item 
\label{prop:WeakNorm:WeakLiminf}
$\norm{x_0} \leq \liminf\limits \norm{x_\alpha}$.
\item 
\label{prop:WeakNorm:WeakStarLiminf}
$\norm{x_0^*} \leq \liminf\limits \norm{x_\alpha^*}$.
\end{enumerate}
\begin{proof}[Proof of \ref{prop:WeakNorm:WeakLiminf}]
By 
\ref{thm:HahnBanach:Norm}, 
we have 
\begin{align*}
\norm{x_0} & = \sup\limits_{x^* \in \partial B_{X^*}(0;1)} \ip{x_0, x^*} \\
& =  \sup\limits_{x^* \in \partial B_{X^*}(0;1)} \liminf\limits \ip{x_\alpha, x^*} \\
& \leq \sup\limits_{x^* \in B_{X^*}(0;1)} \liminf\limits \norm{x_\alpha} \norm{x^*}\\
& \leq \liminf\limits \norm{x_\alpha}
\end{align*}
\end{proof}
\begin{proof}[Proof of \ref{prop:WeakNorm:WeakStarLiminf}]
We have 
\begin{align*}
\norm{x_0^*} & = \sup\limits_{x \in \partial B_{X}(0;1)} \ip{x, x_0^*} \\
& =  \sup\limits_{x \in \partial B_{X}(0;1)} \liminf\limits \ip{x, x_\alpha^*} \\
& \leq \sup\limits_{x \in B_{X}(0;1)} \liminf\limits \norm{x_\alpha^*} \norm{x}\\
& \leq \liminf\limits \norm{x_\alpha^*}
\end{align*}
\end{proof}
\end{prop}
\subsection{Seminorm Adjoints}
\begin{prop}[Linear Operator Notation]
\label{rmk:linearoperatornotation}
    $.$
    When dealing with mappings of 
    spaces of linear operators into
    spaces of other linear operators, 
    or even functions in general, 
    notation can get confusing, and
    presenting such things using ordinary notation without
    ambiguity can often require a plethora of parenthesis, 
    which hamper readability of an arguement. 

    For this reason, at points in this document, 
    I sometimes express the image $\beta(\alpha)$ using 
    \begin{equation*}
        \ip{\alpha, \beta}
    \end{equation*}
    Where $\beta:X \to Y$ 
    and $\alpha \in X$. 

    I combine this notation with usual function notation, 
    particularly in cases similar to the following. 
    For $i \in \{0,1\}$, 
    let $X_i, Y_i, Z_i$ be sets. 
    For $\alpha \in \{X,Y,Z\}$, let 
    $F_\alpha$ be the set of maps $f:\alpha_0 \to \alpha_1$. 
    If $T:F_X \to F_Y$, 
    $y \in Y_0$, 
    and $f \in F_X$, then I would notate
    \begin{equation*}
        \ip{y, Tf}
    \end{equation*}
    rather than $Tf(y)$ or $(T(f)(y))$
\end{prop}

\label{def:adjointoperator}
\newcommand{\AdjointOperator}[0]{
    \bf \hyperref[def:adjointoperator]{Adjoint Operator} \rm
}
\begin{df}[\AdjointOperator]
    Let X, Y, and Z be \SeminormedSpaces
    over a field $\F \in \{\R, \C\}$. .
    Let $T \in BL(X,Y)$. 
    We define the operator
    $T^{\times}_Z:BL(Y,Z) \to BL(X,Z)$ by 
    setting, for $S \in BL(Y,Z)$ 
    and $x \in X$, 
    \begin{equation}
        \ip{x , \T^{\times}_{Z}S } = \ip{Tx, S}
    \end{equation}
    or, equivalently, 
    \begin{equation}
    \T^{\times}_Z S = S \circ T
    \end{equation}

    We call $T^\times_Z$ the \AdjointOperator
    of T relative to the space Z, we denote
    $T^\times_{\F}=T^\times$, and
    we refer to $T^\times:Y^* \to X^*$ as 
    simply the \AdjointOperator of T. 
\end{df}

\begin{prop}[\AdjointOperator]
\label{prop:adjointoperator}
\rm
Let $X$, $Y$, and $Z$ be \SeminormedSpaces
over a \Field $\F \in \{\R, \C\}$.
Let $T \in BL(X,Y)$. 
Let $\T=T^\times_Z$ denote the 
\AdjointOperator of T
relative to the space $Z$. 
Let $Q_Y:Y \to Y/\Ker_Y$ denote the \QuotientMap
The following are true. 
\begin{enumerate}[label=(\roman*), ref={\ref{prop:adjointoperator}~\roman*}]
\item 
\label{prop:Adjoint:Linear}
$\T$ is \Linear. 
\item 
\label{prop:Adjoint:WellDefined}
If $S \in BL(Y,Z)$, then $\T S \in BL(X,Z)$. (That is, the \AdjointOperator is well defined as a concept).
\item 
\label{prop:Adjoint:Continuous}
$\T \in BL\pa{ BL(Y,Z), BL(X,Z)}$. 
\item 
\label{prop:Adjoint:Isometry}
$\norm{\T}=\norm{T}$
\item 
\label{prop:Adjoint:DenseRange}
If $Range(T)$ is dense in $Y$, then $\inf\limits_{\norm{x}=1}\norm{Tx} \leq \inf\limits_{\norm{S}_{BL(Y,Z)} = 1} \norm{\T S}$.
%			To Range(T) dense in Y. 
\item 
\label{prop:Adjoint:NotDenseRange}
If Range(T) is not dense in Y, then 
There exists $S \in BL(Y,Z)$ with $\norm{S} = 1$ and $\norm{\T S} = 0$. 
$\inf\limits_{\norm{S}_{BL(Y,Z)}=1} \norm{\T S} =0$
\item 
\label{prop:Adjoint:Surjective}
$\T$ is \Surjective if and only if T is \Injective and has \SetClosed range in Y. 
\end{enumerate}


\begin{proof}[Proof of \ref{prop:Adjoint:Linear}]
Let $S,R \in BL(Y,Z)$, 
$\alpha \in \F$, 
and $x \in X$. 
Then, 
\begin{align*}
\ip{x, \T(\alpha S+R)} & = \ip{Tx, \alpha S+R}\\
& = \alpha \ip{Tx, S}+ \ip{Tx, R} \\
& = \alpha \ip{x, \T S} + \ip{ x, \T R}\\
& = \ip{x, \alpha \T S} + \ip{x, \T R} \\
& = \ip{x, \alpha \T S + \T R}
\end{align*}
Since $x \in X$ was arbitrary, \Linearity is verified. 
\end{proof}
\begin{proof}[Proof of \ref{prop:Adjoint:WellDefined}]
Let $S \in BL(Y,Z)$. 
Then, 
$\T S = S \circ T$. 
The composition of \ContinuousFunction operators is \ContinuousFunction, so $\T S$ is 
\ContinuousFunction.
The composition of \Linear operators is \Linear, so $\T S$ is \Linear.
Hence, $\T S \in BL(X,Z)$.
\end{proof}
\begin{proof}[Proof of \ref{prop:Adjoint:Continuous}]
Let $S \in BL(Y,Z)$. Then, 
if $x \in X$
\begin{equation*}
\norm{\ip{x, \T S}} = \norm{\ip{Tx, S}} \leq \norm{S} \norm{Tx} \leq \norm{S} \norm{T} \norm{x}
\end{equation*}
Hence $\norm{\T S} \leq \norm{S} \norm{T}$
Since T is \Linear, and since S was arbitrary, 
by part  12 of \ref{prop:BLO}, $\T \in BL\pa{ BL(Y,Z), BL(X,Z)}$.
\end{proof}
\begin{proof}[Proof of \ref{prop:Adjoint:Isometry}]
For any $S \in BL(Y,Z)$, 
$\T S = S \circ T$, so
$\norm{\T S} \leq \norm{S} \norm{T}$. 
Hence $\norm{\T} \leq \norm{T}$. 
Now let $x_0 \in X$. 
Then, by part 4 of 
\ref{thm:hahnbanach}, 
there exists $S \in BL(Y,Z)$ with 
$\norm{S}=1$
and $\norm{STx_0} = \norm{Tx_0}$. 
Hence, 
\begin{align*}
\norm{T x_0} & = \norm{S Tx_0} \\
& = \norm{(S \circ T) x_0} \\
& = \norm{(\T S) x_0} \\
& \leq \norm{\T} \norm{S} \norm{x_0}\\
& = \norm{\T} \norm{x_0}
\end{align*}
Since $x_0 \in X$ is arbitrary, $\norm{T} \leq \norm{\T}$. 
Since the inequality goes both ways, $\norm{T}=\norm{\T}$.
\end{proof}
\begin{proof}[Proof of \ref{prop:Adjoint:DenseRange}]
Let $\Gamma=\inf\limits_{\norm{x}=1} \norm{Tx}$, 
and let $S \in BL(Y,Z)$ with $\norm{S} = 1$. 
Then, 
\begin{equation*}
\{x | \norm{Tx} \leq \Gamma\} \subset B_X(0;1)
\end{equation*}
so 
\begin{equation*}
\sup\limits_{\norm{x}\leq 1} \abs{\ip{Tx, S}} \geq \sup\limits_{\norm{Tx} \leq \Gamma}\abs{\ip{Tx, S}}
\end{equation*}
Also, since $Range(T)$ is dense in $Y$ and $S$ is \ContinuousFunction,
\begin{equation*}
\sup\limits_{\norm{Tx} \leq \Gamma} \abs{\ip{Tx, S}} = \sup\limits_{\norm{y} \leq \Gamma} \abs{\ip{y, S}}
\end{equation*}
From these two we arrive at the inequality
\begin{align*}
\norm{\T S} & = \sup\limits_{\norm{x}  \leq 1} \abs{\ip{x, \T S}}\\
& = \sup\limits_{\norm{x} \leq 1} \abs{\ip{Tx, S}}\\
& \geq \sup\limits_{\norm{Tx} \leq \Gamma} \abs{\ip{Tx, S}}\\
& = \sup\limits_{\norm{y} \leq \Gamma} \abs{\ip{y, S}}\\
& = \Gamma\\
& \inf\limits_{\norm{x} = 1} \norm{Tx}
\end{align*}
Since $S \in \partial B_{BL(Y,Z)}(0;1)$ was arbitrary, we conclude
$\inf\limits_{\norm{S} = 1} \norm{\T S} \geq \inf\limits_{\norm{x} = 1} \norm{Tx}$
\end{proof}
\begin{proof}[Proof of \ref{prop:Adjoint:NotDenseRange}]
Suppose $Range(T)$ is not dense in $Y$. 
Then there exists $y_0 \in Y \setminus \overline{Range(T)}$. 
By 
\ref{thm:HahnBanach:NullspaceOperator}, 
there exists $S_0 \in BL(Y,Z)$ with $S_0\pa{\overline{Range(T)}} = 0$ and 
$\norm{S_0(y_0)} = 1$. 
Set $S= \frac{S_0}{\norm{S_0}}$. 
Then $\norm{S} = 1$. 
Furthermore, 
\begin{align*}
\norm{\T S} & = \sup\limits_{\norm{x} \leq 1} \abs{\ip{x, \T S}}\\
& \leq \sup\limits_{y \in Range(T)} \abs{\ip{y, S}}\\
& = 0
\end{align*}
\end{proof}
\begin{proof}[Proof of \ref{prop:Adjoint:Surjective}]
Let $\T$ be \Surjective and let $x_0 \in X$ with $Tx_0 = 0$. 
Let $\tilde{S} \in BL(X,Z)$. 
Then there exists $S \in BL(Y,Z)$ with $\tilde{S} = \T S$. 
For this $S$, 
\begin{equation*}
\ip{x, \tilde{S}} = \ip{Tx, S} = 0
\end{equation*}
Hence $\tilde{S}x = 0$ for every \Linear \ContinuousFunction $\tilde{S}$. 
This implies $x=0$, so $T$ is \Injective. 
Now suppose 
\end{proof}
\end{prop}

\subsection{Higher Order Duals Of Seminormed Spaces}
\label{def:higherorderdualspaces}
\newcommand{\SecondTopDualSpace}[0]{ 
    \bf \hyperref[def:higherorderdualspaces]{$2^{nd}$ Topological Dual Space} \rm 
}
\newcommand{\ThirdTopDualSpace}[0]{ 
    \bf \hyperref[def:higherorderdualspaces]{$3^{rd}$ Topological Dual Space} \rm 
}
\newcommand{\NthTopDualSpace}[1]{
    \bf \hyperref[def:higherorderdualspaces]{$\pa{#1}^{th}$ Topological Dual Space} \rm
}
\begin{df}[Higher Order Dual Spaces]
    Let X be a 
   \SeminormedSpace. 
    From 
    \ref{def:dualspace}
    we know that the 
    \TopDualSpace
    of X, 
    $X^*$, 
    is also called the 
    \FirstTopDualSpace
    of X. 
    Building on this, 
    for $n \in \{2, 3, 4, ..., \}$
    we call the 
    \FirstTopDualSpace
    of $X^*$ the 
    \SecondTopDualSpace
    of X, 
    we call the 
    \FirstTopDualSpace
    of the 
    \SecondTopDualSpace
    of X the 
    \ThirdTopDualSpace
    of X, and 
    in general the 
    \FirstTopDualSpace
    of the 
    \NthTopDualSpace{n}
    of X
    the 
    \NthTopDualSpace{n+1}
    of X. 
\end{df}

\newcommand{\NthDualSPaceIso}[1]{\textbf{\hyperref[def:higherorderdualspaceisomorphism]{\ensuremath{\pa{#1}^{th}} Dual Space Isomorphism}}\xspace}
\begin{df}[Higher Order Dual Space Isomorphism]
\label{def:higherorderdualspaceisomorphism}
\rm
Let $X$ be a 
\SeminormedSpace
over a \Field
$\F$. 
Let $\Omega:X^* \to (X/\Ker_X)^*$ 
be the 
\Linear 
\Bijective 
\Isometry
defined in 
\ref{thm:dualspaceisomorphism}.
Define 
$\Omega_1=\Omega$.
Also define, for $2 \leq n \in \N$, 
$\Omega_n:X^{n*} \to \pa{X/\Ker_X}^{n*}$
by 
$\Omega_n=\pa{\Omega_{n-1}^{\times}}^{-1}$.
By     
\ref{prop:adjointoperator}
it is clear that the 
adjoint of a \Linear \Bijective \Isometry of \NormedSpaces
is also a \Linear \Bijective \Isometry of \NormedSpaces, and
so each $\Omega_n$ is as well. 

\end{df}

%Removed from here because it is defined for TVS
%\label{def:canonicalembedding}
\newcommand{\CanonicalEmbedding}[0]{
    \bf \hyperref[def:canonicalembedding]{Canonical Embedding} \rm
}
\newcommand{\Reflexive}[0]{
    \bf \hyperref[def:canonicalembedding]{Reflexive} \rm
}
\begin{df}[\CanonicalEmbedding of X into $X^{**}$]
    Let X be a \SeminormedSpace. 
    Define $c_X:X \to X^{**}$ by 
    setting, for each $x^* \in X^*$, 
    for each $x \in X$
    \begin{equation}
        \ip{x^*, c(x)} = \ip{x, x^*}
    \end{equation}
    We call $c_X$ the
    \CanonicalEmbedding
    of X into $X^*$. 
    As normal, if X is understood, 
    we may denote $c_X=c$.
    If c is Surjective, then we 
    say that X is \Reflexive. 

\end{df}


\begin{prop}[Canonical Embedding]
\label{prop:canonicalembedding}
    Let $X$ be a \SeminormedSpace 
    and let $c$ denote its 
    \CanonicalEmbedding. 
    The following are true. 
    \begin{enumerate}
        \item c is well defined
        \item c is Linear. 
        \item c is an isometry. 
        \item c is an injection if and only if X is a \NormedSpace. 
        \item Using the abuse of notation described in 
        \ref{rmk:doubledualnotation}, 
        if $q:X \to X/\Ker$ is the \QuotientMap, 
        then $c_X=c_{X/\Ker} \circ q$. 
        \item $c_X$ is surjective if and only if 
        $c_{X/\Ker}$ is surjective. 
        \item X is \Reflexive if and only if $X/\Ker$ is \Reflexive.
    \end{enumerate}
    \begin{proof}[Proof of 1]
        I just need to show that for any 
        $x \in X$, $c(x)$ is
        conintuous and
        linear. 
        For continuity,
    \end{proof}
    \begin{proof}[Proof of 2]
        Let $\alpha \in \F$ and $x,y \in X$.
        Let $x^* \in X$. 
    \end{proof}
    \begin{proof}[Proof of 3]
    \end{proof}
    \begin{proof}[Proof of 4]
    \end{proof}
    \begin{proof}[Proof of 5]
    \end{proof}

\end{prop}


\subsection{Seminormed Banach Steinhauss Theorem}
\begin{thm}[Banach Steinhauss]
\label{thm:BanachSteinhauss}
\rm
Let $X$ be a \Complete \SeminormedSpace. 
Let $Y$ be a \SeminormedSpace. 
Let $\{T_\alpha\}_{\alpha \in A} \subset BL(X,Y)$. 
Suppose that for each $x \in X$, 
\begin{equation*}
\sup\limits_{\alpha \in A} \norm{T_\alpha x} < \infty
\end{equation*}
Then 
\begin{equation*}
\sup\limits_{\alpha \in A} \norm{T_\alpha} < \infty
\end{equation*}
\begin{proof}
For each $n \in \mathbb{N}$, let 
\begin{equation*}
G_n = \braces{x \in X :\sup\limits_{\alpha \in A} \norm{T_\alpha x } \leq n}
\end{equation*}
By assumption, $\bigcup\limits_{n \in \mathbb{N}} G_n = X$. 
Furthermore, since each $T_\alpha$ is \ContinuousFunction, 
each $G_n$ is \SetClosed. 
Since $X$ is a 
\Complete \PseudometricSpace, 
we can apply
\ref{cor:BaireCategoryTheorem}
to conclude that for some $n_0 \in \mathbb{N}$, 
$\emptyset \neq \InteriorMark{G_{n_0}}$. 
That is, there exists $\delta > 0$ such that 
for every $\alpha \in A$ and for every $x \in B_X(0;\delta)$, 
\begin{equation*}
\norm{T_\alpha x} \leq n_0
\end{equation*}
Since each $T_\alpha$ is linear, for every $x \in \overline{B_X(0;1)}$, 
\begin{equation*}
\norm{T_\alpha x} \leq \frac{n_0}{\delta}
\end{equation*}
Hence $\norm{T_\alpha} \leq \frac{n_0}{\delta}$ for every $\alpha \in A$. 
\end{proof}
\end{thm}
\begin{prop}[\weakly \SetCompact \Norm bounded]
\label{prop:WeaklyCompactIsNormBounded}
\rm
Let $X$ be a \SeminormedSpace. 
Let $A \subset X$ be \weakly \SetCompact.
Then $A$ is \Norm bounded. 
\begin{proof}
Let $c:X \to X^{**}$ denote the \CanonicalEmbedding.
Let $x^* \in X^*$. 
Then since $A$ is \weakly \SetCompact, 
$x^*(A)$ is \SetCompact in $\F$. 
In particular, $x^*(A)$ is bounded. 
Thus 
\begin{equation*}
\sup\limits_{x \in A} \ip{x^*, c(x)} = \sup\limits_{x \in A} \ip{x, x^*} = \sup x^*(A) < \infty
\end{equation*}
By \ref{thm:BanachSteinhauss}, we then conclude
$\sup\limits_{x \in A} \norm{c(x)} < \infty$. 
Since $c$ is an \Isometry, $\sup\limits_{x \in A} \norm{x} < \infty$. 
\end{proof}
\end{prop}






 Similar to in the context of a normed space, if X is a seminormed space, we define the weak topology on X to be the topology on X generated by $X^*$, and the $weak^*$ topology on $X^*$ to be the topology generated by $c(X)$.
Before moving on to the classical theory revamped, I present on more useful result about weak topologies of seminormed spaces. 
\begin{prop}[Weak Quotients]
    \label{prop:weakquotients}
    Let X be a seminormed space and $\{Y_\alpha\}_{\alpha \in A}$ be a collection of topological spaces. For each $\alpha \in A$ let $\phi_\alpha:X \to Y_\alpha$ have the property that for every $x,y \in X$, for every $\alpha \in A$, $\norm{x-y}=0 \implies \phi_\alpha(x)=\phi_\alpha(y)$. 
    For each $\alpha \in A$, define $\tilde{\phi}_\alpha:X/\norm{\cdot}^{-1}\{0\} \to Y_\alpha$ by
    $\tilde{\phi}_{\alpha}[x] = \phi_\alpha x$. Let $\T_w$ denote the weak topology on X induced by $\{\phi_\alpha\}_{\alpha \in A}$, and $\T_{\tilde{w}}$ denote the weak topology on $X/\norm{\cdot}^{-1}\{0\}$ induced by $\{\tilde{\phi}_{\alpha}\}_{\alpha \in A}$. Then 
    \begin{equation}
        (X,\T_w)/\norm{\cdot}^{-1}\{0\} = (X/\norm{\cdot}^{-1}\{0\}, \T_{\tilde{w}})
    \end{equation}
    \begin{proof}
    \end{proof} 
\end{prop} 


Finally, before we move on, recall that if $X,Y$ are Topological vector spaces, we can topologize the set of continuous linear operators from X to Y, denoted $BL(X,Y)$ by saying that $\{T_\alpha\}_{\alpha \in A} \subset BL(X,Y)$ converges to $T \in BL(X,Y)$ if there is a neighborhood U of 0 in X such that $T_{\alpha}x \to Tx$ uniformly for $x \in U$. 
