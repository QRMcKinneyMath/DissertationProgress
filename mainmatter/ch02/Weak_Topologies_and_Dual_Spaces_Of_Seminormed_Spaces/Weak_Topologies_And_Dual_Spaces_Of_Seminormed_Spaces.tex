\section{Weak Topologies And Dual Spaces Of Seminormed Spaces}

\subsection{Basics}
\newcommand{\Bijective}[0]{
    \bf \hyperref[def:SetTheoryDefinitions]{Bijective} \rm
}

\newcommand{\Bijection}[0]{
    \bf \hyperref[def:SetTheoryDefinitions]{Bijection} \rm
}

\newcommand{\Surjection}[0]{
    \bf \hyperref[def:SetTheoryDefinitions]{Surjection} \rm
}
\newcommand{\Surjective}[0]{
    \bf \hyperref[def:SetTheoryDefinitions]{Surjective} \rm
}

\newcommand{\Injection}[0]{
    \bf \hyperref[def:SetTheoryDefinitions]{Injection} \rm
}
\newcommand{\Injective}[0]{
    \bf \hyperref[def:SetTheoryDefinitions]{Injective} \rm
}

\newcommand{\Partition}[0]{
	\bf \hyperref[def:SetTheoryDefinitions]{Partition} \rm
}

\newcommand{\Cover}[0]{
	\bf \hyperref[def:SetTheoryDefinitions]{Cover} \rm
}
\newcommand{\SubCover}[0]{
	\bf \hyperref[def:SetTheoryDefinitions]{SubCover} \rm
}

\newcommand{\Disjoint}[0]{
	\bf \hyperref[def:SetTheoryDefinitions]{Disjoint} \rm
}

\newcommand{\Finite}[0]{
	\bf \hyperref[def:SetTheoryDefinitions]{Finite} \rm
}

\begin{df}[\Bijection]
    \label{def:SetTheoryDefinitions}
\end{df} 
\label{def:TopologicalSpace}
\newcommand{\TopologicalSpaceRef}[0]{
    \bf \hyperref[def:TopologicalSpace]{Topological Space} \rm
}
\newcommand{\TopologyRef}[0]{
        \bf \hyperref[def:TopologicalSpace]{Topology} \rm
}
%A topology named #2 on the space #1 is given by \Topology{#1}{#2}
\newcommand{\Topology}[2]{
    #2_{#1}
}
%A Topological space (#1, #2_#1) is given by \TopologicalSpace{#1}{#2}
\newcommand{\TopologicalSpace}[2]{
    \pa{#1, \Topology{#1}{#2}}
}
\newcommand{\LetBeTopologicalSpace}[2]{
    Let $\TopologicalSpace{#1}{#2}$ be a \TopologicalSpaceRef.
}
\begin{df}[Topological Space]
$\TopologicalSpace{Z}{\T}$
\end{df}

\subsection{Relations}
\label{def:Relation}
\newcommand{\Relation}[0]{
    \bf \hyperref[def:Relation]{Relation} \rm
}
\begin{df}[\Relation]
    Let $X \neq \emptyset$ be a set. 
    We say that $R$ is a \Relation
    on X if $R \subset X \times X$. 
    If $(a,b) \in R$, then we may write
    $a R b$. 
\end{df}
\label{def:ReflexiveRelation}
\newcommand{\ReflexiveRelation}[0]{
    \bf \hyperref[def:ReflexiveRelation]{Reflexive} \rm
}

\newcommand{\RelationReflexivity}[0]{
    \bf \hyperref[def:ReflexiveRelation]{Reflexivity} \rm
}

\begin{df}[\ReflexiveRelation]
    Let $X \neq \emptyset$ be a set. 
    Let $R$ be a \Relation on X. 
    We say that $R$ is \ReflexiveRelation with respect to X if, 
    or equivalently we say that
    $R$ posseses 
    \RelationReflexivity with respect to X
    if 
    $\{(a,a) | a \in X \} \subset R$.
    
    When X is understood, we may simply say that 
    $R$ is \ReflexiveRelation or that $R$
    posesses \RelationReflexivity. 
\end{df}
\label{def:TransitiveRelation}
\newcommand{\TransitiveRelation}[0]{\textbf{\hyperref[def:TransitiveRelation]{Transitive}}\xspace}
\newcommand{\RelationTransitivity}[0]{\textbf{\hyperref[def:TransitiveRelation]{Transitivity}}\xspace}
\begin{df}[\TransitiveRelation]
\rm
\rm
    Let $X \neq \emptyset$ be a set. 
    Let $R$ be a \Relation on X. 
    We say that $R$ is \TransitiveRelation, 
    or equivalently we say that
    $R$ posseses 
    \RelationTransitivity
    if whenever $(a,b) \in R$ and $(b,c) \in R$, 
    we also have $(a,c) \in R$. 
\end{df}

\label{def:Preorder}
\newcommand{\Preordering}[0]{
    \bf \hyperref[def:Preorder]{Preordering} \rm
}

\newcommand{\Preorder}[0]{
    \bf \hyperref[def:Preorder]{Preorder} \rm
}

\newcommand{\PreorderedSet}[0]{
    \bf \hyperref[def:Preorder]{Preordered Set} \rm
}

\begin{df}[\Preorder]
    Let $X \neq \emptyset$ be a set. 
    Let $R$ be a \Relation on $X$. 
    If $R$ is
    \ReflexiveRelation
    and
    \TransitiveRelation
    then we call $R$
    a \Preorder on $X$, 
    or we equivalently call
    $R$ a \Preordering 
    of $X$ and we call 
    $(X,R)$ a \PreorderedSet.
    \end{df}
\label{def:SymmetricRelation}
\newcommand{\SymmetricRelation}[0]{
    \bf \hyperref[def:SymmetricRelation]{Symmetric} \rm
}

\newcommand{\RelationSymmetry}[0]{
    \bf \hyperref[def:SymmetricRelation]{Symmetry} \rm
}

\begin{df}[\SymmetricRelation]
    Let $X \neq \emptyset$ be a set. 
    Let $R$ be a \Relation on X. 
    We say that $R$ is \SymmetricRelation, 
    or equivalently we say that
    $R$ posseses 
    \RelationSymmetry
    if whenever $aRb$, we also have $bRa$. 
\end{df}
\newcommand{\AntiSymmetricRelation}[0]{\textbf{\hyperref[def:AntiSymmetricRelation]{Anti-Symmetric}}\xspace}
\newcommand{\RelationAntiSymmetry}[0]{\textbf{\hyperref[def:SymmetricRelation]{Anti-Symmetry}}\xspace}

\begin{df}[\AntiSymmetricRelation]
\label{def:AntiSymmetricRelation}
\rm
    Let $X$ be a nonempty set.
    Let $R$ be a \Relation on X. 
    We say that $R$ is \AntiSymmetricRelation
    and we say that
    $R$ posseses 
    \RelationAntiSymmetry
	if $R \cap R^{-1} \subset \scIdentity{X}$. 
\end{df}

\label{def:MaximalElement}
\newcommand{\MaximalElement}[0]{
    \bf \hyperref[def:MaximalElement]{Maximal Element} \rm
}

\newcommand{\Maximum}[0]{
    \bf \hyperref[def:MaximalElement]{Maximum} \rm
}

\begin{df}[Upper Bound]
    Let $X \neq \emptyset$ be a set. 
    Let $R$ be a \Relation on X. 
    Let $Y \subset X$.
    Let $a \in Y$. 
    We say that $a$ is a 
    \MaximalElement of $Y$, 
    or equivalently we say that 
    $a$ is a \Maximum of Y 
    if for every $b \in Y$, 
    we have $b \leq a$. 
    
    If we further assume that 
    Y has exactly 1 \MaximalElement 
    then we write 
    $a=max(Y)$. 
\end{df}
\newcommand{\MinimalElement}[0]{\textbf{\hyperref[def:MinimalElement]{Minimal Element}}\xspace}
\newcommand{\Minimum}[0]{\textbf{\hyperref[def:MinimalElement]{Minimum}}\xspace}
\newcommand{\Minima}[0]{\textbf{\hyperref[def:MinimalElement]{Minima}}\xspace}

\begin{df}[\MinimalElement]
\label{def:MinimalElement}
\rm
    Let $X \neq \emptyset$ be a set. 
    Let $R$ be an 
    \Relation on $X$. 
    Let $Y \subset X$.
    Let $a \in Y$. 
    We say that $a$ is a 
    \MinimalElement of $Y$, 
    or equivalently we say that 
    $a$ is a \Minimum of Y 
    if for every $b \in Y$,
	if $b R a$, then 
    we have $a = b$. 
	The Plural of \Minimum is \Minima, 
	and we represent the set of \Minima of Y with 
	respect to the relation $R$ with 
	$\Minima_R(Y)$, or if $R$ is understood, 
	we represent the set of $\Minima$ of Y with 
	$\Minima(Y)$. 
\end{df}

\begin{prop}[\MinimalElement unique if R is \AntiSymmetricRelation]
\label{prop:MinimalElementUnique}
\rm
	Let $X \neq \emptyset$ be a set. 
	Let $R$ be an
	\AntiSymmetricRelation
	\Relation
	on X. 
	Let $Y \subset X$.
	Let $a$ and $b$ be 
	each be a \MinimalElement
	of Y. 
	Then $a=b$. 
	\begin{proof}
		Since $a \in \Minima(Y)$, 
		$a \leq b$. 
		Since $b \in \Minima(Y)$, 
		$b \leq a$. 
		By \RelationAntiSymmetry, 
		$b = a$. 
	\end{proof}
\end{prop}
\label{def:UpperBound}
\newcommand{\UpperBound}[0]{
    \bf \hyperref[def:UpperBound]{Upper Bound} \rm
}

\newcommand{\UpperBounds}[0]{
    \bf \hyperref[def:UpperBound]{Upper Bounds} \rm
}

\newcommand{\BoundedFromAbove}[0]{
    \bf \hyperref[def:UpperBound]{Bounded From Above} \rm
}

\newcommand{\UB}[0]{
	\bf \hyperref[def:UpperBound]{UpperBound} \rm
}

\begin{df}[\UpperBound]
    Let $X \neq \emptyset$ be a set. 
    Let $R$ be a \Relation on X. 
    Let $Y \subset X$.
    Let $a \in X$. 
    We say that $a$ is an 
    \UpperBound for $Y$ if
    for every $x \in Y$, 
    we have $x R a$. 
    If $a$ is an \UpperBound
    then we also say that 
    the set Y is \BoundedFromAbove
    by a. 
	We denote the set of \UpperBounds of 
	$Y$ with respect to the relation $R$ with
	$\UB_R(Y)$. 
	When $R$ is understood, we denote this set with
	$\UB(Y)$. 
\end{df}
\label{def:LowerBound}
\newcommand{\LowerBound}[0]{
    \bf \hyperref[def:LowerBound]{Lower Bound} \rm
}

\newcommand{\BoundedFromBelow}[0]{
    \bf \hyperref[def:LowerBound]{Bounded From Below} \rm
}

\begin{df}[Lower Bound]
    Let $X \neq \emptyset$ be a set. 
    Let $R$ be a \Relation on X. 
    Let $Y \subset X$.
    Let $a \in X$. 
    We say that $a$ is an 
    \LowerBound for $Y$ if
    for every $x \in Y$, 
    we have $a R x$. 
    
    If $a$ is an \LowerBound
    then we also say that 
    the set Y is \BoundedFromBelow
    by a. 
\end{df}
\label{def:LeastUpperBound}

\newcommand{\LeastUpperBound}[0]{
    \bf \hyperref[def:LeastUpperBound]{Least Upper Bound} \rm
}

\newcommand{\LeastUpperBounds}[0]{
    \bf \hyperref[def:LeastUpperBound]{Least Upper Bounds} \rm
}

\newcommand{\Sup}[0]{
    \bf \hyperref[def:LeastUpperBound]{Sup} \rm
}

\newcommand{\Supremum}[0]{
    \bf \hyperref[def:LeastUpperBound]{Supremum} \rm
}

\newcommand{\Suprema}[0]{
    \bf \hyperref[def:LeastUpperBound]{Suprema} \rm
}

\newcommand{\LUB}[0]{
	\bf \hyperref[def:LeastUpperBound]{LUB} \rm
}

\begin{df}[\LeastUpperBound]
    Let $X \neq \emptyset$ be a set. 
    Let $R$ be a \Relation on X. 
    Let $Y \subset X$.
	Let $a \in X$. 
	We say that $a$ is a
	\LeastUpperBound of $Y$ if 
	$a \in \Minima(\UB(Y))$.
	We denote the set of \LeastUpperBounds
	for $Y$ with $\LUB(Y)$.
	If $b \in \LUB(Y)$, then we 
	also call $b$ a 
	\Supremum of $Y$. 
	The Plural of \Supremum is \Suprema.
	If $\LUB(Y)=\{c\}$, 
	then we write $c=\Sup(Y)$. 
\end{df}
\newcommand{\GreatestLowerBound}[0]{\textbf{\hyperref[def:GreatestLowerBound]{Greatest Lower Bound}}\xspace}
\newcommand{\GreatestLowerBounds}[0]{\textbf{\hyperref[def:GreatestLowerBound]{Greatest Lower Bounds}}\xspace}
\newcommand{\Inf}[0]{\textbf{\hyperref[def:GreatestLowerBound]{Inf}}\xspace}
\newcommand{\Infimum}[0]{\textbf{\hyperref[def:GreatestLowerBound]{Infimum}}\xspace}
\newcommand{\Infima}[0]{\textbf{\hyperref[def:GreatestLowerBound]{Infima}}\xspace}
\newcommand{\GLB}[0]{\textbf{\hyperref[def:GreatestLowerBound]{GLB}}\xspace}
\begin{df}[\GreatestLowerBound]
\label{def:GreatestLowerBound}
\rm
    Let $X$ be a nonempty set.
    Let $R$ be a \Relation on X. 
    Let $Y \subset X$.
	Let $a \in X$. 
	We say that $a$ is a
	\GreatestLowerBound of $Y$ if 
	$a \in \Maxima(\LB(Y))$.
	We denote the set of \GreatestLowerBounds
	for $Y$ with $\GLB(Y)$.
	If $b \in \GLB(Y)$, then we 
	also call $b$ an
	\Infimum of $Y$. 
	The Plural of \Infimum is \Infima. 
	If $\GLB(Y)=\{c\}$, then 
	we write $c=\Inf(Y)$.  
\end{df}

\label{def:EquivalenceRelation}
\newcommand{\EquivalenceRelation}[0]{\textbf{\hyperref[def:EquivalenceRelation]{Equivalence Relation}}\xspace}

\begin{df}[\EquivalenceRelation]
    Let $X \neq \emptyset$ be a set.
    Let $\cong$ be a \Preorder on X. 
    We say that $\cong$ is an \EquivalenceRelation on X
    if it is \SymmetricRelation.
\end{df}

\newcommand{\PartialOrder}[0]{\textbf{\hyperref[def:PartialOrder]{Partial Order}}\xspace}
\newcommand{\PartialOrdering}[0]{\textbf{\hyperref[def:PartialOrder]{Partial Ordering}}\xspace}
\newcommand{\Poset}[0]{\textbf{\hyperref[def:PartialOrder]{Partially Ordered Set}}\xspace}

\begin{df}[\PartialOrder]
\label{def:PartialOrder}
\rm
    Let $X$ be a nonempty set.
    Let $\leq$ be an \AntiSymmetricRelation \Preorder
    on $X$. 
    Then we say that $\leq$ is a \PartialOrder on $X$, 
    we say that $\leq$ is a \PartialOrdering of $X$, and
    we refer to the pair $(X,\leq)$ as a 
    \Poset.    
\end{df}

\label{def:Direction}
\newcommand{\Direction}[0]{
    \bf \hyperref[def:Direction]{Direction} \rm
}

\newcommand{\Directions}[0]{
    \bf \hyperref[def:Direction]{Directions} \rm
}

\newcommand{\Directing}[0]{
    \bf \hyperref[def:Direction]{Directing} \rm
}

\newcommand{\Directings}[0]{
    \bf \hyperref[def:Direction]{Directings} \rm
}

\newcommand{\DirectedSet}[0]{
    \bf \hyperref[def:Direction]{Directed Set} \rm
}
\newcommand{\DirectedSets}[0]{
    \bf \hyperref[def:Direction]{Directed Sets} \rm
}
\newcommand{\DirectedSection}[0]{
    \textbf{\hyperref[def:Direction]{Section}}
}
\newcommand{\DirectedSections}[0]{
    \textbf{\hyperref[def:Direction]{Sections}}
}

\begin{df}[\Direction]
    Let $X \neq \emptyset$ be a set.
    Let $\leq$ be a \Preorder on X. 
	If every pair of elements in $X$ has an 
	\UpperBound with respect to $\leq$, then
    we call $\leq$ is a \Direction on $X$, 
	, we call $\leq$ is a \Directing of $X$
	, and we call $(X,\leq)$ is a \DirectedSet.
    If $x_0 \in X$, then we call
    $\{x \in X : x_0 \leq x\}$ the 
    \DirectedSection
    of $x_0$ under $\leq$. 
\end{df}



\subsection{Neighborhood Filter Of A Point}
\label{def:NeighborhoodFilter}   
\newcommand{\NeighborhoodFilter}[0]{
    \bf \hyperref[def:NeighborhoodFilter]{Neighborhood Filter} \rm
}
\newcommand{\NeighborhoodFilters}[0]{
    \bf \hyperref[def:NeighborhoodFilter]{Neighborhood Filters} \rm
}
\newcommand{\NbhFilter}[2]{
    \scU_{#1}\pa{#2}
}
\begin{df}[Neighborhood Filter]
    \LetBeTopologicalSpace{Z}{\T}
    Let $x \in Z$. 
    We denote with $\NbhFilter{\Topology{Z}{\T}}{x}$ the collection of all elements of $\T_Z$ containing x. 
    We call $\NbhFilter{\Topology{Z}{\T}}{x}$ a \NeighborhoodFilter  of $\T_Z$ at x. 
\end{df}


\label{def:RelationOfEqualNeighborhoodFilters}
\newcommand{\RelationOfEqualNeighborhoodFilters}[1]{
    \bf \hyperref[def:RelationOfEqualNeighborhoodFilters]{Relation Of Equal Neighborhood Filters} \rm on #1
}
\begin{df}[relation of equal neighborhood filters]
    
    Let $(Z, \T_Z)$ be a topological space. Define the relation $\cong$ on Z by setting, for $x,y \in Z$, 
    \begin{equation}
        x \cong y \iff \scU_{\T_Z}(x)=\scU_{\T_Z}(y)
    \end{equation}
    We call $\cong$ the \RelationOfEqualNeighborhoodFilters{$(Z,\T_Z)$}
\end{df} 

\begin{prop}[\RelationOfEqualNeighborhoodFilters]
    \label{prop:EqualNeighborhoodFiltersEquivalenceRelation}
    
    The
	\RelationOfEqualNeighborhoodFilters
	$\cong$ on a \TopologicalSpaceRef $(Z,\T_Z)$ forms an 
	\EquivalenceRelation	
	on Z. 
    \begin{proof}
        
        Let $x \in (Z,\T_Z)$. 
        Then $\NbhFilter{\Topology{Z}{\T}}{x}$=$\NbhFilter{\Topology{Z}{\T}}{x}$, so $x \cong x$.
        Thus $\cong$ is 
		\ReflexiveRelation. 
        
        Let $x,y \in (Z,\T_Z)$. 
        Suppose $x \cong y$. 
        Then  $\NbhFilter{\Topology{Z}{\T}}{x} = \NbhFilter{\Topology{Z}{\T}}{y}$
        , so trivially  $\NbhFilter{\Topology{Z}{\T}}{y} =\NbhFilter{\Topology{Z}{\T}}{x}$
        , and thus $y \cong x$.
        Hence, $\cong$ is 
		\SymmetricRelation
        
        Let $x,y,z \in (Z,\T_Z)$.
        Let $x \cong y$ and $y \cong z$. 
        Then, 
         $\NbhFilter{\Topology{Z}{\T}}{x}= \NbhFilter{\Topology{Z}{\T}}{y} =  \NbhFilter{\Topology{Z}{\T}}{z}$
         so that $x \cong z$.
         Thus $\cong$ is \TransitiveRelation
         
         Since $\cong$ is 
		 \ReflexiveRelation
		, \SymmetricRelation
		, and \TransitiveRelation, it is an 
		\EquivalenceRelation. 
        
    \end{proof}
\end{prop}

\subsection{Equivalence Relations, Quotient Sets, and Quotient Maps}
\label{def:EquivalenceClass}
\newcommand{\EquivalenceClass}[0]{\textbf{\hyperref[def:EquivalenceClass]{Equivalence Class}}\xspace}
\newcommand{\EqClass}[2]{\bra{#1}_{\cong}\xspace}
\begin{df}[Equivalence Class]
    
    Let $X \neq \emptyset$.
    Let $\cong$ be an 
	\EquivalenceRelation
	defined on X.  
    Let $x \in X$. 
    We define the set $[x]_{\cong}$ by 
    \begin{equation}
        [x]_{\cong} = \{y \in X | y \cong x\}
    \end{equation} 
    We call $\EqClass{x}{\cong}$ the \EquivalenceClass of x in $(X, \cong)$. 
\end{df}

\begin{prop}[Equivalence Classes Partition]
    \label{prop:EquivalenceClassesPartition}
    
    Let $X \neq \emptyset$. 
    Let $\cong$ be an equivalence relation defined on X. 
    Let $x,y \in X$. 
    The following are true.
    \begin{equation}
        [x]_{\cong}  \cap [y]_{\cong} \neq \emptyset \iff [x]_{\cong} = [y]_{\cong}  \iff x \cong y \iff [x]_{\cong} \subset [y]_{\cong} \iff [y]_{\cong} \subset [x]_{\cong} 
    \end{equation}
    \begin{equation}
        x \in [x]_{\cong} 
    \end{equation}
    \begin{proof} OBVIOUS \end{proof}
\end{prop}  
\label{def:QuotientSet}
\newcommand{\QuotientSet}[0]{
    \bf \hyperref[def:QuotientSet]{Quotient Set} \rm
}
\newcommand{\QuoSet}[2]{
    #1
    /
    #2
}
\newcommand{\LetBeQuotientSet}[2]{
    Let \QuoSet{#1}{#2} be the \QuotientSet of #1 with respect to the relation #2.
}
\begin{df}[Quotient Set]  
    Let $X \neq \emptyset$.
    Let $\cong$ be an 
	\EquivalenceRelation defined on X.
    We define the set $X/\cong$ by 
    \begin{equation}
        \QuoSet{X}{\cong} = \{ [x]_{\cong} : x \in X\}
    \end{equation}
    We call $\QuoSet{X}{\cong}$ the \QuotientSet of X under the relation $\cong$. 
\end{df} 
\begin{rmk}[Quotient Set Partition]
    \label{rmk:quotientsetpartition}
    \ref{prop:EquivalenceClassesPartition}
	, paired with the fact that 
	$x \in [x]_{\cong}$, 
	implies that 
	$X/\cong$ is a partition of X. 
\end{rmk} 
\label{df:quotient_map}
\newcommand{\QuotientMap}[0]{
        \bf \hyperref[df:quotient_map]{Quotient Map} \rm
    }
    
 \newcommand{\QuotientMapInstance}[3]{
     #1 : #2\to #2/#3
}
\begin{df}[Quotient Map]

    Let $X \neq \emptyset$.
    Let $\cong$ be an 
	\EquivalenceRelation 
	on X.
    \LetBeQuotientSet{X}{$\cong$}
    Define $T:X \to X/\cong$ by setting, for each $x \in X$, 
    \begin{equation}
        T(x)=[x]
    \end{equation}    
    We call T the \QuotientMap of X under $\cong$. 
\end{df} 
\begin{prop}[\QuotientMap is  \Surjective]
\label{prop:QuotientMapSurjective}
\rm
    Let $X$ be a nonempty set.
    Let $\cong$ be an 
	\EquivalenceRelation on 
	$X$.
    Let $\QuotientMapInstance{T}{X}{\cong}$  be the 
	\QuotientMap of $X$ under the 
	\Relation
	$\cong$. 
    Then T is a 
	\Surjection. 
    \begin{proof}
       Let $K \in X/\cong$. 
       Then for some $x \in X$, $K=[x]$. 
       Then $T(x) = K$. 
       Since K was arbitrary, we are done. 
    \end{proof}
\end{prop} 


\subsection{Quotient Space Topology}
\label{def:QuotientSpaceTopology}
\newcommand{\QuotientSpaceTopology}[0]{
    \bf \hyperref[def:QuotientSpaceTopology]{Quotient Topology} \rm
}
\newcommand{\QuotientTopologicalSpace}[0]{
    \bf \hyperref[def:QuotientSpaceTopology]{Quotient Topological Space} \rm 
}

\begin{df}[Quotient Space Topology]
    Let $(Z,\T_Z)$ be a topological space. 
    Let $\cong$ be the \RelationOfEqualNeighborhoodFilters{$(Z, \T_Z)$}. 
    Let T be the \QuotientMap of Z under the relation $\cong$. 
    Define $\T_{Z/\cong}$ by
    \begin{equation}
        \T_{Z/\cong} = \left\{ \bigcup_{x \in U}\{T(x)\} \in 2^{Z/\cong}| U \in \T_Z \right\}
    \end{equation}
    By \ref{prop:QuotientSpaceTopology}, $\T_{Z/\cong}$ is a topology on $Z/\cong$.
    We call $\T_{Z/\cong}$ the \QuotientSpaceTopology and we call $\pa{Z/\cong, \T_{Z/\cong}}$ the \QuotientTopologicalSpace of $(Z, \T_Z)$.
    
\end{df}

\begin{prop}[Quotient Space Topology]
    \label{prop:QuotientSpaceTopology}
    
    Let $(Z,\T_Z)$ be a 
	\TopologicalSpaceRef
    with \QuotientTopologicalSpace  $\pa{Z/\cong, \T_{Z/\cong}}$
    and \QuotientMap T.
    
    Then the following are true. 
    \begin{enumerate}
        \item $\T_{Z/\cong}$ is a \TopologyRef on $Z/\cong$. 
        \item $T:(Z, \T_Z) \to (Z/\cong, \T_{Z/\cong})$ is \Continuous. 
        \item If U is \SetOpen (\SetClosed) in $(Z,\T_Z)$ then $T(U)$ and $T(Z\setminus U)$ \Partition $Z/\cong$. 
        \item If U is \SetOpen in $(Z, \T_Z)$, then $T^{-1}(T(U))=U$. 
        \item If K is \SetClosed in $(Z,\T_Z)$, then $T^{-1}T(K)=K$. 
        \item $T:(Z, \T_Z) \to (Z/\cong, \T_{Z/\cong})$ is \MapOpen. 
        \item $T:(Z, \T_Z) \to (Z/\cong, \T_{Z/\cong})$ is \MapClosed.
        \item $(Z, \T_Z)$ is a \Compact space if and only if $(Z/\cong, \T_{Z/\cong})$ is a \Compact space.
        \item If $\scB$ is a \TopologyBasis for $\T_z$, then $\{T(U) | U \in \scB\}$ is a \TopologyBasis for $\T_{Z/\cong}$. 
        \item If T is \Injective, then it is a \Homeomorphism. 
    \end{enumerate} 
    \begin{proof}[Proof of 1]
        Since $\emptyset \in \T_Z$, we have 
        \begin{equation}
            \emptyset = \bigcup\limits_{x \in \emptyset} \{Tx\} \in \T_{Z/\cong}
        \end{equation}
        Since $Z \in \T_Z$, and by \ref{rmk:quotientsetpartition}, 
        \begin{equation} 
            Z/\cong = \bigcup_{x \in Z} \{[x]\}= \bigcup\limits_{x \in Z} \{T(x)\} \in \T_{Z/\cong}
        \end{equation} 
        
        Let $\{U_{\alpha} | \alpha \in A\} \subset \T_{Z/\cong}$. 
        For each $\alpha \in A$, there exists $B_{\alpha} \in \T_{Z}$ such that we have
        \begin{equation} 
            U_{\alpha } = \bigcup_{x \in B_{\alpha}} \{Tx\}
        \end{equation} 
        Since $\bigcup_{\alpha \in A} B_\alpha \in \T_{Z}$, we have 
        \begin{equation}
            \bigcup_{\alpha \in A} U_{\alpha}= \bigcup\limits_{\alpha \in A} \bigcup\limits_{x \in U_\alpha} \{T(x)\} = \bigcup\limits_{x \in \bigcup\limits_{\alpha \in A} B_{\alpha}} \{T(x)\} \in \T_{Z/\cong}
        \end{equation} 
        Let $\{U_i\}_{i=1}^n \subset \T_{Z/\cong}$. 
        For each $i \in \{1, ..., n\}$, there exists $B_i \in \T_{Z}$ such that
        \begin{equation}
            U_i = \bigcup_{x \in B_{i}} \{T(x)\}
        \end{equation}
        Suppose 
        \begin{equation}
            [x_0] \in \bigcap\limits_{i=1}^n \bigcup\limits_{x \in B_i} \{T(x)\}
        \end{equation}
        Then for each $i \in \{1,..., n\}$, there is a $y_i \in B_i$ such that $ y_i \cong x_0$. 
        Since each $B_i$ is \SetOpen, the definition of $\cong$ implies that $x_0 \in B_i$ for every i. Hence, 
        \begin{equation} 
            x_0 \in \bigcap_{i=1}^n B_i
        \end{equation} 
        Implying 
        \begin{equation}
            [x_0] \in  \bigcup\limits_{x \in \bigcap\limits_{i=1}^n B_i} \{[x]\}
        \end{equation} 
        Hence, 
        \begin{equation} 
            \bigcap\limits_{i=1}^n \bigcup\limits_{x \in B_i} \{T(x)\}
            \subset
            \bigcup\limits_{x \in \bigcap\limits_{i=1}^n B_i} \{[x]\}
        \end{equation} 
        Furthermore, since the reverse inclusion is obvious, 
        and since $\bigcap_{i=1}^n B_i \in \T_{Z}$, we have 
        \begin{equation}
            \bigcap_{i=1}^n U_i = \bigcap_{i=1}^n \bigcup_{x \in B_i} \{T(x)\}= \bigcup\limits_{x \in \bigcap\limits_{i=1}^n B_i} \{T(x)\} \in \T_{Z/\cong}
        \end{equation}
    \end{proof}
    \begin{proof}[Proof of 2]
        Let $V \in \T_{Z/\cong}$. 
        Let $x_0 \in T^{-1}(V)$. 
        Then $[x_0] \in V$. 
        By definition, there is a $U \in \T_Z$ such that 
        \begin{equation}
            T(U) \subset \bigcup\limits_{x \in U} \{T(x)\}=V
        \end{equation}
        Hence there is a $y_0 \in U$  such that 
        \begin{equation}
            [x_0] \in T(y_0) = \{[y_0]\}
        \end{equation}
        Therefore, $x \cong y$. 
        Definition of the 
		\RelationOfEqualNeighborhoodFilters 
		implies $\scU(x_0)=\scU(y_0)$. 
        Hence, $x_0 \in U \subset T^{-1}(V)$.
    \end{proof}
    \begin{proof}[Proof of 3]
        Let $K$ be closed in $(Z,\T_Z)$. 
        Then each point $x_0$ in $Z\setminus K$ has some $U_{x_0} \in \scU_{\T_Z}(x_0)$ which is 
		\Disjoint 
		from K.
        Hence $y_0 \not \cong x_0$ for any $y_0 \in K$, $x_0 \in Z\setminus K$. 
        Hence $T(K)$ is 
		\Disjoint 
		from $T\pa{Z \setminus K}$. 
        This fact, paired with \ref{prop:QuotientMapSurjective}, implies $T(Z\setminus K)$ and T(K) 
		is a \Partition of $Z/\cong$.
    \end{proof}
    \begin{proof}[Proof of 4]
        Let $U \in \T_Z$. 
        The nontrivial direction to prove is $T^{-1}\pa{T(U)} \subset U$.
        Let $y \in T^{-1}\pa{T(U)}$. 
        Then $[y]=Ty \in T(U)$.
        Hence, $[y]=T(x)=[x]$ for some $x \in U$. 
        Since $y \cong x$ and $x \in U \in \scU_{\T_Z}(x)$, we have $U \in \scU_{\T_Z}(y)$. 
        Hence $y \in U$.
        Since y was arbitrary, $T^{-1}\pa{T(U)} \subset U$, and equality is obvious because the other direction of inclusion is trivial. 
    \end{proof}
    \begin{proof}[Proof of 5]
        Let K be 
		\SetClosed 
		in $(Z,\T_Z)$. Part 3 Of this result implies $Z/\cong$ is partitioned by $T(K)$ and $T(Z\setminus K)$. 
        
        By part 4 of this proposition, 
        \begin{align*}
            T^{-1}\pa{T(K)}&=T^{-1} \pa{T(Z) \setminus T(Z \setminus K)} \\
            &= T^{-1}\pa{Z/\cong \setminus T(Z \setminus K)}\\
            &=T^{-1}(Z/\cong) \setminus T^{-1}(T(Z\setminus K)) \\
            &= Z \setminus \pa{Z \setminus K} \\
            &= K
        \end{align*}      
    \end{proof}
    \begin{proof}[Proof of 6]
        Let $U \in \T_Z$.
        Then by definition of the \QuotientSpaceTopology
        \begin{equation}
            TU= \bigcup_{x \in U} \{T(x)\}  \in \T_{Z/\cong}
        \end{equation}
    \end{proof}  
    \begin{proof}[Proof of 7] 
        Let K be \SetClosed in $(Z,\T_Z)$. 
        Then $Z \setminus K \in \T_Z$. 
        By Parts 3 and five of this proposition, we know $T(K) = Z/\cong \setminus T(Z\setminus K)$ and also that $T(Z\setminus K) \in \T_{Z/\cong}$. Hence $T(K)$ is closed in $(Z/\cong, \T_{Z/\cong})$. 
    \end{proof} 
    \begin{proof}[Proof of 8]
        Let $(Z,\T_Z)$ be \SetCompact. 
        Let $\{U_{\alpha}\}_{\alpha \in A}$ be an open covering of $(Z/\cong, \T_{Z/\cong})$. 
        Then $\{T^{-1}\pa{U_{\alpha}} | \alpha \in A\}$ is an open covering of $(Z, \T_Z)$. 
        \SetCompactness of $(Z, \T_Z)$ guarantees the existence of a finite subcovering $\{T^{-1}\pa{U_{\alpha_i}} | i \in \{1, ..., n\}\}$. 
        Hence
        $\{U_{\alpha_i} | i \in \{1, ..., n\}\}=\{TT^{-1}(U_{\alpha_i}) | i \in \{1, ..., n\}\}$ is an 
		\OpenCover of $(Z/\cong, \T_{Z/\cong})$. 
         And the \SetCompactness of $(Z/\cong, \T_{Z/\cong})$ is verified. 
         
         
         Now, suppose $(Z/\cong, \T_{Z/\cong})$ is \SetCompact. 
         Let $\{V_{\beta} | \beta \in B\}$ be an \OpenCover of $(Z, \T_Z)$. 
         Since T is an \MapOpen mapping, $\{T(V_{\beta}) | \beta \in B\}$ is an 
		 \OpenCover of $(Z/\cong, \T_{Z/\cong})$ which by 
		 \SetCompactness has a \Finite \SubCover $\{T(V_{\beta_i}) | i \in \{1, ..., n\}\}$. 
         By part 4 of \ref{prop:QuotientSpaceTopology}, 
         $\{V_{\beta_i}| i \in \{1, ..., n\}\} = \{T^{-1}(T(V_{\beta_i})) |i \in \{1, ..., n\}\}$ is then an \OpenSubCover of $(Z, \T_Z)$. 
     %    
    \end{proof}
    \begin{proof}[Proof of 9]
        Let $\scB$ be a basis for $\T_z$ and let $V \in \T_{Z/\cong}$. 
        Then $T^{-1}(Z) \in \T_Z$, and so there is a subcollection $\{U_{\alpha}\}_{\alpha \in A} \subset \scB$ such that $T^{-1}(V) = \bigcup_{\alpha \in A} U_{\alpha}$. 
        Hence, 
        \begin{align*}
            V& =T(T^{-1}(V))\\
            & = T\pa{\bigcup_{\alpha \in A} U_{\alpha}}\\
            & = \bigcup_{\alpha \in A} T(U_{\alpha})
        \end{align*}
     \end{proof} 
     \begin{proof} [Proof of 10]
            If T is \Injective
			, then since it is \Continuous Part 2 of this result, open by part 6 of this result, and \Surjective by \ref{prop:QuotientMapSurjective}, it is a 
			\Bicontinuous 
			\Bijection, that is, a \Homeomorphism. 
         \end{proof}
\end{prop} 
\subsection{Weak Topologies}
\label{def:WeakTopology}
\newcommand{\WeakTopology}[0]{
	\bf \hyperref[def:WeakTopology]{Weak Topology} \rm
}

\begin{df}[\WeakTopology]
	Let X be a set. 
    For each $\alpha \in A$, let 
    $(Y_\alpha, T_\alpha)$ be a 
    \TopologicalSpaceRef, 
    and let $\phi_\alpha:X \to (Y_\alpha, T_\alpha)$. 
    
    Let $\T$ be the Coarsest possible 
    \TopologyRef on X such that 
    for each $\alpha \in A$, 
    $\phi_\alpha:(X, \T) \to (Y, \T_\alpha)$ 
    is continuous. 

    We call $\T$ the
    \WeakTopology on 
    X induced by $\{\phi_\alpha\}_{\alpha \in A}$
\end{df}


\subsection{Pseudometrics}
\label{def:Symmetricmap}
\newcommand{\SymmetricMap}[0]{
    \bf \hyperref[def:Symmetricmap]{Symmetric Map} \rm
}
\newcommand{\CommutativeFunction}[0]{
    \bf \hyperref[def:Symmetricmap]{Commutative} \rm
}
\newcommand{\FunctionCommutativity}[0]{
    \bf \hyperref[def:Symmetricmap]{Function Commutativity} \rm
}
\begin{df}[Triangle Inequality]
    Let X and Y be sets. 
    We say that a map 
    $f:X \times X \to Y$ is a \SymmetricMap 
    if for each 
    $x_0,x_1 \in X$, 
    $f(x_0,x_1)=f(x_1,x_0)$.
    In this situation, 
    we may also refer to $f$ as
    \CommutativeFunction, 
    or say that $f$ posesses 
    \FunctionCommutativity.
\end{df} 

\label{def:TriangleInequality}
\newcommand{\TriangleInequality}[0]{
    \bf \hyperref[def:TriangleInequality]{Triangle Inequality} \rm
}
\begin{df}[Symmetric Map]
    
    Let X be a set and $(Y,+, \leq)$ be a totally ordered magma.
    We say that a map $f:X \times X \to Y$ satisfies the \TriangleInequality if for each $x_0,x_1,x_3 \in X$, we have
    \begin{equation*}
        f(x_0,x_2) \leq  f(x_0,x_1)+f(x_1,x_2)
        \end{equation*}
\end{df} 
\newcommand{\Pseudometric}[0]{\textbf{\hyperref[def:pseudometric]{Pseudometric}}\xspace}
\newcommand{\Pseudometrics}[0]{\textbf{\hyperref[def:pseudometric]{Pseudometrics}}\xspace}
\newcommand{\PseudometricSpaces}[0]{\textbf{\hyperref[def:pseudometric]{Pseudometric Spaces}}\xspace}
\newcommand{\PseudometricSpace}[0]{\textbf{\hyperref[def:pseudometric]{Pseudometric Space}}\xspace}
\begin{df}[\Pseudometric]
\label{def:pseudometric}
\rm
    Let $X$ be a nonempty set.
    Let $d:X \times X \to [0,\infty)$ be \CommutativeFunction, 
    satisfy the \TriangleInequality, and for each $x \in X$, 
    \begin{equation*}
        d(x,x) = 0
    \end{equation*}
    Then we call d a \Pseudometric on X and we call $\pa{X,d}$ a \PseudometricSpace.
    \end{df} 
	
	
	
\newcommand{\Metric}[0]{\textbf{\hyperref[def:metric]{Metric}}\xspace}
\newcommand{\MetricSpace}[0]{\textbf{\hyperref[def:metric]{Metric Space}}\xspace}
\begin{df}[\Metric]
\label{def:metric}
\rm
	Let $(X,d)$ be a \PseudometricSpace. 
	If d has the property that for
	$x,y \in X$, if $x \neq y$, then
	\begin{equation*}
		d(x,y) \neq 0
	\end{equation*}
	Then we call d a \Metric on X 
	and we call $(X,d)$ a
	\MetricSpace
\end{df}

\newcommand{\PseudometricCauchySequence}[0]{\textbf{\hyperref[def:pseudometriccauchysequence]{Pseudometric Cauchy Sequence}}\xspace}
\begin{df}[Pseudometric Cauchy Sequence]
\label{def:pseudometriccauchysequence}
\rm
    Let $(X,d)$ be a \PseudometricSpace.
    We say that a sequence 
	$\{x_i\}_{i \in \N}$ is a 
	\PseudometricCauchySequence
    if, for each 
	$\epsilon > 0$, 
	there exists
	$N \in \N$
	such that for 
    each pair 
	$m,n \in \N$ 
	such that 
	$m>N$ 
	and 
	$n>N$, we have 
    \begin{equation*}
        d(x_m,x_n) < \epsilon
    \end{equation*}
\end{df}


\label{def:pseudometricsequenceconvergence}
\newcommand{\PseudometricConvergence}[0]{
    \bf \hyperref[def:pseudometricsequenceconvergence]{Pseudometric-Convergence} \rm
}
\newcommand{\PseudometricConvergent}[0]{
    \bf \hyperref[def:pseudometricsequenceconvergence]{Pseudometrically-Convergent} \rm
}
\newcommand{\PseudometricConverges}[0]{
    \bf \hyperref[def:pseudometricsequenceconvergence]{Pseudometric-Converges} \rm
}
\begin{df}[Pseudometric Convergence]
    Let $(X,d)$ be a \PseudometricSpace.
	Let $\{x_i\}_{i \in \N}$ be a sequence in $(X,d)$.
    Let $x_0 \in X$.  
    We say that 
	$\{x_i\}_{i \in \N}$ 
	exhibits 
	\PseudometricConvergence 
	to 
	$x_0$ 
	in d,
	or we say that 
	$\{x_i\}_{i \in \N}$  
	\PseudometricConverges 
	to 
	$x_0$ 
	in d, 
	or we say that 
	$\{x_i\}_{i \in \N}$ 
	is 
	\PseudometricConvergent 
	to 
	$x_0 \in d$ 
	if, 
    for every 
	$\epsilon > 0$, 
	there is an 
	$N \in \N$ 
	such that for every 
	$n>N$, 
	we have 
    \begin{equation}
        d(x_0, x_n) < \epsilon
    \end{equation}
\end{df}
\begin{prop}[Convergent Implies Cauchy]
\label{prop:pseudometricconvergenceimpliespseudometriccauchy}

    Let $(X,d)$ be a
    \PseudometricSpace.
    Let $\{x_i\}_{i \in \N}$ be a 
    \PseudometricConvergent sequence. 
    Then $\{x_i\}_{i \in \N}$
    is a \PseudometricCauchySequence.

    \begin{proof}
        Since $\{x_i\}$ converges, let 
        $x_i \to x$. 
        Let $\epsilon > 0$. 
        Then there exists $N \in \N$ 
        such that for $n>N$, we have
        $d(x_i, x) < \frac{\epsilon}{2}$. 
        For this N, if $m,n > N$, then we have 
        \begin{equation}
        d(x_m,x_n) \leq d(x_m,x) + d(x,x_n) < \epsilon
        \end{equation}
        and so the sequence is a
        \PseudometricCauchySequence, as advertised. 
    \end{proof}
\end{prop}

\label{def:uniformlycauchy}
\newcommand{\UniformlyCauchy}[0]{
    \bf \hyperref[def:uniformlycauchy]{Uniformly Cauchy} \rm
}
\begin{df}[Uniformly Cauchy]
	Let $(X_\alpha, d_\alpha)$ be a \PseudometricSpace
	for $\alpha \in A$ where A is some indexing set. 
	For each $\alpha \in A$
	, let $\phi_\alpha :=\{x_i^\alpha\}_{i \in \N} \subset X_{\alpha}$
	be a sequence. 
	We say that the collection $\{\phi_\alpha\}_{\alpha \in A}$ 
	is 
	\UniformlyCauchy if for each $\epsilon > 0$, there exists an 
	$N \in \N$ such that for each pair $m,n \in N$
	such that $m>N$ and $n>N$, and for each $\alpha \in A$, 
	we have 
	\begin{equation}
	d_{\alpha} \pa{x^{\alpha}_n, x^{\alpha}_m} < \epsilon
	\end{equation}
\end{df}

\newcommand{\UniformlyConvergent}[0]{\textbf{\hyperref[def:uniformlyconvergent]{Uniformly Convergent}}\xspace}
\newcommand{\ConvergesUniformly}[0]{\textbf{\hyperref[def:uniformlyconvergent]{Converges Uniformly}}\xspace}
\newcommand{\UniformConvergence}[0]{\textbf{\hyperref[def:uniformlyconvergent]{Uniform Convergence}}\xspace}
\begin{df}[\UniformConvergence]
\label{def:uniformlyconvergent}
\rm
    Let $A$ be a nonempty set.
    For each $\alpha \in A$, 
	let $(X_\alpha, d_\alpha)$ be a \PseudometricSpace
	and let $\phi_\alpha :=\{x_i^\alpha\}_{i \in \N} \subset X_{\alpha}$
	be a \Sequence in $X_\alpha$. 
	We say that the collection $\{\phi_\alpha\}_{\alpha \in A}$ 
    is \UniformlyConvergent to 
    $\{x_\alpha\}_{\alpha \in A} \in \prod\limits_{\alpha \in A} X_\alpha$
    if for each $\epsilon > 0$, 
    there is an $N \in \N$
    such that for each $n>N$, 
    and for every $\alpha \in A$, 
    we have 
    \begin{equation*}
        d_\alpha(x^{\alpha}_i,x_\alpha) < \epsilon
    \end{equation*}

    In this scenario, we may equivalently say that
    $\{\phi_\alpha\}$ demonstrates \UniformConvergence
    to $\{x_\alpha\}_{\alpha \in A}$ 
    or that it \ConvergesUniformly
    to $\{x_\alpha\}_{\alpha \in A}$.
    When we mention \UniformConvergence without
    specifying the limit, we are only claiming that one exists.
\end{df}

\begin{prop}[Uniform Cauchy and Pointwise Convergence implies Uniform Convergence]
\label{prop:uniformlycauchyplusconvergenceimpliesuniformconvergence}
\rm
    Let $A$ be a nonempty set.
    For each $\alpha \in A$, 
	let $(X_\alpha, d_\alpha)$ be a \PseudometricSpace
	and let $\phi_\alpha :=\{x_i^\alpha\}_{i \in \N} \subset X_{\alpha}$
	be a \Sequence. 
    Suppose the collection $\{\phi_\alpha\}_{\alpha \in A}$ 
    is \UniformlyCauchy
    and that each $\phi_\alpha$ 
    is \PseudometricConvergent
    , say $x_i^{\alpha} \to x_\alpha$. 
    Then $\{\phi_\alpha\}_{\alpha \in A}$
    is \UniformlyConvergent
    to $\{x_\alpha\}_{\alpha \in A}$. 
    \begin{proof}
        Let $\epsilon > 0$. 
        Then, since $\{\phi_\alpha\}_{\alpha \in A}$ 
        is \UniformlyCauchy, 
        there is an 
        $N \in \N$
        such that 
        for $m, n>N$, we have
        $d_{\alpha}(x^{\alpha}_n,x^{\alpha}_m) < \frac{\epsilon}{2}$. 
        Since each $\phi_\alpha$ 
        \PseudometricConverges to
        $x_\alpha$, 
        there are $N_{\alpha} \in \N$. 
        such that for any $n_\alpha > N_\alpha$, 
        we have
        $d_\alpha(x^{\alpha}_{n_\alpha}, x_\alpha) < \frac{\epsilon }{2}$. 
        For each $\alpha \in A$, define 
        $M_{\alpha}=max(N+1, N_{\alpha}+1)$.
        Let $n>N$. 
        Then, for any $\alpha \in A$, we have. 
        \begin{align*}
            d_{\alpha}(x_n^{\alpha} , x_\alpha) & \leq d_{\alpha}(x_n^{\alpha} , x^{\alpha}_{M_{\alpha}}) + d_{\alpha}(x_{M_{\alpha}}, x_\alpha)\\
            & < \frac{\epsilon}{2}+\frac{\epsilon}{2} \\
            & = \epsilon
        \end{align*}
        completing the proof. 



    \end{proof} 
\end{prop}


\label{def:pseudometriccomplete}
\newcommand{\PseudometricComplete}[0]{
    \bf \hyperref[def:pseudometriccomplete]{Pseudometric-Complete} \rm
}
\begin{df}[Pseudometric Complete]
    We say that a \PseudometricSpace $(X,d)$ is 
    \PseudometricComplete if each \PseudometricCauchySequence sequence in $(X,d)$ \PseudometricConverges to a limit in $X$. 
    \end{df}
\label{def:pseudometricball}
\newcommand{\OpenBall}[0]{
    \bf \hyperref[def:pseudometricball]{Open Ball} \rm
}
\newcommand{\ClosedBall}[0]{
    \bf \hyperref[def:pseudometricball]{Closed Ball} \rm
}
\begin{df}[Pseudometric Ball]
    Let $(X,d)$ be a \PseudometricSpace. 
    For each $x_0  \in X$ and each $\epsilon > 0$, we define the following.
    \begin{enumerate}
        \item  $B_d(x_0, \epsilon) := \{y \in X | d(x_0,y) < \epsilon\}$ denotes the \OpenBall about $x_0$ with radius $\epsilon$. 
    \item $\overline{B_d}(x_0,\epsilon) := \{y \in X | d(x_0,y) \leq \epsilon \}$ denotes the \ClosedBall about $x_0$ with radius $\epsilon$. 
    \end{enumerate} 
    
     
    \end{df} 
\newcommand{\PseudometricTopology}[0]{\textbf{\hyperref[def:pseudometrictopology]{Pseudometric Topology}}\xspace}
\newcommand{\PseudometricInducedTopology}[0]{\textbf{\hyperref[def:pseudometrictopology]{Pseudometric Topology}}\xspace}

\begin{df}[\PseudometricTopology]
\label{def:pseudometrictopology}
\rm
    Let $(X,d)$ be a \PseudometricSpace, and let $\scB$ be the set of \OpenBall's in $(X,d)$. 
    By \ref{prop:pseudometrictopology}, $\scB$, with the addition
	of $\emptyset$, is the \TopologyBasis for a unique \Topology $\T_d$ on $X$. 
    We call $\T_d$ the \PseudometricInducedTopology induced by $d$ on $X$. 
\end{df}

\label{prop:pseudometrictopology}
\begin{prop}[Pseudometric Topology]
    Let $(X,d)$ by  \PseudometricSpace and let $\scB$ be the set of \OpenBall's in $(X,d)$. 
    The following are true. 
    \begin{enumerate}
        \item There exists a unique topology $\T_d$ on X which $\scB$ is a basis of. That is, the \PseudometricTopology $\T_d$ is well defined. 
        \item The \PseudometricInducedTopology is first countable. That is, each of its points permits a countable neighborhood basis. 
    \end{enumerate}
    \begin{proof}[Proof of 1]
        Uniqueness is guaranteed by closure under arbitrary unions of a topology. 
        For existense, it is sufficient to show that the collection of arbitrary unions
        of elements of $\scB$ is closed under finite intersections. 
        Suppose that for $1\leq i \leq n$, we have $\{U_{\alpha_i} | \alpha_i \in A_i\} \subset \scB$
        and consider the set
        \begin{equation}
            U=\bigcap_{i=1}^n \bigcup_{\alpha_i \in A_i} U_{\alpha_i}
        \end{equation}
        Let $x_0 \in U$. 
        For each $i \in \{1, ..., n\}$, there exists $\alpha_i \in A_i$ such that 
        \begin{equation}
            x_0 \in U_{\alpha_i} = B_d(x_i; \epsilon_i)
        \end{equation}
        For each $i \in \{1, ..., n \}$, define $\delta_i = d(x_0, x_i)$. Then $0 < \delta_i < \epsilon_i$. 
        Then, for each $i \in \{1, ..., n \}$, 
        \begin{equation}
            B_d(x_0; \epsilon_i-\delta_i) \subset U_{\alpha_i} \subset \bigcup_{\alpha_i \in A_i} U_{\alpha_i}
        \end{equation}
        Define 
        \begin{equation}
            \delta_{x_0} = \min\limits_{i=1}^n \pa{ \epsilon_i-\delta_i}
        \end{equation}
        Then $x_0 \in B(x_0; \delta_{x_0} ) \subset U$. 
        If $U=\{x_{\alpha} | \alpha \in A\}$, then the arbitrary nature of $x_0$ above means 
        we can repeat this construction, writing 
        \begin{equation}
            U \subset \bigcup_{\alpha \in A} B(x_{\alpha} ; \delta_{x_{\alpha}} )\subset \bigcup_{\alpha \in A} U = U
        \end{equation}
        Hence, $U \in B$ and the proof is complete. 
    \end{proof}
    \begin{proof}[Proof of 2]
        Let $x_0 \in X$. 
        I claim that 
        \begin{equation}
            \scB_{x_0}:= \left\{ B_d\pa{x_0; \frac{1}{n}} | n \in \N\right\}
        \end{equation}
        is a neighborhood basis for $(X,\T_d)$ at $x_0$. 
        Let $U \in \scU_{\T_d}(x)$ be open in $\T_d$. 
        Since $\scB$ is a basis for $\T_d$, for some $y0 \in X$ and $\epsilon > 0$, 
        $x_0 \in B_d(y_0; \epsilon) \subset U$. 
        Let $\delta = d(x_0, y_0)$. Then $\epsilon - \delta > 0$. 
        Define
        \begin{equation}
            n = \ceil{ \frac{1}{\epsilon - \delta}}
        \end{equation}
        Then we have 
        \begin{equation}
            B_d\pa{x_0 ; \frac{1}{n}} \subset B_d(x_0 : \epsilon - \delta) \subset B(y_0 ; \epsilon) \subset U
        \end{equation}
    \end{proof}
\end{prop}

\label{def:relationofzerodistance}
\newcommand{\RelationOfZeroDistance}[0]{
    \bf \hyperref[def:relationofzerodistance]{Relation Of Zero Distance} \rm
}
\begin{df}[Relation Of Zero Distance]
    Let $(X,d)$ be a \PseudometricSpace. 
    Define the relation  $\cong_d$ on $X \times X$ by setting, for $x,y \in X$, 
    \begin{equation}
        x \cong_d y \iff d(x,y) = 0
    \end{equation}
    We call $\cong_d$ the \RelationOfZeroDistance on $(X,d)$. 
\end{df} 
\begin{prop}[Relation Of Zero Distance is the Relation Of Equal Neighborhood Filters]
    \label{prop:relationofzerodistance}
    \rm
    Let $(X,d)$ be a \PseudometricSpace.
    Let $\cong_{\T_d}$ be the \RelationOfEqualNeighborhoodFilters $(X,\T_d)$. 
    Let $\cong_d$ be the \RelationOfZeroDistance on $(X,d)$. 
    Then $\cong_{\T_d} = \cong_d$. 
    \begin{proof}
        Let $x,y \in X$ and suppose $x_0 \cong_d y_0$.
        Let $U \in \scU_{\T_d}(x_0)$. Then for some $\epsilon > 0$, 
        $x_0 \in B(x_0;\epsilon) \subset U$. 
        Since $x_0 \cong_d y_0$, $d(x_0,y_0) = 0$, so $y_0 \in B(x_0 ; \epsilon) \subset U$. 
        Hence $U \in \scU_{\T_d}(y_0)$. 
        The arbitrary nature of $U \in \scU_{\T_d}(x_0)$ implies 
        $\scU_{\T_d}(x_0) \subset \scU_{\T_d}(y_0)$.
        A reverse construction would just as easily show the reverse inclusion, so we conclude that $x_0 \cong_{\T_d} y_0$. 
        Now suppose $x_0 \cong_{\T_d} y $. Then for each $n \in \N$, 
        $y_0 \in B_{d} \pa{x_0 ; \frac{1}{n}}$.
        Hence $d(x_0, y_0) < \frac{1}{n}$ for each $n \in \bbZ^+$, 
        and so $d(x_0,y_0) = 0$ and $x_0 \cong_d y_0$. 
    \end{proof}
\end{prop}


\newcommand{\PseudometricInducedMetric}[0]{\textbf{\hyperref[def:pseudometricinducedmetric]{Pseudometric Induced Metric}}\xspace}
\newcommand{\MetricInducedByPseudometric}[0]{\textbf{\hyperref[def:pseudometricinducedmetric]{Metric Induced By The Pseudometric}}\xspace}
\begin{df}[\MetricInducedByPseudometric]
    \label{def:pseudometricinducedmetric}
    \rm
    Let $(X,d)$ be a \PseudometricSpace, and let $\cong$ be the \RelationOfZeroDistance, which by \ref{prop:relationofzerodistance} is also the \RelationOfEqualNeighborhoodFilters on $(X,\T_d)$. 
    Define $\tilde{d}: X/\cong \to [0,\infty)$ by 
    \begin{equation*}
        \tilde{d}\pa{\bra{x}, \bra{y}} = d(x,y)
    \end{equation*}
    By \ref{prop:pseudometricinducedmetric}, $\tilde{d}$ is well defined and is in fact a \Metric on $X/\cong$, so we call $\tilde{d}$ the \MetricInducedByPseudometric d on X, or we call it the \PseudometricInducedMetric of $(X,d)$. 
\end{df}

\begin{prop}[Metric Space Induced By Pseudometric Space]
    \label{prop:pseudometricinducedmetric}
    %Let $X$, d, $\cong$, and $\tilde{d}$ be defined as in \ref{def:pseudometricinducedmetric}
    Let $(X,d)$ be a \PseudometricSpace, $\cong$ the \RelationOfZeroDistance on $(X,d)$ and $\tilde{d}$ be defined as in \ref{def:pseudometricinducedmetric}.
    Let $(X/\cong, \T_{X/\cong})$ be the  \QuotientTopologicalSpace with \QuotientMap T, and let $(X/\cong, \T_{\tilde{d}})$ be the topological space induced by the metric space $(X/\cong, \tilde{d})$. 
    The following are true. 
    \begin{enumerate}
        \item $\tilde{d}$ is in fact well defined, and is a metric on $X/\cong$, justifying calling it the \MetricInducedByPseudometric d.
        \item $\T_{X/\cong} = \T_{\tilde{d}}$
        \item T is an isometry from $(X,d)$ to $(X/\cong, \tilde{d})$
        \item $(X/\cong, \tilde{d})$ is complete if and only if $(X, d)$ is \PseudometricComplete.
        \item If $T:$
    \end{enumerate}


\end{prop}
\newcommand{\Pseudometrizable}[0]{\textbf{\hyperref[def:Pseudometrizable]{Pseudometrizable}}\xspace}
\newcommand{\Metrizable}[0]{\textbf{\hyperref[def:Pseudometrizable]{Metrizable}}\xspace}
\begin{df}[(Pseudo)Metrizable]
    \label{def:Pseudometrizable}
    Let $(X,\T)$ be a topological space. 
    \begin{enumerate}
        \item We say that $(X,\T)$ (Or $\T$ or X which it wouldn't cause confusion) is \Pseudometrizable if there exists a pseudometric d on X such that $\T$ is the \PseudometricInducedTopology on $(X,d)$. 
        \item We say that $(X,\T)$ (Or $\T$ or X when it wouldn't cause confusion) is \Metrizable if there exists a metric d on X such that $\T$ is the metric topology on $(X,d)$. 
    \end{enumerate}
\end{df}

\begin{prop}[\Pseudometrizable Prequotient]
    \label{prop:pseudometrizableprequotient}
    \rm
    Let $(X,\T_X)$ be a \TopologicalSpace 
    with \RelationOfEqualNeighborhoodFilters $\cong$, and
    with \QuotientTopologicalSpace  $\pa{X/\cong, \T_{X/\cong}}$
    and \QuotientMap T. Let $\pa{X/\cong, \T_{X/\cong}}$ be \Pseudometrizable with \Pseudometric $\tilde{d}$. 
    The following hold. 
    \begin{enumerate}[label=(\roman*), ref={\ref{prop:pseudometrizableprequotient}~\roman*}]
        \item  
        \label{prop:PseudoPre:Pseudometrizable}
        Define $d:X^2 \to [0,\infty)$ by  $d(x,y) = \tilde{d}\pa{[x],[y]}$. 
        Then $\tilde{d}$ is a \Pseudometric on $X$ which is 
        \PseudometricCompatible with $\scT_X$. 
        \item 
        \label{prop:PseudoPre:Metrizable}
        $\tilde{d}$ is a \Metric $(X/\cong, \T_{X/\cong})$.
        \item 
        \label{prop:PseudoPre:Injective}
        If $T$ is \Injective, then $d$ as defined above is a \Metric on $X$.
    \end{enumerate}
    \begin{proof}[Proof Of \ref{prop:PseudoPre:Pseudometrizable}]
    We first prove $d$ to be a \Pseudometric on $X$.
        First, observe that if $x,y \in X$, then
        $d(x,y) =\tilde{d}([x],[y]) \in [0,\infty)$
        so that d is well defined. 
        Also, 
        $ d(x,y) = \tilde{d}([x],[y])=\tilde{d}([y],[x])=d(y,x)$,
        so d is \CommutativeFunction.
        Furthermore, 
        \begin{align*}
            d(x,z) & = \tilde{d}([x],[z])\\
            & \leq \tilde{d}([x],[y])+\tilde{d}([y], [z])\\
            & = d(x,y)+d(y,z)
        \end{align*}
        so d satisfies the \TriangleInequality. 
        Lastly, 
        $d(x,x)=\tilde{d}([x],[x])=0$, 
        and so $d$ is a \Pseudometric on $X$. 
        
        Let $\T_d$ denote the \PseudometricTopology on $(X,d)$. 
        What remains to show is that $\T_X=\T_d$. 
        Since $d(x,y)=\tilde{d}([x], [y])=\tilde{d}(Tx, Ty)$, T is an \Isometry. 
        Let $x \in U \in \T_X$. Then $[x] \in T(U) \in \T_{X/\cong}$. 
        Hence, there is an $\epsilon > 0$ such that $B_{\tilde{d}}([x], \epsilon) \subset T(U)$. 
        By \ref{prop:QST:OpenSetFiber}, $T^{-1}(B_{\tilde{d}}([x], \epsilon) \subset T^{-1}(T(U))=U$.
        Furthermore, by 
        \ref{prop:QST:QuotientMapContinuous}, 
        $T^{-1}(B_{\tilde{d}}([x], \epsilon) \in \T_X$. 
        Since T is an \Isometry $B_d(x, \epsilon) = T^{-1}(B_{\tilde{d}}([x],\epsilon) \subset U$. 
        Thus we have found an \OpenBall contained in $U$  containing an arbitrary point of U.
        Hence, $\T_X \subset \T_d$. As part of the preceeding arguement we also showed that an  arbitrary d-\OpenBall was in $\T_X$, so $\T_d \subset \T_X$, and so equality holds and we're done. 
    \end{proof} 
    \begin{proof}[Proof of \ref{prop:PseudoPre:Metrizable}]
        Let $x,y \in X$ wtih $[x] \neq [y]$. 
        Then $x \not \cong y$. By \ref{prop:relationofzerodistance}, $x \not \cong_d y$. 
        Hence $\tilde{d}([x],[y])=d(x,y) > 0$. 
    \end{proof}
    \begin{proof}[Proof of \ref{prop:PseudoPre:Injective}]
        Let T be \Injective, and suppose $x,y \in X$ with $x \neq y$. 
        Then $[x]=Tx\neq Ty=[y]$, 
        so by \ref{prop:PseudoPre:Metrizable}, $d(x,y) = \tilde{d}([x],[y]) > 0$. 
        
    \end{proof} 
\end{prop} 


\subsection{Algebraic Structures}
\label{def:AlgebraicDeclarations}


\newcommand{\Group}[0]{
    bf \hyperref[def:AlgebraicDeclarations]{Group} \rm
}
\newcommand{\Groups}[0]{
    bf \hyperref[def:AlgebraicDeclarations]{Groups} \rm
}

\begin{df}[Algebraic Declarations Placeholder]
\end{df}

\label{def:BinaryOperation}
\newcommand{\BinaryOperation}[0]{
    \bf \hyperref[def:BinaryOperation]{Binary Operation} \rm
}
\newcommand{\BinaryOperations}[0]{
    \bf \hyperref[def:BinaryOperation]{Binary Operations} \rm
}

\newcommand{\Operation}[0]{
    \bf \hyperref[def:BinaryOperation]{Operation} \rm
}
\newcommand{\Operations}[0]{
    \bf \hyperref[def:BinaryOperation]{Operations} \rm
}

\begin{df}[\BinaryOperation]
    Let $X \neq \emptyset$
    be a set. 
    We call 
    $T:X \times X \to X$ a 
    \BinaryOperation
    on $X$. 
    In this context, 
    for $x,y \in X$, 
    we sometimes use the notation
    \begin{equation*}
        xTy=T(x,y)
    \end{equation*}
\end{df}


\label{def:Magma}
\newcommand{\Magma}[0]{
    \bf \hyperref[def:Magma]{Magma} \rm
}
\newcommand{\Magmas}[0]{
    \bf \hyperref[dsef:Magma]{Magmas} \rm
}

\begin{df}[\Magma]
    Let $X$ be a set and
    $T:X \times X \to X$ be a 
    \BinaryOperation
    on $X$. 
    We call the pair $(X,T)$ a 
    \Magma.
    When it is clear what operation is being referred to, 
    we may simply refer to $X$
    as the 
    \Magma. 
\end{df}

\label{def:IdentityElement}

\newcommand{\IdentityElement}[0]{
    \bf \hyperref[def:IdentityElement]{Identity Element} \rm
}
\newcommand{\IdentityElements}[0]{
    \bf \hyperref[def:IdentityElement]{Identity Elements} \rm
}
\newcommand{\LeftIdentityElement}[0]{
    \bf \hyperref[def:IdentityElement]{Left Identity Element} \rm
}
\newcommand{\LeftIdentityElements}[0]{
    \bf \hyperref[def:IdentityElement]{Left Identity Elements} \rm
}
\newcommand{\RightIdentityElement}[0]{
    \bf \hyperref[def:IdentityElement]{Right Identity Element} \rm
}
\newcommand{\RightIdentityElements}[0]{
    \bf \hyperref[def:IdentityElement]{Right Identity Elements} \rm
}

\begin{df}[\LeftIdentityElement, \RightIdentityElement]
    Let $(X,L)$ and
    $(X,R)$ be 
    \Magmas.
    Let $l , r \in X$ 
    such that
    for every $x \in X$ 
    we have 
   \begin{align*}
        lLx=x\\
        xRr=x
   \end{align*}
   In such a scenario, we say that
   $l$ is a \LeftIdentityElement 
   of $(X,L)$, and
   we say that 
   $r$ is a 
   \RightIdentityElement
   of $(X,R)$. 
\end{df}

\begin{df}[\IdentityElement]
    Let $(X,\oplus)$ be a 
    \Magma. 
    Let $e \in X$ be both a 
    \LeftIdentityElement 
    and a 
    \RightIdentityElement 
    of $\oplus$. 
    Then, we say that
    $e$ is an \IdentityElement of 
    $(X,\oplus)$. 
\end{df}



\newcommand{\UnitalMagma}[0]{\textbf{\hyperref[def:UnitalMagma]{Unital Magma}}\xspace}
\newcommand{\UnitalMagmas}[0]{\textbf{\hyperref[def:UnitalMagma]{Unital Magmas}}\xspace}
\newcommand{\CommutativeUnitalMagma}[0]{\textbf{\hyperref[def:UnitalMagma]{Commutative Unital Magma}}\xspace}
\newcommand{\CommutativeUnitalMagmas}[0]{\textbf{\hyperref[dsef:UnitalMagma]{Commutative Unital Magmas}}\xspace}
\begin{df}[\UnitalMagma]
\label{def:UnitalMagma}
\rm
    Let $(X,\oplus)$ be a
    \Magma with \IdentityElement $e$. 
    Then we call 
    $(X,\oplus,e)$ a
    \UnitalMagma.
\end{df}

\label{def:InverseElement}

\newcommand{\InverseElement}[0]{
    \bf \hyperref[def:InverseElement]{Inverse} \rm
}
\newcommand{\InvertibleElement}[0]{
    \bf \hyperref[def:InverseElement]{Invertible} \rm
}
\newcommand{\InverseElements}[0]{
    \bf \hyperref[def:InverseElement]{Inverses} \rm
}
\newcommand{\LeftInverseElement}[0]{
    \bf \hyperref[def:InverseElement]{Left Inverse} \rm
}
\newcommand{\LeftInvertibleElement}[0]{
    \bf \hyperref[def:InverseElement]{Left Invertible} \rm
}
\newcommand{\LeftInverseElements}[0]{
    \bf \hyperref[def:InverseElement]{Left Inverses} \rm
}
\newcommand{\RightInverseElement}[0]{
    \bf \hyperref[def:InverseElement]{Right Inverse} \rm
}
\newcommand{\RightInvertibleElement}[0]{
    \bf \hyperref[def:InverseElement]{Right Invertible} \rm
}
\newcommand{\RightInverseElements}[0]{
    \bf \hyperref[def:InverseElement]{Right Inverses} \rm
}

\begin{df}[\LeftInverseElement, \RightInverseElement]
    Let $(X,\oplus,e)$ be a 
    \UnitalMagma.
    Let $l,r \in X$ such that 
    \begin{equation}
        l \oplus r=e
    \end{equation}
    In this scenario, we say that 
    $l$ is a 
    \LeftInverseElement 
    of $r$  
    in
    $(X,\oplus,e)$
    and we say that 
    $r$
    is a 
    \RightInverseElement
    of $l$ 
    in
    $(x,\oplus,e)$.
    Furthermore, we say that 
    $r$ is 
    \LeftInvertibleElement
    in
    $(X,\oplus,e)$
    and that 
    $l$ is 
    \RightInvertibleElement 
    in
    $(X,\oplus,e)$
\end{df}

\begin{df}[\InverseElement]
    Let $(X,\oplus,e)$ be a
    \UnitalMagma. 
    Let $x,y \in X$ such that
    $x$ is a \LeftInverseElement
    of $y$
    and $x$
    is a 
    \RightInverseElement
    of $y$. 
    Then, we say that 
    $x$ is an 
    \InverseElement
    of $y$
    in 
    $(X,\oplus, e)$
    and we say 
    $y$ an 
    \InvertibleElement
    element
    of
    $(X,\oplus,e)$. 
\end{df}



\label{def:AssociativeFunction}
\newcommand{\AssociativeFunction}[0]{
    \bf \hyperref[def:AssociativeFunction]{Associative} \rm
}
\newcommand{\AssociativeOperation}[0]{
    \bf \hyperref[def:AssociativeFunction]{Associative} \rm
}
\newcommand{\FunctionAssociativity}[0]{
    \bf \hyperref[def:AssociativeFunction]{Associativity} \rm
}
\newcommand{\OperationAssociativity}[0]{
    \bf \hyperref[def:AssociativeFunction]{Associativity} \rm
}
\begin{df}[\AssociativeOperation]
	Let $T$ be a 
	\BinaryOperation
	on a set $X$.
    We say that T is 
    \AssociativeFunction 
    and we say that T posses
    \FunctionAssociativity
    if for each $x,y,z \in X$, we have 
    \begin{equation*}
        T\pa{x,T\pa{y,z}}=T\pa{T\pa{x,y},z}
    \end{equation*}
\end{df}

%\label{def:MagmaHomomorphism}
\newcommand{\MagmaHomomorphism}[0]{
    \bf \hyperref[def:MagmaHomomorphism]{Magma Homomorphism} \rm
}
\newcommand{\MagmaHomomorphisms}[0]{
    \bf \hyperref[def:MagmaHomomorphism]{Magma Homomorphisms} \rm
}
\newcommand{\scMagma}[0]{
    \bf \hyperref[def:MagmaHomomorphism]{Magma} \rm
}
\newcommand{\Additive}[0]{
    \bf \hyperref[def:MagmaHomomorphism]{Additive} \rm
}
\newcommand{\Additivity}[0]{
    \bf \hyperref[def:MagmaHomomorphism]{Additivity} \rm
}
\begin{df}[\MagmaHomomorphism]
    Let $(X,\oplus_X)$
    and $(Y,\oplus_Y)$
    be \Magmas.
    Let $T:X \to Y$ satisfy, 
    for each $x_1, x_2 \in X$. 
    \begin{equation*}
        T\pa{x_1 \oplus_X x_2} = T\pa{x_1} \oplus_Y T\pa{x_2}
    \end{equation*}
    Then we call 
    T a \MagmaHomomorphism.
    We represent the collection of
    \MagmaHomomorphisms
    from $(X,\oplus_X)$ 
    to $(Y,\oplus_Y)$
    with 
    $H_{\scMagma}\pa{\pa{X, \oplus_X}, \pa{Y, \oplus_Y}}$, 
    or, when $\oplus_X$ and $\oplus_Y$ are clear, 
    $H_{\scMagma}\pa{X, Y}$. 
    A \MagmaHomomorphism
    is called \Additive
    and posseses the property 
    \Additivity. 
\end{df}

%\input{./Math/Definitions/ch02/PartiallyOrderedMagma}
%\newcommand{\UnitalMagma}[0]{\textbf{\hyperref[def:UnitalMagma]{Unital Magma}}\xspace}
\newcommand{\UnitalMagmas}[0]{\textbf{\hyperref[def:UnitalMagma]{Unital Magmas}}\xspace}
\newcommand{\CommutativeUnitalMagma}[0]{\textbf{\hyperref[def:UnitalMagma]{Commutative Unital Magma}}\xspace}
\newcommand{\CommutativeUnitalMagmas}[0]{\textbf{\hyperref[dsef:UnitalMagma]{Commutative Unital Magmas}}\xspace}
\begin{df}[\UnitalMagma]
\label{def:UnitalMagma}
\rm
    Let $(X,\oplus)$ be a
    \Magma with \IdentityElement $e$. 
    Then we call 
    $(X,\oplus,e)$ a
    \UnitalMagma.
\end{df}

%\label{def:Semigroup}
\newcommand{\Semigroup}[0]{
    \bf \hyperref[def:Semigroup]{Semigroup} \rm
}
\newcommand{\Semigroups}[0]{
    \bf \hyperref[def:Semigroup]{Semigroups} \rm
}

\begin{df}[\Semigroup]
    Let $(X,\oplus)$ be a 
    \Magma.
    Let $\oplus$ 
    be 
    \AssociativeFunction.
    Then we say that 
    $(X,\oplus)$ 
    is a 
    \Semigroup. 
\end{df}

%\label{def:Monoid}

\newcommand{\Monoid}[0]{
    \bf \hyperref[def:Monoid]{Monoid} \rm
}
\newcommand{\Monoids}[0]{
    \bf \hyperref[def:Monoid]{Monoids} \rm
}

\begin{df}[\Monoid]
    Let $(X,\oplus,e)$ be a 
    \UnitalMagma
    and let
    $(X,\oplus)$ 
    be a \Semigroup. 
    Then we call
    $(X,\oplus,e)$ 
    a 
    \Monoid.
\end{df}

%\label{def:Group}

\newcommand{\Group}[0]{
    \bf \hyperref[def:Group]{Group} \rm
}
\newcommand{\Groups}[0]{
    \bf \hyperref[def:Group]{Groups} \rm
}

\begin{df}[\Group]
    Let $(X,\oplus,e)$ 
    be a \Monoid
    such that
    each 
    $x \in X$
    is an
    \InvertibleElement.
    Then we call
    $(X,\oplus,e)$ a 
    \Group. 
\end{df}



\subsection{Vector Spaces}
\label{def:vectorspace}
\newcommand{\VectorSpace}[0]{
    \bf \hyperref[def:vectorspace]{Vector Space} \rm
}

\newcommand{\VectorSpaces}[0]{
    \bf \hyperref[def:vectorspace]{Vector Spaces} \rm
}
\newcommand{\Field}[0]{
    \bf \hyperref[def:vectorspace]{Field} \rm
}
\label{def:scalarhomogeneous}
\newcommand{\ScalarHomogeneous}[0]{
    \bf \hyperref[def:scalarhomogeneous]{Scalar Homogeneous} \rm
}

\newcommand{\ScalarHomogeneity}[0]{
    \bf \hyperref[def:scalarhomogeneous]{Scalar Homogeneity} \rm
}
\begin{df}[Scalar Homogeneous]
    Let V be a vector space over a field $\F \in \{\R, \C\}$. 
    
    We say that a map $p:V \to V$ is \ScalarHomogeneous, if
    , for each $\alpha \in \F$ and each $x \in V$, we have 
    \begin{equation}
        p(\alpha x) = \alpha p(x)
    \end{equation}
    Under these circumstances, we may instead say that the operator 
    p posesses \ScalarHomogeneity.
\end{df}

\label{def:absolutevaluescalarhomogeneous}
\newcommand{\AbsScalarHomogeneous}[0]{
    \bf \hyperref[def:absolutevaluescalarhomogeneous]{Absolutely Scalar Homogeneous} \rm
}

\newcommand{\AbsScalarHomogeneity}[0]{
    \bf \hyperref[def:absolutevaluescalarhomogeneous]{Absolute Scalar Homogeneity} \rm
}
\begin{df}[Scalar Homogeneous]
    Let V be a vector space over a field $\F \in \{\R, \C\}$. 
    
    We say that a map $p:V \to V$ is \AbsScalarHomogeneous, if
    , for each $\alpha \in \F$ and each $x \in V$, we have 
    \begin{equation}
        p(\alpha x) = \abs{\alpha} p(x)
    \end{equation}
    Under these circumstances, we may instead say that the operator 
    p posesses \AbsScalarHomogeneity.
\end{df}


\label{rmk:seminorm}
\begin{rmk}[\ScalarHomogeneous or \AbsScalarHomogeneous operator at 0 is 0]

If V is a vector space over $\mathbb{F} \in \{\R, \C\}$, then for each $x \in V$, $0x=0$.
Hence, if p is a \AbsScalarHomogeneous operator on v, then for any $x \in V$
\begin{equation}
p(0)=p(0x)=|0|p(x)=0p(x)=0
\end{equation}
If instead p is \ScalarHomogeneous operator on V, then we have
\begin{equation}
p(0)=p(0x)=0p(x)=0
\end{equation}
that is, in either case,  p(0)=0. 
\end{rmk}





\label{def:subadditive}
\newcommand{\Subadditive}[0]{
    \bf \hyperref[def:subadditive]{Subadditive} \rm
}

\newcommand{\Subadditivity}[0]{
    \bf \hyperref[def:subadditive]{Subadditivity} \rm
}
\begin{df}[\Subadditive]
Let $(G, \oplus_G)$ be a 
\Magma 
and 
$(H, \oplus_H, \leq)$ 
be a 
\PartiallyOrderedMagma. 
We call a mapping $p:G \to H$ \Subadditive if, for every $x,y \in G$, we have 
\begin{equation}
    p(x\oplus_G y) \leq p(x)\oplus_H p(y)
\end{equation}
Under these circumstances, 
we may also say that
$p$
 posesses $\Subadditivity$. 
\end{df}

\label{def:additive}
\newcommand{\Additive}[0]{
    \bf \hyperref[def:additive]{Additive} \rm
}

\newcommand{\Additivity}[0]{
    \bf \hyperref[def:additive]{Additivity} \rm
}

\newcommand{\MagmaHomomorphism}[0]{
    \bf \hyperref[def:additive]{Magma Homomorphism} \rm
}
\begin{df}[\Additive]
    Let $G$ and $H$ 
    be a Magmas.
    Represent the operation of $G$ and 
    $H$ both with $+$. 
    We call a mapping $p:G \to H$ 
    a \MagmaHomomorphism 
    if, for every 
    $x,y \in G$, we have 
    \begin{equation}
        p(x+y) = p(x)+p(y)
    \end{equation}
    We say that a 
    \MagmaHomomorphism
    is an 
    \Additive 
    operator
    and that 
    \Additive
    operators possess
    the property \Additivity
\end{df}

\label{def:linear}
\newcommand{\Linear}[0]{
    \bf \hyperref[def:linear]{Linear} \rm
}
\newcommand{\Linearity}[0]{
    \bf \hyperref[def:linearity]{Linearity} \rm
}

\begin{df}[\Linear]
    Let V, U be  
    \VectorSpaces
    over a 
    \Field
    $\F$. 
    We say that 
    $T:V \to U$ 
    is \Linear
    or that T possesses 
    \Linearity
    if T is both 
    \Additive
    and \ScalarHomogeneous
\end{df}
\label{def:VectorSpaceSpaceOfLinearOperators}
\newcommand{\SpaceOfLinearOperators}[0]{
    \bf \hyperref[def:VectorSpaceSpaceOfLinearOperators]{Space of Linear Operators} \rm
}

\begin{df}[\SpaceOfLinearOperators]
    Let $U,V$ be \VectorSpaces 
    over the same \Field
    $\F$. 
    We denote with 
    $L(U,V)$ 
    the set of \Linear
    operators $T:U \to V$. 
    We refer to $L(U,V)$ as the 
    \SpaceOfLinearOperators from 
    U to V.
    We endow $L(U,V)$ with 
    the operations of pointwise addition
    and pointwise scalar multiplication, 
    which the reader can verify makes 
    $L(U,V)$ into a \VectorSpace. 
\end{df}


\newcommand{\Balanced}[0]{
    \bf \hyperref[def:BalancedSet]{Balanced} \rm 
}
\newcommand{\BalancedSet}[0]{
    \bf \hyperref[def:BalancedSet]{Balanced Set} \rm 
}
\newcommand{\BalancedSets}[0]{
    \bf \hyperref[def:BalancedSet]{Balanced Sets} \rm 
}

\begin{df}[\Balanced]
\label{def:BalancedSet}
\rm
    Let $V$ be a \VectorSpace 
    over a \Field 
    $\F \in \{\R, \C\}$.
    Let $S \subset V$. 
    We call $S$ a \BalancedSet and
    we say that $S$ is 
    \Balanced if 
    for each
    $\alpha \in \F$ 
    with 
    $\abs{\alpha} \leq 1$
    we have 
    $\alpha S \subset S$. 
\end{df}

\newcommand{\Absorbing}[0]{\textbf{\hyperref[def:AbsorbingSet]{Absorbing}}\xspace}
\newcommand{\AbsorbingSet}[0]{\textbf{\hyperref[def:AbsorbingSet]{Absorbing Set}}\xspace}
\newcommand{\AbsorbingSets}[0]{\textbf{\hyperref[def:AbsorbingSet]{Absorbing Sets}}\xspace}
\newcommand{\Absorbed}[0]{\textbf{\hyperref[def:AbsorbingSet]{Absorbed}}\xspace}
\newcommand{\Absorb}[0]{\textbf{\hyperref[def:AbsorbingSet]{Absorb}}\xspace}
\newcommand{\Absorbs}[0]{\textbf{\hyperref[def:AbsorbingSet]{Absorbs}}\xspace}

\begin{df}[\Absorbing]
\label{def:AbsorbingSet}
\rm
    Let V be a \VectorSpace
    over a 
    \Field
    $\F \in \{\R, \C\}$. 
    Let $A, B \subset V$. 
    We say that $A$ 
    \Absorbs
    $B$ if 
    there exists a 
    $c > 0$ 
    such that
    $B \subset cA$. 
    In such a scenario, 
    $A$ is also said to 
    \Absorb $B$, 
    and we say that $B$ is 
    \Absorbed by $A$.
    If $A$ \Absorbs every singleton in $V$, 
    then we call $A$ an 
    \AbsorbingSet or we say that $A$
    is \Absorbing. 
\end{df}

\label{def:TranslationOperator}
\newcommand{\TranslationOperator}[0] {
    \bf \hyperref[def:TranslationOperator]{Translation Operator} \rm
}

\begin{df}[\TranslationOperator]
    Let $V$ be a 
    \VectorSpace 
    over a 
    \Field $\F$. 
    Let $\alpha \in \F$. 
    We define $T_\alpha:V \to V$ by 
    setting, for each 
    $x \in V$, 
    \begin{equation}
    T_\alpha(x)=\alpha+x
    \end{equation}
    We call $T_\alpha$ 
    the 
    \TranslationOperator
\end{df}

\begin{prop}[\TranslationOperator]
    \label{prop:TranslationOperatorAlgebraicProperties}
    Let $V$
    be a 
    \VectorSpace
    over a 
    \Field
    $\F$. 
    The following are true:
    \begin{enumerate}
        \item If $\alpha, \beta \in \F$, then $T_\alpha \circ T_\beta = T_{\alpha + \beta}$. 
    \end{enumerate}


    \begin{proof}[Proof of 01]
        Let $v \in V$. Then 
        \begin{align*}
            T_{\alpha} \circ T_{\beta} v & = T_\alpha \pa{\beta + v } \\
            & = \alpha + (\beta + v) \\
            & = (\alpha + \beta) + v \\
            & = T_{\alpha+\beta}+v
        \end{align*}
    \end{proof} 

\end{prop}

\label{def:ScalingOperator}
\newcommand{\ScalingOperator}[0] {
    \bf \hyperref[def:ScalingOperator]{Scaling Operator} \rm
}

\begin{df}[\ScalingOperator]
    Let $V$ be a 
    \VectorSpace 
    over a 
    \Field $\F$. 
    Let $\alpha \in \F$. 
    We define $M_\alpha:V \to V$ by 
    setting, for each 
    $x \in V$, 
    \begin{equation}
    M_\alpha(x)=\alpha x
    \end{equation}
    We call $M_\alpha$ the
    \ScalingOperator
\end{df}

\begin{prop}[\ScalingOperator]
    \label{prop:ScalingOperatorAlgebraicProperties}
    Let $V$
    be a 
    \VectorSpace
    over a 
    \Field
    $\F$. 
    The following are true:
    \begin{enumerate}
        \item If $\alpha, \beta \in \F$, then $M_\alpha \circ M_\beta = M_{\alpha * \beta}$. 
    \end{enumerate}


    \begin{proof}[Proof of 01]
        Let $v \in V$. Then 
        \begin{align*}
            M_{\alpha} \circ M_{\beta} v & = M_\alpha \pa{\beta * v } \\
            & = \alpha * (\beta * v) \\
            & = (\alpha * \beta) * v \\
            & = M_{\alpha*\beta}v
        \end{align*}
    \end{proof} 

\end{prop}

\newcommand{\Interval}[0]{\textbf{\hyperref[def:Interval]{Interval}}\xspace}
\newcommand{\Intervals}[0]{\textbf{\hyperref[def:Interval]{Intervals}}\xspace}
\newcommand{\ClosedInterval}[0]{\textbf{\hyperref[def:Interval]{Closed Interval}}\xspace}
\newcommand{\ClosedIntervals}[0]{\textbf{\hyperref[def:Interval]{Closed Intervals}}\xspace}
\newcommand{\OpenInterval}[0]{\textbf{\hyperref[def:Interval]{Open Interval}}\xspace}
\newcommand{\OpenIntervals}[0]{\textbf{\hyperref[def:Interval]{Open Intervals}}\xspace}
\newcommand{\HalfClosedInterval}[0]{\textbf{\hyperref[def:Interval]{Half-Closed Interval}}\xspace}
\newcommand{\HalfClosedIntervals}[0]{\textbf{\hyperref[def:Interval]{Half-Closed Intervals}}\xspace}
\newcommand{\HalfOpenInterval}[0]{\textbf{\hyperref[def:Interval]{Half-Open Interval}}\xspace}
\newcommand{\HalfOpenIntervals}[0]{\textbf{\hyperref[def:Interval]{Half-Open Intervals}}\xspace}
\begin{df}[\Interval]
\label{def:Interval}
\rm
    Let $V$ be a 
    \VectorSpace 
    over a 
    \Field
    $\F \in \{ \R, \C\}$. 
    Let $x,y \in V$. 
    We define the following sets:
    \begin{align*} 
        [x,y] = \{tx+(1-t)y : t \in [0,1] \} \\
        [x,y) = \{tx+(1-t)y : t \in [0,1) \} \\
        (x,y] = \{tx+(1-t)y : t \in (0,1] \} \\
        (x,y) = \{tx+(1-t)y : t \in (0,1) \}
    \end{align*}
    We refer to any of these sets as 
    \Intervals in $V$.
    Even in the absence of a topological structure, 
    we use the following language:
    \begin{enumerate}
        \item $[x,y]$ is called a \ClosedInterval.
        \item $(x,y)$ is called an \OpenInterval.
        \item $(x,y]$ and $[x,y)$ are called \HalfOpenIntervals or \HalfClosedIntervals.
    \end{enumerate}
\end{df}

\label{def:ConvexSet}
\newcommand{\ConvexSet}[0]{
    \bf \hyperref[def:ConvexSet]{Convex} \rm 
}

\begin{df}[\ConvexSet]
    Let $V$ 
    be a 
    \VectorSpace
    over a $\Field$
    $\F \in \{\R, \C\}$. 
    Let 
    $K \subset \F$. 
    We say that 
    $K$ is
    \ConvexSet
    if 
    for every pair $x,y \in K$, 
    we have $[x,y] \subset K$.
\end{df}


%\input{./Math/Definitions/ch02/`
\subsection{Topological Groups}
\label{def:TopologicalGroup}
\newcommand{\TopologicalGroup}[0]{
    \bf \hyperref[def:TopologicalGroup]{Topological Group}  \rm`
}
\begin{df}[\TopologicalGroup]
    Let $(G,+,e)$ be a \Group. 
    Let $g_{-1}:G \to G$ be defined by 
    $g(x)=-x$. 
    Let $\T$ be a 
    \TopologyRef on
    $G$ such that 
    $+:G \times G \to G$ is \Continuous
    and $g_{-1}$ is \Continuous.
    In this scenario, we call $(G, \T)$ a \TopologicalGroup.
\end{df}

\newcommand{\LocalBasis}[0]{\textbf{\hyperref[def:LocalBasis]{Local Basis}}\xspace}
\begin{df}[\LocalBasis]
\label{def:LocalBasis}
\rm
    Let $(G,\T)$ be a 
    \TopologicalGroup
    with \IdentityElement $e$.
    We call a 
    \NeighborhoodBasis of $\T$ about $e$
    a \LocalBasis for $(G,\T)$. 
\end{df}



\subsection{Topological Vector Spaces} 
\label{def:topologicalvectorspace}
\newcommand{\TVS}[0]{
    \bf \hyperref[def:topologicalvectorspace]{Topological Vector Space} \rm
}

\begin{df}[\TVS]
Let $(V,+,\cdot, 0)$ be a 
\VectorSpace over a \Field $\F \in \{\R, \C\}$. 
Let $\T$ be a 
\Topology on $V$
which is 
\VectorSpaceCompatible
with $(V, + , \cdot, 0)$. 
Then we call 
$(V,\T)$ a 
\TVS.
\end{df}

\newcommand{\LocallyConvex}[0]{\textbf{\hyperref[def:LocallyConvex]{Locally Convex}}\xspace}
\newcommand{\LocalConvexity}[0]{\textbf{\hyperref[def:LocallyConvex]{Local Convexity}}\xspace}

\begin{df}[\LocallyConvex]
\label{def:LocallyConvex}
\rm
    We say that a 
    \TVS
    $(X,\T)$ is 
    \LocallyConvex 
    if $(X,\T)$ has a 
    \LocalBasis consisting only of 
    \ConvexSet sets.
    A \LocallyConvex 
    space is said to posess
    \LocalConvexity.
\end{df}


\begin{prop}[Existence of \Balanced \NeighborhoodBasis of 0 in a \TVS]
    \label{prop:ExistenceOfBalancedNeighborhoods}
    \rm
    Let $(X,\T)$ be a 
    \TVS
    over a 
    \Field
    $\F$.
    The following are true. 
    \begin{enumerate}[label=(\roman*), ref={\ref{prop:ExistenceOfBalancedNeighborhoods}~\roman*}]
        \item 
        \label{prop:Bal1}
        If 
            $U \in \scU_{\T}(0)$,
            then there is a 
            \Balanced, \SetOpen
            $V \subset U$
            such that 
            $V \in \scU_{\T}(0)$.
        \item 
        \label{prop:Bal2}
        There exists a 
            \NeighborhoodBasis
            about $0 \in X$ 
            for $\T$ 
            consisting entirely 
            of \Balanced sets. 
        \item 
        \label{prop:Bal3}
        If 
            $U \in \scU_{\T}(0)$ is \ConvexSet,
            then there is a 
            \ConvexSet
            \Balanced, 
            \SetOpen
            $V \subset U$
            such that 
            $V \in \scU_{\T}(0)$.
        \item 
        \label{prop:Bal4}
        If $(X,\T)$ is 
            \LocallyConvex, 
            then there exists a 
            \NeighborhoodBasis
            about $0 \in X$ 
            for $\T$ 
            consisting entirely 
            of 
            \Balanced
            \ConvexSet 
            sets.
    \end{enumerate}

    \begin{proof}[Proof of \ref{prop:Bal1}] 
    Since scalar multiplication is \ContinuousAt $0$, 
    there is an \SetOpen disk $V \subset \mathbb{F}$
    and an \SetOpen $W \subset X$ with $0 \in W$ such that
    $VW \subset U$. 
    $VW$ is clearly balanced. 
    \end{proof}
    \begin{proof}[Proof of \ref{prop:Bal2}] 
    Let $\{U_{\alpha}\}_{\alpha \in A}$ be a 
    \LocalBasis for $\scT$. 
    Then, by \ref{prop:Bal1}, for each $\alpha \in A$, 
    there is a $W_\alpha \subset U_\alpha$ such that
    $W_\alpha$ is \BalancedSet and $0 \in W_\alpha$. 
    Clearly $\{W_\alpha\}_{\alpha \in A}$ forms a 
    \LocalBasis for $X$. 
    \end{proof}
    \begin{proof}[Proof of \ref{prop:Bal3}] 
    By \ref{prop:Bal1}, there is a 
    \BalancedSet \SetOpen $W \subset U$. 
    Let $\alpha \in \mathbb{C}$ with $\abs{\alpha} = 1$. 
    Then $\alpha^{-1}W = \subset W  \subset U$. 
    Hence $W \subset \alpha U$, so 
    if we define $A = \bigcap\limits_{\abs{\alpha} = 1 } \alpha U$, 
    then $0 \in W \subset A$.
    Thus $0 \InteriorMark{A}$. 
    For this reason, it suffices to show $\InteriorMark{A}$ is 
    \BalancedSet.
    It then suffices to show $A$ is \BalancedSet.
    Let $\alpha \in \mathbb{Z}$ with $\abs{\alpha} \leq 1$. 
    Then $\alpha = r \beta$ for some $\beta \in \mathbb{C}$ with $\abs{\beta} = 1$ 
    and $r \in [0,1]$. 
    Since $\alpha U$ is \ConvexSet and contains $0$, 
    $r\alpha U \subset \alpha U$. 
    Hence, 
    \begin{equation*}
    r \beta A = r \beta \bigcap\limits_{\abs{\alpha} = 1} \alpha U = \bigcap\limits_{\abs{\alpha} = 1 } r \alpha U \subset \bigcap\limits_{\abs{\alpha} = 1} \alpha U = A
    \end{equation*}
    Thus $A$ is \BalancedSet.
    \end{proof}
    \begin{proof}[Proof of \ref{prop:Bal4}] 
    This is a similar arguement to that of \ref{prop:Bal2}.
    \end{proof}
\end{prop}

\begin{prop}[Existence of \Balanced \NeighborhoodBasis of 0 in a \TVS]
    \label{prop:ExistenceOfBalancedNeighborhoods}
    \rm
    Let $(X,\T)$ be a 
    \TVS
    over a 
    \Field
    $\F$.
    The following are true. 
    \begin{enumerate}[label=(\roman*), ref={\ref{prop:ExistenceOfBalancedNeighborhoods}~\roman*}]
        \item 
        \label{prop:Bal1}
        If 
            $U \in \scU_{\T}(0)$,
            then there is a 
            \Balanced, \SetOpen
            $V \subset U$
            such that 
            $V \in \scU_{\T}(0)$.
        \item 
        \label{prop:Bal2}
        There exists a 
            \NeighborhoodBasis
            about $0 \in X$ 
            for $\T$ 
            consisting entirely 
            of \Balanced sets. 
        \item 
        \label{prop:Bal3}
        If 
            $U \in \scU_{\T}(0)$ is \ConvexSet,
            then there is a 
            \ConvexSet
            \Balanced, 
            \SetOpen
            $V \subset U$
            such that 
            $V \in \scU_{\T}(0)$.
        \item 
        \label{prop:Bal4}
        If $(X,\T)$ is 
            \LocallyConvex, 
            then there exists a 
            \NeighborhoodBasis
            about $0 \in X$ 
            for $\T$ 
            consisting entirely 
            of 
            \Balanced
            \ConvexSet 
            sets.
    \end{enumerate}

    \begin{proof}[Proof of \ref{prop:Bal1}] 
    Since scalar multiplication is \ContinuousAt $0$, 
    there is an \SetOpen disk $V \subset \mathbb{F}$
    and an \SetOpen $W \subset X$ with $0 \in W$ such that
    $VW \subset U$. 
    $VW$ is clearly balanced. 
    \end{proof}
    \begin{proof}[Proof of \ref{prop:Bal2}] 
    Let $\{U_{\alpha}\}_{\alpha \in A}$ be a 
    \LocalBasis for $\scT$. 
    Then, by \ref{prop:Bal1}, for each $\alpha \in A$, 
    there is a $W_\alpha \subset U_\alpha$ such that
    $W_\alpha$ is \BalancedSet and $0 \in W_\alpha$. 
    Clearly $\{W_\alpha\}_{\alpha \in A}$ forms a 
    \LocalBasis for $X$. 
    \end{proof}
    \begin{proof}[Proof of \ref{prop:Bal3}] 
    By \ref{prop:Bal1}, there is a 
    \BalancedSet \SetOpen $W \subset U$. 
    Let $\alpha \in \mathbb{C}$ with $\abs{\alpha} = 1$. 
    Then $\alpha^{-1}W = \subset W  \subset U$. 
    Hence $W \subset \alpha U$, so 
    if we define $A = \bigcap\limits_{\abs{\alpha} = 1 } \alpha U$, 
    then $0 \in W \subset A$.
    Thus $0 \InteriorMark{A}$. 
    For this reason, it suffices to show $\InteriorMark{A}$ is 
    \BalancedSet.
    It then suffices to show $A$ is \BalancedSet.
    Let $\alpha \in \mathbb{Z}$ with $\abs{\alpha} \leq 1$. 
    Then $\alpha = r \beta$ for some $\beta \in \mathbb{C}$ with $\abs{\beta} = 1$ 
    and $r \in [0,1]$. 
    Since $\alpha U$ is \ConvexSet and contains $0$, 
    $r\alpha U \subset \alpha U$. 
    Hence, 
    \begin{equation*}
    r \beta A = r \beta \bigcap\limits_{\abs{\alpha} = 1} \alpha U = \bigcap\limits_{\abs{\alpha} = 1 } r \alpha U \subset \bigcap\limits_{\abs{\alpha} = 1} \alpha U = A
    \end{equation*}
    Thus $A$ is \BalancedSet.
    \end{proof}
    \begin{proof}[Proof of \ref{prop:Bal4}] 
    This is a similar arguement to that of \ref{prop:Bal2}.
    \end{proof}
\end{prop}

\label{def:topologicalvectorspaceboundedset}
\newcommand{\TVSBounded}[0]{\textbf{\hyperref[def:topologicalvectorspaceboundedset]{TVS-Bounded}}\xspace}

\begin{df}[TVS Bounded Set]
Let $(V,\T)$ be a 
\TVS.
Let $A \subset V$. 
We say that A is \TVSBounded with respect to $\T$,
or when confusion is unlikely we simply say that A is \TVSBounded
if for every $U \in \scU_{\T}(0)$, there exists an $\alpha \in \F$
, $\alpha > 0$
, such that $A \subset \alpha U$. 
\end{df}

\newcommand{\BoundedLinearOperator}[0]{\textbf{\hyperref[def:boundedlinearoperatorinatvs]{Bounded Linear Operator}}\xspace}

\begin{df}[\BoundedLinearOperator]
\label{def:boundedlinearoperatorinatvs}
\rm
Let $\F \in \{\R, \C\}$.
For $i \in \{0,1\}$, let $(V_i,\T_i)$ be  a \TVSs over $\F$.
We say that a \Linear operator $T:(V_1, \T_1) \to (V_2, \T_2)$ is a 
\BoundedLinearOperator
if for each $U \in V_1$ with U \TVSBounded with respect to $\T_0$, 
$TU$ is \TVSBounded with respect to $\T_1$. 
\end{df}



TODO: Clean this Up. It isn't actually clear to me how I should topologize
CL(U,V) for arbitrary TVS's' U and V. 
For now its fine, because so far i've only used the convergence in the case of a 
Seminormed space but eventualy I want to define a topology that
in the case where U and V are seminormed spaces, $CL(U,V)=BL(U,V)=$ the seminormed topology generated by their seminorms.
\label{def:TVSSpaceOfContinuousLinearOperators}
\newcommand{\SpaceOfContinuousLinearOperators}[0]{
    \bf \hyperref[def:TVSSpaceOfContinuousLinearOperators]{Space Of Continuous Linear Operators} \rm
}
\begin{df}[\SpaceOfContinuousLinearOperators]
    Let 
    $(U, \T_U)$
    and $(V, \T_V)$
    each be a \TVS
    over the same $\Field$
    $\F \in \{\R, \C\}$. 
    Let $L(U,V)$ denote the \SpaceOfLinearOperators
    from $U$
    to $V$.
    We denote with 
    $CL((U, \T_U), (V, \T_V))$ 
    the subset of 
    $L(U, V)$ consisting only of the \Continuous operators. 
    When $\T_U$ and $\T_V$ are understood, 
    we may denote
    $CL((U, \T_U), (V, \T_V))=CL(U, V)$ 

    
\end{df}

\label{def:TopologyOfUniformConvergence}
\newcommand{\TopologyOfUniformConvergence}[0]{
    \bf \hyperref[def:TopologyOfUniformConvergence]{Topology of Uniform Convergence} \rm
}
\begin{df}[\TopologyOfUniformConvergence]
    Let X be a set and 
    $(Y, \T_Y)$ be a \TVS.
    Let $\NbhFilter{\T_Y}{0}$ denote the 
    \NeighborhoodFilter of $0 \in (Y, \T_Y)$.
    Suppose $\scF$ is a 
    \VectorSubspace of the set of 
    functions  $T:X \to Y$. 
    Suppose
    $\scG \subset 2^X$  such that 
	$(\scG, \subset)$ is a \DirectedSet.
    For each $x \subset X$ and $y \subset Y$, and  define 
    $M(x, y) = \{f \in \scF | f(x) \subset y\}$
    Now we define 
    $\T(\scF, \T_Y, \scG)= \{f+ M(x,y) | x \in \scG \wedge y \in \NbhFilter{\T_Y}{0} \wedge f \in \scF\}$.
	We call $\T(\scF, \T_Y, \scG)$ the \TopologyOfUniformConvergence of $\scF$ on $\scG$ with respect to $\T_Y$. 
	When $\scF$, $\T_Y$ or $\scG$ are understood they may be omitted from the reference. 
	By \ref{prop:TopologyOfUniformConvergence}, $\T$ is a 
	\TopologyRef on $\scF$. 
\end{df}
\begin{prop}\bf REMOVE\rm \end{prop}

%\begin{prop}[Bounded Linear Operator Continuous]
\label{prop:boundedlinearoperatorsarecontinuous}
Let $(V_i,\T_i)$ be a \TVS over a field $\F \in \{\R, \C\}$ for $i \in \{0,1\}$
Let $T:V_0 \to V_1$ be a 
\BoundedLinearOperator. 
Then T is continous. 
\begin{proof} 
    Let $0_{V_1} \in U \in \T_1$. 
\end{proof} 
\end{prop}
 Not actually True.. In general a Sequentiqally continuous Operator is  is bounded, but bounded only implies continuous if trhe domain is a "BOrnological" space (Defined by bourbaki gives notion of boundeds grounding)

\subsection{Seminormed Spaces}


\newcommand{\Seminorm}[0]{\textbf{\hyperref[def:seminorm]{Seminorm}}\xspace}
\newcommand{\Seminorms}[0]{\textbf{\hyperref[def:seminorm]{Seminorms}}\xspace}
\newcommand{\NonDegenerate}[0]{\textbf{\hyperref[def:seminorm]{Non-Degenerate}}\xspace}
\newcommand{\Degenerate}[0]{\textbf{\hyperref[def:seminorm]{Degenerate}}\xspace}
\newcommand{\SeminormedSpace}[0]{\textbf{\hyperref[def:seminorm]{Seminormed Space}}\xspace}
\newcommand{\SeminormedSpaces}[0]{\textbf{\hyperref[def:seminorm]{Seminormed Spaces}}\xspace}
\begin{df}[Seminorm]
\label{def:seminorm}
    Let 
	$V$ be a 
	\VectorSpace
	over a 
	\Field 
	$\F \in \{ \R, \C\}$.  
    We say that a map 
	$\norm{\cdot}:V \to [0,\infty)$ 
	is a 
	\Seminorm on 
	$V$ 
	if it is both \Subadditive and \AbsScalarHomogeneous. 
	In this case, we refer to $(V, \norm{\cdot})$ as a \SeminormedSpace. 
	We say that $\norm{\cdot}$ is \NonDegenerate if there is at least one $v \in V$ with $\norm{v}>0$. 
	We say that $\norm{\cdot}$ is \Degenerate if it is not \NonDegenerate.  
	We may also refer to the \SeminormedSpace $(V, \norm{\cdot})$ as being
	\Degenerate
	or
	\NonDegenerate. 
\end{df} 





\label{def:norm}
\newcommand{\Norm}[0]{
    \bf \hyperref[def:norm]{Norm} \rm
}
\label{def:normedspace}
\newcommand{\NormedSpace}[0]{
    \bf \hyperref[def:normedspace]{Normed Space} \rm
}\newcommand{\NormedSpaces}[0]{
    \bf \hyperref[def:normedspace]{Normed Spaces} \rm
}
\begin{df}[Norm]
    Let $(V,\norm{\cdot})$ be a \SeminormedSpace.
    If the following implication is true for $x \in V$, then we refer to $\norm{\cdot}$ as a \Norm on V, and we call $(V, \norm{\cdot})$ a \NormedSpace.
    \begin{equation}
    x \neq 0 \implies \norm{x} \neq 0
    \end{equation}
\end{df}

\begin{prop}[Subadditive Operator On a Group Induces a Metric]
    \label{prop:subadditiveinducestriangleinequality}
    Let $(G,+, e)$ be a group and let $(H,+,\leq)$ be a totally ordered magma. 
    Let $p:G \to H$ be \Subadditive. 
    define $d:G \times G \to H$ by setting, for each $x,y \in G$, 
    \begin{equation}
        d(x,y) =  p(x+(-y))
    \end{equation}

    Then d satisfies the triangle inequality. 

    \begin{proof}
    let $x,y, z \in G$. Then
    \begin{align*}
        d(x,z) &= p(x+(-z))\\
        & = p(x+e+(-z))\\
        & = p(x+(-y)+y+(-z))\\
        & \leq p(x+(-y))+p(y+(-z))\\
        & = d(x,y)+d(y,z)
    \end{align*}
    completing the proof. 
    \end{proof} 
\end{prop}
 
\label{def:seminormtopology}
\newcommand{\SeminormTopology}[0]{
    \bf \hyperref[def:seminormtopology]{Seminorm Topology} \rm
}

\newcommand{\SeminormInducedPseudometric}[0]{
    \bf \hyperref[def:seminormtopology]{Pseudometric induced by the Seminorm} \rm
}

\newcommand{\SeminormSpaceInducedPseudometricSpace}[0]{
    \bf \hyperref[def:seminormtopology]{Pseudometric Space induced by the Seminormed Space} \rm
}


\begin{df}[Seminorm Topology]
    Let $(X,\norm{\cdot})$ be a \SeminormedSpace.
    define $d_{\norm{\cdot}}:V \times V \to [0,\infty)$  by setting,
    for $x,y \in X$, 
    \begin{equation}
    d_{\norm{\cdot}}(x,y) = \norm{x-y}
    \end{equation}
    Observe the following: 
    \begin{enumerate}
        \item \ref{rmk:seminorm} guarantees that $d_{\norm{\cdot}}(x,x)=0$ for $x \in X$. 
        \item 
        \ref{prop:subadditiveinducestriangleinequality} guarantees that d satisfies the \TriangleInequality. 
        \item d is a \SymmetricMap, as we have 
    \begin{equation}
        d(x,y)_{\norm{\cdot}}=\norm{x-y}=|-1|\norm{x-y}=\norm{y-x}=d(y,x)
    \end{equation}
    \end{enumerate}

    Hence, $d_{\norm{\cdot}}$  is a \Pseudometric on X, which we call the \SeminormInducedPseudometric on X. 
    We refer to $(X, d_{\norm{\cdot}})$ as the \SeminormSpaceInducedPseudometricSpace $(X,\norm{\cdot}$. 
    We refer to the \PseudometricTopology induced by $d_{\norm{\cdot}}$ as the \SeminormTopology induced by $\norm{\cdot}$, and unless otherwise specified, when we reference $(X,\norm{\cdot})$, we consider it to be endowed with this topology. 

\end{df}

\label{def:seminormkernel}
\newcommand{\SeminormKernel}[0]{\textbf{\hyperref[def:seminormkernel]{Seminorm Kernel}}\xspace}
\newcommand{\SeminormKernels}[0]{\textbf{\hyperref[def:seminormkernel]{Seminorm Kernels}}\xspace}
\newcommand{\Ker}[0]{\textbf{\ensuremath{\mathcal{K}\rm^{ernel}}}\xspace}


\begin{df}[Seminorm Kernel]
Let $(V, \norm{\cdot})$ be a \SeminormedSpace. 
Define the set $\Ker_{(V,\norm{\cdot})}$ by 
\begin{equation}
\Ker_{(B,\norm{\cdot})}=\{x \in V | \norm{x}=0\}
\end{equation}
We call this set the \SeminormKernel of the space $\Ker_{(V,\norm{\cdot})}$. 
When confusion is unlikely, we may denote this set with
$\Ker$, $\Ker_V$, or even $\Ker_{\norm{\cdot}}$, or we may just refer to it
as the \SeminormKernel, the \SeminormKernel of $V$, or the \SeminormKernel of $\norm{\cdot}$. 
\end{df}

\begin{prop}[Seminorm Kernel is a vector Subspace]
\label{prop:seminormkernelisavectorsubspace}
    Let $(X,\norm{\cdot})$ be a \SeminormedSpace over a field $\F \in \{\R, \C\}$  
    with corresponding \SeminormKernel $\Ker$. 
    Then the following are true. 
    \begin{enumerate}
        \item $\Ker$ is a vector subspace of X. 
        \item $\Ker$ is closed in the \SeminormTopology on X.
        \item $\Ker=X$ if and only if X is \Degenerate. 
    \end{enumerate}


    \begin{proof}[Proof of One]
        \Subadditivity implies that, if $x,y \in \Ker$, then $\norm{x+y} \leq \norm{x}+\norm{y}=0$. 
        By \ScalarHomogeneity, if $x \in \Ker$  and $\alpha \in \F$, $\norm{\alpha x} =|\alpha| \norm{x}=0$
        so $\Ker$ is in fact a vector subspace of X. 
    \end{proof}
    \begin{proof}[Proof of Two]
        
        If $x \in X \setminus  \Ker$
        then $\norm{x} = \alpha > 0$ for some positive $\alpha$. 
        Hence $B(x;\alpha/2)$ is an open set containing x disjoint from $\Ker$. 
       We can then write $X \setminus \Ker$ as the union of all such open sets to see that $\Ker$ is closed. 
    \end{proof}
	
	\begin{proof}[Proof of Three]
		Direct application of the definitions of the \SeminormKernel
		and \Degenerate \Seminorm. 
	\end{proof} 
\end{prop}

\newcommand{\EquivelanceModKernel}[0]{\textbf{\hyperref[def:equivalencemodseminormkernel]{Equivalence MOD-$\Ker$}}\xspace}
\newcommand{\EquivalenceModKernel}[0]{\textbf{\hyperref[def:equivalencemodseminormkernel]{Equivalence MOD-$\Ker$}}\xspace}
\newcommand{\EquivalentModKernel}[0]{\textbf{\hyperref[def:equivalencemodseminormkernel]{Equivalent MOD-$\Ker$}}\xspace}
\newcommand{\SeminormKernelQuotientVectorSpace}[0]{\textbf{\hyperref[def:equivalencemodseminormkernel]{Seminorm Kernel Quotient Vector Space}}\xspace}

\begin{df}[Quotient Space Mod Kernel]
\label{def:equivalencemodseminormkernel}
\rm
Let $\F \in \{\R,\C\}$. 
Let $(X,\norm{\cdot})$ be a \SeminormedSpace over $\F$.
Denote the \SeminormKernel of $(X,\norm{\cdot})$ with  $\Ker$.
By \ref{prop:SeminormKernel:VectorSubspace} and \ref{prop:SeminormKernel:Closure0}, 
$\Ker$ is a \SetClosed \VectorSubspace of $X$. 
We endow $X/\Ker$ with the \QuotientVectorSpace structure, 
which we call the 
\SeminormKernelQuotientVectorSpace of the \SeminormedSpace $(X,\norm{\cdot})$.
\end{df}


\label{prop:equivalencemodkernelispseudometricequivalence}
\begin{prop}[Equivalence Mod Kernel is Pseudometric Equivalence]
    Let $(X,\norm{\cdot})$ be a seminromed space.
    with \SeminormKernel $\Ker$.
    Let $d$ denote the \SeminormInducedPseudometric.
	Let $\cong_{d}$ denote the 
	\RelationOfZeroDistance with respect to d. 
    
    Then $\cong_{\Ker}=\cong_{d}$. 
    \begin{proof}
        Let $x,y \in X$ and let $x \cong_{\Ker}y$.
        Then, since $x-y \in \Ker$, 
        Then $d(x,y) := \norm{x-y} =0$, so $x \cong_d y$. 
        Hence $\cong_{\Ker} \subset \cong_{d}$ 


        Now let $x,y \in X$ with $x \cong_d y$. 
        Then $\norm{x-y}=d(x,y) = 0$, so $x-y \in \Ker$
        , and therefore $x \cong_{\Ker} y$. 
        Hence, $\cong_{d} \subset \cong_{\Ker}$. 

        Since inclusion goes both directions, $\cong_{\Ker} = \cong_d$.

    \end{proof} 
\end{prop}

\label{def:quotientnormspace}
\newcommand{\QuotientNorm}[0]{
    \bf \hyperref[def:quotientnormspace]{Quotient Norm} \rm
}
\newcommand{\QuotientNormedSpace}[0]{
    \bf \hyperref[def:quotientnormspace]{Quotient Normed Space} \rm
}

\begin{df}[Quotient Norm Space]
Let $(X,\norm{\cdot})$ be a \SeminormedSpace
with \SeminormInducedPseudometric $d$, 
\SeminormKernel $\Ker$, and
\SeminormKernelQuotientVectorSpace $X/\Ker$.
Let $\tilde{d}:X/\Ker \times X/\Ker \to [0,\infty)$ be the \MetricInducedByPseudometric.

Define $\norm{\cdot}_{\Ker} : X/\Ker \to [0,\infty)$ by 
\begin{equation}
\norm{[x]}_{\Ker} = \tilde{d}([x], [0])
\end{equation}

By $\ref{prop:quotientnormspace}$, $(X/\Ker, \norm{\cdot}_{\Ker})$ is a normed space which we call the \QuotientNormedSpace of $(X,\norm{\cdot})$, and we call $\norm{\cdot}_{\Ker}$ the \QuotientNorm. 
Whenever we refer to $X/\Ker$, unless otherwise specified, we endow it with this norm and the topology generated by this norm.
Furthermore, whenever we consider $X/\Ker$, unless otherwise specified, we consider it as 
possesing the topology generated by the norm $\norm{\cdot}_{\Ker}$. 
\end{df}

\begin{prop}[Quotient Normed Space]
\label{prop:quotientnormspace}
\rm
Let $(X,\norm{\cdot})$ be a \SeminormedSpace
with \SeminormInducedPseudometric $d$, 
\SeminormKernel $\Ker$, and
\SeminormKernelQuotientVectorSpace $X/\Ker$.
Let $\tilde{d}:X/\Ker \times X/\Ker \to [0,\infty)$ be the \MetricInducedByPseudometric.
Let $T:X \to X/\Ker$ denote the \QuotientMap of X into $X/\Ker$ 
(Recalling that the 
\RelationOfEqualNeighborhoodFilters equals the 
\RelationOfZeroDistance equals the relation of 
\EquivelanceModKernel), so they would all produce the same quotient map)
Let $\norm{\cdot}_{\Ker}$ denote the \QuotientNorm.

The following are true. 
\begin{enumerate}[label=(\roman*), ref={\ref{prop:quotientnormspace}~\roman*}]
\item 
\label{prop:QNS:WellDefined}
$\norm{\cdot}_{\Ker}$ is a \Norm on $X/\Ker$. 
\item 
\label{prop:QNS:Compatible}
$\tilde{d}$  is the \SeminormInducedPseudometric $\norm{\cdot}_{\Ker}$, 
and thus they produce the same \Topology. 
\item 
\label{prop:QNS:Topology}
T has all of the properties described in $\ref{prop:QuotientSpaceTopology}$. 
\item 
\label{prop:QNS:Linear}
T is \Linear.
\item 
\label{prop:QNS:Surjective}
T is \Surjective. 
\item 
\label{prop:QNS:Isometry}
T is an \Isometry. 
\item 
\label{prop:QNS:Injective}
T is \Injective if and only if $\norm{\cdot}$ is a \Norm. 
\begin{proof}[Proof of \ref{prop:QNS:WellDefined}]
    First, note that 
    $Range(\norm{\cdot}_{\Ker}) \subset Range(\tilde{d}) \subset [0,\infty)$,\
    so that $\norm{\cdot}_{\Ker}$ has the correct \FunctionDomain and \FunctionCodomain. 
    For \Subadditivity, let $[x],[y] \in X/\Ker$. Then 
    \begin{align*}
        \norm{[x]+[y]}_{\Ker}& = \norm{[x+y]}_{\Ker}\\
        & = \tilde{d}\pa{[x+y], [0]}\\
        & = d(x+y, 0)\\
        & = \norm{x+y} \\
        & \leq \norm{x}+\norm{y}\\
        & = d(x,0)+d(y,0)\\
        & = \tilde{d}\pa{[x],[0]}+ \tilde{d}\pa{[y],[0]}\\
        & = \norm{[x]}_{\Ker}+\norm{[y]}_{\Ker}
    \end{align*}
    For \AbsScalarHomogeneity, let $\alpha \in \F$ and $[x] \in X/\Ker$. 
    Then, 
    \begin{align*}
        \norm{[\alpha x]}_{\Ker} & = \tilde{d}\pa{[\alpha x], [0]}\\
        & = d(\alpha x, 0) \\
        & = \norm{\alpha x}\\
        & = \abs{\alpha} \norm{x} \\
        & = \abs{\alpha} \norm{[x]}_{\Ker}
    \end{align*}
    Finally, suppose $[x] \neq 0$. 
    Then, since the additive identity of $X/\Ker$ is $\Ker$, $x \not \in \Ker$. 
    Hence $\norm{[x]}_{\Ker} = \tilde{d}([x], 0) = d(x,0) =\norm{x} > 0$. 

\end{proof}
\begin{proof}[Proof of \ref{prop:QNS:Compatible}] 
Let $D$ denote the \SeminormInducedPseudometric $\norm{\cdot}_{\Ker}$. 
Then, for $[x], [y] \in X/\Ker$, 
\begin{align*}
\tilde{d}([x], [y]) & = d(x,y)\\
& = \norm{x-y}\\
& = \norm{x-y-0}\\
& = d(x-y, 0)\\
& = \tilde{d}([x-y],0)\\
& = \norm{[x-y]}_{\Ker}\\
& = \norm{[x]-[y]}_{\Ker}\\
& = D\pa{[x], [y]}
\end{align*}
Since these two \Pseudometric's are equal, they produce the same \Topology. 
Furthermore, by applying \ref{prop:pseudometricinducedmetric}, we see that the 
\Topology generated by $\norm{\cdot}_{\Ker}$ is also the \QuotientSpaceTopology on $X/\Ker$. 
\end{proof}
\begin{proof}[Proof of \ref{prop:QNS:Topology}]
T is the topological \QuotientMap and the \Norm \Topology is the \QuotientSpaceTopology, so the assumptions of $\ref{prop:QuotientSpaceTopology}$ are satisfied. 
\end{proof} 
\begin{proof}[Proof of \ref{prop:QNS:Linear}] 
This is a direct consequence of \ref{prop:QuotientVectorSpace:QuotientMapLinear}.
\end{proof}
\begin{proof}[Proof of \ref{prop:QNS:Surjective}] 
This is a direct consequence of \ref{prop:QuotientMapSurjective}.
\end{proof}
\begin{proof}[Proof of \ref{prop:QNS:Isometry}] 
This is a direct consequence of \ref{def:MFPM:IsIsometricSurjection}.
\end{proof}
\begin{proof}[Proof of \ref{prop:QNS:Injective}] 
This is a direct consequence of \ref{def:MFPM:Injection}.
\end{proof}

\end{enumerate} 

\end{prop} 

\begin{rmk}[Quotient Normed Space]
\label{rmk:quotientnormedspace}
\rm
    If $(X,\norm{\cdot}_X)$
    is a \NormedSpace
    then by parts
    \ref{prop:QNS:Linear}, 
    \ref{prop:QNS:Surjective}, 
    \ref{prop:QNS:Isometry}, 
    and
    \ref{prop:QNS:Injective}, 
    $T:X \to \Ker_X$
    is an isomorphism of 
    \NormedSpaces satisfying
    $Tx=\{x\}$.
    For this reason, 
    as an abuse of notation, 
    later in this document,
    I may not distinguish between the quotient
    $X/\Ker_X$ and the space $X$ if
    X is a \NormedSpace, 
    and similarly, I may not distinguish between 
    $x \in X$ and $\{x\} \in X/\Ker_X$. 
\end{rmk}

\begin{prop}
\label{prop:quotientspreservecompleteness}
Let $(X,\norm{\cdot})$ be a \SeminormedSpace with \QuotientNormedSpace $(X/\Ker, \norm{\cdot}_{\Ker})$. 

Then X is \PseudometricComplete if and only if $X/\Ker$ is complete. 

\begin{proof}
Let X be \PseudometricComplete. 
Let $\{[x_i]\}_{i \in \N} \subset X/\Ker$ be a \PseudometricCauchySequence. 
Let $\epsilon > 0$. 
Then there is an $N \in \N$ such that for $m,n > N$ we have 
\begin{equation}
\norm{[x_m-x_n]}_{\Ker} < \epsilon
\end{equation}

For this N, we have 
\begin{equation}
\norm{x_m-x_n} = \norm{[x_m-x_n]}_{\Ker} < \epsilon
\end{equation}
so that $\{x_i\}_{i \in \N}$ is a \PseudometricCauchySequence. 
Since X is \PseudometricComplete, 
there is a 
$x \in X$ such that $\norm{x_i-x} \to 0$, 
but since T is an isometry, 
\begin{equation}
\norm{[x]-[x_i]}=\norm{[x_i-x]}_{\Ker} \to 0
\end{equation}
and so 
$[x_i] \to [x]$.
so that $X/\Ker$ is complete. 

Now suppose instead that $X/\Ker$ is complete 
and suppose $\{x_i\}_{i \in \N}$ is a \PseudometricCauchySequence in X. 
Since $\norm{[x_i-x_j]}_{\Ker} = \norm{x_i-x_j}$, 
$\{[x_i]\}_{i \in \N}$ is a \PseudometricCauchySequence in $X/\Ker$, which therefore has a 
limit $y \in X/\Ker$. Since T is surjective, $y=[x]$ for some $x \in X$, and it is easy to see that
$x_i \to x$ so that $X$ is \PseudometricComplete. 

\end{proof}
\end{prop}

\label{def:BLO} 
\newcommand{\SpaceOfBoundedLinearOperators}[0]{ 
    \bf \hyperref[def:BLO]{Space of Bounded Linear Operators} \rm
}
\newcommand{\OperatorSeminorm}[0]{
    \bf \hyperref[def:BLO]{Operator Seminorm} \rm
}
\newcommand{\OperatorNorm}[0]{
    \bf \hyperref[def:BLO]{Operator Norm} \rm
}
\begin{df}[Space of Continuous Linear Operators From a Seminormed Space into a Normed Space]
Let $(X,\norm{\cdot}_X)$ be a \NonDegenerate \SeminormedSpace.
Let $(Y, \norm{\cdot}_Y)$ be a \SeminormedSpace.
We denote with $BL\pa{(X,\norm{\cdot}_X), (Y, \norm{\cdot}_Y)}$ 
the collection of
\ContinuousFunction
\Linear
operators
$T:(X, \norm{\cdot}_X) \to (Y, \norm{\cdot}_Y)$. 
When the topologies on X and Y are understood, we denote this set with
$BL\pa{X,Y}$. 
We refer to $BL\pa{X,Y}$ as the \SpaceOfBoundedLinearOperators 
from $(X, \norm{\cdot}_X)$ to $(Y, \norm{\cdot}_Y)$ 
, or when $\norm{\cdot}_X$ and $\norm{\cdot}_Y$ are understood, 
from X to Y. 

We endow $BL\pa{X,Y}$ with the algebraic operations
of pointwise scalar multiplication
and pointwise addition, making $BL\pa{X,Y}$ a vector space. 

We define $\norm{\cdot}:BL(X,Y) \to [0,\infty)$ by defining, 
for $T \in BL(X,Y)$
\begin{equation}
    \norm{T} = \sup\limits_{\norm{x}_X \neq 0} \frac{\norm{Tx}_Y}{\norm{x}_X}
\end{equation}
As will be proven in \ref{prop:BLO}, $\norm{\cdot}$ is a \Seminorm on $BL(X,Y)$, which 
we refer to as the \OperatorSeminorm on $BL(X,Y)$. induced by the
\Seminorm $\norm{\cdot}_X$ on X and the \Seminorm $\norm{\cdot}_Y$ on Y. 

In the case that $\norm{\cdot}_{Y}$ is a \Norm, rather than just a \Seminorm, by \ref{prop:BLO}
, $\norm{\cdot}$ is a \Norm on $BL(X,Y)$, which we instead call the \OperatorNorm. 
\end{df}

\begin{prop}[Space of Bounded Linear Operators On Seminormed Spaces]
\label{prop:BLO} 
Let $(X,\norm{\cdot}_X)$ be a \SeminormedSpace. 
Let $(Y, \norm{\cdot}_Y)$ be a \SeminormedSpace.
Let $BL(X,Y)$ denote the \SpaceOfBoundedLinearOperators from X to Y. 
Let $\norm{\cdot}$ denote the \OperatorSeminorm. 

The following are true. 
\begin{enumerate}
%For Item 1, may have to prove result connecting 
%pseudometric topology continuity to $\epsilon-delta$ cotninuity wrt the pseudometric. 
\item $\norm{\cdot}$ is in fact a well-defined \Seminorm on $BL(X,Y)$. 
\item If $\norm{\cdot}_Y$ is a \Norm, then so is $\norm{\cdot}$. 
\item If $T \in BL(X,Y)$ and $\alpha \in (0,\infty)$, then $\norm{T} = \sup\limits_{\norm{x}_X =\alpha} \frac{\norm{Tx}_Y}{\norm{x}_X}$. 
\item If $T \in BL(X,Y)$ and $\alpha \in (0,\infty)$, , then $\norm{T} = \sup\limits_{0<\norm{x}_X \leq \alpha} \frac{\norm{Tx}_Y}{\norm{x}_X}= \sup\limits_{0<\norm{x}_X < \alpha} \frac{\norm{Tx}_Y}{\norm{x}_X}$. 
\item If $T \in BL(X,Y)$ and $x \in X$, then $\norm{Tx}_Y \leq \norm{T} \norm{x}_X$. 
\item $S:X \to Y$ is linear
, $S(\Ker_X) \subset \Ker_Y$
, and $\sup\limits_{\norm{x}_X \neq 0} \frac{\norm{Sx}_Y}{\norm{x}_X} < \infty$
, if and only if $S \in BL(X,Y)$. 
\item A sequence $\{T_i\}_{i \in \N}$ is a \PseudometricCauchySequence
    if and only if
    there exists an $\alpha > 0$ 
    such that the collection of sequences 
    $\{\{T_ix\}_{i \in \N} | x \in B_X(0;\alpha)\}$ is
    \UniformlyCauchy
    if and only if
    for every $\beta > 0$, 
    the collection of sequences 
    $\{\{T_ix\}_{i \in \N} | x \in B_X(0;\beta)\}$ is
    is \UniformlyCauchy
\item If $T_i \to T$ with respect to $\norm{\cdot}$, then $T_ix \to Tx$ with respect to $\norm{\cdot}_Y$ for each $x \in X$
\item A sequence $\{T_i\}_{i \in \N} \subset BL(X,Y)$ 
    converges %TODO: Turn converges into a macro (net based) referenced via \Coverges, and insert that here. 
    with respect to $\norm{\cdot}$ 
    if and only if it is a \PseudometricCauchySequence 
    and for each $x_\alpha$ 
    in some Hamel basis $\{x_\alpha\}_{\alpha \in A} \subset X$,
    the sequence $\{T_ix_\alpha\}_{\alpha \in A}$
    converges with respect to $\norm{\cdot}_Y$. 
\item $BL(X,Y)$ is complete if and only if Y is. 

\item $\norm{\cdot}$ is \NonDegenerate if and only if Y is. \bf THIS WILL NEED TO BER MOVED LATER, UNTIL AFTER THE SEMINORMED HAHN BANACH THEOREM \rm
\end{enumerate}


\begin{proof}[Proof of 1] 
    Since X is nondegenerate, there exists at least 1 $x \in X$ with $\norm{x}_X \neq 0$, 
    so for each $T \in BL(X,Y)$, the set that the supremum is being taken over is nonempty.
    Also, it is clear that $Range(\norm{\cdot}) \subset [0,\infty)$, 

    For \Subadditivity, let $T_i \in BL(X,Y)$ for $i \in \{0,1\}$. and $x \in X$ with $\norm{x} > 0$.
    Then, since $\norm{\cdot}_Y$ is \Subadditive, 
    \begin{align*}
    \frac{\norm{(T_0+T_1)x}_Y}{\norm{x}_X} \leq \frac{\norm{T_0x}_Y}{\norm{x}_X}+ \frac{\norm{T_1x}_Y}{\norm{x}_X}
    \end{align*}
    Since this is true for each x with $\norm{x}_X \neq 0$, taking the supremum of each side yields

    \begin{align*}
    \sup\limits_{\norm{x}_X \neq 0} \pa{\frac{\norm{(T_0+T_1)x}_Y}{\norm{x}_X}} & \leq\sup\limits_{\norm{x}_X \neq 0} \pa{ \frac{\norm{T_0x}_Y}{\norm{x}_X}+ \frac{\norm{T_1x}_Y}{\norm{x}_X}}\\
& \leq\sup\limits_{\norm{x}_X \neq 0} \pa{ \frac{\norm{T_0x}_Y}{\norm{x}_X}} + \sup\limits_{\norm{x}_X \neq 0} \pa{\frac{\norm{T_1x}_Y}{\norm{x}_X}}\\
    \end{align*}
    Hence, $\norm{T_0+T_1} \leq \norm{T_0}+\norm{T_1}$ so that $\norm{\cdot}$ is \Subadditive. 
    For \ScalarHomogeneity, let $T \in BL(X,Y)$, $\alpha \in \F$, and $x \in X$ with $\norm{x}_X \neq 0$. 
    Then 
    \begin{align*}
        \frac{\norm{(\alpha T)x}_Y}{\norm{x}_X} = \frac{\norm{\alpha (Tx)}_Y}{\norm{x}_X} = \abs{\alpha} \frac{\norm{Tx}_Y}{\norm{x}_X}
    \end{align*}
    Hence taking the supremum finishes the proof.
\end{proof}
\begin{proof}[Proof of 2] 
   Let $T \neq 0 \in BL(X,Y)$. Then for some $x \in X$, $Tx \neq 0$. 
   Then $Tx$ has a neighborhood U disjoint from $0_Y$, 
   Hence $x \in T^{-1}(U)$ but not $0_X \in T^{-1}(U)$, since $T0_X = 0_Y$.
   Since U is a neighborhood of x disjoint from 0, 
   there is an $\epsilon > 0$ such that $0_X \subset \complement U \subset \complement \overline{B_X}(x;\epsilon)$,
   and therefore $\norm{x}_X > \epsilon$. 
   Since $\norm{x}_X > 0$, it is ranged over in the supremum defining $\norm{T}$, and so
   \begin{equation}
   0 < \frac{\norm{Tx}_Y}{\norm{x}_X} \leq \sup\limits_{\norm{x}_X \neq 0} \frac{\norm{Tx}_X}{\norm{x}_X}=\norm{T}
   \end{equation}
\end{proof}
\begin{proof}[Proof of 3] 
    Let $\alpha \in (0,\infty)$
   Let $T \in BL(X,Y)$. 
   Then, there is a sequence $\{x_i\} \subset X$ with each $\norm{x_i}_X \neq 0$ 
   such that 
   \begin{equation}
    \frac{\norm{Tx_i}_Y}{\norm{x_i}_X} \to \norm{T}
    \end{equation}
    For each $i \in \N$, define $y_i =\alpha  x_i/\norm{x_i}_X$. 
    then each $\norm{y_i} = \alpha$, 
    and by \ScalarHomogeneity
    of T, we have 
    \begin{equation}
    \frac{\norm{Ty_i}_Y}{\norm{y_i}_X} = \frac{\norm{Tx_i}_Y}{\norm{x_i}_X} \to \norm{T}
    \end{equation}
    , completing the proof. 
\end{proof}
\begin{proof}[Proof of 4] 
If we define, for 
$T \in BL(X,Y)$, 
$f(T) = \sup\limits_{0 < norm{x}_X \leq \alpha} \frac{\norm{Tx}_Y}{\norm{x}_X}$, then
since $\norm{\cdot}^{-1}((0,\alpha))\subset \norm{\cdot}^{-1}((0,\infty))$, we have $f(T) \leq \norm{T} $
and since $\norm{\cdot}^{-1}(\{\alpha\})\subset \norm{\cdot}^{-1}((0,\alpha))$, we have $\norm{T} \leq f(T)$. proving the first equality.
The second is found by applying the same arguement to $\alpha/2$ and realizing that $(0,\alpha/2] \subset (0,\alpha)$. 
\end{proof}
\begin{proof}[Proof of 5]
Let $T \in BL(X,Y)$ and $x \in X$. 
If $\norm{Tx}_Y \neq 0$, then $B_Y(Tx, \frac{\norm{Tx}_Y}{2})$ is a neighborhood of $Tx$ disjoint from 0.
Continuity of T impliese $x$ then has a neighborhood disjoint from $0 \in T^{-1}(0)$, implying
that $\norm{x}_X \neq 0$. 

Hence if $\norm{x}_X = 0$, then we know $\norm{Tx}_Y = 0$, so that the relation
\begin{equation}
\norm{Tx}_Y \leq \norm{T} \norm{x}_X
\end{equation}

If $\norm{x}_X \neq 0$, then by definition of supremum, 
\begin{equation*} 
\frac{\norm{Tx}_Y}{\norm{x}_X} \leq \norm{T}
\end{equation*}
so that $\norm{Tx}_Y \leq \norm{T} \norm{x}_X$. 
\end{proof}
\begin{proof}[Proof of 6]
    I assume the first 3 conditions
    and show that $S \in BL(X,Y)$.
    It is necessary and sufficient to 
    show that S is continuous.
    Let $F= \sup\limits_{\norm{x}_X \neq 0} \frac{\norm{Sx}_Y}{\norm{x}_X}$.
    If $F=0$, then $S(X) \subset \Ker_Y$. 
    Every neighborhood of every point in $\Ker_Y$
    contains $\Ker_Y$, so in that case continuity holds. 
    Suppose $F \neq 0$. 
    By translation invariance of the topology, 
    it is sufficient to consider neighborhoods of $0_Y \in Y$. 
    Let $\epsilon > 0$. 
    Define $V=B_X\pa{0; \frac{\epsilon}{F}}$. 
    Let $x_0 \in V$. 
    If $\norm{x_0}_X = 0$, then
    $S(x_0) \in S(\Ker_X) \subset \Ker_Y \subset B_Y(0;\epsilon)$. 
    If $\norm{x_0}_X \neq 0$, then 
    $\norm{Sx}_Y \leq F \norm{x}_X < \epsilon$, so
    $s(x_0) \in B_Y(0;\epsilon)$. 
    Hence $S\pa{B_X\pa{0;\frac{\epsilon}{F}}} \subset B_Y(0; \epsilon)$.
    so S is continuous,and this direction fo the proof is complete.

    Suppose conversely that $S \in BL(X,Y)$. 
    Then S is linear by definition
    , and the supremum expression is finite by part 1 
    of this result. 
    Since S is linear, $S0_X = 0_Y$. 
    Since S is continuous, 
    \begin{align*}
        S(\Ker_X) &= S\pa{\overline{\{0_X\}}}\\
        & \subset \overline{S\pa{\{0_X\}}}\\
        & =\overline{\{0_Y\}} \\
        & = \Ker_Y
    \end{align*}



\end{proof}
\begin{proof}[Proof of 7]
    $(3 \implies 2)$ is trivial, as is $(2 \implies 3)$.
    
    I now prove $(1\implies 3)$. 
    Let $\{T_i\}_{i \in \N}$ be a 
    \PseudometricCauchySequence.
    Let $\beta > 0$. 
    Let $\epsilon > 0$. 
    Then there exists $N \in \N$
    such that for $m,n > N$, 
    \begin{equation*}
    \norm{T_n-T_m} < \frac{\epsilon}{\beta}
    \end{equation*}
    Let $x \in B_X(0;\beta)$. 
    Then 
    \begin{align*}
        \norm{T_mx-T_nx}_Y & = \norm{(T_m-T_n)x}_Y\\
        & \leq \norm{T_m-T_n} \norm{x}_X\\
        & < \epsilon
    \end{align*}
    Since $x \in B_X(0;\beta)$ was arbitrary,
    $\{\{T_ix\}_{i \in \N} | x \in B_X(0;\beta)\}$ is
    \UniformlyCauchy.

    I now prove $(3 \implies 1)$. 
    Let $\epsilon > 0$.
    Then there is an $N \in \N$ 
    such that for $m,n>N$, 
    for each $x \in B_X(0;2)$, 
    \begin{equation*}
    \norm{T_mx-T_nx} < \epsilon
    \end{equation*}
    In particular, if $\norm{x}=1$, then 
    \begin{equation}
    \frac{\norm{(T_m-T-n)x}_Y}{\norm{x}_X} =\norm{(T_m-T_n)x}_Y < \epsilon
    \end{equation}
    Hence, by taking the supremum over such x
    and applying part 3 of this result, 
    $\norm{T_m-T_n} < \epsilon$. 
\end{proof}
\begin{proof}[Proof of 8]
    Let $T_i \to T$. 
    Let $x \in X$. 
    If $x \in \Ker_X$, then $T_i(x) \in \Ker_Y$ for $i \in \N$ and $T_x \in \Ker_Y$, 
    so convergence is obvious. 
    Suppose $\norm{x}_X > 0$. 
    Let $\epsilon > 0$. 
    Then there exists $N \in \N$ such that
    for $n>N$, $\norm{T_n-T} < \frac{\epsilon}{\norm{x}_X}$.
    For such n, 
    \begin{equation*}
    \norm{T_ix-T_x}_Y \leq \norm{T_i-T} \norm{x}_X < \epsilon
    \end{equation*}
\end{proof}
\begin{proof}[Proof of 9]
%Pick up here, and prove it
\end{proof}
\begin{proof}[Proof of 6] %Remove
    Let $T_i \subset BL(X,Y)$ conveerge, say $T_i \to T \in BL(X,Y)$. 
    Let $\epsilon > 0$. 
    Then there is an $N \in \N$
    such that for $n>N$, 
    we have 
    $\norm{T_i-T} < \epsilon$. 
    For these n, for any $x \in \overline{B_X}(0;1)$, we have
    \begin{align*}
    \norm{T_ix-Tx}_Y& = \norm{(T_i-T)x}_Y \\
    & \leq \norm{T_i-T} \norm{x}_X\\
    & < \epsilon 
    \end{align*}
    So that $T_ix \to Tx$ uniformly for $x \in \overline{B_X}(0;1)$. 

    Suppose conversely that we have a sequence
    $\{T_i\}_{i \in \N}$ 
    such that $T_ix \to y(x)$ for each $x \in \overline{B_X}(0;1)$. 
    I proceed through three separate claims.
    
    \bf claim 01 \rm if $T_ix \to y(x)$ uniformly for $x \in \overline{B_X}(0;1)$, then
    there is a linear operator $T:X \to Y$ such that pointiwise everywhere in X, $T_ix \to Tx$. 
    \bf FOR CLAIM 01 \rm: Take a hamel bnasis $\{x_\alpha\}$, normalize it to get $y(z_\alpha)$ define T in terms of linear combinations. SHow that T is linear. Then work on the boundedness arguement. 
    

    \bf claim 02 \rm The linear operator T guaranteed to exist in claim 01 is bounded, that is, 
    that $\norm{T}$ is well defined and finite.

    \bf claim 03 \rm  $\norm{T_i-T} \to 0$.


    \bf Proof of claim 01 \rm Let $x \in X$.
    Define $z=x/\norm{x}_X$. 
    Then $z \in \overline{B_X}(0;1)$. 
    Hence, by assumption, there is a (not necessisarily unique)
    $y(z) \in Y$ such that 
    $T_iz \to y(z)$. 
    Let $\epsilon > 0$. 
    Then, there is an $N \in \N$ such that for $n>N$, we have
    $\norm{T_nz-y(z)}_Y < \frac{\epsilon}{\norm{x}_X}$. 
    For these n, 
    \begin{align*}
    \norm{T_nx-\norm{x}_Xy(z)}_Y & = \norm{x}_X \norm{T_nz-y(z)} \\
    & < \epsilon
    \end{align*}
    Hence $T_i(x) \to \norm{x}_Xy(z)$
    , so our candidate map is $T:X \to Y$ defined by 
    \begin{equation*}
        \begin{dcases}
            T(x) = y(x)  & x \in \overline{B_X}0;1
        \end{dcases}
    \end{equation*}
    Hence there is a map $T:X \to Y$ such that $T_ix \to Tx$ pointwise for $x \in X$. 
    %This map is linear, 
    %as if $\alpha\in \F$ 
    %and $x,y \in X$, and 
    %$\epsilon > 0$, then
    %There is an $N \in \N$ 
    %such that for $n>N$, we have
    %\begin{equation}
    %FILL IN CONDITION
    %\end{equation}
    %and for these n, we have 
    %\begin{align*}
    %T(\alpha x + y)
    %\end{align*}
\end{proof}

\end{prop}


\label{def:handedquotientoperators}
\newcommand{\CodomainQuotientOperator}[0]{
    \bf \hyperref[def:handedquotientoperators]{Codomain Quotient Operator} \rm
}
\newcommand{\CodomainQuotientMap}[0]{
    \bf \hyperref[def:handedquotientoperators]{Codomain Quotient Map} \rm
}
\begin{df}[Codomain Quotient Operator]
    Let X and Y be \SeminormedSpaces.
    Define $\scQ_Y:BL(X,Y) \to BL(X, Y/\Ker_Y)$ by setting, 
    for each $x \in X$, 
    \begin{equation*} 
        \scQ_YTx = [Tx]
    \end{equation*}
    Let $T \in BL(X,Y)$. 
    We call $\scQ_Y$ the \CodomainQuotientMap of X and Y
    and we call $\scQ_YT$ the 
    \CodomainQuotientOperator
    of T.
\end{df}

\begin{prop}[Codomain Quotient Operator]
\label{prop:handedquotientoperators}
    Let X and Y be 
    \SeminormedSpaces
    with \CodomainQuotientMap $\scQ_Y$. 
    The following are true. 
    \begin{enumerate}
        \item $\scQ_Y$ is a well defined continuous linear surjective isometry. 
        \item If Y is a \NormedSpace, then $\scQ_Y$ is invertible with a continuous inverse. 
    \end{enumerate}
    \begin{proof}[Proof Of 1]
        Since $Tx \in Y$ for any $x \in X$, 
        $[Tx]_Y$ is defined for any $x \in X$. 
        Furthermore, if $q_y:Y \to Y/\Ker$
        is the \QuotientMap of Y under 
        \EquivalenceModKernel, then 
        $\scQ_YT = q_y \circ T$. 
        By \ref{prop:quotientnormspace}, 
        $q_y$ is linear and an isometryu, and hence continuous.
        Therefore, 
        $q_y \in BL(Y, Y/\Ker)$. 
        Hence $\scQ_Y$ is well defined. 

        For linearity, let $\alpha \in \F$
        and $S,T \in BL(X,Y)$. 
        Let $x \in X$. 
        Then, 
        \begin{align*}
            \scQ_Y\pa{\alpha T+S}x & = \bra{\pa{\alpha T+S}x}_Y\\
            & = \bra{\alpha Tx+ Sx}_Y\\
            & = \bra{\alpha Tx}_Y+ \bra{Sx}_Y\\
            & = \alpha \bra{Tx}_Y+\bra{Sx_Y}\\
            & = \alpha \scQ_YTx+ \scQ_YSx\\
            & = \pa{\alpha \scQ_YT+\scQ_YS}x
        \end{align*}

        For being an isometry, 
        let $T \in BL(X,Y)$ and 
        let $x \in X$. Then, since $\norm{\bra{Tx}_Y}_{Y/\Ker} = \norm{Tx}_Y$, 
        \begin{align*}
            \frac{\norm{\scQ Tx}_{Y/\Ker}}{\norm{x}_X} & = \frac{\norm{\bra{Tx}_Y}_{Y/\Ker}}{\norm{x}_X} \\
            & = \frac{\norm{Tx}_Y}{\norm{x}_X} 
        \end{align*}
        and thus taking the norm over
        x with $\norm{x}_X \neq 0$ will yield the
        same result. Hence $\norm{T} = \norm{\scQ_Y T}$. 

        For surjectivity, let $\tilde{T} \in BL(X, Y/\Ker_Y)$. 
        Let $\{x_{\alpha}\}_{\alpha \in A}$ be a hamel basis for $X$. 
        For each $\alpha \in A$, let $y_{\alpha} \in \tilde{T}x_{\alpha}$. 
        Define $T:X \to Y$ by 
        \begin{equation}
            T\pa{\sum_{i=1}^n \beta_{\alpha_i} x_{\alpha_i}} = \sum_{i=1}^n \beta_{\alpha_i} y_{\alpha_i}
        \end{equation}
        T is obviously linear
        and has the property $[Tx]=\tilde{T}x$. 
        and since $\tilde{T} \in BL(X,Y/\Ker)$, 
        $\tilde{T}\Ker_X \subset \Ker_{ (Y/\Ker_Y)}=0$. 
        Hence $T \Ker_X \subset \Ker_Y$. 
        Furthermore, if $x \in X$ with $\norm{x}_X \neq 0$, then
        \begin{align*}
        \frac{\norm{Tx}_Y}{\norm{x}_X} & = \frac{\norm{\bra{Tx}_Y}_{Y/\Ker}}{\norm{x}_X}\\
        & = \frac{\norm{\tilde{T}x}_{Y/\Ker}}{\norm{x}_X}
        \end{align*}
        Therefore $T$ is bounded.
        Hence $T \in BL(X,Y)$, and $\scQ_YT=\tilde{T}$. 
        Thus we have surjectivity, and are done.
    \end{proof}
    \begin{proof}[Proof Of 2]
        If $Y$ is a \NormedSpace, 
        %then $q_y:Y \to Y/\Ker_Y$ is 
        a linear isometric homeomorphism by 
        \ref{prop:quotientnormspace}. 
        In particular, in this case, 
        $q_y$ is injective, meaning that 
        if $T,S \in BL(X,Y)$ where
        $T \neq S$, then 
        $Tx_0 \neq Sx_0$ for some $x_0 \in X$. 
        For this $x_0$, $q_yTx_0 \neq q_ySx_0$, so 
        $\scQ_YT \neq \scQ_YS$. 
        Therefore $\scQ_Y$ is injective, and therefore a bijection. 
        The inverse of an isometry is also an isometry 
        and therefore continuous, finishing this proof. 
    \end{proof}
\end{prop}

\label{def:quotientoperator}
\newcommand{\QuotientOperator}[0]{\textbf{\hyperref[def:quotientoperator]{Quotient Operator}}\xspace}
\newcommand{\OperatorQuotientMap}[0]{\textbf{\hyperref[def:quotientoperator]{Operator Quotient Map}}\xspace}
\begin{df}[Quotient Operator]
    Let $X,Y$ be \SeminormedSpaces
    with \SeminormKernels $\Ker_X$, $\Ker_Y$. 
    Define $Q:BL(X,Y) \to BL(X/\Ker_X, Y/\Ker_Y)$ by 
    setting, for $T \in BL(X,Y)$, 
    for $x \in X$, 
    \begin{equation}
    QT\bra{x}_X=\bra{Tx}_Y
    \end{equation}
    We call Q the \OperatorQuotientMap of X and Y and
    we call QT the \QuotientOperator of T. 
\end{df}



\begin{prop}[Quotient Operator]
\label{prop:quotientoperator}
    Let $X,Y$ be \SeminormedSpaces
    with \SeminormKernels $\Ker_X$, $\Ker_Y$
    and \OperatorQuotientMap Q. 
    Then Q is a well-defined linear surjective isometry. 
    \begin{proof} 
        We first show that Q is well defined. 
        Let $T \in BL(X,Y)$ and 
        let $x_0, x_1 \in X$ such that $\bra{x_0}=\bra{x_1}$. 
        Then $\norm{x_0-x_1}_X = 0$, so since T is continuous, 
        $\norm{Tx_0-Tx_1}_Y = 0$. 
        Hence $Tx_0 \cong Tx_1$, so
        $\bra{Tx_0} = \bra{Tx_1}$. 


        For linearity, let $\alpha\in \F$, and let
        $T,S \in BL(X,Y)$. 
        Let $x \in X$. 
        Then 
        \begin{align*}
            Q\pa{\alpha T+S}\bra{x}_X & = \bra{\pa{\alpha T+S}x}_Y\\
            & = \alpha \bra{Tx}_Y + \bra{Sx}_Y\\
            & = \alpha QT[x]_X +QS[x]_X\\
            & = \pa{\alpha QT+QS}[x]_X
        \end{align*}
        Since $x \in X$ was arbitrary, Q is linear. 

        As for being an isometry, let $T \in BL(X,Y)$ and let $x \in X$. 
        Since $\norm{\bra{x}}=\norm{x}$ and $\norm{Tx}=\norm{\bra{Tx}}$, 
        we have 
        \begin{align*}
        \frac{\norm{QT\bra{x}_{X/\Ker_X}}_{Y/\Ker_Y}}{\norm{\bra{x}}_{X/\Ker_X}}  & =  \frac{\norm{\bra{Tx}}_{Y/\Ker_Y}}{\norm{\bra{X}}_{X/\Ker_X}}\\
        & = \frac{\norm{Tx}_Y}{\norm{x}_X} 
        \end{align*}
        and so taking the supremum over $\norm{x} \neq 0$ gives us 
        that this is an isometry. 
        

        For surjectivity, let $\tilde{T} \in BL(X/\Ker_X, Y/\Ker_Y)$.
        Let $\{x_\alpha\}_{\alpha \in A}$ be a Hamel basis for X. 
        For each $\alpha \in A$, 
        let $y_\alpha \in \tilde{T}[x_\alpha]_X$. 
        Now define 
        \begin{equation}
            T\sum_{i=1}^n \beta_i x_{\alpha_i} = \sum_{i=1}^n \beta_i y_{\alpha_i}
        \end{equation}
        Then $T:X \to Y$ is obviously linear, and
        $Tx \in \tilde{T}[x]_X$ for $x \in X$. 
        Hence, 
        \begin{equation}
            \frac{`\norm{Tx}_{Y}}{\norm{x}_X} = \frac{\norm{\tilde{T}[x]_X}_{Y/\Ker_Y}}{\norm{[x]_X}_{X/\Ker_X}}
        \end{equation}
        so T is bounded, and hence $T \in BL(X,Y)$, 
        but that also implies that by definition, 
        $QT=\tilde{T}$, so we have proven surjectivity. 
    \end{proof}

\end{prop}

\label{def:canonicalisomorphism}
\newcommand{\CanonicalIso}[0]{
    \bf \hyperref[def:canonicalisomorphism]{Canonical Isomorphism Of The Quotient Space Of Continuous Linear Operators} \rm
}

\begin{df}[Canonical Isomorphism Of The Quotient Space Of Continuous Linear Operators]
    Let $X,Y$ be \SeminormedSpaces
    with \SeminormKernels $\Ker_X$, $\Ker_Y$.
    Let $\Ker$ denote the \SeminormKernel of $BL(X,Y)$. 
    Let Q denote the \OperatorQuotientMap of X and Y.
    Define $\Theta_{(X,Y)}:BL(X,Y)/\Ker \to BL(X/\Ker_X, Y/\Ker_Y)$ by 
    setting, for each $T \in BL(X,Y)$. 
    \begin{equation}
        \Theta_{(X,Y)}(\bra{T}) = QT
    \end{equation}
    We call $\Theta_{(X,Y)}$ the \CanonicalIso from X to Y. 
    When X and Y are understood, we may denote the
    \CanonicalIso simply with $\Theta$. 
    By \ref{prop:canonicalisomorphism}, $\Theta_{(X,Y)}$
    is an isomorphism of \NormedSpaces.
    That is, $\Theta$ is Linear, Bijective, Bicontinuous, and an isometry. 
\end{df}




\begin{prop}[Canonical Isomorphism Of The Quotient Space Of Continuous Linear Operators]
\label{prop:canonicalisomorphism}
    Let $X,Y$ be \SeminormedSpaces.
    Let $\Theta$ denote the \CanonicalIso
    from X to Y. 
    Then $\Theta$ is
    a bijective, bicontinuous, linear, isometry. 
    \begin{proof}
       By \ref{prop:quotientnormspace}, part 1, 
       $Y/\Ker_Y$ is a \NormedSpace, 
       Hence by \ref{prop:BLO}, part 2, 
       \newline
       $BL(X/\Ker_X, Y/\Ker_Y)$ is a \NormedSpace. 
       Similarly, by \ref{prop:quotientnormspace}, part 1, 
       $BL(X,Y)/\Ker$ is a normed space. 
       Hence, it is sufficient to show that $\Theta$ is a
       well-defined surjective linear isometry. 

       For well definedness, let $T,S \in BL(X,Y)$ with $[T]=[S]$. 
       Then, $\norm{T-S}=0$, so 
       if $x \in X$, $\norm{Tx-Sx}=0$. 
       Hence $Tx \cong Sx$ and since x was arbitrary, 
       $QT=QS$. 
        
       Let q denote the \QuotientMap $q:BL(X,Y) \to BL(X,Y)/\Ker$. 
       By parts 4, 5, and 6 of \ref{prop:quotientnormspace}, 
       q is a linear surjective isometry. 
       Also, by definition, $\Theta \circ q = Q$. 
       Since Q is surjective, $\Theta$ is surjective. 
       Since $Q$ is an isometry, and $q$ is a surjecive isometry, 
       $Theta$ is an isometry. 
       Since Q is linear, and since q is surjective and linear, 
       $\Theta$ is linear. 
    \end{proof}


\end{prop}

\newcommand{\SemiTopDualSpace}[0]{\textbf{\hyperref[def:topologicaldualspace]{Topological Dual Space}}\xspace}
\begin{df}\bf REMOVE \rm
\label{def:topologicaldualspace}
\rm
\end{df}

\label{def:dualspace}
\newcommand{\TopDualSpace}[0]{
    \bf \hyperref[def:dualspace]{Topological Dual Space} \rm
}
\begin{df}[Dual Space]
    Let $(X,\norm{\cdot})($ be a 
    \SeminormedSpace.
    We call $BL(X, \F)$ the 
    \TopDualSpace of 
    $(X,\norm{\cdot})$,
    and we denote 
    $BL(X,\F)$ with the symbol 
    $X^*$. 
    If $x^* \in X^*$, then
    we use the notational convention
    of writing, for $x \in X$. 
    \begin{equation}
    \ip{x, x^*}:= x^*(x)
    \end{equation}
\end{df}
\begin{rmk}[\TopDualSpace is a \NormedSpace]
    Let $X$ be a 
    \SeminormedSpace.
    Then, using 
    \ref{prop:BLO}, 
    since $\F$ is a 
    \NormedSpace, 
    so is $X^*$. 
\end{rmk}

\begin{thm}[\TopDualSpace Isomorphism]
\label{thm:dualspaceisomorphism}
    Let X be a 
    \SeminormedSpace.
    Define $\Omega:X^* \to \pa{X/\Ker_X}^*$
    by setting, for $x^* \in X$, 
    and for $x \in X$, 
    \begin{equation}
        \ip{x, x^*} = \ip{[x], \Omega x^*}
    \end{equation}
    Then $\Omega$ is a 
    Linear, 
    Bijective, 
    Isometric, 
    Bicontinuous operator. 
    That is, $X^*$ and $(X/\Ker_X)^*$ are 
    isomorphic, and that isomorphism is explicitly
    given by $\Omega$. 
    \begin{proof}
        Consider the following
        \begin{equation}
            BL(X,\F) \overset{q}{\to} BL(X,\F)/\Ker_{BL(X/\F)} \overset{\Theta}{\to} BL(X/\Ker_X, \F/\Ker_{\F}) \overset{\scQ_{\F}^{-1}}{\to} BL(X/\Ker_X,\F)
        \end{equation}
        where
        q is the \QuotientMap, 
        which is an linear bijective bicontinuous isometry in this case
        by parts 4, 5, 6, and 7 of 
        of \ref{prop:quotientnormspace}, 
        $\Theta$ is the \CanonicalIso, 
        which is a linear bijective bicontinuous isometry by 
        \ref{prop:canonicalisomorphism}
        and $\scQ_{\F}$ is the \CodomainQuotientMap.
        which is in this case a linear, bijective, bicontinuous isometry
        by \ref{prop:handedquotientoperators}


        Since $\Omega=\scQ_{\F}^{-1} \circ \Theta \circ q$, 
        and since each of the described properties
        are preserved under composition, 
        $\Omega$ is also a 
        linear bijective bicontinuous isometry. 
    \end{proof}
\end{thm}



%\begin{prop}
    \label{prop:dualspacepushing}
    Let $X,Y$ be \SeminormedSpaces.
    Let $T:X \to Y$ be a
    linear, 
    surjective
    isometry. 
    Define $T^*:X^* \to Y^*$ by 
    setting, for $x^* \in X^*$ and
    $x \in X$, 
    \begin{equation}
    \ip{Tx, T^*x^*} = \ip{x,x^*}
    \end{equation}
    Then $T^*$ is a Linear Bijective Isometry. 
    \begin{proof} 
        I first need to show that $T^*$ is well defined.
        Suppose $Tx \cong Ty$. 
        Since T is an isometry, $x \cong y$. 
        Hence, $\ip{x,x^*} = \ip{y,x^*}$ for all 
        $x^* \in X^*$. so the equation 
        defining $T^*x^*$ is at least consistent. 
        Is $T^*x^*$ linear? Yes, as 
        if $x_1,y_1 \in Y$ and $\alpha \in \F$, 
        then there are $x,y \in X$ such that 
        $Tx=x_1$, $Ty=y_1$, and 
        \begin{align*}
            \ip{\alpha x_1+y_1, T^*x^*} & = \ip{\alpha x+y, x^*}\\
            & = \alpha \ip{x,x^*}+\ip{y,x^*}\\
            & = \alpha \ip{x_1, T^*x^*}+ \ip{y_1, T^*x^*}
        \end{align*}
        Boundedness of $T^*x^*$ will be shown during the 
        proof that $T^*$ is an isometry. 



    For linearity, let $x_i^* \in X^*$ for 
    $i \in \{0,1\}$ and let $\alpha \in \F$. 
    Let $y \in Y$. 
    Since T is surjective, there exists $x \in X$ 
    such that $y=Tx$.
    Then
    \begin{align*}
        \ip{y, T^*\pa{\alpha x_0^*+x_1^*}} & = \ip{x, \alpha x_0^*+x_1^*} \\
        & = \alpha \ip{x, x_0^*} + \ip{x, x_1^*} \\
        & = \alpha \ip{y, Tx_0^*}+ \ip{y, Tx_1^*}\\
        & = \ip{y, \alpha T^* x_0^* + T^* x_1^*}
    \end{align*}

    For surjectivity, let $y^* \in Y^*$. 
    To prove the existence of $x^* \in X^*$ 
    such that $T^*x^* = y^*$, 
    it is sufficient to find$ x^*\in X$
    with $Kernel(T x^*)=Kernel(y^*)$. 


    To see that $T^*$ is an isometry, 
    let $x^* \in X^*$. Then, 
    since T is surjective and an isometry, 
    \begin{equation*}
    \{y \in Y| \norm{y} \neq 0  \} = \{Tx | \norm{x} \neq 0\}
    \end{equation*}
    , which allows us to say that
    \begin{align*}
    \norm{T^*x^*} &= \sup\limits_{\norm{y} \neq 0} \frac{\abs{\ip{y, T^*x}}}{\norm{y}}\\
    &= \sup\limits_{\norm{x} \neq 0} \frac{\abs{\ip{Tx, T^*x^*}}}{\norm{Tx}} \\
    & = \sup\limits_{\norm{x} \neq 0} \frac{\abs{\ip{x,x^*}}}{\norm{x}}\\
    & = \norm{x^*}
    \end{align*}
    so $T^*$ is an isometry. 
    Since $T^*$ is an isometry, and 
    $X^*$ and $Y^*$ are both normed spaces by 
    \ref{thm:dualspaceisomorphism}, 
    $T^*$ is surjective. 

    \end{proof}



\end{prop}


\begin{thm}[Hahn Banach Theorem For Seminormed Spaces]
\label{thm:hahnbanach}
\rm
Let $(X,\norm{\cdot})$ be a \SeminormedSpace,
let $x_i \in X$ for $i \in \{0,1\}$ such that 
$\norm{x_0-x_1}_X \neq 0$, and
let $X^*$ denote $X's$
\TopDualSpace. 
The following are true. 
\begin{enumerate}[label=(\roman*), ref={\ref{thm:hahnbanach}~\roman*}]
    \item 
	\label{thm:HahnBanach:Extension1}
	If $Z \subset X$ is a subspace
        and $z^* \in Z^*$, then there 
        is an extension $x^*$ of $z^*$, 
        $x^* \in X^*$ such that 
        \begin{equation*}
        \norm{z^*}_{Z^*} = \norm{x^*}_{X^*}
        \end{equation*}
     \item 
	 \label{thm:HahnBanach:Point1}
	 If $x \in X$, 
        with $\norm{x} \neq 0$, 
        then there exists an
        $x^* \in X$ with 
        $\norm{x^*}=1$ and 
        $\ip{x,x^*} = \norm{x}_X$. 
    \item 
	\label{thm:HahnBanach:Norm}
	If $x \in X$, then 
    \begin{equation*}
        \norm{x}_X = \sup\limits_{0 \neq x^* \in X^*} \frac{\ip{x,x^*}}{\norm{x^*}}
    \end{equation*}
	
    \item 
	\label{thm:HahnBanach:Operator1}
	If $Y$ is a 
        \NonDegenerate
        \SeminormedSpace, and if 
        $x_0 \in X$, with 
        $\norm{x_0} \neq 0$, 
        then there exists
        an $S \in BL(X,Y)$ with 
        $\norm{S} = 1$ and 
        \begin{equation*}
            \norm{Sx_0} = \norm{x_0}
        \end{equation*}
\end{enumerate}


\begin{proof}[Proof of \ref{thm:HahnBanach:Extension1}]
    For $\alpha \in \{Z,X\}$, let 
    $\Omega_\alpha:\alpha^* \to (\alpha/\Ker_\alpha)^*$ denote the isomorphism
    defined in 
    \ref{thm:dualspaceisomorphism}.
    Let $q$ denote the quotient operator $q:X \to X/\Ker$. 
    Define $T:Z/\Ker_Z \to q(Z)$ bv $T([z]_{\cong_Z} ) = [z]_{\cong_X}$. %Make a separate Result
    Since Z is endowed with the subspace Topology,                       %Make a separate Result
    T is obviously a \Linear \Isometric \Homeomorphism.          %Make a separate Result
    
    %Then $\Omega_Zz^* \in (Z/\Ker_Z)^*$ satisfies
    %$\norm{\Omega_Zz^*}_{Z/\Ker_Z}=\norm{z^*}_{Z^*}$ an 
    Define $\Gamma_Z:(Z/\Ker_Z)^* \to q(Z)^*$ by setting, 
    for $\phi^* \in (Z/\Ker_Z)^*$, 
    for $[z]_Z \in Z/\Ker_Z$, 
    \begin{equation*}
        \ip{T[z]_Z, \Gamma_Z \phi^*} = \ip{[z]_Z, \phi^*}
    \end{equation*}
    Then $\Gamma_Z$ is a \Linear Bijective \Isometry. 
    Hence $\Gamma_Z \circ \Omega_Z z^* \in q(Z)^*$ with 
    $\norm{\Gamma_Z \circ \Omega_Z z^*}_{q(Z)^*} = \norm{z^*}_{Z^*}$. 

    Thus we can apply the Hahn Banach theorem for \NormedSpaces to claim 
    the existence of $x_q^* \in (X/\Ker_X)^*$ where
    $x_q^*$ is an extension of $\Gamma_Z \circ \Omega_Z z^*$ and
    \begin{equation*}
        \norm{x_q^*}_{(X/\Ker_X)^*} = \norm{\Gamma_Z \circ \Omega_Z z^*}_{(q(Z))^*} = \norm{z^*}_{Z^*}
    \end{equation*}
    Finally, letting $x^* = \Omega_X^{-1} x_q^*$, we have 
    $x^* \in X^*$, 
    $\norm{x^*}_{X^*} = \norm{x_q^*}_{(X/\Ker_X)^*} =\norm{z^*}_{Z^*}$, 
    and 
    if $z \in Z$, then 
    \begin{align*}
        \ip{z, x^*} & = \ip{[z]_X, x_q^*} \\
        & = \ip{[z]_X, \Gamma_Z \circ \Omega_Zz^*}\\
        & = \ip{[z]_Z, \Omega_Z z^*}\\
        & = \ip{z, z^*}
    \end{align*}
\end{proof}

\begin{proof}[Proof of 2]
    Let $Z=span(x)$. 
    Define $z^* \in Z^*$ by 
    $\ip{\alpha x, z^*} = \alpha \norm{x}$. 
    Then $\norm{z^*} = 1$. 
    Also, by part 1 of this result, 
    it has an extension $x^* \in X^*$ with 
    $\norm{x^*} = \norm{z^*} =1$ 
    and $\ip{x,x^*} = \norm{x}$. 
\end{proof}
\begin{proof}[Proof of 3]
    If $\norm{x} = 0$, then
    for every $x^* \in X$, $\ip{x,x^*} = 0$.
    Hence 
    \begin{equation*} 
    \norm{x}_X = \sup\limits_{0 \neq x^* \in X^*} \frac{\ip{x,x^*}}{\norm{x^*}} = \sup\limits_{x^* \in \partial B_{X^*}(0;1)} \frac{\ip{x,x^*}}{\norm{x^*}}=0
    \end{equation*}

    Otherwise, let  $x^* \in X^*$ guaranteed to 
    exist by part 2 which satisfies $\norm{x^*}=1$, 
    $\ip{x,x^*} = \norm{x}$. 
    Then 
    \begin{align*}
    \norm{x} & = \frac{\ip{x,x^*}}{\norm{x^*}} \\
    & \leq \sup\limits_{x^* \in \partial B_{X^*}(0;1)} \frac{\ip{x,x^*}}{\norm{x^*}} \\
    & \leq    \sup\limits_{0 \neq x^* \in X^*} \frac{\ip{x,x^*}}{\norm{x^*}} 
    \end{align*}
    The other direction of the inequality
    falls directly from the definition of 
    the norm on $X^*$, and is trivial, so 
    we are done. 
\end{proof}
\begin{proof}[Proof of 4]
    By part 2 of this result, there exists $x_0^* \in X^*$ with 
    $\norm{x_0^*} = 1$ and $\ip{x_0, x_0^*}=\norm{x_0}$. 
    Since Y is \NonDegenerate, there
    exists $y_0 \in Y$ with $\norm{y_0} = 1$. 
    Define $T: \F \to Y$ by $T \alpha = \alpha y$.
    Then $\norm{T} = \norm{y} = 1$. 
    Define $S: X \to Y$ by $S=T \circ x_0^*$. 
    Then $\norm{S} \leq \norm{T} \norm{x_0^*} = 1$, and
    $\norm{S x_0} = \norm{\ip{x_0, x_0^*} y} = \ip{x_0, x_0^*} = \norm{x_0}$. 
    Hence $\norm{S} \geq 1$ and therefore $\norm{S} = 1$. 
\end{proof}
\end{thm}


\begin{prop}[Linear Operator Notation]
\label{rmk:linearoperatornotation}
    $.$
    When dealing with mappings of 
    spaces of linear operators into
    spaces of other linear operators, 
    or even functions in general, 
    notation can get confusing, and
    presenting such things using ordinary notation without
    ambiguity can often require a plethora of parenthesis, 
    which hamper readability of an arguement. 

    For this reason, at points in this document, 
    I sometimes express the image $\beta(\alpha)$ using 
    \begin{equation*}
        \ip{\alpha, \beta}
    \end{equation*}
    Where $\beta:X \to Y$ 
    and $\alpha \in X$. 

    I combine this notation with usual function notation, 
    particularly in cases similar to the following. 
    For $i \in \{0,1\}$, 
    let $X_i, Y_i, Z_i$ be sets. 
    For $\alpha \in \{X,Y,Z\}$, let 
    $F_\alpha$ be the set of maps $f:\alpha_0 \to \alpha_1$. 
    If $T:F_X \to F_Y$, 
    $y \in Y_0$, 
    and $f \in F_X$, then I would notate
    \begin{equation*}
        \ip{y, Tf}
    \end{equation*}
    rather than $Tf(y)$ or $(T(f)(y))$
\end{prop}

\label{def:adjointoperator}
\newcommand{\AdjointOperator}[0]{
    \bf \hyperref[def:adjointoperator]{Adjoint Operator} \rm
}
\begin{df}[\AdjointOperator]
    Let X, Y, and Z be \SeminormedSpaces
    over a field $\F \in \{\R, \C\}$. .
    Let $T \in BL(X,Y)$. 
    We define the operator
    $T^{\times}_Z:BL(Y,Z) \to BL(X,Z)$ by 
    setting, for $S \in BL(Y,Z)$ 
    and $x \in X$, 
    \begin{equation}
        \ip{x , \T^{\times}_{Z}S } = \ip{Tx, S}
    \end{equation}
    or, equivalently, 
    \begin{equation}
    \T^{\times}_Z S = S \circ T
    \end{equation}

    We call $T^\times_Z$ the \AdjointOperator
    of T relative to the space Z, we denote
    $T^\times_{\F}=T^\times$, and
    we refer to $T^\times:Y^* \to X^*$ as 
    simply the \AdjointOperator of T. 
\end{df}

\begin{prop}[\AdjointOperator]
\label{prop:adjointoperator}
\rm
Let $X$, $Y$, and $Z$ be \SeminormedSpaces
over a \Field $\F \in \{\R, \C\}$.
Let $T \in BL(X,Y)$. 
Let $\T=T^\times_Z$ denote the 
\AdjointOperator of T
relative to the space $Z$. 
Let $Q_Y:Y \to Y/\Ker_Y$ denote the \QuotientMap
The following are true. 
\begin{enumerate}[label=(\roman*), ref={\ref{prop:adjointoperator}~\roman*}]
\item 
\label{prop:Adjoint:Linear}
$\T$ is \Linear. 
\item 
\label{prop:Adjoint:WellDefined}
If $S \in BL(Y,Z)$, then $\T S \in BL(X,Z)$. (That is, the \AdjointOperator is well defined as a concept).
\item 
\label{prop:Adjoint:Continuous}
$\T \in BL\pa{ BL(Y,Z), BL(X,Z)}$. 
\item 
\label{prop:Adjoint:Isometry}
$\norm{\T}=\norm{T}$
\item 
\label{prop:Adjoint:DenseRange}
If $Range(T)$ is dense in $Y$, then $\inf\limits_{\norm{x}=1}\norm{Tx} \leq \inf\limits_{\norm{S}_{BL(Y,Z)} = 1} \norm{\T S}$.
%			To Range(T) dense in Y. 
\item 
\label{prop:Adjoint:NotDenseRange}
If Range(T) is not dense in Y, then 
There exists $S \in BL(Y,Z)$ with $\norm{S} = 1$ and $\norm{\T S} = 0$. 
$\inf\limits_{\norm{S}_{BL(Y,Z)}=1} \norm{\T S} =0$
\item 
\label{prop:Adjoint:Surjective}
$\T$ is \Surjective if and only if T is \Injective and has \SetClosed range in Y. 
\end{enumerate}


\begin{proof}[Proof of \ref{prop:Adjoint:Linear}]
Let $S,R \in BL(Y,Z)$, 
$\alpha \in \F$, 
and $x \in X$. 
Then, 
\begin{align*}
\ip{x, \T(\alpha S+R)} & = \ip{Tx, \alpha S+R}\\
& = \alpha \ip{Tx, S}+ \ip{Tx, R} \\
& = \alpha \ip{x, \T S} + \ip{ x, \T R}\\
& = \ip{x, \alpha \T S} + \ip{x, \T R} \\
& = \ip{x, \alpha \T S + \T R}
\end{align*}
Since $x \in X$ was arbitrary, \Linearity is verified. 
\end{proof}
\begin{proof}[Proof of \ref{prop:Adjoint:WellDefined}]
Let $S \in BL(Y,Z)$. 
Then, 
$\T S = S \circ T$. 
The composition of \ContinuousFunction operators is \ContinuousFunction, so $\T S$ is 
\ContinuousFunction.
The composition of \Linear operators is \Linear, so $\T S$ is \Linear.
Hence, $\T S \in BL(X,Z)$.
\end{proof}
\begin{proof}[Proof of \ref{prop:Adjoint:Continuous}]
Let $S \in BL(Y,Z)$. Then, 
if $x \in X$
\begin{equation*}
\norm{\ip{x, \T S}} = \norm{\ip{Tx, S}} \leq \norm{S} \norm{Tx} \leq \norm{S} \norm{T} \norm{x}
\end{equation*}
Hence $\norm{\T S} \leq \norm{S} \norm{T}$
Since T is \Linear, and since S was arbitrary, 
by part  12 of \ref{prop:BLO}, $\T \in BL\pa{ BL(Y,Z), BL(X,Z)}$.
\end{proof}
\begin{proof}[Proof of \ref{prop:Adjoint:Isometry}]
For any $S \in BL(Y,Z)$, 
$\T S = S \circ T$, so
$\norm{\T S} \leq \norm{S} \norm{T}$. 
Hence $\norm{\T} \leq \norm{T}$. 
Now let $x_0 \in X$. 
Then, by part 4 of 
\ref{thm:hahnbanach}, 
there exists $S \in BL(Y,Z)$ with 
$\norm{S}=1$
and $\norm{STx_0} = \norm{Tx_0}$. 
Hence, 
\begin{align*}
\norm{T x_0} & = \norm{S Tx_0} \\
& = \norm{(S \circ T) x_0} \\
& = \norm{(\T S) x_0} \\
& \leq \norm{\T} \norm{S} \norm{x_0}\\
& = \norm{\T} \norm{x_0}
\end{align*}
Since $x_0 \in X$ is arbitrary, $\norm{T} \leq \norm{\T}$. 
Since the inequality goes both ways, $\norm{T}=\norm{\T}$.
\end{proof}
\begin{proof}[Proof of \ref{prop:Adjoint:DenseRange}]
Let $\Gamma=\inf\limits_{\norm{x}=1} \norm{Tx}$, 
and let $S \in BL(Y,Z)$ with $\norm{S} = 1$. 
Then, 
\begin{equation*}
\{x | \norm{Tx} \leq \Gamma\} \subset B_X(0;1)
\end{equation*}
so 
\begin{equation*}
\sup\limits_{\norm{x}\leq 1} \abs{\ip{Tx, S}} \geq \sup\limits_{\norm{Tx} \leq \Gamma}\abs{\ip{Tx, S}}
\end{equation*}
Also, since $Range(T)$ is dense in $Y$ and $S$ is \ContinuousFunction,
\begin{equation*}
\sup\limits_{\norm{Tx} \leq \Gamma} \abs{\ip{Tx, S}} = \sup\limits_{\norm{y} \leq \Gamma} \abs{\ip{y, S}}
\end{equation*}
From these two we arrive at the inequality
\begin{align*}
\norm{\T S} & = \sup\limits_{\norm{x}  \leq 1} \abs{\ip{x, \T S}}\\
& = \sup\limits_{\norm{x} \leq 1} \abs{\ip{Tx, S}}\\
& \geq \sup\limits_{\norm{Tx} \leq \Gamma} \abs{\ip{Tx, S}}\\
& = \sup\limits_{\norm{y} \leq \Gamma} \abs{\ip{y, S}}\\
& = \Gamma\\
& \inf\limits_{\norm{x} = 1} \norm{Tx}
\end{align*}
Since $S \in \partial B_{BL(Y,Z)}(0;1)$ was arbitrary, we conclude
$\inf\limits_{\norm{S} = 1} \norm{\T S} \geq \inf\limits_{\norm{x} = 1} \norm{Tx}$
\end{proof}
\begin{proof}[Proof of \ref{prop:Adjoint:NotDenseRange}]
Suppose $Range(T)$ is not dense in $Y$. 
Then there exists $y_0 \in Y \setminus \overline{Range(T)}$. 
By 
\ref{thm:HahnBanach:NullspaceOperator}, 
there exists $S_0 \in BL(Y,Z)$ with $S_0\pa{\overline{Range(T)}} = 0$ and 
$\norm{S_0(y_0)} = 1$. 
Set $S= \frac{S_0}{\norm{S_0}}$. 
Then $\norm{S} = 1$. 
Furthermore, 
\begin{align*}
\norm{\T S} & = \sup\limits_{\norm{x} \leq 1} \abs{\ip{x, \T S}}\\
& \leq \sup\limits_{y \in Range(T)} \abs{\ip{y, S}}\\
& = 0
\end{align*}
\end{proof}
\begin{proof}[Proof of \ref{prop:Adjoint:Surjective}]
Let $\T$ be \Surjective and let $x_0 \in X$ with $Tx_0 = 0$. 
Let $\tilde{S} \in BL(X,Z)$. 
Then there exists $S \in BL(Y,Z)$ with $\tilde{S} = \T S$. 
For this $S$, 
\begin{equation*}
\ip{x, \tilde{S}} = \ip{Tx, S} = 0
\end{equation*}
Hence $\tilde{S}x = 0$ for every \Linear \ContinuousFunction $\tilde{S}$. 
This implies $x=0$, so $T$ is \Injective. 
Now suppose 
\end{proof}
\end{prop}

\label{def:higherorderdualspaces}
\newcommand{\SecondTopDualSpace}[0]{ 
    \bf \hyperref[def:higherorderdualspaces]{$2^{nd}$ Topological Dual Space} \rm 
}
\newcommand{\ThirdTopDualSpace}[0]{ 
    \bf \hyperref[def:higherorderdualspaces]{$3^{rd}$ Topological Dual Space} \rm 
}
\newcommand{\NthTopDualSpace}[1]{
    \bf \hyperref[def:higherorderdualspaces]{$\pa{#1}^{th}$ Topological Dual Space} \rm
}
\begin{df}[Higher Order Dual Spaces]
    Let X be a 
   \SeminormedSpace. 
    From 
    \ref{def:dualspace}
    we know that the 
    \TopDualSpace
    of X, 
    $X^*$, 
    is also called the 
    \FirstTopDualSpace
    of X. 
    Building on this, 
    for $n \in \{2, 3, 4, ..., \}$
    we call the 
    \FirstTopDualSpace
    of $X^*$ the 
    \SecondTopDualSpace
    of X, 
    we call the 
    \FirstTopDualSpace
    of the 
    \SecondTopDualSpace
    of X the 
    \ThirdTopDualSpace
    of X, and 
    in general the 
    \FirstTopDualSpace
    of the 
    \NthTopDualSpace{n}
    of X
    the 
    \NthTopDualSpace{n+1}
    of X. 
\end{df}

\newcommand{\NthDualSPaceIso}[1]{\textbf{\hyperref[def:higherorderdualspaceisomorphism]{\ensuremath{\pa{#1}^{th}} Dual Space Isomorphism}}\xspace}
\begin{df}[Higher Order Dual Space Isomorphism]
\label{def:higherorderdualspaceisomorphism}
\rm
Let $X$ be a 
\SeminormedSpace
over a \Field
$\F$. 
Let $\Omega:X^* \to (X/\Ker_X)^*$ 
be the 
\Linear 
\Bijective 
\Isometry
defined in 
\ref{thm:dualspaceisomorphism}.
Define 
$\Omega_1=\Omega$.
Also define, for $2 \leq n \in \N$, 
$\Omega_n:X^{n*} \to \pa{X/\Ker_X}^{n*}$
by 
$\Omega_n=\pa{\Omega_{n-1}^{\times}}^{-1}$.
By     
\ref{prop:adjointoperator}
it is clear that the 
adjoint of a \Linear \Bijective \Isometry of \NormedSpaces
is also a \Linear \Bijective \Isometry of \NormedSpaces, and
so each $\Omega_n$ is as well. 

\end{df}

\label{def:canonicalembedding}
\newcommand{\CanonicalEmbedding}[0]{
    \bf \hyperref[def:canonicalembedding]{Canonical Embedding} \rm
}
\newcommand{\Reflexive}[0]{
    \bf \hyperref[def:canonicalembedding]{Reflexive} \rm
}
\begin{df}[\CanonicalEmbedding of X into $X^{**}$]
    Let X be a \SeminormedSpace. 
    Define $c_X:X \to X^{**}$ by 
    setting, for each $x^* \in X^*$, 
    for each $x \in X$
    \begin{equation}
        \ip{x^*, c(x)} = \ip{x, x^*}
    \end{equation}
    We call $c_X$ the
    \CanonicalEmbedding
    of X into $X^*$. 
    As normal, if X is understood, 
    we may denote $c_X=c$.
    If c is Surjective, then we 
    say that X is \Reflexive. 

\end{df}


\begin{prop}[Canonical Embedding]
\label{prop:canonicalembedding}
    Let $X$ be a \SeminormedSpace 
    and let $c$ denote its 
    \CanonicalEmbedding. 
    The following are true. 
    \begin{enumerate}
        \item c is well defined
        \item c is Linear. 
        \item c is an isometry. 
        \item c is an injection if and only if X is a \NormedSpace. 
        \item Using the abuse of notation described in 
        \ref{rmk:doubledualnotation}, 
        if $q:X \to X/\Ker$ is the \QuotientMap, 
        then $c_X=c_{X/\Ker} \circ q$. 
        \item $c_X$ is surjective if and only if 
        $c_{X/\Ker}$ is surjective. 
        \item X is \Reflexive if and only if $X/\Ker$ is \Reflexive.
    \end{enumerate}
    \begin{proof}[Proof of 1]
        I just need to show that for any 
        $x \in X$, $c(x)$ is
        conintuous and
        linear. 
        For continuity,
    \end{proof}
    \begin{proof}[Proof of 2]
        Let $\alpha \in \F$ and $x,y \in X$.
        Let $x^* \in X$. 
    \end{proof}
    \begin{proof}[Proof of 3]
    \end{proof}
    \begin{proof}[Proof of 4]
    \end{proof}
    \begin{proof}[Proof of 5]
    \end{proof}

\end{prop}


\label{def:SeminormWeakTopology}
\newcommand{\SeminormWeakTopology}[0]{
    \bf \hyperref[def:SeminormWeakTopology]{Seminorm Weak Topology} \rm
}
\newcommand{\NormWeakTopology}[0]{
    \bf \hyperref[def:SeminormWeakTopology]{Norm Weak Topology} \rm
}
\newcommand{\SeminormWeakStarTopology}[0]{
    \bf \hyperref[def:SeminormWeakTopology]{Seminorm Weak-* Topology} \rm
}
\newcommand{\NormWeakStarTopology}[0]{
    \bf \hyperref[def:SeminormWeakTopology]{Norm Weak-* Topology} \rm
}
\newcommand{\weak}[0]{
    \bf \hyperref[def:SeminormWeakTopology]{$\mathfrak{weak}$}\rm
}

\newcommand{\weakstar}[0]{
    \bf \hyperref[def:SeminormWeakTopology]{$\mathfrak{weak}^*$}\rm
}

\begin{df}[Weak Topologies Relating To Seminormed and Normed Spaces]
    %Let $(X,\norm{\dot})$ be a \SemiNormed space. 
    latex 

    \weak
    
    \weakstar


    


\end{df}                                

 Similar to in the context of a normed space, if X is a seminormed space, we define the weak topology on X to be the topology on X generated by $X^*$, and the $weak^*$ topology on $X^*$ to be the topology generated by $c(X)$.
Before moving on to the classical theory revamped, I present on more useful result about weak topologies of seminormed spaces. 
\begin{prop}[Weak Quotients]
    \label{prop:weakquotients}
    Let X be a seminormed space and $\{Y_\alpha\}_{\alpha \in A}$ be a collection of topological spaces. For each $\alpha \in A$ let $\phi_\alpha:X \to Y_\alpha$ have the property that for every $x,y \in X$, for every $\alpha \in A$, $\norm{x-y}=0 \implies \phi_\alpha(x)=\phi_\alpha(y)$. 
    For each $\alpha \in A$, define $\tilde{\phi}_\alpha:X/\norm{\cdot}^{-1}\{0\} \to Y_\alpha$ by
    $\tilde{\phi}_{\alpha}[x] = \phi_\alpha x$. Let $\T_w$ denote the weak topology on X induced by $\{\phi_\alpha\}_{\alpha \in A}$, and $\T_{\tilde{w}}$ denote the weak topology on $X/\norm{\cdot}^{-1}\{0\}$ induced by $\{\tilde{\phi}_{\alpha}\}_{\alpha \in A}$. Then 
    \begin{equation}
        (X,\T_w)/\norm{\cdot}^{-1}\{0\} = (X/\norm{\cdot}^{-1}\{0\}, \T_{\tilde{w}})
    \end{equation}
    \begin{proof}
    \end{proof} 
\end{prop} 


Finally, before we move on, recall that if $X,Y$ are Topological vector spaces, we can topologize the set of continuous linear operators from X to Y, denoted $BL(X,Y)$ by saying that $\{T_\alpha\}_{\alpha \in A} \subset BL(X,Y)$ converges to $T \in BL(X,Y)$ if there is a neighborhood U of 0 in X such that $T_{\alpha}x \to Tx$ uniformly for $x \in U$. 