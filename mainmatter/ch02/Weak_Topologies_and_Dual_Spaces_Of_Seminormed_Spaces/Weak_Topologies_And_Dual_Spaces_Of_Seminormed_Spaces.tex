\section{Weak Topologies And Dual Spaces Of Seminormed Spaces}
\subsection{Quotient Maps}
\label{def:TopologicalSpace}
\newcommand{\TopologicalSpaceRef}[0]{
    \bf \hyperref[def:TopologicalSpace]{Topological Space} \rm
}
\newcommand{\TopologyRef}[0]{
        \bf \hyperref[def:TopologicalSpace]{Topology} \rm
}
%A topology named #2 on the space #1 is given by \Topology{#1}{#2}
\newcommand{\Topology}[2]{
    #2_{#1}
}
%A Topological space (#1, #2_#1) is given by \TopologicalSpace{#1}{#2}
\newcommand{\TopologicalSpace}[2]{
    \pa{#1, \Topology{#1}{#2}}
}
\newcommand{\LetBeTopologicalSpace}[2]{
    Let $\TopologicalSpace{#1}{#2}$ be a \TopologicalSpaceRef.
}
\begin{df}[Topological Space]
$\TopologicalSpace{Z}{\T}$
\end{df}

\label{def:NeighborhoodFilter}   
\newcommand{\NeighborhoodFilter}[0]{
    \bf \hyperref[def:NeighborhoodFilter]{Neighborhood Filter} \rm
}
\newcommand{\NeighborhoodFilters}[0]{
    \bf \hyperref[def:NeighborhoodFilter]{Neighborhood Filters} \rm
}
\newcommand{\NbhFilter}[2]{
    \scU_{#1}\pa{#2}
}
\begin{df}[Neighborhood Filter]
    \LetBeTopologicalSpace{Z}{\T}
    Let $x \in Z$. 
    We denote with $\NbhFilter{\Topology{Z}{\T}}{x}$ the collection of all elements of $\T_Z$ containing x. 
    We call $\NbhFilter{\Topology{Z}{\T}}{x}$ a \NeighborhoodFilter  of $\T_Z$ at x. 
\end{df}



\label{def:RelationOfEqualNeighborhoodFilters}
\newcommand{\RelationOfEqualNeighborhoodFilters}[1]{
    \bf \hyperref[def:RelationOfEqualNeighborhoodFilters]{Relation Of Equal Neighborhood Filters} \rm on #1
}
\begin{df}[relation of equal neighborhood filters]
    
    Let $(Z, \T_Z)$ be a topological space. Define the relation $\cong$ on Z by setting, for $x,y \in Z$, 
    \begin{equation}
        x \cong y \iff \scU_{\T_Z}(x)=\scU_{\T_Z}(y)
    \end{equation}
    We call $\cong$ the \RelationOfEqualNeighborhoodFilters{$(Z,\T_Z)$}
\end{df} 


\begin{prop}[\RelationOfEqualNeighborhoodFilters]
    \label{prop:EqualNeighborhoodFiltersEquivalenceRelation}
    
    The
	\RelationOfEqualNeighborhoodFilters
	$\cong$ on a \TopologicalSpaceRef $(Z,\T_Z)$ forms an 
	\EquivalenceRelation	
	on Z. 
    \begin{proof}
        
        Let $x \in (Z,\T_Z)$. 
        Then $\NbhFilter{\Topology{Z}{\T}}{x}$=$\NbhFilter{\Topology{Z}{\T}}{x}$, so $x \cong x$.
        Thus $\cong$ is 
		\ReflexiveRelation. 
        
        Let $x,y \in (Z,\T_Z)$. 
        Suppose $x \cong y$. 
        Then  $\NbhFilter{\Topology{Z}{\T}}{x} = \NbhFilter{\Topology{Z}{\T}}{y}$
        , so trivially  $\NbhFilter{\Topology{Z}{\T}}{y} =\NbhFilter{\Topology{Z}{\T}}{x}$
        , and thus $y \cong x$.
        Hence, $\cong$ is 
		\SymmetricRelation
        
        Let $x,y,z \in (Z,\T_Z)$.
        Let $x \cong y$ and $y \cong z$. 
        Then, 
         $\NbhFilter{\Topology{Z}{\T}}{x}= \NbhFilter{\Topology{Z}{\T}}{y} =  \NbhFilter{\Topology{Z}{\T}}{z}$
         so that $x \cong z$.
         Thus $\cong$ is \TransitiveRelation
         
         Since $\cong$ is 
		 \ReflexiveRelation
		, \SymmetricRelation
		, and \TransitiveRelation, it is an 
		\EquivalenceRelation. 
        
    \end{proof}
\end{prop}

\label{def:EquivalenceClass}
\newcommand{\EquivalenceClass}[0]{\textbf{\hyperref[def:EquivalenceClass]{Equivalence Class}}\xspace}
\newcommand{\EqClass}[2]{\bra{#1}_{\cong}\xspace}
\begin{df}[Equivalence Class]
    
    Let $X \neq \emptyset$.
    Let $\cong$ be an 
	\EquivalenceRelation
	defined on X.  
    Let $x \in X$. 
    We define the set $[x]_{\cong}$ by 
    \begin{equation}
        [x]_{\cong} = \{y \in X | y \cong x\}
    \end{equation} 
    We call $\EqClass{x}{\cong}$ the \EquivalenceClass of x in $(X, \cong)$. 
\end{df}


\begin{prop}[Equivalence Classes Partition]
    \label{prop:EquivalenceClassesPartition}
    
    Let $X \neq \emptyset$. 
    Let $\cong$ be an equivalence relation defined on X. 
    Let $x,y \in X$. 
    The following are true.
    \begin{equation}
        [x]_{\cong}  \cap [y]_{\cong} \neq \emptyset \iff [x]_{\cong} = [y]_{\cong}  \iff x \cong y \iff [x]_{\cong} \subset [y]_{\cong} \iff [y]_{\cong} \subset [x]_{\cong} 
    \end{equation}
    \begin{equation}
        x \in [x]_{\cong} 
    \end{equation}
    \begin{proof} OBVIOUS \end{proof}
\end{prop}  

\label{def:QuotientSet}
\newcommand{\QuotientSet}[0]{
    \bf \hyperref[def:QuotientSet]{Quotient Set} \rm
}
\newcommand{\QuoSet}[2]{
    #1
    /
    #2
}
\newcommand{\LetBeQuotientSet}[2]{
    Let \QuoSet{#1}{#2} be the \QuotientSet of #1 with respect to the relation #2.
}
\begin{df}[Quotient Set]  
    Let $X \neq \emptyset$.
    Let $\cong$ be an 
	\EquivalenceRelation defined on X.
    We define the set $X/\cong$ by 
    \begin{equation}
        \QuoSet{X}{\cong} = \{ [x]_{\cong} : x \in X\}
    \end{equation}
    We call $\QuoSet{X}{\cong}$ the \QuotientSet of X under the relation $\cong$. 
\end{df} 

\begin{rmk}[Quotient Set Partition]
    \label{rmk:quotientsetpartition}
    \ref{prop:EquivalenceClassesPartition}
	, paired with the fact that 
	$x \in [x]_{\cong}$, 
	implies that 
	$X/\cong$ is a partition of X. 
\end{rmk} 

\label{df:quotient_map}
\newcommand{\QuotientMap}[0]{
        \bf \hyperref[df:quotient_map]{Quotient Map} \rm
    }
    
 \newcommand{\QuotientMapInstance}[3]{
     #1 : #2\to #2/#3
}
\begin{df}[Quotient Map]

    Let $X \neq \emptyset$.
    Let $\cong$ be an 
	\EquivalenceRelation 
	on X.
    \LetBeQuotientSet{X}{$\cong$}
    Define $T:X \to X/\cong$ by setting, for each $x \in X$, 
    \begin{equation}
        T(x)=[x]
    \end{equation}    
    We call T the \QuotientMap of X under $\cong$. 
\end{df} 

\begin{prop}[\QuotientMap is  \Surjective]
\label{prop:QuotientMapSurjective}
\rm
    Let $X$ be a nonempty set.
    Let $\cong$ be an 
	\EquivalenceRelation on 
	$X$.
    Let $\QuotientMapInstance{T}{X}{\cong}$  be the 
	\QuotientMap of $X$ under the 
	\Relation
	$\cong$. 
    Then T is a 
	\Surjection. 
    \begin{proof}
       Let $K \in X/\cong$. 
       Then for some $x \in X$, $K=[x]$. 
       Then $T(x) = K$. 
       Since K was arbitrary, we are done. 
    \end{proof}
\end{prop} 


\label{def:QuotientSpaceTopology}
\newcommand{\QuotientSpaceTopology}[0]{
    \bf \hyperref[def:QuotientSpaceTopology]{Quotient Topology} \rm
}
\newcommand{\QuotientTopologicalSpace}[0]{
    \bf \hyperref[def:QuotientSpaceTopology]{Quotient Topological Space} \rm 
}

\begin{df}[Quotient Space Topology]
    Let $(Z,\T_Z)$ be a topological space. 
    Let $\cong$ be the \RelationOfEqualNeighborhoodFilters{$(Z, \T_Z)$}. 
    Let T be the \QuotientMap of Z under the relation $\cong$. 
    Define $\T_{Z/\cong}$ by
    \begin{equation}
        \T_{Z/\cong} = \left\{ \bigcup_{x \in U}\{T(x)\} \in 2^{Z/\cong}| U \in \T_Z \right\}
    \end{equation}
    By \ref{prop:QuotientSpaceTopology}, $\T_{Z/\cong}$ is a topology on $Z/\cong$.
    We call $\T_{Z/\cong}$ the \QuotientSpaceTopology and we call $\pa{Z/\cong, \T_{Z/\cong}}$ the \QuotientTopologicalSpace of $(Z, \T_Z)$.
    
\end{df}


\begin{prop}[Quotient Space Topology]
    \label{prop:QuotientSpaceTopology}
    
    Let $(Z,\T_Z)$ be a 
	\TopologicalSpaceRef
    with \QuotientTopologicalSpace  $\pa{Z/\cong, \T_{Z/\cong}}$
    and \QuotientMap T.
    
    Then the following are true. 
    \begin{enumerate}
        \item $\T_{Z/\cong}$ is a \TopologyRef on $Z/\cong$. 
        \item $T:(Z, \T_Z) \to (Z/\cong, \T_{Z/\cong})$ is \Continuous. 
        \item If U is \SetOpen (\SetClosed) in $(Z,\T_Z)$ then $T(U)$ and $T(Z\setminus U)$ \Partition $Z/\cong$. 
        \item If U is \SetOpen in $(Z, \T_Z)$, then $T^{-1}(T(U))=U$. 
        \item If K is \SetClosed in $(Z,\T_Z)$, then $T^{-1}T(K)=K$. 
        \item $T:(Z, \T_Z) \to (Z/\cong, \T_{Z/\cong})$ is \MapOpen. 
        \item $T:(Z, \T_Z) \to (Z/\cong, \T_{Z/\cong})$ is \MapClosed.
        \item $(Z, \T_Z)$ is a \Compact space if and only if $(Z/\cong, \T_{Z/\cong})$ is a \Compact space.
        \item If $\scB$ is a \TopologyBasis for $\T_z$, then $\{T(U) | U \in \scB\}$ is a \TopologyBasis for $\T_{Z/\cong}$. 
        \item If T is \Injective, then it is a \Homeomorphism. 
    \end{enumerate} 
    \begin{proof}[Proof of 1]
        Since $\emptyset \in \T_Z$, we have 
        \begin{equation}
            \emptyset = \bigcup\limits_{x \in \emptyset} \{Tx\} \in \T_{Z/\cong}
        \end{equation}
        Since $Z \in \T_Z$, and by \ref{rmk:quotientsetpartition}, 
        \begin{equation} 
            Z/\cong = \bigcup_{x \in Z} \{[x]\}= \bigcup\limits_{x \in Z} \{T(x)\} \in \T_{Z/\cong}
        \end{equation} 
        
        Let $\{U_{\alpha} | \alpha \in A\} \subset \T_{Z/\cong}$. 
        For each $\alpha \in A$, there exists $B_{\alpha} \in \T_{Z}$ such that we have
        \begin{equation} 
            U_{\alpha } = \bigcup_{x \in B_{\alpha}} \{Tx\}
        \end{equation} 
        Since $\bigcup_{\alpha \in A} B_\alpha \in \T_{Z}$, we have 
        \begin{equation}
            \bigcup_{\alpha \in A} U_{\alpha}= \bigcup\limits_{\alpha \in A} \bigcup\limits_{x \in U_\alpha} \{T(x)\} = \bigcup\limits_{x \in \bigcup\limits_{\alpha \in A} B_{\alpha}} \{T(x)\} \in \T_{Z/\cong}
        \end{equation} 
        Let $\{U_i\}_{i=1}^n \subset \T_{Z/\cong}$. 
        For each $i \in \{1, ..., n\}$, there exists $B_i \in \T_{Z}$ such that
        \begin{equation}
            U_i = \bigcup_{x \in B_{i}} \{T(x)\}
        \end{equation}
        Suppose 
        \begin{equation}
            [x_0] \in \bigcap\limits_{i=1}^n \bigcup\limits_{x \in B_i} \{T(x)\}
        \end{equation}
        Then for each $i \in \{1,..., n\}$, there is a $y_i \in B_i$ such that $ y_i \cong x_0$. 
        Since each $B_i$ is \SetOpen, the definition of $\cong$ implies that $x_0 \in B_i$ for every i. Hence, 
        \begin{equation} 
            x_0 \in \bigcap_{i=1}^n B_i
        \end{equation} 
        Implying 
        \begin{equation}
            [x_0] \in  \bigcup\limits_{x \in \bigcap\limits_{i=1}^n B_i} \{[x]\}
        \end{equation} 
        Hence, 
        \begin{equation} 
            \bigcap\limits_{i=1}^n \bigcup\limits_{x \in B_i} \{T(x)\}
            \subset
            \bigcup\limits_{x \in \bigcap\limits_{i=1}^n B_i} \{[x]\}
        \end{equation} 
        Furthermore, since the reverse inclusion is obvious, 
        and since $\bigcap_{i=1}^n B_i \in \T_{Z}$, we have 
        \begin{equation}
            \bigcap_{i=1}^n U_i = \bigcap_{i=1}^n \bigcup_{x \in B_i} \{T(x)\}= \bigcup\limits_{x \in \bigcap\limits_{i=1}^n B_i} \{T(x)\} \in \T_{Z/\cong}
        \end{equation}
    \end{proof}
    \begin{proof}[Proof of 2]
        Let $V \in \T_{Z/\cong}$. 
        Let $x_0 \in T^{-1}(V)$. 
        Then $[x_0] \in V$. 
        By definition, there is a $U \in \T_Z$ such that 
        \begin{equation}
            T(U) \subset \bigcup\limits_{x \in U} \{T(x)\}=V
        \end{equation}
        Hence there is a $y_0 \in U$  such that 
        \begin{equation}
            [x_0] \in T(y_0) = \{[y_0]\}
        \end{equation}
        Therefore, $x \cong y$. 
        Definition of the 
		\RelationOfEqualNeighborhoodFilters 
		implies $\scU(x_0)=\scU(y_0)$. 
        Hence, $x_0 \in U \subset T^{-1}(V)$.
    \end{proof}
    \begin{proof}[Proof of 3]
        Let $K$ be closed in $(Z,\T_Z)$. 
        Then each point $x_0$ in $Z\setminus K$ has some $U_{x_0} \in \scU_{\T_Z}(x_0)$ which is 
		\Disjoint 
		from K.
        Hence $y_0 \not \cong x_0$ for any $y_0 \in K$, $x_0 \in Z\setminus K$. 
        Hence $T(K)$ is 
		\Disjoint 
		from $T\pa{Z \setminus K}$. 
        This fact, paired with \ref{prop:QuotientMapSurjective}, implies $T(Z\setminus K)$ and T(K) 
		is a \Partition of $Z/\cong$.
    \end{proof}
    \begin{proof}[Proof of 4]
        Let $U \in \T_Z$. 
        The nontrivial direction to prove is $T^{-1}\pa{T(U)} \subset U$.
        Let $y \in T^{-1}\pa{T(U)}$. 
        Then $[y]=Ty \in T(U)$.
        Hence, $[y]=T(x)=[x]$ for some $x \in U$. 
        Since $y \cong x$ and $x \in U \in \scU_{\T_Z}(x)$, we have $U \in \scU_{\T_Z}(y)$. 
        Hence $y \in U$.
        Since y was arbitrary, $T^{-1}\pa{T(U)} \subset U$, and equality is obvious because the other direction of inclusion is trivial. 
    \end{proof}
    \begin{proof}[Proof of 5]
        Let K be 
		\SetClosed 
		in $(Z,\T_Z)$. Part 3 Of this result implies $Z/\cong$ is partitioned by $T(K)$ and $T(Z\setminus K)$. 
        
        By part 4 of this proposition, 
        \begin{align*}
            T^{-1}\pa{T(K)}&=T^{-1} \pa{T(Z) \setminus T(Z \setminus K)} \\
            &= T^{-1}\pa{Z/\cong \setminus T(Z \setminus K)}\\
            &=T^{-1}(Z/\cong) \setminus T^{-1}(T(Z\setminus K)) \\
            &= Z \setminus \pa{Z \setminus K} \\
            &= K
        \end{align*}      
    \end{proof}
    \begin{proof}[Proof of 6]
        Let $U \in \T_Z$.
        Then by definition of the \QuotientSpaceTopology
        \begin{equation}
            TU= \bigcup_{x \in U} \{T(x)\}  \in \T_{Z/\cong}
        \end{equation}
    \end{proof}  
    \begin{proof}[Proof of 7] 
        Let K be \SetClosed in $(Z,\T_Z)$. 
        Then $Z \setminus K \in \T_Z$. 
        By Parts 3 and five of this proposition, we know $T(K) = Z/\cong \setminus T(Z\setminus K)$ and also that $T(Z\setminus K) \in \T_{Z/\cong}$. Hence $T(K)$ is closed in $(Z/\cong, \T_{Z/\cong})$. 
    \end{proof} 
    \begin{proof}[Proof of 8]
        Let $(Z,\T_Z)$ be \SetCompact. 
        Let $\{U_{\alpha}\}_{\alpha \in A}$ be an open covering of $(Z/\cong, \T_{Z/\cong})$. 
        Then $\{T^{-1}\pa{U_{\alpha}} | \alpha \in A\}$ is an open covering of $(Z, \T_Z)$. 
        \SetCompactness of $(Z, \T_Z)$ guarantees the existence of a finite subcovering $\{T^{-1}\pa{U_{\alpha_i}} | i \in \{1, ..., n\}\}$. 
        Hence
        $\{U_{\alpha_i} | i \in \{1, ..., n\}\}=\{TT^{-1}(U_{\alpha_i}) | i \in \{1, ..., n\}\}$ is an 
		\OpenCover of $(Z/\cong, \T_{Z/\cong})$. 
         And the \SetCompactness of $(Z/\cong, \T_{Z/\cong})$ is verified. 
         
         
         Now, suppose $(Z/\cong, \T_{Z/\cong})$ is \SetCompact. 
         Let $\{V_{\beta} | \beta \in B\}$ be an \OpenCover of $(Z, \T_Z)$. 
         Since T is an \MapOpen mapping, $\{T(V_{\beta}) | \beta \in B\}$ is an 
		 \OpenCover of $(Z/\cong, \T_{Z/\cong})$ which by 
		 \SetCompactness has a \Finite \SubCover $\{T(V_{\beta_i}) | i \in \{1, ..., n\}\}$. 
         By part 4 of \ref{prop:QuotientSpaceTopology}, 
         $\{V_{\beta_i}| i \in \{1, ..., n\}\} = \{T^{-1}(T(V_{\beta_i})) |i \in \{1, ..., n\}\}$ is then an \OpenSubCover of $(Z, \T_Z)$. 
     %    
    \end{proof}
    \begin{proof}[Proof of 9]
        Let $\scB$ be a basis for $\T_z$ and let $V \in \T_{Z/\cong}$. 
        Then $T^{-1}(Z) \in \T_Z$, and so there is a subcollection $\{U_{\alpha}\}_{\alpha \in A} \subset \scB$ such that $T^{-1}(V) = \bigcup_{\alpha \in A} U_{\alpha}$. 
        Hence, 
        \begin{align*}
            V& =T(T^{-1}(V))\\
            & = T\pa{\bigcup_{\alpha \in A} U_{\alpha}}\\
            & = \bigcup_{\alpha \in A} T(U_{\alpha})
        \end{align*}
     \end{proof} 
     \begin{proof} [Proof of 10]
            If T is \Injective
			, then since it is \Continuous Part 2 of this result, open by part 6 of this result, and \Surjective by \ref{prop:QuotientMapSurjective}, it is a 
			\Bicontinuous 
			\Bijection, that is, a \Homeomorphism. 
         \end{proof}
\end{prop} 

\subsection{Pseudometrics}

\label{def:Symmetricmap}
\newcommand{\SymmetricMap}[0]{
    \bf \hyperref[def:Symmetricmap]{Symmetric Map} \rm
}
\newcommand{\CommutativeFunction}[0]{
    \bf \hyperref[def:Symmetricmap]{Commutative} \rm
}
\newcommand{\FunctionCommutativity}[0]{
    \bf \hyperref[def:Symmetricmap]{Function Commutativity} \rm
}
\begin{df}[Triangle Inequality]
    Let X and Y be sets. 
    We say that a map 
    $f:X \times X \to Y$ is a \SymmetricMap 
    if for each 
    $x_0,x_1 \in X$, 
    $f(x_0,x_1)=f(x_1,x_0)$.
    In this situation, 
    we may also refer to $f$ as
    \CommutativeFunction, 
    or say that $f$ posesses 
    \FunctionCommutativity.
\end{df} 


\label{def:TriangleInequality}
\newcommand{\TriangleInequality}[0]{
    \bf \hyperref[def:TriangleInequality]{Triangle Inequality} \rm
}
\begin{df}[Symmetric Map]
    
    Let X be a set and $(Y,+, \leq)$ be a totally ordered magma.
    We say that a map $f:X \times X \to Y$ satisfies the \TriangleInequality if for each $x_0,x_1,x_3 \in X$, we have
    \begin{equation*}
        f(x_0,x_2) \leq  f(x_0,x_1)+f(x_1,x_2)
        \end{equation*}
\end{df} 

\newcommand{\Pseudometric}[0]{\textbf{\hyperref[def:pseudometric]{Pseudometric}}\xspace}
\newcommand{\Pseudometrics}[0]{\textbf{\hyperref[def:pseudometric]{Pseudometrics}}\xspace}
\newcommand{\PseudometricSpaces}[0]{\textbf{\hyperref[def:pseudometric]{Pseudometric Spaces}}\xspace}
\newcommand{\PseudometricSpace}[0]{\textbf{\hyperref[def:pseudometric]{Pseudometric Space}}\xspace}
\begin{df}[\Pseudometric]
\label{def:pseudometric}
\rm
    Let $X$ be a nonempty set.
    Let $d:X \times X \to [0,\infty)$ be \CommutativeFunction, 
    satisfy the \TriangleInequality, and for each $x \in X$, 
    \begin{equation*}
        d(x,x) = 0
    \end{equation*}
    Then we call d a \Pseudometric on X and we call $\pa{X,d}$ a \PseudometricSpace.
    \end{df} 
	
	
	
\newcommand{\Metric}[0]{\textbf{\hyperref[def:metric]{Metric}}\xspace}
\newcommand{\MetricSpace}[0]{\textbf{\hyperref[def:metric]{Metric Space}}\xspace}
\begin{df}[\Metric]
\label{def:metric}
\rm
	Let $(X,d)$ be a \PseudometricSpace. 
	If d has the property that for
	$x,y \in X$, if $x \neq y$, then
	\begin{equation*}
		d(x,y) \neq 0
	\end{equation*}
	Then we call d a \Metric on X 
	and we call $(X,d)$ a
	\MetricSpace
\end{df}


\newcommand{\PseudometricCauchySequence}[0]{\textbf{\hyperref[def:pseudometriccauchysequence]{Pseudometric Cauchy Sequence}}\xspace}
\begin{df}[Pseudometric Cauchy Sequence]
\label{def:pseudometriccauchysequence}
\rm
    Let $(X,d)$ be a \PseudometricSpace.
    We say that a sequence 
	$\{x_i\}_{i \in \N}$ is a 
	\PseudometricCauchySequence
    if, for each 
	$\epsilon > 0$, 
	there exists
	$N \in \N$
	such that for 
    each pair 
	$m,n \in \N$ 
	such that 
	$m>N$ 
	and 
	$n>N$, we have 
    \begin{equation*}
        d(x_m,x_n) < \epsilon
    \end{equation*}
\end{df}



\label{def:pseudometricsequenceconvergence}
\newcommand{\PseudometricConvergence}[0]{
    \bf \hyperref[def:pseudometricsequenceconvergence]{Pseudometric-Convergence} \rm
}
\newcommand{\PseudometricConvergent}[0]{
    \bf \hyperref[def:pseudometricsequenceconvergence]{Pseudometrically-Convergent} \rm
}
\newcommand{\PseudometricConverges}[0]{
    \bf \hyperref[def:pseudometricsequenceconvergence]{Pseudometric-Converges} \rm
}
\begin{df}[Pseudometric Convergence]
    Let $(X,d)$ be a \PseudometricSpace.
	Let $\{x_i\}_{i \in \N}$ be a sequence in $(X,d)$.
    Let $x_0 \in X$.  
    We say that 
	$\{x_i\}_{i \in \N}$ 
	exhibits 
	\PseudometricConvergence 
	to 
	$x_0$ 
	in d,
	or we say that 
	$\{x_i\}_{i \in \N}$  
	\PseudometricConverges 
	to 
	$x_0$ 
	in d, 
	or we say that 
	$\{x_i\}_{i \in \N}$ 
	is 
	\PseudometricConvergent 
	to 
	$x_0 \in d$ 
	if, 
    for every 
	$\epsilon > 0$, 
	there is an 
	$N \in \N$ 
	such that for every 
	$n>N$, 
	we have 
    \begin{equation}
        d(x_0, x_n) < \epsilon
    \end{equation}
\end{df}

\label{def:pseudometriccomplete}
\newcommand{\PseudometricComplete}[0]{
    \bf \hyperref[def:pseudometriccomplete]{Pseudometric-Complete} \rm
}
\begin{df}[Pseudometric Complete]
    We say that a \PseudometricSpace $(X,d)$ is 
    \PseudometricComplete if each \PseudometricCauchySequence sequence in $(X,d)$ \PseudometricConverges to a limit in $X$. 
    \end{df}

\label{def:pseudometricball}
\newcommand{\OpenBall}[0]{
    \bf \hyperref[def:pseudometricball]{Open Ball} \rm
}
\newcommand{\ClosedBall}[0]{
    \bf \hyperref[def:pseudometricball]{Closed Ball} \rm
}
\begin{df}[Pseudometric Ball]
    Let $(X,d)$ be a \PseudometricSpace. 
    For each $x_0  \in X$ and each $\epsilon > 0$, we define the following.
    \begin{enumerate}
        \item  $B_d(x_0, \epsilon) := \{y \in X | d(x_0,y) < \epsilon\}$ denotes the \OpenBall about $x_0$ with radius $\epsilon$. 
    \item $\overline{B_d}(x_0,\epsilon) := \{y \in X | d(x_0,y) \leq \epsilon \}$ denotes the \ClosedBall about $x_0$ with radius $\epsilon$. 
    \end{enumerate} 
    
     
    \end{df} 

\newcommand{\PseudometricTopology}[0]{\textbf{\hyperref[def:pseudometrictopology]{Pseudometric Topology}}\xspace}
\newcommand{\PseudometricInducedTopology}[0]{\textbf{\hyperref[def:pseudometrictopology]{Pseudometric Topology}}\xspace}

\begin{df}[\PseudometricTopology]
\label{def:pseudometrictopology}
\rm
    Let $(X,d)$ be a \PseudometricSpace, and let $\scB$ be the set of \OpenBall's in $(X,d)$. 
    By \ref{prop:pseudometrictopology}, $\scB$, with the addition
	of $\emptyset$, is the \TopologyBasis for a unique \Topology $\T_d$ on $X$. 
    We call $\T_d$ the \PseudometricInducedTopology induced by $d$ on $X$. 
\end{df}


\label{prop:pseudometrictopology}
\begin{prop}[Pseudometric Topology]
    Let $(X,d)$ by  \PseudometricSpace and let $\scB$ be the set of \OpenBall's in $(X,d)$. 
    The following are true. 
    \begin{enumerate}
        \item There exists a unique topology $\T_d$ on X which $\scB$ is a basis of. That is, the \PseudometricTopology $\T_d$ is well defined. 
        \item The \PseudometricInducedTopology is first countable. That is, each of its points permits a countable neighborhood basis. 
    \end{enumerate}
    \begin{proof}[Proof of 1]
        Uniqueness is guaranteed by closure under arbitrary unions of a topology. 
        For existense, it is sufficient to show that the collection of arbitrary unions
        of elements of $\scB$ is closed under finite intersections. 
        Suppose that for $1\leq i \leq n$, we have $\{U_{\alpha_i} | \alpha_i \in A_i\} \subset \scB$
        and consider the set
        \begin{equation}
            U=\bigcap_{i=1}^n \bigcup_{\alpha_i \in A_i} U_{\alpha_i}
        \end{equation}
        Let $x_0 \in U$. 
        For each $i \in \{1, ..., n\}$, there exists $\alpha_i \in A_i$ such that 
        \begin{equation}
            x_0 \in U_{\alpha_i} = B_d(x_i; \epsilon_i)
        \end{equation}
        For each $i \in \{1, ..., n \}$, define $\delta_i = d(x_0, x_i)$. Then $0 < \delta_i < \epsilon_i$. 
        Then, for each $i \in \{1, ..., n \}$, 
        \begin{equation}
            B_d(x_0; \epsilon_i-\delta_i) \subset U_{\alpha_i} \subset \bigcup_{\alpha_i \in A_i} U_{\alpha_i}
        \end{equation}
        Define 
        \begin{equation}
            \delta_{x_0} = \min\limits_{i=1}^n \pa{ \epsilon_i-\delta_i}
        \end{equation}
        Then $x_0 \in B(x_0; \delta_{x_0} ) \subset U$. 
        If $U=\{x_{\alpha} | \alpha \in A\}$, then the arbitrary nature of $x_0$ above means 
        we can repeat this construction, writing 
        \begin{equation}
            U \subset \bigcup_{\alpha \in A} B(x_{\alpha} ; \delta_{x_{\alpha}} )\subset \bigcup_{\alpha \in A} U = U
        \end{equation}
        Hence, $U \in B$ and the proof is complete. 
    \end{proof}
    \begin{proof}[Proof of 2]
        Let $x_0 \in X$. 
        I claim that 
        \begin{equation}
            \scB_{x_0}:= \left\{ B_d\pa{x_0; \frac{1}{n}} | n \in \N\right\}
        \end{equation}
        is a neighborhood basis for $(X,\T_d)$ at $x_0$. 
        Let $U \in \scU_{\T_d}(x)$ be open in $\T_d$. 
        Since $\scB$ is a basis for $\T_d$, for some $y0 \in X$ and $\epsilon > 0$, 
        $x_0 \in B_d(y_0; \epsilon) \subset U$. 
        Let $\delta = d(x_0, y_0)$. Then $\epsilon - \delta > 0$. 
        Define
        \begin{equation}
            n = \ceil{ \frac{1}{\epsilon - \delta}}
        \end{equation}
        Then we have 
        \begin{equation}
            B_d\pa{x_0 ; \frac{1}{n}} \subset B_d(x_0 : \epsilon - \delta) \subset B(y_0 ; \epsilon) \subset U
        \end{equation}
    \end{proof}
\end{prop}


\label{def:relationofzerodistance}
\newcommand{\RelationOfZeroDistance}[0]{
    \bf \hyperref[def:relationofzerodistance]{Relation Of Zero Distance} \rm
}
\begin{df}[Relation Of Zero Distance]
    Let $(X,d)$ be a \PseudometricSpace. 
    Define the relation  $\cong_d$ on $X \times X$ by setting, for $x,y \in X$, 
    \begin{equation}
        x \cong_d y \iff d(x,y) = 0
    \end{equation}
    We call $\cong_d$ the \RelationOfZeroDistance on $(X,d)$. 
\end{df}

\begin{prop}[Relation Of Zero Distance is the Relation Of Equal Neighborhood Filters]
    \label{prop:relationofzerodistance}
    \rm
    Let $(X,d)$ be a \PseudometricSpace.
    Let $\cong_{\T_d}$ be the \RelationOfEqualNeighborhoodFilters $(X,\T_d)$. 
    Let $\cong_d$ be the \RelationOfZeroDistance on $(X,d)$. 
    Then $\cong_{\T_d} = \cong_d$. 
    \begin{proof}
        Let $x,y \in X$ and suppose $x_0 \cong_d y_0$.
        Let $U \in \scU_{\T_d}(x_0)$. Then for some $\epsilon > 0$, 
        $x_0 \in B(x_0;\epsilon) \subset U$. 
        Since $x_0 \cong_d y_0$, $d(x_0,y_0) = 0$, so $y_0 \in B(x_0 ; \epsilon) \subset U$. 
        Hence $U \in \scU_{\T_d}(y_0)$. 
        The arbitrary nature of $U \in \scU_{\T_d}(x_0)$ implies 
        $\scU_{\T_d}(x_0) \subset \scU_{\T_d}(y_0)$.
        A reverse construction would just as easily show the reverse inclusion, so we conclude that $x_0 \cong_{\T_d} y_0$. 
        Now suppose $x_0 \cong_{\T_d} y $. Then for each $n \in \N$, 
        $y_0 \in B_{d} \pa{x_0 ; \frac{1}{n}}$.
        Hence $d(x_0, y_0) < \frac{1}{n}$ for each $n \in \bbZ^+$, 
        and so $d(x_0,y_0) = 0$ and $x_0 \cong_d y_0$. 
    \end{proof}
\end{prop}



\newcommand{\PseudometricInducedMetric}[0]{\textbf{\hyperref[def:pseudometricinducedmetric]{Pseudometric Induced Metric}}\xspace}
\newcommand{\MetricInducedByPseudometric}[0]{\textbf{\hyperref[def:pseudometricinducedmetric]{Metric Induced By The Pseudometric}}\xspace}
\begin{df}[\MetricInducedByPseudometric]
    \label{def:pseudometricinducedmetric}
    \rm
    Let $(X,d)$ be a \PseudometricSpace, and let $\cong$ be the \RelationOfZeroDistance, which by \ref{prop:relationofzerodistance} is also the \RelationOfEqualNeighborhoodFilters on $(X,\T_d)$. 
    Define $\tilde{d}: X/\cong \to [0,\infty)$ by 
    \begin{equation*}
        \tilde{d}\pa{\bra{x}, \bra{y}} = d(x,y)
    \end{equation*}
    By \ref{prop:pseudometricinducedmetric}, $\tilde{d}$ is well defined and is in fact a \Metric on $X/\cong$, so we call $\tilde{d}$ the \MetricInducedByPseudometric d on X, or we call it the \PseudometricInducedMetric of $(X,d)$. 
\end{df}


\begin{prop}[Metric Space Induced By Pseudometric Space]
    \label{prop:pseudometricinducedmetric}
    %Let $X$, d, $\cong$, and $\tilde{d}$ be defined as in \ref{def:pseudometricinducedmetric}
    Let $(X,d)$ be a \PseudometricSpace, $\cong$ the \RelationOfZeroDistance on $(X,d)$ and $\tilde{d}$ be defined as in \ref{def:pseudometricinducedmetric}.
    Let $(X/\cong, \T_{X/\cong})$ be the  \QuotientTopologicalSpace with \QuotientMap T, and let $(X/\cong, \T_{\tilde{d}})$ be the topological space induced by the metric space $(X/\cong, \tilde{d})$. 
    The following are true. 
    \begin{enumerate}
        \item $\tilde{d}$ is in fact well defined, and is a metric on $X/\cong$, justifying calling it the \MetricInducedByPseudometric d.
        \item $\T_{X/\cong} = \T_{\tilde{d}}$
        \item T is an isometry from $(X,d)$ to $(X/\cong, \tilde{d})$
        \item $(X/\cong, \tilde{d})$ is complete if and only if $(X, d)$ is \PseudometricComplete.
        \item If $T:$
    \end{enumerate}


\end{prop}

\newcommand{\Pseudometrizable}[0]{\textbf{\hyperref[def:Pseudometrizable]{Pseudometrizable}}\xspace}
\newcommand{\Metrizable}[0]{\textbf{\hyperref[def:Pseudometrizable]{Metrizable}}\xspace}
\begin{df}[(Pseudo)Metrizable]
    \label{def:Pseudometrizable}
    Let $(X,\T)$ be a topological space. 
    \begin{enumerate}
        \item We say that $(X,\T)$ (Or $\T$ or X which it wouldn't cause confusion) is \Pseudometrizable if there exists a pseudometric d on X such that $\T$ is the \PseudometricInducedTopology on $(X,d)$. 
        \item We say that $(X,\T)$ (Or $\T$ or X when it wouldn't cause confusion) is \Metrizable if there exists a metric d on X such that $\T$ is the metric topology on $(X,d)$. 
    \end{enumerate}
\end{df}


\begin{prop}[\Pseudometrizable Prequotient]
    \label{prop:pseudometrizableprequotient}
    \rm
    Let $(X,\T_X)$ be a \TopologicalSpace 
    with \RelationOfEqualNeighborhoodFilters $\cong$, and
    with \QuotientTopologicalSpace  $\pa{X/\cong, \T_{X/\cong}}$
    and \QuotientMap T. Let $\pa{X/\cong, \T_{X/\cong}}$ be \Pseudometrizable with \Pseudometric $\tilde{d}$. 
    The following hold. 
    \begin{enumerate}[label=(\roman*), ref={\ref{prop:pseudometrizableprequotient}~\roman*}]
        \item  
        \label{prop:PseudoPre:Pseudometrizable}
        Define $d:X^2 \to [0,\infty)$ by  $d(x,y) = \tilde{d}\pa{[x],[y]}$. 
        Then $\tilde{d}$ is a \Pseudometric on $X$ which is 
        \PseudometricCompatible with $\scT_X$. 
        \item 
        \label{prop:PseudoPre:Metrizable}
        $\tilde{d}$ is a \Metric $(X/\cong, \T_{X/\cong})$.
        \item 
        \label{prop:PseudoPre:Injective}
        If $T$ is \Injective, then $d$ as defined above is a \Metric on $X$.
    \end{enumerate}
    \begin{proof}[Proof Of \ref{prop:PseudoPre:Pseudometrizable}]
    We first prove $d$ to be a \Pseudometric on $X$.
        First, observe that if $x,y \in X$, then
        $d(x,y) =\tilde{d}([x],[y]) \in [0,\infty)$
        so that d is well defined. 
        Also, 
        $ d(x,y) = \tilde{d}([x],[y])=\tilde{d}([y],[x])=d(y,x)$,
        so d is \CommutativeFunction.
        Furthermore, 
        \begin{align*}
            d(x,z) & = \tilde{d}([x],[z])\\
            & \leq \tilde{d}([x],[y])+\tilde{d}([y], [z])\\
            & = d(x,y)+d(y,z)
        \end{align*}
        so d satisfies the \TriangleInequality. 
        Lastly, 
        $d(x,x)=\tilde{d}([x],[x])=0$, 
        and so $d$ is a \Pseudometric on $X$. 
        
        Let $\T_d$ denote the \PseudometricTopology on $(X,d)$. 
        What remains to show is that $\T_X=\T_d$. 
        Since $d(x,y)=\tilde{d}([x], [y])=\tilde{d}(Tx, Ty)$, T is an \Isometry. 
        Let $x \in U \in \T_X$. Then $[x] \in T(U) \in \T_{X/\cong}$. 
        Hence, there is an $\epsilon > 0$ such that $B_{\tilde{d}}([x], \epsilon) \subset T(U)$. 
        By \ref{prop:QST:OpenSetFiber}, $T^{-1}(B_{\tilde{d}}([x], \epsilon) \subset T^{-1}(T(U))=U$.
        Furthermore, by 
        \ref{prop:QST:QuotientMapContinuous}, 
        $T^{-1}(B_{\tilde{d}}([x], \epsilon) \in \T_X$. 
        Since T is an \Isometry $B_d(x, \epsilon) = T^{-1}(B_{\tilde{d}}([x],\epsilon) \subset U$. 
        Thus we have found an \OpenBall contained in $U$  containing an arbitrary point of U.
        Hence, $\T_X \subset \T_d$. As part of the preceeding arguement we also showed that an  arbitrary d-\OpenBall was in $\T_X$, so $\T_d \subset \T_X$, and so equality holds and we're done. 
    \end{proof} 
    \begin{proof}[Proof of \ref{prop:PseudoPre:Metrizable}]
        Let $x,y \in X$ wtih $[x] \neq [y]$. 
        Then $x \not \cong y$. By \ref{prop:relationofzerodistance}, $x \not \cong_d y$. 
        Hence $\tilde{d}([x],[y])=d(x,y) > 0$. 
    \end{proof}
    \begin{proof}[Proof of \ref{prop:PseudoPre:Injective}]
        Let T be \Injective, and suppose $x,y \in X$ with $x \neq y$. 
        Then $[x]=Tx\neq Ty=[y]$, 
        so by \ref{prop:PseudoPre:Metrizable}, $d(x,y) = \tilde{d}([x],[y]) > 0$. 
        
    \end{proof} 
\end{prop} 







\subsection{Seminormed Spaces}
If $(X,\norm{\cdot})$ is a seminormed space, that is a vector space on which $\norm{\cdot}$ would be a norm if not for the fact that it allows some nonzero vector x to have $\norm{x}=0$, then $\norm{\cdot}$ induces a pseudometric on X. 
If $K=\norm{\cdot}^{-1}(\{0\})$, then if $x-y \in K$ and $z \in X$, then $\norm{(z+x)-(z+y)}=0$ by the triangle inequality, so since K is a vector subspace, $X/\norm{\cdot}=\{x+K: x \in X\}$.
Hence $X/K$ is clearly a normed space with norm $\norm{[x]} = \norm{x}$, and $X/K$ will preserve completeness and incompleteness of X. 

\begin{prop}[Seminorm Linear Operators]
    \label{prop:seminormlinearoperators}
    Let $(X,\norm{\cdot}_X)$ be a seminormed space and $(Y,\norm{\cdot}_Y)$ a normed space.
    For each continuous linear operator $T:X \to Y$, define 
    \begin{equation}
        \norm{T} = \sup\limits_{\norm{x}_X \neq 0} \frac{\norm{Tx}_Y}{\norm{x}_X}
    \end{equation}
    Then $\norm{\cdot}$ is a norm on the space of continuous linear operators from X to Y, which we shall denote with $BL(X,Y)$. 
    Further, if $Q:BL(X,Y) \to BL(X/\norm{\cdot}_X^{-1}\{0\},Y)$ is defined by $QT[x]=Tx$, then Q is a well defined linear bijective isometry. 
    \begin{proof}
    \end{proof} 
\end{prop} 
As a consequence of the above proposition, even if $X$ is just a seminormed space, $X^*$ is still a Banach space which is isomorphic to $(X/K)^*$, implying the possibility of extending several results known for normed spaces into the context of seminorms. Further, since $X/K$ can be embedded into X, several existence results, such as Helly's theorem, can also be generalized to the case of a seminormed space. In the context of a seminormed space, the canonical embedding $c:X \to X^{**}$ ceases to be injective but remains a linear isometry, and we shall continue to use the nomenclature that X is \bf reflexive \rm if $c$ is surjective. Since $X^{**}=(X/\norm{\cdot}^{-1}\{0\})^{**}$, it is no surprise that $c(X)=c(X/\norm{\cdot}^{-1}\{0\})$. Hence, a seminormed space is reflexive if and only if it's induced normed space is reflexive.  The weak topology on a set X induced by set of mappings $\{\phi_\alpha:X \to Y_\alpha\}$ where each $Y_\alpha$ is a topological space is the coarsest topology on X which makes each $\phi_\alpha$ continuous. Similar to in the context of a normed space, if X is a seminormed space, we define the weak topology on X to be the topology on X generated by $X^*$, and the $weak^*$ topology on $X^*$ to be the topology generated by $c(X)$.
Before moving on to the classical theory revamped, I present on more useful result about weak topologies of seminormed spaces. 
\begin{prop}[Weak Quotients]
    \label{prop:weakquotients}
    Let X be a seminormed space and $\{Y_\alpha\}_{\alpha \in A}$ be a collection of topological spaces. For each $\alpha \in A$ let $\phi_\alpha:X \to Y_\alpha$ have the property that for every $x,y \in X$, for every $\alpha \in A$, $\norm{x-y}=0 \implies \phi_\alpha(x)=\phi_\alpha(y)$. 
    For each $\alpha \in A$, define $\tilde{\phi}_\alpha:X/\norm{\cdot}^{-1}\{0\} \to Y_\alpha$ by
    $\tilde{\phi}_{\alpha}[x] = \phi_\alpha x$. Let $\T_w$ denote the weak topology on X induced by $\{\phi_\alpha\}_{\alpha \in A}$, and $\T_{\tilde{w}}$ denote the weak topology on $X/\norm{\cdot}^{-1}\{0\}$ induced by $\{\tilde{\phi}_{\alpha}\}_{\alpha \in A}$. Then 
    \begin{equation}
        (X,\T_w)/\norm{\cdot}^{-1}\{0\} = (X/\norm{\cdot}^{-1}\{0\}, \T_{\tilde{w}})
    \end{equation}
    \begin{proof}
    \end{proof} 
\end{prop} 


Finally, before we move on, recall that if $X,Y$ are Topological vector spaces, we can topologize the set of continuous linear operators from X to Y, denoted $BL(X,Y)$ by saying that $\{T_\alpha\}_{\alpha \in A} \subset BL(X,Y)$ converges to $T \in BL(X,Y)$ if there is a neighborhood U of 0 in X such that $T_{\alpha}x \to Tx$ uniformly for $x \in U$. 