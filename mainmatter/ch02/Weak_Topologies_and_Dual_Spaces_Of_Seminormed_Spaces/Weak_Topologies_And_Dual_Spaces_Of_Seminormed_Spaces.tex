\section{Weak Topologies And Dual Spaces Of Seminormed Spaces}
\subsection{Quotient Maps}

\newcommand{\Homeomorphism}[0]{
    \bf \hyperref[def:Homeomorphism]{Homeomorphism} \rm
}








\newcommand{\NeighborhoodBasis}[0]{
	\bf \hyperref[def:TopologicalSpace]{Neighborhood Basis} \rm
}

\newcommand{\Bicontinuous}[0]{
	\bf \hyperref[def:TopologicalSpace]{Bicontinuous} \rm
}

\newcommand{\LetBeTopologicalSpace}[2]{
    Let $\scTopologicalSpace{#1}{#2}$ be a \TopologicalSpace.
}


\begin{df}[Homeomorphism]
    \label{def:Homeomorphism}
    \end{df} 







\label{def:NeighborhoodFilter}   
\newcommand{\NbhFilter}[2]{
    \scU_{#1}\pa{#2}
}


\label{def:RelationOfEqualNeighborhoodFilters}
\newcommand{\RelationOfEqualNeighborhoodFilters}[1]{
    \bf \hyperref[def:RelationOfEqualNeighborhoodFilters]{Relation Of Equal Neighborhood Filters} \rm on #1
}
\begin{df}[Relation of Equal \NeighborhoodFilters]
    Let $(Z, \T_Z)$ be a \TopologicalSpace
	Define the relation 
	$\cong \subset Z \times Z$ 
	by setting, for $x,y \in Z$, 
    \begin{equation}
        x \cong y \iff \scU_{\T_Z}(x)=\scU_{\T_Z}(y)
    \end{equation}
    We call $\cong$ the \RelationOfEqualNeighborhoodFilters{$(Z,\T_Z)$}
\end{df} 


\begin{prop}[\RelationOfEqualNeighborhoodFilters]
    \label{prop:EqualNeighborhoodFiltersEquivalenceRelation}
    
    The
	\RelationOfEqualNeighborhoodFilters
	$\cong$ on a \TopologicalSpaceRef $(Z,\T_Z)$ forms an 
	\EquivalenceRelation	
	on Z. 
    \begin{proof}
        
        Let $x \in (Z,\T_Z)$. 
        Then $\NbhFilter{\Topology{Z}{\T}}{x}$=$\NbhFilter{\Topology{Z}{\T}}{x}$, so $x \cong x$.
        Thus $\cong$ is 
		\ReflexiveRelation. 
        
        Let $x,y \in (Z,\T_Z)$. 
        Suppose $x \cong y$. 
        Then  $\NbhFilter{\Topology{Z}{\T}}{x} = \NbhFilter{\Topology{Z}{\T}}{y}$
        , so trivially  $\NbhFilter{\Topology{Z}{\T}}{y} =\NbhFilter{\Topology{Z}{\T}}{x}$
        , and thus $y \cong x$.
        Hence, $\cong$ is 
		\SymmetricRelation
        
        Let $x,y,z \in (Z,\T_Z)$.
        Let $x \cong y$ and $y \cong z$. 
        Then, 
         $\NbhFilter{\Topology{Z}{\T}}{x}= \NbhFilter{\Topology{Z}{\T}}{y} =  \NbhFilter{\Topology{Z}{\T}}{z}$
         so that $x \cong z$.
         Thus $\cong$ is \TransitiveRelation
         
         Since $\cong$ is 
		 \ReflexiveRelation
		, \SymmetricRelation
		, and \TransitiveRelation, it is an 
		\EquivalenceRelation. 
        
    \end{proof}
\end{prop}
\label{def:EquivalenceClass}
\newcommand{\EquivalenceClass}[0]{\textbf{\hyperref[def:EquivalenceClass]{Equivalence Class}}\xspace}
\newcommand{\EqClass}[2]{\bra{#1}_{\cong}\xspace}
\begin{df}[Equivalence Class]
    
    Let $X \neq \emptyset$.
    Let $\cong$ be an 
	\EquivalenceRelation
	defined on X.  
    Let $x \in X$. 
    We define the set $[x]_{\cong}$ by 
    \begin{equation}
        [x]_{\cong} = \{y \in X | y \cong x\}
    \end{equation} 
    We call $\EqClass{x}{\cong}$ the \EquivalenceClass of x in $(X, \cong)$. 
\end{df}

\begin{prop}[Equivalence Classes Partition]
    \label{prop:EquivalenceClassesPartition}
    
    Let $X \neq \emptyset$. 
    Let $\cong$ be an equivalence relation defined on X. 
    Let $x,y \in X$. 
    The following statements are equivalent. 
    \begin{enumerate}
        \item $[x]_{\cong}  \cap [y]_{\cong} \neq \emptyset$
        \item $x \cong y$
        \item $[x]_{\cong} = [y]_{\cong}$
        \item $[x]_{\cong} \subset [y]_{\cong}$
        \item $[y]_{\cong} \subset [x]_{\cong}$ 
    \end{enumerate}
    
\begin{proof}[Proof That $1 \implies 2$]
Suppose $M:=[x]_{\cong} \cap [y]_{\cong} \neq \emptyset$. 
Then there exists $z \in M$.
Then $z \cong x$, so by symmetry, $x \cong z$. 
But by transitivity, pair with $z \cong y$, we conclude $x \cong y$. 
\end{proof}
\begin{proof}[Proof That $2 \implies 4$]
    Let $x \cong y$ and let $z \in [x]_{\cong}$. 
    Then $z \cong x \cong y$, so $z \cong y$ and $z \in [y]_{\cong}$.
    Since z was arbitrary, we're done. 
\end{proof}
\begin{proof}[Proof That $2 \implies 5$]
    Let $x \cong y$. By symmetry, $y \cong x$, so by $(2 \implies 4)$, we are done. 
\end{proof}
\begin{proof}[Proof That $2 \implies 3$]
    Since  $2 \implies 4$ and $2 \implies 5$ and 5 and 4 together imply 3, we have this. 
\end{proof}
\begin{proof}[Proof That $5 \implies 1$]
    Let $[y]_{\cong} \subset [x]_{\cong}$. 
    Then $y \in [y]_{\cong} = [y]_{\cong} \cap [x]_{\cong} $.
    Hence 1 holds. 
\end{proof}

\end{prop}  
\newcommand{\QuotientSet}[0]{\textbf{\hyperref[def:QuotientSet]{Quotient Set}}\xspace}
\newcommand{\QuoSet}[2]{\ensuremath{#1/#2}\xspace}
\newcommand{\LetBeQuotientSet}[2]{
    Let \ensuremath{\QuoSet{#1}{#2}} be the \QuotientSet of \ensuremath{#1} with respect to the relation \ensuremath{#2}.
}
\begin{df}[Quotient Set]  
\label{def:QuotientSet}
\rm
    Let $X \neq \emptyset$.
    Let $\cong$ be an 
	\EquivalenceRelation defined on X.
    We define the set $X/\cong$ by 
    \begin{equation}
        \QuoSet{X}{\cong} = \braces{ [x]_{\cong} : x \in X}
    \end{equation}
    We call $\QuoSet{X}{\cong}$ the \QuotientSet of X under the relation $\cong$. 
\end{df} 

\begin{rmk}[Quotient Set Partition]
    \label{rmk:quotientsetpartition}
    
    By \ref{prop:EquivalenceClassesPartition}, $X/\cong$ is a partition of X. 
\end{rmk} 
\newcommand{\QuotientMap}[0]{\textbf{\hyperref[df:quotient_map]{Quotient Map}}\xspace}
    
 \newcommand{\QuotientMapInstance}[3]{ #1 : #2\to #2/#3 }
\begin{df}[Quotient Map]
\label{df:quotient_map}
\rm
    Let $X \neq \emptyset$.
    Let $\cong$ be an 
	\EquivalenceRelation 
	on X.
    \LetBeQuotientSet{X}{\cong}
    Define $T:X \to X/\cong$ by setting, for each $x \in X$, 
    \begin{equation}
        T(x)=[x]
    \end{equation}    
    We call T the \QuotientMap of X under $\cong$. 
\end{df} 

\label{prop:QuotientMapSurjective}
\begin{prop}[Quotient Map Surjective]
    Let $X \neq \emptyset$. 
    Let $\cong$ be an equivalence relation on X.
    Let $\QuotientMapInstance{T}{X}{\cong}$  be the \QuotientMap of X under the relation $\cong$. 
    Then T is a surjection. 
    \begin{proof}
       Let $K \in Z/\cong$. 
       Then for some $x \in Z$, $K=[x]$. 
       Then $T(x) = K$. 
       Since K was arbitrary, we are done. 
    \end{proof}
\end{prop} 
\label{def:QuotientSpaceTopology}
\newcommand{\QuotientSpaceTopology}[0]{
    \bf \hyperref[def:QuotientSpaceTopology]{Quotient Topology} \rm
}
\newcommand{\QuotientTopologicalSpace}[0]{
    \bf \hyperref[def:QuotientSpaceTopology]{Quotient Topological Space} \rm 
}

\begin{df}[Quotient Space Topology]
    Let $(Z,\T_Z)$ be a topological space. 
    Let $\cong$ be the \RelationOfEqualNeighborhoodFilters{$(Z, \T_Z)$}. 
    Let T be the \QuotientMap of Z under the relation $\cong$. 
    Define $\T_{Z/\cong}$ by
    \begin{equation}
        \T_{Z/\cong} = \left\{ \bigcup_{x \in U}\{T(x)\} \in 2^{Z/\cong}| U \in \T_Z \right\}
    \end{equation}
    By \ref{prop:QuotientSpaceTopology}, $\T_{Z/\cong}$ is a topology on $Z/\cong$.
    We call $\T_{Z/\cong}$ the \QuotientSpaceTopology and we call $\pa{Z/\cong, \T_{Z/\cong}}$ the \QuotientTopologicalSpace of $(Z, \T_Z)$.
    
\end{df}

\begin{prop}[Quotient Space Topology]
    \label{prop:QuotientSpaceTopology}
    
    Let $(Z,\T_Z)$ be a topological space 
    with \QuotientTopologicalSpace  $\pa{Z/\cong, \T_{Z/\cong}}$
    and \QuotientMap T.
    
    Then the following are true. 
    \begin{enumerate}
        \item $\T_{Z/\cong}$ is a topology on $Z/\cong$. 
        \item $T:(Z, \T_Z) \to (Z/\cong, \T_{Z/\cong})$ is continuous. 
        \item If U is open (closed) in $(Z,\T_Z)$ then $T(U)$ and $T(Z\setminus U)$ partition $Z/\cong$. 
        \item If U is open in $(Z, \T_Z)$, then $T^{-1}(T(U))=U$. 
        \item If K is closed in $(Z,\T_Z)$, then $T^{-1}T(K)=K$. 
        \item $T:(Z, \T_Z) \to (Z/\cong, \T_{Z/\cong})$ is an open mapping. 
        \item $T:(Z, \T_Z) \to (Z/\cong, \T_{Z/\cong})$ is a  closed mapping.
        \item $(Z, \T_Z)$ is a compact space if and only if $(Z/\cong, \T_{Z/\cong})$ is a compact space
        
    \end{enumerate} 
    \begin{proof}[Proof of 1]
        Since $\emptyset \in \T_Z$, we have 
        \begin{equation}
            \emptyset = \bigcup\limits_{x \in \emptyset} \{Tx\} \in \T_{Z/\cong}
        \end{equation}
        Since $Z \in \T_Z$, and by \ref{rmk:quotientsetpartition}, 
        \begin{equation} 
            Z/\cong = \bigcup_{x \in Z} \{[x]\}= \bigcup\limits_{x \in Z} \{T(x)\} \in \T_{Z/\cong}
        \end{equation} 
        
        Let $\{U_{\alpha} | \alpha \in A\} \subset \T_{Z/\cong}$. 
        For each $\alpha \in A$, there exists $B_{\alpha} \in \T_{Z}$ such that we have
        \begin{equation} 
            U_{\alpha } = \bigcup_{x \in B_{\alpha}} \{Tx\}
        \end{equation} 
        Since $\bigcup_{\alpha \in A} B_\alpha \in \T_{Z}$, we have 
        \begin{equation}
            \bigcup_{\alpha \in A} U_{\alpha}= \bigcup\limits_{\alpha \in A} \bigcup\limits_{x \in U_\alpha} \{T(x)\} = \bigcup\limits_{x \in \bigcup\limits_{\alpha \in A} B_{\alpha}} \{T(x)\} \in \T_{Z/\cong}
        \end{equation} 
        Let $\{U_i\}_{i=1}^n \subset \T_{Z/\cong}$. 
        For each $i \in \{1, ..., n\}$, there exists $B_i \in \T_{Z}$ such that
        \begin{equation}
            U_i = \bigcup_{x \in B_{i}} \{T(x)\}
        \end{equation}
        Suppose 
        \begin{equation}
            [x_0] \in \bigcap\limits_{i=1}^n \bigcup\limits_{x \in B_i} \{T(x)\}
        \end{equation}
        Then for each $i \in \{1,..., n\}$, there is a $y_i \in B_i$ such that $ y_i \cong x_0$. 
        Since each $B_i$ is open, the definition of $\cong$ implies that $x_0 \in B_i$ for every i. Hence, 
        \begin{equation} 
            x_0 \in \bigcap_{i=1}^n B_i
        \end{equation} 
        Implying 
        \begin{equation}
            [x_0] \in  \bigcup\limits_{x \in \bigcap\limits_{i=1}^n B_i} \{[x]\}
        \end{equation} 
        Hence, 
        \begin{equation} 
            \bigcap\limits_{i=1}^n \bigcup\limits_{x \in B_i} \{T(x)\}
            \subset
            \bigcup\limits_{x \in \bigcap\limits_{i=1}^n B_i} \{[x]\}
        \end{equation} 
        Furthermore, since the reverse inclusion is obvious, 
        and since $\bigcap_{i=1}^n B_i \in \T_{Z}$, we have 
        \begin{equation}
            \bigcap_{i=1}^n U_i = \bigcap_{i=1}^n \bigcup_{x \in B_i} \{T(x)\}= \bigcup\limits_{x \in \bigcap\limits_{i=1}^n B_i} \{T(x)\} \in \T_{Z/\cong}
        \end{equation}
    \end{proof}
    \begin{proof}[Proof of 2]
        Let $V \in \T_{Z/\cong}$. 
        Let $x_0 \in T^{-1}V$. 
        Then $[x_0] \in V$. 
        By definition, there is a $U \in \T_Z$ such that 
        \begin{equation}
            T(U) \subset \bigcup\limits_{x \in U} \{T(x)\}=V
        \end{equation}
        Hence there is a $y_0 \in U$  such that 
        \begin{equation}
            [x_0] \in T(y_0) = \{[y_0]\}
        \end{equation}
        Therefore, $x \cong y$. 
        Definition of the relation of equal neighborhood filters implies $\scU(x_0)=\scU(y_0)$. 
        Hence, $x_0 \in U \subset T^{-1}(V)$.
    \end{proof}
    \begin{proof}[Proof of 3]
        Let $K$ be closed in $(Z,\T_Z)$. 
        Then each point $x_0$ in $Z\setminus K$ has some $U_{x_0} \in \scU_{\T_Z}(x_0)$ which is disjoint from K.
        Hence $y_0 \not \cong x_0$ for any $y_0 \in K$, $x_0 \in Z\setminus K$. 
        Hence $T(K)$ is disjoint from $T\pa{Z \setminus K}$. 
        This fact, paired with \ref{prop:QuotientMapSurjective}, implies $T(Z\setminus K)$ and T(K) partition $Z/\cong$.
    \end{proof}
    \begin{proof}[Proof of 4]
        Let $U \in \T_Z$. 
        The nontrivial direction to prove is $T^{-1}\pa{T(U)} \subset U$.
        Let $y \in T^{-1}\pa{T(U)}$. 
        Then $[y]=Ty \in T(U)$.
        Hence, $[y]=T(x)=[x]$ for some $x \in U$. 
        Since $y \cong x$ and $x \in U \in \scU_{\T_Z}(x)$, we have $U \in \scU_{\T_Z}(y)$. 
        Hence $y \in U$.
        Since y was arbitrary, $T^{-1}\pa{T(U)} \subset U$, and equality is obvious because the other direction of inclusion is trivial. 
    \end{proof}
    \begin{proof}[Proof of 5]
        Let K be closed in $(Z,\T_Z)$. Part 3 Of this result implies $Z/\cong$ is partitioned by $T(K)$ and $T(Z\setminus K)$. 
        
        By part 4 of this proposition, 
        \begin{align*}
            T^{-1}\pa{T(K)}&=T^{-1} \pa{T(Z) \setminus T(Z \setminus K)} \\
            &= T^{-1}\pa{Z/\cong \setminus T(Z \setminus K)}\\
            &=T^{-1}(Z/\cong) \setminus T^{-1}(T(Z\setminus K)) \\
            &= Z \setminus \pa{Z \setminus K} \\
            &= K
        \end{align*}      
    \end{proof}
    \begin{proof}[Proof of 6]
        Let $U \in \T_Z$.
        Then by definition of the \QuotientSpaceTopology
        \begin{equation}
            TU= \bigcup_{x \in U} \{T(x)\}  \in \T_{Z/\cong}
        \end{equation}
    \end{proof}  
    \begin{proof}[Proof of 7] 
        Let K be closed in $(Z,\T_Z)$. 
        Then $Z \setminus K \in \T_Z$. 
        By Parts 3 and five of this proposition, we know $T(K) = Z/\cong \setminus T(Z\setminus K)$ and also that $T(Z\setminus K) \in \T_{Z/\cong}$. Hence $T(K)$ is closed in $(Z/\cong, \T_{Z/\cong})$. 
    \end{proof} 
    \begin{proof}[Proof of 8]
        Let $(Z,\T_Z)$ be compact. 
        Let $\{U_{\alpha}\}_{\alpha \in A}$ be an open covering of $(Z/\cong, \T_{Z/\cong})$. 
        Then $\{T^{-1}\pa{U_{\alpha}} | \alpha \in A\}$ is an open covering of $(Z, \T_Z)$. 
        Compactness of $(Z, \T_Z)$ guarantees the existence of a finite subcovering $\{T^{-1}\pa{U_{\alpha_i}} | i \in \{1, ..., n\}\}$. 
        Hence
        $\{U_{\alpha_i} | i \in \{1, ..., n\}\}=\{TT^{-1}(U_{\alpha_i}) | i \in \{1, ..., n\}\}$ is an open covering of $(Z/\cong, \T_{Z/\cong})$. 
         And the compactness of $(Z/\cong, \T_{Z/\cong})$ is verified. 
         
         
         Now, suppose $(Z/\cong, \T_{Z/\cong})$ is compact. 
         Let $\{V_{\beta} | \beta \in B\}$ be an open covering of $(Z, \T_Z)$. 
         Since T is an open mapping, $\{T(V_{\beta}) | \beta \in B\}$ is an open covering of $(Z/\cong, \T_{Z/\cong})$ which by compactness has a finite subcover $\{T(V_{\beta_i}) | i \in \{1, ..., n\}\}$. 
         By part 4 of \ref{prop:QuotientSpaceTopology}, 
         $\{V_{\beta_i}| i \in \{1, ..., n\}\} = \{T^{-1}(T(V_{\beta_i})) |i \in \{1, ..., n\}\}$ is then an open subcovering of $(Z, \T_Z)$. 
     %    
    \end{proof}
\end{prop} 

\subsection{Pseudometrics}
\label{def:Symmetricmap}
\newcommand{\SymmetricMap}[0]{
    \bf \hyperref[def:Symmetricmap]{Symmetric Map} \rm
}
\newcommand{\CommutativeFunction}[0]{
    \bf \hyperref[def:Symmetricmap]{Commutative} \rm
}
\newcommand{\FunctionCommutativity}[0]{
    \bf \hyperref[def:Symmetricmap]{Commutativity} \rm
}
\begin{df}[\CommutativeFunction]
    Let X and Y be sets. 
    We say that a map 
    $f:X \times X \to Y$ is a \SymmetricMap 
    if for each 
    $x_0,x_1 \in X$, 
    $f(x_0,x_1)=f(x_1,x_0)$.
    In this situation, 
    we may also refer to $f$ as
    \CommutativeFunction, 
    or say that $f$ posesses 
    \FunctionCommutativity.
\end{df} 

\label{def:TriangleInequality}
\newcommand{\TriangleInequality}[0]{
    \bf \hyperref[def:TriangleInequality]{Triangle Inequality} \rm
}
\begin{df}[Symmetric Map]
    
    Let X be a set and $(Y,+, \leq)$ be a totally ordered magma.
    We say that a map $f:X \times X \to Y$ satisfies the \TriangleInequality if for each $x_0,x_1,x_3 \in X$, we have
    \begin{equation*}
        f(x_0,x_2) \leq  f(x_0,x_1)+f(x_1,x_2)
        \end{equation*}
\end{df} 
\label{def:pseudometric}
\newcommand{\Pseudometric}[0]{
    \bf \hyperref[def:pseudometric]{Pseudometric} \rm
}
\newcommand{\PseudometricSpace}[0]{
    \bf \hyperref[def:pseudometric]{Pseudometric Space} \rm
}

\begin{df}[Pseudometric]
    Let $X \neq \emptyset$. 
    Let $d:X \times X \to [0,\infty)$ be a \SymmetricMap that satisfies the \TriangleInequality.
    Under these conditions we call d a \Pseudometric on X and we call $\pa{X,d}$ a \PseudometricSpace.
    \end{df} 
\label{def:pseudometriccauchysequence}
\newcommand{\PseudometricCauchySequence}[0]{
    \bf \hyperref[def:pseudometriccauchysequence]{Pseudometric Cauchy Sequence} \rm
}
\begin{df}[Pseudometric Cauchy Sequence]

    Let $(X,d)$ be a \PseudometricSpace.
    We say that a sequence $\{x_i\}_{i \in \N}$ is a \PseudometricCauchySequence
    if, for each $\epsilon > 0$, there exists an $N \in \N$, sucht that for 
    each pair $m,n \in \N$ such that $m>N$ and $n>N$, we have 
    \begin{equation}
        d(x_m,x_n) < \epsilon
    \end{equation}
\end{df}
\label{def:uniformlycauchy}
\newcommand{\UniformlyCauchy}[0]{
    \bf \hyperref[def:uniformlycauchy]{Uniformly Cauchy} \rm
}
\begin{df}[Uniformly Cauchy]
	Let $(X_\alpha, d_\alpha)$ be a \PseudometricSpace
	for $\alpha \in A$ where A is some indexing set. 
	For each $\alpha \in A$
	, let $\phi_\alpha :=\{x_i^\alpha\}_{i \in \N} \subset X_{\alpha}$
	be a sequence. 
	We say that the collection $\{\phi_\alpha\}_{\alpha \in A}$ 
	is 
	\UniformlyCauchy if for each $\epsilon > 0$, there exists an 
	$N \in \N$ such that for each pair $m,n \in N$
	such that $m>N$ and $n>N$, and for each $\alpha \in A$, 
	we have 
	\begin{equation}
	d_{\alpha} \pa{x^{\alpha}_n, x^{\alpha}_m} < \epsilon
	\end{equation}
\end{df}
\label{def:pseudometricsequenceconvergence}
\newcommand{\PseudometricConvergence}[0]{\textbf{\hyperref[def:pseudometricsequenceconvergence]{Pseudometric-Convergence}}\xspace}
\newcommand{\PseudometricConvergent}[0]{\textbf{\hyperref[def:pseudometricsequenceconvergence]{Pseudometrically-Convergent}}\xspace}
\newcommand{\PseudometricConverges}[0]{\textbf{\hyperref[def:pseudometricsequenceconvergence]{Pseudometric-Converges}}\xspace}
\begin{df}[Pseudometric Convergence]
    Let $(X,d)$ be a \PseudometricSpace.
	Let $\{x_i\}_{i \in \N}$ be a sequence in $(X,d)$.
    Let $x_0 \in X$.  
    We say that 
	$\{x_i\}_{i \in \N}$ 
	exhibits 
	\PseudometricConvergence 
	to 
	$x_0$ 
	in d,
	or we say that 
	$\{x_i\}_{i \in \N}$  
	\PseudometricConverges 
	to 
	$x_0$ 
	in d, 
	or we say that 
	$\{x_i\}_{i \in \N}$ 
	is 
	\PseudometricConvergent 
	to 
	$x_0 \in d$ 
	if, 
    for every 
	$\epsilon > 0$, 
	there is an 
	$N \in \N$ 
	such that for every 
	$n>N$, 
	we have 
    \begin{equation}
        d(x_0, x_n) < \epsilon
    \end{equation}
\end{df}

\label{def:pseudometriccomplete}
\newcommand{\PseudometricComplete}[0]{
    \bf \hyperref[def:pseudometriccomplete]{Pseudometric-Complete} \rm
}
\newcommand{\Complete}[0]{
    \bf \hyperref[def:pseudometriccomplete]{Complete} \rm
}
\begin{df}[Pseudometric Complete]
    We say that a \PseudometricSpace $(X,d)$ is 
    \PseudometricComplete if each 
	\PseudometricCauchySequence 
	sequence in $(X,d)$ 
	\PseudometricConverges to a limit in $X$.


	In the case that d is a \Metric, then
	being \PseudometricComplete is 
	equivalent to beging \Complete
	in the classical sense, so 
	we will commonly refer to a \PseudometricSpace
	which is \PseudometricComplete as simply
	being \Complete. 
    \end{df}
\label{def:pseudometricball}
\newcommand{\OpenBall}[0]{
    \bf \hyperref[def:pseudometricball]{Open Ball} \rm
}
\newcommand{\ClosedBall}[0]{
    \bf \hyperref[def:pseudometricball]{Closed Ball} \rm
}
\begin{df}[Pseudometric Ball]
    Let $(X,d)$ be a \PseudometricSpace. 
    For each $x_0  \in X$ and each $\epsilon > 0$, we define the following.
    \begin{enumerate}
        \item  $B_d(x_0, \epsilon) := \{y \in X | d(x_0,y) < \epsilon\}$ denotes the \OpenBall about $x_0$ with radius $\epsilon$. 
    \item $\overline{B_d}(x_0,\epsilon) := \{y \in X | d(x_0,y) \leq \epsilon \}$ denotes the \ClosedBall about $x_0$ with radius $\epsilon$. 
    \end{enumerate} 
    
     
    \end{df} 
\label{def:pseudometrictopology}
\newcommand{\PseudometricTopology}[0]{
    \bf \hyperref[def:pseudometrictopology]{Pseudometric Topology} \rm
}
\newcommand{\PseudometricInducedTopology}[0]{
    \bf \hyperref[def:pseudometrictopology]{Pseudometric Topology} \rm
}
\begin{df}[Pseudometric Topology]
    Let $(X,d)$ be a \PseudometricSpace, and let $\scB$ be the set of \OpenBall's in $(X,d)$. 
    By \ref{prop:pseudometrictopology}, $\scB$ is the basis for a unique topology $\T_d$ on X. 
    We call $\T_d$ the \PseudometricInducedTopology induced by $d$ on X. 

\end{df}
\label{prop:pseudometrictopology}
\begin{prop}[Pseudometric Topology]
    Let $(X,d)$ by  \PseudometricSpace and let $\scB$ be the set of \OpenBall's in $(X,d)$. 
    The following are true. 
    \begin{enumerate}
        \item There exists a unique topology $\T_d$ on X which $\scB$ is a basis of. That is, the \PseudometricTopology $\T_d$ is well defined. 
        \item The \PseudometricInducedTopology is first countable. That is, each of its points permits a countable neighborhood basis. 
    \end{enumerate}
    \begin{proof}[Proof of 1]
        Uniqueness is guaranteed by closure under arbitrary unions of a topology. 
        For existense, it is sufficient to show that the collection of arbitrary unions
        of elements of $\scB$ is closed under finite intersections. 
        Suppose that for $1\leq i \leq n$, we have $\{U_{\alpha_i} | \alpha_i \in A_i\} \subset \scB$
        and consider the set
        \begin{equation}
            U=\bigcap_{i=1}^n \bigcup_{\alpha_i \in A_i} U_{\alpha_i}
        \end{equation}
        Let $x_0 \in U$. 
        For each $i \in \{1, ..., n\}$, there exists $\alpha_i \in A_i$ such that 
        \begin{equation}
            x_0 \in U_{\alpha_i} = B_d(x_i; \epsilon_i)
        \end{equation}
        For each $i \in \{1, ..., n \}$, define $\delta_i = d(x_0, x_i)$. Then $0 < \delta_i < \epsilon_i$. 
        Then, for each $i \in \{1, ..., n \}$, 
        \begin{equation}
            B_d(x_0; \epsilon_i-\delta_i) \subset U_{\alpha_i} \subset \bigcup_{\alpha_i \in A_i} U_{\alpha_i}
        \end{equation}
        Define 
        \begin{equation}
            \delta_{x_0} = \min\limits_{i=1}^n \pa{ \epsilon_i-\delta_i}
        \end{equation}
        Then $x_0 \in B(x_0; \delta_{x_0} ) \subset U$. 
        If $U=\{x_{\alpha} | \alpha \in A\}$, then the arbitrary nature of $x_0$ above means 
        we can repeat this construction, writing 
        \begin{equation}
            U \subset \bigcup_{\alpha \in A} B(x_{\alpha} ; \delta_{x_{\alpha}} )\subset \bigcup_{\alpha \in A} U = U
        \end{equation}
        Hence, $U \in B$ and the proof is complete. 
    \end{proof}
    \begin{proof}[Proof of 2]
        Let $x_0 \in X$. 
        I claim that 
        \begin{equation}
            \scB_{x_0}:= \left\{ B_d\pa{x_0; \frac{1}{n}} | n \in \N\right\}
        \end{equation}
        is a neighborhood basis for $(X,\T_d)$ at $x_0$. 
        Let $U \in \scU_{\T_d}(x)$ be open in $\T_d$. 
        Since $\scB$ is a basis for $\T_d$, for some $y0 \in X$ and $\epsilon > 0$, 
        $x_0 \in B_d(y_0; \epsilon) \subset U$. 
        Let $\delta = d(x_0, y_0)$. Then $\epsilon - \delta > 0$. 
        Define
        \begin{equation}
            n = \ceil{ \frac{1}{\epsilon - \delta}}
        \end{equation}
        Then we have 
        \begin{equation}
            B_d\pa{x_0 ; \frac{1}{n}} \subset B_d(x_0 : \epsilon - \delta) \subset B(y_0 ; \epsilon) \subset U
        \end{equation}
    \end{proof}
\end{prop}

\label{def:relationofzerodistance}
\newcommand{\RelationOfZeroDistance}[0]{
    \bf \hyperref[def:relationofzerodistance]{Relation Of Zero Distance} \rm
}
\begin{df}[Relation Of Zero Distance]
    Let $(X,d)$ be a \PseudometricSpace. 
    Define the relation  $\cong_d$ on $X \times X$ by setting, for $x,y \in X$, 
    \begin{equation}
        x \cong_d y \iff d(x,y) = 0
    \end{equation}
    We call $\cong_d$ the \RelationOfZeroDistance on $(X,d)$. 
\end{df}

\begin{prop}[Relation Of Zero Distance is the Relation Of Equal Neighborhood Filters]
    \label{prop:relationofzerodistance}
    Let $(X,d)$ be a \PseudometricSpace.
    Let $\cong_{\T_d}$ be the \RelationOfEqualNeighborhoodFilters $(X,\T_d)$. 
    Let $\cong_d$ be the \RelationOfZeroDistance on $(X,d)$. 
    Then $\cong_{\T_d} = \cong_d$. 
    \begin{proof}
        Let $x,y \in X$ and suppose $x_0 \cong_d y_0$.
        Let $U \in \scU_{\T_d}(x_0)$. Then for some $\epsilon > 0$, 
        $x_0 \in B(x_0;\epsilon) \subset U$. 
        Since $x_0 \cong_d y_0$, $d(x_0,y_0) = 0$, so $y_0 \in B(x_0 ; \epsilon) \subset U$. 
        Hence $U \in \scU_{\T_d}(y_0)$. 
        The arbitrary nature of $U \in \scU_{\T_d}(x_0)$ implies 
        \begin{equation}
            \scU_{\T_d}(x_0) \subset \scU_{\T_d}(y_0)
        \end{equation}
        A reverse construction would just as easily show the reverse inclusion, so we conclude that $x_0 \cong_{\T_d} y_0$. 
        Now suppose $x_0 \cong_{\T_d} y $. Then or each $n \in \N$, 
        \begin{equation}
            y_0 \in B_{d} \pa{x_0 ; \frac{1}{n}}
        \end{equation}
        Hence $d(x_0, y_0) < \frac{1}{n}$ for each natural n, therefore $d(x_0,y_0) = 0$ and $x_0 \cong_d y_0$. 
    \end{proof}
\end{prop}

\newcommand{\PseudometricInducedMetric}[0]{
    \bf \hyperref[def:pseudometricinducedmetric]{Pseudometric Induced Metric} \rm
}
\newcommand{\MetricInducedByPseudometric}[0]{
    \bf \hyperref[def:pseudometricinducedmetric]{Metric Induced By The Pseudometric} \rm
}
\begin{df}[Metric Space Induced By Pseudometric]
    \label{def:pseudometricinducedmetric}
    Let $(X,d)$ be a \PseudometricSpace, and let $\cong$ be the \RelationOfZeroDistance, which by \ref{prop:relationofzerodistance} is also the \RelationOfEqualNeighborhoodFilters $(X,\T_d)$. 
    Define $\tilde{d}: X/\cong \to [0,\infty)$ by 
    \begin{equation}
        \tilde{d}\pa{\bra{x}, \bra{y}} = d(x,y)
    \end{equation}
    By \ref{prop:pseudometricinducedmetric}, $\tilde{d}$ is well defined and is in fact a metric on $X/\cong$, so we call $\tilde{d}$ the \MetricInducedByPseudometric d on X, or we call it the \PseudometricInducedMetric of $(X,d)$. 
\end{df}
\begin{prop}[Metric Space Induced By Pseudometric Space]
    \label{prop:pseudometricinducedmetric}
    %Let $X$, d, $\cong$, and $\tilde{d}$ be defined as in \ref{def:pseudometricinducedmetric}
    Let $(X,d)$ be a \PseudometricSpace, $\cong$ the \RelationOfZeroDistance on $(X,d)$ and $\tilde{d}$ be defined as in \ref{def:pseudometricinducedmetric}.
    Let $(X/\cong, \T_{X/\cong})$ be the  \QuotientTopologicalSpace with \QuotientMap T, and let $(X/\cong, \T_{\tilde{d}})$ be the topological space induced by the metric space $(X/\cong, \tilde{d})$. 
    The following are true. 
    \begin{enumerate}
        \item $\tilde{d}$ is in fact well defined, and is a metric on $X/\cong$, justifying calling it the \MetricInducedByPseudometric d.
        \item $\T_{X/\cong} = \T_{\tilde{d}}$
        \item T is an isometric surjection $(X,d)$ to $(X/\cong, \tilde{d})$
        \item $(X/\cong, \tilde{d})$ is complete if and only if $(X, d)$ is \PseudometricComplete.

    \end{enumerate}
    \begin{proof}[Proof of 01]
        First we show that $\tilde{d}$ is well defined as a mapping, that is, that if
        $x_0,y_0 \in X$ and $x_1 \cong x_0$ and $y_1 \cong y_0$, then we should have 
        \begin{equation}
            \tilde{d}\pa{\bra{x_0},\bra{y_0}}=\tilde{d}\pa{\bra{x_1},\bra{y_1}}
        \end{equation}
        
        This is easy, as
        \begin{align*}
            d(x_0,y_0) & \leq d(x_0,x_1)+d(x_1,y_1)+d(y_1,y_0)\\
            & = d(x_1,y_1)\\
            & \leq d(x_1,x_0)+d(x_0,y_0)+d(y_0,y_1)\\
            &=d(x_0,y_0)
         \end{align*}
         Nonnegativity falls directly from the nonnegativity of d. 
         Proving that $\tilde{d}$ is a \SymmetricMap is equally trivial
         \begin{align*}
             \tilde{d}\pa{\bra{x}, \bra{y}}= d(x,y) = d(y,x) = \tilde{d}\pa{\bra{y}, \bra{x}}
         \end{align*}
         Proving that $\tilde{d}$ satisfies the \TriangleInequality is similarly simple, letting $x_0,y_0,z_0 \in X$, we have
         \begin{align*}
             \tilde{d}\pa{\bra{x_0}, \bra{z_0}} & = d(x_0,z_0) \\
             & \leq d(x_0, y_0)+d(y_0, z_0)\\
             & = \tilde{d}\pa{\bra{x_0}, \bra{y_0}}+ \tilde{d}\pa{ \bra{y_0}, \bra{z_0}}
         \end{align*}
         
         All that remains is to show positivity on nonequal arguements. Let $x_0, y_0 \in X$ such that $\bra{x_0} \neq \bra{y_0}$. Then $x_0 \not \cong y_0$. Hence \begin{equation*}
             \tilde{d}\pa{\bra{x_0}, \bra{y_0}}=d(x_0,y_0) \neq 0
             \end{equation*}
    \end{proof}
    \begin{proof}[Proof of 02]
        By \ref{prop:QuotientSpaceTopology}, part 9, $\scB_{\cong}:=\{T(B_d(x;\epsilon)) | x \in X, \epsilon > 0\}$ is a basis for $\T_{X/\cong}$. 
        By definition, $\scB_{\tilde{d}}:=\{B_{\tilde{d}}(\bra{x}; \epsilon) | x \in X , \epsilon > 0 \}$ is a basis for $\T_{\tilde{x}}$. 
        
        I claim that for each $x \in X$ and $\epsilon > 0$, 
        \begin{equation}
            T\pa{B_d(x;\epsilon)} = B_{\tilde{d}} \pa{\bra{x}; \epsilon}
        \end{equation}
        To see this, 
        suppose $\tilde{y} \in T\pa{B_d(x;\epsilon)}$. 
        Then $\tilde{y}=T(y)$ for some $y \in B_d(x;\epsilon)$. 
        Hence 
        \begin{align*}
            \tilde{d}(\tilde{y},\bra{x})& =\tilde{d}(T(y),\bra{x})\\
            &=\tilde{d}(\bra{y},\bra{x})\\
            & = d(y,x) \\
            & < \epsilon
        \end{align*}
        Hence $\tilde{y} \in B_d(\bra{x} ; \epsilon)$, and so 
                \begin{equation}
            T\pa{B_d(x;\epsilon)} \subset  B_{\tilde{d}} \pa{\bra{x}; \epsilon}
        \end{equation}
        Suppose $\bra{y} \in B_{\tilde{d}}\pa{\bra{x} ; \epsilon}$. 
        Then $d(x,y) = \tilde{d}()\bra{x},\bra{y}) < \epsilon$, so $y \in B_d(x; \epsilon)$. 
        Hence $[y]=T(y) \in T\pa{B_d(x;\epsilon)}$, so the reverse inclusion also holds, and so the above claim holds. 
        This, paired witht he fact that 
        \begin{equation}
            \{[x] | x \in X\}= X/\cong
        \end{equation}
        finishes the result. 
        
    \end{proof}
    \begin{proof}[Proof of 03]
        Falls directly from the definition $T(x)=\bra{x}$, hence
        \begin{equation}
            d(x,y) = \tilde{d}\pa{\bra{x}, \bra{y}} = \tilde{d}\pa{T(x), T(y)}
        \end{equation}
        
        T is surjective by \ref{prop:QuotientMapSurjective}.
    \end{proof}
    \begin{proof}[Proof of 04]
        Let $(X,d)$ be \PseudometricComplete. 
        Let $\{[x_i]\}_{i \in \N} \subset (X/\cong, \tilde{d})$ be a \PseudometricCauchySequence. 
        Let $\epsilon > 0$. 
        Then there exists $N \in \N$ such that for $m,n > N$, we have 
        \begin{equation}
            d(x_m, x_n) = \tilde{d}(Tx_m, Ty_m) =\tilde{d}([x_m], [x_n]) < \epsilon
        \end{equation}
        So the sequence $\{x_i\}_{ i \in \N} \subset (X,d)$ is \PseudometricCauchySequence. 
        Since $(X,d)$ is \PseudometricComplete, this sequence has a limit, say $x_i \to x \in (X,d)$. 
        But, we have $[x_i]=Tx_i \to Tx = [x]$, so $\{[x_i]\}$ is convergent, and since that sequence was arbitrary, $(X/\cong, \tilde{d})$ is \PseudometricComplete. 
        
        Let $(X/\cong, \tilde{d})$ be \PseudometricComplete. 
        Let $\{x_i\} \subset X$ be a \PseudometricCauchySequence.
        Let $\epsilon > 0$. Then there exist $N \in \N$ such that for $m,n > N$, we have
        \begin{equation}
            \tilde{d}\pa{[x_m], [x_n]} = \tilde{d}(Tx_m, Tx_n) = d(x_m, x_n) < \epsilon
        \end{equation}
        so that $\{[x_i]\}_{i \in \N}$ is also a \PseudometricCauchySequence. 
        Since $(X/\cong, \tilde{d})$ is \PseudometricComplete, this sequence has a limit, say $[x_i] \to y \in X/\cong$. 
        Since T is surjective, for some $x \in X$, $Tx \in y$, and so
        \begin{equation}
            d(x, x_i) = \tilde{d}(Tx, Tx_i) =\tilde{d}(y, [x_i]) \to 0
        \end{equation}
        meaning $x_i \to x$ and we are done. 
                
\end{proof}
\end{prop}
\newcommand{\Pseudometrizable}[0]{\textbf{\hyperref[def:Pseudometrizable]{Pseudometrizable}}\xspace}
\newcommand{\Metrizable}[0]{\textbf{\hyperref[def:Pseudometrizable]{Metrizable}}\xspace}
\begin{df}[(Pseudo)Metrizable]
    \label{def:Pseudometrizable}
    Let $(X,\T)$ be a topological space. 
    \begin{enumerate}
        \item We say that $(X,\T)$ (Or $\T$ or X which it wouldn't cause confusion) is \Pseudometrizable if there exists a pseudometric d on X such that $\T$ is the \PseudometricInducedTopology on $(X,d)$. 
        \item We say that $(X,\T)$ (Or $\T$ or X when it wouldn't cause confusion) is \Metrizable if there exists a metric d on X such that $\T$ is the metric topology on $(X,d)$. 
    \end{enumerate}
\end{df}

\begin{prop}[Pseudometrizable Prequotient]
    \label{prop:pseudometrizableprequotient}
    Let $(X,\T_X)$ be a topological space 
    with \QuotientTopologicalSpace  $\pa{X/\cong, \T_{X/\cong}}$
    and \QuotientMap T. Let $\pa{X/\cong, \T_{X/\cong}}$ be \Pseudometrizable with \Pseudometric $\tilde{d}$. 
    
    The following hold. 
    \begin{enumerate}
        \item  $(X,\T_X)$ is \Pseudometrizable. 
        \item $(X/\cong, \T_{X/\cong})$ is \Metrizable. 
        \item If T is injective, then $(X,\T_X)$ is metrizable. 
    \end{enumerate}
    \begin{proof}[Proof Of One]
        Define $d:X \times X \to [0,\infty)]$ by 
        \begin{equation*}
            d(x,y) = \tilde{d}\pa{[x], [y]}.
        \end{equation*}
        Then
        \begin{equation*}
            d(x,y) =\tilde{d}([x],[y]) \in [0,\infty)
        \end{equation*}
        so that d is well defined. 
        
        Also, 
        \begin{equation*}
            d(x,y) = \tilde{d}([x],[y])=\tilde{d}([y],[x])=d(y,x)
        \end{equation*}
        , so d is a \SymmetricMap.
        
        Also, 
        \begin{align*}
            d(x,z) & = \tilde{d}([x],[z])\\
            & \leq \tilde{d}([x],[y])+\tilde{d}([y], [z])\\
            & = d(x,y)+d(y,z)
        \end{align*}
        so d satisfies the \TriangleInequality. 
        Also, 
        \begin{equation}
            d(x,x)=\tilde{d}([x],[x])=0
        \end{equation}
        and so d is a \Pseudometric on X. 
        
        Let $\T_d$ denote the \PseudometricTopology on $(X,d)$. What remains to show is that $\T_X=\T_d$. 
        
        
           By \ref{def:pseudometricinducedmetric} paired with how d is defined, $\tilde{d}$ is the \PseudometricInducedMetric of $(X,d)$. Let $\cong_d$ denote the \RelationOfZeroDistance on $(X,d)$, and 
           let $\cong_{\T_X}$ denote the \RelationOfEqualNeighborhoodFilters on $(X,\T_X)$. 
           
           %claim: $\cong_d=\cong_{\T_X}$ to use a theormem. 

    \end{proof} 
    \begin{proof}[Proof of Two]
    \end{proof}
    \begin{proof}[Proof of Three]
    \end{proof} 
\end{prop} 


\subsection{Topological Vector Spaces} 
\label{def:topologicalvectorspace}
\newcommand{\TVS}[0]{
    \bf \hyperref[def:topologicalvectorspace]{Topological Vector Space} \rm
}

\begin{df}[\TVS]
Let $(V,+,\cdot, 0)$ be a 
\VectorSpace over a \Field $\F \in \{\R, \C\}$. 
Let $\T$ be a 
\TopologyRef on V such that 
$(V, +, 0)$ is a 
\TopologicalGroup
and 
$\cdot: \F \times V \to V$ 
is
\Continuous. 
Then we call $(V,\T)$ a \TVS. 
\end{df}

\label{def:topologicalvectorspaceboundedset}
\newcommand{\TVSBounded}[0]{
    \bf \hyperref[def:topologicalvectorspaceboundedset]{TVS-Bounded} \rm
}

\begin{df}[TVS Bounded Set]
Let $(V,\T)$ be a 
\TVS.
Let $A \subset V$. 
We say that A is \TVSBounded with respect to $\T$,
or when confusion is unlikely we simply say that A is \TVSBounded
if for every $U \in \scU_{\T}(0)$, there exists an $\alpha \in \F$
, $\alpha > 0$
, such that $A \subset \alpha U$. 
\end{df}

\label{def:boundedlinearoperatorinatvs}
\newcommand{\BLO}[0]{
    \bf \hyperref[def:boundedlinearoperatorinatvs]{Bounded Linear Operator} \rm
}

\begin{df}[TVS Bounded Linear Operator]
Let $(V_i,\T_i)$ be a \TVS over $\F \in \{\R, \C\}$ for $i \in \{0,1\}$. 
We say that a linear operator $T:(V_1, \T_1) \to (V_2, \T_2)$ is a \BLO
if for each $U \in V_1$ with U \TVSBounded with respect to $\T_0$, 
$TU$ is \TVSBounded with respect to $\T_1$. 
\end{df}

\begin{prop}[Bounded Linear Operator Continuous]
\label{prop:boundedlinearoperatorsarecontinuous}
Let $(V_i,\T_i)$ be a \TVS over a field $\F \in \{\R, \C\}$ for $i \in \{0,1\}$
Let $T:V_0 \to V_1$ be a \BLO. 
Then T is continous. 
\begin{proof} 
    Let $0_{V_1} \in U \in \T_1$. 
\end{proof} 
\end{prop}




\subsection{Seminormed Spaces}
\label{def:subadditive}
\newcommand{\Subadditive}[0]{
    \bf \hyperref[def:subadditive]{Subadditive} \rm
}

\newcommand{\Subadditivity}[0]{
    \bf \hyperref[def:subadditive]{Subadditivity} \rm
}
\begin{df}[Subadditive]
Let G be a Magma and H be a Totally ordered Magma. Represent the operation of g and h both with +. 
We call a mapping $p:G \to H$ \Subadditive if, for every $x,y \in G$, we have 
\begin{equation}
    p(x+y) \leq p(x)+p(y)
\end{equation}
Under these circumstances, we may also say that the operator p posesses $\Subadditivity$. 
\end{df}

\label{def:scalarhomogeneous}
\newcommand{\ScalarHomogeneous}[0]{
    \bf \hyperref[def:scalarhomogeneous]{Scalar Homogeneous} \rm
}

\newcommand{\ScalarHomogeneity}[0]{
    \bf \hyperref[def:scalarhomogeneous]{Scalar Homogeneity} \rm
}
\begin{df}[Scalar Homogeneous]
    Let V be a vector space over a field $\F \in \{\R, \C\}$. 
    We say that a map $p:V \to V$ is \ScalarHomogeneous, if
    , for each $\alpha \in \F$ and each $x \in V$, we have 
    \begin{equation}
        p(\alpha x) = \abs{\alpha} p(x)
    \end{equation}
    Under these circumstances, we may instead say that the operator p posesses \ScalarHomogeneity.
    
\end{df}


\label{rmk:seminorm}
\begin{rmk}[Scalar Homogeneous operator at 0 is 0]

If V is a vector space over $\mathbb{F} \in \{\R, \C\}$, then for each $x \in V$, $0x=0$.
Hence, if p is a \ScalarHomogeneous operator on v, then for any $x \in V$
\begin{equation}
p(0)=p(0x)=|0|p(x)=0p(x)=0
\end{equation}
that is, p(0)=0. 
\end{rmk}





\label{def:seminorm}
\newcommand{\Seminorm}[0]{
    \bf \hyperref[def:seminorm]{Seminorm} \rm
}\newcommand{\Seminorms}[0]{
    \bf \hyperref[def:seminorm]{Seminorms} \rm
}
\newcommand{\NonDegenerate}[0]{
	\bf \hyperref[def:seminorm]{Non-Degenerate} \rm
}
\newcommand{\Degenerate}[0]{
	\bf \hyperref[def:seminorm]{Degenerate} \rm
}
\label{def:seminormedspace}
\newcommand{\SeminormedSpace}[0]{
    \bf \hyperref[def:seminormedspace]{Seminormed Space} \rm
}
\begin{df}[Seminorm]
    Let V be a vector space over a field $\F \in \{ \R, \C\}$.  
    We say that a map $\norm{\cdot}:V \to [0,\infty)$ is a \Seminorm on V 
	if it is both \Subadditive and \ScalarHomogeneous. 
	In this case, we refer to $(V, \norm{\cdot})$ as a \SeminormedSpace. 
	We say that $\norm{\cdot}$ is \NonDegenerate if there is at least one $v \in V$ with $\norm{v}>0$. 
	We say that $\norm{\cdot}$ is \Degenerate if it is not \NonDegenerate.  
	We may also refer to the \SeminormedSpace $(V, \norm{\cdot})$ as being
	\Degenerate
	or
	\NonDegenerate. 
\end{df} 





\label{def:norm}
\newcommand{\Norm}[0]{
    \bf \hyperref[def:norm]{Norm} \rm
}
\label{def:normedspace}
\newcommand{\NormedSpace}[0]{
    \bf \hyperref[def:normedspace]{Normed Space} \rm
}
\begin{df}[Norm]
    Let $(V,\norm{\cdot})$ be a \SeminormedSpace.
    If the following implication is true for $x \in V$, then we refer to $\norm{\cdot}$ as a \Norm on V, and we call $(V, \norm{\cdot})$ a \NormedSpace.
    \begin{equation}
    x \neq 0 \implies \norm{x} \neq 0
    \end{equation}
\end{df}

\begin{prop}[Subadditive Operator On a Group Induces a Metric]
    \label{prop:subadditiveinducestriangleinequality}
    Let $(G,+, e)$ be a group and let $(H,+,\leq)$ be a totally ordered magma. 
    Let $p:G \to H$ be \Subadditive. 
    define $d:G \times G \to H$ by setting, for each $x,y \in G$, 
    \begin{equation}
        d(x,y) =  p(x+(-y))
    \end{equation}

    Then d satisfies the triangle inequality. 

    \begin{proof}
    let $x,y, z \in G$. Then
    \begin{align*}
        d(x,z) &= p(x+(-z))\\
        & = p(x+e+(-z))\\
        & = p(x+(-y)+y+(-z))\\
        & \leq p(x+(-y))+p(y+(-z))\\
        & = d(x,y)+d(y,z)
    \end{align*}
    completing the proof. 
    \end{proof} 
\end{prop}
 
\newcommand{\SeminormTopology}[0]{\textbf{\hyperref[def:seminormtopology]{Seminorm Topology}}\xspace}
\newcommand{\SeminormInducedPseudometric}[0]{\textbf{\hyperref[def:seminormtopology]{Pseudometric induced by the Seminorm}}\xspace}
\newcommand{\SeminormSpaceInducedPseudometricSpace}[0]{\textbf{\hyperref[def:seminormtopology]{Pseudometric Space induced by the Seminormed Space}}\xspace}

\begin{df}[Seminorm Topology]
\label{def:seminormtopology}
\rm
    Let $(X,\norm{\cdot})$ be a \SeminormedSpace.
    define $d_{\norm{\cdot}}:V \times V \to [0,\infty)$  by setting,
    for $x,y \in X$, 
    \begin{equation}
    d_{\norm{\cdot}}(x,y) = \norm{x-y}
    \end{equation}
    Observe the following: 
    \begin{enumerate}
        \item \ref{rmk:seminorm} guarantees that $d_{\norm{\cdot}}(x,x)=0$ for $x \in X$. 
        \item 
        \ref{prop:subadditiveinducestriangleinequality} guarantees that d satisfies the \TriangleInequality. 
        \item d is a \SymmetricMap, as we have 
    \begin{equation}
        d(x,y)_{\norm{\cdot}}=\norm{x-y}=|-1|\norm{x-y}=\norm{y-x}=d(y,x)
    \end{equation}
    \end{enumerate}

    Hence, $d_{\norm{\cdot}}$  is a \Pseudometric on X, which we call the \SeminormInducedPseudometric on X. 
    We refer to $(X, d_{\norm{\cdot}})$ as the \SeminormSpaceInducedPseudometricSpace $(X,\norm{\cdot}$. 
    We refer to the \PseudometricTopology induced by $d_{\norm{\cdot}}$ as the \SeminormTopology induced by $\norm{\cdot}$, and unless otherwise specified, when we reference $(X,\norm{\cdot})$, we consider it to be endowed with this topology. 

\end{df}

\label{def:seminormkernel}
\newcommand{\SeminormKernel}[0]{
    \bf \hyperref[def:seminormkernel]{Seminorm Kernel} \rm
}
\newcommand{\SeminormKernels}[0]{
    \bf \hyperref[def:seminormkernel]{Seminorm Kernels} \rm
}
\newcommand{\Ker}[0]{
   \bf\mathcal{K}\rm^{ernel}
}


\begin{df}[Seminorm Kernel]
Let $(V, \norm{\cdot})$ be a \SeminormedSpace. 
Define the set $\Ker_{(V,\norm{\cdot})}$ by 
\begin{equation}
\Ker_{(B,\norm{\cdot})}=\{x \in V | \norm{x}=0\}
\end{equation}
We call this set the \SeminormKernel of the space $\Ker_{(V,\norm{\cdot})}$. 
When confusion is unlikely, we may denote this set with
$\Ker$, $\Ker_V$, or even $\Ker_{\norm{\cdot}}$, or we may just refer to it
as the \SeminormKernel, the \SeminormKernel of $V$, or the \SeminormKernel of $\norm{\cdot}$. 
\end{df}

\begin{prop}[Seminorm Kernel is a vector Subspace]
\label{prop:seminormkernelisavectorsubspace}
    Let $(X,\norm{\cdot})$ be a \SeminormedSpace over a field $\F \in \{\R, \C\}$  
    with corresponding \SeminormKernel $\Ker$. 
    Then the following are true. 
    \begin{enumerate}
        \item $\Ker$ is a vector subspace of X. 
        \item $\Ker$ is closed in the \SeminormTopology on X.
        \item By part 1 of this result, $\Ker$ is a subgroup of the additive structure of $X$. Hence, we can talk about the 
    \end{enumerate}


    \begin{proof}[Proof of One]
        \Subadditivity implies that, if $x,y \in \Ker$, then $\norm{x+y} \leq \norm{x}+\norm{y}=0$. 
        By \ScalarHomogeneity, if $x \in \Ker$  and $\alpha \in \F$, $\norm{\alpha x} =|\alpha| \norm{x}=0$
        so $\Ker$ is in fact a vector subspace of X. 
    \end{proof}
    \begin{proof}[Proof of Two]
        
        If $x \in X \setminus  \Ker$
        then $\norm{x} = \alpha > 0$ for some positive $\alpha$. 
        Hence $B(x;\alpha/2)$ is an open set containing x disjoint from $\Ker$. 
       We can then write $X \setminus \Ker$ as the union of all such open sets to see that $\Ker$ is closed. 
    \end{proof}
\end{prop}

\label{def:equivalencemodseminormkernel}
\newcommand{\EquivelanceModKernel}[0]{
    \bf \hyperref[def:equivalencemodseminormkernel]{Equivalence MOD-$\Ker$} \rm
}

\newcommand{\EquivalentModKernel}[0]{
    \bf \hyperref[def:equivalencemodseminormkernel]{Equivalent MOD-$\Ker$} \rm
}
\newcommand{\SeminormKernelQuotientVectorSpace}[0]{
    \bf \hyperref[def:equivalencemodseminormkernel]{Seminorm Kernel Quotient Vector Space} \rm
}

\begin{df}[Quotient Space Mod Kernel]
Let $(X,\norm{\cdot})$ be a \SeminormedSpace over a field $\F \in \{\R,\C\}$.
with \SeminormKernel $\Ker$.
By \ref{prop:seminormkernelisavectorsubspace}, part 1, 
$\Ker$ is a vector subspace of $X's$ algebraic structure, and so if we define 
$\cong_{\Ker} \subset X \times X$ by setting, for $x,y \in X$
\begin{equation}
x \cong_{\Ker} y \iff x-y \in \Ker
\end{equation}
Then one recognizes $\cong_{\Ker}$ as \EquivelanceModKernel as would be commonly spoken of in Module or Vector Space theory. 
From this, alot of nice properties fall out. We list them here, without proof just to nail down notation. 
For proof, see any undergraduate algebra text.
\begin{enumerate}
%\item We denote the \QuotientSpace $X/\cong$ with $X/\Ker$. 
\item If $x \cong_{\Ker} y$, then we say that x and y are \EquivalentModKernel. 
\item For $x \in X$
    , we denote the \EquivalenceClass $[x]_{\cong_{\Ker}}$ with 
    $[x]_{\Ker}$ or 
    with $x+\Ker$, or 
    when confusion is unlikely, simply $[x]$. 
\item We denote $X/\cong_{\Ker}$ with $X/\Ker$. 
\item If we define $\oplus:X/\Ker \times X/\Ker \to X/\Ker$ by setting
    , for $x,y \in X$, $[x]_{\Ker}\oplus[y]_{\Ker}=[x+y]_{\Ker}$
    , then $\oplus$ is well defined and endows $X/\Ker$ with a group structure. 
\item If we further define $\odot:\F \times X/\Ker \to X/\Ker$ by 
    $\alpha [x]_{\Ker}=[\alpha x]_{\Ker}$
    , then $\pa{X/\Ker, \oplus, \odot, [0]_{\Ker}}$ is a Vector space over $\F$. 
\item Unless otherwise specified
    , when referring to the set $X/\Ker$
    , we endow it with the above vector space structure
    , and we call this space the \SeminormKernelQuotientVectorSpace 
    of the seminormed space $(X, \norm{\cdot})$.
\end{enumerate}
\end{df}


\label{prop:equivalencemodkernelispseudometricequivalence}
\begin{prop}[Equivalence Mod Kernel is Pseudometric Equivalence]
    Let $(X,\norm{\cdot})$ be a seminromed space.
    with \SeminormKernel $\Ker$.
    Let $d$ denote the \SeminormInducedPseudometric.
	Let $\cong_{d}$ denote the 
	\RelationOfZeroDistance with respect to d. 
    
    Then $\cong_{\Ker}=\cong_{d}$. 
    \begin{proof}
        Let $x,y \in X$ and let $x \cong_{\Ker}y$.
        Then, since $x-y \in \Ker$, 
        Then $d(x,y) := \norm{x-y} =0$, so $x \cong_d y$. 
        Hence $\cong_{\Ker} \subset \cong_{d}$ 


        Now let $x,y \in X$ with $x \cong_d y$. 
        Then $\norm{x-y}=d(x,y) = 0$, so $x-y \in \Ker$
        , and therefore $x \cong_{\Ker} y$. 
        Hence, $\cong_{d} \subset \cong_{\Ker}$. 

        Since inclusion goes both directions, $\cong_{\Ker} = \cong_d$.

    \end{proof} 
\end{prop}

\label{def:quotientnormspace}
\newcommand{\QuotientNorm}[0]{
    \bf \hyperref[def:quotientnormspace]{Quotient Norm} \rm
}
\newcommand{\QuotientNormedSpace}[0]{
    \bf \hyperref[def:quotientnormspace]{Quotient Normed Space} \rm
}

\begin{df}[Quotient Norm Space]
Let $(X,\norm{\cdot})$ be a \SeminormedSpace
with \SeminormInducedPseudometric $d$, 
\SeminormKernel $\Ker$, and
\SeminormKernelQuotientVectorSpace $X/\Ker$.
Let $\tilde{d}:X/\Ker \times X/\Ker \to [0,\infty)$ be the \MetricInducedByPseudometric.

Define $\norm{\cdot}_{\Ker} : X/\Ker \to [0,\infty)$ by 
\begin{equation}
\norm{[x]}_{\Ker} = \tilde{d}([x], [0])
\end{equation}

By $\ref{prop:quotientnormspace}$, $(X/\Ker, \norm{\cdot}_{\Ker})$ is a normed space which we call the \QuotientNormedSpace of $(X,\norm{\cdot})$, and we call $\norm{\cdot}_{\Ker}$ the \QuotientNorm. 
Whenever we refer to $X/\Ker$, unless otherwise specified, we endow it with this norm and the topology generated by this norm.
Furthermore, whenever we consider $X/\Ker$, unless otherwise specified, we consider it as 
possesing the topology generated by the norm $\norm{\cdot}_{\Ker}$. 
\end{df}

\begin{prop}[Quotient Normed Space]
\label{prop:quotientnormspace}

Let $(X,\norm{\cdot})$ be a \SeminormedSpace
with \SeminormInducedPseudometric $d$, 
\SeminormKernel $\Ker$, and
\SeminormKernelQuotientVectorSpace $X/\Ker$.
Let $\tilde{d}:X/\Ker \times X/\Ker \to [0,\infty)$ be the \MetricInducedByPseudometric.
Let $T:X \to X/\Ker$ denote the \QuotientMap of X into $X/\Ker$ 
(Recalling that the 
\RelationOfEqualNeighborhoodFilters equals the 
\RelationOfZeroDistance equals the relation of 
\EquivelanceModKernel), so they would all produce the same quotient map)
Let $\norm{\cdot}_{\Ker}$ denote the \QuotientNorm.

The following are true. 
\begin{enumerate}
\item $\norm{\cdot}_{\Ker}$ is a norm on $X/\Ker$. 
\item $\tilde{d}$  is the \SeminormInducedPseudometric $\norm{\cdot}_{\Ker}$, and thus they produce the same topology. 
\item T has all of the properties described in $\ref{prop:QuotientSpaceTopology}$. 
\item T is Linear.
\item T is Surjective. 
\item T is an isometry. 
\item T is injective if and only if $\norm{\cdot}$ is a norm. 
\begin{proof}[Proof of 1]
    First, note that 
    $Range(\norm{\cdot}_{\Ker}) \subset Range(\tilde{d}) \subset [0,\infty)$,\
    so that $\norm{\cdot}_{\Ker}$ has the correct domain and codomain. 
    For \Subadditivity, let $[x],[y] \in X/\Ker$. Then 
    \begin{align*}
        \norm{[x]+[y]}_{\Ker}& = \norm{[x+y]}_{\Ker}\\
        & = \tilde{d}\pa{[x+y], [0]}\\
        & = d(x+y, 0)\\
        & = \norm{x+y} \\
        & \leq \norm{x}+\norm{y}\\
        & = d(x,0)+d(y,0)\\
        & = \tilde{d}\pa{[x],[0]}+ \tilde{d}\pa{[y],[0]}\\
        & = \norm{[x]}_{\Ker}+\norm{[y]}_{\Ker}
    \end{align*}
    For \ScalarHomogeneity, let $\alpha \in \F$ and $[x] \in X/\Ker$. 
    Then, 
    \begin{align*}
        \norm{[\alpha x]}_{\Ker} & = \tilde{d}\pa{[\alpha x], [0]}\\
        & = d(\alpha x, 0) \\
        & = \norm{\alpha x}\\
        & = \abs{\alpha} \norm{x} \\
        & = \abs{\alpha} \norm{[x]}_{\Ker}
    \end{align*}
    Finally, suppose $[x] \neq 0$. 
    Then, since the additive identity of $X/\Ker$ is $\Ker$, $x \not \in \Ker$. 
    Hence $\norm{[x]}_{\Ker} = \tilde{d}([x], 0) = d(x,0) =\norm{x} > 0$. 

\end{proof}
\begin{proof}[Proof of 2] 
Let $D$ denote the \SeminormInducedPseudometric $\norm{\cdot}_{\Ker}$. 
Then, for $[x], [y] \in X/\Ker$, 
\begin{align*}
\tilde{d}([x], [y]) & = d(x,y)\\
& = \norm{x-y}\\
& = \norm{x-y-0}\\
& = d(x-y, 0)\\
& = \tilde{d}([x-y],0)\\
& = \norm{[x-y]}_{\Ker}\\
& = \norm{[x]-[y]}_{\Ker}
& = D\pa{[x], [y]}
\end{align*}
Since these two \Pseudometric's are equal, they produce the same topology. 
Furthermore, by applying \ref{prop:pseudometricinducedmetric}, we see that the 
topology generated by $\norm{\cdot}_{\Ker}$ is also the \QuotientSpaceTopology on $X/\Ker$. 
\end{proof}
\begin{proof}[Proof of 3]
T is the topological \QuotientMap and the norm topology is the \QuotientSpaceTopology, so the assumptions of $\ref{prop:QuotientSpaceTopology}$ are satisfied. 
\end{proof} 
\begin{proof}[Proof of 4] 
Let $x,y \in X$ and $\alpha \in \F$. Then 
\begin{align*}
T(\alpha x + y) & = [\alpha x + y] \\
& = (\alpha x + y ) + \Ker\\
& =\alpha \pa{x+\Ker} + \pa{y+ \Ker}\\
& = \alpha [x] + [y]\\
& = \alpha T(x) + T(y) 
\end{align*}
\end{proof}
\begin{proof}[Proof of 5] 
Direct consequence of \ref{prop:QuotientMapSurjective}
\end{proof}
\begin{proof}[Proof of 6] 
Consequence of part 2 of this result combined with 
\end{proof}
\begin{proof}[Proof of 7] 
If $\norm{\cdot}$ is a \Norm, then $\Ker={0}$, so 
$Tx=Ty \implies T(x-y) =0 \implies x-y \in \Ker \implies x-y=0 \implies x=y$. 
\end{proof}

\end{enumerate} 

\end{prop} 

\begin{prop}
\label{prop:quotientspreservecompleteness}
Let $(X,\norm{\cdot})$ be a \SeminormedSpace with \QuotientNormedSpace $(X/\Ker, \norm{\cdot}_{\Ker})$. 

Then X is \PseudometricComplete if and only if $X/\Ker$ is complete. 

\begin{proof}
Let X be \PseudometricComplete. 
Let $\{[x_i]\}_{i \in \N} \subset X/\Ker$ be a \PseudometricCauchySequence. 
Let $\epsilon > 0$. 
Then there is an $N \in \N$ such that for $m,n > N$ we have 
\begin{equation}
\norm{[x_m-x_n]}_{\Ker} < \epsilon
\end{equation}

For this N, we have 
\begin{equation}
\norm{x_m-x_n} = \norm{[x_m-x_n]}_{\Ker} < \epsilon
\end{equation}
so that $\{x_i\}_{i \in \N}$ is a \PseudometricCauchySequence. 
Since X is \PseudometricComplete, 
there is a 
$x \in X$ such that $\norm{x_i-x} \to 0$, 
but since T is an isometry, 
\begin{equation}
\norm{[x]-[x_i]}=\norm{[x_i-x]}_{\Ker} \to 0
\end{equation}
and so 
$[x_i] \to [x]$.
so that $X/\Ker$ is complete. 

Now suppose instead that $X/\Ker$ is complete 
and suppose $\{x_i\}_{i \in \N}$ is a \PseudometricCauchySequence in X. 
Since $\norm{[x_i-x_j]}_{\Ker} = \norm{x_i-x_j}$, 
$\{[x_i]\}_{i \in \N}$ is a \PseudometricCauchySequence in $X/\Ker$, which therefor has a 
limit $y \in X/\Ker$. Since T is surjective, $y=[x]$ for some $x \in X$, and it is easy to see that
$x_i \to x$ so that $X$ is \PseudometricComplete. 

\end{proof}
\end{prop}

\label{def:BLO} 
\newcommand{\SpaceOfBoundedLinearOperators}[0]{ 
    \bf \hyperref[def:BLO]{Space of Bounded Linear Operators} \rm
}
\newcommand{\OperatorSeminorm}[0]{
    \bf \hyperref[def:BLO]{Operator Seminorm} \rm
}
\newcommand{\OperatorNorm}[0]{
    \bf \hyperref[def:BLO]{Operator Norm} \rm
}
\begin{df}[Space of Continuous Linear Operators From a Seminormed Space into a Normed Space]
Let $(X,\norm{\cdot}_X)$ be a \NonDegenerate \SeminormedSpace.
Let $(Y, \norm{\cdot}_Y)$ be a \SeminormedSpace.
We denote with $BL\pa{(X,\norm{\cdot}_X), (Y, \norm{\cdot}_Y)}$ 
the collection of
\Continuous
\Linear
operators
$T:(X, \norm{\cdot}_X) \to (Y, \norm{\cdot}_Y)$. 
When the topologies on X and Y are understood, we denote this set with
$BL\pa{X,Y}$. 
We refer to $BL\pa{X,Y}$ as the \SpaceOfBoundedLinearOperators 
from $(X, \norm{\cdot}_X)$ to $(Y, \norm{\cdot}_Y)$ 
, or when $\norm{\cdot}_X$ and $\norm{\cdot}_Y$ are understood, 
from X to Y. 

We endow $BL\pa{X,Y}$ with the algebraic operations
of pointwise scalar multiplication
and pointwise addition, making $BL\pa{X,Y}$ a vector space. 

We define $\norm{\cdot}:BL(X,Y) \to [0,\infty)$ by defining, 
for $T \in BL(X,Y)$
\begin{equation}
    \norm{T} = \sup\limits_{\norm{x}_X \neq 0} \frac{\norm{Tx}_Y}{\norm{x}_X}
\end{equation}
As will be proven in \ref{prop:BLO}, $\norm{\cdot}$ is a \Seminorm on $BL(X,Y)$, which 
we refer to as the \OperatorSeminorm on $BL(X,Y)$. induced by the
\Seminorm $\norm{\cdot}_X$ on X and the \Seminorm $\norm{\cdot}_Y$ on Y. 

In the case that $\norm{\cdot}_{Y}$ is a \Norm, rather than just a \Seminorm, by \ref{prop:BLO}
, $\norm{\cdot}$ is a \Norm on $BL(X,Y)$, which we instead call the \OperatorNorm. 
\end{df}

\begin{prop} 
\label{prop:BLO} 
Let $(X,\norm{\cdot}_X)$ be a \SeminormedSpace. 
Let $(Y, \norm{\cdot}_Y)$ be a \SeminormedSpace.
Let $BL(X,Y)$ denote the \SpaceOfBoundedLinearOperators from X to Y. 
Let $\norm{\cdot}$ denote the \OperatorSeminorm. 

The following are true. 
\begin{enumerate}
%For Item 1, may have to prove result connecting 
%pseudometric topology continuity to $\epsilon-delta$ cotninuity wrt the pseudometric. 
\item $\norm{\cdot}$ is in fact a well-defined \Seminorm on $BL(X,Y)$. 
\item If $\norm{\cdot}_Y$ is a \Norm, then so is $\norm{\cdot}$. 
\item $\norm{\cdot}$ is \NonDegenerate if and only if Y is. 
\item $BL(X,Y)$ is complete if and only if Y is. 
\item Convergence of a sequence $\{T_i\}_{i \in \N} \subset BL(X,Y)$ with respect to $\norm{\cdot}$
is equivalent to the following condition: $T_ix \to Tx$ uniformly for $x \in B_X(0;1)$. 
\end{enumerate}


\begin{proof}[Proof of 1] 
    Since X is nondegenerate, there exists at least 1 $x \in X$ with $\norm{x}_X \neq 0$, 
    so for each $T \in BL(X,Y)$, the set that the supremum is being taken over is nonempty.
    Also, it is clear that $Range(\norm{\cdot}) \subset [0,\infty)$, 

    For \Subadditivity, let $T_i \in BL(X,Y)$ for $i \in \{0,1\}$. and $x \in X$ with $\norm{x} > 0$.
    Then, since $\norm{\cdot}_Y$ is \Subadditive, 
    \begin{align*}
    \frac{\norm{(T_0+T_1)x}_Y}{\norm{x}_X} \leq \frac{\norm{T_0x}_Y}{\norm{x}_X}+ \frac{\norm{T_1x}_Y}{\norm{x}_X}
    \end{align*}
    Since this is true for each x with $\norm{x}_X \neq 0$, taking the supremum of each side yields

    \begin{align*}
    \sup\limits_{\norm{x}_X \neq 0} \pa{\frac{\norm{(T_0+T_1)x}_Y}{\norm{x}_X}} & \leq\sup\limits_{\norm{x}_X \neq 0} \pa{ \frac{\norm{T_0x}_Y}{\norm{x}_X}+ \frac{\norm{T_1x}_Y}{\norm{x}_X}}\\
& \leq\sup\limits_{\norm{x}_X \neq 0} \pa{ \frac{\norm{T_0x}_Y}{\norm{x}_X}} + \sup\limits_{\norm{x}_X \neq 0} \pa{\frac{\norm{T_1x}_Y}{\norm{x}_X}}\\
    \end{align*}
    Hence, $\norm{T_0+T_1} \leq \norm{T_0}+\norm{T_1}$ so that $\norm{\cdot}$ is \Subadditive. 
    For \ScalarHomogeneity, let $T \in BL(X,Y)$, $\alpha \in \F$, and $x \in X$ with $\norm{x}_X \neq 0$. 
    Then 
    \begin{align*}
        \frac{\norm{(\alpha T)x}_Y}{\norm{x}_X} = \frac{\norm{\alpha (Tx)}_Y}{\norm{x}_X} = \abs{\alpha} \frac{\norm{Tx}_Y}{\norm{x}_X}
    \end{align*}
    Hence taking the supremum finishes the proof.
\end{proof}
\begin{proof}[Proof of 2] 
   Let $T \neq 0 \in BL(X,Y)$. Then for some $x \in X$, $Tx \neq 0$. 
   Then $Tx$ has a neighborhood U disjoint from $0_Y$, 
   Hence $x \in T^{-1}(U)$ but not $0_X \in T^{-1}(U)$, since $T0_X = 0_Y$.
   Since U is a neighborhood of x disjoint from 0, 
   there is an $\epsilon > 0$ such that $0_X \subset \complement U \subset \complement B_X(x;\epsilon)$,
   and therefore $\norm{x}_X > \epsilon$. 
   Since $\norm{x}_X > 0$, it is ranged over in the supremum defining $\norm{T}$, and so
   \begin{equation}
   0 < \frac{\norm{Tx}_Y}{\norm{x}_X} \leq \sup\limits_{\norm{x}_X \neq 0} \frac{\norm{Tx}_X}{\norm{x}_X}=\norm{T}
   \end{equation}
\end{proof}
\begin{proof}[Proof of 3] 
\end{proof}
\begin{proof}[Proof of 4] 
\end{proof}

\end{prop}


since K is a vector subspace, $X/\norm{\cdot}=\{x+K: x \in X\}$.
Hence $X/K$ is clearly a normed space with norm $\norm{[x]} = \norm{x}$, and $X/K$ will preserve completeness and incompleteness of X. 

\begin{prop}[Seminorm Linear Operators]
    \label{prop:seminormlinearoperators}
    Let $(X,\norm{\cdot}_X)$ be a seminormed space and $(Y,\norm{\cdot}_Y)$ a normed space.
    For each continuous linear operator $T:X \to Y$, define 
    \begin{equation}
        \norm{T} = \sup\limits_{\norm{x}_X \neq 0} \frac{\norm{Tx}_Y}{\norm{x}_X}
    \end{equation}
    Then $\norm{\cdot}$ is a norm on the space of continuous linear operators from X to Y, which we shall denote with $BL(X,Y)$. 
    Further, if $Q:BL(X,Y) \to BL(X/\norm{\cdot}_X^{-1}\{0\},Y)$ is defined by $QT[x]=Tx$, then Q is a well defined linear bijective isometry. 
    \begin{proof}
    \end{proof} 
\end{prop} 
As a consequence of the above proposition, even if $X$ is just a seminormed space, $X^*$ is still a Banach space which is isomorphic to $(X/K)^*$, implying the possibility of extending several results known for normed spaces into the context of seminorms. Further, since $X/K$ can be embedded into X, several existence results, such as Helly's theorem, can also be generalized to the case of a seminormed space. In the context of a seminormed space, the canonical embedding $c:X \to X^{**}$ ceases to be injective but remains a linear isometry, and we shall continue to use the nomenclature that X is \bf reflexive \rm if $c$ is surjective. Since $X^{**}=(X/\norm{\cdot}^{-1}\{0\})^{**}$, it is no surprise that $c(X)=c(X/\norm{\cdot}^{-1}\{0\})$. Hence, a seminormed space is reflexive if and only if it's induced normed space is reflexive.  The weak topology on a set X induced by set of mappings $\{\phi_\alpha:X \to Y_\alpha\}$ where each $Y_\alpha$ is a topological space is the coarsest topology on X which makes each $\phi_\alpha$ continuous. Similar to in the context of a normed space, if X is a seminormed space, we define the weak topology on X to be the topology on X generated by $X^*$, and the $weak^*$ topology on $X^*$ to be the topology generated by $c(X)$.
Before moving on to the classical theory revamped, I present on more useful result about weak topologies of seminormed spaces. 
\begin{prop}[Weak Quotients]
    \label{prop:weakquotients}
    Let X be a seminormed space and $\{Y_\alpha\}_{\alpha \in A}$ be a collection of topological spaces. For each $\alpha \in A$ let $\phi_\alpha:X \to Y_\alpha$ have the property that for every $x,y \in X$, for every $\alpha \in A$, $\norm{x-y}=0 \implies \phi_\alpha(x)=\phi_\alpha(y)$. 
    For each $\alpha \in A$, define $\tilde{\phi}_\alpha:X/\norm{\cdot}^{-1}\{0\} \to Y_\alpha$ by
    $\tilde{\phi}_{\alpha}[x] = \phi_\alpha x$. Let $\T_w$ denote the weak topology on X induced by $\{\phi_\alpha\}_{\alpha \in A}$, and $\T_{\tilde{w}}$ denote the weak topology on $X/\norm{\cdot}^{-1}\{0\}$ induced by $\{\tilde{\phi}_{\alpha}\}_{\alpha \in A}$. Then 
    \begin{equation}
        (X,\T_w)/\norm{\cdot}^{-1}\{0\} = (X/\norm{\cdot}^{-1}\{0\}, \T_{\tilde{w}})
    \end{equation}
    \begin{proof}
    \end{proof} 
\end{prop} 


Finally, before we move on, recall that if $X,Y$ are Topological vector spaces, we can topologize the set of continuous linear operators from X to Y, denoted $BL(X,Y)$ by saying that $\{T_\alpha\}_{\alpha \in A} \subset BL(X,Y)$ converges to $T \in BL(X,Y)$ if there is a neighborhood U of 0 in X such that $T_{\alpha}x \to Tx$ uniformly for $x \in U$. 