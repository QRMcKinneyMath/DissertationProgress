\chapter{Non-Hausdorff Analysis}
\section{Pseudometrics}
\subsection{Introduction}
\label{def:TriangleInequality}
\newcommand{\TriangleInequality}[0]{
    \bf \hyperref[def:TriangleInequality]{Triangle Inequality} \rm
}
\begin{df}[Symmetric Map]
    
    Let X be a set and $(Y,+, \leq)$ be a totally ordered magma.
    We say that a map $f:X \times X \to Y$ satisfies the \TriangleInequality if for each $x_0,x_1,x_3 \in X$, we have
    \begin{equation*}
        f(x_0,x_2) \leq  f(x_0,x_1)+f(x_1,x_2)
        \end{equation*}
\end{df} 
\newcommand{\Pseudometric}[0]{\textbf{\hyperref[def:pseudometric]{Pseudometric}}\xspace}
\newcommand{\Pseudometrics}[0]{\textbf{\hyperref[def:pseudometric]{Pseudometrics}}\xspace}
\newcommand{\PseudometricSpaces}[0]{\textbf{\hyperref[def:pseudometric]{Pseudometric Spaces}}\xspace}
\newcommand{\PseudometricSpace}[0]{\textbf{\hyperref[def:pseudometric]{Pseudometric Space}}\xspace}
\begin{df}[\Pseudometric]
\label{def:pseudometric}
\rm
    Let $X$ be a nonempty set.
    Let $d:X \times X \to [0,\infty)$ be \CommutativeFunction, 
    satisfy the \TriangleInequality, and for each $x \in X$, 
    \begin{equation*}
        d(x,x) = 0
    \end{equation*}
    Then we call d a \Pseudometric on X and we call $\pa{X,d}$ a \PseudometricSpace.
    \end{df} 
	
	
	
\newcommand{\Metric}[0]{\textbf{\hyperref[def:metric]{Metric}}\xspace}
\newcommand{\MetricSpace}[0]{\textbf{\hyperref[def:metric]{Metric Space}}\xspace}
\begin{df}[\Metric]
\label{def:metric}
\rm
	Let $(X,d)$ be a \PseudometricSpace. 
	If d has the property that for
	$x,y \in X$, if $x \neq y$, then
	\begin{equation*}
		d(x,y) \neq 0
	\end{equation*}
	Then we call d a \Metric on X 
	and we call $(X,d)$ a
	\MetricSpace
\end{df}

\newcommand{\Isometry}[0]{\textbf{\hyperref[def:isometry]{Isometry}}\xspace}
\newcommand{\Isometries}[0]{\textbf{\hyperref[def:isometry]{Isometries}}\xspace}
\newcommand{\Isometric}[0]{\textbf{\hyperref[def:isometry]{Isometric}}\xspace}
\newcommand{\Isometrically}[0]{\textbf{\hyperref[def:isometry]{Isometrically}}\xspace}
\begin{df}[\Isometry]
\label{def:isometry}
\rm
    For $i \in \{0,1\}$, 
    let $(X_i,d_i)$ be \PseudometricSpaces.
    Let $f:X_0 \to X_1$ satisfy, for each $x,y \in X_0$, 
    \begin{equation*}
    d_0(x,y) = d_1\pa{f(x_0), f(x_1)}
    \end{equation*}
    Then we call $f$ an \Isometry between $X_0$ and $Range(f)$, 
    we say that
    $(X_0, d_0)$ and $(X_1,d_1)$ are 
    \Isometric, 
    and we say that $f$ operates \Isometrically.
\end{df}

\subsection{Cauchy Sequences, Convergence, and Completeness}
\newcommand{\PseudometricCauchySequence}[0]{\textbf{\hyperref[def:pseudometriccauchysequence]{Pseudometric Cauchy Sequence}}\xspace}
\begin{df}[Pseudometric Cauchy Sequence]
\label{def:pseudometriccauchysequence}
\rm
    Let $(X,d)$ be a \PseudometricSpace.
    We say that a sequence 
	$\{x_i\}_{i \in \N}$ is a 
	\PseudometricCauchySequence
    if, for each 
	$\epsilon > 0$, 
	there exists
	$N \in \N$
	such that for 
    each pair 
	$m,n \in \N$ 
	such that 
	$m>N$ 
	and 
	$n>N$, we have 
    \begin{equation*}
        d(x_m,x_n) < \epsilon
    \end{equation*}
\end{df}


\label{def:pseudometricsequenceconvergence}
\newcommand{\PseudometricConvergence}[0]{
    \bf \hyperref[def:pseudometricsequenceconvergence]{Pseudometric-Convergence} \rm
}
\newcommand{\PseudometricConvergent}[0]{
    \bf \hyperref[def:pseudometricsequenceconvergence]{Pseudometrically-Convergent} \rm
}
\newcommand{\PseudometricConverges}[0]{
    \bf \hyperref[def:pseudometricsequenceconvergence]{Pseudometric-Converges} \rm
}
\begin{df}[Pseudometric Convergence]
    Let $(X,d)$ be a \PseudometricSpace.
	Let $\{x_i\}_{i \in \N}$ be a sequence in $(X,d)$.
    Let $x_0 \in X$.  
    We say that 
	$\{x_i\}_{i \in \N}$ 
	exhibits 
	\PseudometricConvergence 
	to 
	$x_0$ 
	in d,
	or we say that 
	$\{x_i\}_{i \in \N}$  
	\PseudometricConverges 
	to 
	$x_0$ 
	in d, 
	or we say that 
	$\{x_i\}_{i \in \N}$ 
	is 
	\PseudometricConvergent 
	to 
	$x_0 \in d$ 
	if, 
    for every 
	$\epsilon > 0$, 
	there is an 
	$N \in \N$ 
	such that for every 
	$n>N$, 
	we have 
    \begin{equation}
        d(x_0, x_n) < \epsilon
    \end{equation}
\end{df}
\begin{prop}[Convergent Implies Cauchy]
\label{prop:pseudometricconvergenceimpliespseudometriccauchy}

    Let $(X,d)$ be a
    \PseudometricSpace.
    Let $\{x_i\}_{i \in \N}$ be a 
    \PseudometricConvergent sequence. 
    Then $\{x_i\}_{i \in \N}$
    is a \PseudometricCauchySequence.

    \begin{proof}
        Since $\{x_i\}$ converges, let 
        $x_i \to x$. 
        Let $\epsilon > 0$. 
        Then there exists $N \in \N$ 
        such that for $n>N$, we have
        $d(x_i, x) < \frac{\epsilon}{2}$. 
        For this N, if $m,n > N$, then we have 
        \begin{equation}
        d(x_m,x_n) \leq d(x_m,x) + d(x,x_n) < \epsilon
        \end{equation}
        and so the sequence is a
        \PseudometricCauchySequence, as advertised. 
    \end{proof}
\end{prop}

\label{def:uniformlycauchy}
\newcommand{\UniformlyCauchy}[0]{
    \bf \hyperref[def:uniformlycauchy]{Uniformly Cauchy} \rm
}
\begin{df}[Uniformly Cauchy]
	Let $(X_\alpha, d_\alpha)$ be a \PseudometricSpace
	for $\alpha \in A$ where A is some indexing set. 
	For each $\alpha \in A$
	, let $\phi_\alpha :=\{x_i^\alpha\}_{i \in \N} \subset X_{\alpha}$
	be a sequence. 
	We say that the collection $\{\phi_\alpha\}_{\alpha \in A}$ 
	is 
	\UniformlyCauchy if for each $\epsilon > 0$, there exists an 
	$N \in \N$ such that for each pair $m,n \in N$
	such that $m>N$ and $n>N$, and for each $\alpha \in A$, 
	we have 
	\begin{equation}
	d_{\alpha} \pa{x^{\alpha}_n, x^{\alpha}_m} < \epsilon
	\end{equation}
\end{df}

\newcommand{\UniformlyConvergent}[0]{\textbf{\hyperref[def:uniformlyconvergent]{Uniformly Convergent}}\xspace}
\newcommand{\ConvergesUniformly}[0]{\textbf{\hyperref[def:uniformlyconvergent]{Converges Uniformly}}\xspace}
\newcommand{\UniformConvergence}[0]{\textbf{\hyperref[def:uniformlyconvergent]{Uniform Convergence}}\xspace}
\begin{df}[\UniformConvergence]
\label{def:uniformlyconvergent}
\rm
    Let $A$ be a nonempty set.
    For each $\alpha \in A$, 
	let $(X_\alpha, d_\alpha)$ be a \PseudometricSpace
	and let $\phi_\alpha :=\{x_i^\alpha\}_{i \in \N} \subset X_{\alpha}$
	be a \Sequence in $X_\alpha$. 
	We say that the collection $\{\phi_\alpha\}_{\alpha \in A}$ 
    is \UniformlyConvergent to 
    $\{x_\alpha\}_{\alpha \in A} \in \prod\limits_{\alpha \in A} X_\alpha$
    if for each $\epsilon > 0$, 
    there is an $N \in \N$
    such that for each $n>N$, 
    and for every $\alpha \in A$, 
    we have 
    \begin{equation*}
        d_\alpha(x^{\alpha}_i,x_\alpha) < \epsilon
    \end{equation*}

    In this scenario, we may equivalently say that
    $\{\phi_\alpha\}$ demonstrates \UniformConvergence
    to $\{x_\alpha\}_{\alpha \in A}$ 
    or that it \ConvergesUniformly
    to $\{x_\alpha\}_{\alpha \in A}$.
    When we mention \UniformConvergence without
    specifying the limit, we are only claiming that one exists.
\end{df}

\begin{prop}[Uniform Cauchy and Pointwise Convergence implies Uniform Convergence]
\label{prop:uniformlycauchyplusconvergenceimpliesuniformconvergence}
\rm
    Let $A$ be a nonempty set.
    For each $\alpha \in A$, 
	let $(X_\alpha, d_\alpha)$ be a \PseudometricSpace
	and let $\phi_\alpha :=\{x_i^\alpha\}_{i \in \N} \subset X_{\alpha}$
	be a \Sequence. 
    Suppose the collection $\{\phi_\alpha\}_{\alpha \in A}$ 
    is \UniformlyCauchy
    and that each $\phi_\alpha$ 
    is \PseudometricConvergent
    , say $x_i^{\alpha} \to x_\alpha$. 
    Then $\{\phi_\alpha\}_{\alpha \in A}$
    is \UniformlyConvergent
    to $\{x_\alpha\}_{\alpha \in A}$. 
    \begin{proof}
        Let $\epsilon > 0$. 
        Then, since $\{\phi_\alpha\}_{\alpha \in A}$ 
        is \UniformlyCauchy, 
        there is an 
        $N \in \N$
        such that 
        for $m, n>N$, we have
        $d_{\alpha}(x^{\alpha}_n,x^{\alpha}_m) < \frac{\epsilon}{2}$. 
        Since each $\phi_\alpha$ 
        \PseudometricConverges to
        $x_\alpha$, 
        there are $N_{\alpha} \in \N$. 
        such that for any $n_\alpha > N_\alpha$, 
        we have
        $d_\alpha(x^{\alpha}_{n_\alpha}, x_\alpha) < \frac{\epsilon }{2}$. 
        For each $\alpha \in A$, define 
        $M_{\alpha}=max(N+1, N_{\alpha}+1)$.
        Let $n>N$. 
        Then, for any $\alpha \in A$, we have. 
        \begin{align*}
            d_{\alpha}(x_n^{\alpha} , x_\alpha) & \leq d_{\alpha}(x_n^{\alpha} , x^{\alpha}_{M_{\alpha}}) + d_{\alpha}(x_{M_{\alpha}}, x_\alpha)\\
            & < \frac{\epsilon}{2}+\frac{\epsilon}{2} \\
            & = \epsilon
        \end{align*}
        completing the proof. 



    \end{proof} 
\end{prop}


\label{def:pseudometriccomplete}
\newcommand{\PseudometricComplete}[0]{
    \bf \hyperref[def:pseudometriccomplete]{Pseudometric-Complete} \rm
}
\begin{df}[Pseudometric Complete]
    We say that a \PseudometricSpace $(X,d)$ is 
    \PseudometricComplete if each \PseudometricCauchySequence sequence in $(X,d)$ \PseudometricConverges to a limit in $X$. 
    \end{df}
\subsection{Pseudometric Topologies}
\label{def:pseudometricball}
\newcommand{\OpenBall}[0]{
    \bf \hyperref[def:pseudometricball]{Open Ball} \rm
}
\newcommand{\ClosedBall}[0]{
    \bf \hyperref[def:pseudometricball]{Closed Ball} \rm
}
\begin{df}[Pseudometric Ball]
    Let $(X,d)$ be a \PseudometricSpace. 
    For each $x_0  \in X$ and each $\epsilon > 0$, we define the following.
    \begin{enumerate}
        \item  $B_d(x_0, \epsilon) := \{y \in X | d(x_0,y) < \epsilon\}$ denotes the \OpenBall about $x_0$ with radius $\epsilon$. 
    \item $\overline{B_d}(x_0,\epsilon) := \{y \in X | d(x_0,y) \leq \epsilon \}$ denotes the \ClosedBall about $x_0$ with radius $\epsilon$. 
    \end{enumerate} 
    
     
    \end{df} 
\newcommand{\PseudometricTopology}[0]{\textbf{\hyperref[def:pseudometrictopology]{Pseudometric Topology}}\xspace}
\newcommand{\PseudometricInducedTopology}[0]{\textbf{\hyperref[def:pseudometrictopology]{Pseudometric Topology}}\xspace}

\begin{df}[\PseudometricTopology]
\label{def:pseudometrictopology}
\rm
    Let $(X,d)$ be a \PseudometricSpace, and let $\scB$ be the set of \OpenBall's in $(X,d)$. 
    By \ref{prop:pseudometrictopology}, $\scB$, with the addition
	of $\emptyset$, is the \TopologyBasis for a unique \Topology $\T_d$ on $X$. 
    We call $\T_d$ the \PseudometricInducedTopology induced by $d$ on $X$. 
\end{df}

\label{prop:pseudometrictopology}
\begin{prop}[Pseudometric Topology]
    Let $(X,d)$ by  \PseudometricSpace and let $\scB$ be the set of \OpenBall's in $(X,d)$. 
    The following are true. 
    \begin{enumerate}
        \item There exists a unique topology $\T_d$ on X which $\scB$ is a basis of. That is, the \PseudometricTopology $\T_d$ is well defined. 
        \item The \PseudometricInducedTopology is first countable. That is, each of its points permits a countable neighborhood basis. 
    \end{enumerate}
    \begin{proof}[Proof of 1]
        Uniqueness is guaranteed by closure under arbitrary unions of a topology. 
        For existense, it is sufficient to show that the collection of arbitrary unions
        of elements of $\scB$ is closed under finite intersections. 
        Suppose that for $1\leq i \leq n$, we have $\{U_{\alpha_i} | \alpha_i \in A_i\} \subset \scB$
        and consider the set
        \begin{equation}
            U=\bigcap_{i=1}^n \bigcup_{\alpha_i \in A_i} U_{\alpha_i}
        \end{equation}
        Let $x_0 \in U$. 
        For each $i \in \{1, ..., n\}$, there exists $\alpha_i \in A_i$ such that 
        \begin{equation}
            x_0 \in U_{\alpha_i} = B_d(x_i; \epsilon_i)
        \end{equation}
        For each $i \in \{1, ..., n \}$, define $\delta_i = d(x_0, x_i)$. Then $0 < \delta_i < \epsilon_i$. 
        Then, for each $i \in \{1, ..., n \}$, 
        \begin{equation}
            B_d(x_0; \epsilon_i-\delta_i) \subset U_{\alpha_i} \subset \bigcup_{\alpha_i \in A_i} U_{\alpha_i}
        \end{equation}
        Define 
        \begin{equation}
            \delta_{x_0} = \min\limits_{i=1}^n \pa{ \epsilon_i-\delta_i}
        \end{equation}
        Then $x_0 \in B(x_0; \delta_{x_0} ) \subset U$. 
        If $U=\{x_{\alpha} | \alpha \in A\}$, then the arbitrary nature of $x_0$ above means 
        we can repeat this construction, writing 
        \begin{equation}
            U \subset \bigcup_{\alpha \in A} B(x_{\alpha} ; \delta_{x_{\alpha}} )\subset \bigcup_{\alpha \in A} U = U
        \end{equation}
        Hence, $U \in B$ and the proof is complete. 
    \end{proof}
    \begin{proof}[Proof of 2]
        Let $x_0 \in X$. 
        I claim that 
        \begin{equation}
            \scB_{x_0}:= \left\{ B_d\pa{x_0; \frac{1}{n}} | n \in \N\right\}
        \end{equation}
        is a neighborhood basis for $(X,\T_d)$ at $x_0$. 
        Let $U \in \scU_{\T_d}(x)$ be open in $\T_d$. 
        Since $\scB$ is a basis for $\T_d$, for some $y0 \in X$ and $\epsilon > 0$, 
        $x_0 \in B_d(y_0; \epsilon) \subset U$. 
        Let $\delta = d(x_0, y_0)$. Then $\epsilon - \delta > 0$. 
        Define
        \begin{equation}
            n = \ceil{ \frac{1}{\epsilon - \delta}}
        \end{equation}
        Then we have 
        \begin{equation}
            B_d\pa{x_0 ; \frac{1}{n}} \subset B_d(x_0 : \epsilon - \delta) \subset B(y_0 ; \epsilon) \subset U
        \end{equation}
    \end{proof}
\end{prop}

\subsection{Quotients Of Pseudometric Spaces}
\label{def:relationofzerodistance}
\newcommand{\RelationOfZeroDistance}[0]{
    \bf \hyperref[def:relationofzerodistance]{Relation Of Zero Distance} \rm
}
\begin{df}[Relation Of Zero Distance]
    Let $(X,d)$ be a \PseudometricSpace. 
    Define the relation  $\cong_d$ on $X \times X$ by setting, for $x,y \in X$, 
    \begin{equation}
        x \cong_d y \iff d(x,y) = 0
    \end{equation}
    We call $\cong_d$ the \RelationOfZeroDistance on $(X,d)$. 
\end{df} 
\begin{prop}[Relation Of Zero Distance is the Relation Of Equal Neighborhood Filters]
    \label{prop:relationofzerodistance}
    \rm
    Let $(X,d)$ be a \PseudometricSpace.
    Let $\cong_{\T_d}$ be the \RelationOfEqualNeighborhoodFilters $(X,\T_d)$. 
    Let $\cong_d$ be the \RelationOfZeroDistance on $(X,d)$. 
    Then $\cong_{\T_d} = \cong_d$. 
    \begin{proof}
        Let $x,y \in X$ and suppose $x_0 \cong_d y_0$.
        Let $U \in \scU_{\T_d}(x_0)$. Then for some $\epsilon > 0$, 
        $x_0 \in B(x_0;\epsilon) \subset U$. 
        Since $x_0 \cong_d y_0$, $d(x_0,y_0) = 0$, so $y_0 \in B(x_0 ; \epsilon) \subset U$. 
        Hence $U \in \scU_{\T_d}(y_0)$. 
        The arbitrary nature of $U \in \scU_{\T_d}(x_0)$ implies 
        $\scU_{\T_d}(x_0) \subset \scU_{\T_d}(y_0)$.
        A reverse construction would just as easily show the reverse inclusion, so we conclude that $x_0 \cong_{\T_d} y_0$. 
        Now suppose $x_0 \cong_{\T_d} y $. Then for each $n \in \N$, 
        $y_0 \in B_{d} \pa{x_0 ; \frac{1}{n}}$.
        Hence $d(x_0, y_0) < \frac{1}{n}$ for each $n \in \bbZ^+$, 
        and so $d(x_0,y_0) = 0$ and $x_0 \cong_d y_0$. 
    \end{proof}
\end{prop}


\newcommand{\PseudometricInducedMetric}[0]{\textbf{\hyperref[def:pseudometricinducedmetric]{Pseudometric Induced Metric}}\xspace}
\newcommand{\MetricInducedByPseudometric}[0]{\textbf{\hyperref[def:pseudometricinducedmetric]{Metric Induced By The Pseudometric}}\xspace}
\begin{df}[\MetricInducedByPseudometric]
    \label{def:pseudometricinducedmetric}
    \rm
    Let $(X,d)$ be a \PseudometricSpace, and let $\cong$ be the \RelationOfZeroDistance, which by \ref{prop:relationofzerodistance} is also the \RelationOfEqualNeighborhoodFilters on $(X,\T_d)$. 
    Define $\tilde{d}: X/\cong \to [0,\infty)$ by 
    \begin{equation*}
        \tilde{d}\pa{\bra{x}, \bra{y}} = d(x,y)
    \end{equation*}
    By \ref{prop:pseudometricinducedmetric}, $\tilde{d}$ is well defined and is in fact a \Metric on $X/\cong$, so we call $\tilde{d}$ the \MetricInducedByPseudometric d on X, or we call it the \PseudometricInducedMetric of $(X,d)$. 
\end{df}

\begin{prop}[Metric Space Induced By Pseudometric Space]
    \label{prop:pseudometricinducedmetric}
    %Let $X$, d, $\cong$, and $\tilde{d}$ be defined as in \ref{def:pseudometricinducedmetric}
    Let $(X,d)$ be a \PseudometricSpace, $\cong$ the \RelationOfZeroDistance on $(X,d)$ and $\tilde{d}$ be defined as in \ref{def:pseudometricinducedmetric}.
    Let $(X/\cong, \T_{X/\cong})$ be the  \QuotientTopologicalSpace with \QuotientMap T, and let $(X/\cong, \T_{\tilde{d}})$ be the topological space induced by the metric space $(X/\cong, \tilde{d})$. 
    The following are true. 
    \begin{enumerate}
        \item $\tilde{d}$ is in fact well defined, and is a metric on $X/\cong$, justifying calling it the \MetricInducedByPseudometric d.
        \item $\T_{X/\cong} = \T_{\tilde{d}}$
        \item T is an isometry from $(X,d)$ to $(X/\cong, \tilde{d})$
        \item $(X/\cong, \tilde{d})$ is complete if and only if $(X, d)$ is \PseudometricComplete.
        \item If $T:$
    \end{enumerate}


\end{prop}
\newcommand{\Pseudometrizable}[0]{\textbf{\hyperref[def:Pseudometrizable]{Pseudometrizable}}\xspace}
\newcommand{\Metrizable}[0]{\textbf{\hyperref[def:Pseudometrizable]{Metrizable}}\xspace}
\begin{df}[(Pseudo)Metrizable]
    \label{def:Pseudometrizable}
    Let $(X,\T)$ be a topological space. 
    \begin{enumerate}
        \item We say that $(X,\T)$ (Or $\T$ or X which it wouldn't cause confusion) is \Pseudometrizable if there exists a pseudometric d on X such that $\T$ is the \PseudometricInducedTopology on $(X,d)$. 
        \item We say that $(X,\T)$ (Or $\T$ or X when it wouldn't cause confusion) is \Metrizable if there exists a metric d on X such that $\T$ is the metric topology on $(X,d)$. 
    \end{enumerate}
\end{df}

\begin{prop}[\Pseudometrizable Prequotient]
    \label{prop:pseudometrizableprequotient}
    \rm
    Let $(X,\T_X)$ be a \TopologicalSpace 
    with \RelationOfEqualNeighborhoodFilters $\cong$, and
    with \QuotientTopologicalSpace  $\pa{X/\cong, \T_{X/\cong}}$
    and \QuotientMap T. Let $\pa{X/\cong, \T_{X/\cong}}$ be \Pseudometrizable with \Pseudometric $\tilde{d}$. 
    The following hold. 
    \begin{enumerate}[label=(\roman*), ref={\ref{prop:pseudometrizableprequotient}~\roman*}]
        \item  
        \label{prop:PseudoPre:Pseudometrizable}
        Define $d:X^2 \to [0,\infty)$ by  $d(x,y) = \tilde{d}\pa{[x],[y]}$. 
        Then $\tilde{d}$ is a \Pseudometric on $X$ which is 
        \PseudometricCompatible with $\scT_X$. 
        \item 
        \label{prop:PseudoPre:Metrizable}
        $\tilde{d}$ is a \Metric $(X/\cong, \T_{X/\cong})$.
        \item 
        \label{prop:PseudoPre:Injective}
        If $T$ is \Injective, then $d$ as defined above is a \Metric on $X$.
    \end{enumerate}
    \begin{proof}[Proof Of \ref{prop:PseudoPre:Pseudometrizable}]
    We first prove $d$ to be a \Pseudometric on $X$.
        First, observe that if $x,y \in X$, then
        $d(x,y) =\tilde{d}([x],[y]) \in [0,\infty)$
        so that d is well defined. 
        Also, 
        $ d(x,y) = \tilde{d}([x],[y])=\tilde{d}([y],[x])=d(y,x)$,
        so d is \CommutativeFunction.
        Furthermore, 
        \begin{align*}
            d(x,z) & = \tilde{d}([x],[z])\\
            & \leq \tilde{d}([x],[y])+\tilde{d}([y], [z])\\
            & = d(x,y)+d(y,z)
        \end{align*}
        so d satisfies the \TriangleInequality. 
        Lastly, 
        $d(x,x)=\tilde{d}([x],[x])=0$, 
        and so $d$ is a \Pseudometric on $X$. 
        
        Let $\T_d$ denote the \PseudometricTopology on $(X,d)$. 
        What remains to show is that $\T_X=\T_d$. 
        Since $d(x,y)=\tilde{d}([x], [y])=\tilde{d}(Tx, Ty)$, T is an \Isometry. 
        Let $x \in U \in \T_X$. Then $[x] \in T(U) \in \T_{X/\cong}$. 
        Hence, there is an $\epsilon > 0$ such that $B_{\tilde{d}}([x], \epsilon) \subset T(U)$. 
        By \ref{prop:QST:OpenSetFiber}, $T^{-1}(B_{\tilde{d}}([x], \epsilon) \subset T^{-1}(T(U))=U$.
        Furthermore, by 
        \ref{prop:QST:QuotientMapContinuous}, 
        $T^{-1}(B_{\tilde{d}}([x], \epsilon) \in \T_X$. 
        Since T is an \Isometry $B_d(x, \epsilon) = T^{-1}(B_{\tilde{d}}([x],\epsilon) \subset U$. 
        Thus we have found an \OpenBall contained in $U$  containing an arbitrary point of U.
        Hence, $\T_X \subset \T_d$. As part of the preceeding arguement we also showed that an  arbitrary d-\OpenBall was in $\T_X$, so $\T_d \subset \T_X$, and so equality holds and we're done. 
    \end{proof} 
    \begin{proof}[Proof of \ref{prop:PseudoPre:Metrizable}]
        Let $x,y \in X$ wtih $[x] \neq [y]$. 
        Then $x \not \cong y$. By \ref{prop:relationofzerodistance}, $x \not \cong_d y$. 
        Hence $\tilde{d}([x],[y])=d(x,y) > 0$. 
    \end{proof}
    \begin{proof}[Proof of \ref{prop:PseudoPre:Injective}]
        Let T be \Injective, and suppose $x,y \in X$ with $x \neq y$. 
        Then $[x]=Tx\neq Ty=[y]$, 
        so by \ref{prop:PseudoPre:Metrizable}, $d(x,y) = \tilde{d}([x],[y]) > 0$. 
        
    \end{proof} 
\end{prop} 

\subsection{Non-Hausdorff Baire Category Theorem}
\newcommand{\NowhereDense}[0]{\textbf{\hyperref[def:NowhereDense]{Nowhere Dense}}\xspace}
\begin{df}[\NowhereDense]
\label{def:NowhereDense}
\rm
Let $X$ be a \TopologicalSpace. 
Let $A \subset X$ such that 
$\InteriorMark{\ClosureMark{A}} = \emptyset$.
Then we say that $A$ is \NowhereDense
\end{df}
\newcommand{\FirstCategory}[0]{\textbf{\hyperref[def:Category]{First Category}}\xspace}
\newcommand{\SecondCategory}[0]{\textbf{\hyperref[def:Category]{Second Category}}\xspace}
\begin{df}[\FirstCategory, \SecondCategory]
\label{def:Category}
\rm
Let $X$ be a \TopologicalSpace. 
Let $A \subset X$ be a \Countable union of 
\NowhereDense sets in $X$.
Then we say that $A$ is of the \FirstCategory. 
If $B \subset X$ is not of the \FirstCategory, 
then we say that $B$ is of the \SecondCategory. 
\end{df}

\begin{prop}[Baire Category]
\label{prop:BaireCategoryTheorem}
\rm
Let $(X, \T)$ be a \Complete \PseudometricSpace. 
Let $\{U_i\}_{i \in \mathbb{N}} \subset \scT$.
Suppose for each $i \in \mathbb{N}$, $U_i$ is 
\TopologyDense in $X$.
Define $U = \bigcap\limits_{i \in \mathbb{N}} U_i$.
Then $U$ is \TopologyDense in $X$. 
\begin{proof}
Let  $\emptyset \neq V_0 \in \T$.
We must show $V_0 \cap U \neq \emptyset$. 
Since $U_1$ is \TopologyDense, 
there exists $0 , \epsilon_1 < \frac{1}{1}$ and $x_1 \in X$ such that 
\begin{equation*}
\overline{B(x_1,\epsilon_1)} \subset B(x_1, 2 \epsilon_1) \subset U_1 \cap V_0
\end{equation*}
Set $V_1 = B(x_1;\epsilon_1)$. 
Continuing recursively, for each $n \in \mathbb{N}$
we get and $0 < \epsilon_n < \frac{1}{n}$ 
and an $x_n \in X$ such that 
for every $n \in \mathbb{N}$. 
\begin{equation*}
V_{n} = B(x_n, \epsilon_n) \subset \overline{B(x_n, \epsilon_n)} \subset B(x_n, 2\epsilon_n) \subset U_n \cap V_{n-1}
\end{equation*}
The sequence $\{x_k\}_{k \in \mathbb{N}}$ is clearly 
Cauchy, and so since $X$ is \Complete it posesses 
a limit $x_0$. 
Observe that if $n > k$, then $x_n \in \overline{B(x_k, \epsilon_k)}$. 
Hence, for every $n \in \mathbb{N}$, $x_0 \in \overline{B(x_k, \epsilon_k)}$, so 
\begin{equation*}
x_0 \in \bigcap\limits_{n \in \mathbb{N}} \overline{B(x_n, \epsilon_n)} \subset \bigcap\limits_{n \in \mathbb{N}} U_n \cap V_0 = V_0 \cap \bigcap\limits_{n \in \mathbb{N}} U_n
\end{equation*}
\end{proof}
\end{prop}

\begin{cor}
\label{cor:BaireCategoryTheorem}
\rm
Let $X$ be an \Infinite  \Complete \PseudometricSpace. 
Then $X$ is of \SecondCategory in itself.
That is to say, $X$ is not the \Countable
union of \NowhereDense sets.
%\begin{proof}
%Let $\{A_i\}_{i \in \mathbb{N}}$ be a collection of 
%\NowhereDense subsets of $X$. 
%Then for each $i \in \mathbb{N}$, $X \setminus \overline{A_i}$ is an
%\SetOpen \TopologyDense subset of  $X$. 
%By \ref{prop:BaireCategoryTheorem}
%$\bigcap\limits_{i \in \mathbb{N}} \pa{X \setminus \overline{A_i}}$ 
%is \TopologyDense in $X$.
%Hence  if $V$ is \SetOpen, then $V \cap \bigcap\limits_{i \in \mathbb{N}} X \setminus \overline{A_i} \neq \emptyset$. 
%When then implies $x \neq \bigcup\limits_{i \in \mathbb{N}} A_i$.
%\end{proof}
\end{cor}