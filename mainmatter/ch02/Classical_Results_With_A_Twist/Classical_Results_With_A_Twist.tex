
\section{Classical Results With A Twist}

By \ref{prop:seminormlinearoperators} and \ref{prop:weakquotients}, many of the classical theorems relating a normed space and its duals still hold in the context of a seminormed space without too much alteration of the proofs. Since the author has not seen these results presented in this context, they are presented with proof below. 
\subsection{Helly}
In this subsection, we develop Helly's theorem in the context of a seminormed space, which will serve as valuable lemma throughout this document. Its location here is due to the fact that  is is a generalization of a lemma commonly used to prove the Goldstine Theorem. 
\begin{thm}[Helly's Theorem]
    \label{thm:helly}
    Let $(X,\norm{\cdot})$ be a seminormed space $M>0$, $\{\alpha_i\}_{i=1}^n \subset \R$, and $\{x_i^*\}_{i=1}^n \subset X^*$. Then the following are equivalent. 
    \begin{enumerate}
        \item For every $\epsilon > 0$, there is an $x_\epsilon \in X$ such that $\norm{x_\epsilon}< M+\epsilon$ and $\ip{x,x_i^*}=\alpha_i$ for $1 \leq i \leq n$. 
        \item For every $\epsilon > 0$, there is an $x_\epsilon \in X$ such that $||x_\epsilon|| \leq M$ and $\abs{\ip{x_\epsilon,x_i^*}-\alpha_i}< \epsilon$ for $1 \leq i \leq n$. 
        \item For each $\{\beta_i\}_{i=1}^n \subset \C$, 
        \begin{equation}
            \label{eq:helly}
            \abs{\sum_{i=1}^n \beta_i \alpha_i} \leq M\norm{\sum_{i=1}^n \beta_i x_i^*}
        \end{equation}
    \end{enumerate}
    \begin{proof} $(1 \implies 2)$
        Obvious since $\{x_i^*\}_{i=1}^n \subset X^*$. 
    \end{proof}
    \begin{proof} $(2 \implies 3)$
        Let $\{\beta_i\}_{i=1}^n \subset \C$ and choose $\epsilon >0$. 
        Then
        \begin{equation}
            \begin{split}
                \abs{\sum_{i=1}^n \beta_i \alpha_i }&= \abs{ \sum_{i=1}^n \beta_i \pa{ \alpha_i-\ip{x_\epsilon,x_i^*}+\ip{x_\epsilon,x_i^*}}}\\
                & < \sum_{i=1}^n \epsilon \abs{\beta_i} + \abs{ \ip{ x_{\epsilon} , \sum_{i=1}^n \beta_ix_i^*}}\\
                & \leq \epsilon \sum_{i=1}^n \abs{\beta_i} + M \norm{\sum_{i=1}^n \beta_i x_i^*}
            \end{split} 
        \end{equation}
        Since $\epsilon$ was arbitrary, the desired inequality holds. 
    \end{proof} 
    \begin{proof} $(3 \implies 1)$
        If each $x_i^*=0$, then by \ref{eq:helly}, each $\alpha_i=0$, so $x=0$ works for any $\epsilon$. 
        Otherwise, we without loss of generality reorder $\{x_i^*\}_{i=1}^n$ so that $\{x_i^*\}_{i=1}^m$ is linearly independent and $span\{x_i\}_{i=1}^m=span\{x_i\}_{i=1}^n$. 
        Define $S:X \to \C^{m}$ by $Sx=\bra{\ip{x,x_1^*},\cdots,\ip{x,x_m^*}}$, and recognize that linear independence of $\{x_i\}_{i=1}^n$ guarantees that S is surjective, so there exists 
        $\tilde{x} \in S^{-1} \{ \bra{\alpha_1,\cdots,\alpha_m}\}$ and furthermore
        \begin{equation}
            S^{-1}\{ \bra{\alpha_1,\cdots,\alpha_m}\}=\tilde{x}+ \bigcap_{i=1}^m kern(x_i^*) = \tilde{x}+ \bigcap_{i=1}^n kern(x_i^*):=\tilde{x}+K 
        \end{equation}
        If $m < j \leq n$, then $x_j^* \in span(x_i^*,\cdots,x_n^*)$ so  $x_j = \sum_{i=1}^n \gamma_i x_i^*$.
        Now we choose $\beta_i=\gamma_i$ for $1 \leq i \leq m$, $\beta_j=-1$, and other $\beta_k's=0$. 
        \begin{align} 
            \abs{\ip{\tilde{x},x_j^*}-\alpha_j}&= \abs{\ip{\tilde{x},\sum_{i=1}^m \beta_i x_i^*}-\alpha_j} = \abs{ \pa{\sum_{i=1}^m \beta_i \ip{\tilde{x},x_i^*}}+(-1)\alpha_j}\\
            &=\abs{\sum_{i=1}^n \beta_i \alpha_i} \\
            &\leq M \norm{\sum_{i=1}^n \beta_i x_i^*} \\
            & =M\norm{\sum_{i=1}^m \gamma_i x_i^* + (-1) x_j^*}
        \end{align}
        Hence $\ip{x,x_i^*} = \alpha_i$ for $1 \leq i \leq n$ and any $x \in \tilde{x}+K$. 
        By the Hahn Banach Theorem applied to the semi norm on X induced by the seminorm on the quotient space $X/K$, there is an $x^* \in X^*$ such that $\sup\limits_{\norm{x} \neq 0} \frac{\abs{\ip{x,x^*}}}{\norm{x}} =1$, $\ip{\tilde{x},x^*} = d(\tilde{x},K)$, and $K \subset ker(x^*)$. Since $K \subset ker(x^*)$, $x^* \in span \{ x_i^*\}_{i=1}^n$, say $x^* = \sum_{i=1}^n \mu_i x_i^*$. 
        Hence, 
        \begin{equation}
            \begin{split}
                d(\tilde{x},K) = \ip{\tilde{x},x^*} & = \sum_{i=1}^n \mu_i \ip{\tilde{x},x_i^*}\\
                & = \sum_{i=1}^n \mu_i \alpha_i\\
                & \leq M\norm{\sum_{i=1}^n \mu_i x_i^*}=M 
            \end{split}
        \end{equation}
        Thus, given $\epsilon > 0$, we find an $z_\epsilon \in K$ such that $||\tilde{x}-z_\epsilon|| < M+\epsilon$ and let $x_\epsilon=\tilde{x}-z_\epsilon$. 
        
    \end{proof}
    
\end{thm}

\begin{cor}
    \label{cor:helly}
    Let X be a seminormed space, $x^{**} \in X^{**}$, $\{x_i^*\}_{i=1}^n \subset X^*$, and $\epsilon > 0$. Then there exists the following.
    \begin{enumerate}
        \item An $x_1 \in X$ such that $\norm{x_1} \leq \norm{x^{**}}+\epsilon$ and for $1 \leq i \leq n$, $\ip{x_1,x_i^*} = \ip{x_i^*,x^{**}}$.  
        \item An $x_2 \in X$ such that $\norm{x_2} \leq \norm{x^{**}}$ and for $1 \leq i \leq n$, $\abs{\ip{x_1,x_i^*}-\ip{x_i^*,x^{**}}} < \epsilon$. 
    \end{enumerate} 
    \begin{proof} For $1 \leq i \leq n$, let $\alpha_i=\ip{x_i^*,x^{**}}$, and let $\{\beta_i\}_{i=1}^n \subset \C$. Then 
        \begin{equation}
            \abs{\sum_{i=1}^n \beta_i \alpha_i} = \abs{ \ip{\sum_{i=1}^n \beta_i x_i^*,x^{**}}} \leq \norm{x^{**}} \norm{\sum_{i=1}^n \beta_i x_i^*}
        \end{equation}
        An application of \ref{thm:helly} completes the proof. 
    \end{proof} 
\end{cor} 
\subsection{Goldstine} 
\begin{thm}[Goldstine]
    \label{thm:goldstine}
    Let X be a seminormed space, $c:X \to X^{**}$ denote the canonical embedding, B denote the closed unit ball of X, and $B_1$ denote the closed unit ball of $X^{**}$. Then $c(B)$ is $weak^*$ dense in $B_1$.   
    \begin{proof}
        Note that the $weak^*$ topology on $X^{**}$ has as a basis sets of the form
        \begin{equation}
            U\pa{x^{**},\epsilon,\{x_i^*\}_{i=1}^n}:=\{ y^{**} \in X^{**} : \pa{\forall 1 \leq i \leq n}\pa{ \abs{\ip{x_i^*,x^{**}-y^{**}}} < \epsilon }
        \end{equation}
        where we range over all $x^{**} \in X^{**}$, $\epsilon > 0$, and finite subsets $\{x_i^*\}_{i=1}^n$ of $X^*$. 
        Let $x^{**} \in B_1$, $\epsilon > 0$, and $\{x_i^*\}_{i=1}^n \subset X^*$. 
        Let $\epsilon > 0$. Then by \ref{cor:helly}, there exists an $x \in B$ such that for $1 \leq i \leq n$, $\abs{\ip{x_i^*,x^{**}-c(x)}}= \abs{\ip{x_i^*,x^{**}}-\ip{x,x_i^*}} < \epsilon$, from which the desired result follows.
    \end{proof}
\end{thm} 
\subsection{Banach Alaoglu}
The following well known result concerning the $weak^*$ compactness of the unit ball of a Banach space was first proven in the separable case by Banach, and then generalized in 1940 by Alaoglu \cite{alaoglu40} to Banach spaces. Generalizations of this result in a general TVS satisfying sufficient conditions have also been shown but the form presented here comes from \cite{hester07}, who drops the assumption of completeness for one direction of the implication. 
\begin{thm}[Banach-Alaoglu-Morales]
    \label{thm:banachalaoglu}
    Let X be a normed space and define $B=\{x^* \in X^*: ||x^*|| \leq 1\}$. Then B is $weak^*$ compact. 
    \begin{proof}
        Let $\mathbb{F}$ denote $X's$ field, and for $x \in X$, define $D_x=\{y \in \mathbb{F}: \abs{y} \leq \norm{x}\}$. 
        Then each $D_x$ is Hausdorff and compact so by Tychonoff's theorem, $D:=\prod_{x \in X} D_x$ is compact and Hausdorff when endowed with the product topology. 
        If $T \in D$, then $T:X \to \mathbb{F}$ and $\abs{Tx} \leq \norm{x}$ for each $x \in X$, so $D \cap X^* \subset B$. 
        It is also clear that $B \subset D$, so $D \cap X^* = B$. 
        Let $\{\gamma_\alpha\}_{\alpha \in A}$ be a net in B converging to $\gamma \in D$ in D's product topology. 
        Then, letting $\pi_x$ denote the $x^{th}$ projection, for each $x \in X$, 
        \begin{equation}
            \gamma_{\alpha}(x) = \pi_x(\gamma_\alpha) \to \pi_x(\gamma) = \gamma(x)
        \end{equation}
        If $\alpha \in \mathbb{F}$ and $x,y \in X$, then 
        \begin{equation}
            \ip{\alpha x+y, \gamma_\alpha} \to \ip{ \alpha x+y, \gamma}
        \end{equation}
        and also
        \begin{equation}
            \ip{\alpha x+y,\gamma_\alpha} = \alpha \ip{x,\gamma_\alpha}+ \ip{y,\gamma_\alpha} \to \alpha \ip{x,\gamma}+\ip{y,\gamma}
        \end{equation}
        which implies $\gamma$ is linear since D is Hausdorff, and hence $\gamma \in B$. 
        Thus $B$ is closed in D. What remains to be shown is that the $weak*$ topology on B is the subspace topology on B induced by $D's$ topology, since a Closed subset of a compact Hausdorff space is compact. 
        For notation, denote with $\mathcal{T}_D$ the subspace topology on B induced by D's topology, and denote with $\mathcal{T}_w$ the subspace topology on B induced by the  $weak^*$ topology on $X^*$. 
        To see that $\mathcal{T}_w \subset \mathcal{T}_D$, let $\{\gamma_\alpha\}_{\alpha \in A} \subset B$ such that $\gamma_\alpha \overset{\mathcal{T}_D}{\to} \gamma$. For each $x \in X$, letting c be the canonical embedding, 
        \begin{equation}
            \ip{\gamma_{\alpha}, c(x)}= \ip{x,\gamma_\alpha} = \pi_x(\gamma_\alpha) \to \pi_x(\gamma) = \ip{x,\gamma} = \ip{\gamma, c(x)}
        \end{equation}
        Hence $\gamma_\alpha \overset{\mathcal{T}_w}{\to} \gamma$, so $\mathcal{T}_w \subset \mathcal{T}_D$.  
        To see that $\mathcal{T}_D \subset \mathcal{T}_w$, fix $x\in X$ and let $\{\gamma_\alpha\}_{\alpha \in A} \subset B$ such that $\gamma_\alpha \overset{\mathcal{T}_w}{\to} \gamma$. 
        Then $\pi_x(\gamma_\alpha) = \ip{x,\gamma_\alpha} \to \gamma(x) = \pi_x(\gamma)$, so by definition of the product topology $\gamma_\alpha \overset{\mathcal{T}_D}{\to} \gamma$, implying $\mathcal{T}_D \subset \mathcal{T}_w$.  Hence B is $weak^{*}$ is compact. 
    \end{proof}
\end{thm} 

\begin{cor}[Banach Alaoglu Seminorm]
    Let X be a seminormed space and define $B=\{x^* \in X^*: \norm{x^*} \leq 1\}$. Then B is $weak^*$ compact. 
    \begin{proof} 
        This is  a consequence of the fact that the $weak^*$ topology on $X^*$ is identical to the $weak^*$ topology on the dual space of $X/\norm{\cdot}^{-1}\{0\}$. 
    \end{proof} 
\end{cor} 
This gives us the useable result
\begin{cor}[Banach-Alaoglu-Morales]
    \label{thm:banachalaoglumorales}
    Let X be a seminormed space and $C \subset X^*$ 
    \begin{enumerate}
        \item If X is complete and $C$ $weak^*$ compact, then $C$ is $weak^*$ closed and bounded. 
        \item If $C$ is $weak^*$ closed and bounded, then $C$ is $weak^*$ compact. 
    \end{enumerate}
    \begin{proof}(1)
        Since $C$ is $weak^*$ compact, it is $weak^*$ closed since the $weak^*$ topology is Hausdorff. Since $c(x):(X,\mathcal{T}_{w^*}) \to \mathbb{C}$ is continuous for each $x \in X$, for each $x \in X$, $c(x)(C)=$ is compact and therefore bounded. Hence, for every $x \in X$, $\{\abs{\ip{x,c}} : c \in C\}$ is bounded, so by the Banach Steinhaus, C is bounded. 
    \end{proof}
    \begin{proof}(2)
        Since $C$ is bounded, it is contained in some closed ball B which we know to be $weak^*$ compact by \ref{thm:banachalaoglu}. Since the $weak^*$ topology on B is compact and  Hausdorff and $C$ is closed in this topology, it is compact in this topology. Since the subspace topology on C induced by the $weak^*$ topology on $X^*$ equals this topology, we are done.  
    \end{proof} 
\end{cor} 
\subsection{Eberlein-Smulian}
The purpose of this section is to provide a characterization of weakly compact subsets of a complete seminormed space X, which will serve to increase the applicability of the results regarding weakly compactly generated spaces covered later in this document. 
The first main result of this section, \ref{thm:eberleinsmulian}, serves to show that even though weak topologies of Banach Spaces are not in general metrizable, an equivalence between weak compactness and sequential compactness exists. From this  result $(1 \implies 2)$ was first presented in the case of normed spaces in \cite{smulian40}, and then $(2 \implies 1)$ was proven in the case of normed spaces in \cite{eberlein47}. Several different proofs have been given in the years since, and the one present here is based on that present in \cite{whitley67}, which is also followed in \cite{diestel84}. 
We begin with a few lemmas. 
\begin{lem}[Metrizable Weak]
    \label{lem:metrizableweak}
    If X is a seminormed space and $X^*$ contains a countable set that separates points mod $K:=\norm{\cdot}^{-1}\{0\}$,
    then subspace topology induced by the weak topology on any weakly compact subset A of X is pseudometrizable. 
    \begin{proof}
        As a consequence of \ref{prop:pseudometrizableprequotient} and \ref{prop:weakquotients}, it is sufficient to let X be a normed space and $\{x_i^*\}_{i \in \N}$ separate points in X. 
        Let $M=2\sup\limits_{x \in A} ||x||$, and define d to be the metric on A defined by, for $x,y \in A$, 
        \begin{equation}
            d(x,y) = \sum_{k \in \N} \frac{\abs{\ip{x-y,x_k^*}}}{\norm{x_k^*} 2^k}
        \end{equation}
        Let $x \in A$, $\epsilon > 0$ be arbitrary, and define 
        \begin{equation}
            n=\ceil{2+log_2\pa{\frac{M}{\epsilon}}} \tab[1cm] U=A \cap \bigcap_{k=1}^n
            \left\{y \in X : \abs{\ip{x-y,x_k^*}} < \frac{\norm{x_k^*}2^{k-1}\epsilon}{n}\right\}
        \end{equation}
        The U is open in the subspace topology on A induced by $X's$ weak topology. Furthermore, if $y \in U$, then 
        \begin{equation}
            \begin{split}
                d(x,y) &= \sum_{k \in \N} \frac{\abs{\ip{x-y,x_k^*}}}{\norm{x_k^*}2^k} \\
                & \leq \sum_{k=1}^n \frac{\abs{\ip{x-y,x_k^*}}}{\norm{x_k^*}2^k} + \sum_{k=n+1}^\infty \frac{2M}{2^k}\\
                & < \sum_{k=1}^n \frac{\epsilon}{2n}+ \frac{M}{2^{n-1}}< \epsilon
            \end{split} 
        \end{equation}
        So that $U \subset B_d(x;\epsilon)$. 
        This implies $Id:(A,\mathcal{T}_w) \to (A,\mathcal{T}_d)$ is continuous. 
        Since a continuous injection from a compact space into a Hausdorff space is a homeomorphism, the subspace topology on A induced by the weak topology equals the topology on A induced by d, and so A's weak topology is metrizable. 
    \end{proof} 
\end{lem}
\begin{lem}
    \label{lem:finiteselection}
    Let X be a seminormed space and $Y \subset X^{**}$ be a finite dimensional vector subspace. Then there exists a finite set $Z \subset \partial B_{X^*}(0;1)$ such that for each $y^{**} \in Y$, 
    \begin{equation}
        \norm{y^{**}} \leq 2 \max\limits_{z^* \in Z} \abs{\ip{z^*,y^{**}}}
    \end{equation}
    \begin{proof}
        Let $S=\partial B_{X^{**}}(0;1) \cap Y$. Then, since Y is finite dimensional, S is compact, and therefore permits a $\frac{1}{4}-net$ $\{s_i\}_{i=1}^n$. 
        Now let $\{z_k^*\}_{k=1}^n  \subset \partial B_{X^*}(0;1)$ such that for each k, $\ip{z_k^*,s_i} > \frac{3}{4}$.
        Let $s \in S$ then there is a k such that $\norm{s-s_k}< \frac{1}{4}$. for this k, we have
        \begin{equation}
            \ip{z_k^*,s} = \ip{z_k^*,s_k} + \ip{z_k^*,s-s_k} \geq \frac{3}{4}-\frac{1}{4} = \frac{1}{2}
        \end{equation}
    \end{proof} 
\end{lem}

\begin{thm}[Eberlein-Smulian]
    \label{thm:eberleinsmulian}
    Let X be a seminormed space and $A \subset X$. Then the following are equivalent. 
    \begin{enumerate}
        \item A is weakly compact.
        \item A is weakly sequentially compact.
    \end{enumerate}
    \begin{proof} $(1 \implies 2)$
        Let $A \subset X$ be weakly compact, and let $\{x_i\}_{i \in \N} \subset A$. Define $S=\overline{span\{x_i: i \in \N\}}$. Since S is closed and convex, it is weakly closed, and so $A \cap S$ is weakly compact as well. 
        By construction, S is separable, and so contains a countable dense set $\{y_i\}_{i \in \N}$. By Hahn-Banach, for each $i \in \N$, there exists $y_i^* \in S^*$ such that $\ip{y_i,y_i^*}=1$, and continuity of each $y_i^*$ implies $\{y_i^*\}_{i \in \N}$  separates points in S mod $\norm{\cdot}^{-1}\{0\}$.
        Hence we can apply \ref{lem:metrizableweak} to claim that the subspace topology on $A \cap S$ induced by $S's$ weak topology is metrizable, and therefore $\{x_i\}_{i \in \N}$ has a sub-sequence $\{x_{n_i}\}_{i \in \N}$ which is weakly S-convergent, and therefore weakly X-convergent since subspace topologies are no less fine than the topologies that induce them. 
        Since $A \subset X$ is weakly closed, this sequence converges within A, and so A is weakly sequentially compact. 
    \end{proof}
    \begin{proof} $(2 \implies 1)$. 
        Let $A \subset X$ be weakly sequentially compact, let c denote the canonical embedding of $X$ into $X^{**}$, and let $x^{**}$ in the $weak^*$ closure of $c(A)$. Let $x_1^1 \in X^*$ have norm 1. By assumption, there exists $a_1^{**} \in c(A)$ such that $\abs{\ip{x_1^*,x^{**}-a_1^{**}}} < 1$. 
        By \ref{lem:finiteselection}, there exists $\{x_1^2,\cdots,x_{n_2}^2\} \subset \partial B_{X^*}(0;1)$ such that for each $y^{**} \in span\left\{x^{**},x^{**}-a_1^{**}\right\}$, 
        \begin{equation}
            \norm{y^{**}} \leq 2 \max\limits_{1 \leq k \leq n_2} \abs{\ip{x_k^2,y^{**}}}
        \end{equation}
        Also, since $x^{**}$ is in the $weak^*$ closure of $c(A)$, there exists $a_2^{**} \in c(A) \cap U_2$ where
        \begin{equation}
            U_2=\{y^{**} \in X^{**} : (\forall 1 \leq j \leq 2)(\forall 1 \leq k \leq n_j)(\abs{\ip{x_k^j,x^{**}-y^{**}}} < \frac{1}{2})\}
        \end{equation}
        Continuing inductively, , we construct a sequence $\{a_n^{**}\}_{n \in \N} \subset c(A)$ such that for each $j \in \N$,  $\{x_k^{j}\}_{k=1}^{n_j} \subset \partial B_{X^*}(0;1)$ such that for every $y^{**} \in span\left\{x^{**},x^{**}-a_1^{**}, \cdots, x^{**}-a_{j-1}^{**}\right\}$, we have 
        \begin{equation}
            \norm{y^{**}} \leq 2  \max\limits_{1 \leq k \leq n_j} \abs{\ip{x_k^j,x^{**}-y^{**}}}
        \end{equation}
        and  $a_j^{**}\in c(A) \cap U_j$ where $U_j$ is the $\{x_k^{m}\}_{1 \leq m \leq j,1 \leq k \leq n_m}$ $weak^*$ neighborhood about $x^{**}$ of radius $\frac{1}{j}$.
        For each $k \in \N$, let $a_k=c^{-1}(c(a_k))$. Since A is sequentially weakly compact, $\{a_k\}_{k \in \N}$ has a weak cluster point $x \in A$. 
        Also, $x \in \overline{span\{a_i\}_{i \in \N}}$ because this is a weakly closed set, implying 
        $c(x) \in \overline{span\{a_i^{**}\}_{i \in \N}}$, which then implies 
        $c(x) \in \overline{span\{x^{**},x^{**}-a_1^{**},x^{**}-a_2^{**},\cdots\}}$. 
        By continuity of the norm and each element of $\{x_i^k\}_{k \in \N, 1 \leq i \leq n_k}$, we conclude that for each element $y^{**}$ of 
        $\overline{\{x^{**},x^{**}-a_1^{**},x^{**}-a_2^{**},\cdots\}}$, 
        \begin{equation}
            \norm{y^{**}} \leq 2 \sup\limits_{k \in \N, 1 \leq i \leq n_k} \abs{\ip{x_i^k,y^{**}}} 
        \end{equation}
        This is useful, because for each $k \in \N$, $1 \leq i \leq n_k$, we have, for large enough j,
        \begin{equation}
            \begin{split}
                \abs{\ip{x_i^k, x^{**}-c(x)}} & \leq \abs{\ip{x_i^k,x^{**}-a_j^{**}}} + \abs{\ip{a_j^{**}-c(x),x_i^k}}\\
                & \leq \frac{1}{j}+ \abs{ \ip{x_i^k, a_j-x}}
            \end{split}
        \end{equation}
        which can be made arbitrarily small, and so $\abs{\ip{x_i^k,x^{**}-c(x)}}=0$, implying that 
        \begin{equation}
            \norm{x^{**}-c(x)} \leq 2 \sup\limits_{k \in \N, 1 \leq i \leq n_k} \abs{\ip{x_i^k,x^{**}-x}}=0
        \end{equation}
        So $x^{**}=c(x)$, and therefore $c(A)$ is $weak^*$ closed. Since A is weakly-sequentially compact, $c(A)$ is $weak^*$ sequentially compact and therefore bounded by Banach Steinhaus. 
        By \ref{thm:banachalaoglu}, bounded $weak^*$ closed sets are compact, and so $c(A)$ is $weak^*$ compact. 
        Since the weak topology on $A/\norm{\cdot}^{-1}\{0\}$  is the same as the $weak^*$ topology on $c(A)$, $A / \norm{\cdot}^{-1}\{0\}$ is weakly compact. 
        To see that A is weakly compact, apply \ref{prop:pseudometricembedding}.
    \end{proof} 
\end{thm} 

\subsection{Bishop-Phelps}
In this subsection, I develop a result due to \cite{bishopphelps63} which will prove useful throughout this document.
I begin by presenting the concept of a convex cone and a trio of lemmas which are commonly utilized in the proof of this result. 
\begin{df}[Convex Cone] \rm
    \label{df:ConvexCone}
    Let $X$ be a seminormed space over $\R$. If $K \subset X$ is convex and closed under positive scalar multiples, then we call it a \bf convex cone\rm. If J is a convex cone in X, $C \subset X$, $x_0 \in C$, and $(J+x_0) \cap C = \{x_0\}$, then we say that J \bf supports \rm $C$ at $x_0$. If $x^* \in \partial B_{X^*}(0;1)$ and $\alpha >0$ then we define 
    \begin{equation}
        K(x^*,\alpha):=\{x \in X: ||x|| \leq \alpha \ip{x,x^*}\}
    \end{equation}
\end{df} 

\begin{rmk}\rm
    \label{rmk:ClosedConvexCone}
    Let X be a seminormed space, $x^* \in \partial B_{X^*}(0;1)$, and $\alpha > 0$. The following are true. 
    \begin{enumerate} 
        \item $K(x^*,\alpha)$ is a closed convex cone. 
        \item If $\alpha > 1$, $Int(K(x^*,\alpha)) \neq \emptyset$. 
    \end{enumerate} 
    \begin{proof}(1)
        If $\{x_n\} \subset K(x^*,\alpha)$ converges, say $x_n \to x$, then continuity of $x^*$ implies $\ip{x_n,x^*} \to \ip{x,x^*}$. Hence, given $\epsilon > 0$, there exists $N>0$ such that for $n>N$ we have $max\pa{\abs{||x||-||x_n||},\abs{\ip{x-x_n,x^*}}} < \epsilon$, so that for all $n>N$,
        \begin{equation}
            \norm{x}\leq \norm{x_n} +\epsilon < \alpha \ip{x_n,x^*} + \epsilon < \alpha \ip{x,x^*}+ (\alpha+1)\epsilon
        \end{equation}
        So $x \in K(x^*,\alpha)$ closedness is verified. It is obvious that $K(x^*,\alpha)$ is closed under positive scalar multiples, and for convexity, if $x,y \in K(x^*,\alpha)$ and $t \in [0,1]$, then
        \begin{equation}
            \norm{tx+(1-t)y} \leq t\norm{x}+(1-t)\norm{y} \leq t \alpha \ip{x,x^*}+(1-t) \alpha \ip{y,x^*} = \alpha \ip{tx+(1-t)y,x^*}
        \end{equation}
    \end{proof} 
    \begin{proof}(2)
        By definition of the norm on $X^*$, there is an $x \in \overline{B_X(0;1)}$ such that $2/\pa{\alpha \pa{1+\frac{1}{\alpha}}}< \ip{x,x^*}$, implying by linearity that $1/\alpha < \ip{\frac{1+(1/\alpha)}{2}x,x^*}$. By continuity of $x^*$ we find a neighborhood U of $\pa{1+\frac{1}{\alpha}}/2$ contained in $B_X(0;1)$ such that for each $y \in U$, $1/\alpha < \ip{y,x^*}$. This implies $||y|| \leq 1 < \alpha \ip{y,x^*}$, so $U \subset K(x^*,\alpha)$.
    \end{proof} 
\end{rmk}
\begin{lem}[Bishop-Phelps Lemma]
    \label{lem:BishopPhelps}
    
    Let X be a complete seminormed space, $x^*,y^* \in \partial B_{X^*}(0;1)$, $C \subset X$ closed and convex, $1>\epsilon >0$, and $k>1+\frac{2}{\epsilon}$. The following are true. 
    \begin{enumerate}
        \item If $x^*$ is bounded on C, then for each $z \in C$, there is an $x_0 \in X$ such that $K(x^*,\epsilon)$ supports $C$ at $x_0$ and $x_0 \in K(x^*,\epsilon)+z$. 
        \item If $\abs{\ip{x,y^*}}\leq \frac{\epsilon}{2}$ for each $x \in Kern(x^*) \cap \overline{B_X(0;1)}$, then 
        \begin{equation}
            min\pa{\norm{x^*+y^*},\norm{x^*-y^*}} \leq \epsilon
        \end{equation}
        \item If $y^*$ is nonnegative on $K(x^*,k)$, then $\norm{x^*-y^*}\leq \epsilon$.
    \end{enumerate}
    \begin{proof}(1)
        Let $x^*$ be bounded on C and define, for $x,y \in X$, $y \lesssim x$ if and only if $x-y \in K(x^*,\epsilon)$. 
        Fix $z \in C$. 
        Define $Z=C \cap \pa{ K(x^*,\epsilon)+z}$. Since C and $K(x^*,\epsilon)$ are closed, so is Z. 
        Let $\mathcal{C}=\{x_\alpha\}_{\alpha \in A}$ be a chain in  where $(A,\leq)$ is a totally ordered set and $x_\alpha \lesssim x_\beta \iff \alpha \leq \beta$.  
        If $x_\alpha,x_\beta \in \mathcal{C}$, where $x_\beta \lesssim x_\alpha$, then $x_\alpha-x_\beta \in K(x^*,\epsilon)$, so $0 \leq ||x_\alpha-x_\beta|| \leq \epsilon \ip{x_\alpha-x_\beta,x^*}$, implying $\ip{x_\beta,x^*} \leq \ip{x_\alpha,x^*}$. Thus we conclude $\{\ip{x_\alpha,x^*}\}_{\alpha \in A}$ is a monotone bounded net in $\R$ that is therefore Cauchy, which by the following inequality 
        \begin{equation}
            ||x_\beta-x_\alpha|| \leq \epsilon \ip{x_\alpha-x_\beta,x^*} = \epsilon \pa{ \ip{x_\alpha,x^*}-\ip{x_\beta,x^*}} \to 0
        \end{equation}
        implies $\mathcal{C}$ is a Cauchy net and therefore converges, say $x_\alpha \to y_0 \in Z$. Continuity of the norm and $x^*$ imply together that $y_0$ is an upper bound for $\mathcal{C}$.  Since $\mathcal{C}$ was an arbitrary chain in Z, $Z$ has a maximal element $x_0$.
        By definition, $x_0 \in Z:=K(x^*,\epsilon)+z$.
        Since $x_0 \in Z \subset C$, $x_0 \in C$. Further, since $0 \in K(x^*,\epsilon)$, $x_0 \in K(x^*,\epsilon) \cap C$. 
        Let $y \in (K(x^*,\epsilon)+x_0) \cap C$. 
        Then $y-x_0 \in K(x^*,\epsilon)$ so that $z \lesssim x_0 \lesssim y$, meaning $y \in Z$ and therefore $y=x_0$ since $x_0$ is maximal. 
        Hence $(K(x^*,\epsilon)+x_0 ) \cap C=\{x_0\}$, so we are done. 
    \end{proof}
    \begin{proof}(2)
        By assumption, $\norm{y^*|_{Kern(x^*)}} \leq \frac{\epsilon}{2}$, so by the Hahn-Banach theorem, we can find a $\tilde{y^*} \in X^*$ extending $y^*|_{Kern(x^*)}$ such that $\norm{\tilde{y^*}}\leq \frac{\epsilon}{2}$. 
        Since $y^*-\tilde{y^*} \neq 0$,  $codim\pa{kern(x^*)}=1$, and  $kern \pa{x^*} \subset kern\pa{y^*-\tilde{y^*}}$,we conclude $kern\pa{y^*-\tilde{y^*}}=kern(x^*)$. Hence, for some $\alpha \in \R$, $y^*-\tilde{y^*}=\alpha x^*$. 
        For this alpha, we have 
        \begin{equation}
            \abs{1-|\alpha|} = \abs{\norm{y^*}-\norm{\tilde{y^*}-y^*}} \leq \norm{\tilde{y^*}} \leq \frac{\epsilon}{2}
        \end{equation}
        If $\alpha \geq 0$, 
        \begin{equation} 
            \norm{x^*-y^*} =\norm{x^*-\pa{\alpha x^*+\tilde{y^*}}}=\norm{\pa{1-\alpha}x^*-\tilde{y^*}} \leq |1-\alpha|+\norm{\tilde{y^*}} \leq \epsilon
        \end{equation}
        If $\alpha \leq 0$, then 
        \begin{equation} 
            \norm{x^*+y^*} = \norm{x^*+\pa{\alpha x^*+\tilde{y^*}}} = \norm{(1+\alpha)x^*+\tilde{y^*}} \leq |1+\alpha| + \norm{\tilde{y^*}} \leq \epsilon
        \end{equation}
    \end{proof}
    \begin{proof}(3)
        Since $||x^*||=1$, there exists $x \in \partial B_X(0;1)$ such that $\ip{x,x^*} > \frac{1}{k} \pa{1+\frac{2}{\epsilon}}$. 
        If $y \in Kern(x^*) \cap \overline{B_X(0;1)}$, then \begin{equation}
            \norm{x \pm \frac{2}{\epsilon}y} \leq 1+\frac{2}{\epsilon} < k \ip{x,x^*}=k \ip{x \pm \frac{2}{\epsilon}y,x^*}
        \end{equation}
        so $x \pm \frac{2}{\epsilon}y \in K(x^*,k)$, so by assumption $\ip{x \pm \frac{2}{\epsilon}y,y^*} \geq 0$. 
        Since this occurs for both positive and negative, $\abs{\ip{y,y^*}}=\frac{\epsilon}{2}\abs{\ip{\frac{2}{\epsilon}y,y^*}} \leq \frac{\epsilon}{2}\ip{y^*,x} \leq \frac{\epsilon}{2}||x||=\frac{\epsilon}{2}$. 
        Hence by part 2, either $||x^*-y^*|| \leq \epsilon$, or $||x^*+y^*|| \leq \epsilon$. 
        Since $||x^*||=1$, there exists $x \in \partial B_X(0;1)$ such that $\frac{||x||}{k} \leq max \pa{\epsilon, \frac{1}{k}} < \ip{x,x^*}$, so that $x \in K(x^*,k)$, implying $\ip{x,y^*} \geq 0$, and therefore $\epsilon < \ip{x_0,x^*+y^*} \leq \norm{x^*+y^*}$. Hence we conclude $\norm{x^*-y^*} \leq \epsilon$. 
        
    \end{proof} 
\end{lem}
\begin{thm}[Bishop-Phelps Theorem]
    
    
    Let X be a complete seminormed space, 
    $C \subset X$ be closed, bounded, and convex, 
    and define $M:=\{f \in X^*| (\exists x_0 \in C)(\ip{x_0,f}=\sup\limits_{x \in C}\ip{x,f})\}$.
    Then $\overline{M}=X^*$
    \begin{proof}
        Since M is a vector subspace independent of translations of C, we assume without loss of generality that $0 \in C$ and that it is sufficient to show that M is dense in $\partial B_{X^*}(0;1)$. 
        Let $x^* \in \partial B_{X^*}(0;1)$.
        Let $\epsilon \in (0,1)$ and let $1+\frac{2}{\epsilon} < k$. by \ref{rmk:ClosedConvexCone}, $K(x^*,k)$ is a closed convex cone with nonempty interior. Applying \ref{lem:BishopPhelps}, part one, there is $x_0 \in C$ with $x_0 \in K(x^*,k)$ and $\pa{K(x^*,k)+x_0} \cap C = \{x_0\}$. 
        By Hahn Banach, there exists $y^* \in \partial B_{X^*}(0;1)$ satisfying
        \begin{equation}
            \sup\limits_{x \in C} \ip{x,y^*} = \ip{x_0,y^*} = \inf\limits_{x \in K(x^*,k)+x_0} \ip{x,y^*} = \inf\limits_{\tilde{x} \in K(x^*,k)} \ip{\tilde{x},y^*}+ \ip{x_0,y^*}
        \end{equation}
        Hence $y^*$ is positive on $K(x^*,k)$, so by \ref{lem:BishopPhelps} part 3, $||x^*-y^*||< \epsilon$, so we are done since $y^* \in M$ and $x^*$ was arbitrary. 
    \end{proof} 
\end{thm}
