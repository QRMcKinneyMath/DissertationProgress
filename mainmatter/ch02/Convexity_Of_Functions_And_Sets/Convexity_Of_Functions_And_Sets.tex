
\section{Convexity Of Functions And Sets}
\newcommand{\ConvexFunction}[0]{\textbf{\hyperref[def:ConvexFunction]{Convex}}\xspace}
\newcommand{\StrictlyConvexFunction}[0]{\textbf{\hyperref[def:ConvexFunction]{Strictly Convex}}\xspace}
\begin{df}[\ConvexFunction]
\label{def:ConvexFunction}
\rm
Let $\scF \in \{\bbR, \bbC\}$. 
Let $V$ be a \VectorSpace over $\scF$. 
Let $T:V \to (-\infty,\infty]$
We say that $T$ is \ConvexFunction if for each $x,y \in V$, 
\begin{equation*}
T\pa{\frac{x+y}{2}} \leq \frac{T(x)+T(y)}{2}
\end{equation*}
We say that $T$ is \StrictlyConvexFunction if for each $x,y \in V$, 
\begin{equation*}
T\pa{\frac{x+y}{2}} < \frac{T(x)+T(y)}{2}
\end{equation*}
\end{df}

\newcommand{\EffectiveDomain}[0]{\textbf{\hyperref[def:EffectiveDomain]{Effective Domain}}\xspace}
\newcommand{\ProperFunction}[0]{\textbf{\hyperref[def:EffectiveDomain]{Proper}}\xspace}
\newcommand{\Epigraph}[0]{\textbf{\hyperref[def:EffectiveDomain]{Epigraph}}\xspace}
\begin{df}[\EffectiveDomain]
\label{def:EffectiveDomain}
\rm
Let $X$ be a set and let $T:X \to (-\infty,\infty]$. 
We call $T^{-1}(\mathbb{R})$ the \EffectiveDomain of $T$ 
and we denote the \EffectiveDomain of $T$ with $D(T)$. 
If $D(T) \neq \emptyset$, we say that $T$ is \ProperFunction.
Define $Epi(T) = \{(x,t) : (x \in D(T))(t \geq T(x))\}$. 
We call $Epi(T)$ the \Epigraph of $T$. 
\end{df}

\begin{prop}[\ConvexFunction function]
\label{prop:ConvexFunction}
\rm
Let $\scF \in \{\bbC, \bbR\}$. 
Let $X$ be a \VectorSpace over $\scF$. 
The following are equivalent.
\begin{enumerate}[label=(\roman*), ref={\ref{prop:ConvexFunction}~\roman*}]
\item 
\label{prop:ConvexFunction:BinaryConvex}
$T$ is \ConvexFunction
and for each $x,y \in X$, $T\pa{[x,y]}$ is bounded from above. 
\item
\label{prop:ConvexFunction:Convex}
For each $x,y \in X$ and for each $\lambda \in (0,1)$, 
\begin{equation*}
T\pa{\lambda x + \pa{1-\lambda}y} \leq \lambda T(x) + \pa{1-\lambda}T(y)
\end{equation*}
\item
\label{prop:ConvexFunction:SumConvex}
For every $\{\lambda_i\}_{i=1}^n \subset (0,1)$ such that $\sum\limits_{i=1}^n \lambda_i = 1$
and for every $\{x_i\}_{i=1}^n \subset X$ where each $x_i$ is distinct, , 
\begin{equation*}
T\pa{\sum\limits_{i=1}^n \lambda_i x_i} \leq \sum\limits_{i=1}^n \lambda_i T\pa{x_i}
\end{equation*}
\item
\label{prop:ConvexFunction:ConvexInequality}
For each $x,y,z \in X$ with $x \neq z$ and $y = \lambda z + \pa{1-\lambda} x$ where $\lambda \in (0,1)$, we have 
\begin{equation*}
\frac{T(y)-T(x)}{\lambda} \leq T(z)-T(x) \leq \frac{T(z)-T(y)}{1-\lambda}
\end{equation*}
\item
\label{prop:ConvexFunction:EpigraphConvex}
$Epi(T)$ is \ConvexSet
\end{enumerate}
\begin{proof}[Proof of \ref{prop:ConvexFunction:BinaryConvex} implies  \ref{prop:ConvexFunction:Convex}]
Let $x,y \in X$. 
Let $M$ be the collection of all $n \in \bbZ^+$ such that 
for all $\lambda \in (0,1)$
there exists nonnegative integers $j_n$ and $k_n$ with $j+k = 2^n-1$, 
$\frac{j_n}{2^n} \leq \lambda$ and $\frac{k_n}{2^n} \leq 1-\lambda$ 
and there exists $\lambda_n \in (0,1)$ such that 
\begin{equation*}
T\pa{\lambda x + (1-\lambda)y} \leq \frac{j_n}{2^n}T(x) + \frac{k_n}{2^n}T(y) + \frac{1}{2^n}T\pa{\lambda_n x + (1-\lambda_n) y}
\end{equation*}
I first prove $1 \in M$. 
Without loss of generality, suppose $\lambda \geq \frac{1}{2}$. 
Define $\lambda_1 = 2\pa{\lambda - \frac{1}{2}}$. Then $\lambda_1 \in (0,1)$, $\lambda = \frac{\lambda_1}{2} + \frac{1}{2}$, and $2(1-\lambda) = 1-\lambda_1$
\begin{align*}
T\pa{\lambda x + (1-\lambda)y} & = T\pa{\frac{1}{2} x + \frac{1}{2} \lambda_1 x +\frac{2}{2}\pa{1-\lambda}y} \\
& = T\pa{ \frac{1}{2} x + \frac{1}{2} \pa{\lambda_1 x + \pa{1-\lambda_1}y}} \\
& \leq \frac{1}{2} T(x) + \frac{1}{2} T\pa{\lambda_1 x + (1-\lambda_1) y}
\end{align*}
Since $\frac{1}{2} \leq \lambda$ and $0 \leq 1-\lambda$, $1 \in M$. 
Now suppose $k \in M$. 
Then since $1 \in M$. There are integers $a$ and $b$ with either $a=0$ and $b=1$ 
or with $a=1$ and $b=0$ and there is a $\lambda_{k+1} \in [0,1]$ such that
\begin{equation*}
T\pa{\lambda_k x +(1-\lambda_k)y} \leq  \frac{a}{2}T(x)+\frac{b}{2}T(y)+\frac{1}{2}T\pa{\lambda_{k+1} x+(1-\lambda_{k+1})y}
\end{equation*}
Hence 
\begin{align*}
T\pa{\lambda x + (1-\lambda) y } &\leq \frac{j_k}{2^k}T(x) + \frac{m_k}{2^k}T(y) + \frac{1}{2^k} T\pa{\lambda_k x + (1-\lambda_k)y}\\
& \leq \frac{j_k}{2^k}T(x)+\frac{m_k}{2^k}T(y) + \frac{1}{2^k} \pa{\frac{a}{2}T(x)+\frac{b}{2}T(y)+\frac{1}{2}T\pa{\lambda_{k+1}x+(1-\lambda_{k+1}T(y)}}\\
& = \frac{2j_k+a}{2^{k+1}}T(x) + \frac{2m_k+b}{2^{k+1}} + \frac{1}{2^{k+1}} T\pa{\lambda_{k+1}x+(1-\lambda_{k+1})y}
\end{align*}
where we have 
\begin{equation*}
2j_k+a+2m_k+b=2\pa{j_k+m_k}+a+b = 2\pa{2^k-1}+1=2^{k+1}-1
\end{equation*}
so that $k+1 \in M$.
Hence, $M = \mathbb{Z}^+$. 
Thus, for each $n \in \mathbb{N}$, if $\Gamma$ is an $\UpperBound$ for $T([x,y])$, 
we have 
\begin{equation*}
T\pa{\lambda x + (1-\lambda)y} \leq \frac{j_n}{2^n}T(x) + \frac{m_n}{2^n} T(y) + \frac{\Gamma}{2^n}
\end{equation*}
Now since $\frac{j_n}{2^n} \leq \lambda$, $\frac{m_n}{2^n} \leq 1-\lambda$, and $\frac{j_n}{2^n}+\frac{m_n}{2^n} \to 1$, it is clear that $\frac{j_n}{2^n} \to \lambda$ and $\frac{m_n}{2^n} \to 1-\lambda$, so 
clearly the result holds. 
\end{proof}
\begin{proof}[Proof of \ref{prop:ConvexFunction:Convex} implies  \ref{prop:ConvexFunction:SumConvex}]
        I utilize induction on n. 
        Since $f$ is \ConvexFunction, the proposition holds for $n=1$ and $n=2$. 
        Let $k \in \N$.
        Suppose that for any $(\psi_1,\cdots,\psi_k)\in (0,1)^k$ such that $\sum_{j=1}^k \psi_j=1$, and for each $(x_1,\cdots,x_k) \in X^k$, we have
        \begin{equation*}
            f\pa{\sum_{j=1}^k \lambda_j x_j} \leq \sum_{j=1}^k \lambda_j f(x_j)
        \end{equation*}
        Let $(\lambda_1,\cdots,\lambda_{k+1}) \in [0,1]^{k+1}$. 
        Let $(x_1,\cdots,x_{k+1}) \in X^{k+1}$.
        Then 
        \begin{equation*}
            \sum_{j=1}^k \frac{\lambda_j}{1-\lambda_{k+1}}=1 \tab[1cm] \pa{\frac{\lambda_1}{1-\lambda_{k+1}},\cdots,\frac{\lambda_k}{1-\lambda_{k+1}}} \in (0,1)^k
        \end{equation*}
        Since $f$ is \ConvexFunction, , 
        \begin{equation*}
            \begin{split}
                f \pa{ \sum_{j=1}^{k+1} \lambda_j x_j} & = f\pa{ \lambda_{k+1}x_{k+1} + (1-\lambda_{k+1}) \sum_{j=1}^k \frac{\lambda_j}{1-\lambda_{k+1}} x_j}\\
                & \leq \lambda_{k+1} f(x_{k+1}) + (1-\lambda_{k+1}) f\pa{\sum_{j=1}^k \frac{\lambda_j}{1-\lambda_{k+1}} x_j} \\
                & \leq \sum_{j=1}^{k+1} \lambda_j f(x_j)
            \end{split}
        \end{equation*}


\end{proof}
\begin{proof}[Proof of \ref{prop:ConvexFunction:SumConvex} if and only if   \ref{prop:ConvexFunction:ConvexInequality}]
For the first inequality, we have 
\begin{equation*}
T(y) = T\pa{\lambda_z + (1-\lambda)x } \leq \lambda T(z)+(1-\lambda)T(x)
\end{equation*}
which is equivalent to 
\begin{equation*}
T(y) -T(x) \leq \lambda \pa{T(z)-T(x)}
\end{equation*}
which is equivalent to
\begin{equation*}
\frac{T(y)-T(x)}{\lambda} \leq T(z) - T(x)
\end{equation*}

For the second inequality, 
\begin{equation*}
T(y) \leq \lambda T(z) + (1-\lambda)T(x)
\end{equation*}
which is equivalent to
\begin{equation*}
-\lambda T(z)+T(z)-\pa{1-\lambda}T(x) \leq T(z)-T(y)
\end{equation*}
which is equivalent to 
\begin{equation*}
\pa{1-\lambda}\pa{T(z)-T(x)} \leq T(z)-T(y)
\end{equation*}
which is equivalent to 
\begin{equation*}
T(z)-T(x) \leq \frac{T(z)-T(y)}{1-\lambda}
\end{equation*}
\end{proof}
\begin{proof}[Proof of \ref{prop:ConvexFunction:ConvexInequality} implies  \ref{prop:ConvexFunction:BinaryConvex}]
\end{proof}
\begin{proof}[Proof of \ref{prop:ConvexFunction:Convex} implies  \ref{prop:ConvexFunction:EpigraphConvex}]
\end{proof}
\begin{proof}[Proof of \ref{prop:ConvexFunction:EpigraphConvex} implies  \ref{prop:ConvexFunction:Convex}]
\end{proof}
\end{prop}

\begin{df}[Convex Functions]
    \label{df:convexfunction}
    Let $X$ be a vector space, Y a topological vector space, $\scU$ the set of neighborhoods of 0 in Y except Y itself, $f:X \to (-\infty,\infty]$, and $g:Y \to (-\infty,\infty]$.
    Let $x,y \in X$. 
    \begin{enumerate}
        \item We call $D(f):= f^{-1}\pa{\R}$ the \bf effective domain \rm of $f$. 
        \item If $D(f) \neq \emptyset$, then we call $f$ \bf proper.\rm
        \item We call $Epi\pa{f} :=\{ (x,t) \in X \times \R : f(x) \leq t \}$ the \bf Epigraph \rm of $f$. 
        \item We say that $C \subset X$ is \bf strictly convex \rm if for each $x,y \in C$, for each $z_0 \in (x,y)$, and for each $z_1 \in X$, there is a $t > 0$ such that $[z_0,z_0+tz_1] \subset C$. 
        \item If g is a convex function, then then we define the \bf modulus of local uniform convexity \rm of g, $\tilde{\Delta}:\scU \times Y\to \R$ by
        \begin{equation}
            \tilde{\Delta} \pa{U, x} = \frac{1}{2} \inf\limits_{y \in Y \setminus (x+U)} \left\{ f(x)+f(y)-2f\pa{\frac{x+y}{2}}\right\}
        \end{equation}
        \item If g is a convex function, then we define the \bf modulus of uniform convexity \rm of g, $\Delta:\scU \to \R$ by  $\Delta(U) = \inf\limits_{g(x)=1} \tilde{\Delta}(U,x)$. 
        \item If g is a convex function, then we say that g is \bf locally uniformly convex  at \rm  $x \in Y$ if for each $U \in \scU$, $\tilde{\Delta}(U,x) > 0$. 
        \item We say that g is \bf locally uniformly convex \rm if it is \bf locally uniformly convex \rm  at each of its points. 
        \item We say that g is \bf uniformly convex \rm if for each $U \in \scU$, $\Delta(U)>0$. 
        \item We say that g is \bf lower semi-continuous\rm, or LSC if it is continuous with respect to the topology on $(-\infty, \infty]$ generated by sets of the form $(-\infty,\alpha)$ where $\alpha \in \R$, along with $(-\infty,\infty]$ itself. 
    \end{enumerate} 
\end{df} 
\begin{rmk}[Basis Independence]
    It is easy to see that a mapping is locally uniform convex at a point (locally uniformly convex) [uniformly convex] if we define $\Delta$, $\tilde{\Delta}$ in terms of a single neighborhood basis of Y at 0 instead of all neighborhoods of 0 in Y.
\end{rmk} 

\begin{rmk}[Strictly Convex Real Valued]
    \label{rmk:strictlyconvexrealvalued}If X is a vector space and $f:X \to (-\infty,\infty]$ is strictly convex, and f is finite everywhere.
    \begin{proof}
        If $f(x)= \infty$  where $x \in X$ and f is strictly convex, then we must have $\infty = f(x) < \frac{f(0)+f(2x)}{2}$, a contradiction. 
    \end{proof} 
\end{rmk} 

\begin{prop} 
    Let X be a vector space and $T:X \to (-\infty,\infty]$. Then the following are equivalent. 
    \begin{enumerate}
        \item T is (strictly) convex.
        \item For each $x_1 \neq x_2 \in X$ and $\lambda \in (0,1)$.
        \begin{equation}
            T\pa{\lambda x_1 + (1-\lambda )x_2} \pa{<} \leq \lambda Tx_1 + (1-\lambda)Tx_2
        \end{equation} 
        \item For each $\{\lambda_i\}_{i=1}^n \subset (0,1)$ which sums to 1, for each $\{x_i\}_{i=1}^n \subset X$, 
        \begin{equation}
            T\pa{\sum_{j=1}^n \lambda_j  x_j } \pa{<}\leq \sum_{j=1}^n \lambda_j Tx_j
        \end{equation}
        \item If $x_1 \neq x_3 \in X$ and $x_2 \in (x_1,x_3)$, say $x_2=\lambda x_1+(1-\lambda)x_3$ then 
        \begin{equation} 
            \frac{Tx_2-Tx_1}{\lambda} \pa{<}\leq Tx_3-Tx_1 \pa{<}\leq \frac{Tx_3-Tx_2}{1-\lambda}
        \end{equation}
        \item $Epi(T)$ is (strictly) convex
    \end{enumerate} 
    \begin{proof} $(1 \implies  2)$
        
    \end{proof}
    \begin{proof}$(2 \implies 3)$
        I utilize induction on n. 
        Since f is assumed to be (strictly) convex, the proposition holds for $n=1$ and $n=2$. 
        Let $k \in \N$.
        Suppose that for any $(\psi_1,\cdots,\psi_k)\in [0,1]^k$ such that $\sum_{j=1}^k \psi_j=1$, and for each $(x_1,\cdots,x_k) \in X^k$, we have
        \begin{equation}
            f\pa{\sum_{j=1}^k \lambda_j x_j} \pa{<}\leq \sum_{j=1}^k \lambda_j f(x_j)
        \end{equation}
        Let $(\lambda_1,\cdots,\lambda_{k+1}) \in [0,1]^{k+1}$. Let $(x_1,\cdots,x_{k+1}) \in X^{k+1}$. Without loss of generality, we assume $\lambda_{k+1} \neq 0$. 
        Then 
        \begin{equation}
            \sum_{j=1}^k \frac{\lambda_j}{1-\lambda_{k+1}}=1 \tab[1cm] \pa{\frac{\lambda_1}{1-\lambda_{k+1}},\cdots,\frac{\lambda_k}{1-\lambda_{k+1}}} \in [0,1]^k
        \end{equation}
        Hence, because f is (strictly) convex, 
        \begin{equation}
            \begin{split}
                f \pa{ \sum_{j=1}^{k+1} \lambda_j x_j} & = f\pa{ \lambda_{k+1}x_{k+1} + (1-\lambda_{k+1}) \sum_{j=1}^k \frac{\lambda_j}{1-\lambda_{k+1}} x_j}\\
                & \pa{<}\leq \lambda_{k+1} f(x_{k+1}) + (1-\lambda_{k+1}) f\pa{\sum_{j=1}^k \frac{\lambda_j}{1-\lambda_{k+1}} x_j} \\
                & \pa{<}\leq \sum_{j=1}^{k+1} \lambda_j f(x_j)
            \end{split}
        \end{equation}
    \end{proof}
    \begin{proof}$(3 \implies 4)$
    \end{proof} 
    \begin{proof} $(4 \implies 1)$. 
    \end{proof} 
    \begin{proof} $(2 \iff 5)$
    \end{proof} 
\end{prop} 
The epigraph of a function provides us with a nice characterization of lower semi-continuity. 
\begin{prop}[Convex Continuity]
    \label{prop:convexcontinuity}
    Let $X$ be a locally convex space and $f:X \to (-\infty,\infty]$. The following conditions are equivalent. 
    \begin{enumerate}
        \item f is LSC on X. %cioranescu3
        \item $Epi(f)$ is closed in $X \times \R$. 
    \end{enumerate} 
    \begin{proof}
        Define $F:X \times \R \to \tilde{\R}$ by $F(x,\alpha)=f(x)-\alpha$. 
        Then $f$ is (weakly) LSC on $X$ if and only if $F$ is (weakly) LSC on $X \times \R$. 
        Suppose $f$ is (weakly) LSC.
        Then $F$ is (weakly) LSC, so $F^{-1}((-\infty,0])=Epi(f)$ is (weakly) closed, so we're done. 
        Suppose $Epi(f)$ is (weakly) closed. 
        Then $F^{-1}((-\infty,0])$ is (weakly) closed. 
        Further, for any $\beta \in \R$, $F^{-1}((-\infty,\beta])=F^{-1}((-\infty,0])-(0,\beta)$, and so is also closed. 
        Hence $F$ is (weakly) LSC, and so $f$ is too. 
    \end{proof} 
\end{prop}
In the case of a convex function, the above proposition allows us to equate weak and strong lower semicontinuity.
\begin{cor}[Weak To Strong Convex]
    \label{cor:weaktostrongconvex}
    Let X be locally convex Hausdorff space  and $f:X \to (-\infty,\infty]$ be convex. Then $f$ is LSC if and only if it is weakly LSC. 
    \begin{proof}
        Since $Epi(f)$ is convex, it is closed if and only if it is weakly closed, allowing us to apply \ref{prop:convexcontinuity}.
    \end{proof} 
\end{cor} 
\begin{thm}[Point Continuous]
    \label{thm:pointcontinuous}
    Let X be a locally convex space and $f:X \to (-\infty,\infty]$ be convex and proper. Then $f$ is bounded on some  open set if and only if $f$ is continuous on the interior of its domain. 
    \begin{proof} $(implies)$
        Without loss of generality, we assume that f is bounded from above by M on a (weakly) open set $\scU$ which is symmetric and contains 0. Further, we can also suppose $f(0)=0$. 
        For each $\epsilon \in (0,1)$ and each $x \in \epsilon \scU$, we have 
        \begin{equation}
            f(x) = f\pa{ \epsilon \frac{x}{\epsilon} + (1-\epsilon)0}  \leq \epsilon f\pa{ \frac{x}{\epsilon}} \leq \epsilon M
        \end{equation} 
        and
        since $0=\frac{x}{1+\epsilon}+\pa{1-\frac{1}{1+\epsilon} \pa{\frac{-x}{\epsilon}}}$, 
        \begin{equation}
            0=f(0)=f\pa{ \frac{x}{1+\epsilon} + \pa{1-\frac{1}{1+\epsilon}} \pa{\frac{-x}{\epsilon}}} \leq \frac{f(x)}{1+\epsilon}+\frac{\epsilon f\pa{-\frac{x}{\epsilon}}}{1+\epsilon}
        \end{equation} 
        so, since $\frac{-x}{\epsilon} \in \scU$, 
        \begin{equation} 
            -\epsilon M \leq -\epsilon f\pa{ \frac{x}{\epsilon}} \leq f(x)
        \end{equation}
        Hence, $|f(x)| \leq \epsilon M$ for $x \in \epsilon \scU$, and f is (weakly) continuous at 0. 
        Hence, it is sufficient to show that for any $y$ in the (weak) interior of $D(f)$, there is a (weak) neighborhood of y on which f is bounded from above. 
        To see this, let $y$ in the (weak) interior of $ D(f)$. 
        Since scalar multiplication is continuous, there is a $\rho >1$ such that $\rho y \in D(f)$. If $\scU_y=y+\pa{1-\frac{1}{\rho}}\scU$, then $x \in \scU_y$ can be written, for some $z \in \scU$, as
        \begin{equation}
            x=y+\pa{1-\frac{1}{\rho}}z=\frac{1}{\rho}(\rho y) + \pa{1-\frac{1}{\rho}}z
        \end{equation}
        Since f is convex, $D(f)$ is convex, and so $x \in D(f)$, implying $\scU_y \subset D(f)$. 
        Since f is a convex function, we also have that 
        \begin{equation}
            f(x) \leq \frac{1}{\rho} f(\rho y)+ \pa{1-\frac{1}{\rho}}f(x) \leq \frac{1}{\rho} f(\rho y) + \pa{ 1-\frac{1}{\rho}} M
        \end{equation}
        so that f is bounded on $\scU_y$ and is therefore continuous at y.
    \end{proof}
    \begin{proof}$(\impliedby)$
        This is obvious. 
    \end{proof} 
\end{thm} 
\begin{cor} Let $X$ be a locally convex, space and $f:X \to (-\infty,\infty]$ be LSC at some point in its effective domain. 
    \begin{enumerate}
        \item if $f$ is convex, then it is continuous on the interior of $D(f)$. 
        \item If $f$ is strictly convex, then it is continuous on X. 
    \end{enumerate} 
\end{cor}
