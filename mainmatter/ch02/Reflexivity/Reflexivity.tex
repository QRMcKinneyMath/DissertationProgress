\subsection{Reflexivity Results}
\begin{rmk}
\rm
Recall that a \SeminormedSpace $X$ is said to be \Reflexive if $c(X)=X^{**}$. 
Since $X^{**}$ is always \Complete and $c$ an \Isometry, any \Reflexive space  \Complete.
Due to the Banach-Alaoglu theorem, 
the \ClosedUnitBall of a \Reflexive space
\weakly \SetCompact. 
For this reason and others, 
\Reflexivity is a condition of interest to many mathematicians. 
We begin with a basic result. 
\end{rmk}
\begin{lem}[Reflexive Separable]
\label{lem:ReflexiveSeparable}
\rm
Let $X$ be a \Complete \SeminormedSpace.
The following conditions are equivalent. 
\begin{enumerate}[label=(\roman*), ref={\ref{lem:ReflexiveSeparable}~\roman*}]
\item 
\label{lem:James:Reflexive}
X is \Reflexive.
\item 
\label{lem:James:WeaklyCompact}
The \ClosedUnitBall of $X$ is \weakly \SetCompact. 
\item 
\label{lem:James:ReflexiveSubspace}
Each \SetClosed \TopologySeparable subspace of $X$ is \Reflexive. 
%\item 
%\label{lem:James:FiniteIntersectionProperty}
%All collections of \SetClosed, bounded, \ConvexSet sets in $X$ possessing the 
%\FiniteIntersectionProperty have nonempty intersections.
\item 
\label{lem:James:WeaklySequentiallyCompact}
The \ClosedUnitBall of $X$ is \weakly sequentially \SetCompact.
\end{enumerate} 
\begin{proof}[Proof of \ref{lem:James:Reflexive} implies \ref{lem:James:WeaklyCompact}]
Let $X$ be \Reflexive. 
By \ref{thm:eberleinsmulian} it is sufficient to show that any sequence $\{x_i\}_{i \in \N} \subset \overline{B_X(0;1)}$ has a \weak cluster point $x \in \overline{B_X(0;1)}$. 
Since $\overline{B_{X^{**}}(0;1)}$ is $\weakstar$ \SetCompact, 
$\{c(x_i)\}_{i \in \N}$ has a subsequence $\{c(x_{n_i})\}_{i \in \N}$ such that $c(x_{n_i}) \overset{w^*}{\to} \tilde{x} \in \overline{B_{X^{**}}(0;1)}$. 
Since $X$ is \Reflexive, for some $x \in \overline{B_X(0;1)}$, $c(x)= \tilde{x}$. 
Let $x^* \in X^*$. Then,
\begin{equation*}
\lim\limits_{i \to \infty} \abs{\ip{x_{n_i}-x,x^*}} =\lim\limits_{i \to \infty} \abs{\ip{x^*,c(x_{n_i})-\tilde{x}}} =0
\end{equation*}
Hence, $x_{n_i} \overset{w}{\to} x$, completing the proof. 
\end{proof}
\begin{proof}[Proof of \ref{lem:James:WeaklyCompact} implies \ref{lem:James:Reflexive}]
Suppose the \ClosedUnitBall of $X$ is \weakly \SetCompact, 
and let $x^{**} \in \overline{B_{X^{**}}(0;1)}$. 
By \ref{thm:GoldstinesTheorem}, 
there is a sequence $\{x_n\}_{n \in \N} \subset \overline{B_X(0;1)}$ 
such that $c(x_n) \overset{w^*}{\to}  x^{**}$. 
By assumption, there is a subsequence $\{x_{n_k}\}_{k \in \N}$ 
such that $x_{n_k} \overset{w}{\to} x \in \overline{B_X(0;1)}$. 
This implies $c(x_{n_k}) \overset{w^*}{\to} c(x)$, 
and so since the $\weakstar$ \Topology is \Hausdorff,  $x^{**}=c(x) \in c\pa{\overline{B_X(0;1)}}$. 
\end{proof}
\begin{proof}[Proof of \ref{lem:James:WeaklyCompact} $\implies$ \ref{lem:James:ReflexiveSubspace}]
Let $X$ be \Reflexive.
Let $S \subset X$ be a \SetClosed \TopologySeparable \VectorSubspace. 
Then $S$ is \weakly \SetClosed.
Let $\{x_i\}_{i \in \N} \subset S \cap \overline{B_X(0;1)}$. 
By  \ref{thm:eberleinsmulian}, $\overline{B_X(0;1)}$ is
\weakly sequentially \SetCompact.
Hence, there exists $x_0 \in S \cap \overline{B_X(0;1)}$ and 
a subsequence $\{x_{n_k}\}_{k \in \mathbb{N}}$ such that 
$x_0 \in  \lim\limits_{k \to \infty} x_{n_k}$, 
where convergence is with respect to the \weak \Topology on $X$.
By the Hahn-Banach extension theorem, 
the\weak \Topology on $S$ is \TopologyCoarser than the 
\SubspaceTopology on $S$ induced by the \weak \Topology on $X$.
Hence $x_{n_k} \to x_0$ in the \weak \Topology on $S$.
Thus $S \cap \overline{B_X(0;1)}$ is \weakly sequentially 
\SetCompact with respect to $S^*$. 
By \ref{thm:eberleinsmulian} and 
$\ref{lem:James:WeaklyCompact} \implies \ref{lem:James:Reflexive}$, this direction is complete.
\end{proof} 
\begin{proof}[Proof of \ref{lem:James:ReflexiveSubspace} $\implies$ \ref{lem:James:WeaklyCompact}]
Let $\{x_n\}_{n \in \N} \subset \overline{B_X(0;1)}$. 
Then since $S:=\overline{span\{x_i\}_{i \in \N}}$ is a \SetClosed \TopologySeparable \VectorSubspace, 
$\{x_n\}_{n \in \N}$ has an $S-\weakly$ convergent subsequence, $x_{n_k} \overset{S-w}{\to} x \in S$.  
If $x^* \in X^*$, then $x^*|_{S} \in S^*$, so that 
\begin{equation*}
\abs{\ip{x-x_{n_k},x^*}} = \abs{\ip{x-x_{n_k}, x^*|_{S}}} \to 0
\end{equation*}
Hence $x_{n_k} \overset{w}{\to} x$.
Since $\{x_i\}_{i \in \N} \subset S \cap \overline{B_X(0;1)}$ was arbitrary, 
$\overline{B_X(0;1)}$ is \weakly sequentially \SetCompact.
An application of \ref{thm:eberleinsmulian} finishes the proof. 
\end{proof} 
%\begin{proof}[Proof of \ref{lem:James:WeaklyCompact} $\implies$ \ref{lem:James:FiniteIntersectionProperty}] 
%Let $\{X_\alpha\}_{\alpha \in A}$ be a collection of 
%\Norm bounded
%\ConvexSet
%\SetClosed 
%sets
%possessing the 
%\FiniteIntersectionProperty.
%Then there exists $\gamma > 0$ and a $\beta \in A$ such that 
%$\{X_\beta \cap X_\alpha\}_{\alpha \in A}$ is 
%a collection of 
%\ConvexSet 
%\SetClosed
%sets each contained in $\overline{B_X(0;\gamma)}$. 
%Since each $X_{\beta} \cap X_{\alpha}$ is 
%\ConvexSet and \SetClosed, each is \weakly \SetClosed. 
%Since each is a \weakly \SetClosed subset of $\overline{B_X(0;\gamma)}$, 
%which is \weakly \SetCompact by assumption, 
%an application of 
%$\ref{prop:Compact:FIP} \iff \ref{prop:CompactFilter:Compact}$
%implies that 
%\begin{equation*}
%\bigcap\limits_{\alpha \in A} X_\alpha = \bigcap\limits_{\alpha \in A} \pa{X_\alpha \cap X_\beta} \neq \emptyset
%\end{equation*}
%\end{proof}
%\begin{proof}[Proof of \ref{lem:James:FiniteIntersectionProperty} $\implies$ \ref{lem:James:WeaklyCompact}] 
%\end{proof} 
\begin{proof}[Proof of \ref{lem:James:WeaklyCompact} $\iff$ \ref{lem:James:WeaklySequentiallyCompact}]
This result is a direct consequence of \ref{thm:eberleinsmulian}.
\end{proof} 
\end{lem} 
\begin{prop}[\Reflexive Dual]
\label{prop:ReflexiveDual}
\rm
Let $X$ be a \Complete \SeminormedSpace over $\F$.
Then $X$ is \Reflexive if and only if $X^*$ is \Reflexive. 
\begin{proof}
%Let $X$ be \Reflexive. 
%By \ref{}, 
%$\overline{B_X(0;1)}$ is \weakly \SetCompact.

Let $X$ be \Reflexive. 
Let $c_*$ denote the \CanonicalEmbedding of $X^*$. 
Let $c$ denote the \CanonicalEmbedding of $X$. 
Let $\tilde{j} \in X^{***}$. 
Define $j:X \to \F$ by 
$\ip{x,j} = \ip{c(x), \tilde{j}}$. 
Let $\alpha \in \F$ and $x,y \in X$. 
Then 
\begin{equation*}
\ip{\alpha x + y, j} = \ip{c\pa{\alpha x + y}, \tilde{j}} = \alpha \ip{c(x), \tilde{j}} + \ip{c(y), \tilde{j}} = \alpha \ip{x, j} + \ip{y, j}
\end{equation*}
so that $j$ is \Linear. 
$j$ is also \ContinuousFunction, as 
\begin{equation*}
\abs{\ip{x, j}} = \abs{\ip{c(x), \tilde{j}} }\leq \norm{c(x)} \norm{\tilde{j}} = \norm{x} \norm{\tilde{j}}
\end{equation*}
Furthermore, if $g^{*} \in X^{**}$, then 
there exists $g \in X$ such that $c(g)=g^*$ so that 
\begin{equation*}
\ip{g^*, \tilde{j}} = \ip{c(g), \tilde{j}} = \ip{g, j} = \ip{j, c(g)} = \ip{j, g^*} = \ip{g^*, c_*\pa{j}}
\end{equation*}
Since $g^* \in X^*$ was arbitrary, $\tilde{j}=c_*\pa{j}$.
Since $\tilde{j} \in X^{***}$ was arbitrary, $c_*$ is \Surjective and
$X^*$ is therefore \Reflexive. 

Suppose now that $X^*$ is \Reflexive. 
Then by the above implication, $X^{**}$ is \Reflexive. 
By 
\ref{lem:ReflexiveSeparable}, 
each \SetClosed \TopologySeparable subspace of $X^{**}$ is \Reflexive. 
Let $K \subset X$ be a \SetClosed \TopologySeparable subspace of $X$. 
Then $c(K)$ is a \SetClosed \TopologySeparable subspace of $X^*$. 
This implies $c(K)$ is \Reflexive, which implies 
$K$ is \Reflexive.
Hence, by \ref{lem:ReflexiveSeparable}, 
$X$ is \Reflexive.
\end{proof}
\end{prop}

\subsection{James}
\begin{rmk}
\rm
As an easy application of \ref{thm:BanachAlaogluMorales} and \ref{thm:eberleinsmulian}, 
for a \Reflexive space X, 
all $x^* \in X^*$ attain their \Norm.  
The converse of this fact was, 
for a time, 
an open question of considerable interest. 
This problem has thoroughly been studied in the \Hausdorff case, albiet in a 
piecemeal manner.
The result was first tackled by James in \cite{james50} 
under the added assumption that every space $Y$ isomorphic to 
X has the property that each $y^* \in Y^*$ attains its 
\Norm and that $X$ permits a \SchauderBasis. 
This result was rapidly improved by Klee in \cite{klee50} who 
dropped the assumption of the existence a \SchauderBasis.
and then by James again in \cite{james57} who proved the 
result in the case of a \TopologySeparable $X$. 
The question of the converse in a \BanachSpace was finally answered to the affirmative in \cite{james64reflexivity}, building on the arguments in \cite{james57}.
I prove that the \Hausdorff assumption is unnecessary, 
showing in steps that the result holds in a general 
\Complete \SeminormedSpace.
\end{rmk}|
\newcommand{\Colimit}[0]{\textbf{\hyperref[def:Colimit]{Colimit}}\xspace}
\begin{df}[\Colimit]
\label{def:Colimit}
\rm
Let $X$ be a \TVS. 
Let $\{x_i^*\}_{i \in N} \subset X^*$. 
We define 
\begin{equation*}
Colim\{x_i^*\}_{i \in \N} = \braces{x^* \in X^* : \pa{\forall x \in X}\pa{ \liminf\limits_{i \to \infty} \ip{x, x_i^*} \leq \ip{x, x^*} \leq \limsup\limits_{i \to \infty} \ip{x,x_i^*}}}
\end{equation*}
We call $Colim\{x_i^*\}_{i \in \N}$ the 
\Colimit of $\{x_i\}_{i \in \N}$. 
\end{df} 
\begin{rmk}[CoLim Nonempty]
\label{rmk:colimnonempty}
\rm
Let $X$ be a \Complete \SeminormedSpace.
Let $\{x_i^*\}_{i \in \N} \subset X^*$ be bounded. 
Then $CoLim\{x_i^*\}_{i \in \N} \neq \emptyset$
\begin{proof}
Since $\{x_i^*\}_{i \in \N}$ is bounded, 
by 
\ref{thm:BanachAlaogluMorales}
and
\ref{thm:eberleinsmulian}, 
it has a subsequence with a \weakstar limit $x^*$.
We have $x^* \in CoLim\{x_i^*\}_{i \in \N}$.
\end{proof} 
\end{rmk} 
\begin{lem}
\label{lem:james}
\rm
Let $X$ be a \Complete \SeminormedSpace.
Let $\alpha \in (0,1)$.
Let $\{x_i^*\}_{i \in \N} \subset X^*$.
Let $\{\beta_i\}_{i \in \N} \subset (0,1)$ such that $\sum_{i \in \N} \beta_i=1$. 
The following are true 
\begin{enumerate}[label=(\roman*), ref={\ref{lem:james}~\roman*}]
\item
\label{lem:james:Boundary}
If $\{x_i^*\}_{i \in \N} \subset \partial B_{X^*}(0;1)$ and $d\pa{ 0, \overline{co}\{x_i^*\}_{i \in \N}} \geq \alpha$ then there exists $\gamma \in [\alpha, 1]$ and 
$\{y_i^*\}_{i \in \N} \subset X^*$ such that for every $n \in \N$, 
\begin{equation*}
y_i^* \in \overline{co}\{x_j\}_{j \geq n} \tab[.5cm] \norm{\sum\limits_{j \in \N} \beta_j y_j^*} = \gamma \tab[.5cm] \norm{\sum\limits_{j=1}^n \beta_j y_j^*} < \gamma \pa{ 1-\alpha \pa{\sum\limits_{j=n+1}^{\infty} \beta_j}}
\end{equation*}
\item
\label{lem:james:ClosedBall}
If $\{x_i^*\}_{i \in \N} \subset \overline{ B_{X^*}(0;1)}$ and $d\pa{ 0, \overline{co}\{x_i^*\}_{i \in \N}-CoLim\{x_i\}_{i \in \N}} \geq \alpha$ then there exists $\gamma \in [\alpha, 2]$, there exists
$\{y_i^*\}_{i \in \N} \subset \overline{B_{X^*}(0;1)}$, and there exists
$y^* \in CoLim\{y_i^*\}_{i \in \N}$ such that 
\begin{equation*}
\norm{\sum\limits_{j \in \N} \beta_j \pa{y_j^*-y^*}} = \gamma
\end{equation*}
and
for every $n \in \N$, 
\begin{equation*}
\norm{\sum\limits_{j=1}^n \beta_j \pa{y_j^*-y^*}} < \gamma \pa{1-\alpha \pa{\sum\limits_{j=n+1}^{\infty} \beta_j}}
\end{equation*}
\end{enumerate}
\begin{proof}[Proof of \ref{lem:james:Boundary}]
There exists a sequence $\{\delta_i\}_{i \in \N} \subset (0,1)$ such that 
\begin{equation*} 
\sum_{i \in \N} \frac{\beta_i \delta_i}{\pa{ \sum\limits_{j=i+1}^{\infty} \beta_i} \pa{ \sum\limits_{j=i}^{\infty} \beta_i}} < 1-\alpha
\end{equation*} 
Define $\gamma_1 = d\pa{0,\overline{co}\{x_i^*\}_{i \in \N}}$.
Let $y_1^* \in \overline{co}\{x_i^*\}_{i \in \N}$ such that 
$\norm{y_1^*} \leq \gamma_1\pa{1+\delta_1}$.
From here, for each $n \geq 1$, we define $\gamma_{n+1}\in \R$ by 
\begin{equation*}
\gamma_{n+1}=\inf\left\{ \norm{\pa{\sum_{i=1}^n \beta_i y_i^*}+ \pa{1-\sum_{i=1}^n \beta_i}y^*}: y^* \in \overline{co}\{x_i^*\}_{i \geq {n+1}}\right\}
\end{equation*}
and we choose $y_{n+1}^* \in \overline{co}\{x_i^*\}_{i \geq n+1}$ such that  
\begin{equation*}\norm{\sum_{i=1}^n \beta_i y_i^* + \pa{ 1- \sum_{i=1}^n \beta_i } y_{n+1}^*} < \gamma_{n+1} \pa{1+\delta_{n+1}}
\end{equation*}
For each $n \in \N$, 
\begin{equation*}
\frac{\beta_n}{1-\sum\limits_{i=1}^{n-1}\beta_i} y_n^* + \frac{1-\sum\limits_{i=1}^n \beta_i}{1-\sum\limits_{i=1}^{n-1} \beta_i} \overline{co}\{x_i\}_{i \geq n+1} \subset \overline{co} \{x_i\}_{i \geq n}
\end{equation*}
Hence, 
\begin{equation*}
\sum\limits_{i=1}^n \beta_i y_i^* + \pa{1-\sum\limits_{i=1}^n \beta_i }\overline{co}\{x_i^*\}_{i \geq n+1}  \subset \sum\limits_{i=1}^{n-1} \beta_i y_i^* + \pa{1-\sum\limits_{i=1}^{n-1} \beta_i }\overline{co}\{x_i^*\}_{i \geq n}
\end{equation*}
Thus $\{\gamma_i\}_{i \in \N}$ is nondecreasing.
Furthermore, since $\{x_i^*\}_{i \in \N} \subset \overline{B_{X^*}(0;1)}$ and $\sum\limits_{i \in \N} \beta_i=1$, we have, for every i, 
\begin{equation*}
\alpha \leq \gamma_i \nearrow  \norm{\sum_{k \in \N} \beta_k y_k^*} \leq 1
\end{equation*}
Define $\gamma = \lim\limits_{i \to \infty} \gamma_i$.
What is left to be shown is that for every $n \in \N$, 
\begin{equation*}
\norm{\sum_{j=1}^n \beta_j y_j^*} < \gamma \pa{1-\alpha \pa{ \sum_{j=n+1}^\infty \beta_j}}
\end{equation*}
Let $n \in \N$. Then, 
\begin{align*}
\norm{\sum_{j=1}^n \beta_j (y_j^*)} & = \norm{ \pa{\pa{ \frac{\sum_{j=n}^\infty \beta_j}{\sum_{j=n}^\infty \beta_j}} \pa{\sum_{j=1}^{n-1} \beta_j y_j^*}}+ \pa{\pa{ \frac{\sum_{j=n}^\infty \beta_j}{\sum_{j=n}^\infty \beta_j}}\pa{\beta_n y_n^*}}}\\
& = \norm{ \pa{\pa{ \frac{\beta_n+\sum_{j=n+1}^\infty \beta_j}{\sum_{j=n}^\infty \beta_j}} \pa{\sum_{j=1}^{n-1} \beta_j y_j^*}}+ \pa{\pa{ \frac{\beta_n\sum_{j=n}^\infty \beta_j}{\sum_{j=n}^\infty \beta_j}}\pa{ y_n^*}}}\\
& \leq \frac{\beta_n}{\sum_{j=n}^{\infty} \beta_j} \norm{ \sum_{j=1}^{n-1} \beta_j y_j^*+ \pa{ \sum_{j=n}^\infty \beta_j} y_n^*}+ \frac{\sum_{j=n+1}^{\infty} \beta_j}{\sum_{j=n}^{\infty} \beta_j} \norm{\sum_{j=1}^{n-1} \beta_j y_j^*}\\
& \leq \frac{\beta_n}{\sum_{j=n}^{\infty} \beta_j} \norm{ \sum_{j=1}^{n-1} \beta_j y_j^*+ \pa{ 1-\sum_{j=1}^{n-1} \beta_j} y_n^*}+ \frac{\sum_{j=n+1}^{\infty} \beta_j}{\sum_{j=n}^{\infty} \beta_j} \norm{\sum_{j=1}^{n-1} \beta_j y_j^*}\\
& < \frac{\beta_n}{\sum_{j=n}^{\infty} \beta_j} \pa{\gamma_n}\pa{1+\delta_n}+\frac{\sum_{j=n+1}^{\infty} \beta_j}{\sum_{j=n}^{\infty} \beta_j} \norm{\sum_{j=1}^{n-1} \beta_j y_j^*}\\
& = \pa{ \sum_{j=n+1}^{\infty} \beta_j} \pa{ \pa{ \frac{\beta_n \gamma_n \pa{1+\delta_n}}{\pa{\sum_{j=n}^\infty \beta_j}\pa{\sum_{j=n+1}^\infty \beta_j}}}+ \pa{\frac{1}{\sum_{j=n}^\infty \beta_j}} \norm{\sum_{j=1}^{n-1} \beta_j y_j^*}}
\end{align*}
Hence, for any $n \in \N$, 
\begin{align*}
\norm{\sum_{j=1}^n \beta_j y_j^*} & < \pa{\sum_{j=n+1}^\infty \beta_j} \pa{ \pa{ \frac{\beta_n\gamma_n\pa{1+\delta_n}}{\pa{\sum_{j=n}^\infty \beta_j}\pa{ \sum_{j=n+1}^\infty \beta_j}}} + \pa{\frac{1}{\sum_{j=n}^\infty \beta_j}} \norm{\sum_{j=1}^{n-1} \beta_j y_j^*}}\\
& < \pa{\sum_{j=n+1}^{\infty} \beta_j} \sum_{j=1}^n \frac{\beta_j \gamma_j\pa{1+\delta_j}}{\pa{\sum_{k=j}^\infty \beta_k}\pa{ \sum_{k=j+1}^\infty \beta_k}}\\
& \leq \gamma \pa{\sum_{j=n+1}^{\infty} \beta_j} \sum_{j=1}^n \frac{\beta_j \pa{1+\delta_j}}{\pa{\sum_{k=j}^\infty \beta_k}\pa{ \sum_{k=j+1}^\infty \beta_k}}\\
&\leq \gamma \pa{ \sum_{j=n+1}^\infty \beta_j}\pa{ \pa{ \sum_{j=1}^n \frac{\beta_j}{\pa{\sum_{k=j}^\infty \beta_k}\pa{ \sum_{k=j+1}^\infty \beta_k}}}+ \pa{1-\alpha}}\\
& = \gamma \pa{ \sum_{j=n+1}^\infty \beta_j}\pa{ \pa{ \sum_{j=1}^n\pa{\frac{1}{\sum_{k=j}^\infty \beta_k}-\frac{1}{\sum_{k=j+1}^\infty \beta_k}}}+ \pa{1-\alpha}}\\
& = \gamma\pa{\sum_{j=n+1}^\infty \beta_j} \pa{ \frac{1}{\sum_{j=n+1}^\infty \beta_j} - \frac{1}{\sum_{j=1}^{\infty} \beta_j}+1-\alpha}\\
&=\gamma \pa{ \sum_{j=n+1}^\infty \beta_j} \pa{ \frac{1}{\sum_{j=n+1}^\infty \beta_j}- \alpha}\\
& = \gamma \pa{ 1- \alpha \pa{\sum_{j=n+1}^\infty \beta_j}}
\end{align*}
completing the proof.
\end{proof}
\begin{proof}[Proof of \ref{lem:james:ClosedBall}]
As before, let $\{\delta_i\}_{i \in \N} \subset (0,1)$ such that 
\begin{equation*} 
\sum_{i \in \N} \frac{\beta_i \delta_i}{\pa{ \sum\limits_{j=i+1}^{\infty} \beta_i} \pa{ \sum\limits_{j=i}^{\infty} \beta_i}} < 1-\alpha
\end{equation*} 
Define $\{x_i^0\}_{i \in \N} = \{x_i^*\}_{i \in \N}$.
Define
\begin{equation*}
\gamma_1 = \inf\left\{ \sup\limits_{y^* \in CoLim\{\phi_i\}_{ i\in \N}} \left\{ \norm{x^*-y^*}\right\} : x^* \in \overline{co}\{x_i^0\}_{i \in \N}, \phi_k \in \overline{co}\{x_i^0\}_{i \geq k}, k \in \N \right\}
\end{equation*}
Choose $\{\phi_i^1\}_{i \in \N} \subset X^*$
such that
for every $k \in \N$, $\phi_k^1 \in \overline{co}\{x_i^*\}_{i \geq k}$ 
and for all $y^* \in \overline{co}\{x_i^0\}_{i \in \N}$, 
and for all $w \in CoLim \{\phi_i^1\}_{i \in \N}$, we have
\begin{equation*}
\norm{y^*-w} \leq \gamma_1 \pa{1+\frac{\delta}{2}}
\end{equation*}
Choose $y_1^* \in \overline{co}\{x_i^*\}_{i \in \N}$ such that
there exists $w' \in CoLim \{\phi_i^1\}_{i \in \N}$
such that 
\begin{equation*} 
\gamma_1 < \norm{y_1^*-w'} \leq \gamma_1 \pa{1+\frac{\delta_1}{2}} < \gamma_1(1+\delta_1)
\end{equation*}
Then there exists $\tilde{x} \in \overline{B_X(0;1)}$ 
such that $\gamma_1 \pa{1-\delta_1} < \ip{\tilde{x},y_1^*-w'}$. 
Since $w' \in CoLim\{\phi_i^1\}_{i \in \N}$, 
we can extract a subsequence $\{x_i^1\}_{i \in \N}$ of $\{\phi_i^1\}_{i \in \N}$  
such that 
$\lim\limits_{i \to \infty} \ip{\tilde{x}, x_i^1}$ exists and
equals $\liminf\limits_{i \to \infty} \ip{\tilde{x}, \phi_i^1}$.
Thus, 
for any $w \in CoLim\{x_i^1\}_{i \in \N}$, we have 
\begin{equation*}
\ip{\tilde{x},w} = \lim\limits_{i \to \infty} \ip{\tilde{x},x_i^1} = \liminf\limits_{i \to \infty} \ip{\tilde{x},\phi_i^1} \leq \ip{\tilde{x},w'}
\end{equation*}
And so, for any $w \in CoLim\{x_i^1\}_{i \in \N}$, we have
\begin{equation*} 
\gamma_1  \pa{1-\delta_1} < \ip{\tilde{x},y_1^*-w}
\end{equation*}
Continuing inductively, for $i \in \N$, set 
\begin{equation*}
\gamma_{i+1}= \inf \left\{ \sup \left\{ \norm{\pa{\sum_{j=1}^{i} \beta_j y_j^*} + \pa{\pa{\sum_{j=i+1}^{\infty}\beta_j }y^*}-w}: w \in CoLim\{\phi_i\}_{i \in\N}\right\}\right\}
\end{equation*}
Where the \Infimum is taken over all $y^* \in \overline{co}\{x_j^{i}\}_{j \geq i+1}$ and all $\{\phi_i\}_{i \in \N} \subset X^*$ such that $\phi_k \in \overline{co} \{x_j^i\}_{j \geq k}$. 
Next, pick $\{\phi_j^{i+1}\}_{j \in \N} \subset X^*$ such that for every $k \in \N$,
for every $\phi_k^{i+1} \in \overline{co} \{x_j^i\}_{j \geq k}$,
for all $y^* \in \overline{co}\{x_j^{i}\}_{j \in \N}$, 
and for all $w \in CoLim\{\phi_j^{i+1}\}_{j \in \N}$, we have 
\begin{equation*}
 \norm{ \sum_{j=1}^i \beta_j y_j^*+ \pa{ \sum_{j=i+1}^{\infty} \beta_j} y^*-w} < \gamma_{i+1}\pa{1+\delta_{i+1}}
\end{equation*}
Next, pick $y_{i+1}^* \in \overline{co}\{x_j^i\}_{j \geq i+1}$ 
and pick $w' \in CoLim\{\phi_j^{i+1}\}_{j \in \N}$ such that
\begin{equation*}
\gamma_{i+1}< \norm{ \sum_{j=1}^i \beta_j y_j^*+ \pa{ \sum_{j=i+1}^{\infty} \beta_j} y_{i+1}^*-w'} < \gamma_{i+1}\pa{1+\delta_{i+1}}
\end{equation*}
Next, pick $\tilde{x} \in \overline{B_X(0;1)}$ satisfying
\begin{equation*}
\gamma_{i+1}(1-\delta_{i+1}) < \ip{\tilde{x}, \sum_{j=1}^i \beta_j y_j^*+ \pa{ \pa{ \sum_{j=i+1}^\infty \beta_j} y_j^*}-w'}
\end{equation*}
and apply the fact that since $\liminf\limits_{j \to \infty} \ip{\tilde{x},\phi_j^{i+1}} \leq \ip{\tilde{x},w}$, we can find a subsequence $\{x_j^{i+1}\}_{j \in \N}$ of $\{\phi_j^{i+1}\}_{j \in \N}$ such that for every $w \in CoLim\{x_j^{i+1}\}_{j \in \N}$ we have 
\begin{equation*}
\gamma_{i+1}(1-\delta_{i+1}) < \ip{\tilde{x}, \sum_{j=1}^i \beta_j y_j^*+ \pa{ \pa{ \sum_{j=i+1}^\infty \beta_j} y_j^*}-w}
\end{equation*}
completing our construction. 
Clearly, $CoLim\{y_j^*\}_{j \in \N} \subset CoLim\{\phi_j^i\}_{j \in \N}$ for every $i \in \N$. Hence, for every $w \in CoLim\{y_j^*\}_{ j\in \N}$, for ever $i \in \N$, we have 
\begin{equation*}
\gamma_i(1-\delta_i) < \norm{\sum_{j=1}^{i-1} \beta_j y_j^*+ \pa{\pa{\sum_{j=i}^\infty \beta_j} y_i^*}-w} < \gamma_i(1+\delta_i)
\end{equation*}
Also, since $\{y_j^*\}_{j \in \N} \subset \overline{co}\{x_j\}_{j \in \N} \subset \overline{B_{X^*}(0;1)}$, $CoLim\{y_j^*\}_{j \in \N} \subset \overline{co}\{y_j^*\}_{j \in \N} \subset \overline{B_{X^*}(0;1)}$.
By definition, $\gamma_1 \geq \alpha$, and
$\{\gamma_i\}_{i \in \N}$ is a nondecreasing \Sequence since it is defined by taking the \Infimum over a set which never gains new elements as $i$ increases. Further, $\norm{w} \leq 1$ for $w \in CoLim\{y_j^*\}_{j \in \N}$ implies that for every $n$, $\gamma_n \leq 2$, so by monotone convergence, $\gamma_i \nearrow \gamma= \norm{\sum_{j \in \N}\beta_j \pa{y_j-w}}\leq 2$. 
As for the final estimate, let $n \in \N$ and let $y^* \in CoLim\{y_j^*\}_{j \in \N}$.
Then,

\begin{align*}
\norm{\sum\limits_{j=1}^n \beta_j (y_j^*-y^*)} 
& = \norm{ \pa{\pa{ \frac{\sum_{j=n}^\infty \beta_j}{\sum_{j=n}^\infty \beta_j}}} \pa{\sum_{j=1}^{n-1} \beta_j (y_j^*-y^*)}}+ \pa{\pa{ \frac{\sum_{j=n}^\infty \beta_j}{\sum_{j=n}^\infty \beta_j}}\pa{\beta_n (y_n^*-y^*)}}\\
& = \norm{ \pa{\pa{ \frac{\beta_n+\sum_{j=n+1}^\infty \beta_j}{\sum_{j=n}^\infty \beta_j}} \pa{\sum_{j=1}^{n-1} \beta_j (y_j^*-y^*)}} + \pa{\pa{ \frac{\beta_n\sum_{j=n}^\infty \beta_j}{\sum_{j=n}^\infty \beta_j}}\pa{ (y_n^*-y^*)}}}\\%%HOWDOIBRINGTHISNORMDOWN
& \leq \frac{\beta_n}{\sum_{j=n}^{\infty} \beta_j} \norm{ \sum_{j=1}^{n-1} \beta_j (y_j^*-y^*)+ \pa{ \sum_{j=n}^\infty \beta_j} (y_n^*-y^*)} 
+ \frac{\sum_{j=n+1}^{\infty} \beta_j}{\sum_{j=n}^{\infty} \beta_j} \norm{\sum_{j=1}^{n-1} \beta_j (y_j^*-y^*)}\\
& \leq \frac{\beta_n}{\sum_{j=n}^{\infty} \beta_j} \norm{ \sum_{j=1}^{n-1} \beta_j (y_j^*-y^*)+ \pa{ 1-\sum_{j=1}^{n-1} \beta_j} (y_n^*-y^*)} \\
& + \frac{\sum_{j=n+1}^{\infty} \beta_j}{\sum_{j=n}^{\infty} \beta_j} \norm{\sum_{j=1}^{n-1} \beta_j (y_j^*-y^*)}\\
& < \frac{\beta_n}{\sum_{j=n}^{\infty} \beta_j} \pa{\gamma_n}\pa{1+\delta_n}+\frac{\sum_{j=n+1}^{\infty} \beta_j}{\sum_{j=n}^{\infty} \beta_j} \norm{\sum_{j=1}^{n-1} \beta_j (y_j^*-y^*)}\\
& = \pa{ \sum_{j=n+1}^{\infty} \beta_j} \pa{ \pa{ \frac{\beta_n \gamma_n \pa{1+\delta_n}}{\pa{\sum_{j=n}^\infty \beta_j}\pa{\sum_{j=n+1}^\infty \beta_j}}}+ \pa{\frac{1}{\sum_{j=n}^\infty \beta_j}}
 *\norm{\sum_{j=1}^{n-1} \beta_j (y_j^*-y^*)}}
\end{align*}
Hence, for every $i \in \N$, we have 
\begin{equation*}
\begin{split}
\norm{\sum_{j=1}^i \beta_j (y_j^*-y^*)} & < \pa{\sum_{j=i+1}^\infty \beta_j} \\
& \tab[.5cm]*\pa{ \pa{ \frac{\beta_i\gamma_i\pa{1+\delta_i}}{\pa{\sum_{j=i}^\infty \beta_j}\pa{ \sum_{j=i+1}^\infty \beta_j}}} + \pa{\frac{1}{\sum_{j=i}^\infty \beta_j}} \norm{\sum_{j=1}^{i-1} \beta_j (y_j^*-y^*)}}\\
& < \pa{\sum_{j=i+1}^{\infty} \beta_j} \sum_{j=1}^i \frac{\beta_j \gamma_j\pa{1+\delta_j}}{\pa{\sum_{k=j}^\infty \beta_k}\pa{ \sum_{k=j+1}^\infty \beta_k}}\\
& \leq \gamma \pa{\sum_{j=i+1}^{\infty} \beta_j} \sum_{j=1}^i \frac{\beta_j \pa{1+\delta_j}}{\pa{\sum_{k=j}^\infty \beta_k}\pa{ \sum_{k=j+1}^\infty \beta_k}}\\
&\leq \gamma \pa{ \sum_{j=i+1}^\infty \beta_j}\pa{ \pa{ \sum_{j=1}^i \frac{\beta_j}{\pa{\sum_{k=j}^\infty \beta_k}\pa{ \sum_{k=j+1}^\infty \beta_k}}}+ \pa{1-\alpha}}\\
& = \gamma \pa{ \sum_{j=i+1}^\infty \beta_j}\pa{ \pa{ \sum_{j=1}^i\pa{\frac{1}{\sum_{k=j+1}^\infty \beta_k}-\frac{1}{\sum_{k=j}^\infty \beta_k}}}+ \pa{1-\alpha}}\\
& = \gamma\pa{\sum_{j=i+1}^\infty \beta_j} \pa{ \frac{1}{\sum_{j=i+1}^\infty \beta_j} - \frac{1}{\sum_{j=1}^{\infty} \beta_j}+1-\alpha}\\
&=\gamma \pa{ \sum_{j=i+1}^\infty \beta_j} \pa{ \frac{1}{\sum_{j=i+1}^\infty \beta_j}- \alpha}\\
& = \gamma \pa{ 1- \alpha \pa{\sum_{j=i+1}^\infty \beta_j}}
\end{split} 
\end{equation*}
\end{proof} 
\end{lem}

\begin{rmk}
\rm
It is worth noting that in the following two theorems, the assumption that $X$
is \Complete is necessary, as demonstrated by \cite{james71}.
\end{rmk}
\begin{thm}[James Separable]
\label{thm:JamesSeparable}
\rm
Let $X$ be a \TopologySeparable
\Complete \SeminormedSpace.
The following are equivalent.
\begin{enumerate}[label=(\roman*), ref={\ref{thm:JamesSeparable}~\roman*}]
\item 
\label{thm:JamesSeparable:NotReflexive}
$X$ is not \Reflexive.
\item 
\label{thm:JamesSeparable:Colimit}
For every $\alpha \in (0,1)$ there exists a \Sequence $\{x_i^*\}_{i \in \N} \subset \overline{B_{X^*}(0;1)}$ satisfying $d\pa{0,\overline{co}\{x_i^*\}_{i \in \N}} \geq \alpha$ and $x_i^* \overset{w^*}{\to} 0$. 
\item 
\label{thm:JamesSeparable:Sequence}
For every $\alpha \in (0,1)$ and $\{\beta_i\}_{i \in \N} \subset (0,1)$ satisfying $\sum_{i \in \N} \beta_i = 1$, there is a $\gamma \in [\alpha,1]$ and $\{y_i^*\}_{i \in \N} \subset X^*$ such that 
$y_i^* \overset{w^*}{\to} 0$ and for each $i \in \N$, 
\begin{equation*}
\norm{\sum_{j \in \N} \beta_j y_j^*} = \gamma \tab[1cm] \norm{\sum_{j=1}^i \beta_j y_j^*} \leq \gamma \pa{ 1- \alpha \pa{ \sum_{j=i+1}^{\infty} \beta_j}}
\end{equation*}
\item 
\label{thm:JamesSeparable:Norm}
There exists $x^* \in X^*$ not attaining its \Norm. 
\end{enumerate} 
\begin{proof}[Proof of \ref{thm:JamesSeparable:NotReflexive} implies \ref{thm:JamesSeparable:Colimit}]
Let $\alpha \in (0,1)$. 
Since X is not \Reflexive and $c(X)$ is \Complete, 
by Riesz's lemma \cite{kreyszig89},  
there exists an $x^{**} \in B_{X^{**}}(0;1)$ such that $d(x^{**},c(X)) > \alpha$. 
Since $X$ is \TopologySeparable it has a \Countable \TopologyDense set $\{x_i\}_{i \in \N}$. 
Fix $i \in \N$, 
let $\alpha_1=\alpha_2=\cdots=\alpha_{i-1}=0$, 
$\alpha_i=\alpha$, 
and let $\{\beta_j\}_{j=1}^i \subset \C$ where $\beta_i \neq 0$ without loss of generality. 
Then, since $c(X)$ is a \VectorSubspace,  
\begin{align*}
\abs{\sum_{j=1}^i \beta_j \alpha_j}&= \abs{\beta_i \alpha_i} \\
& = \abs{\beta_i} \alpha \\
& \leq \frac{\abs{\beta_i} \alpha}{d\pa{x^{**},c(X)}} \norm{x^{**}+ \sum_{j=1}^{i-1} \frac{\beta_j}{\beta_i}c(x_j)}\\
& = \frac{\alpha}{d\pa{x^{**},c(X)}} \norm{\beta_i x^{**} + \sum_{j=1}^{i-1} \beta_j c(x_j)}
\end{align*}
Since $\alpha < d\pa{x^{**}, c(X)}$, for some $\epsilon > 0$, $\epsilon + \frac{\alpha}{d\pa{x^{**},c(X)}}<1$, so by \ref{thm:HellysTheorem}, since $X^{**}=\pa{X^*}^*$, there exists an $x_i^* \in \overline{B_{X^*}(0;1)}$ such that for $1 \leq j \leq i-1$ we have $\ip{x_j,x_i^*}=\ip{x_i^*,c(x_j)} = 0$ and $\ip{x_i^*,c(x_i)} \geq \alpha$.
Using this method we construct a sequence $\{x_i^*\}_{i \in \N} \subset \overline{B_{X^*}(0;1)}$ such that for each $1 \leq j \leq i-1$, $\ip{x_j,x_i^*} = 0$ and $\ip{x_i,x^{**}} \geq \alpha$. Without loss of generality, we let $\norm{x_i^*} = 1$.  
Since $\{x_j\}_{j \in \N}$ is \TopologyDense in $X$ and 
$\{x_j^*\}_{j \in \N}$  is bounded, $x_j^* \overset{w^*}{\to} 0.$
Furthermore, any convex combination of the $(x_i^*)'s$ satisfies 
\begin{equation*}
\alpha \leq \ip{\sum_{j=1}^n \lambda_j x_{k_j}^*, x^{**}}\leq \norm{x^{**}} \norm{\sum_{j=1}^n \lambda_j x_{k_j}^*} \leq \norm{\sum_{j=1}^n \lambda_j x_{k_j}^*}
\end{equation*}
so that $d\pa{0,\overline{co}\{x_i^*\}_{i \in \N}} \geq \alpha$, completing the proof. \end{proof}
\begin{proof}[Proof of \ref{thm:JamesSeparable:Colimit} implies \ref{thm:JamesSeparable:Sequence}]
Let $\alpha \in (0,1)$. 
Let $\{\beta_i\}_{i \in \N} \subset (0,1)$ 
such that $\sum\limits_{i \in \N} \beta_i = 1$. 
Let $\{x_i\}_{i \in \N} \subset \overline{B_{X^*}(0;1)}$ such that 
$d\pa{0, \overline{co}\{x_i^*\}_{i \in \N}} \geq \alpha$
and $x_i^* \wsto 0$ which is guaranteed to exist 
by \ref{thm:JamesSeparable:Colimit}.
By \ref{lem:james:Boundary}, 
there exists $\gamma \in [\alpha , 1 ]$ 
and $\{y_i^*\}_{i \in \N} \subset X^*$ such that for each $n \in \N$, 
\begin{equation*}
y_i^* \in \overline{co}\{x_j\}_{j \geq n} \tab[0.25cm] \norm{\sum\limits_{j \in \N} \beta_j y_j^*}= \gamma \tab[0.25cm] \norm{\sum\limits_{j=1}^n \beta_j y_j^*} < \gamma \pa{1-\alpha \pa{\sum\limits_{j={n+1}}^{\infty} \beta_j}}
\end{equation*}
Let $x \in X$
Let $\epsilon > 0$. 
Since $x_i^* \wsto 0$, there exists $N \in \N$ such that $n>N$ implies $\abs{\ip{x,x_n^*}} < \frac{\epsilon}{2}$. 
Let $n>N$. Then there exists $K \in \N$ and  $\{\lambda_i\}_{i  =n}^K \subset [0,1]$ such that 
$\norm{y_i-\sum\limits_{j=n}^K \lambda_j x_j^*} < \frac{\epsilon}{2\norm{x}}$ and $\sum\limits_{j=n}^K \lambda_j = 1$
Hence, 
\begin{align*}
\abs{\ip{x, y} } & \leq \abs{\ip{x, y-\sum\limits_{j=n}^k\lambda_j x_j^*}} + \abs{\ip{x, \sum\limits_{j=n}^K \lambda_j x_j^*}}\\
& \leq \norm{x} \norm{y-\sum\limits_{j=n}^K \lambda_j x_j^*} + \frac{\epsilon}{2}\\
& < \epsilon
\end{align*}
Since $x \in X$ was arbitrary, we conclude $y_i^* \wsto 0$. 
Thus, \ref{thm:JamesSeparable:Sequence} holds.
\end{proof} 
\begin{proof}[Proof of \ref{thm:JamesSeparable:Sequence} implies \ref{thm:JamesSeparable:Norm}]
Let $\alpha \in (0,1)$.
Let $\{\beta_i\}_{i \in \N} \subset (0,1)$ such that $\sum\limits_{i \in \N} \beta_i = 1$. 
By \ref{thm:JamesSeparable:Sequence}, there exists 
a $\gamma \in [\alpha, 1]$ and a 
$\{y_i^*\}_{i \in \N} \subset X^*$ such that 
$y_i^* \overset{w^*}{\to} 0$ and for each $i \in \N$, 
\begin{equation*}
\norm{\sum_{j \in \N} \beta_j y_j^*} = \gamma \tab[1cm] \norm{\sum_{j=1}^i \beta_j y_j^*} \leq \gamma \pa{ 1- \alpha \pa{ \sum_{j=i+1}^{\infty} \beta_j}}
\end{equation*}
Define $x^* = \sum_{j \in \N} \beta_j y_j^*$ and let $x \in \overline{B_X(0;1)}$. 
Since $y_j^* \overset{w^*}{\to} 0$, 
for some $N \in \N$, $\ip{x,y_j^*} < \gamma \alpha$ for every $j > \N$. Then
\begin{equation*}
\begin{split}
\abs{\ip{x, x^*}} & \leq \abs{ \ip{x,\sum_{j=1}^N \beta_j y_j^*}} + \abs{\ip{x,\sum_{j=N+1}^{\infty} \beta_j y_j^*}}\\
& < \abs{\ip{x,\sum_{j=1}^N \beta_j y_j^*}} + \alpha \gamma \sum_{j=N+1}^{\infty} \beta_j\\
& \leq \norm{\sum_{j=1}^N \beta_j y_j^*} + \alpha \gamma \sum_{j=N+1}^\infty \beta_j \\
& \leq \gamma \pa{1-\alpha \sum_{j=N+1}^\infty \beta_j}+ \gamma \alpha \sum_{j=N+1}^\infty \beta_j \\
&= \gamma \\
&= \norm{\sum_{j \in \N} \beta_j y_j^*} \\
& = \norm{x^*}
\end{split}
\end{equation*}
Since the inequality is strict and $x \in \overline{B_X(0;1)}$ was arbitrary, we are done. 
\end{proof} 
\begin{proof}[Proof of \ref{thm:JamesSeparable:Norm} implies \ref{thm:JamesSeparable:NotReflexive}]
I prove via contrapositive. 
Let $X$ be \Reflexive.
Let $x^* \in X$, 
Then there is a \Sequence $\{x_n\}_{n \in \N} \subset \overline{B_X(0;1)}$ such that $\ip{x_n,x^*} \to \norm{x^*}$.  
By  \ref{lem:ReflexiveSeparable}, $\{x_n\}_{n \in \N}$ has a \weakly convergent subsequence $x_{k_n} \overset{w}{\to} x \in \overline{B_X(0;1)}$. 
This $x$ satisfies $\ip{x,x^*} = \norm{x^*}$. 
\end{proof} 
\end{thm} 
\begin{thm}[James]
\label{thm:James}
\rm
Let $X$ be a 
\Complete \SeminormedSpace.
The following conditions are equivalent.
\begin{enumerate}[label=(\roman*), ref={\ref{thm:James}~\roman*}]
\item 
\label{thm:James:NotReflexive}
$X$ is not \Reflexive.
\item 
\label{thm:James:1}
For each $\alpha \in (0,1)$, there exists an $\{x_i^*\}_{i \in \N} \subset \overline{B_{X^*}(0;1)}$ and a \VectorSubspace $Y \subset X$ 
such that $d\pa{ \overline{co}\{x_i^*\}_{i \in \N}-Y^{\perp},0} \geq \alpha$ and that $\ip{y,x_i^*} \to 0$ for each $y \in Y$, 
where $Y^{\perp} = \{y \in X^* : Y \subset Kernel(y)\}$. 
\item 
\label{thm:James:2}
For every $\alpha \in (0,1)$ and $\{\beta_i\}_{i \in \N} \subset [0,\infty)$ such that $\sum_{i \in \N} \beta_i = 1$, there is a $\gamma \in [0,2]$ and  $\{y_i^*\}_{i \in \N} \subset \overline{B_{X^*}(0;1)}$ such that for each $y^* \in CoLim\{y_i^*\}_{i \in \N}$, each $i \in \N$, 
\begin{equation*} 
\norm{\sum_{j \in \N} \beta_j \pa{ y_j^*-y^*}} = \gamma \tab[1cm] \norm{\sum_{j=1}^i \beta_j \pa{ y_j^*-y_j}} < \gamma \pa{1-\alpha \pa{ \sum_{j=i+1}^{\infty} \beta_j}}
\end{equation*}
\item 
\label{thm:James:Norm}
There exists $x^* \in X^*$ which doesn't achieve its \Norm.  
\end{enumerate}
\begin{proof}[Proof of \ref{thm:James:NotReflexive} implies \ref{thm:James:1}]
Let $X$ be not-\Reflexive.
By \ref{lem:ReflexiveSeparable}, $X$ contains a \SetClosed \TopologySeparable
\VectorSubspace $S$.
An application of \ref{thm:JamesSeparable} implies the existence of a \Sequence $\{x_i\}_{i \in \N} \subset \overline{B_{S^*}(0;1)}$ such that 
\begin{equation*}
d_s\pa{0,\overline{co}\{x_i\}_{i \in \N}} \geq \alpha \tab[1cm] x_i \overset{S-w^*}{\to} 0
\end{equation*}
Let $y^{\perp} \in S^{\perp}$. 
For each $i \in \N$, let $x_i^*$ be a Hahn-Banach extension of $x_i$ living in $\overline{B_{X^*}(0;1)}$.
Let $x^* \in \overline{co}\{x_i^*\}_{i \in \N}$. 
Then $x:=x^*|_{S} \in \overline{co}\{x_i\}_{i \in \N}$. 
If $y^{\perp} \in S^{\perp}$, then 
\begin{equation*}
\norm{x^*-y^{\perp}} \geq \norm{x-\pa{y^{\perp}|_S}}_S = \norm{x} \geq \alpha
\end{equation*}
so $d\pa{ \overline{co}\{x_i^*\}_{i \in \N}-S^{\perp},0} \geq \alpha$.
Also, 
If $y \in S$, then since $x_i=x_i^*|_S$, 
$\ip{y,x_i^*}=\ip{y,x_i} \to 0$.
Hence, $x_i \overset{S-w^{*}}{\to} 0$.
Thus \ref{thm:James:1} holds with $Y=S$. 
\end{proof}
\begin{proof}[Proof of \ref{thm:James:1} implies \ref{thm:James:2}]
Let $\alpha  \in (0,1)$. 
Let $\{\beta_i\}_{i \in \N} \subset [0,\infty)$ such that $\sum\limits_{i \in \N} \beta_i =1$. 
By 
\ref{thm:James:1}, 
there exists 
$\{x_i^*\}_{i \in \N} \subset \overline{B_{X^*}(0;1)}$
and a \VectorSubspace $Y \subset X$
such that 
$\ip{y, x_i^*} \to 0$ for each $y \in Y$ 
and 
$d\pa{\overline{co}\{x_i^*\}_{i \in \N}-Y^{\perp}, 0} \geq \alpha$. 
Since $\ip{y, x_i^*} \to 0$ for each $y \in Y$, we conclude $CoLim\{x_i^*\}_{i \in \N} \subset Y^{\perp}$. 
Hence $d\pa{\overline{co}\{x_i^*\}_{i \in \N} - CoLim\{x_i^*\}_{i \in \N}, 0} \geq \alpha$. 
Thus we can apply 
\ref{lem:james:ClosedBall}
to claim the existence of $\gamma \in [\alpha, 2]$, 
the existence of $\{y_i^*\}_{i \in \N} \subset \overline{B_{X^*}(0;1)}$, 
and the existence of $y^* \in CoLim\{y_i^*\}_{i \in \N}$ such that 
\begin{equation*}
\norm{\sum\limits_{j \in \N} \beta_j \pa{y_j^*-y^*}} = \gamma
\end{equation*}
and
for every $n \in \N$, 
\begin{equation*}
\norm{\sum\limits_{j=1}^n \beta_j \pa{y_j^*-y^*}} < \gamma \pa{1-\alpha \pa{\sum\limits_{j=n+1}^{\infty} \beta_j}}
\end{equation*}
so that \ref{thm:James:2} holds.
\end{proof}
\begin{proof}[Proof of \ref{thm:James:2} implies \ref{thm:James:Norm}]
Let $\alpha \in (0,1)$. 
Define $\eta=\frac{\alpha^2}{4}$.
Let $\{\beta\}_{i \in \N} \subset (0,\infty)$ such that 
For each $n \in \N$, 
$\beta{n+1} <\eta \beta$ and $\sum_{k \in \N} \beta = 1$.
By 
\ref{thm:James:2}, there exists 
$\{y_i^*\}_{i \in \N} \subset \overline{B_{X^*}(0;1)}$
such that for each $y^* \in CoLim\{y_i^*\}_{i \in \N}$ and for each $i \in \N$, 
\begin{equation*} 
\norm{\sum_{j \in \N} \beta_j \pa{ y_j^*-y^*}} = \gamma \tab[1cm] \norm{\sum_{j=1}^i \beta_j \pa{ y_j^*-y_j}} < \gamma \pa{1-\alpha \pa{ \sum_{j=i+1}^{\infty} \beta_j}}
\end{equation*}
By 
\ref{rmk:colimnonempty}, there exists $y^* \in CoLim\{y_i^*\}_{i \in \N}$. 
Let $x \in \overline{B_X(0;1)}$. 
Since $\alpha \leq \gamma$, there exists $i \in \N$ such that 
\begin{equation*}
\ip{x,y_{i+1}^*-y^*} < \alpha^2 - 2\eta \leq \alpha \gamma - 2 \eta
\end{equation*}
For this x, we have 
\begin{equation*}
\begin{split}
\ip{x, \sum_{j \in \N} \beta_j (y_j^*-y^*)}& < \sum_{j=1}^i \beta_j \ip{x,y_j^*-y^*}+ \beta_{i+1} \pa{\alpha \gamma - 2 \eta} + \sum_{j=i+2}^{\infty} \beta_j \ip{x,y_j^*-y^*}\\
& \leq \norm{\sum_{j=1}^i \beta_j(y_j^*-y^*)}+ \beta_{i+1} \pa{\alpha \gamma -2\eta}+ 2 \sum_{j=i+2}^{\infty} \beta_j\\
& \leq \gamma \pa{1-\alpha \pa{\sum_{j=i+1}^{\infty} \beta_j}}+ \beta_{i+1}(\alpha \gamma -2\eta)+ 2 \sum_{j=i+2}^{\infty} \beta_j\\
& = \gamma -\gamma \alpha \sum_{j=i+2}^{\infty} \beta_j-2 \eta \beta_{i+1} + 2 \sum_{j=i+1}^{\infty} \eta \beta_j\\
& \leq \gamma - \pa{\gamma \alpha - 2 \eta} \sum_{j=i+1}^{\infty} \beta_j < \gamma = \norm{\sum_{j \in \N} \beta_j \pa{y_j^*-y^*}}
\end{split}
\end{equation*}
Since $x \in \overline{B_X(0;1)}$ was arbitrary, we are done. 
\end{proof}
\begin{proof}[Proof of \ref{thm:James:Norm} implies \ref{thm:James:NotReflexive}]
I prove via contrapositive.
Let $X$ be \Reflexive. 
Let $x^* \in X$.
Then there is a sequence $\{x_n\}_{n \in \N} \subset \overline{B_X(0;1)}$ such that $\ip{x_n,x^*} \to \norm{x^*}$.  
By  
\ref{lem:ReflexiveSeparable}, 
$\{x_n\}_{n \in \N}$ has \weakly convergent a subsequence $x_{k_n} \wto x \in \overline{B_X(0;1)}$. 
This x satisfies $\ip{x,x^*} = \norm{x^*} = \norm{x^*}\norm{x}$.
\end{proof} 
\end{thm} 
\begin{cor} 
\label{cor:JamesBrief}
\rm
Let $X$ be a \Complete \SeminormedSpace.
The following conditions are equivalent.
\begin{enumerate}
\item X is \Reflexive..
\item Each element of $X^*$ attains its \Norm. 
\end{enumerate}
\begin{proof}
This result is a direct consequence of \ref{thm:James}.
\end{proof} 
\end{cor} 
